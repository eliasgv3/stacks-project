\IfFileExists{stacks-project.cls}{%
\documentclass{stacks-project}
}{%
\documentclass{amsart}
}

% For dealing with references we use the comment environment
\usepackage{verbatim}
\newenvironment{reference}{\comment}{\endcomment}
%\newenvironment{reference}{}{}
\newenvironment{slogan}{\comment}{\endcomment}
\newenvironment{history}{\comment}{\endcomment}

% For commutative diagrams we use Xy-pic
\usepackage[all]{xy}

% We use 2cell for 2-commutative diagrams.
\xyoption{2cell}
\UseAllTwocells

% We use multicol for the list of chapters between chapters
\usepackage{multicol}

% This is generall recommended for better output
\usepackage{lmodern}
\usepackage[T1]{fontenc}

% For cross-file-references
\usepackage{xr-hyper}

% Package for hypertext links:
\usepackage{hyperref}

% For any local file, say "hello.tex" you want to link to please
% use \externaldocument[hello-]{hello}
\externaldocument[introduction-]{introduction}
\externaldocument[conventions-]{conventions}
\externaldocument[sets-]{sets}
\externaldocument[categories-]{categories}
\externaldocument[topology-]{topology}
\externaldocument[sheaves-]{sheaves}
\externaldocument[sites-]{sites}
\externaldocument[stacks-]{stacks}
\externaldocument[fields-]{fields}
\externaldocument[algebra-]{algebra}
\externaldocument[brauer-]{brauer}
\externaldocument[homology-]{homology}
\externaldocument[derived-]{derived}
\externaldocument[simplicial-]{simplicial}
\externaldocument[more-algebra-]{more-algebra}
\externaldocument[smoothing-]{smoothing}
\externaldocument[modules-]{modules}
\externaldocument[sites-modules-]{sites-modules}
\externaldocument[injectives-]{injectives}
\externaldocument[cohomology-]{cohomology}
\externaldocument[sites-cohomology-]{sites-cohomology}
\externaldocument[dga-]{dga}
\externaldocument[dpa-]{dpa}
\externaldocument[sdga-]{sdga}
\externaldocument[hypercovering-]{hypercovering}
\externaldocument[schemes-]{schemes}
\externaldocument[constructions-]{constructions}
\externaldocument[properties-]{properties}
\externaldocument[morphisms-]{morphisms}
\externaldocument[coherent-]{coherent}
\externaldocument[divisors-]{divisors}
\externaldocument[limits-]{limits}
\externaldocument[varieties-]{varieties}
\externaldocument[topologies-]{topologies}
\externaldocument[descent-]{descent}
\externaldocument[perfect-]{perfect}
\externaldocument[more-morphisms-]{more-morphisms}
\externaldocument[flat-]{flat}
\externaldocument[groupoids-]{groupoids}
\externaldocument[more-groupoids-]{more-groupoids}
\externaldocument[etale-]{etale}
\externaldocument[chow-]{chow}
\externaldocument[intersection-]{intersection}
\externaldocument[pic-]{pic}
\externaldocument[weil-]{weil}
\externaldocument[adequate-]{adequate}
\externaldocument[dualizing-]{dualizing}
\externaldocument[duality-]{duality}
\externaldocument[discriminant-]{discriminant}
\externaldocument[derham-]{derham}
\externaldocument[local-cohomology-]{local-cohomology}
\externaldocument[algebraization-]{algebraization}
\externaldocument[curves-]{curves}
\externaldocument[resolve-]{resolve}
\externaldocument[models-]{models}
\externaldocument[functors-]{functors}
\externaldocument[equiv-]{equiv}
\externaldocument[pione-]{pione}
\externaldocument[etale-cohomology-]{etale-cohomology}
\externaldocument[proetale-]{proetale}
\externaldocument[relative-cycles-]{relative-cycles}
\externaldocument[more-etale-]{more-etale}
\externaldocument[trace-]{trace}
\externaldocument[crystalline-]{crystalline}
\externaldocument[spaces-]{spaces}
\externaldocument[spaces-properties-]{spaces-properties}
\externaldocument[spaces-morphisms-]{spaces-morphisms}
\externaldocument[decent-spaces-]{decent-spaces}
\externaldocument[spaces-cohomology-]{spaces-cohomology}
\externaldocument[spaces-limits-]{spaces-limits}
\externaldocument[spaces-divisors-]{spaces-divisors}
\externaldocument[spaces-over-fields-]{spaces-over-fields}
\externaldocument[spaces-topologies-]{spaces-topologies}
\externaldocument[spaces-descent-]{spaces-descent}
\externaldocument[spaces-perfect-]{spaces-perfect}
\externaldocument[spaces-more-morphisms-]{spaces-more-morphisms}
\externaldocument[spaces-flat-]{spaces-flat}
\externaldocument[spaces-groupoids-]{spaces-groupoids}
\externaldocument[spaces-more-groupoids-]{spaces-more-groupoids}
\externaldocument[bootstrap-]{bootstrap}
\externaldocument[spaces-pushouts-]{spaces-pushouts}
\externaldocument[spaces-chow-]{spaces-chow}
\externaldocument[groupoids-quotients-]{groupoids-quotients}
\externaldocument[spaces-more-cohomology-]{spaces-more-cohomology}
\externaldocument[spaces-simplicial-]{spaces-simplicial}
\externaldocument[spaces-duality-]{spaces-duality}
\externaldocument[formal-spaces-]{formal-spaces}
\externaldocument[restricted-]{restricted}
\externaldocument[spaces-resolve-]{spaces-resolve}
\externaldocument[formal-defos-]{formal-defos}
\externaldocument[defos-]{defos}
\externaldocument[cotangent-]{cotangent}
\externaldocument[examples-defos-]{examples-defos}
\externaldocument[algebraic-]{algebraic}
\externaldocument[examples-stacks-]{examples-stacks}
\externaldocument[stacks-sheaves-]{stacks-sheaves}
\externaldocument[criteria-]{criteria}
\externaldocument[artin-]{artin}
\externaldocument[quot-]{quot}
\externaldocument[stacks-properties-]{stacks-properties}
\externaldocument[stacks-morphisms-]{stacks-morphisms}
\externaldocument[stacks-limits-]{stacks-limits}
\externaldocument[stacks-cohomology-]{stacks-cohomology}
\externaldocument[stacks-perfect-]{stacks-perfect}
\externaldocument[stacks-introduction-]{stacks-introduction}
\externaldocument[stacks-more-morphisms-]{stacks-more-morphisms}
\externaldocument[stacks-geometry-]{stacks-geometry}
\externaldocument[moduli-]{moduli}
\externaldocument[moduli-curves-]{moduli-curves}
\externaldocument[examples-]{examples}
\externaldocument[exercises-]{exercises}
\externaldocument[guide-]{guide}
\externaldocument[desirables-]{desirables}
\externaldocument[coding-]{coding}
\externaldocument[obsolete-]{obsolete}
\externaldocument[fdl-]{fdl}
\externaldocument[index-]{index}

% Theorem environments.
%
\theoremstyle{plain}
\newtheorem{theorem}[subsection]{Theorem}
\newtheorem{proposition}[subsection]{Proposition}
\newtheorem{lemma}[subsection]{Lemma}

\theoremstyle{definition}
\newtheorem{definition}[subsection]{Definition}
\newtheorem{example}[subsection]{Example}
\newtheorem{exercise}[subsection]{Exercise}
\newtheorem{situation}[subsection]{Situation}

\theoremstyle{remark}
\newtheorem{remark}[subsection]{Remark}
\newtheorem{remarks}[subsection]{Remarks}

\numberwithin{equation}{subsection}

% Macros
%
\def\lim{\mathop{\mathrm{lim}}\nolimits}
\def\colim{\mathop{\mathrm{colim}}\nolimits}
\def\Spec{\mathop{\mathrm{Spec}}}
\def\Hom{\mathop{\mathrm{Hom}}\nolimits}
\def\Ext{\mathop{\mathrm{Ext}}\nolimits}
\def\SheafHom{\mathop{\mathcal{H}\!\mathit{om}}\nolimits}
\def\SheafExt{\mathop{\mathcal{E}\!\mathit{xt}}\nolimits}
\def\Sch{\mathit{Sch}}
\def\Mor{\mathop{\mathrm{Mor}}\nolimits}
\def\Ob{\mathop{\mathrm{Ob}}\nolimits}
\def\Sh{\mathop{\mathit{Sh}}\nolimits}
\def\NL{\mathop{N\!L}\nolimits}
\def\CH{\mathop{\mathrm{CH}}\nolimits}
\def\proetale{{pro\text{-}\acute{e}tale}}
\def\etale{{\acute{e}tale}}
\def\QCoh{\mathit{QCoh}}
\def\Ker{\mathop{\mathrm{Ker}}}
\def\Im{\mathop{\mathrm{Im}}}
\def\Coker{\mathop{\mathrm{Coker}}}
\def\Coim{\mathop{\mathrm{Coim}}}

% Boxtimes
%
\DeclareMathSymbol{\boxtimes}{\mathbin}{AMSa}{"02}

%
% Macros for moduli stacks/spaces
%
\def\QCohstack{\mathcal{QC}\!\mathit{oh}}
\def\Cohstack{\mathcal{C}\!\mathit{oh}}
\def\Spacesstack{\mathcal{S}\!\mathit{paces}}
\def\Quotfunctor{\mathrm{Quot}}
\def\Hilbfunctor{\mathrm{Hilb}}
\def\Curvesstack{\mathcal{C}\!\mathit{urves}}
\def\Polarizedstack{\mathcal{P}\!\mathit{olarized}}
\def\Complexesstack{\mathcal{C}\!\mathit{omplexes}}
% \Pic is the operator that assigns to X its picard group, usage \Pic(X)
% \Picardstack_{X/B} denotes the Picard stack of X over B
% \Picardfunctor_{X/B} denotes the Picard functor of X over B
\def\Pic{\mathop{\mathrm{Pic}}\nolimits}
\def\Picardstack{\mathcal{P}\!\mathit{ic}}
\def\Picardfunctor{\mathrm{Pic}}
\def\Deformationcategory{\mathcal{D}\!\mathit{ef}}


% OK, start here.
%
\begin{document}

\title{Algebraic Stacks}


\maketitle

\phantomsection
\label{section-phantom}

\tableofcontents

\section{Introduction}
\label{section-introduction}

\noindent
This is where we define algebraic stacks and make some very elementary
observations. The general philosophy will be to have no separation
conditions whatsoever and add those conditions necessary to make lemmas,
propositions, theorems true/provable. Thus the notions discussed here
differ slightly from those in other places in the literature, e.g.,
\cite{LM-B}.

\medskip\noindent
This chapter is not an introduction to algebraic stacks.
For an informal discussion of algebraic stacks, please take a look at
Introducing Algebraic Stacks, Section
\ref{stacks-introduction-section-introduction}.


\section{Conventions}
\label{section-conventions}

\noindent
The conventions we use in this chapter are the same as those in the
chapter on algebraic spaces. For convenience we repeat them here.

\medskip\noindent
We work in a suitable big fppf site $\Sch_{fppf}$
as in Topologies, Definition \ref{topologies-definition-big-fppf-site}.
So, if not explicitly stated otherwise all schemes will be objects
of $\Sch_{fppf}$. We discuss what changes if you change the big
fppf site in
Section \ref{section-change-big-site}.

\medskip\noindent
We will always work relative to a base $S$ contained in $\Sch_{fppf}$.
And we will then work with the big fppf site $(\Sch/S)_{fppf}$, see
Topologies, Definition \ref{topologies-definition-big-small-fppf}.
The absolute case can be recovered by taking
$S = \Spec(\mathbf{Z})$.

\medskip\noindent
If $U, T$ are schemes over $S$, then we denote
$U(T)$ for the set of $T$-valued points {\it over} $S$.
In a formula: $U(T) = \Mor_S(T, U)$.

\medskip\noindent
Note that any fpqc covering is a universal effective
epimorphism, see
Descent, Lemma \ref{descent-lemma-fpqc-universal-effective-epimorphisms}.
Hence the topology on $\Sch_{fppf}$
is weaker than the canonical topology and all representable presheaves
are sheaves.








\section{Notation}
\label{section-notation}

\noindent
We use the letters $S, T, U, V, X, Y$ to indicate schemes.
We use the letters $\mathcal{X}, \mathcal{Y}, \mathcal{Z}$ to indicate
categories (fibred, fibred in groupoids, stacks, ...)
over $(\Sch/S)_{fppf}$. We use small case letters
$f$, $g$ for functors such as $f : \mathcal{X} \to \mathcal{Y}$
over $(\Sch/S)_{fppf}$.
We use capital $F$, $G$, $H$ for algebraic spaces over $S$, and more
generally for presheaves of sets on $(\Sch/S)_{fppf}$.
(In future chapters we will revert to using also $X$, $Y$, etc
for algebraic spaces.)

\medskip\noindent
The reason for these choices is that we want to clearly distinguish between
the different types of objects in this chapter, to build the foundations.









\section{Representable categories fibred in groupoids}
\label{section-representable}

\noindent
Let $S$ be a scheme contained in $\Sch_{fppf}$.
The basic object of study in this chapter will be a
category fibred in groupoids
$p : \mathcal{X} \to (\Sch/S)_{fppf}$, see
Categories, Definition \ref{categories-definition-fibred-groupoids}.
We will often simply say ``let $\mathcal{X}$ be a category fibred
in groupoids over $(\Sch/S)_{fppf}$'' to indicate
this situation. A $1$-morphism $\mathcal{X} \to \mathcal{Y}$ of categories
fibred in groupoids over $(\Sch/S)_{fppf}$ will be a $1$-morphism
in the $2$-category of categories fibred in groupoids over
$(\Sch/S)_{fppf}$, see
Categories,
Definition \ref{categories-definition-categories-fibred-in-groupoids-over-C}.
It is simply a functor $\mathcal{X} \to \mathcal{Y}$ over
$(\Sch/S)_{fppf}$.
We recall this is really a $(2, 1)$-category and that all $2$-fibre products
exist.

\medskip\noindent
Let $\mathcal{X}$ be a category fibred in groupoids over
$(\Sch/S)_{fppf}$. Recall that $\mathcal{X}$
is said to be {\it representable} if there exists a
scheme $U \in \Ob((\Sch/S)_{fppf})$ and an
equivalence
$$
j : \mathcal{X} \longrightarrow (\Sch/U)_{fppf}
$$
of categories over $(\Sch/S)_{fppf}$, see
Categories,
Definition \ref{categories-definition-representable-fibred-category}.
We will sometimes say that $\mathcal{X}$ is
{\it representable by a scheme} to distinguish from the case
where $\mathcal{X}$ is representable by an algebraic space (see
below).

\medskip\noindent
If $\mathcal{X}, \mathcal{Y}$ are fibred in groupoids and
representable by $U, V$, then we have
\begin{equation}
\label{equation-morphisms-schemes}
\Mor_{\textit{Cat}/(\Sch/S)_{fppf}}(\mathcal{X}, \mathcal{Y})
\Big/
2\text{-isomorphism}
=
\Mor_{\Sch/S}(U, V)
\end{equation}
see
Categories,
Lemma \ref{categories-lemma-morphisms-representable-fibred-categories}.
More precisely, any $1$-morphism $\mathcal{X} \to \mathcal{Y}$
gives rise to a morphism $U \to V$. Conversely, given a morphism
of schemes $U \to V$ over $S$ there exists a $1$-morphism
$\phi : \mathcal{X} \to \mathcal{Y}$ which gives rise to $U \to V$
and which is unique up to unique $2$-isomorphism.






\section{The 2-Yoneda lemma}
\label{section-2-yoneda}

\noindent
Let $U \in \Ob((\Sch/S)_{fppf})$, and let $\mathcal{X}$ be a
category fibred in groupoids over $(\Sch/S)_{fppf}$.
We will frequently use the $2$-Yoneda lemma, see
Categories, Lemma \ref{categories-lemma-yoneda-2category}.
Technically it says that there is an equivalence of categories
$$
\Mor_{\textit{Cat}/(\Sch/S)_{fppf}}(
(\Sch/U)_{fppf}, \mathcal{X})
\longrightarrow
\mathcal{X}_U, \quad
f \longmapsto f(U/U).
$$
It says that $1$-morphisms $(\Sch/U)_{fppf} \to \mathcal{X}$
correspond to objects $x$ of the fibre category $\mathcal{X}_U$.
Namely, given a $1$-morphism $f : (\Sch/U)_{fppf} \to \mathcal{X}$
we obtain the object $x = f(U/U) \in \Ob(\mathcal{X}_U)$.
Conversely, given a choice of pullbacks for $\mathcal{X}$ as in
Categories,
Definition \ref{categories-definition-pullback-functor-fibred-category},
and an object $x$ of $\mathcal{X}_U$, we obtain a functor
$(\Sch/U)_{fppf} \to \mathcal{X}$ defined by the rule
$$
(\varphi : V \to U) \longmapsto \varphi^*x
$$
on objects. By abuse of notation we use
$x : (\Sch/U)_{fppf} \to \mathcal{X}$
to indicate this functor. It indeed has the property that $x(U/U) = x$
and moreover, given any other functor $f$ with $f(U/U) = x$ there exists
a unique $2$-isomorphism $x \to f$. In other words the functor $x$
is well determined by the object $x$ up to unique $2$-isomorphism.

\medskip\noindent
We will use this without further mention in the following.





\section{Representable morphisms of categories fibred in groupoids}
\label{section-representable-morphism}


\noindent
Let $\mathcal{X}$, $\mathcal{Y}$ be categories fibred in groupoids
over $(\Sch/S)_{fppf}$. Let $f : \mathcal{X} \to \mathcal{Y}$
be a {\it representable $1$-morphism}, see
Categories, Definition
\ref{categories-definition-representable-map-categories-fibred-in-groupoids}.
This means that for every $U \in \Ob((\Sch/S)_{fppf})$ and
any $y \in \Ob(\mathcal{Y}_U)$ the $2$-fibre product
$(\Sch/U)_{fppf} \times_{y, \mathcal{Y}} \mathcal{X}$
is representable. Choose a representing object $V_y$ and an equivalence
$$
(\Sch/V_y)_{fppf}
\longrightarrow
(\Sch/U)_{fppf} \times_{y, \mathcal{Y}} \mathcal{X}.
$$
The projection
$(\Sch/V_y)_{fppf} \to
(\Sch/U)_{fppf} \times_\mathcal{Y} \mathcal{Y}
\to (\Sch/U)_{fppf}$
comes from a morphism of schemes $f_y : V_y \to U$, see
Section \ref{section-representable}. We represent this by the diagram
\begin{equation}
\label{equation-representable}
\vcenter{
\xymatrix{
V_y \ar@{~>}[r] \ar[d]_{f_y} &
(\Sch/V_y)_{fppf} \ar[d] \ar[r] &
\mathcal{X} \ar[d]^f \\
U \ar@{~>}[r] &
(\Sch/U)_{fppf} \ar[r]^-y &
\mathcal{Y}
}
}
\end{equation}
where the squiggly arrows represent the $2$-Yoneda embedding.
Here are some lemmas about this notion that work in great generality
(namely, they work for categories fibred in groupoids over any
base category which has fibre products).

\begin{lemma}
\label{lemma-morphism-schemes-gives-representable-transformation}
Let $f : X \to Y$ be a morphism of $(\Sch/S)_{fppf}$.
Then the $1$-morphism induced by $f$
$$
(\Sch/X)_{fppf} \longrightarrow (\Sch/Y)_{fppf}
$$
is a representable $1$-morphism.
\end{lemma}

\begin{proof}
This is formal and relies only on the fact that
the category $(\Sch/S)_{fppf}$ has fibre products.
\end{proof}

\begin{lemma}
\label{lemma-representable-morphism-equivalent}
Let $S$ be an object of $\Sch_{fppf}$.
Consider a $2$-commutative diagram
$$
\xymatrix{
\mathcal{X}' \ar[r] \ar[d]_{f'} & \mathcal{X} \ar[d]^f \\
\mathcal{Y}' \ar[r] & \mathcal{Y}
}
$$
of $1$-morphisms of categories fibred in groupoids over
$(\Sch/S)_{fppf}$.
Assume the horizontal arrows are equivalences.
Then $f$ is representable if and only if $f'$ is representable.
\end{lemma}

\begin{proof}
Omitted.
\end{proof}

\begin{lemma}
\label{lemma-composition-representable-transformations}
Let $S$ be a scheme contained in $\Sch_{fppf}$.
Let $\mathcal{X}, \mathcal{Y}, \mathcal{Z}$
be categories fibred in groupoids over $(\Sch/S)_{fppf}$
Let $f : \mathcal{X} \to \mathcal{Y}$, $g : \mathcal{Y} \to \mathcal{Z}$
be representable $1$-morphisms. Then
$$
g \circ f : \mathcal{X} \longrightarrow \mathcal{Z}
$$
is a representable $1$-morphism.
\end{lemma}

\begin{proof}
This is entirely formal and works in any category.
\end{proof}

\begin{lemma}
\label{lemma-base-change-representable-transformations}
Let $S$ be a scheme contained in $\Sch_{fppf}$.
Let $\mathcal{X}, \mathcal{Y}, \mathcal{Z}$
be categories fibred in groupoids over $(\Sch/S)_{fppf}$
Let $f : \mathcal{X} \to \mathcal{Y}$ be a representable $1$-morphism.
Let $g : \mathcal{Z} \to \mathcal{Y}$ be any $1$-morphism.
Consider the fibre product diagram
$$
\xymatrix{
\mathcal{Z} \times_{g, \mathcal{Y}, f} \mathcal{X} \ar[r]_-{g'} \ar[d]_{f'} &
\mathcal{X} \ar[d]^f \\
\mathcal{Z} \ar[r]^g & \mathcal{Y}
}
$$
Then the base change $f'$ is a representable $1$-morphism.
\end{lemma}

\begin{proof}
This is entirely formal and works in any category.
\end{proof}

\begin{lemma}
\label{lemma-product-representable-transformations}
Let $S$ be a scheme contained in $\Sch_{fppf}$.
Let $\mathcal{X}_i, \mathcal{Y}_i$ be categories fibred in groupoids over
$(\Sch/S)_{fppf}$, $i = 1, 2$.
Let $f_i : \mathcal{X}_i \to \mathcal{Y}_i$, $i = 1, 2$
be representable $1$-morphisms.
Then
$$
f_1 \times f_2 :
\mathcal{X}_1 \times \mathcal{X}_2
\longrightarrow
\mathcal{Y}_1 \times \mathcal{Y}_2
$$
is a representable $1$-morphism.
\end{lemma}

\begin{proof}
Write $f_1 \times f_2$ as the composition
$\mathcal{X}_1 \times \mathcal{X}_2 \to
\mathcal{Y}_1 \times \mathcal{X}_2 \to
\mathcal{Y}_1 \times \mathcal{Y}_2$.
The first arrow is the base change of $f_1$ by the map
$\mathcal{Y}_1 \times \mathcal{X}_2 \to \mathcal{Y}_1$, and the second arrow
is the base change of $f_2$ by the map
$\mathcal{Y}_1 \times \mathcal{Y}_2 \to \mathcal{Y}_2$.
Hence this lemma is a formal
consequence of Lemmas \ref{lemma-composition-representable-transformations}
and \ref{lemma-base-change-representable-transformations}.
\end{proof}



\section{Split categories fibred in groupoids}
\label{section-split}

\noindent
Let $S$ be a scheme contained in $\Sch_{fppf}$.
Recall that given a ``presheaf of groupoids''
$$
F : (\Sch/S)_{fppf}^{opp} \longrightarrow \textit{Groupoids}
$$
we get a category fibred in groupoids $\mathcal{S}_F$ over
$(\Sch/S)_{fppf}$, see
Categories, Example \ref{categories-example-functor-groupoids}.
Any category fibred in groupoids isomorphic (!) to one of these
is called a {\it split category fibred in groupoids}.
Any category fibred in groupoids is equivalent to a split one.

\medskip\noindent
If $F$ is a presheaf of sets then $\mathcal{S}_F$ is
fibred in sets, see
Categories,
Definition \ref{categories-definition-category-fibred-sets},
and
Categories, Example \ref{categories-example-presheaf}.
The rule $F \mapsto \mathcal{S}_F$ is in some sense fully faithful
on presheaves, see
Categories, Lemma \ref{categories-lemma-2-category-fibred-sets}.
If $F, G$ are presheaves, then
$$
\mathcal{S}_{F \times G}
=
\mathcal{S}_F \times_{(\Sch/S)_{fppf}} \mathcal{S}_G
$$
and if $F \to H$ and $G \to H$ are maps of presheaves of sets, then
$$
\mathcal{S}_{F \times_H G} =
\mathcal{S}_F \times_{\mathcal{S}_H} \mathcal{S}_G
$$
where the right hand sides are $2$-fibre products. This is immediate
from the definitions as the fibre categories of
$\mathcal{S}_F, \mathcal{S}_G, \mathcal{S}_H$ have only identity morphisms.

\medskip\noindent
An even more special case is where $F = h_X$ is a representable
presheaf. In this case we have
$\mathcal{S}_{h_X} = (\Sch/X)_{fppf}$, see
Categories,
Example \ref{categories-example-fibred-category-from-functor-of-points}.

\medskip\noindent
We will use the notation $\mathcal{S}_F$ without further mention in the
following.




\section{Categories fibred in groupoids representable by algebraic spaces}
\label{section-representable-by-algebraic-spaces}

\noindent
A slightly weaker notion than being representable is the notion of
being representable by algebraic spaces which we discuss in this section.
This discussion might have been avoided had we worked with some category
$\textit{Spaces}_{fppf}$ of algebraic spaces instead of the category
$\Sch_{fppf}$. However, it seems to us natural to consider the
category of schemes as the natural collection of ``test objects'' over
which the fibre categories of an algebraic stack are defined.

\medskip\noindent
In analogy with Categories, Definitions
\ref{categories-definition-representable-fibred-category}
we make the following definition.

\begin{definition}
\label{definition-representable-by-algebraic-space}
Let $S$ be a scheme contained in $\Sch_{fppf}$.
A category fibred in groupoids $p : \mathcal{X} \to (\Sch/S)_{fppf}$
is called {\it representable by an algebraic space over $S$}
if there exists an algebraic space $F$ over $S$ and an equivalence
$j : \mathcal{X} \to \mathcal{S}_F$
of categories over $(\Sch/S)_{fppf}$.
\end{definition}

\noindent
We continue our abuse of notation in suppressing the equivalence $j$
whenever we encounter such a situation.
It follows formally from the above that if $\mathcal{X}$ is
representable (by a scheme), then it is representable by an
algebraic space. Here is the analogue of
Categories,
Lemma \ref{categories-lemma-characterize-representable-fibred-category}.

\begin{lemma}
\label{lemma-characterize-representable-by-space}
Let $S$ be a scheme contained in $\Sch_{fppf}$.
Let $p : \mathcal{X} \to (\Sch/S)_{fppf}$
be a category fibred in groupoids.
Then $\mathcal{X}$ is representable by an algebraic space over $S$
if and only if the following conditions are satisfied:
\begin{enumerate}
\item $\mathcal{X}$ is fibred in setoids\footnote{This means that
it is fibred in groupoids and objects in the fibre categories
have no nontrivial automorphisms, see Categories,
Definition \ref{categories-definition-category-fibred-sets}.}, and
\item the presheaf $U \mapsto \Ob(\mathcal{X}_U)/\!\!\cong$ is
an algebraic space.
\end{enumerate}
\end{lemma}

\begin{proof}
Omitted, but see Categories,
Lemma \ref{categories-lemma-characterize-representable-fibred-category}.
\end{proof}

\noindent
If $\mathcal{X}, \mathcal{Y}$ are fibred in groupoids and
representable by algebraic spaces $F, G$ over $S$, then we have
\begin{equation}
\label{equation-morphisms-spaces}
\Mor_{\textit{Cat}/(\Sch/S)_{fppf}}(\mathcal{X}, \mathcal{Y})
\Big/
2\text{-isomorphism}
=
\Mor_{\Sch/S}(F, G)
\end{equation}
see
Categories, Lemma \ref{categories-lemma-2-category-fibred-setoids}.
More precisely, any $1$-morphism $\mathcal{X} \to \mathcal{Y}$
gives rise to a morphism $F \to G$. Conversely, given a morphism
of sheaves $F \to G$ over $S$ there exists a $1$-morphism
$\phi : \mathcal{X} \to \mathcal{Y}$ which gives rise to $F \to G$
and which is unique up to unique $2$-isomorphism.



\section{Morphisms representable by algebraic spaces}
\label{section-morphisms-representable-by-algebraic-spaces}

\noindent
In analogy with Categories, Definition
\ref{categories-definition-representable-map-categories-fibred-in-groupoids}
we make the following definition.

\begin{definition}
\label{definition-representable-by-algebraic-spaces}
Let $S$ be a scheme contained in $\Sch_{fppf}$.
A $1$-morphism $f : \mathcal{X} \to \mathcal{Y}$ of
categories fibred in groupoids over $(\Sch/S)_{fppf}$
is called {\it representable by algebraic spaces} if
for any $U \in \Ob((\Sch/S)_{fppf})$
and any $y : (\Sch/U)_{fppf} \to \mathcal{Y}$
the category fibred in groupoids
$$
(\Sch/U)_{fppf} \times_{y, \mathcal{Y}} \mathcal{X}
$$
over $(\Sch/U)_{fppf}$
is representable by an algebraic space over $U$.
\end{definition}

\noindent
Choose an algebraic space $F_y$ over $U$ which represents
$(\Sch/U)_{fppf} \times_{y, \mathcal{Y}} \mathcal{X}$.
We may think of $F_y$ as an algebraic space over $S$
which comes equipped with a canonical morphism $f_y : F_y \to U$
over $S$, see
Spaces, Section \ref{spaces-section-change-base-scheme}.
Here is the diagram
\begin{equation}
\label{equation-representable-by-algebraic-spaces}
\vcenter{
\xymatrix{
F_y \ar[d]_{f_y} &
(\Sch/U)_{fppf} \times_{y, \mathcal{Y}} \mathcal{X}
\ar@{~>}[l] \ar[d]_{\text{pr}_0} \ar[r]_-{\text{pr}_1} &
\mathcal{X} \ar[d]^f \\
U &
(\Sch/U)_{fppf} \ar@{~>}[l] \ar[r]^-y &
\mathcal{Y}
}
}
\end{equation}
where the squiggly arrows represent the construction which associates
to a stack fibred in setoids its associated sheaf of isomorphism classes
of objects. The right square is
$2$-commutative, and is a $2$-fibre product square.

\medskip\noindent
Here is the analogue of Categories,
Lemma \ref{categories-lemma-criterion-representable-map-stack-in-groupoids}.

\begin{lemma}
\label{lemma-criterion-map-representable-spaces-fibred-in-groupoids}
Let $S$ be a scheme contained in $\Sch_{fppf}$.
Let $f : \mathcal{X} \to \mathcal{Y}$ be a $1$-morphism
of categories fibred in groupoids over $(\Sch/S)_{fppf}$.
The following are necessary and sufficient conditions for
$f$ to be representable by algebraic spaces:
\begin{enumerate}
\item for each scheme $U/S$ the
functor $f_U : \mathcal{X}_U \longrightarrow \mathcal{Y}_U$
between fibre categories is faithful, and
\item for each $U$ and each $y \in \Ob(\mathcal{Y}_U)$ the presheaf
$$
(h : V \to U)
\longmapsto
\{(x, \phi) \mid x \in \Ob(\mathcal{X}_V), \phi : h^*y \to f(x)\}/\cong
$$
is an algebraic space over $U$.
\end{enumerate}
Here we have made a choice of pullbacks for $\mathcal{Y}$.
\end{lemma}

\begin{proof}
This follows from the description of fibre categories of the $2$-fibre products
$(\Sch/U)_{fppf} \times_{y, \mathcal{Y}} \mathcal{X}$ in
Categories, Lemma \ref{categories-lemma-identify-fibre-product}
combined with
Lemma \ref{lemma-characterize-representable-by-space}.
\end{proof}

\noindent
Here are some lemmas about this notion that work in great generality.

\begin{lemma}
\label{lemma-representable-by-spaces-morphism-equivalent}
Let $S$ be an object of $\Sch_{fppf}$.
Consider a $2$-commutative diagram
$$
\xymatrix{
\mathcal{X}' \ar[r] \ar[d]_{f'} & \mathcal{X} \ar[d]^f \\
\mathcal{Y}' \ar[r] & \mathcal{Y}
}
$$
of $1$-morphisms of categories fibred in groupoids over
$(\Sch/S)_{fppf}$.
Assume the horizontal arrows are equivalences.
Then $f$ is representable by algebraic spaces
if and only if $f'$ is representable by algebraic spaces.
\end{lemma}

\begin{proof}
Omitted.
\end{proof}

\begin{lemma}
\label{lemma-morphism-spaces-gives-representable-by-spaces}
Let $S$ be an object of $\Sch_{fppf}$.
Let $f : \mathcal{X} \to \mathcal{Y}$
be a $1$-morphism of categories fibred in groupoids over $S$.
If $\mathcal{X}$ and $\mathcal{Y}$ are representable by
algebraic spaces over $S$, then the $1$-morphism $f$
is representable by algebraic spaces.
\end{lemma}

\begin{proof}
Omitted. This relies only on the fact that
the category of algebraic spaces over $S$ has fibre products,
see Spaces, Lemma \ref{spaces-lemma-fibre-product-spaces}.
\end{proof}

\begin{lemma}
\label{lemma-map-presheaves-representable-by-algebraic-spaces}
Let $S$ be an object of $\Sch_{fppf}$.
Let $a : F \to G$ be a map of presheaves of sets on $(\Sch/S)_{fppf}$.
Denote $a' : \mathcal{S}_F  \to \mathcal{S}_G$ the associated
map of categories fibred in sets.
Then $a$ is representable by algebraic spaces (see
Bootstrap,
Definition \ref{bootstrap-definition-morphism-representable-by-spaces})
if and only if $a'$ is representable by algebraic spaces.
\end{lemma}

\begin{proof}
Omitted.
\end{proof}

\begin{lemma}
\label{lemma-map-fibred-setoids-representable-algebraic-spaces}
Let $S$ be an object of $\Sch_{fppf}$.
Let $f : \mathcal{X} \to \mathcal{Y}$ be a $1$-morphism of
categories fibred in setoids over $(\Sch/S)_{fppf}$.
Let $F$, resp.\ $G$ be the presheaf which to $T$ associates
the set of isomorphism classes of objects of
$\mathcal{X}_T$, resp.\ $\mathcal{Y}_T$.
Let $a : F \to G$ be the map of presheaves corresponding to $f$.
Then $a$ is representable by algebraic spaces (see
Bootstrap,
Definition \ref{bootstrap-definition-morphism-representable-by-spaces})
if and only if $f$ is representable by algebraic spaces.
\end{lemma}

\begin{proof}
Omitted. Hint: Combine
Lemmas \ref{lemma-representable-by-spaces-morphism-equivalent}
and \ref{lemma-map-presheaves-representable-by-algebraic-spaces}.
\end{proof}

\begin{lemma}
\label{lemma-base-change-representable-by-spaces}
Let $S$ be a scheme contained in $\Sch_{fppf}$.
Let $\mathcal{X}, \mathcal{Y}, \mathcal{Z}$
be categories fibred in groupoids over $(\Sch/S)_{fppf}$.
Let $f : \mathcal{X} \to \mathcal{Y}$ be a $1$-morphism
representable by algebraic spaces.
Let $g : \mathcal{Z} \to \mathcal{Y}$ be any $1$-morphism.
Consider the fibre product diagram
$$
\xymatrix{
\mathcal{Z} \times_{g, \mathcal{Y}, f} \mathcal{X} \ar[r]_-{g'} \ar[d]_{f'} &
\mathcal{X} \ar[d]^f \\
\mathcal{Z} \ar[r]^g & \mathcal{Y}
}
$$
Then the base change $f'$ is a $1$-morphism representable by
algebraic spaces.
\end{lemma}

\begin{proof}
This is formal.
\end{proof}

\begin{lemma}
\label{lemma-base-change-by-space-representable-by-space}
Let $S$ be a scheme contained in $\Sch_{fppf}$.
Let $\mathcal{X}, \mathcal{Y}, \mathcal{Z}$
be categories fibred in groupoids over $(\Sch/S)_{fppf}$
Let $f : \mathcal{X} \to \mathcal{Y}$,
$g : \mathcal{Z} \to \mathcal{Y}$ be $1$-morphisms.
Assume
\begin{enumerate}
\item $f$ is representable by algebraic spaces, and
\item $\mathcal{Z}$ is representable by an algebraic space over $S$.
\end{enumerate}
Then the $2$-fibre product
$\mathcal{Z} \times_{g, \mathcal{Y}, f} \mathcal{X}$
is representable by an algebraic space.
\end{lemma}

\begin{proof}
This is a reformulation of
Bootstrap, Lemma \ref{bootstrap-lemma-representable-by-spaces-over-space}.
First note that
$\mathcal{Z} \times_{g, \mathcal{Y}, f} \mathcal{X}$
is fibred in setoids over $(\Sch/S)_{fppf}$.
Hence it is equivalent to $\mathcal{S}_F$ for some presheaf
$F$ on $(\Sch/S)_{fppf}$, see
Categories, Lemma \ref{categories-lemma-setoid-fibres}.
Moreover, let $G$ be an algebraic space which represents
$\mathcal{Z}$. The $1$-morphism
$\mathcal{Z} \times_{g, \mathcal{Y}, f} \mathcal{X} \to \mathcal{Z}$
is representable by algebraic spaces by
Lemma \ref{lemma-base-change-representable-by-spaces}.
And $\mathcal{Z} \times_{g, \mathcal{Y}, f} \mathcal{X} \to \mathcal{Z}$
corresponds to a morphism $F \to G$ by
Categories, Lemma \ref{categories-lemma-2-category-fibred-setoids}.
Then $F \to G$ is representable by algebraic spaces by
Lemma \ref{lemma-map-fibred-setoids-representable-algebraic-spaces}.
Hence
Bootstrap, Lemma \ref{bootstrap-lemma-representable-by-spaces-over-space}
implies that $F$ is an algebraic space as desired.
\end{proof}

\noindent
Let $S$, $\mathcal{X}$, $\mathcal{Y}$, $\mathcal{Z}$, $f$, $g$ be as in
Lemma \ref{lemma-base-change-by-space-representable-by-space}.
Let $F$ and $G$ be algebraic spaces over $S$ such that
$F$ represents $\mathcal{Z} \times_{g, \mathcal{Y}, f} \mathcal{X}$
and $G$ represents $\mathcal{Z}$. The $1$-morphism
$f' : \mathcal{Z} \times_{g, \mathcal{Y}, f} \mathcal{X} \to \mathcal{Z}$
corresponds to a morphism $f' : F \to G$ of algebraic spaces
by (\ref{equation-morphisms-spaces}).
Thus we have the following diagram
\begin{equation}
\label{equation-representable-by-algebraic-spaces-on-space}
\vcenter{
\xymatrix{
F \ar[d]_{f'} &
\mathcal{Z} \times_{g, \mathcal{Y}, f} \mathcal{X}
\ar@{~>}[l] \ar[d] \ar[r] &
\mathcal{X} \ar[d]^f \\
G &
\mathcal{Z} \ar@{~>}[l] \ar[r]^-g &
\mathcal{Y}
}
}
\end{equation}
where the squiggly arrows represent the construction which associates
to a stack fibred in setoids its associated sheaf of isomorphism classes
of objects.

\begin{lemma}
\label{lemma-composition-representable-by-spaces}
Let $S$ be a scheme contained in $\Sch_{fppf}$.
Let $\mathcal{X}, \mathcal{Y}, \mathcal{Z}$
be categories fibred in groupoids over $(\Sch/S)_{fppf}$.
If $f : \mathcal{X} \to \mathcal{Y}$, $g : \mathcal{Y} \to \mathcal{Z}$
are $1$-morphisms representable by algebraic spaces, then
$$
g \circ f : \mathcal{X} \longrightarrow \mathcal{Z}
$$
is a $1$-morphism representable by algebraic spaces.
\end{lemma}

\begin{proof}
This follows from
Lemma \ref{lemma-base-change-by-space-representable-by-space}.
Details omitted.
\end{proof}

\begin{lemma}
\label{lemma-product-representable-by-spaces}
Let $S$ be a scheme contained in $\Sch_{fppf}$.
Let $\mathcal{X}_i, \mathcal{Y}_i$ be categories fibred in groupoids over
$(\Sch/S)_{fppf}$, $i = 1, 2$.
Let $f_i : \mathcal{X}_i \to \mathcal{Y}_i$, $i = 1, 2$
be $1$-morphisms representable by algebraic spaces.
Then
$$
f_1 \times f_2 :
\mathcal{X}_1 \times \mathcal{X}_2
\longrightarrow
\mathcal{Y}_1 \times \mathcal{Y}_2
$$
is a $1$-morphism representable by algebraic spaces.
\end{lemma}

\begin{proof}
Write $f_1 \times f_2$ as the composition
$\mathcal{X}_1 \times \mathcal{X}_2 \to
\mathcal{Y}_1 \times \mathcal{X}_2 \to
\mathcal{Y}_1 \times \mathcal{Y}_2$.
The first arrow is the base change of $f_1$ by the map
$\mathcal{Y}_1 \times \mathcal{X}_2 \to \mathcal{Y}_1$, and the second arrow
is the base change of $f_2$ by the map
$\mathcal{Y}_1 \times \mathcal{Y}_2 \to \mathcal{Y}_2$.
Hence this lemma is a formal
consequence of Lemmas \ref{lemma-composition-representable-by-spaces}
and \ref{lemma-base-change-representable-by-spaces}.
\end{proof}

\begin{lemma}
\label{lemma-get-a-stack}
\begin{reference}
Lemma in an email of Matthew Emerton dated June 15, 2016
\end{reference}
Let $S$ be a scheme contained in $\Sch_{fppf}$.
Let $\mathcal{X} \to \mathcal{Z}$ and $\mathcal{Y} \to \mathcal{Z}$
be $1$-morphisms of categories fibred in groupoids over $(\Sch/S)_{fppf}$.
If $\mathcal{X} \to \mathcal{Z}$ is representable by algebraic spaces
and $\mathcal{Y}$ is a stack in groupoids, then
$\mathcal{X} \times_\mathcal{Z} \mathcal{Y}$ is a stack in groupoids.
\end{lemma}

\begin{proof}
The property of a morphism being representable by algebraic spaces
is preserved under base-change
(Lemma \ref{lemma-base-change-by-space-representable-by-space}),
and so, passing to the base-change
$\mathcal{X} \times_\mathcal{Z} \mathcal{Y}$ over $\mathcal{Y}$,
we may reduce to the case of a morphism of categories
fibred in groupoids $\mathcal{X} \to \mathcal{Y}$
which is representable by algebraic spaces, and
whose target is a stack in groupoids; our goal is then to prove
that $\mathcal{X}$ is also a stack in groupoids.
This follows from Stacks, Lemma
\ref{stacks-lemma-relative-sheaf-over-stack-is-stack}
whose assumptions are satisfied as a result of
Lemma \ref{lemma-criterion-map-representable-spaces-fibred-in-groupoids}.
\end{proof}







\section{Properties of morphisms representable by algebraic spaces}
\label{section-representable-properties}

\noindent
Here is the definition that makes this work.

\begin{definition}
\label{definition-relative-representable-property}
Let $S$ be a scheme contained in $\Sch_{fppf}$.
Let $f : \mathcal{X} \to \mathcal{Y}$ be a $1$-morphism
of categories fibred in groupoids over $(\Sch/S)_{fppf}$.
Assume $f$ is representable by algebraic spaces.
Let $\mathcal{P}$ be a property of morphisms of algebraic spaces which
\begin{enumerate}
\item is preserved under any base change, and
\item is fppf local on the base, see
Descent on Spaces,
Definition \ref{spaces-descent-definition-property-morphisms-local}.
\end{enumerate}
In this case we say that $f$ has {\it property $\mathcal{P}$} if for every
$U \in \Ob((\Sch/S)_{fppf})$ and
any $y \in \mathcal{Y}_U$ the resulting morphism of algebraic spaces
$f_y : F_y \to U$, see
diagram (\ref{equation-representable-by-algebraic-spaces}),
has property $\mathcal{P}$.
\end{definition}

\noindent
It is important to note that we will only use this definition for
properties of morphisms that are stable under base change, and
local in the fppf topology on the target. This is
not because the definition doesn't make sense otherwise; rather it
is because we may want to give a different definition which is
better suited to the property we have in mind.

\begin{lemma}
\label{lemma-property-morphism-equivalent}
Let $S$ be an object of $\Sch_{fppf}$.
Let $\mathcal{P}$ be as in
Definition \ref{definition-relative-representable-property}.
Consider a $2$-commutative diagram
$$
\xymatrix{
\mathcal{X}' \ar[r] \ar[d]_{f'} & \mathcal{X} \ar[d]^f \\
\mathcal{Y}' \ar[r] & \mathcal{Y}
}
$$
of $1$-morphisms of categories fibred in groupoids over
$(\Sch/S)_{fppf}$.
Assume the horizontal arrows are equivalences and $f$ (or equivalently $f'$)
is representable by algebraic spaces.
Then $f$ has $\mathcal{P}$ if and only if $f'$ has $\mathcal{P}$.
\end{lemma}

\begin{proof}
Note that this makes sense by
Lemma \ref{lemma-representable-by-spaces-morphism-equivalent}.
Proof omitted.
\end{proof}

\noindent
Here is a sanity check.

\begin{lemma}
\label{lemma-map-presheaves-representable-by-spaces-transformation-property}
Let $S$ be a scheme contained in $\Sch_{fppf}$.
Let $a : F \to G$ be a map of presheaves on $(\Sch/S)_{fppf}$.
Let $\mathcal{P}$ be as in
Definition \ref{definition-relative-representable-property}.
Assume $a$ is representable by algebraic spaces.
Then $a : F \to G$ has property $\mathcal{P}$ (see
Bootstrap, Definition \ref{bootstrap-definition-property-transformation})
if and only if the corresponding morphism
$\mathcal{S}_F \to \mathcal{S}_G$ of categories fibred in groupoids
has property $\mathcal{P}$.
\end{lemma}

\begin{proof}
Note that the lemma makes sense by
Lemma \ref{lemma-map-presheaves-representable-by-algebraic-spaces}.
Proof omitted.
\end{proof}

\begin{lemma}
\label{lemma-map-fibred-setoids-property}
Let $S$ be an object of $\Sch_{fppf}$. Let $\mathcal{P}$ be as in
Definition \ref{definition-relative-representable-property}.
Let $f : \mathcal{X} \to \mathcal{Y}$ be a $1$-morphism of
categories fibred in setoids over $(\Sch/S)_{fppf}$.
Let $F$, resp.\ $G$ be the presheaf which to $T$ associates
the set of isomorphism classes of objects of
$\mathcal{X}_T$, resp.\ $\mathcal{Y}_T$.
Let $a : F \to G$ be the map of presheaves corresponding to $f$.
Then $a$ has $\mathcal{P}$ if and only if $f$ has $\mathcal{P}$.
\end{lemma}

\begin{proof}
The lemma makes sense by
Lemma \ref{lemma-map-fibred-setoids-representable-algebraic-spaces}.
The lemma follows on combining
Lemmas \ref{lemma-property-morphism-equivalent}
and \ref{lemma-map-presheaves-representable-by-spaces-transformation-property}.
\end{proof}

\begin{lemma}
\label{lemma-composition-representable-transformations-property}
Let $S$ be a scheme contained in $\Sch_{fppf}$.
Let $\mathcal{X}$, $\mathcal{Y}$, $\mathcal{Z}$ be categories fibred
in groupoids over $(\Sch/S)_{fppf}$.
Let $\mathcal{P}$ be a property as in
Definition \ref{definition-relative-representable-property}
which is stable under composition.
Let $f : \mathcal{X} \to \mathcal{Y}$,
$g : \mathcal{Y} \to \mathcal{Z}$ be $1$-morphisms which
are representable by algebraic spaces.
If $f$ and $g$ have property $\mathcal{P}$ so does
$g \circ f : \mathcal{X} \to \mathcal{Z}$.
\end{lemma}

\begin{proof}
Note that the lemma makes sense by
Lemma \ref{lemma-composition-representable-by-spaces}.
Proof omitted.
\end{proof}

\begin{lemma}
\label{lemma-base-change-representable-transformations-property}
Let $S$ be a scheme contained in $\Sch_{fppf}$.
Let $\mathcal{X}, \mathcal{Y}, \mathcal{Z}$
be categories fibred in groupoids over $(\Sch/S)_{fppf}$.
Let $\mathcal{P}$ be a property as in
Definition \ref{definition-relative-representable-property}.
Let $f : \mathcal{X} \to \mathcal{Y}$ be a $1$-morphism
representable by algebraic spaces.
Let $g : \mathcal{Z} \to \mathcal{Y}$ be any $1$-morphism.
Consider the $2$-fibre product diagram
$$
\xymatrix{
\mathcal{Z} \times_{g, \mathcal{Y}, f} \mathcal{X} \ar[r]_-{g'} \ar[d]_{f'} &
\mathcal{X} \ar[d]^f \\
\mathcal{Z} \ar[r]^g & \mathcal{Y}
}
$$
If $f$ has $\mathcal{P}$, then the base change $f'$
has $\mathcal{P}$.
\end{lemma}

\begin{proof}
The lemma makes sense by
Lemma \ref{lemma-base-change-representable-by-spaces}.
Proof omitted.
\end{proof}

\begin{lemma}
\label{lemma-descent-representable-transformations-property}
Let $S$ be a scheme contained in $\Sch_{fppf}$.
Let $\mathcal{X}, \mathcal{Y}, \mathcal{Z}$
be categories fibred in groupoids over $(\Sch/S)_{fppf}$.
Let $\mathcal{P}$ be a property as in
Definition \ref{definition-relative-representable-property}.
Let $f : \mathcal{X} \to \mathcal{Y}$ be a $1$-morphism
representable by algebraic spaces.
Let $g : \mathcal{Z} \to \mathcal{Y}$ be any $1$-morphism.
Consider the fibre product diagram
$$
\xymatrix{
\mathcal{Z} \times_{g, \mathcal{Y}, f} \mathcal{X} \ar[r]_-{g'} \ar[d]_{f'} &
\mathcal{X} \ar[d]^f \\
\mathcal{Z} \ar[r]^g & \mathcal{Y}
}
$$
Assume that for every scheme $U$ and object $x$ of $\mathcal{Y}_U$,
there exists an fppf covering $\{U_i \to U\}$ such that $x|_{U_i}$
is in the essential image of the functor
$g : \mathcal{Z}_{U_i} \to \mathcal{Y}_{U_i}$.
In this case, if $f'$ has $\mathcal{P}$, then $f$ has $\mathcal{P}$.
\end{lemma}

\begin{proof}
Proof omitted. Hint: Compare with the proof of
Spaces,
Lemma \ref{spaces-lemma-descent-representable-transformations-property}.
\end{proof}

\begin{lemma}
\label{lemma-product-representable-transformations-property}
Let $S$ be a scheme contained in $\Sch_{fppf}$.
Let $\mathcal{P}$ be a property as in
Definition \ref{definition-relative-representable-property}
which is stable under composition.
Let $\mathcal{X}_i, \mathcal{Y}_i$ be categories fibred in groupoids over
$(\Sch/S)_{fppf}$, $i = 1, 2$.
Let $f_i : \mathcal{X}_i \to \mathcal{Y}_i$, $i = 1, 2$
be $1$-morphisms representable by algebraic spaces.
If $f_1$ and $f_2$ have property $\mathcal{P}$ so does
$
f_1 \times f_2 :
\mathcal{X}_1 \times \mathcal{X}_2
\to
\mathcal{Y}_1 \times \mathcal{Y}_2
$.
\end{lemma}

\begin{proof}
The lemma makes sense by
Lemma \ref{lemma-product-representable-by-spaces}.
Proof omitted.
\end{proof}

\begin{lemma}
\label{lemma-representable-transformations-property-implication}
Let $S$ be a scheme contained in $\Sch_{fppf}$.
Let $\mathcal{X}$, $\mathcal{Y}$ be categories fibred in groupoids
over $(\Sch/S)_{fppf}$.
Let $f : \mathcal{X} \to \mathcal{Y}$ be a $1$-morphism representable
by algebraic spaces.
Let $\mathcal{P}$, $\mathcal{P}'$ be properties as in
Definition \ref{definition-relative-representable-property}.
Suppose that for any morphism of algebraic spaces $a : F \to G$
we have $\mathcal{P}(a) \Rightarrow \mathcal{P}'(a)$.
If $f$ has property $\mathcal{P}$ then
$f$ has property $\mathcal{P}'$.
\end{lemma}

\begin{proof}
Formal.
\end{proof}

\begin{lemma}
\label{lemma-open-fibred-category-is-full}
Let $S$ be a scheme contained in $\Sch_{fppf}$.
Let $j : \mathcal X \to \mathcal Y$ be a $1$-morphism of
categories fibred in groupoids over $(\Sch/S)_{fppf}$.
Assume $j$ is representable by algebraic spaces and a monomorphism
(see
Definition \ref{definition-relative-representable-property}
and
Descent on Spaces, Lemma
\ref{spaces-descent-lemma-descending-property-monomorphism}).
Then $j$ is fully faithful on fibre categories.
\end{lemma}

\begin{proof}
We have seen in
Lemma \ref{lemma-criterion-map-representable-spaces-fibred-in-groupoids}
that $j$ is faithful on fibre categories. Consider a scheme $U$,
two objects $u, v$ of $\mathcal{X}_U$, and an isomorphism
$t : j(u) \to j(v)$ in $\mathcal{Y}_U$. We have to construct an
isomorphism in $\mathcal{X}_U$ between $u$ and $v$.
By the $2$-Yoneda lemma (see Section \ref{section-2-yoneda})
we think of $u$, $v$ as $1$-morphisms
$u, v : (\Sch/U)_{fppf} \to \mathcal{X}$
and we consider the $2$-fibre product
$$
(\Sch/U)_{fppf} \times_{j \circ v, \mathcal{Y}} \mathcal{X}.
$$
By assumption this is representable by an algebraic space
$F_{j \circ v}$, over $U$ and the morphism
$F_{j \circ v} \to U$ is a monomorphism.
But since $(1_U, v, 1_{j(v)})$ gives a $1$-morphism of
$(\Sch/U)_{fppf}$ into the displayed $2$-fibre product,
we see that $F_{j \circ v} = U$ (here we use
that if $V \to U$ is a monomorphism of algebraic spaces which has a
section, then $V = U$). Therefore the $1$-morphism projecting to
the first coordinate
$$
(\Sch/U)_{fppf} \times_{j \circ v, \mathcal{Y}} \mathcal{X}
\to (\Sch/U)_{fppf}
$$
is an equivalence of fibre categories.
Since $(1_U, u, t)$ and $(1_U, v, 1_{j(v)})$ give two
objects in $((\Sch/U)_{fppf} \times_{j \circ v, \mathcal{Y}}
\mathcal{X})_U$ which have the same first coordinate, there must
be a $2$-morphism between them in the $2$-fibre product.
This is by definition a morphism $\tilde t : u \to v$ such that
$j(\tilde t) = t$.
\end{proof}

\noindent
Here is a characterization of those categories fibred in groupoids
for which the diagonal is representable by algebraic spaces.

\begin{lemma}
\label{lemma-representable-diagonal}
Let $S$ be a scheme contained in $\Sch_{fppf}$.
Let $\mathcal{X}$ be a category fibred in groupoids over
$(\Sch/S)_{fppf}$. The following are equivalent:
\begin{enumerate}
\item the diagonal $\mathcal{X} \to \mathcal{X} \times \mathcal{X}$
is representable by algebraic spaces,
\item for every scheme $U$ over $S$, and any
$x, y \in \Ob(\mathcal{X}_U)$ the sheaf
$\mathit{Isom}(x, y)$ is an algebraic space over $U$,
\item for every scheme $U$ over $S$, and any $x \in \Ob(\mathcal{X}_U)$
the associated $1$-morphism $x : (\Sch/U)_{fppf} \to \mathcal{X}$
is representable by algebraic spaces,
\item for every pair of schemes $T_1, T_2$ over $S$, and any
$x_i \in \Ob(\mathcal{X}_{T_i})$, $i = 1, 2$ the $2$-fibre product
$(\Sch/T_1)_{fppf} \times_{x_1, \mathcal{X}, x_2}
(\Sch/T_2)_{fppf}$
is representable by an algebraic space,
\item for every representable category fibred in groupoids $\mathcal{U}$
over $(\Sch/S)_{fppf}$ every $1$-morphism
$\mathcal{U} \to \mathcal{X}$ is representable by algebraic spaces,
\item for every pair $\mathcal{T}_1, \mathcal{T}_2$ of representable
categories fibred in groupoids over $(\Sch/S)_{fppf}$ and any
$1$-morphisms $x_i : \mathcal{T}_i \to \mathcal{X}$, $i = 1, 2$ the
$2$-fibre product $\mathcal{T}_1 \times_{x_1, \mathcal{X}, x_2} \mathcal{T}_2$
is representable by an algebraic space,
\item for every category fibred in groupoids $\mathcal{U}$
over $(\Sch/S)_{fppf}$ which is
representable by an algebraic space every $1$-morphism
$\mathcal{U} \to \mathcal{X}$ is representable by algebraic spaces,
\item for every pair $\mathcal{T}_1, \mathcal{T}_2$ of categories fibred
in groupoids over $(\Sch/S)_{fppf}$ which are representable
by algebraic spaces, and any $1$-morphisms
$x_i : \mathcal{T}_i \to \mathcal{X}$ the
$2$-fibre product $\mathcal{T}_1 \times_{x_1, \mathcal{X}, x_2} \mathcal{T}_2$
is representable by an algebraic space.
\end{enumerate}
\end{lemma}

\begin{proof}
The equivalence of (1) and (2) follows from
Stacks, Lemma \ref{stacks-lemma-isom-as-2-fibre-product}
and the definitions.
Let us prove the equivalence of (1) and (3).
Write $\mathcal{C} = (\Sch/S)_{fppf}$ for the base category.
We will use some of the observations of the proof of the similar
Categories, Lemma \ref{categories-lemma-representable-diagonal-groupoids}.
We will use the symbol $\cong$ to mean ``equivalence of categories fibred
in groupoids over $\mathcal{C} = (\Sch/S)_{fppf}$''.
Assume (1). Suppose given $U$ and $x$ as in (3). For any scheme $V$
and $y \in \Ob(\mathcal{X}_V)$ we see (compare reference above) that
$$
\mathcal{C}/U
\times_{x, \mathcal{X}, y}
\mathcal{C}/V
\cong
(\mathcal{C}/U \times_S V)
\times_{(x, y), \mathcal{X} \times \mathcal{X}, \Delta}
\mathcal{X}
$$
which is representable by an algebraic space by assumption. Conversely,
assume (3). Consider any scheme $U$ over $S$ and a pair $(x, x')$
of objects of $\mathcal{X}$ over $U$. We have to show that
$\mathcal{X} \times_{\Delta, \mathcal{X} \times \mathcal{X}, (x, x')} U$
is representable by an algebraic space. This is clear because
(compare reference above)
$$
\mathcal{X}
\times_{\Delta, \mathcal{X} \times \mathcal{X}, (x, x')}
\mathcal{C}/U
\cong
(\mathcal{C}/U \times_{x, \mathcal{X}, x'} \mathcal{C}/U)
\times_{\mathcal{C}/U \times_S U, \Delta}
\mathcal{C}/U
$$
and the right hand side is representable by an algebraic space by assumption
and the fact that the category of algebraic spaces over $S$ has fibre products
and contains $U$ and $S$.

\medskip\noindent
The equivalences
(3) $\Leftrightarrow$ (4),
(5) $\Leftrightarrow$ (6),
and
(7) $\Leftrightarrow$ (8)
are formal. The equivalences
(3) $\Leftrightarrow$ (5) and
(4) $\Leftrightarrow$ (6)
follow from
Lemma \ref{lemma-representable-by-spaces-morphism-equivalent}.
Assume (3), and let $\mathcal{U} \to \mathcal{X}$ be as in (7).
To prove (7) we have to show that for every scheme $V$ and $1$-morphism
$y : (\Sch/V)_{fppf} \to \mathcal{X}$ the $2$-fibre product
$(\Sch/V)_{fppf} \times_{y, \mathcal{X}} \mathcal{U}$
is representable by an algebraic space. Property (3) tells us
that $y$ is representable by algebraic spaces hence
Lemma \ref{lemma-base-change-by-space-representable-by-space}
implies what we want. Finally, (7) directly implies (3).
\end{proof}

\noindent
In the situation of the lemma, for any $1$-morphism
$x : (\Sch/U)_{fppf} \to \mathcal{X}$ as in the lemma, it makes sense
to say that $x$ has property $\mathcal{P}$, for any property as in
Definition \ref{definition-relative-representable-property}.
In particular this holds for
$\mathcal{P} = $ ``surjective'',
$\mathcal{P} = $ ``smooth'', and
$\mathcal{P} = $ ``\'etale'',
see
Descent on Spaces,
Lemmas \ref{spaces-descent-lemma-descending-property-surjective},
\ref{spaces-descent-lemma-descending-property-smooth}, and
\ref{spaces-descent-lemma-descending-property-etale}.
We will use these three cases in the definitions
of algebraic stacks below.








\section{Stacks in groupoids}
\label{section-stacks}

\noindent
Let $S$ be a scheme contained in $\Sch_{fppf}$.
Recall that a category $p : \mathcal{X} \to (\Sch/S)_{fppf}$
over $(\Sch/S)_{fppf}$ is said to be a
{\it stack in groupoids} (see
Stacks, Definition \ref{stacks-definition-stack-in-groupoids})
if and only if
\begin{enumerate}
\item $p : \mathcal{X} \to (\Sch/S)_{fppf}$ is fibred
in groupoids over $(\Sch/S)_{fppf}$,
\item for all $U \in \Ob((\Sch/S)_{fppf})$,
for all $x, y\in \Ob(\mathcal{X}_U)$ the presheaf
$\mathit{Isom}(x, y)$ is a sheaf on the site $(\Sch/U)_{fppf}$, and
\item for all coverings $\mathcal{U} = \{U_i \to U\}$ in
$(\Sch/S)_{fppf}$, all descent data $(x_i, \phi_{ij})$
for $\mathcal{U}$ are effective.
\end{enumerate}
For examples see
Examples of Stacks,
Section \ref{examples-stacks-section-examples-stacks-in-groupoids}
ff.










\section{Algebraic stacks}
\label{section-algebraic-stacks}

\noindent
Here is the definition of an algebraic stack. We remark that condition
(2) implies we can make sense out of the condition in part (3) that
$(\Sch/U)_{fppf} \to \mathcal{X}$
is smooth and surjective, see discussion following
Lemma \ref{lemma-representable-diagonal}.

\begin{definition}
\label{definition-algebraic-stack}
Let $S$ be a base scheme contained in $\Sch_{fppf}$.
An {\it algebraic stack over $S$} is a category
$$
p : \mathcal{X} \to (\Sch/S)_{fppf}
$$
over $(\Sch/S)_{fppf}$ with the following properties:
\begin{enumerate}
\item The category $\mathcal{X}$ is a stack in groupoids over
$(\Sch/S)_{fppf}$.
\item The diagonal
$\Delta : \mathcal{X} \to \mathcal{X} \times \mathcal{X}$
is representable by algebraic spaces.
\item There exists a scheme $U \in \Ob((\Sch/S)_{fppf})$
and a $1$-morphism $(\Sch/U)_{fppf} \to \mathcal{X}$
which is surjective and smooth\footnote{In future chapters we will denote
this simply $U \to \mathcal{X}$ as is customary in the literature. Another
good alternative would be to formulate this condition as the existence of a
representable category fibred in groupoids $\mathcal{U}$ and a surjective
smooth $1$-morphism $\mathcal{U} \to \mathcal{X}$.}.
\end{enumerate}
\end{definition}

\noindent
There are some differences with other definitions found in the literature.

\medskip\noindent
The first is that we require $\mathcal{X}$ to be a stack in groupoids
in the fppf topology, whereas in many references the \'etale topology is
used. It somehow seems to us that the fppf topology is the natural topology
to work with. In the end the resulting $2$-category of algebraic stacks
ends up being the same. This is explained in
Criteria for Representability, Section \ref{criteria-section-stacks-etale}.

\medskip\noindent
The second is that we only require the diagonal map of $\mathcal{X}$ to be
representable by algebraic spaces, whereas in most references some other
conditions are imposed. Our point of view is to try to prove a certain
number of the results that follow only assuming that the diagonal
of $\mathcal{X}$ be representable by algebraic spaces, and simply add
an additional hypothesis wherever this is necessary. It has the added
benefit that any algebraic space (as defined in
Spaces, Definition \ref{spaces-definition-algebraic-space})
gives rise to an algebraic stack.

\medskip\noindent
The third is that in some papers it is required that there exists a
scheme $U$ and a surjective and \'etale morphism $U \to \mathcal{X}$.
In the groundbreaking paper \cite{DM} where algebraic stacks were first
introduced Deligne and Mumford used this definition and showed that
the moduli stack of stable genus $g > 1$ curves is an algebraic stack
which has an \'etale covering by a scheme. Michael Artin, see
\cite{ArtinVersal}, realized that many
natural results on algebraic stacks generalize to the case where one
only assume a smooth covering by a scheme. Hence our choice above.
To distinguish the two cases one sees the terms ``Deligne-Mumford stack''
and ``Artin stack'' used in the literature. We will reserve the term
``Artin stack'' for later use (insert future reference here), and continue
to use ``algebraic stack'', but we will use ``Deligne-Mumford stack''
to indicate those algebraic stacks which have an \'etale covering by a
scheme.

\begin{definition}
\label{definition-deligne-mumford}
Let $S$ be a scheme contained in $\Sch_{fppf}$.
Let $\mathcal{X}$ be an algebraic stack over $S$.
We say $\mathcal{X}$ is a {\it Deligne-Mumford stack} if there exists
a scheme $U$ and a surjective \'etale morphism
$(\Sch/U)_{fppf} \to \mathcal{X}$.
\end{definition}

\noindent
We will compare our notion of a Deligne-Mumford stack with
the notion as defined in the paper by Deligne and Mumford later
(see insert future reference here).

\medskip\noindent
The category of algebraic stacks over $S$ forms a $2$-category.
Here is the precise definition.

\begin{definition}
\label{definition-morphism-algebraic-stacks}
Let $S$ be a scheme contained in $\Sch_{fppf}$.
The {\it $2$-category of algebraic stacks over $S$} is the
sub $2$-category of the $2$-category of categories fibred in
groupoids over $(\Sch/S)_{fppf}$ (see
Categories,
Definition \ref{categories-definition-categories-fibred-in-groupoids-over-C})
defined as follows:
\begin{enumerate}
\item Its objects are those categories fibred in groupoids
over $(\Sch/S)_{fppf}$ which are algebraic stacks over $S$.
\item Its $1$-morphisms $f : \mathcal{X} \to \mathcal{Y}$ are
any functors of categories over $(\Sch/S)_{fppf}$, as in
Categories, Definition \ref{categories-definition-categories-over-C}.
\item Its $2$-morphisms are transformations between functors
over $(\Sch/S)_{fppf}$, as in
Categories, Definition \ref{categories-definition-categories-over-C}.
\end{enumerate}
\end{definition}

\noindent
In other words this $2$-category is the full sub $2$-category of
$\textit{Cat}/(\Sch/S)_{fppf}$ whose objects are algebraic stacks.
Note that every $2$-morphism is automatically an isomorphism.
Hence this is actually a $(2, 1)$-category and not just a $2$-category.

\medskip\noindent
We will see later (insert future reference here) that this $2$-category
has $2$-fibre products.

\medskip\noindent
Similar to the remark above the $2$-category of algebraic stacks over $S$ is a
full sub $2$-category of the $2$-category of categories fibred in groupoids
over $(\Sch/S)_{fppf}$. It turns out that it is closed under
equivalences. Here is the precise statement.

\begin{lemma}
\label{lemma-equivalent}
Let $S$ be a scheme contained in $\Sch_{fppf}$.
Let $\mathcal{X}$, $\mathcal{Y}$ be categories over $(\Sch/S)_{fppf}$.
Assume $\mathcal{X}$, $\mathcal{Y}$ are equivalent as categories over
$(\Sch/S)_{fppf}$. Then $\mathcal{X}$ is an algebraic stack if and
only if $\mathcal{Y}$ is an algebraic stack. Similarly, $\mathcal{X}$
is a Deligne-Mumford stack if and only if $\mathcal{Y}$ is a Deligne-Mumford
stack.
\end{lemma}

\begin{proof}
Assume $\mathcal{X}$ is an algebraic stack (resp.\ a Deligne-Mumford stack). By
Stacks, Lemma \ref{stacks-lemma-stack-in-groupoids-equivalent}
this implies that $\mathcal{Y}$ is a stack in groupoids over
$\Sch_{fppf}$. Choose an equivalence $f : \mathcal{X} \to \mathcal{Y}$
over $\Sch_{fppf}$. This gives a $2$-commutative diagram
$$
\xymatrix{
\mathcal{X} \ar[r]_f \ar[d]_{\Delta_\mathcal{X}} &
\mathcal{Y} \ar[d]^{\Delta_\mathcal{Y}} \\
\mathcal{X} \times \mathcal{X} \ar[r]^{f \times f} &
\mathcal{Y} \times \mathcal{Y}
}
$$
whose horizontal arrows are equivalences. This implies that
$\Delta_\mathcal{Y}$ is representable by algebraic spaces according to
Lemma \ref{lemma-representable-by-spaces-morphism-equivalent}.
Finally, let $U$ be a scheme over $S$, and let
$x : (\Sch/U)_{fppf} \to \mathcal{X}$ be a $1$-morphism which
is surjective and smooth (resp.\ \'etale). Considering the diagram
$$
\xymatrix{
(\Sch/U)_{fppf} \ar[r]_{\text{id}} \ar[d]_x &
(\Sch/U)_{fppf} \ar[d]^{f \circ x} \\
\mathcal{X} \ar[r]^f &
\mathcal{Y}
}
$$
and applying
Lemma \ref{lemma-property-morphism-equivalent}
we conclude that $f \circ x$ is surjective and smooth (resp.\ \'etale)
as desired.
\end{proof}




\section{Algebraic stacks and algebraic spaces}
\label{section-stacks-spaces}

\noindent
In this section we discuss some simple criteria which imply that an
algebraic stack is an algebraic space. The main result is that this
happens exactly when objects of fibre categories have no nontrivial
automorphisms. This is not a triviality! Before we come to this
we first do a sanity check.

\begin{lemma}
\label{lemma-representable-algebraic}
Let $S$ be a scheme contained in $\Sch_{fppf}$.
\begin{enumerate}
\item A category fibred in groupoids
$p : \mathcal{X} \to (\Sch/S)_{fppf}$
which is representable by an algebraic space is a Deligne-Mumford stack.
\item If $F$ is an algebraic space over $S$, then the associated
category fibred in groupoids
$p : \mathcal{S}_F \to (\Sch/S)_{fppf}$
is a Deligne-Mumford stack.
\item If $X \in \Ob((\Sch/S)_{fppf})$, then
$(\Sch/X)_{fppf} \to (\Sch/S)_{fppf}$ is
a Deligne-Mumford stack.
\end{enumerate}
\end{lemma}

\begin{proof}
It is clear that (2) implies (3).
Parts (1) and (2) are equivalent by Lemma \ref{lemma-equivalent}.
Hence it suffices to prove (2).
First, we note that $\mathcal{S}_F$ is stack in sets since
$F$ is a sheaf (Stacks, Lemma
\ref{stacks-lemma-stack-in-setoids-characterize}).
A fortiori it is a stack in groupoids. Second the diagonal
morphism $\mathcal{S}_F \to \mathcal{S}_F  \times \mathcal{S}_F$
is the same as the morphism $\mathcal{S}_F \to \mathcal{S}_{F \times F}$
which comes from the diagonal of $F$. Hence this is representable
by algebraic spaces according to
Lemma \ref{lemma-morphism-spaces-gives-representable-by-spaces}.
Actually it is even representable (by schemes), as the diagonal of
an algebraic space is representable, but we do not need this.
Let $U$ be a scheme and let $h_U \to F$ be a surjective \'etale morphism.
We may think of this as a surjective \'etale morphism of algebraic spaces.
Hence by
Lemma
\ref{lemma-map-presheaves-representable-by-spaces-transformation-property}
the corresponding $1$-morphism $(\Sch/U)_{fppf} \to \mathcal{S}_F$
is surjective and \'etale.
\end{proof}

\noindent
The following result says that a Deligne-Mumford stack whose inertia
is trivial ``is'' an algebraic space. This lemma will be obsoleted by
the stronger
Proposition \ref{proposition-algebraic-stack-no-automorphisms}
below which says that this holds more generally for algebraic stacks...

\begin{lemma}
\label{lemma-algebraic-stack-no-automorphisms}
Let $S$ be a scheme contained in $\Sch_{fppf}$.
Let $\mathcal{X}$ be an algebraic stack over $S$.
The following are equivalent
\begin{enumerate}
\item $\mathcal{X}$ is a Deligne-Mumford stack and is a stack in setoids,
\item $\mathcal{X}$ is a Deligne-Mumford stack such that the
canonical $1$-morphism $\mathcal{I}_\mathcal{X} \to \mathcal{X}$
is an equivalence, and
\item $\mathcal{X}$ is representable by an algebraic space.
\end{enumerate}
\end{lemma}

\begin{proof}
The equivalence of (1) and (2) follows from
Stacks, Lemma \ref{stacks-lemma-characterize-stack-in-setoids}.
The implication (3) $\Rightarrow$ (1) follows from
Lemma \ref{lemma-representable-algebraic}.
Finally, assume (1). By
Stacks, Lemma \ref{stacks-lemma-stack-in-setoids-characterize}
there exists a sheaf $F$ on $(\Sch/S)_{fppf}$
and an equivalence $j : \mathcal{X} \to \mathcal{S}_F$. By
Lemma \ref{lemma-map-presheaves-representable-by-algebraic-spaces}
the fact that $\Delta_\mathcal{X}$ is representable by algebraic
spaces, means that $\Delta_F : F \to F \times F$
is representable by algebraic spaces.
Let $U$ be a scheme, and let $x : (\Sch/U)_{fppf} \to \mathcal{X}$
be a surjective \'etale morphism. The composition
$j \circ x : (\Sch/U)_{fppf} \to \mathcal{S}_F$
corresponds to a morphism $h_U \to F$ of sheaves. By
Bootstrap, Lemma \ref{bootstrap-lemma-representable-diagonal}
this morphism is representable by algebraic spaces.
Hence by
Lemma \ref{lemma-map-fibred-setoids-property}
we conclude that $h_U \to F$ is surjective and \'etale.
Finally, we apply
Bootstrap, Theorem \ref{bootstrap-theorem-bootstrap}
to see that $F$ is an algebraic space.
\end{proof}

\begin{proposition}
\label{proposition-algebraic-stack-no-automorphisms}
Let $S$ be a scheme contained in $\Sch_{fppf}$.
Let $\mathcal{X}$ be an algebraic stack over $S$.
The following are equivalent
\begin{enumerate}
\item $\mathcal{X}$ is a stack in setoids,
\item the canonical $1$-morphism $\mathcal{I}_\mathcal{X} \to \mathcal{X}$
is an equivalence, and
\item $\mathcal{X}$ is representable by an algebraic space.
\end{enumerate}
\end{proposition}

\begin{proof}
The equivalence of (1) and (2) follows from
Stacks, Lemma \ref{stacks-lemma-characterize-stack-in-setoids}.
The implication (3) $\Rightarrow$ (1) follows from
Lemma \ref{lemma-algebraic-stack-no-automorphisms}.
Finally, assume (1). By
Stacks, Lemma \ref{stacks-lemma-stack-in-setoids-characterize}
there exists an equivalence $j : \mathcal{X} \to \mathcal{S}_F$
where $F$ is a sheaf on $(\Sch/S)_{fppf}$.  By
Lemma \ref{lemma-map-presheaves-representable-by-algebraic-spaces}
the fact that $\Delta_\mathcal{X}$ is representable by algebraic
spaces, means that $\Delta_F : F \to F \times F$
is representable by algebraic spaces.
Let $U$ be a scheme and let $x : (\Sch/U)_{fppf} \to \mathcal{X}$
be a surjective smooth morphism. The composition
$j \circ x : (\Sch/U)_{fppf} \to \mathcal{S}_F$
corresponds to a morphism $h_U \to F$ of sheaves. By
Bootstrap, Lemma \ref{bootstrap-lemma-representable-diagonal}
this morphism is representable by algebraic spaces.
Hence by
Lemma \ref{lemma-map-fibred-setoids-property}
we conclude that $h_U \to F$ is surjective and smooth.
In particular it is surjective, flat and locally of finite presentation
(by
Lemma \ref{lemma-representable-transformations-property-implication}
and the fact that a smooth morphism of algebraic spaces is flat and
locally of finite presentation, see
Morphisms of Spaces,
Lemmas \ref{spaces-morphisms-lemma-smooth-locally-finite-presentation} and
\ref{spaces-morphisms-lemma-smooth-flat}).
Finally, we apply
Bootstrap, Theorem \ref{bootstrap-theorem-final-bootstrap}
to see that $F$ is an algebraic space.
\end{proof}






\section{2-Fibre products of algebraic stacks}
\label{section-2-fibre-products}

\noindent
The $2$-category of algebraic stacks has products and $2$-fibre products.
The first lemma is really a special case of
Lemma \ref{lemma-2-fibre-product}
but its proof is slightly easier.

\begin{lemma}
\label{lemma-product-spaces}
Let $S$ be a scheme contained in $\Sch_{fppf}$.
Let $\mathcal{X}$, $\mathcal{Y}$ be algebraic stacks over $S$.
Then $\mathcal{X} \times_{(\Sch/S)_{fppf}} \mathcal{Y}$
is an algebraic stack, and is a product in the $2$-category of
algebraic stacks over $S$.
\end{lemma}

\begin{proof}
An object of $\mathcal{X} \times_{(\Sch/S)_{fppf}} \mathcal{Y}$
over $T$ is just a pair $(x, y)$ where $x$ is an object of $\mathcal{X}_T$
and $y$ is an object of $\mathcal{Y}_T$. Hence it is immediate from
the definitions that
$\mathcal{X} \times_{(\Sch/S)_{fppf}} \mathcal{Y}$ is a
stack in groupoids. If $(x, y)$ and $(x', y')$ are
two objects of $\mathcal{X} \times_{(\Sch/S)_{fppf}} \mathcal{Y}$
over $T$, then
$$
\mathit{Isom}((x, y), (x', y')) =
\mathit{Isom}(x, x') \times \mathit{Isom}(y, y').
$$
Hence it follows from the equivalences in
Lemma \ref{lemma-representable-diagonal}
and the fact that the category of algebraic spaces has products
that the diagonal of $\mathcal{X} \times_{(\Sch/S)_{fppf}} \mathcal{Y}$
is representable by algebraic spaces.
Finally, suppose that $U, V \in \Ob((\Sch/S)_{fppf})$,
and let $x, y$ be surjective smooth morphisms
$x : (\Sch/U)_{fppf} \to \mathcal{X}$,
$y : (\Sch/V)_{fppf} \to \mathcal{Y}$.
Note that
$$
(\Sch/U \times_S V)_{fppf} =
(\Sch/U)_{fppf}
\times_{(\Sch/S)_{fppf}} (\Sch/V)_{fppf}.
$$
The object $(\text{pr}_U^*x, \text{pr}_V^*y)$ of
$\mathcal{X} \times_{(\Sch/S)_{fppf}} \mathcal{Y}$ over
$(\Sch/U \times_S V)_{fppf}$ thus defines a $1$-morphism
$$
(\Sch/U \times_S V)_{fppf}
\longrightarrow
\mathcal{X} \times_{(\Sch/S)_{fppf}} \mathcal{Y}
$$
which is the composition of base changes of $x$ and $y$, hence
is surjective and smooth, see
Lemmas \ref{lemma-base-change-representable-transformations-property} and
\ref{lemma-composition-representable-transformations-property}.
We conclude that $\mathcal{X} \times_{(\Sch/S)_{fppf}} \mathcal{Y}$
is indeed an algebraic stack. We omit the verification that it
really is a product.
\end{proof}

\begin{lemma}
\label{lemma-2-fibre-product-general}
Let $S$ be a scheme contained in $\Sch_{fppf}$.
Let $\mathcal{Z}$ be a stack in groupoids over $(\Sch/S)_{fppf}$
whose diagonal is representable by algebraic spaces.
Let $\mathcal{X}$, $\mathcal{Y}$ be algebraic stacks over $S$.
Let $f : \mathcal{X} \to \mathcal{Z}$, $g : \mathcal{Y} \to \mathcal{Z}$
be $1$-morphisms of stacks in groupoids. Then the $2$-fibre product
$\mathcal{X} \times_{f, \mathcal{Z}, g} \mathcal{Y}$ is an algebraic stack.
\end{lemma}

\begin{proof}
We have to check conditions (1), (2), and (3) of
Definition \ref{definition-algebraic-stack}.
The first condition follows from
Stacks, Lemma \ref{stacks-lemma-2-product-stacks-in-groupoids}.

\medskip\noindent
The second condition we have to check is that the $\mathit{Isom}$-sheaves
are representable by algebraic spaces. To do this, suppose that
$T$ is a scheme over $S$, and $u, v$ are objects of
$(\mathcal{X} \times_{f, \mathcal{Z}, g} \mathcal{Y})_T$.
By our construction of $2$-fibre products (which goes all the way
back to
Categories, Lemma \ref{categories-lemma-2-product-categories-over-C})
we may write $u = (x, y, \alpha)$ and $v = (x', y', \alpha')$.
Here $\alpha : f(x) \to g(y)$ and similarly for $\alpha'$.
Then it is clear that
$$
\xymatrix{
\mathit{Isom}(u, v) \ar[d] \ar[rr] & &
\mathit{Isom}(y, y') \ar[d]^{\phi \mapsto g(\phi) \circ \alpha} \\
\mathit{Isom}(x, x') \ar[rr]^-{\psi \mapsto \alpha' \circ f(\psi)} & &
\mathit{Isom}(f(x), g(y'))
}
$$
is a cartesian diagram of sheaves on $(\Sch/T)_{fppf}$.
Since by assumption the sheaves
$\mathit{Isom}(y, y')$, $\mathit{Isom}(x, x')$, $\mathit{Isom}(f(x), g(y'))$
are algebraic spaces (see
Lemma \ref{lemma-representable-diagonal})
we see that $\mathit{Isom}(u, v)$
is an algebraic space.

\medskip\noindent
Let $U, V \in \Ob((\Sch/S)_{fppf})$,
and let $x, y$ be surjective smooth morphisms
$x : (\Sch/U)_{fppf} \to \mathcal{X}$,
$y : (\Sch/V)_{fppf} \to \mathcal{Y}$.
Consider the morphism
$$
(\Sch/U)_{fppf}
\times_{f \circ x, \mathcal{Z}, g \circ y}
(\Sch/V)_{fppf}
\longrightarrow
\mathcal{X} \times_{f, \mathcal{Z}, g} \mathcal{Y}.
$$
As the diagonal of $\mathcal{Z}$ is representable by algebraic spaces
the source of this arrow is representable by an algebraic space $F$, see
Lemma \ref{lemma-representable-diagonal}.
Moreover, the morphism is the composition
of base changes of $x$ and $y$, hence surjective and smooth, see
Lemmas \ref{lemma-base-change-representable-transformations-property} and
\ref{lemma-composition-representable-transformations-property}.
Choosing a scheme $W$ and a surjective \'etale morphism $W \to F$
we see that the composition of the displayed $1$-morphism
with the corresponding $1$-morphism
$$
(\Sch/W)_{fppf}
\longrightarrow
(\Sch/U)_{fppf}
\times_{f \circ x, \mathcal{Z}, g \circ y}
(\Sch/V)_{fppf}
$$
is surjective and smooth which proves the last condition.
\end{proof}

\begin{lemma}
\label{lemma-2-fibre-product}
Let $S$ be a scheme contained in $\Sch_{fppf}$.
Let $\mathcal{X}, \mathcal{Y}, \mathcal{Z}$ be algebraic stacks over $S$.
Let $f : \mathcal{X} \to \mathcal{Z}$, $g : \mathcal{Y} \to \mathcal{Z}$
be $1$-morphisms of algebraic stacks. Then the $2$-fibre product
$\mathcal{X} \times_{f, \mathcal{Z}, g} \mathcal{Y}$ is an algebraic stack.
It is also the $2$-fibre product in the $2$-category of algebraic stacks
over $(\Sch/S)_{fppf}$.
\end{lemma}

\begin{proof}
The fact that $\mathcal{X} \times_{f, \mathcal{Z}, g} \mathcal{Y}$ is an
algebraic stack follows from the stronger
Lemma \ref{lemma-2-fibre-product-general}.
The fact that $\mathcal{X} \times_{f, \mathcal{Z}, g} \mathcal{Y}$
is a $2$-fibre product in the $2$-category of algebraic stacks over $S$
follows formally from the fact that the $2$-category of algebraic stacks
over $S$ is a full sub $2$-category of the $2$-category of stacks in
groupoids over $(\Sch/S)_{fppf}$.
\end{proof}







\section{Algebraic stacks, overhauled}
\label{section-overhaul}

\noindent
Some basic results on algebraic stacks.

\begin{lemma}
\label{lemma-lift-morphism-presentations}
Let $S$ be a scheme contained in $\Sch_{fppf}$.
Let $f : \mathcal{X} \to \mathcal{Y}$ be a $1$-morphism of algebraic
stacks over $S$.
Let $V \in \Ob((\Sch/S)_{fppf})$.
Let $y : (\Sch/V)_{fppf} \to \mathcal{Y}$ be surjective and smooth.
Then there exists an object $U \in \Ob((\Sch/S)_{fppf})$
and a $2$-commutative diagram
$$
\xymatrix{
(\Sch/U)_{fppf} \ar[r]_a \ar[d]_x &
(\Sch/V)_{fppf} \ar[d]^y \\
\mathcal{X} \ar[r]^f & \mathcal{Y}
}
$$
with $x$ surjective and smooth.
\end{lemma}

\begin{proof}
First choose $W \in \Ob((\Sch/S)_{fppf})$ and a surjective
smooth $1$-morphism $z : (\Sch/W)_{fppf} \to \mathcal{X}$.
As $\mathcal{Y}$ is an algebraic stack we may choose an equivalence
$$
j :
\mathcal{S}_F
\longrightarrow
(\Sch/W)_{fppf}
\times_{f \circ z, \mathcal{Y}, y}
(\Sch/V)_{fppf}
$$
where $F$ is an algebraic space. By
Lemma \ref{lemma-base-change-representable-transformations-property}
the morphism
$\mathcal{S}_F \to (\Sch/W)_{fppf}$ is surjective and smooth
as a base change of $y$. Hence by
Lemma \ref{lemma-composition-representable-transformations-property}
we see that $\mathcal{S}_F \to \mathcal{X}$ is surjective and smooth.
Choose an object $U \in \Ob((\Sch/S)_{fppf})$
and a surjective \'etale morphism $U \to F$. Then applying
Lemma \ref{lemma-composition-representable-transformations-property}
once more we obtain the desired properties.
\end{proof}

\noindent
This lemma is a generalization of
Proposition \ref{proposition-algebraic-stack-no-automorphisms}.

\begin{lemma}
\label{lemma-characterize-representable-by-algebraic-spaces}
Let $S$ be a scheme contained in $\Sch_{fppf}$.
Let $f : \mathcal{X} \to \mathcal{Y}$ be a $1$-morphism of algebraic
stacks over $S$. The following are equivalent:
\begin{enumerate}
\item for $U \in \Ob((\Sch/S)_{fppf})$
the functor $f : \mathcal{X}_U \to \mathcal{Y}_U$ is faithful,
\item the functor $f$ is faithful, and
\item $f$ is representable by algebraic spaces.
\end{enumerate}
\end{lemma}

\begin{proof}
Parts (1) and (2) are equivalent by general properties of $1$-morphisms
of categories fibred in groupoids, see
Categories, Lemma \ref{categories-lemma-equivalence-fibred-categories}.
We see that (3) implies (2) by
Lemma \ref{lemma-criterion-map-representable-spaces-fibred-in-groupoids}.
Finally, assume (2).
Let $U$ be a scheme. Let $y \in \Ob(\mathcal{Y}_U)$.
We have to prove that
$$
\mathcal{W} = (\Sch/U)_{fppf} \times_{y, \mathcal{Y}} \mathcal{X}
$$
is representable by an algebraic space over $U$. Since
$(\Sch/U)_{fppf}$ is an algebraic stack we see from
Lemma \ref{lemma-2-fibre-product}
that $\mathcal{W}$ is an algebraic stack.
On the other hand the explicit description of objects of $\mathcal{W}$
as triples $(V, x, \alpha : y(V) \to f(x))$ and the fact that $f$ is
faithful, shows that the fibre categories of $\mathcal{W}$ are setoids. Hence
Proposition \ref{proposition-algebraic-stack-no-automorphisms}
guarantees that $\mathcal{W}$ is representable by an algebraic space.
\end{proof}

\begin{lemma}
\label{lemma-smooth-surjective-morphism-implies-algebraic}
Let $S$ be a scheme contained in $\Sch_{fppf}$.
Let $u : \mathcal{U} \to \mathcal{X}$ be a $1$-morphism of
stacks in groupoids over $(\Sch/S)_{fppf}$. If
\begin{enumerate}
\item $\mathcal{U}$ is representable by an algebraic space, and
\item $u$ is representable by algebraic spaces, surjective and smooth,
\end{enumerate}
then $\mathcal X$ is an algebraic stack over $S$.
\end{lemma}

\begin{proof}
We have to show that $\Delta : \mathcal{X} \to \mathcal{X} \times \mathcal{X}$
is representable by algebraic spaces, see
Definition \ref{definition-algebraic-stack}.
Given two schemes $T_1$, $T_2$ over $S$ denote
$\mathcal{T}_i = (\Sch/T_i)_{fppf}$ the associated representable
fibre categories. Suppose given $1$-morphisms
$f_i : \mathcal{T}_i \to \mathcal{X}$.
According to
Lemma \ref{lemma-representable-diagonal}
it suffices to prove that the $2$-fibered
product $\mathcal{T}_1 \times_\mathcal{X} \mathcal{T}_2$
is representable by an algebraic space. By
Stacks, Lemma
\ref{stacks-lemma-2-fibre-product-stacks-in-setoids-over-stack-in-groupoids}
this is in any case a stack in setoids. Thus
$\mathcal{T}_1 \times_\mathcal{X} \mathcal{T}_2$ corresponds
to some sheaf $F$ on $(\Sch/S)_{fppf}$, see
Stacks, Lemma \ref{stacks-lemma-stack-in-setoids-characterize}.
Let $U$ be the algebraic space which represents $\mathcal{U}$.
By assumption
$$
\mathcal{T}_i' = \mathcal{U} \times_{u, \mathcal{X}, f_i} \mathcal{T}_i
$$
is representable by an algebraic space $T'_i$ over $S$. Hence
$\mathcal{T}_1' \times_\mathcal{U} \mathcal{T}_2'$ is representable
by the algebraic space $T'_1 \times_U T'_2$.
Consider the commutative diagram
$$
\xymatrix{
&
\mathcal{T}_1 \times_{\mathcal X} \mathcal{T}_2 \ar[rr]\ar'[d][dd] & &
\mathcal{T}_1 \ar[dd] \\
\mathcal{T}_1' \times_\mathcal{U} \mathcal{T}_2' \ar[ur]\ar[rr]\ar[dd] & &
\mathcal{T}_1' \ar[ur]\ar[dd] \\
&
\mathcal{T}_2 \ar'[r][rr] & &
\mathcal X \\
\mathcal{T}_2' \ar[rr]\ar[ur] & &
\mathcal{U} \ar[ur] }
$$
In this diagram the bottom square, the right square, the back square, and
the front square are $2$-fibre products. A formal argument then shows
that $\mathcal{T}_1' \times_\mathcal{U} \mathcal{T}_2' \to
\mathcal{T}_1 \times_{\mathcal X} \mathcal{T}_2$
is the ``base change'' of $\mathcal{U} \to \mathcal{X}$, more precisely
the diagram
$$
\xymatrix{
\mathcal{T}_1' \times_\mathcal{U} \mathcal{T}_2' \ar[d] \ar[r] &
\mathcal{U} \ar[d] \\
\mathcal{T}_1 \times_{\mathcal X} \mathcal{T}_2 \ar[r] &
\mathcal{X}
}
$$
is a $2$-fibre square.
Hence $T'_1 \times_U T'_2 \to F$ is representable by algebraic spaces,
smooth, and surjective, see
Lemmas \ref{lemma-map-fibred-setoids-representable-algebraic-spaces},
\ref{lemma-base-change-representable-by-spaces},
\ref{lemma-map-fibred-setoids-property}, and
\ref{lemma-base-change-representable-transformations-property}.
Therefore $F$ is an algebraic space by
Bootstrap, Theorem \ref{bootstrap-theorem-final-bootstrap}
and we win.
\end{proof}

\noindent
An application of
Lemma \ref{lemma-smooth-surjective-morphism-implies-algebraic}
is that something which is an algebraic space over an algebraic stack
is an algebraic stack. This is the analogue of
Bootstrap, Lemma \ref{bootstrap-lemma-representable-by-spaces-over-space}.
Actually, it suffices to assume the morphism
$\mathcal{X} \to \mathcal{Y}$ is ``algebraic'', as we will see in
Criteria for Representability,
Lemma \ref{criteria-lemma-algebraic-morphism-to-algebraic}.

\begin{lemma}
\label{lemma-representable-morphism-to-algebraic}
Let $S$ be a scheme contained in $\Sch_{fppf}$.
Let $\mathcal{X} \to \mathcal{Y}$ be a morphism of stacks in groupoids
over $(\Sch/S)_{fppf}$. Assume that
\begin{enumerate}
\item $\mathcal{X} \to \mathcal{Y}$ is representable by algebraic spaces, and
\item $\mathcal{Y}$ is an algebraic stack over $S$.
\end{enumerate}
Then $\mathcal{X}$ is an algebraic stack over $S$.
\end{lemma}

\begin{proof}
Let $\mathcal{V} \to \mathcal{Y}$ be a surjective smooth $1$-morphism
from a representable stack in groupoids to $\mathcal{Y}$. This exists by
Definition \ref{definition-algebraic-stack}.
Then the $2$-fibre product
$\mathcal{U} = \mathcal{V} \times_{\mathcal Y} \mathcal X$
is representable by an algebraic space by
Lemma \ref{lemma-base-change-by-space-representable-by-space}.
The $1$-morphism $\mathcal{U} \to \mathcal X$ is representable by algebraic
spaces, smooth, and surjective, see
Lemmas \ref{lemma-base-change-representable-by-spaces} and
\ref{lemma-base-change-representable-transformations-property}.
By
Lemma \ref{lemma-smooth-surjective-morphism-implies-algebraic}
we conclude that $\mathcal{X}$ is an algebraic stack.
\end{proof}

\begin{lemma}
\label{lemma-open-fibred-category-is-algebraic}
\begin{reference}
Removing the hypothesis that $j$ is a monomorphism was observed
in an email from Matthew Emerton dates June 15, 2016
\end{reference}
Let $S$ be a scheme contained in $\Sch_{fppf}$.
Let $j : \mathcal X \to \mathcal Y$ be a $1$-morphism of
categories fibred in groupoids over $(\Sch/S)_{fppf}$.
Assume $j$ is representable by algebraic spaces.
Then, if $\mathcal{Y}$ is a stack in groupoids
(resp.\ an algebraic stack), so is $\mathcal{X}$.
\end{lemma}

\begin{proof}
The statement on algebraic stacks will follow from the statement on
stacks in groupoids by Lemma \ref{lemma-representable-morphism-to-algebraic}.
If $j$ is representable by algebraic spaces, then $j$ is
faithful on fibre categories and for each $U$ and each
$y \in \Ob(\mathcal{Y}_U)$ the presheaf
$$
(h : V \to U)
\longmapsto
\{(x, \phi) \mid x \in \Ob(\mathcal{X}_V), \phi : h^*y \to f(x)\}/\cong
$$
is an algebraic space over $U$. See
Lemma \ref{lemma-criterion-map-representable-spaces-fibred-in-groupoids}.
In particular this presheaf is a sheaf and the conclusion follows
from Stacks, Lemma \ref{stacks-lemma-relative-sheaf-over-stack-is-stack}.
\end{proof}



\section{From an algebraic stack to a presentation}
\label{section-stack-to-presentation}

\noindent
Given an algebraic stack over $S$ we obtain a groupoid in algebraic spaces
over $S$ whose associated quotient stack is the algebraic stack.

\medskip\noindent
Recall that if $(U, R, s, t, c)$ is a groupoid in algebraic spaces over $S$
then $[U/R]$ denotes the quotient stack associated to this datum, see
Groupoids in Spaces,
Definition \ref{spaces-groupoids-definition-quotient-stack}.
In general $[U/R]$ is {\bf not} an algebraic stack. In particular the
stack $[U/R]$ occurring in the following lemma is in general not
algebraic.

\begin{lemma}
\label{lemma-map-space-into-stack}
Let $S$ be a scheme contained in $\Sch_{fppf}$.
Let $\mathcal{X}$ be an algebraic stack over $S$.
Let $\mathcal{U}$ be an algebraic stack over $S$ which
is representable by an algebraic space.
Let $f : \mathcal{U} \to \mathcal{X}$ be a 1-morphism. Then
\begin{enumerate}
\item the $2$-fibre product
$\mathcal{R} = \mathcal{U} \times_{f, \mathcal{X}, f} \mathcal{U}$
is representable by an algebraic space,
\item there is a canonical equivalence
$$
\mathcal{U} \times_{f, \mathcal{X}, f} \mathcal{U}
\times_{f, \mathcal{X}, f} \mathcal{U} =
\mathcal{R} \times_{\text{pr}_1, \mathcal{U}, \text{pr}_0} \mathcal{R},
$$
\item the projection $\text{pr}_{02}$ induces via (2) a $1$-morphism
$$
\text{pr}_{02} :
\mathcal{R} \times_{\text{pr}_1, \mathcal{U}, \text{pr}_0} \mathcal{R}
\longrightarrow
\mathcal{R}
$$
\item let $U$, $R$ be the algebraic spaces representing
$\mathcal{U}, \mathcal{R}$ and $t, s : R \to U$ and
$c : R \times_{s, U, t} R \to R$ are the morphisms corresponding
to the $1$-morphisms
$\text{pr}_0, \text{pr}_1 : \mathcal{R} \to \mathcal{U}$
and
$\text{pr}_{02} :
\mathcal{R} \times_{\text{pr}_1, \mathcal{U}, \text{pr}_0} \mathcal{R} \to
\mathcal{R}$ above, then the quintuple $(U, R, s, t, c)$ is a groupoid in
algebraic spaces over $S$,
\item the morphism $f$ induces a canonical $1$-morphism
$f_{can} : [U/R] \to \mathcal{X}$
of stacks in groupoids over $(\Sch/S)_{fppf}$, and
\item the $1$-morphism $f_{can} : [U/R] \to \mathcal{X}$ is fully faithful.
\end{enumerate}
\end{lemma}

\begin{proof}
Proof of (1). By definition $\Delta_\mathcal{X}$ is representable
by algebraic spaces so
Lemma \ref{lemma-representable-diagonal}
applies to show that $\mathcal{U} \to \mathcal{X}$ is representable
by algebraic spaces. Hence the result follows from
Lemma \ref{lemma-base-change-by-space-representable-by-space}.

\medskip\noindent
Let $T$ be a scheme over $S$. By construction of the $2$-fibre product (see
Categories, Lemma \ref{categories-lemma-2-product-categories-over-C})
we see that the objects of the fibre category $\mathcal{R}_T$
are triples $(a, b, \alpha)$ where $a, b \in \Ob(\mathcal{U}_T)$
and $\alpha : f(a) \to f(b)$
is a morphism in the fibre category $\mathcal{X}_T$.

\medskip\noindent
Proof of (2). The equivalence comes from repeatedly applying
Categories, Lemmas \ref{categories-lemma-associativity-2-fibre-product} and
\ref{categories-lemma-2-fibre-product-erase-factor}.
Let us identify
$\mathcal{U} \times_\mathcal{X} \mathcal{U} \times_\mathcal{X} \mathcal{U}$
with
$(\mathcal{U} \times_\mathcal{X} \mathcal{U})
\times_\mathcal{X} \mathcal{U}$.
If $T$ is a scheme over $S$, then on fibre categories over $T$
this equivalence maps the object
$((a, b, \alpha), c, \beta)$ on the left hand side
to the object $((a, b, \alpha), (b, c, \beta))$ of the right hand side.

\medskip\noindent
Proof of (3). The $1$-morphism $\text{pr}_{02}$ is constructed in the proof of
Categories, Lemma \ref{categories-lemma-triple-2-fibre-product-pr02}.
In terms of the description of objects of the fibre category
above we see that $((a, b, \alpha), (b, c, \beta))$
maps to $(a, c, \beta \circ \alpha)$.

\medskip\noindent
Unfortunately, this is {\it not compatible} with our conventions on
groupoids where we always have $j = (t, s) : R \to U$, and we ``think''
of a $T$-valued point $r$ of $R$ as a morphism $r : s(r) \to t(r)$.
However, this does not affect the proof of (4), since the opposite of
a groupoid is a groupoid. But in the proof of (5) it is responsible
for the inverses in the displayed formula below.

\medskip\noindent
Proof of (4). Recall that the sheaf $U$ is isomorphic to the sheaf
$T \mapsto \Ob(\mathcal{U}_T)/\!\cong$, and
similarly for $R$, see
Lemma \ref{lemma-characterize-representable-by-space}.
It follows from
Categories,
Lemma \ref{categories-lemma-category-fibred-setoids-presheaves-products}
that this description is compatible with $2$-fibre products
so we get a similar matching of
$\mathcal{R} \times_{\text{pr}_1, \mathcal{U}, \text{pr}_0} \mathcal{R}$
and $R \times_{s, U, t} R$.
The morphisms $t, s : R \to U$ and $c : R \times_{s, U, t} R \to R$
we get from the general equality (\ref{equation-morphisms-spaces}).
Explicitly these maps are the transformations of functors that come
from letting $\text{pr}_0$, $\text{pr}_0$, $\text{pr}_{02}$
act on isomorphism classes of objects of fibre categories.
Hence to show that we obtain a groupoid in algebraic
spaces it suffices to show that for every scheme $T$ over $S$
the structure
$$
(\Ob(\mathcal{U}_T)/\!\cong,
\Ob(\mathcal{R}_T)/\!\cong,
\text{pr}_1, \text{pr}_0, \text{pr}_{02})
$$
is a groupoid which is clear from our description of objects of
$\mathcal{R}_T$ above.

\medskip\noindent
Proof of (5). We will eventually apply
Groupoids in Spaces,
Lemma \ref{spaces-groupoids-lemma-quotient-stack-2-coequalizer}
to obtain the functor $[U/R] \to \mathcal{X}$.
Consider the $1$-morphism $f : \mathcal{U} \to \mathcal{X}$.
We have a $2$-arrow $\tau : f \circ \text{pr}_1 \to f \circ \text{pr}_0$
by definition of $\mathcal{R}$ as the $2$-fibre product.
Namely, on an object $(a, b, \alpha)$ of $\mathcal{R}$ over $T$ it is
the map $\alpha^{-1} : b \to a$. We claim that
$$
\tau \circ \text{id}_{\text{pr}_{02}} =
(\tau \star \text{id}_{\text{pr}_0})
\circ
(\tau \star \text{id}_{\text{pr}_1}).
$$
This identity says that given an object
$((a, b, \alpha), (b, c, \beta))$ of
$\mathcal{R} \times_{\text{pr}_1, \mathcal{U}, \text{pr}_0} \mathcal{R}$
over $T$, then the composition of
$$
\xymatrix{
c \ar[r]^{\beta^{-1}} & b \ar[r]^{\alpha^{-1}} & a
}
$$
is the same as the arrow $(\beta \circ \alpha)^{-1} : a \to c$. This is
clearly true, hence the claim holds. In this way we see that all the
assumption of
Groupoids in Spaces,
Lemma \ref{spaces-groupoids-lemma-quotient-stack-2-coequalizer}
are satisfied for the structure
$(\mathcal{U}, \mathcal{R}, \text{pr}_0, \text{pr}_1, \text{pr}_{02})$
and the $1$-morphism $f$ and the $2$-morphism $\tau$.
Except, to apply the lemma we need to prove this holds
for the structure $(\mathcal{S}_U, \mathcal{S}_R, s, t, c)$
with suitable morphisms.

\medskip\noindent
Now there should be some general abstract nonsense
argument which transfer these data between the two, but it seems to
be quite long. Instead, we use the following trick.
Pick a quasi-inverse $j^{-1} : \mathcal{S}_U \to \mathcal{U}$
of the canonical equivalence $j : \mathcal{U} \to \mathcal{S}_U$ which comes
from $U(T) = \Ob(\mathcal{U}_T)/\!\!\cong$.
This just means that for every scheme $T/S$ and every
object $a \in \mathcal{U}_T$ we have picked out a particular
element of its isomorphism class, namely $j^{-1}(j(a))$.
Using $j^{-1}$ we may therefore see $\mathcal{S}_U$
as a subcategory of $\mathcal{U}$. Having chosen this subcategory
we can consider those objects $(a, b, \alpha)$ of $\mathcal{R}_T$
such that $a, b$ are objects of $(\mathcal{S}_U)_T$, i.e., such
that $j^{-1}(j(a)) = a$ and $j^{-1}(j(b)) = b$. Then it is clear that
this forms a subcategory of $\mathcal{R}$ which maps isomorphically
to $\mathcal{S}_R$ via the canonical equivalence
$\mathcal{R} \to \mathcal{S}_R$. Moreover, this is clearly compatible
with forming the $2$-fibre product
$\mathcal{R} \times_{\text{pr}_1, \mathcal{U}, \text{pr}_0} \mathcal{R}$.
Hence we see that we may simply restrict
$f$ to $\mathcal{S}_U$ and restrict $\tau$ to a transformation
between functors $\mathcal{S}_R \to \mathcal{X}$. Hence it is clear that
the displayed equality of
Groupoids in Spaces,
Lemma \ref{spaces-groupoids-lemma-quotient-stack-2-coequalizer}
holds since it holds even as an equality of transformations of functors
$\mathcal{R} \times_{\text{pr}_1, \mathcal{U}, \text{pr}_0} \mathcal{R}
\to \mathcal{X}$ before restricting to the subcategory
$\mathcal{S}_{R \times_{s, U, t} R}$.

\medskip\noindent
This proves that
Groupoids in Spaces,
Lemma \ref{spaces-groupoids-lemma-quotient-stack-2-coequalizer}
applies and we get our desired morphism of stacks
$f_{can} : [U/R] \to \mathcal{X}$. We briefly spell out how
$f_{can}$ is defined in this special case.
On an object $a$ of $\mathcal{S}_U$ over $T$
we have $f_{can}(a) = f(a)$, where we think of
$\mathcal{S}_U \subset \mathcal{U}$ by the chosen embedding above.
If $a, b$ are objects of $\mathcal{S}_U$ over $T$, then a morphism
$\varphi : a \to b$ in $[U/R]$ is by definition an object of the
form $\varphi = (b, a, \alpha)$ of $\mathcal{R}$ over $T$. (Note
switch.) And the rule in the proof of
Groupoids in Spaces,
Lemma \ref{spaces-groupoids-lemma-quotient-stack-2-coequalizer}
is that
\begin{equation}
\label{equation-on-morphisms}
f_{can}(\varphi) = \Big(f(a) \xrightarrow{\alpha^{-1}} f(b)\Big).
\end{equation}
Proof of (6). Both $[U/R]$ and $\mathcal{X}$ are stacks.
Hence given a scheme $T/S$ and objects $a, b$ of $[U/R]$
over $T$ we obtain a transformation of fppf sheaves
$$
\mathit{Isom}(a, b) \longrightarrow \mathit{Isom}(f_{can}(a), f_{can}(b))
$$
on $(\Sch/T)_{fppf}$. We have to show that this is an
isomorphism. We may work fppf locally on $T$, hence we may assume that
$a, b$ come from morphisms $a, b : T \to U$. By the embedding
$\mathcal{S}_U \subset \mathcal{U}$ above we may also think of $a, b$ as
objects of $\mathcal{U}$ over $T$. In
Groupoids in Spaces,
Lemma \ref{spaces-groupoids-lemma-quotient-stack-morphisms}
we have seen that the left hand sheaf is represented by the algebraic space
$$
R \times_{(t, s), U \times_S U, (b, a)} T
$$
over $T$. On the other hand, the right hand side is by
Stacks, Lemma \ref{stacks-lemma-isom-as-2-fibre-product}
equal to the sheaf associated to the following stack in setoids:
$$
\mathcal{X}
\times_{\mathcal{X} \times \mathcal{X}, (f \circ b, f \circ a)} T =
\mathcal{X}
\times_{\mathcal{X} \times \mathcal{X}, (f, f)}
(\mathcal{U} \times \mathcal{U})
\times_{\mathcal{U} \times \mathcal{U}, (b, a)} T =
\mathcal{R}
\times_{(\text{pr}_0, \text{pr}_1), \mathcal{U} \times \mathcal{U}, (b, a)} T
$$
which is representable by the fibre product displayed above.
At this point we have shown that the two $\mathit{Isom}$-sheaves
are isomorphic. Our $1$-morphism $f_{can} : [U/R] \to \mathcal{X}$ induces
this isomorphism on $\mathit{Isom}$-sheaves by
Equation (\ref{equation-on-morphisms}).
\end{proof}

\noindent
We can use the previous very abstract lemma to produce
presentations.

\begin{lemma}
\label{lemma-stack-presentation}
Let $S$ be a scheme contained in $\Sch_{fppf}$.
Let $\mathcal{X}$ be an algebraic stack over $S$.
Let $U$ be an algebraic space over $S$.
Let $f : \mathcal{S}_U \to \mathcal{X}$ be a surjective smooth morphism.
Let $(U, R, s, t, c)$ be the groupoid in algebraic spaces
and $f_{can} : [U/R] \to \mathcal{X}$ be the result of applying
Lemma \ref{lemma-map-space-into-stack}
to $U$ and $f$. Then
\begin{enumerate}
\item the morphisms $s$, $t$ are smooth, and
\item the $1$-morphism $f_{can} : [U/R] \to \mathcal{X}$
is an equivalence.
\end{enumerate}
\end{lemma}

\begin{proof}
The morphisms $s, t$ are smooth by
Lemmas \ref{lemma-property-morphism-equivalent} and
\ref{lemma-map-presheaves-representable-by-spaces-transformation-property}.
As the $1$-morphism $f$ is smooth and
surjective it is clear that given any scheme $T$ and any object
$a \in \Ob(\mathcal{X}_T)$ there exists a smooth and surjective
morphism $T' \to T$ such that $a|_T'$ comes from an object of
$[U/R]_{T'}$. Since $f_{can} : [U/R] \to \mathcal{X}$
is fully faithful, we deduce that
$[U/R] \to \mathcal{X}$ is essentially surjective as
descent data on objects are effective on both sides, see
Stacks, Lemma \ref{stacks-lemma-characterize-essentially-surjective-when-ff}.
\end{proof}

\begin{remark}
\label{remark-flat-fp-presentation}
If the morphism $f : \mathcal{S}_U \to \mathcal{X}$ of
Lemma \ref{lemma-stack-presentation}
is only assumed surjective, flat and locally of finite presentation, then
it will still be the case that $f_{can} : [U/R] \to \mathcal{X}$ is an
equivalence. In this case the morphisms $s$, $t$ will be flat and
locally of finite presentation, but of course not smooth in general.
\end{remark}

\noindent
Lemma \ref{lemma-stack-presentation}
suggests the following definitions.

\begin{definition}
\label{definition-smooth-groupoid}
Let $S$ be a scheme. Let $B$ be an algebraic space over $S$.
Let $(U, R, s, t, c)$ be a groupoid in algebraic spaces over $B$.
We say $(U, R, s, t, c)$ is a {\it smooth groupoid}\footnote{This terminology
might be a bit confusing: it does not imply that $[U/R]$ is smooth
over anything.}
if $s, t : R \to U$ are smooth morphisms of algebraic spaces.
\end{definition}

\begin{definition}
\label{definition-presentation}
Let $\mathcal{X}$ be an algebraic stack over $S$.
A {\it presentation} of $\mathcal{X}$ is given by a smooth groupoid
$(U, R, s, t, c)$ in algebraic spaces over $S$, and an
equivalence $f : [U/R] \to \mathcal{X}$.
\end{definition}

\noindent
We have seen above that every algebraic stack has a presentation.
Our next task is to show that every smooth groupoid in algebraic
spaces over $S$ gives rise to an algebraic stack.


\section{The algebraic stack associated to a smooth groupoid}
\label{section-smooth-groupoid-gives-algebraic-stack}

\noindent
In this section we start with a smooth groupoid in algebraic spaces
and we show that the associated quotient stack is an algebraic stack.

\begin{lemma}
\label{lemma-diagonal-quotient-stack}
Let $S$ be a scheme contained in $\Sch_{fppf}$.
Let $(U, R, s, t, c)$ be a groupoid in algebraic spaces over $S$.
Then the diagonal of $[U/R]$ is representable by algebraic spaces.
\end{lemma}

\begin{proof}
It suffices to show that the $\mathit{Isom}$-sheaves are algebraic
spaces, see
Lemma \ref{lemma-representable-diagonal}.
This follows from
Bootstrap, Lemma \ref{bootstrap-lemma-quotient-stack-isom}.
\end{proof}

\begin{lemma}
\label{lemma-smooth-quotient-smooth-presentation}
Let $S$ be a scheme contained in $\Sch_{fppf}$.
Let $(U, R, s, t, c)$ be a smooth groupoid in algebraic spaces over $S$.
Then the morphism $\mathcal{S}_U \to [U/R]$ is smooth and surjective.
\end{lemma}

\begin{proof}
Let $T$ be a scheme and let $x : (\Sch/T)_{fppf} \to [U/R]$
be a $1$-morphism. We have to show that the projection
$$
\mathcal{S}_U \times_{[U/R]} (\Sch/T)_{fppf}
\longrightarrow
(\Sch/T)_{fppf}
$$
is surjective and smooth. We already know that the left hand side
is representable by an algebraic space $F$, see
Lemmas \ref{lemma-diagonal-quotient-stack} and
\ref{lemma-representable-diagonal}.
Hence we have to show the corresponding morphism $F \to T$ of
algebraic spaces is surjective and smooth.
Since we are working with properties of morphisms of algebraic
spaces which are local on the target in the fppf topology we
may check this fppf locally on $T$. By construction, there exists
an fppf covering $\{T_i \to T\}$ of $T$ such that
$x|_{(\Sch/T_i)_{fppf}}$ comes from a morphism
$x_i : T_i \to U$. (Note that $F \times_T T_i$ represents the
$2$-fibre product $\mathcal{S}_U \times_{[U/R]} (\Sch/T_i)_{fppf}$
so everything is compatible with the base change via $T_i \to T$.)
Hence we may assume that $x$ comes from $x : T \to U$.
In this case we see that
$$
\mathcal{S}_U \times_{[U/R]} (\Sch/T)_{fppf}
=
(\mathcal{S}_U \times_{[U/R]} \mathcal{S}_U)
\times_{\mathcal{S}_U} (\Sch/T)_{fppf}
=
\mathcal{S}_R \times_{\mathcal{S}_U} (\Sch/T)_{fppf}
$$
The first equality by
Categories, Lemma \ref{categories-lemma-2-fibre-product-erase-factor}
and the second equality by
Groupoids in Spaces,
Lemma \ref{spaces-groupoids-lemma-quotient-stack-2-cartesian}.
Clearly the last $2$-fibre product is represented by the algebraic
space $F = R \times_{s, U, x} T$ and the projection
$R \times_{s, U, x} T \to T$ is smooth as the base change of
the smooth morphism of algebraic spaces $s : R \to U$.
It is also surjective as $s$ has a section (namely the identity
$e : U \to R$ of the groupoid).
This proves the lemma.
\end{proof}

\noindent
Here is the main result of this section.

\begin{theorem}
\label{theorem-smooth-groupoid-gives-algebraic-stack}
Let $S$ be a scheme contained in $\Sch_{fppf}$.
Let $(U, R, s, t, c)$ be a smooth groupoid in algebraic spaces over $S$.
Then the quotient stack $[U/R]$ is an algebraic stack over $S$.
\end{theorem}

\begin{proof}
We check the three conditions of
Definition \ref{definition-algebraic-stack}.
By construction we have that $[U/R]$ is a stack in groupoids
which is the first condition.

\medskip\noindent
The second condition follows from the stronger
Lemma \ref{lemma-diagonal-quotient-stack}.

\medskip\noindent
Finally, we have to show there exists a scheme $W$ over $S$
and a surjective smooth $1$-morphism
$(\Sch/W)_{fppf} \longrightarrow \mathcal{X}$.
First choose $W \in \Ob((\Sch/S)_{fppf})$ and a
surjective \'etale morphism $W \to U$. Note that this
gives a surjective \'etale morphism $\mathcal{S}_W \to \mathcal{S}_U$
of categories fibred in sets, see
Lemma
\ref{lemma-map-presheaves-representable-by-spaces-transformation-property}.
Of course then $\mathcal{S}_W \to \mathcal{S}_U$ is also surjective and
smooth, see
Lemma \ref{lemma-representable-transformations-property-implication}.
Hence $\mathcal{S}_W \to \mathcal{S}_U \to [U/R]$ is surjective
and smooth by a combination of
Lemmas \ref{lemma-smooth-quotient-smooth-presentation} and
\ref{lemma-composition-representable-transformations-property}.
\end{proof}





\section{Change of big site}
\label{section-change-big-site}

\noindent
In this section we briefly discuss what happens when we change big sites.
The upshot is that we can always enlarge the big site at will, hence we
may assume any set of schemes we want to consider is contained in the big
fppf site over which we consider our algebraic space.
We encourage the reader to skip this section.

\medskip\noindent
Pullbacks of stacks is defined in
Stacks, Section \ref{stacks-section-inverse-image}.

\begin{lemma}
\label{lemma-change-big-site}
Suppose given big sites $\Sch_{fppf}$ and $\Sch'_{fppf}$.
Assume that $\Sch_{fppf}$ is contained in $\Sch'_{fppf}$,
see Topologies, Section \ref{topologies-section-change-alpha}.
Let $S$ be an object of $\Sch_{fppf}$.
Let $f : (\Sch'/S)_{fppf} \to (\Sch/S)_{fppf}$ the morphism
of sites corresponding to the inclusion functor
$u : (\Sch/S)_{fppf} \to (\Sch'/S)_{fppf}$.
Let $\mathcal{X}$ be a stack in groupoids over $(\Sch/S)_{fppf}$.
\begin{enumerate}
\item if $\mathcal{X}$ is representable by some
$X \in \Ob((\Sch/S)_{fppf})$, then
$f^{-1}\mathcal{X}$ is representable too, in fact it is representable by the
same scheme $X$, now viewed as an object of $(\Sch'/S)_{fppf}$,
\item if $\mathcal{X}$ is representable by
$F \in \Sh((\Sch/S)_{fppf})$ which is
an algebraic space, then $f^{-1}\mathcal{X}$ is representable
by the algebraic space $f^{-1}F$,
\item if $\mathcal{X}$ is an algebraic stack, then $f^{-1}\mathcal{X}$
is an algebraic stack, and
\item if $\mathcal{X}$ is a Deligne-Mumford stack, then $f^{-1}\mathcal{X}$
is a Deligne-Mumford stack too.
\end{enumerate}
\end{lemma}

\begin{proof}
Let us prove (3). By
Lemma \ref{lemma-stack-presentation}
we may write $\mathcal{X} = [U/R]$ for some smooth
groupoid in algebraic spaces $(U, R, s, t, c)$. By
Groupoids in Spaces,
Lemma \ref{spaces-groupoids-lemma-quotient-stack-change-big-site}
we see that $f^{-1}[U/R] = [f^{-1}U/f^{-1}R]$.
Of course $(f^{-1}U, f^{-1}R, f^{-1}s, f^{-1}t, f^{-1}c)$
is a smooth groupoid in algebraic spaces too. Hence (3) is proved.

\medskip\noindent
Now the other cases (1), (2), (4) each mean that $\mathcal{X}$ has
a presentation $[U/R]$ of a particular kind, and hence translate into the
same kind of presentation for $f^{-1}\mathcal{X} = [f^{-1}U/f^{-1}R]$.
Whence the lemma is proved.
\end{proof}

\noindent
It is not true (in general) that the restriction of an algebraic space
over the bigger site is an algebraic space over the smaller site (simply
by reasons of cardinality). Hence we can only ever use a simple lemma of this
kind to enlarge the base category and never to shrink it.

\begin{lemma}
\label{lemma-fully-faithful}
Suppose $\Sch_{fppf}$ is contained in $\Sch'_{fppf}$.
Let $S$ be an object of $\Sch_{fppf}$. Denote
$\textit{Algebraic-Stacks}/S$ the $2$-category of algebraic stacks over $S$
defined using $\Sch_{fppf}$. Similarly, denote
$\textit{Algebraic-Stacks}'/S$ the $2$-category of algebraic stacks over $S$
defined using $\Sch'_{fppf}$. The rule
$\mathcal{X} \mapsto f^{-1}\mathcal{X}$ of
Lemma \ref{lemma-change-big-site}
defines a functor of $2$-categories
$$
\textit{Algebraic-Stacks}/S \longrightarrow \textit{Algebraic-Stacks}'/S
$$
which defines equivalences of morphism categories
$$
\Mor_{\textit{Algebraic-Stacks}/S}(\mathcal{X}, \mathcal{Y})
\longrightarrow
\Mor_{\textit{Algebraic-Stacks}'/S}(f^{-1}\mathcal{X}, f^{-1}\mathcal{Y})
$$
for every objects $\mathcal{X}, \mathcal{Y}$ of
$\textit{Algebraic-Stacks}/S$. An object
$\mathcal{X}'$ of $\textit{Algebraic-Stacks}'/S$
is equivalence to $f^{-1}\mathcal{X}$ for some
$\mathcal{X}$ in $\textit{Algebraic-Stacks}/S$
if and only if it has a presentation $\mathcal{X} = [U'/R']$
with $U', R'$ isomorphic to $f^{-1}U$, $f^{-1}R$ for some
$U, R \in \textit{Spaces}/S$.
\end{lemma}

\begin{proof}
The statement on morphism categories is a consequence of the more general
Stacks, Lemma \ref{stacks-lemma-bigger-site}.
The characterization of the ``essential image'' follows from the description
of $f^{-1}$ in the proof of
Lemma \ref{lemma-change-big-site}.
\end{proof}


\section{Change of base scheme}
\label{section-change-base-scheme}

\noindent
In this section we briefly discuss what happens when we change base schemes.
The upshot is that given a morphism $S \to S'$ of base schemes, any algebraic
stack over $S$ can be viewed as an algebraic stack over $S'$.

\begin{lemma}
\label{lemma-category-of-spaces-over-smaller-base-scheme}
Let $\Sch_{fppf}$ be a big fppf site.
Let $S \to S'$ be a morphism of this site.
The constructions A and B of
Stacks, Section \ref{stacks-section-localize}
above give isomorphisms of $2$-categories
$$
\left\{
\begin{matrix}
2\text{-category of algebraic}\\
\text{stacks }\mathcal{X}\text{ over }S
\end{matrix}
\right\}
\leftrightarrow
\left\{
\begin{matrix}
2\text{-category of pairs }(\mathcal{X}', f)\text{ consisting of an}\\
\text{algebraic stack }\mathcal{X}'\text{ over }S'\text{ and a morphism}\\
f : \mathcal{X}' \to (\Sch/S)_{fppf}\text{ of algebraic stacks over }S'
\end{matrix}
\right\}
$$
\end{lemma}

\begin{proof}
The statement makes sense as the functor
$j : (\Sch/S)_{fppf} \to (\Sch/S')_{fppf}$
is the localization functor associated to the object $S/S'$
of $(\Sch/S')_{fppf}$. By
Stacks, Lemma \ref{stacks-lemma-localize-stacks}
the only thing to show is that the constructions A and B
preserve the subcategories of algebraic stacks.
For example, if $\mathcal{X} = [U/R]$ then construction A
applied to $\mathcal{X}$ just produces
$\mathcal{X}' = \mathcal{X}$. Conversely, if $\mathcal{X}' = [U'/R']$
the morphism $p$ induces morphisms of algebraic spaces
$U' \to S$ and $R' \to S$, and then $\mathcal{X} = [U'/R']$
but now viewed as a stack over $S$. Hence the lemma is clear.
\end{proof}

\begin{definition}
\label{definition-viewed-as}
Let $\Sch_{fppf}$ be a big fppf site.
Let $S \to S'$ be a morphism of this site.
If $p : \mathcal{X} \to (\Sch/S)_{fppf}$
is an algebraic stack over $S$, then
$\mathcal{X}$ {\it viewed as an algebraic stack over $S'$}
is the algebraic stack
$$
\mathcal{X} \longrightarrow (\Sch/S')_{fppf}
$$
gotten by applying construction A of
Lemma \ref{lemma-category-of-spaces-over-smaller-base-scheme}
to $\mathcal{X}$.
\end{definition}

\noindent
Conversely, what if we start with an algebraic stack $\mathcal{X}'$
over $S'$ and we want to get an algebraic stack over $S$?
Well, then we consider the $2$-fibre product
$$
\mathcal{X}'_S
=
(\Sch/S)_{fppf} \times_{(\Sch/S')_{fppf}} \mathcal{X}'
$$
which is an algebraic stack over $S'$ according to
Lemma \ref{lemma-2-fibre-product}.
Moreover, it comes equipped with a natural $1$-morphism
$p : \mathcal{X}'_S \to (\Sch/S)_{fppf}$ and hence by
Lemma \ref{lemma-category-of-spaces-over-smaller-base-scheme}
it corresponds in a canonical way to an algebraic stack over $S$.

\begin{definition}
\label{definition-change-of-base}
Let $\Sch_{fppf}$ be a big fppf site.
Let $S \to S'$ be a morphism of this site.
Let $\mathcal{X}'$ be an algebraic stack over $S'$.
The {\it change of base of $\mathcal{X}'$} is the
algebraic space $\mathcal{X}'_S$ over $S$ described above.
\end{definition}




\begin{multicols}{2}[\section{Other chapters}]
\noindent
Preliminaries
\begin{enumerate}
\item \hyperref[introduction-section-phantom]{Introduction}
\item \hyperref[conventions-section-phantom]{Conventions}
\item \hyperref[sets-section-phantom]{Set Theory}
\item \hyperref[categories-section-phantom]{Categories}
\item \hyperref[topology-section-phantom]{Topology}
\item \hyperref[sheaves-section-phantom]{Sheaves on Spaces}
\item \hyperref[sites-section-phantom]{Sites and Sheaves}
\item \hyperref[stacks-section-phantom]{Stacks}
\item \hyperref[fields-section-phantom]{Fields}
\item \hyperref[algebra-section-phantom]{Commutative Algebra}
\item \hyperref[brauer-section-phantom]{Brauer Groups}
\item \hyperref[homology-section-phantom]{Homological Algebra}
\item \hyperref[derived-section-phantom]{Derived Categories}
\item \hyperref[simplicial-section-phantom]{Simplicial Methods}
\item \hyperref[more-algebra-section-phantom]{More on Algebra}
\item \hyperref[smoothing-section-phantom]{Smoothing Ring Maps}
\item \hyperref[modules-section-phantom]{Sheaves of Modules}
\item \hyperref[sites-modules-section-phantom]{Modules on Sites}
\item \hyperref[injectives-section-phantom]{Injectives}
\item \hyperref[cohomology-section-phantom]{Cohomology of Sheaves}
\item \hyperref[sites-cohomology-section-phantom]{Cohomology on Sites}
\item \hyperref[dga-section-phantom]{Differential Graded Algebra}
\item \hyperref[dpa-section-phantom]{Divided Power Algebra}
\item \hyperref[sdga-section-phantom]{Differential Graded Sheaves}
\item \hyperref[hypercovering-section-phantom]{Hypercoverings}
\end{enumerate}
Schemes
\begin{enumerate}
\setcounter{enumi}{25}
\item \hyperref[schemes-section-phantom]{Schemes}
\item \hyperref[constructions-section-phantom]{Constructions of Schemes}
\item \hyperref[properties-section-phantom]{Properties of Schemes}
\item \hyperref[morphisms-section-phantom]{Morphisms of Schemes}
\item \hyperref[coherent-section-phantom]{Cohomology of Schemes}
\item \hyperref[divisors-section-phantom]{Divisors}
\item \hyperref[limits-section-phantom]{Limits of Schemes}
\item \hyperref[varieties-section-phantom]{Varieties}
\item \hyperref[topologies-section-phantom]{Topologies on Schemes}
\item \hyperref[descent-section-phantom]{Descent}
\item \hyperref[perfect-section-phantom]{Derived Categories of Schemes}
\item \hyperref[more-morphisms-section-phantom]{More on Morphisms}
\item \hyperref[flat-section-phantom]{More on Flatness}
\item \hyperref[groupoids-section-phantom]{Groupoid Schemes}
\item \hyperref[more-groupoids-section-phantom]{More on Groupoid Schemes}
\item \hyperref[etale-section-phantom]{\'Etale Morphisms of Schemes}
\end{enumerate}
Topics in Scheme Theory
\begin{enumerate}
\setcounter{enumi}{41}
\item \hyperref[chow-section-phantom]{Chow Homology}
\item \hyperref[intersection-section-phantom]{Intersection Theory}
\item \hyperref[pic-section-phantom]{Picard Schemes of Curves}
\item \hyperref[weil-section-phantom]{Weil Cohomology Theories}
\item \hyperref[adequate-section-phantom]{Adequate Modules}
\item \hyperref[dualizing-section-phantom]{Dualizing Complexes}
\item \hyperref[duality-section-phantom]{Duality for Schemes}
\item \hyperref[discriminant-section-phantom]{Discriminants and Differents}
\item \hyperref[derham-section-phantom]{de Rham Cohomology}
\item \hyperref[local-cohomology-section-phantom]{Local Cohomology}
\item \hyperref[algebraization-section-phantom]{Algebraic and Formal Geometry}
\item \hyperref[curves-section-phantom]{Algebraic Curves}
\item \hyperref[resolve-section-phantom]{Resolution of Surfaces}
\item \hyperref[models-section-phantom]{Semistable Reduction}
\item \hyperref[functors-section-phantom]{Functors and Morphisms}
\item \hyperref[equiv-section-phantom]{Derived Categories of Varieties}
\item \hyperref[pione-section-phantom]{Fundamental Groups of Schemes}
\item \hyperref[etale-cohomology-section-phantom]{\'Etale Cohomology}
\item \hyperref[crystalline-section-phantom]{Crystalline Cohomology}
\item \hyperref[proetale-section-phantom]{Pro-\'etale Cohomology}
\item \hyperref[relative-cycles-section-phantom]{Relative Cycles}
\item \hyperref[more-etale-section-phantom]{More \'Etale Cohomology}
\item \hyperref[trace-section-phantom]{The Trace Formula}
\end{enumerate}
Algebraic Spaces
\begin{enumerate}
\setcounter{enumi}{64}
\item \hyperref[spaces-section-phantom]{Algebraic Spaces}
\item \hyperref[spaces-properties-section-phantom]{Properties of Algebraic Spaces}
\item \hyperref[spaces-morphisms-section-phantom]{Morphisms of Algebraic Spaces}
\item \hyperref[decent-spaces-section-phantom]{Decent Algebraic Spaces}
\item \hyperref[spaces-cohomology-section-phantom]{Cohomology of Algebraic Spaces}
\item \hyperref[spaces-limits-section-phantom]{Limits of Algebraic Spaces}
\item \hyperref[spaces-divisors-section-phantom]{Divisors on Algebraic Spaces}
\item \hyperref[spaces-over-fields-section-phantom]{Algebraic Spaces over Fields}
\item \hyperref[spaces-topologies-section-phantom]{Topologies on Algebraic Spaces}
\item \hyperref[spaces-descent-section-phantom]{Descent and Algebraic Spaces}
\item \hyperref[spaces-perfect-section-phantom]{Derived Categories of Spaces}
\item \hyperref[spaces-more-morphisms-section-phantom]{More on Morphisms of Spaces}
\item \hyperref[spaces-flat-section-phantom]{Flatness on Algebraic Spaces}
\item \hyperref[spaces-groupoids-section-phantom]{Groupoids in Algebraic Spaces}
\item \hyperref[spaces-more-groupoids-section-phantom]{More on Groupoids in Spaces}
\item \hyperref[bootstrap-section-phantom]{Bootstrap}
\item \hyperref[spaces-pushouts-section-phantom]{Pushouts of Algebraic Spaces}
\end{enumerate}
Topics in Geometry
\begin{enumerate}
\setcounter{enumi}{81}
\item \hyperref[spaces-chow-section-phantom]{Chow Groups of Spaces}
\item \hyperref[groupoids-quotients-section-phantom]{Quotients of Groupoids}
\item \hyperref[spaces-more-cohomology-section-phantom]{More on Cohomology of Spaces}
\item \hyperref[spaces-simplicial-section-phantom]{Simplicial Spaces}
\item \hyperref[spaces-duality-section-phantom]{Duality for Spaces}
\item \hyperref[formal-spaces-section-phantom]{Formal Algebraic Spaces}
\item \hyperref[restricted-section-phantom]{Algebraization of Formal Spaces}
\item \hyperref[spaces-resolve-section-phantom]{Resolution of Surfaces Revisited}
\end{enumerate}
Deformation Theory
\begin{enumerate}
\setcounter{enumi}{89}
\item \hyperref[formal-defos-section-phantom]{Formal Deformation Theory}
\item \hyperref[defos-section-phantom]{Deformation Theory}
\item \hyperref[cotangent-section-phantom]{The Cotangent Complex}
\item \hyperref[examples-defos-section-phantom]{Deformation Problems}
\end{enumerate}
Algebraic Stacks
\begin{enumerate}
\setcounter{enumi}{93}
\item \hyperref[algebraic-section-phantom]{Algebraic Stacks}
\item \hyperref[examples-stacks-section-phantom]{Examples of Stacks}
\item \hyperref[stacks-sheaves-section-phantom]{Sheaves on Algebraic Stacks}
\item \hyperref[criteria-section-phantom]{Criteria for Representability}
\item \hyperref[artin-section-phantom]{Artin's Axioms}
\item \hyperref[quot-section-phantom]{Quot and Hilbert Spaces}
\item \hyperref[stacks-properties-section-phantom]{Properties of Algebraic Stacks}
\item \hyperref[stacks-morphisms-section-phantom]{Morphisms of Algebraic Stacks}
\item \hyperref[stacks-limits-section-phantom]{Limits of Algebraic Stacks}
\item \hyperref[stacks-cohomology-section-phantom]{Cohomology of Algebraic Stacks}
\item \hyperref[stacks-perfect-section-phantom]{Derived Categories of Stacks}
\item \hyperref[stacks-introduction-section-phantom]{Introducing Algebraic Stacks}
\item \hyperref[stacks-more-morphisms-section-phantom]{More on Morphisms of Stacks}
\item \hyperref[stacks-geometry-section-phantom]{The Geometry of Stacks}
\end{enumerate}
Topics in Moduli Theory
\begin{enumerate}
\setcounter{enumi}{107}
\item \hyperref[moduli-section-phantom]{Moduli Stacks}
\item \hyperref[moduli-curves-section-phantom]{Moduli of Curves}
\end{enumerate}
Miscellany
\begin{enumerate}
\setcounter{enumi}{109}
\item \hyperref[examples-section-phantom]{Examples}
\item \hyperref[exercises-section-phantom]{Exercises}
\item \hyperref[guide-section-phantom]{Guide to Literature}
\item \hyperref[desirables-section-phantom]{Desirables}
\item \hyperref[coding-section-phantom]{Coding Style}
\item \hyperref[obsolete-section-phantom]{Obsolete}
\item \hyperref[fdl-section-phantom]{GNU Free Documentation License}
\item \hyperref[index-section-phantom]{Auto Generated Index}
\end{enumerate}
\end{multicols}


\bibliography{my}
\bibliographystyle{amsalpha}

\end{document}
