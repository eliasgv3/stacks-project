\IfFileExists{stacks-project.cls}{%
\documentclass{stacks-project}
}{%
\documentclass{amsart}
}

% For dealing with references we use the comment environment
\usepackage{verbatim}
\newenvironment{reference}{\comment}{\endcomment}
%\newenvironment{reference}{}{}
\newenvironment{slogan}{\comment}{\endcomment}
\newenvironment{history}{\comment}{\endcomment}

% For commutative diagrams we use Xy-pic
\usepackage[all]{xy}

% We use 2cell for 2-commutative diagrams.
\xyoption{2cell}
\UseAllTwocells

% We use multicol for the list of chapters between chapters
\usepackage{multicol}

% This is generall recommended for better output
\usepackage{lmodern}
\usepackage[T1]{fontenc}

% For cross-file-references
\usepackage{xr-hyper}

% Package for hypertext links:
\usepackage{hyperref}

% For any local file, say "hello.tex" you want to link to please
% use \externaldocument[hello-]{hello}
\externaldocument[introduction-]{introduction}
\externaldocument[conventions-]{conventions}
\externaldocument[sets-]{sets}
\externaldocument[categories-]{categories}
\externaldocument[topology-]{topology}
\externaldocument[sheaves-]{sheaves}
\externaldocument[sites-]{sites}
\externaldocument[stacks-]{stacks}
\externaldocument[fields-]{fields}
\externaldocument[algebra-]{algebra}
\externaldocument[brauer-]{brauer}
\externaldocument[homology-]{homology}
\externaldocument[derived-]{derived}
\externaldocument[simplicial-]{simplicial}
\externaldocument[more-algebra-]{more-algebra}
\externaldocument[smoothing-]{smoothing}
\externaldocument[modules-]{modules}
\externaldocument[sites-modules-]{sites-modules}
\externaldocument[injectives-]{injectives}
\externaldocument[cohomology-]{cohomology}
\externaldocument[sites-cohomology-]{sites-cohomology}
\externaldocument[dga-]{dga}
\externaldocument[dpa-]{dpa}
\externaldocument[sdga-]{sdga}
\externaldocument[hypercovering-]{hypercovering}
\externaldocument[schemes-]{schemes}
\externaldocument[constructions-]{constructions}
\externaldocument[properties-]{properties}
\externaldocument[morphisms-]{morphisms}
\externaldocument[coherent-]{coherent}
\externaldocument[divisors-]{divisors}
\externaldocument[limits-]{limits}
\externaldocument[varieties-]{varieties}
\externaldocument[topologies-]{topologies}
\externaldocument[descent-]{descent}
\externaldocument[perfect-]{perfect}
\externaldocument[more-morphisms-]{more-morphisms}
\externaldocument[flat-]{flat}
\externaldocument[groupoids-]{groupoids}
\externaldocument[more-groupoids-]{more-groupoids}
\externaldocument[etale-]{etale}
\externaldocument[chow-]{chow}
\externaldocument[intersection-]{intersection}
\externaldocument[pic-]{pic}
\externaldocument[weil-]{weil}
\externaldocument[adequate-]{adequate}
\externaldocument[dualizing-]{dualizing}
\externaldocument[duality-]{duality}
\externaldocument[discriminant-]{discriminant}
\externaldocument[derham-]{derham}
\externaldocument[local-cohomology-]{local-cohomology}
\externaldocument[algebraization-]{algebraization}
\externaldocument[curves-]{curves}
\externaldocument[resolve-]{resolve}
\externaldocument[models-]{models}
\externaldocument[functors-]{functors}
\externaldocument[equiv-]{equiv}
\externaldocument[pione-]{pione}
\externaldocument[etale-cohomology-]{etale-cohomology}
\externaldocument[proetale-]{proetale}
\externaldocument[relative-cycles-]{relative-cycles}
\externaldocument[more-etale-]{more-etale}
\externaldocument[trace-]{trace}
\externaldocument[crystalline-]{crystalline}
\externaldocument[spaces-]{spaces}
\externaldocument[spaces-properties-]{spaces-properties}
\externaldocument[spaces-morphisms-]{spaces-morphisms}
\externaldocument[decent-spaces-]{decent-spaces}
\externaldocument[spaces-cohomology-]{spaces-cohomology}
\externaldocument[spaces-limits-]{spaces-limits}
\externaldocument[spaces-divisors-]{spaces-divisors}
\externaldocument[spaces-over-fields-]{spaces-over-fields}
\externaldocument[spaces-topologies-]{spaces-topologies}
\externaldocument[spaces-descent-]{spaces-descent}
\externaldocument[spaces-perfect-]{spaces-perfect}
\externaldocument[spaces-more-morphisms-]{spaces-more-morphisms}
\externaldocument[spaces-flat-]{spaces-flat}
\externaldocument[spaces-groupoids-]{spaces-groupoids}
\externaldocument[spaces-more-groupoids-]{spaces-more-groupoids}
\externaldocument[bootstrap-]{bootstrap}
\externaldocument[spaces-pushouts-]{spaces-pushouts}
\externaldocument[spaces-chow-]{spaces-chow}
\externaldocument[groupoids-quotients-]{groupoids-quotients}
\externaldocument[spaces-more-cohomology-]{spaces-more-cohomology}
\externaldocument[spaces-simplicial-]{spaces-simplicial}
\externaldocument[spaces-duality-]{spaces-duality}
\externaldocument[formal-spaces-]{formal-spaces}
\externaldocument[restricted-]{restricted}
\externaldocument[spaces-resolve-]{spaces-resolve}
\externaldocument[formal-defos-]{formal-defos}
\externaldocument[defos-]{defos}
\externaldocument[cotangent-]{cotangent}
\externaldocument[examples-defos-]{examples-defos}
\externaldocument[algebraic-]{algebraic}
\externaldocument[examples-stacks-]{examples-stacks}
\externaldocument[stacks-sheaves-]{stacks-sheaves}
\externaldocument[criteria-]{criteria}
\externaldocument[artin-]{artin}
\externaldocument[quot-]{quot}
\externaldocument[stacks-properties-]{stacks-properties}
\externaldocument[stacks-morphisms-]{stacks-morphisms}
\externaldocument[stacks-limits-]{stacks-limits}
\externaldocument[stacks-cohomology-]{stacks-cohomology}
\externaldocument[stacks-perfect-]{stacks-perfect}
\externaldocument[stacks-introduction-]{stacks-introduction}
\externaldocument[stacks-more-morphisms-]{stacks-more-morphisms}
\externaldocument[stacks-geometry-]{stacks-geometry}
\externaldocument[moduli-]{moduli}
\externaldocument[moduli-curves-]{moduli-curves}
\externaldocument[examples-]{examples}
\externaldocument[exercises-]{exercises}
\externaldocument[guide-]{guide}
\externaldocument[desirables-]{desirables}
\externaldocument[coding-]{coding}
\externaldocument[obsolete-]{obsolete}
\externaldocument[fdl-]{fdl}
\externaldocument[index-]{index}

% Theorem environments.
%
\theoremstyle{plain}
\newtheorem{theorem}[subsection]{Theorem}
\newtheorem{proposition}[subsection]{Proposition}
\newtheorem{lemma}[subsection]{Lemma}

\theoremstyle{definition}
\newtheorem{definition}[subsection]{Definition}
\newtheorem{example}[subsection]{Example}
\newtheorem{exercise}[subsection]{Exercise}
\newtheorem{situation}[subsection]{Situation}

\theoremstyle{remark}
\newtheorem{remark}[subsection]{Remark}
\newtheorem{remarks}[subsection]{Remarks}

\numberwithin{equation}{subsection}

% Macros
%
\def\lim{\mathop{\mathrm{lim}}\nolimits}
\def\colim{\mathop{\mathrm{colim}}\nolimits}
\def\Spec{\mathop{\mathrm{Spec}}}
\def\Hom{\mathop{\mathrm{Hom}}\nolimits}
\def\Ext{\mathop{\mathrm{Ext}}\nolimits}
\def\SheafHom{\mathop{\mathcal{H}\!\mathit{om}}\nolimits}
\def\SheafExt{\mathop{\mathcal{E}\!\mathit{xt}}\nolimits}
\def\Sch{\mathit{Sch}}
\def\Mor{\mathop{\mathrm{Mor}}\nolimits}
\def\Ob{\mathop{\mathrm{Ob}}\nolimits}
\def\Sh{\mathop{\mathit{Sh}}\nolimits}
\def\NL{\mathop{N\!L}\nolimits}
\def\CH{\mathop{\mathrm{CH}}\nolimits}
\def\proetale{{pro\text{-}\acute{e}tale}}
\def\etale{{\acute{e}tale}}
\def\QCoh{\mathit{QCoh}}
\def\Ker{\mathop{\mathrm{Ker}}}
\def\Im{\mathop{\mathrm{Im}}}
\def\Coker{\mathop{\mathrm{Coker}}}
\def\Coim{\mathop{\mathrm{Coim}}}

% Boxtimes
%
\DeclareMathSymbol{\boxtimes}{\mathbin}{AMSa}{"02}

%
% Macros for moduli stacks/spaces
%
\def\QCohstack{\mathcal{QC}\!\mathit{oh}}
\def\Cohstack{\mathcal{C}\!\mathit{oh}}
\def\Spacesstack{\mathcal{S}\!\mathit{paces}}
\def\Quotfunctor{\mathrm{Quot}}
\def\Hilbfunctor{\mathrm{Hilb}}
\def\Curvesstack{\mathcal{C}\!\mathit{urves}}
\def\Polarizedstack{\mathcal{P}\!\mathit{olarized}}
\def\Complexesstack{\mathcal{C}\!\mathit{omplexes}}
% \Pic is the operator that assigns to X its picard group, usage \Pic(X)
% \Picardstack_{X/B} denotes the Picard stack of X over B
% \Picardfunctor_{X/B} denotes the Picard functor of X over B
\def\Pic{\mathop{\mathrm{Pic}}\nolimits}
\def\Picardstack{\mathcal{P}\!\mathit{ic}}
\def\Picardfunctor{\mathrm{Pic}}
\def\Deformationcategory{\mathcal{D}\!\mathit{ef}}


% OK, start here.
%
\begin{document}

\title{The Cotangent Complex}


\maketitle

\phantomsection
\label{section-phantom}

\tableofcontents

\section{Introduction}
\label{section-introduction}

\noindent
The goal of this chapter is to construct the cotangent complex of a
ring map, of a morphism of schemes, and of a morphism of algebraic spaces.
Some references are the notes \cite{quillenhomology}, the paper
\cite{quillencohomology}, and the books
\cite{Andre} and \cite{cotangent}.




\section{Advice for the reader}
\label{section-advice-reader}

\noindent
In writing this chapter we have tried to minimize
the use of simplicial techniques. We view the choice of a {\it resolution}
$P_\bullet$ of a ring $B$ over a ring $A$ as a tool to calculating the
{\it homology} of abelian sheaves on the category $\mathcal{C}_{B/A}$, see
Remark \ref{remark-resolution}. This is similar to the role played
by a ``good cover'' to compute cohomology using the {\v C}ech complex.
To read a bit on homology on categories, please visit
Cohomology on Sites, Section \ref{sites-cohomology-section-homology}.
The derived lower shriek functor $L\pi_!$ is to homology what
$R\Gamma(\mathcal{C}_{B/A}, -)$ is to cohomology. The category
$\mathcal{C}_{B/A}$, studied in Section \ref{section-compute-L-pi-shriek},
is the opposite of the category of factorizations $A \to P \to B$ where $P$
is a polynomial algebra over $A$. This category comes with maps of sheaves
of rings
$$
\underline{A} \longrightarrow \mathcal{O} \longrightarrow \underline{B}
$$
where over the object $U =  (P \to B)$ we have $\mathcal{O}(U) = P$. 
It turns out that we obtain the cotangent complex of $B$ over $A$ as
$$
L_{B/A} =
L\pi_!(\Omega_{\mathcal{O}/\underline{A}} \otimes_\mathcal{O} \underline{B})
$$
see Lemma \ref{lemma-compute-cotangent-complex}. We have consistently tried
to use this point of view to prove the basic properties of cotangent
complexes of ring maps. In particular, all of the results can be proven
without relying on the existence of standard resolutions, although we have
not done so. The theory is quite satisfactory, except that
perhaps the proof of the fundamental triangle
(Proposition \ref{proposition-triangle}) uses just a little
bit more theory on derived lower shriek functors.
To provide the reader with an alternative,
we give a rather complete sketch of an approach to this result
based on simple properties of standard resolutions in
Remarks \ref{remark-triangle} and \ref{remark-explicit-map}.

\medskip\noindent
Our approach to the cotangent complex for morphisms of ringed topoi,
morphisms of schemes, morphisms of algebraic spaces, etc
is to deduce as much as possible from the case of ``plain ring maps''
discussed above.





\section{The cotangent complex of a ring map}
\label{section-cotangent-ring-map}

\noindent
Let $A$ be a ring. Let $\textit{Alg}_A$ be the category of $A$-algebras.
Consider the pair of adjoint functors $(U, V)$ where
$V : \textit{Alg}_A \to \textit{Sets}$ is the forgetful functor and
$U : \textit{Sets} \to \textit{Alg}_A$ assigns to a set $E$ the polynomial
algebra $A[E]$ on $E$ over $A$. Let $X_\bullet$ be the simplicial object of
$\text{Fun}(\textit{Alg}_A, \textit{Alg}_A)$ constructed in
Simplicial, Section \ref{simplicial-section-standard}.

\medskip\noindent
Consider an $A$-algebra $B$. Denote $P_\bullet = X_\bullet(B)$ the resulting
simplicial $A$-algebra. Recall that $P_0 = A[B]$, $P_1 = A[A[B]]$, and so on.
In particular each term $P_n$ is a polynomial $A$-algebra.
Recall also that there is an augmentation
$$
\epsilon : P_\bullet \longrightarrow B
$$
where we view $B$ as a constant simplicial $A$-algebra.

\begin{definition}
\label{definition-standard-resolution}
Let $A \to B$ be a ring map. The {\it standard resolution of $B$ over $A$}
is the augmentation $\epsilon : P_\bullet \to B$ with terms
$$
P_0 = A[B],\quad P_1 = A[A[B]],\quad \ldots
$$
and maps as constructed above.
\end{definition}

\noindent
It will turn out that we can use the standard resolution
to compute left derived functors in certain settings.

\begin{definition}
\label{definition-cotangent-complex-ring-map}
The {\it cotangent complex} $L_{B/A}$ of a ring map $A \to B$
is the complex of $B$-modules associated to the simplicial $B$-module
$$
\Omega_{P_\bullet/A} \otimes_{P_\bullet, \epsilon} B
$$
where $\epsilon : P_\bullet \to B$ is the standard resolution
of $B$ over $A$.
\end{definition}

\noindent
In Simplicial, Section \ref{simplicial-section-complexes} we associate a
chain complex to a simplicial module, but here we work with cochain complexes.
Thus the term $L_{B/A}^{-n}$ in degree $-n$ is the $B$-module
$\Omega_{P_n/A} \otimes_{P_n, \epsilon_n} B$ and $L_{B/A}^m = 0$
for $m > 0$.

\begin{remark}
\label{remark-variant-cotangent-complex}
Let $A \to B$ be a ring map. Let $\mathcal{A}$ be the category of
arrows $\psi : C \to B$ of $A$-algebras and let $\mathcal{S}$ be
the category of maps $E \to B$ where $E$ is a set. There are adjoint
functors $V : \mathcal{A} \to \mathcal{S}$ (the forgetful functor)
and $U : \mathcal{S} \to \mathcal{A}$ which sends $E \to B$ to
$A[E] \to B$. Let $X_\bullet$ be the simplicial object of
$\text{Fun}(\mathcal{A}, \mathcal{A})$ constructed in
Simplicial, Section \ref{simplicial-section-standard}.
The diagram
$$
\xymatrix{
\mathcal{A} \ar[d] \ar[r] & \mathcal{S} \ar@<1ex>[l] \ar[d] \\
\textit{Alg}_A \ar[r] & \textit{Sets} \ar@<1ex>[l]
}
$$
commutes. It follows that $X_\bullet(\text{id}_B : B \to B)$
is equal to the standard resolution of $B$ over $A$.
\end{remark}

\begin{lemma}
\label{lemma-colimit-cotangent-complex}
Let $A_i \to B_i$ be a system of ring maps over a directed index
set $I$. Then $\colim L_{B_i/A_i} = L_{\colim B_i/\colim A_i}$.
\end{lemma}

\begin{proof}
This is true because the forgetful functor
$V : A\textit{-Alg} \to \textit{Sets}$ and its adjoint
$U : \textit{Sets} \to A\textit{-Alg}$ commute with filtered colimits.
Moreover, the functor $B/A \mapsto \Omega_{B/A}$ does as well
(Algebra, Lemma \ref{algebra-lemma-colimit-differentials}).
\end{proof}





\section{Simplicial resolutions and derived lower shriek}
\label{section-compute-L-pi-shriek}

\noindent
Let $A \to B$ be a ring map. Consider the category whose objects are
$A$-algebra maps $\alpha : P \to B$ where $P$ is a polynomial algebra over $A$
(in some set\footnote{It suffices to consider sets of cardinality
at most the cardinality of $B$.} of variables) and whose
morphisms $s : (\alpha : P \to B) \to (\alpha' : P' \to B)$ are
$A$-algebra homomorphisms $s : P \to P'$ with $\alpha' \circ s = \alpha$.
Let $\mathcal{C} = \mathcal{C}_{B/A}$ denote the {\bf opposite}
of this category. The reason for
taking the opposite is that we want to think of objects
$(P, \alpha)$ as corresponding to the diagram of affine schemes
$$
\xymatrix{
\Spec(B) \ar[d] \ar[r] & \Spec(P) \ar[ld] \\
\Spec(A)
}
$$
We endow $\mathcal{C}$ with the chaotic topology
(Sites, Example \ref{sites-example-indiscrete}), i.e., we endow
$\mathcal{C}$ with the structure of a site where coverings are given by
identities so that all presheaves are sheaves.
Moreover, we endow $\mathcal{C}$ with two sheaves of rings. The first
is the sheaf $\mathcal{O}$ which sends to object $(P, \alpha)$ to $P$.
Then second is the constant sheaf $B$, which we will denote
$\underline{B}$. We obtain the following diagram of morphisms of
ringed topoi
\begin{equation}
\label{equation-pi}
\vcenter{
\xymatrix{
(\Sh(\mathcal{C}), \underline{B}) \ar[r]_i \ar[d]_\pi &
(\Sh(\mathcal{C}), \mathcal{O}) \\
(\Sh(*), B)
}
}
\end{equation}
The morphism $i$ is the identity on underlying topoi and
$i^\sharp : \mathcal{O} \to \underline{B}$ is the obvious map.
The map $\pi$ is as in Cohomology on Sites, Example
\ref{sites-cohomology-example-category-to-point}.
An important role will be played in the following
by the derived functors
$
Li^* : D(\mathcal{O}) \longrightarrow D(\underline{B})
$
left adjoint to $Ri_* = i_* : D(\underline{B}) \to D(\mathcal{O})$ and
$
L\pi_! : D(\underline{B}) \longrightarrow D(B)
$
left adjoint to $\pi^* = \pi^{-1} : D(B) \to D(\underline{B})$.

\begin{lemma}
\label{lemma-identify-pi-shriek}
With notation as above let $P_\bullet$ be a simplicial $A$-algebra
endowed with an augmentation $\epsilon : P_\bullet \to B$.
Assume each $P_n$ is a polynomial algebra over $A$ and $\epsilon$
is a trivial Kan fibration on underlying simplicial sets. Then
$$
L\pi_!(\mathcal{F}) = \mathcal{F}(P_\bullet, \epsilon)
$$
in $D(\textit{Ab})$, resp.\ $D(B)$ functorially in $\mathcal{F}$ in
$\textit{Ab}(\mathcal{C})$, resp.\ $\textit{Mod}(\underline{B})$.
\end{lemma}

\begin{proof}
We will use the criterion of Cohomology on Sites, Lemma
\ref{sites-cohomology-lemma-compute-by-cosimplicial-resolution} to prove this.
Given an object $U = (Q, \beta)$ of $\mathcal{C}$ we have to show that
$$
S_\bullet = \Mor_\mathcal{C}((Q, \beta), (P_\bullet, \epsilon))
$$
is homotopy equivalent to a singleton.
Write $Q = A[E]$ for some set $E$ (this is possible by our choice of
the category $\mathcal{C}$). We see that
$$
S_\bullet = \Mor_{\textit{Sets}}((E, \beta|_E), (P_\bullet, \epsilon))
$$
Let $*$ be the constant simplicial set on a singleton. For $b \in B$
let $F_{b, \bullet}$ be the simplicial set defined by the cartesian
diagram
$$
\xymatrix{
F_{b, \bullet} \ar[r] \ar[d] & P_\bullet \ar[d]_\epsilon \\
{*} \ar[r]^b & B
}
$$
With this notation $S_\bullet = \prod_{e \in E} F_{\beta(e), \bullet}$.
Since we assumed $\epsilon$ is a trivial Kan fibration we see that
$F_{b, \bullet} \to *$ is a trivial Kan fibration
(Simplicial, Lemma \ref{simplicial-lemma-trivial-kan-base-change}).
Thus $S_\bullet \to *$ is a trivial Kan fibration
(Simplicial, Lemma \ref{simplicial-lemma-product-trivial-kan}).
Therefore $S_\bullet$ is homotopy equivalent to $*$
(Simplicial, Lemma \ref{simplicial-lemma-trivial-kan-homotopy}).
\end{proof}

\noindent
In particular, we can use the standard resolution of $B$ over $A$
to compute derived lower shriek.

\begin{lemma}
\label{lemma-pi-shriek-standard}
Let $A \to B$ be a ring map. Let $\epsilon : P_\bullet \to B$
be the standard resolution of $B$ over $A$. Let $\pi$ be as in
(\ref{equation-pi}). Then
$$
L\pi_!(\mathcal{F}) = \mathcal{F}(P_\bullet, \epsilon)
$$
in $D(\textit{Ab})$, resp.\ $D(B)$ functorially in $\mathcal{F}$ in
$\textit{Ab}(\mathcal{C})$, resp.\ $\textit{Mod}(\underline{B})$.
\end{lemma}

\begin{proof}[First proof]
We will apply Lemma \ref{lemma-identify-pi-shriek}.
Since the terms $P_n$ are polynomial algebras we see the first
assumption of that lemma is satisfied. The second assumption is proved
as follows. By
Simplicial, Lemma \ref{simplicial-lemma-standard-simplicial-homotopy}
the map $\epsilon$ is a homotopy equivalence of underlying
simplicial sets. By
Simplicial, Lemma \ref{simplicial-lemma-homotopy-equivalence}
this implies $\epsilon$ induces a quasi-isomorphism of associated
complexes of abelian groups. By
Simplicial, Lemma \ref{simplicial-lemma-qis-simplicial-abelian-groups}
this implies that $\epsilon$ is a trivial Kan fibration of underlying
simplicial sets.
\end{proof}

\begin{proof}[Second proof]
We will use the criterion of Cohomology on Sites, Lemma
\ref{sites-cohomology-lemma-compute-by-cosimplicial-resolution}.
Let $U = (Q, \beta)$ be an object of $\mathcal{C}$.
We have to show that
$$
S_\bullet = \Mor_\mathcal{C}((Q, \beta), (P_\bullet, \epsilon))
$$
is homotopy equivalent to a singleton. Write $Q = A[E]$ for some set $E$
(this is possible by our choice of the category $\mathcal{C}$). Using the
notation of Remark \ref{remark-variant-cotangent-complex} we see that
$$
S_\bullet = \Mor_\mathcal{S}((E \to B), i(P_\bullet \to B))
$$
By Simplicial, Lemma \ref{simplicial-lemma-standard-simplicial-homotopy}
the map $i(P_\bullet \to B) \to i(B \to B)$ is a homotopy equivalence
in $\mathcal{S}$. Hence $S_\bullet$ is homotopy equivalent to
$$
\Mor_\mathcal{S}((E \to B), (B \to B)) = \{*\}
$$
as desired.
\end{proof}

\begin{lemma}
\label{lemma-compute-cotangent-complex}
Let $A \to B$ be a ring map.  Let $\pi$ and $i$ be as in (\ref{equation-pi}).
There is a canonical isomorphism
$$
L_{B/A} = L\pi_!(Li^*\Omega_{\mathcal{O}/A}) =
L\pi_!(i^*\Omega_{\mathcal{O}/A}) =
L\pi_!(\Omega_{\mathcal{O}/A} \otimes_\mathcal{O} \underline{B})
$$
in $D(B)$.
\end{lemma}

\begin{proof}
For an object $\alpha : P \to B$ of the category $\mathcal{C}$
the module $\Omega_{P/A}$ is a free $P$-module. Thus
$\Omega_{\mathcal{O}/A}$ is a flat $\mathcal{O}$-module. Hence
$Li^*\Omega_{\mathcal{O}/A} = i^*\Omega_{\mathcal{O}/A}$ is the sheaf
of $\underline{B}$-modules which associates to $\alpha : P \to A$ the
$B$-module $\Omega_{P/A} \otimes_{P, \alpha} B$.
By Lemma \ref{lemma-pi-shriek-standard}
we see that the right hand side is computed by
the value of this sheaf on the standard resolution which is our
definition of the left hand side
(Definition \ref{definition-cotangent-complex-ring-map}).
\end{proof}

\begin{lemma}
\label{lemma-pi-lower-shriek-constant-sheaf}
If $A \to B$ is a ring map, then $L\pi_!(\pi^{-1}M) = M$
with $\pi$ as in (\ref{equation-pi}).
\end{lemma}

\begin{proof}
This follows from Lemma \ref{lemma-identify-pi-shriek} which tells us
$L\pi_!(\pi^{-1}M)$ is computed by $(\pi^{-1}M)(P_\bullet, \epsilon)$
which is the constant simplicial object on $M$.
\end{proof}

\begin{lemma}
\label{lemma-identify-H0}
If $A \to B$ is a ring map, then $H^0(L_{B/A}) = \Omega_{B/A}$.
\end{lemma}

\begin{proof}
We will prove this by a direct calculation.
We will use the identification of Lemma \ref{lemma-compute-cotangent-complex}.
There is clearly a map from $\Omega_{\mathcal{O}/A} \otimes \underline{B}$
to the constant sheaf with value $\Omega_{B/A}$. Thus this map induces
a map
$$
H^0(L_{B/A}) = H^0(L\pi_!(\Omega_{\mathcal{O}/A} \otimes \underline{B}))
= \pi_!(\Omega_{\mathcal{O}/A} \otimes \underline{B}) \to \Omega_{B/A}
$$
By choosing an object $P \to B$ of $\mathcal{C}_{B/A}$ with $P \to B$
surjective we see that this map is surjective (by
Algebra, Lemma \ref{algebra-lemma-differential-surjective}).
To show that it is injective, suppose that $P \to B$ is an object
of $\mathcal{C}_{B/A}$ and that $\xi \in \Omega_{P/A} \otimes_P B$
is an element which maps to zero in $\Omega_{B/A}$.
We first choose factorization $P \to P' \to B$ such that $P' \to B$
is surjective and $P'$ is a polynomial algebra over $A$.
We may replace $P$ by $P'$. If $B = P/I$, then the kernel
$\Omega_{P/A} \otimes_P B \to \Omega_{B/A}$ is the image of
$I/I^2$ (Algebra, Lemma \ref{algebra-lemma-differential-seq}).
Say $\xi$ is the image of $f \in I$.
Then we consider the two maps $a, b : P' = P[x] \to P$, the first of which
maps $x$ to $0$ and the second of which maps $x$ to $f$ (in both
cases $P[x] \to B$ maps $x$ to zero). We see that $\xi$ and $0$
are the image of $\text{d}x \otimes 1$ in $\Omega_{P'/A} \otimes_{P'} B$.
Thus $\xi$ and $0$ have the same image in the colimit (see
Cohomology on Sites, Example \ref{sites-cohomology-example-category-to-point})
$\pi_!(\Omega_{\mathcal{O}/A} \otimes \underline{B})$ as desired.
\end{proof}

\begin{lemma}
\label{lemma-pi-lower-shriek-polynomial-algebra}
If $B$ is a polynomial algebra over the ring $A$, then
with $\pi$ as in (\ref{equation-pi}) we have that
$\pi_!$ is exact and $\pi_!\mathcal{F} = \mathcal{F}(B \to B)$.
\end{lemma}

\begin{proof}
This follows from Lemma \ref{lemma-identify-pi-shriek} which tells us
the constant simplicial algebra on $B$ can be used to compute $L\pi_!$.
\end{proof}

\begin{lemma}
\label{lemma-cotangent-complex-polynomial-algebra}
If $B$ is a polynomial algebra over the ring $A$, then
$L_{B/A}$ is quasi-isomorphic to $\Omega_{B/A}[0]$.
\end{lemma}

\begin{proof}
Immediate from
Lemmas \ref{lemma-compute-cotangent-complex} and
\ref{lemma-pi-lower-shriek-polynomial-algebra}.
\end{proof}





\section{Constructing a resolution}
\label{section-polynomial}

\noindent
In the Noetherian finite type case we can construct a ``small'' simplicial
resolution for finite type ring maps.

\begin{lemma}
\label{lemma-polynomial}
Let $A$ be a Noetherian ring. Let $A \to B$ be a finite type ring map.
Let $\mathcal{A}$ be the category of $A$-algebra maps $C \to B$. Let
$n \geq 0$ and let $P_\bullet$ be a simplicial object of $\mathcal{A}$
such that
\begin{enumerate}
\item $P_\bullet \to B$ is a trivial Kan fibration of simplicial sets,
\item $P_k$ is finite type over $A$ for $k \leq n$,
\item $P_\bullet = \text{cosk}_n \text{sk}_n P_\bullet$ as simplicial
objects of $\mathcal{A}$.
\end{enumerate}
Then $P_{n + 1}$ is a finite type $A$-algebra.
\end{lemma}

\begin{proof}
Although the proof we give of this lemma is straightforward, it is a bit
messy. To clarify the idea we explain what happens for low $n$ before giving
the proof in general. For example, if $n = 0$, then (3) means that
$P_1 = P_0 \times_B P_0$. Since the ring map $P_0 \to B$ is surjective, this
is of finite type over $A$ by
More on Algebra, Lemma \ref{more-algebra-lemma-fibre-product-finite-type}.

\medskip\noindent
If $n = 1$, then (3) means that
$$
P_2 = \{(f_0, f_1, f_2) \in P_1^3 \mid
d_0f_0 = d_0f_1,\ d_1f_0 = d_0f_2,\ d_1f_1 = d_1f_2 \}
$$
where the equalities take place in $P_0$. Observe that the triple
$$
(d_0f_0, d_1f_0, d_1f_1) = (d_0f_1, d_0f_2, d_1f_2)
$$
is an element of the fibre product $P_0 \times_B P_0 \times_B P_0$ over $B$
because the maps $d_i : P_1 \to P_0$ are morphisms over $B$. Thus we get
a map
$$
\psi : P_2 \longrightarrow P_0 \times_B P_0 \times_B P_0
$$
The fibre of $\psi$ over an element
$(g_0, g_1, g_2) \in P_0 \times_B P_0 \times_B P_0$
is the set of triples $(f_0, f_1, f_2)$ of $1$-simplices
with $(d_0, d_1)(f_0) = (g_0, g_1)$, $(d_0, d_1)(f_1) = (g_0, g_2)$,
and $(d_0, d_1)(f_2) = (g_1, g_2)$. As $P_\bullet \to B$ is a trivial
Kan fibration the map $(d_0, d_1) : P_1 \to P_0 \times_B P_0$ is
surjective. Thus we see that $P_2$ fits into the cartesian diagram
$$
\xymatrix{
P_2 \ar[d] \ar[r] & P_1^3 \ar[d] \\
P_0 \times_B P_0 \times_B P_0 \ar[r] & (P_0 \times_B P_0)^3
}
$$
By More on Algebra, Lemma \ref{more-algebra-lemma-formal-consequence}
we conclude. The general case is similar, but requires a bit more notation.

\medskip\noindent
The case $n > 1$. By Simplicial, Lemma \ref{simplicial-lemma-cosk-above-object}
the condition $P_\bullet = \text{cosk}_n \text{sk}_n P_\bullet$
implies the same thing is true in the category of simplicial
$A$-algebras and hence in the category of sets (as the forgetful
functor from $A$-algebras to sets commutes with limits). Thus
$$
P_{n + 1} =
\Mor(\Delta[n + 1], P_\bullet) =
\Mor(\text{sk}_n \Delta[n + 1], \text{sk}_n P_\bullet)
$$
by Simplicial, Lemma \ref{simplicial-lemma-simplex-map} and
Equation (\ref{simplicial-equation-cosk}). We will prove by induction
on $1 \leq k < m \leq n + 1$ that the ring
$$
Q_{k, m} = \Mor(\text{sk}_k \Delta[m], \text{sk}_k P_\bullet)
$$
is of finite type over $A$. The case $k = 1$, $1 < m \leq n + 1$
is entirely similar to the discussion above in the case $n = 1$.
Namely, there is a cartesian diagram
$$
\xymatrix{
Q_{1, m} \ar[d] \ar[r] & P_1^N \ar[d] \\
P_0 \times_B \ldots \times_B P_0 \ar[r] & (P_0 \times_B P_0)^N
}
$$
where $N = {m + 1 \choose 2}$. We conclude as before.

\medskip\noindent
Let $1 \leq k_0 \leq n$ and assume $Q_{k, m}$ is of finite type
over $A$ for all $1 \leq k \leq k_0$ and $k < m \leq n + 1$.
For $k_0 + 1 < m \leq n + 1$ we claim there is a cartesian square
$$
\xymatrix{
Q_{k_0 + 1, m} \ar[d] \ar[r] & P_{k_0 + 1}^N \ar[d] \\
Q_{k_0, m} \ar[r] & Q_{k_0, k_0 + 1}^N
}
$$
where $N$ is the number of nondegenerate $(k_0 + 1)$-simplices
of $\Delta[m]$. Namely, to see this is true, think of an element of
$Q_{k_0 + 1, m}$ as a function $f$ from the $(k_0 + 1)$-skeleton
of $\Delta[m]$ to $P_\bullet$. We can restrict $f$ to the $k_0$-skeleton
which gives the left vertical map of the diagram. We can also restrict
to each nondegenerate $(k_0 + 1)$-simplex which gives the top horizontal
arrow. Moreover, to give such an $f$ is the same thing as giving its
restriction to $k_0$-skeleton and to each nondegenerate
$(k_0 + 1)$-face, provided these agree on the overlap, and this
is exactly the content of the diagram. Moreover, the fact that
$P_\bullet \to B$ is a trivial Kan fibration implies that
the map
$$
P_{k_0} \to Q_{k_0, k_0 + 1} = \Mor(\partial \Delta[k_0 + 1], P_\bullet)
$$
is surjective as every map $\partial \Delta[k_0 + 1] \to B$ can be extended
to $\Delta[k_0 + 1] \to B$ for $k_0 \geq 1$ (small argument about constant
simplicial sets omitted). Since by induction hypothesis the rings
$Q_{k_0, m}$, $Q_{k_0, k_0 + 1}$ are finite type $A$-algebras, so is
$Q_{k_0 + 1, m}$ by
More on Algebra, Lemma \ref{more-algebra-lemma-formal-consequence}
once more.
\end{proof}

\begin{proposition}
\label{proposition-polynomial}
Let $A$ be a Noetherian ring. Let $A \to B$ be a finite type ring map.
There exists a simplicial $A$-algebra $P_\bullet$ with an augmentation
$\epsilon : P_\bullet \to B$ such that each $P_n$ is a polynomial algebra
of finite type over $A$ and such that $\epsilon$ is a trivial
Kan fibration of simplicial sets.
\end{proposition}

\begin{proof}
Let $\mathcal{A}$ be the category of $A$-algebra maps $C \to B$.
In this proof our simplicial objects and skeleton and coskeleton
functors will be taken in this category.

\medskip\noindent
Choose a polynomial algebra $P_0$ of finite type over $A$ and a surjection
$P_0 \to B$. As a first approximation we take
$P_\bullet = \text{cosk}_0(P_0)$. In other words, $P_\bullet$ is the simplicial
$A$-algebra with terms $P_n = P_0 \times_A \ldots \times_A P_0$.
(In the final paragraph of the proof this simplicial object will
be denoted $P^0_\bullet$.) By
Simplicial, Lemma \ref{simplicial-lemma-cosk-minus-one-equivalence}
the map $P_\bullet \to B$ is a trivial Kan fibration of simplicial sets.
Also, observe that $P_\bullet = \text{cosk}_0 \text{sk}_0 P_\bullet$.

\medskip\noindent
Suppose for some $n \geq 0$ we have constructed $P_\bullet$
(in the final paragraph of the proof this will be $P^n_\bullet$)
such that
\begin{enumerate}
\item[(a)] $P_\bullet \to B$ is a trivial Kan fibration of simplicial sets,
\item[(b)] $P_k$ is a finitely generated polynomial algebra for
$0 \leq k \leq n$, and
\item[(c)] $P_\bullet = \text{cosk}_n \text{sk}_n P_\bullet$
\end{enumerate}
By Lemma \ref{lemma-polynomial}
we can find a finitely generated polynomial algebra $Q$ over $A$
and a surjection $Q \to P_{n + 1}$. Since $P_n$ is a polynomial algebra
the $A$-algebra maps $s_i : P_n \to P_{n + 1}$ lift to maps
$s'_i : P_n \to Q$. Set $d'_j : Q \to P_n$ equal to the composition of
$Q \to P_{n + 1}$ and $d_j : P_{n + 1} \to P_n$.
We obtain a truncated simplicial object $P'_\bullet$ of $\mathcal{A}$
by setting $P'_k = P_k$ for $k \leq n$ and $P'_{n + 1} = Q$ and morphisms
$d'_i = d_i$ and $s'_i = s_i$ in degrees $k \leq n - 1$ and using the
morphisms $d'_j$ and $s'_i$ in degree $n$. Extend this to a full simplicial
object $P'_\bullet$ of $\mathcal{A}$ using $\text{cosk}_{n + 1}$. By
functoriality of the coskeleton functors there is a morphism
$P'_\bullet \to P_\bullet$ of simplicial objects extending the
given morphism of $(n + 1)$-truncated simplicial objects.
(This morphism will be denoted $P^{n + 1}_\bullet \to P^n_\bullet$
in the final paragraph of the proof.)

\medskip\noindent
Note that conditions (b) and (c) are satisfied for $P'_\bullet$ with $n$
replaced by $n + 1$. We claim the map $P'_\bullet \to P_\bullet$ satisfies
assumptions (1), (2), (3), and (4) of
Simplicial, Lemmas \ref{simplicial-lemma-section}
with $n + 1$ instead of $n$. Conditions (1) and (2) hold by construction.
By Simplicial, Lemma \ref{simplicial-lemma-cosk-above-object}
we see that we have
$P_\bullet = \text{cosk}_{n + 1}\text{sk}_{n + 1}P_\bullet$
and
$P'_\bullet = \text{cosk}_{n + 1}\text{sk}_{n + 1}P'_\bullet$
not only in $\mathcal{A}$ but also in the category of $A$-algebras,
whence in the category of sets (as the forgetful functor from $A$-algebras
to sets commutes with all limits). This proves (3) and (4). Thus the lemma
applies and $P'_\bullet \to P_\bullet$ is a trivial Kan fibration. By
Simplicial, Lemma \ref{simplicial-lemma-trivial-kan-composition}
we conclude that $P'_\bullet \to B$ is a trivial Kan fibration and (a)
holds as well.

\medskip\noindent
To finish the proof we take the inverse limit $P_\bullet = \lim P^n_\bullet$
of the sequence of simplicial algebras
$$
\ldots \to P^2_\bullet \to P^1_\bullet \to P^0_\bullet
$$
constructed above. The map $P_\bullet \to B$ is a trivial Kan fibration by
Simplicial, Lemma \ref{simplicial-lemma-limit-trivial-kan}.
However, the construction above stabilizes in each degree
to a fixed finitely generated polynomial algebra as desired.
\end{proof}

\begin{lemma}
\label{lemma-pi-shriek-finite}
Let $A$ be a Noetherian ring. Let $A \to B$ be a finite type ring map.
Let $\pi$, $\underline{B}$ be as in (\ref{equation-pi}).
If $\mathcal{F}$ is an $\underline{B}$-module such that
$\mathcal{F}(P, \alpha)$ is a finite $B$-module for all
$\alpha : P = A[x_1, \ldots, x_n] \to B$, then the cohomology modules
of $L\pi_!(\mathcal{F})$ are finite $B$-modules.
\end{lemma}

\begin{proof}
By Lemma \ref{lemma-identify-pi-shriek} and
Proposition \ref{proposition-polynomial}
we can compute $L\pi_!(\mathcal{F})$ by a complex
constructed out of the values of $\mathcal{F}$ on finite type
polynomial algebras.
\end{proof}

\begin{lemma}
\label{lemma-cotangent-finite}
Let $A$ be a Noetherian ring. Let $A \to B$ be a finite type ring map.
Then $H^n(L_{B/A})$ is a finite $B$-module for all $n \in \mathbf{Z}$.
\end{lemma}

\begin{proof}
Apply Lemmas \ref{lemma-compute-cotangent-complex} and
\ref{lemma-pi-shriek-finite}.
\end{proof}

\begin{remark}[Resolutions]
\label{remark-resolution}
Let $A \to B$ be any ring map. Let us call an augmented simplicial $A$-algebra
$\epsilon : P_\bullet \to B$ a {\it resolution of $B$ over $A$} if
each $P_n$ is a polynomial algebra and $\epsilon$ is a trivial Kan fibration
of simplicial sets. If $P_\bullet \to B$ is an augmentation of a simplicial
$A$-algebra with each $P_n$ a polynomial algebra surjecting onto $B$, then
the following are equivalent
\begin{enumerate}
\item $\epsilon : P_\bullet \to B$ is a resolution of $B$ over $A$,
\item $\epsilon : P_\bullet \to B$ is a quasi-isomorphism on
associated complexes,
\item $\epsilon : P_\bullet \to B$ induces a homotopy equivalence
of simplicial sets.
\end{enumerate}
To see this use Simplicial, Lemmas
\ref{simplicial-lemma-trivial-kan-homotopy},
\ref{simplicial-lemma-homotopy-equivalence}, and
\ref{simplicial-lemma-qis-simplicial-abelian-groups}.
A resolution $P_\bullet$ of $B$ over $A$ gives a cosimplicial object
$U_\bullet$ of $\mathcal{C}_{B/A}$ as in Cohomology on Sites, Lemma
\ref{sites-cohomology-lemma-compute-by-cosimplicial-resolution}
and it follows that
$$
L\pi_!\mathcal{F} = \mathcal{F}(P_\bullet)
$$
functorially in $\mathcal{F}$, see Lemma \ref{lemma-identify-pi-shriek}.
The (formal part of the) proof of Proposition \ref{proposition-polynomial}
shows that resolutions exist. We also have seen in the first proof of
Lemma \ref{lemma-pi-shriek-standard} that the standard resolution of $B$
over $A$ is a resolution (so that this terminology doesn't lead to a conflict).
However, the argument in the proof of Proposition \ref{proposition-polynomial}
shows the existence of resolutions without appealing to the simplicial
computations in Simplicial, Section \ref{simplicial-section-standard}.
Moreover, for {\it any} choice of resolution we have a canonical isomorphism
$$
L_{B/A} = \Omega_{P_\bullet/A} \otimes_{P_\bullet, \epsilon} B
$$
in $D(B)$ by Lemma \ref{lemma-compute-cotangent-complex}. The freedom to
choose an arbitrary resolution can be quite useful.
\end{remark}

\begin{lemma}
\label{lemma-O-homology-B-homology}
Let $A \to B$ be a ring map. Let $\pi$, $\mathcal{O}$, $\underline{B}$
be as in (\ref{equation-pi}). For any $\mathcal{O}$-module $\mathcal{F}$
we have
$$
L\pi_!(\mathcal{F}) = L\pi_!(Li^*\mathcal{F}) =
L\pi_!(\mathcal{F} \otimes_\mathcal{O}^\mathbf{L} \underline{B})
$$
in $D(\textit{Ab})$.
\end{lemma}

\begin{proof}
It suffices to verify the assumptions of Cohomology on Sites, Lemma
\ref{sites-cohomology-lemma-O-homology-qis}
hold for $\mathcal{O} \to \underline{B}$ on $\mathcal{C}_{B/A}$.
We will use the results of Remark \ref{remark-resolution} without
further mention. Choose a resolution $P_\bullet$ of $B$ over $A$ to get a
suitable cosimplicial object $U_\bullet$ of $\mathcal{C}_{B/A}$.
Since $P_\bullet \to B$ induces a quasi-isomorphism on associated
complexes of abelian groups we see that $L\pi_!\mathcal{O} = B$.
On the other hand $L\pi_!\underline{B}$ is computed by
$\underline{B}(U_\bullet) = B$. This verifies the second assumption of
Cohomology on Sites, Lemma
\ref{sites-cohomology-lemma-O-homology-qis}
and we are done with the proof.
\end{proof}

\begin{lemma}
\label{lemma-apply-O-B-comparison}
Let $A \to B$ be a ring map. Let $\pi$, $\mathcal{O}$, $\underline{B}$
be as in (\ref{equation-pi}). We have
$$
L\pi_!(\mathcal{O}) = L\pi_!(\underline{B}) = B
\quad\text{and}\quad
L_{B/A} = L\pi_!(\Omega_{\mathcal{O}/A} \otimes_\mathcal{O} \underline{B}) =
L\pi_!(\Omega_{\mathcal{O}/A})
$$
in $D(\textit{Ab})$.
\end{lemma}

\begin{proof}
This is just an application of Lemma \ref{lemma-O-homology-B-homology}
(and the first equality on the right is
Lemma \ref{lemma-compute-cotangent-complex}).
\end{proof}

\noindent
Here is a special case of the fundamental triangle that is easy to prove.

\begin{lemma}
\label{lemma-special-case-triangle}
Let $A \to B \to C$ be ring maps. If $B$ is a polynomial algebra over
$A$, then there is a distinguished triangle 
$L_{B/A} \otimes_B^\mathbf{L} C \to L_{C/A} \to L_{C/B} \to
L_{B/A} \otimes_B^\mathbf{L} C[1]$ in $D(C)$.
\end{lemma}

\begin{proof}
We will use the observations of Remark \ref{remark-resolution}
without further mention. Choose a resolution $\epsilon : P_\bullet \to C$
of $C$ over $B$ (for example the standard resolution). Since $B$ is a
polynomial algebra over $A$ we see that $P_\bullet$ is also a resolution of
$C$ over $A$. Hence $L_{C/A}$ is computed by
$\Omega_{P_\bullet/A} \otimes_{P_\bullet, \epsilon} C$
and $L_{C/B}$ is computed by
$\Omega_{P_\bullet/B} \otimes_{P_\bullet, \epsilon} C$.
Since for each $n$ we have the short exact sequence
$0 \to \Omega_{B/A} \otimes_B P_n \to \Omega_{P_n/A} \to \Omega_{P_n/B} \to 0$
(Algebra, Lemma \ref{algebra-lemma-ses-formally-smooth})
and since $L_{B/A} = \Omega_{B/A}[0]$
(Lemma \ref{lemma-cotangent-complex-polynomial-algebra})
we obtain the result.
\end{proof}

\begin{example}
\label{example-resolution-length-two}
Let $A \to B$ be a ring map. In this example we 
will construct an ``explicit'' resolution $P_\bullet$ of $B$ over $A$ of
length $2$. To do this we follow the procedure of the proof of
Proposition \ref{proposition-polynomial}, see also the discussion in
Remark \ref{remark-resolution}.

\medskip\noindent
We choose a surjection $P_0 = A[u_i] \to B$ where $u_i$ is a set of
variables. Choose generators $f_t \in P_0$, $t \in T$ of the ideal
$\Ker(P_0 \to B)$. We choose $P_1 = A[u_i, x_t]$ with face maps
$d_0$ and $d_1$ the unique $A$-algebra maps with $d_j(u_i) = u_i$
and $d_0(x_t) = 0$ and $d_1(x_t) = f_t$. The map $s_0 : P_0 \to P_1$
is the unique $A$-algebra map with $s_0(u_i) = u_i$. It is clear that
$$
P_1 \xrightarrow{d_0 - d_1} P_0 \to B \to 0
$$
is exact, in particular the map $(d_0, d_1) : P_1 \to P_0 \times_B P_0$
is surjective. Thus, if $P_\bullet$ denotes the $1$-truncated
simplicial $A$-algebra given by $P_0$, $P_1$, $d_0$, $d_1$, and $s_0$, then
the augmentation $\text{cosk}_1(P_\bullet) \to B$ is a trivial Kan fibration.
The next step of the procedure in the proof of
Proposition \ref{proposition-polynomial}
is to choose a polynomial algebra $P_2$ and a surjection
$$
P_2 \longrightarrow \text{cosk}_1(P_\bullet)_2
$$
Recall that
$$
\text{cosk}_1(P_\bullet)_2 =
\{(g_0, g_1, g_2) \in P_1^3 \mid d_0(g_0) = d_0(g_1),
d_1(g_0) = d_0(g_2), d_1(g_1) = d_1(g_2)\}
$$
Thinking of $g_i \in P_1$ as a polynomial in $x_t$ the conditions
are
$$
g_0(0) = g_1(0),\quad
g_0(f_t) = g_2(0),\quad
g_1(f_t) = g_2(f_t)
$$
Thus $\text{cosk}_1(P_\bullet)_2$ contains the elements
$y_t = (x_t, x_t, f_t)$ and $z_t = (0, x_t, x_t)$.
Every element $G$ in $\text{cosk}_1(P_\bullet)_2$ is
of the form $G = H + (0, 0, g)$ where $H$ is in the image
of $A[u_i, y_t, z_t] \to \text{cosk}_1(P_\bullet)_2$. Here
$g \in P_1$ is a polynomial with vanishing
constant term such that $g(f_t) = 0$ in $P_0$. Observe that
\begin{enumerate}
\item $g = x_t x_{t'} - f_t x_{t'}$ and
\item $g = \sum r_t x_t$ with $r_t \in P_0$ if $\sum r_t f_t = 0$ in $P_0$
\end{enumerate}
are elements of $P_1$ of the desired form. Let
$$
Rel = \Ker(\bigoplus\nolimits_{t \in T} P_0 \longrightarrow P_0),\quad
(r_t) \longmapsto \sum r_tf_t
$$
We set $P_2 = A[u_i, y_t, z_t, v_r, w_{t, t'}]$ where
$r = (r_t) \in Rel$, with map
$$
P_2 \longrightarrow \text{cosk}_1(P_\bullet)_2
$$
given by $y_t  \mapsto (x_t, x_t, f_t)$,
$z_t \mapsto (0, x_t, x_t)$,
$v_r \mapsto (0, 0, \sum r_t x_t)$, and
$w_{t, t'} \mapsto (0, 0, x_t x_{t'} - f_t x_{t'})$. A calculation
(omitted) shows that this map is surjective. Our choice of the
map displayed above determines the maps $d_0, d_1, d_2 : P_2 \to P_1$.
Finally, the procedure in the proof of
Proposition \ref{proposition-polynomial}
tells us to choose the maps $s_0, s_1 : P_1 \to P_2$ lifting the two
maps $P_1 \to \text{cosk}_1(P_\bullet)_2$. It is clear that we can take
$s_i$ to be the unique $A$-algebra maps determined by
$s_0(x_t) = y_t$ and $s_1(x_t) = z_t$.
\end{example}








\section{Functoriality}
\label{section-functoriality}

\noindent
In this section we consider a commutative square
\begin{equation}
\label{equation-commutative-square}
\vcenter{
\xymatrix{
B \ar[r] & B' \\
A \ar[u] \ar[r] & A' \ar[u]
}
}
\end{equation}
of ring maps. We claim there is a canonical $B$-linear map of complexes
$$
L_{B/A} \longrightarrow L_{B'/A'}
$$
associated to this diagram. Namely, if $P_\bullet \to B$ is the
standard resolution of $B$ over $A$ and $P'_\bullet \to B'$ is the
standard resolution of $B'$ over $A'$, then there is a canonical map
$P_\bullet \to P'_\bullet$
of simplicial $A$-algebras compatible with the augmentations
$P_\bullet \to B$ and $P'_\bullet \to B'$. This can be seen in terms
of the construction of standard resolutions in
Simplicial, Section \ref{simplicial-section-standard}
but in the special case at hand it probably suffices to say simply
that the maps
$$
P_0 = A[B] \longrightarrow A'[B'] = P'_0,\quad
P_1 = A[A[B]] \longrightarrow A'[A'[B']] = P'_1,
$$
and so on are given by the given maps $A \to A'$ and $B \to B'$.
The desired map $L_{B/A} \to L_{B'/A'}$ then comes from the associated
maps $\Omega_{P_n/A} \to \Omega_{P'_n/A'}$.

\medskip\noindent
Another description of the functoriality map can be given as follows.
Let $\mathcal{C} = \mathcal{C}_{B/A}$ and $\mathcal{C}' = \mathcal{C}_{B'/A}'$
be the categories considered in Section \ref{section-compute-L-pi-shriek}.
There is a functor
$$
u : \mathcal{C} \longrightarrow \mathcal{C}',\quad
(P, \alpha) \longmapsto (P \otimes_A A', c \circ (\alpha \otimes 1))
$$
where $c : B \otimes_A A' \to B'$ is the obvious map. As discussed in
Cohomology on Sites, Example
\ref{sites-cohomology-example-morphism-categories}
we obtain a morphism of topoi $g : \Sh(\mathcal{C}) \to \Sh(\mathcal{C}')$
and a commutative diagram of maps of ringed topoi
\begin{equation}
\label{equation-double-square}
\vcenter{
\xymatrix{
(\Sh(\mathcal{C}'), \underline{B}) \ar[d]_\pi &
(\Sh(\mathcal{C}'), \underline{B'}) \ar[d]_\pi \ar[l]^h &
(\Sh(\mathcal{C}), \underline{B'}) \ar[d]_{\pi'} \ar[l]^g \\
(\Sh(*), B) &
(\Sh(*), B') \ar[l]_f &
(\Sh(*), B') \ar[l]
}
}
\end{equation}
Here $h$ is the identity on underlying topoi and given by the ring map
$B \to B'$ on sheaves of rings. 
By Cohomology on Sites, Remark
\ref{sites-cohomology-remark-morphism-fibred-categories}
given $\mathcal{F}$ on $\mathcal{C}$ and $\mathcal{F}'$ on $\mathcal{C}'$
and a transformation $t : \mathcal{F} \to g^{-1}\mathcal{F}'$
we obtain a canonical map $L\pi_!(\mathcal{F}) \to L\pi'_!(\mathcal{F}')$.
If we apply this to the sheaves
$$
\mathcal{F} : (P, \alpha) \mapsto \Omega_{P/A} \otimes_P B,\quad
\mathcal{F}' : (P', \alpha') \mapsto \Omega_{P'/A'} \otimes_{P'} B',
$$
and the transformation $t$ given by the canonical maps
$$
\Omega_{P/A} \otimes_P B \longrightarrow
\Omega_{P \otimes_A A'/A'} \otimes_{P \otimes_A A'} B'
$$
to get a canonical map
$$
L\pi_!(\Omega_{\mathcal{O}/A} \otimes_\mathcal{O} \underline{B})
\longrightarrow
L\pi'_!(\Omega_{\mathcal{O}'/A'} \otimes_{\mathcal{O}'} \underline{B'})
$$
By Lemma \ref{lemma-compute-cotangent-complex} this gives
$L_{B/A} \to L_{B'/A'}$. We omit the verification that this map
agrees with the map defined above in terms of simplicial
resolutions.

\begin{lemma}
\label{lemma-flat-base-change}
Assume (\ref{equation-commutative-square}) induces a quasi-isomorphism
$B \otimes_A^\mathbf{L} A' = B'$. Then, with notation as in
(\ref{equation-double-square}) and
$\mathcal{F}' \in \textit{Ab}(\mathcal{C}')$,
we have $L\pi_!(g^{-1}\mathcal{F}') = L\pi'_!(\mathcal{F}')$.
\end{lemma}

\begin{proof}
We use the results of Remark \ref{remark-resolution} without
further mention. We will apply Cohomology on Sites, Lemma
\ref{sites-cohomology-lemma-get-it-now}. Let $P_\bullet \to B$ be a resolution.
If we can show that $u(P_\bullet) = P_\bullet \otimes_A A' \to B'$
is a quasi-isomorphism, then we are done. The complex of $A$-modules
$s(P_\bullet)$ associated to $P_\bullet$ (viewed as a simplicial $A$-module)
is a free $A$-module resolution of $B$. Namely, $P_n$ is a free $A$-module and
$s(P_\bullet) \to B$ is a quasi-isomorphism. Thus $B \otimes_A^\mathbf{L} A'$
is computed by $s(P_\bullet) \otimes_A A' = s(P_\bullet \otimes_A A')$.
Therefore the assumption of the lemma signifies that
$\epsilon' : P_\bullet \otimes_A A' \to B'$ is a quasi-isomorphism.
\end{proof}

\noindent
The following lemma in particular applies when $A \to A'$ is flat
and $B' = B \otimes_A A'$ (flat base change).

\begin{lemma}
\label{lemma-flat-base-change-cotangent-complex}
If (\ref{equation-commutative-square}) induces a quasi-isomorphism
$B \otimes_A^\mathbf{L} A' = B'$, then the functoriality map
induces an isomorphism
$$
L_{B/A} \otimes_B^\mathbf{L} B' \longrightarrow L_{B'/A'}
$$
\end{lemma}

\begin{proof}
We will use the notation introduced in Equation (\ref{equation-double-square}).
We have
$$
L_{B/A} \otimes_B^\mathbf{L} B' =
L\pi_!(\Omega_{\mathcal{O}/A} \otimes_\mathcal{O} \underline{B})
\otimes_B^\mathbf{L} B' =
L\pi_!(Lh^*(\Omega_{\mathcal{O}/A} \otimes_\mathcal{O} \underline{B}))
$$
the first equality by Lemma \ref{lemma-compute-cotangent-complex}
and the second by Cohomology on Sites, Lemma
\ref{sites-cohomology-lemma-change-of-rings}.
Since $\Omega_{\mathcal{O}/A}$ is a flat $\mathcal{O}$-module,
we see that $\Omega_{\mathcal{O}/A} \otimes_\mathcal{O} \underline{B}$
is a flat $\underline{B}$-module. Thus
$Lh^*(\Omega_{\mathcal{O}/A} \otimes_\mathcal{O} \underline{B}) =
\Omega_{\mathcal{O}/A} \otimes_\mathcal{O} \underline{B'}$
which is equal to
$g^{-1}(\Omega_{\mathcal{O}'/A'} \otimes_{\mathcal{O}'} \underline{B'})$
by inspection.
we conclude by Lemma \ref{lemma-flat-base-change}
and the fact that $L_{B'/A'}$ is computed by
$L\pi'_!(\Omega_{\mathcal{O}'/A'} \otimes_{\mathcal{O}'} \underline{B'})$.
\end{proof}

\begin{remark}
\label{remark-homotopy-triangle}
Suppose that we are given a square (\ref{equation-commutative-square})
such that there exists an arrow $\kappa : B \to A'$ making the diagram
commute:
$$
\xymatrix{
B \ar[r]_\beta \ar[rd]_\kappa & B' \\
A \ar[u] \ar[r]^\alpha & A' \ar[u]
}
$$
In this case we claim the functoriality map $P_\bullet \to P'_\bullet$
is homotopic to the composition $P_\bullet \to B \to A' \to P'_\bullet$.
Namely, using $\kappa$ the functoriality map factors as
$$
P_\bullet \to P_{A'/A', \bullet} \to P'_\bullet
$$
where $P_{A'/A', \bullet}$ is the standard resolution of $A'$ over $A'$.
Since $A'$ is the polynomial algebra on the empty set over $A'$ we
see from Simplicial, Lemma \ref{simplicial-lemma-standard-simplicial-homotopy}
that the augmentation $\epsilon_{A'/A'} : P_{A'/A', \bullet} \to A'$
is a homotopy equivalence of simplicial rings. Observe that the homotopy
inverse map $c : A' \to P_{A'/A', \bullet}$ constructed in the proof of
that lemma is just the structure morphism, hence
we conclude what we want because the two compositions
$$
\xymatrix{
P_\bullet \ar[r] &
P_{A'/A', \bullet} \ar@<1ex>[rr]^{\text{id}}
\ar@<-1ex>[rr]_{c \circ \epsilon_{A'/A'}} & &
P_{A'/A', \bullet} \ar[r] &
P'_\bullet
}
$$
are the two maps discussed above and these are homotopic
(Simplicial, Remark \ref{simplicial-remark-homotopy-pre-post-compose}).
Since the second map $P_\bullet \to P'_\bullet$ induces the zero
map $\Omega_{P_\bullet/A} \to \Omega_{P'_\bullet/A'}$ we conclude
that the functoriality map $L_{B/A} \to L_{B'/A'}$ is homotopic
to zero in this case.
\end{remark}

\begin{lemma}
\label{lemma-cotangent-complex-product}
Let $A \to B$ and $A \to C$ be ring maps.
Then the map $L_{B \times C/A} \to L_{B/A} \oplus L_{C/A}$ is
an isomorphism in $D(B \times C)$.
\end{lemma}

\begin{proof}
Although this lemma can be deduced from the fundamental triangle
we will give a direct and elementary proof of this now.
Factor the ring map $A \to B \times C$ as $A \to A[x] \to B \times C$
where $x \mapsto (1, 0)$. By Lemma \ref{lemma-special-case-triangle}
we have a distinguished triangle
$$
L_{A[x]/A} \otimes_{A[x]}^\mathbf{L} (B \times C) \to L_{B \times C/A} \to
L_{B \times C/A[x]} \to L_{A[x]/A} \otimes_{A[x]}^\mathbf{L} (B \times C)[1]
$$
in $D(B \times C)$. Similarly we have the distinguished triangles
$$
\begin{matrix}
L_{A[x]/A} \otimes_{A[x]}^\mathbf{L} B \to L_{B/A} \to L_{B/A[x]}
\to L_{A[x]/A} \otimes_{A[x]}^\mathbf{L} B[1] \\
L_{A[x]/A} \otimes_{A[x]}^\mathbf{L} C \to L_{C/A} \to L_{C/A[x]}
\to L_{A[x]/A} \otimes_{A[x]}^\mathbf{L} C[1]
\end{matrix}
$$
Thus it suffices to prove the result for $B \times C$ over $A[x]$.
Note that $A[x] \to A[x, x^{-1}]$ is flat, that
$(B \times C) \otimes_{A[x]} A[x, x^{-1}] = B \otimes_{A[x]} A[x, x^{-1}]$,
and that $C \otimes_{A[x]} A[x, x^{-1}] = 0$.
Thus by base change (Lemma \ref{lemma-flat-base-change-cotangent-complex})
the map $L_{B \times C/A[x]} \to L_{B/A[x]} \oplus L_{C/A[x]}$
becomes an isomorphism after inverting $x$.
In the same way one shows that the map becomes an isomorphism after
inverting $x - 1$. This proves the lemma.
\end{proof}




\section{The fundamental triangle}
\label{section-triangle}

\noindent
In this section we consider a sequence of ring maps $A \to B \to C$.
It is our goal to show that this triangle gives rise to a distinguished
triangle
\begin{equation}
\label{equation-triangle}
L_{B/A} \otimes_B^\mathbf{L} C \to L_{C/A} \to L_{C/B} \to
L_{B/A} \otimes_B^\mathbf{L} C[1]
\end{equation}
in $D(C)$. This will be proved in Proposition \ref{proposition-triangle}.
For an alternative approach see Remark \ref{remark-triangle}.

\medskip\noindent
Consider the category $\mathcal{C}_{C/B/A}$
wich is the {\bf opposite} of the category whose objects are
$(P \to B, Q \to C)$ where
\begin{enumerate}
\item $P$ is a polynomial algebra over $A$,
\item $P \to B$ is an $A$-algebra homomorphism,
\item $Q$ is a polynomial algebra over $P$, and
\item $Q \to C$ is a $P$-algebra-homomorphism.
\end{enumerate}
We take the opposite as we want to think of $(P \to B, Q \to C)$
as corresponding to the commutative diagram
$$
\xymatrix{
\Spec(C) \ar[d] \ar[r] & \Spec(Q) \ar[d] \\
\Spec(B) \ar[d] \ar[r] & \Spec(P) \ar[dl] \\
\Spec(A)
}
$$
Let $\mathcal{C}_{B/A}$, $\mathcal{C}_{C/A}$, $\mathcal{C}_{C/B}$
be the categories considered in Section \ref{section-compute-L-pi-shriek}.
There are functors
$$
\begin{matrix}
u_1 : \mathcal{C}_{C/B/A} \to \mathcal{C}_{B/A}, &
(P \to B, Q \to C) \mapsto (P \to B) \\
u_2 : \mathcal{C}_{C/B/A} \to \mathcal{C}_{C/A}, &
(P \to B, Q \to C) \mapsto (Q \to C) \\
u_3 : \mathcal{C}_{C/B/A} \to \mathcal{C}_{C/B}, &
(P \to B, Q \to C) \mapsto (Q \otimes_P B \to C)
\end{matrix}
$$
These functors induce corresponding morphisms of topoi $g_i$.
Let us denote $\mathcal{O}_i = g_i^{-1}\mathcal{O}$ so that we
get morphisms of ringed topoi
\begin{equation}
\label{equation-three-maps}
\begin{matrix}
g_1 : (\Sh(\mathcal{C}_{C/B/A}), \mathcal{O}_1)
\longrightarrow (\Sh(\mathcal{C}_{B/A}), \mathcal{O}) \\
g_2 : (\Sh(\mathcal{C}_{C/B/A}), \mathcal{O}_2)
\longrightarrow (\Sh(\mathcal{C}_{C/A}), \mathcal{O}) \\
g_3 : (\Sh(\mathcal{C}_{C/B/A}), \mathcal{O}_3)
\longrightarrow (\Sh(\mathcal{C}_{C/B}), \mathcal{O})
\end{matrix}
\end{equation}
Let us denote
$\pi : \Sh(\mathcal{C}_{C/B/A}) \to \Sh(*)$,
$\pi_1 : \Sh(\mathcal{C}_{B/A}) \to \Sh(*)$,
$\pi_2 : \Sh(\mathcal{C}_{C/A}) \to \Sh(*)$, and
$\pi_3 : \Sh(\mathcal{C}_{C/B}) \to \Sh(*)$,
so that $\pi = \pi_i \circ g_i$.
We will obtain our distinguished triangle from the identification
of the cotangent complex in Lemma \ref{lemma-compute-cotangent-complex}
and the following lemmas.

\begin{lemma}
\label{lemma-triangle-ses}
With notation as in (\ref{equation-three-maps}) set
$$
\begin{matrix}
\Omega_1 = \Omega_{\mathcal{O}/A} \otimes_\mathcal{O} \underline{B}
\text{ on }\mathcal{C}_{B/A} \\
\Omega_2 = \Omega_{\mathcal{O}/A} \otimes_\mathcal{O} \underline{C}
\text{ on }\mathcal{C}_{C/A} \\
\Omega_3 = \Omega_{\mathcal{O}/B} \otimes_\mathcal{O} \underline{C}
\text{ on }\mathcal{C}_{C/B}
\end{matrix}
$$
Then we have a canonical short exact sequence of sheaves
of $\underline{C}$-modules
$$
0 \to g_1^{-1}\Omega_1 \otimes_{\underline{B}} \underline{C} \to
g_2^{-1}\Omega_2 \to
g_3^{-1}\Omega_3 \to 0
$$
on $\mathcal{C}_{C/B/A}$.
\end{lemma}

\begin{proof}
Recall that $g_i^{-1}$ is gotten by simply precomposing with $u_i$.
Given an object $U = (P \to B, Q \to C)$ we have a split
short exact sequence
$$
0 \to \Omega_{P/A} \otimes Q \to \Omega_{Q/A} \to \Omega_{Q/P} \to 0
$$
for example by Algebra, Lemma \ref{algebra-lemma-ses-formally-smooth}.
Tensoring with $C$ over $Q$ we obtain a short exact sequence
$$
0 \to \Omega_{P/A} \otimes C \to \Omega_{Q/A} \otimes C \to
\Omega_{Q/P} \otimes C \to 0
$$
We have $\Omega_{P/A} \otimes C = \Omega_{P/A} \otimes B \otimes C$
whence this is the value of
$g_1^{-1}\Omega_1 \otimes_{\underline{B}} \underline{C}$
on $U$. The module $\Omega_{Q/A} \otimes C$ is the value of
$g_2^{-1}\Omega_2$ on $U$.
We have $\Omega_{Q/P} \otimes C = \Omega_{Q \otimes_P B/B} \otimes C$
by Algebra, Lemma \ref{algebra-lemma-differentials-base-change}
hence this is the value of
$g_3^{-1}\Omega_3$ on $U$. Thus the short exact sequence of the
lemma comes from assigning to $U$ the last displayed short exact
sequence.
\end{proof}

\begin{lemma}
\label{lemma-polynomial-on-top}
With notation as in (\ref{equation-three-maps})
suppose that $C$ is a polynomial algebra over $B$. Then
$L\pi_!(g_3^{-1}\mathcal{F}) = L\pi_{3, !}\mathcal{F} = \pi_{3, !}\mathcal{F}$
for any abelian sheaf $\mathcal{F}$ on $\mathcal{C}_{C/B}$
\end{lemma}

\begin{proof}
Write $C = B[E]$ for some set $E$. Choose a resolution
$P_\bullet \to B$ of $B$ over $A$. For every $n$ consider
the object $U_n = (P_n \to B, P_n[E] \to C)$ of $\mathcal{C}_{C/B/A}$.
Then $U_\bullet$ is a cosimplicial object of $\mathcal{C}_{C/B/A}$.
Note that $u_3(U_\bullet)$ is the constant cosimplicial
object of $\mathcal{C}_{C/B}$ with value $(C \to C)$.
We will prove that the object $U_\bullet$ of $\mathcal{C}_{C/B/A}$
satisfies the hypotheses of
Cohomology on Sites, Lemma
\ref{sites-cohomology-lemma-compute-by-cosimplicial-resolution}.
This implies the lemma as it shows that $L\pi_!(g_3^{-1}\mathcal{F})$
is computed by the constant simplicial abelian group
$\mathcal{F}(C \to C)$ which is the value of
$L\pi_{3, !}\mathcal{F} = \pi_{3, !}\mathcal{F}$ by
Lemma \ref{lemma-pi-lower-shriek-polynomial-algebra}.

\medskip\noindent
Let $U = (\beta : P \to B, \gamma : Q \to C)$ be an object of
$\mathcal{C}_{C/B/A}$. We may write $P = A[S]$ and $Q = A[S \amalg T]$
by the definition of our category $\mathcal{C}_{C/B/A}$. We have to show that
$$
\Mor_{\mathcal{C}_{C/B/A}}(U_\bullet, U)
$$
is homotopy equivalent to a singleton simplicial set $*$. Observe that this
simplicial set is the product
$$
\prod\nolimits_{s \in S} F_s \times \prod\nolimits_{t \in T} F'_t
$$
where $F_s$ is the corresponding simplicial set for
$U_s = (A[\{s\}] \to B, A[\{s\}] \to C)$
and $F'_t$ is the corresponding simplicial set for
$U_t = (A \to B, A[\{t\}] \to C)$. Namely, the object $U$
is the product $\prod U_s \times \prod U_t$ in $\mathcal{C}_{C/B/A}$.
It suffices each $F_s$ and $F'_t$ is homotopy equivalent to $*$, see
Simplicial, Lemma \ref{simplicial-lemma-products-homotopy}.
The case of $F_s$ follows as $P_\bullet \to B$ is a trivial Kan
fibration (as a resolution) and $F_s$ is the fibre of this map over
$\beta(s)$. (Use Simplicial, Lemmas
\ref{simplicial-lemma-trivial-kan-base-change} and
\ref{simplicial-lemma-trivial-kan-homotopy}).
The case of $F'_t$ is more interesting. Here we are saying that
the fibre of
$$
P_\bullet[E] \longrightarrow C = B[E]
$$
over $\gamma(t) \in C$ is homotopy equivalent to a point. In fact we
will show this map is a trivial Kan fibration. Namely,
$P_\bullet \to B$ is a trivial can fibration. For any ring $R$
we have
$$
R[E] =
\colim_{\Sigma \subset \text{Map}(E, \mathbf{Z}_{\geq 0})\text{ finite}}
\prod\nolimits_{I \in \Sigma} R
$$
(filtered colimit). Thus the displayed map of simplicial sets is a
filtered colimit of trivial Kan fibrations, whence a trivial Kan fibration
by Simplicial, Lemma \ref{simplicial-lemma-filtered-colimit-trivial-kan}.
\end{proof}

\begin{lemma}
\label{lemma-triangle-compute-lower-shriek}
With notation as in (\ref{equation-three-maps}) we have
$Lg_{i, !} \circ g_i^{-1} = \text{id}$ for $i = 1, 2, 3$
and hence also $L\pi_! \circ g_i^{-1} = L\pi_{i, !}$ for
$i = 1, 2, 3$.
\end{lemma}

\begin{proof}
Proof for $i = 1$. We claim the functor $\mathcal{C}_{C/B/A}$
is a fibred category over $\mathcal{C}_{B/A}$
Namely, suppose given $(P \to B, Q \to C)$
and a morphism $(P' \to B) \to (P \to B)$ of $\mathcal{C}_{B/A}$.
Recall that this means we have an $A$-algebra homomorphism
$P \to P'$ compatible with maps to $B$. Then we set $Q' = Q \otimes_P P'$
with induced map to $C$ and the morphism
$$
(P' \to B, Q' \to C) \longrightarrow (P \to B, Q \to C)
$$
in $\mathcal{C}_{C/B/A}$ (note reversal arrows again) is strongly cartesian
in $\mathcal{C}_{C/B/A}$ over $\mathcal{C}_{B/A}$. Moreover, observe
that the fibre category of $u_1$ over $P \to B$ is the category
$\mathcal{C}_{C/P}$. Let $\mathcal{F}$ be an abelian sheaf on
$\mathcal{C}_{B/A}$. Since we have a fibred category we may apply
Cohomology on Sites, Lemma
\ref{sites-cohomology-lemma-compute-left-derived-pi-shriek}.
Thus $L_ng_{1, !}g_1^{-1}\mathcal{F}$ is the (pre)sheaf
which assigns to $U \in \Ob(\mathcal{C}_{B/A})$ the $n$th homology of
$g_1^{-1}\mathcal{F}$ restricted to the fibre category over $U$.
Since these restrictions are constant the desired result follows from
Lemma \ref{lemma-pi-lower-shriek-constant-sheaf}
via our identifications of fibre categories above.

\medskip\noindent
The case $i = 2$.
We claim $\mathcal{C}_{C/B/A}$ is a fibred category over $\mathcal{C}_{C/A}$
is a fibred category. Namely, suppose given $(P \to B, Q \to C)$
and a morphism $(Q' \to C) \to (Q \to C)$ of $\mathcal{C}_{C/A}$.
Recall that this means we have a $B$-algebra homomorphism
$Q \to Q'$ compatible with maps to $C$. Then
$$
(P \to B, Q' \to C) \longrightarrow (P \to B, Q \to C)
$$
is strongly cartesian in $\mathcal{C}_{C/B/A}$ over $\mathcal{C}_{C/A}$.
Note that the fibre category of $u_2$ over $Q \to C$ has an final
(beware reversal arrows) object, namely, $(A \to B, Q \to C)$. Let
$\mathcal{F}$ be an abelian sheaf on $\mathcal{C}_{C/A}$.
Since we have a fibred category we may apply
Cohomology on Sites, Lemma
\ref{sites-cohomology-lemma-compute-left-derived-pi-shriek}.
Thus $L_ng_{2, !}g_2^{-1}\mathcal{F}$ is the (pre)sheaf
which assigns to $U \in \Ob(\mathcal{C}_{C/A})$ the $n$th homology of
$g_1^{-1}\mathcal{F}$ restricted to the fibre category over $U$.
Since these restrictions are constant the desired result follows from
Cohomology on Sites, Lemma \ref{sites-cohomology-lemma-initial-final}
because the fibre categories all have final objects.

\medskip\noindent
The case $i = 3$. In this case we will apply
Cohomology on Sites, Lemma
\ref{sites-cohomology-lemma-compute-left-derived-g-shriek}
to $u = u_3 : \mathcal{C}_{C/B/A} \to \mathcal{C}_{C/B}$
and $\mathcal{F}' = g_3^{-1}\mathcal{F}$ for some abelian sheaf
$\mathcal{F}$ on $\mathcal{C}_{C/B}$.
Suppose $U = (\overline{Q} \to C)$ is an object of $\mathcal{C}_{C/B}$.
Then $\mathcal{I}_U = \mathcal{C}_{\overline{Q}/B/A}$ (again beware
of reversal of arrows). The sheaf $\mathcal{F}'_U$ is given by the
rule $(P \to B, Q \to \overline{Q}) \mapsto \mathcal{F}(Q \otimes_P B \to C)$.
In other words, this sheaf is the pullback of a sheaf
on $\mathcal{C}_{\overline{Q}/C}$ via the morphism
$\Sh(\mathcal{C}_{\overline{Q}/B/A}) \to \Sh(\mathcal{C}_{\overline{Q}/B})$.
Thus Lemma \ref{lemma-polynomial-on-top} shows that
$H_n(\mathcal{I}_U, \mathcal{F}'_U) = 0$ for $n > 0$
and equal to $\mathcal{F}(\overline{Q} \to C)$ for $n = 0$.
The aforementioned Cohomology on Sites, Lemma
\ref{sites-cohomology-lemma-compute-left-derived-g-shriek}
implies that $Lg_{3, !}(g_3^{-1}\mathcal{F}) = \mathcal{F}$ and
the proof is done.
\end{proof}

\begin{proposition}
\label{proposition-triangle}
Let $A \to B \to C$ be ring maps. There is a canonical distinguished
triangle
$$
L_{B/A} \otimes_B^\mathbf{L} C \to L_{C/A} \to L_{C/B} \to
L_{B/A} \otimes_B^\mathbf{L} C[1]
$$
in $D(C)$.
\end{proposition}

\begin{proof}
Consider the short exact sequence of sheaves of
Lemma \ref{lemma-triangle-ses}
and apply the derived functor $L\pi_!$ to obtain a distinguished
triangle
$$
L\pi_!(g_1^{-1}\Omega_1 \otimes_{\underline{B}} \underline{C}) \to
L\pi_!(g_2^{-1}\Omega_2) \to
L\pi_!(g_3^{-1}\Omega_3) \to
L\pi_!(g_1^{-1}\Omega_1 \otimes_{\underline{B}} \underline{C})[1]
$$
in $D(C)$. Using Lemmas \ref{lemma-triangle-compute-lower-shriek} and
\ref{lemma-compute-cotangent-complex}
we see that the second and third terms agree with $L_{C/A}$ and $L_{C/B}$
and the first one equals
$$
L\pi_{1, !}(\Omega_1 \otimes_{\underline{B}} \underline{C}) =
L\pi_{1, !}(\Omega_1) \otimes_B^\mathbf{L} C =
L_{B/A} \otimes_B^\mathbf{L} C
$$
The first equality by Cohomology on Sites, Lemma
\ref{sites-cohomology-lemma-change-of-rings}
(and flatness of $\Omega_1$ as a sheaf of modules over $\underline{B}$)
and the second by Lemma \ref{lemma-compute-cotangent-complex}.
\end{proof}

\begin{remark}
\label{remark-triangle}
We sketch an alternative, perhaps simpler, proof of the existence of
the fundamental triangle.
Let $A \to B \to C$ be ring maps and assume that $B \to C$ is injective.
Let $P_\bullet \to B$ be the standard resolution of $B$ over $A$ and
let $Q_\bullet \to C$ be the standard resolution of $C$ over $B$.
Picture
$$
\xymatrix{
P_\bullet : &
A[A[A[B]]] \ar[d]
\ar@<2ex>[r]
\ar@<0ex>[r]
\ar@<-2ex>[r]
&
A[A[B]] \ar[d]
\ar@<1ex>[r]
\ar@<-1ex>[r]
\ar@<1ex>[l]
\ar@<-1ex>[l]
&
A[B] \ar[d] \ar@<0ex>[l] \ar[r] &
B \\
Q_\bullet : &
A[A[A[C]]]
\ar@<2ex>[r]
\ar@<0ex>[r]
\ar@<-2ex>[r]
&
A[A[C]]
\ar@<1ex>[r]
\ar@<-1ex>[r]
\ar@<1ex>[l]
\ar@<-1ex>[l]
&
A[C] \ar@<0ex>[l] \ar[r] &
C
}
$$
Observe that since $B \to C$ is injective, the ring $Q_n$ is a
polynomial algebra over $P_n$ for all $n$. Hence we obtain a cosimplicial
object in $\mathcal{C}_{C/B/A}$ (beware reversal arrows).
Now set $\overline{Q}_\bullet = Q_\bullet \otimes_{P_\bullet} B$.
The key to the proof of Proposition \ref{proposition-triangle}
is to show that $\overline{Q}_\bullet$ is a resolution of $C$ over $B$.
This follows from Cohomology on Sites, Lemma
\ref{sites-cohomology-lemma-O-homology-qis}
applied to $\mathcal{C} = \Delta$, $\mathcal{O} = P_\bullet$,
$\mathcal{O}' = B$, and $\mathcal{F} = Q_\bullet$ (this uses that $Q_n$
is flat over $P_n$; see Cohomology on Sites, Remark
\ref{sites-cohomology-remark-simplicial-modules} to relate simplicial modules
to sheaves). The key fact implies that the distinguished triangle of
Proposition \ref{proposition-triangle}
is the distinguished triangle associated to the short exact sequence
of simplicial $C$-modules
$$
0 \to
\Omega_{P_\bullet/A} \otimes_{P_\bullet} C \to
\Omega_{Q_\bullet/A} \otimes_{Q_\bullet} C \to
\Omega_{\overline{Q}_\bullet/B} \otimes_{\overline{Q}_\bullet} C \to 0
$$
which is deduced from the short exact sequences
$0 \to \Omega_{P_n/A} \otimes_{P_n} Q_n \to \Omega_{Q_n/A} \to
\Omega_{Q_n/P_n} \to 0$ of
Algebra, Lemma \ref{algebra-lemma-ses-formally-smooth}.
Namely, by Remark \ref{remark-resolution} and the key fact the complex on the
right hand side represents $L_{C/B}$ in $D(C)$.

\medskip\noindent
If $B \to C$ is not injective, then we can use the above to get a
fundamental triangle for $A \to B \to B \times C$. Since
$L_{B \times C/B} \to L_{B/B} \oplus L_{C/B}$ and
$L_{B \times C/A} \to L_{B/A} \oplus L_{C/A}$
are quasi-isomorphism in $D(B \times C)$
(Lemma \ref{lemma-cotangent-complex-product})
this induces the desired distinguished triangle in $D(C)$
by tensoring with the flat ring map $B \times C \to C$.
\end{remark}

\begin{remark}
\label{remark-explicit-map}
Let $A \to B \to C$ be ring maps with $B \to C$ injective.
Recall the notation $P_\bullet$, $Q_\bullet$, $\overline{Q}_\bullet$ of
Remark \ref{remark-triangle}.
Let $R_\bullet$ be the standard resolution of $C$ over $B$.
In this remark we explain how to get the canonical identification
of $\Omega_{\overline{Q}_\bullet/B} \otimes_{\overline{Q}_\bullet} C$
with $L_{C/B} = \Omega_{R_\bullet/B} \otimes_{R_\bullet} C$.
Let $S_\bullet \to B$ be the standard resolution of $B$ over $B$.
Note that the functoriality map $S_\bullet \to R_\bullet$ identifies
$R_n$ as a polynomial algebra over $S_n$ because $B \to C$ is injective.
For example in degree $0$ we have the map $B[B] \to B[C]$, in degree
$1$ the map $B[B[B]] \to B[B[C]]$, and so on. Thus
$\overline{R}_\bullet = R_\bullet \otimes_{S_\bullet} B$
is a simplicial polynomial algebra
over $B$ as well and it follows (as in Remark \ref{remark-triangle}) from
Cohomology on Sites, Lemma
\ref{sites-cohomology-lemma-O-homology-qis}
that $\overline{R}_\bullet \to C$ is a resolution. Since we have
a commutative diagram
$$
\xymatrix{
Q_\bullet \ar[r] & R_\bullet \\
P_\bullet \ar[u] \ar[r] & S_\bullet \ar[u] \ar[r] & B
}
$$
we obtain a canonical map
$\overline{Q}_\bullet = Q_\bullet \otimes_{P_\bullet} B \to
\overline{R}_\bullet$. Thus the maps
$$
L_{C/B} = \Omega_{R_\bullet/B} \otimes_{R_\bullet} C
\longrightarrow
\Omega_{\overline{R}_\bullet/B} \otimes_{\overline{R}_\bullet} C
\longleftarrow
\Omega_{\overline{Q}_\bullet/B} \otimes_{\overline{Q}_\bullet} C
$$
are quasi-isomorphisms (Remark \ref{remark-resolution}) and composing
one with the inverse of the other gives the desired identification.
\end{remark}





\section{Localization and \'etale ring maps}
\label{section-localization}

\noindent
In this section we study what happens if we localize our rings.
Let $A \to A' \to B$ be ring maps such that $B = B \otimes_A^\mathbf{L} A'$.
This happens for example if $A' = S^{-1}A$ is the localization of $A$
at a multiplicative subset $S \subset A$. In this
case for an abelian sheaf $\mathcal{F}'$ on $\mathcal{C}_{B/A'}$
the homology of $g^{-1}\mathcal{F}'$ over $\mathcal{C}_{B/A}$ agrees with
the homology of $\mathcal{F}'$ over $\mathcal{C}_{B/A'}$, see
Lemma \ref{lemma-flat-base-change} for a precise statement.

\begin{lemma}
\label{lemma-localize-at-bottom}
Let $A \to A' \to B$ be ring maps such that $B = B \otimes_A^\mathbf{L} A'$.
Then $L_{B/A} = L_{B/A'}$ in $D(B)$.
\end{lemma}

\begin{proof}
According to the discussion above (i.e., using
Lemma \ref{lemma-flat-base-change})
and Lemma \ref{lemma-compute-cotangent-complex}
we have to show that the sheaf given
by the rule $(P \to B) \mapsto \Omega_{P/A} \otimes_P B$ on $\mathcal{C}_{B/A}$
is the pullback of the sheaf given by the rule
$(P \to B) \mapsto \Omega_{P/A'} \otimes_P B$.
The pullback functor $g^{-1}$ is given by precomposing with the
functor $u : \mathcal{C}_{B/A} \to \mathcal{C}_{B/A'}$,
$(P \to B) \mapsto (P \otimes_A A' \to B)$.
Thus we have to show that
$$
\Omega_{P/A} \otimes_P B =
\Omega_{P \otimes_A A'/A'} \otimes_{(P \otimes_A A')} B
$$
By Algebra, Lemma \ref{algebra-lemma-differentials-base-change}
the right hand side is equal to
$$
(\Omega_{P/A} \otimes_A A') \otimes_{(P \otimes_A A')} B
$$
Since $P$ is a polynomial algebra over $A$ the module
$\Omega_{P/A}$ is free and the equality is obvious.
\end{proof}

\begin{lemma}
\label{lemma-derived-diagonal}
Let $A \to B$ be a ring map such that $B = B \otimes_A^\mathbf{L} B$.
Then $L_{B/A} = 0$ in $D(B)$.
\end{lemma}

\begin{proof}
This is true because $L_{B/A} = L_{B/B} = 0$ by
Lemmas \ref{lemma-localize-at-bottom} and
\ref{lemma-cotangent-complex-polynomial-algebra}.
\end{proof}

\begin{lemma}
\label{lemma-bootstrap}
Let $A \to B$ be a ring map such that $\text{Tor}^A_i(B, B) = 0$ for $i > 0$
and such that $L_{B/B \otimes_A B} = 0$.
Then $L_{B/A} = 0$ in $D(B)$.
\end{lemma}

\begin{proof}
By Lemma \ref{lemma-flat-base-change-cotangent-complex} we see that
$L_{B/A} \otimes_B^\mathbf{L} (B \otimes_A B) = L_{B \otimes_A B/B}$.
Now we use the distinguished triangle (\ref{equation-triangle})
$$
L_{B \otimes_A B/B} \otimes^\mathbf{L}_{(B \otimes_A B)} B \to
L_{B/B} \to L_{B/B \otimes_A B} \to
L_{B \otimes_A B/B} \otimes^\mathbf{L}_{(B \otimes_A B)} B[1]
$$
associated to the ring maps $B \to B \otimes_A B \to B$ and the vanishing of
$L_{B/B}$ (Lemma \ref{lemma-cotangent-complex-polynomial-algebra}) and
$L_{B/B \otimes_A B}$ (assumed) to see that
$$
0 =
L_{B \otimes_A B/B} \otimes^\mathbf{L}_{(B \otimes_A B)} B =
L_{B/A} \otimes_B^\mathbf{L} (B \otimes_A B)
\otimes^\mathbf{L}_{(B \otimes_A B)} B = L_{B/A}
$$
as desired.
\end{proof}

\begin{lemma}
\label{lemma-when-zero}
The cotangent complex $L_{B/A}$ is zero in each of the following cases:
\begin{enumerate}
\item $A \to B$ and $B \otimes_A B \to B$ are flat, i.e., $A \to B$
is weakly \'etale
(More on Algebra, Definition \ref{more-algebra-definition-weakly-etale}),
\item $A \to B$ is a flat epimorphism of rings,
\item $B = S^{-1}A$ for some multiplicative subset $S \subset A$,
\item $A \to B$ is unramified and flat,
\item $A \to B$ is \'etale,
\item $A \to B$ is a filtered colimit of ring maps for which
the cotangent complex vanishes,
\item $B$ is a henselization of a local ring of $A$,
\item $B$ is a strict henselization of a local ring of $A$, and
\item add more here.
\end{enumerate}
\end{lemma}

\begin{proof}
In case (1) we may apply
Lemma \ref{lemma-derived-diagonal}
to the surjective flat ring map $B \otimes_A B \to B$
to conclude that $L_{B/B \otimes_A B} = 0$ and then we use
Lemma \ref{lemma-bootstrap}
to conclude. The cases (2) -- (5) are each special cases of (1).
Part (6) follows from Lemma \ref{lemma-colimit-cotangent-complex}.
Parts (7) and (8) follows from the fact that (strict) henselizations
are filtered colimits of \'etale ring extensions of $A$, see
Algebra, Lemmas \ref{algebra-lemma-henselization-different} and
\ref{algebra-lemma-strict-henselization-different}.
\end{proof}

\begin{lemma}
\label{lemma-localize-on-top}
Let $A \to B \to C$ be ring maps such that $L_{C/B} = 0$.
Then $L_{C/A} = L_{B/A} \otimes_B^\mathbf{L} C$.
\end{lemma}

\begin{proof}
This is a trivial consequence of 
the distinguished triangle (\ref{equation-triangle}).
\end{proof}

\begin{lemma}
\label{lemma-localize}
Let $A \to B$ be ring maps and $S \subset A$, $T \subset B$ multiplicative
subsets such that $S$ maps into $T$.
Then $L_{T^{-1}B/S^{-1}A} = L_{B/A} \otimes_B T^{-1}B$
in $D(T^{-1}B)$.
\end{lemma}

\begin{proof}
Lemma \ref{lemma-localize-on-top} shows that
$L_{T^{-1}B/A} = L_{B/A} \otimes_B T^{-1}B$
and Lemma \ref{lemma-localize-at-bottom}
shows that $L_{T^{-1}B/A} = L_{T^{-1}B/S^{-1}A}$.
\end{proof}

\begin{lemma}
\label{lemma-cotangent-complex-henselization}
Let $A \to B$ be a local ring homomorphism of local rings.
Let $A^h \to B^h$, resp.\ $A^{sh} \to B^{sh}$ be the induced
maps of henselizations, resp.\ strict henselizations.
Then
$$
L_{B^h/A^h} = L_{B^h/A} = L_{B/A} \otimes_B^\mathbf{L} B^h
\quad\text{resp.}\quad
L_{B^{sh}/A^{sh}} = L_{B^{sh}/A} = L_{B/A} \otimes_B^\mathbf{L} B^{sh}
$$
in $D(B^h)$, resp.\ $D(B^{sh})$.
\end{lemma}

\begin{proof}
The complexes $L_{A^h/A}$, $L_{A^{sh}/A}$, $L_{B^h/B}$, and
$L_{B^{sh}/B}$ are all zero by Lemma \ref{lemma-when-zero}.
Using the fundamental distinguished triangle (\ref{equation-triangle})
for $A \to B \to B^h$ we obtain
$L_{B^h/A} = L_{B/A} \otimes_B^\mathbf{L} B^h$.
Using the fundamental triangle for $A \to A^h \to B^h$
we obtain $L_{B^h/A^h} = L_{B^h/A}$.
Similarly for strict henselizations.
\end{proof}




\section{Smooth ring maps}
\label{section-smooth}

\noindent
Let $C \to B$ be a surjection of rings with kernel $I$. Let us call such
a ring map ``weakly quasi-regular'' if $I/I^2$ is a flat $B$-module and
$\text{Tor}_*^C(B, B)$ is the exterior algebra on $I/I^2$.
The generalization to ``smooth ring maps'' of what is done in
Lemma \ref{lemma-when-zero} for ``\'etale ring maps'' is to look
at flat ring maps $A \to B$ such that the multiplication map
$B \otimes_A B \to B$ is weakly quasi-regular. For the moment we just stick to
smooth ring maps.

\begin{lemma}
\label{lemma-when-projective}
If $A \to B$ is a smooth ring map, then $L_{B/A} = \Omega_{B/A}[0]$.
\end{lemma}

\begin{proof}
We have the agreement in cohomological degree $0$ by
Lemma \ref{lemma-identify-H0}.
Thus it suffices to prove the other cohomology groups
are zero. It suffices to prove this locally on $\Spec(B)$ as
$L_{B_g/A} = (L_{B/A})_g$ for $g \in B$ by Lemma \ref{lemma-localize-on-top}.
Thus we may assume that $A \to B$ is standard smooth
(Algebra, Lemma \ref{algebra-lemma-smooth-syntomic}), i.e.,
that we can factor $A \to B$ as
$A \to A[x_1, \ldots, x_n] \to B$ with $A[x_1, \ldots, x_n] \to B$
\'etale. In this case Lemmas \ref{lemma-when-zero} and
Lemma \ref{lemma-localize-on-top} show that
$L_{B/A} = L_{A[x_1, \ldots, x_n]/A} \otimes B$
whence the conclusion by
Lemma \ref{lemma-cotangent-complex-polynomial-algebra}.
\end{proof}





\section{Positive characteristic}
\label{section-positive-characteristic}

\noindent
In this section we fix a prime number $p$.
If $A$ is a ring with $p = 0$ in $A$, then $F_A : A \to A$
denotes the Frobenius endomorphism $a \mapsto a^p$.

\begin{lemma}
\label{lemma-frobenius-homotopy}
Let $A \to B$ be a ring map with $p = 0$ in $A$. Let $P_\bullet$ be the
standard resolution of $B$ over $A$. The map $P_\bullet \to P_\bullet$
induced by the diagram
$$
\xymatrix{
B \ar[r]_{F_B} & B \\
A \ar[u] \ar[r]^{F_A} & A \ar[u]
}
$$
discussed in Section \ref{section-functoriality} is homotopic to the Frobenius
endomorphism $P_\bullet \to P_\bullet$ given by Frobenius on each $P_n$.
\end{lemma}

\begin{proof}
Let $\mathcal{A}$ be the category of $\mathbf{F}_p$-algebra maps
$A \to B$. Let $\mathcal{S}$ be the category of pairs $(A, E)$
where $A$ is an $\mathbf{F}_p$-algebra and $E$ is a set. Consider the
adjoint functors
$$
V : \mathcal{A} \to \mathcal{S}, \quad (A \to B) \mapsto (A, B)
$$
and
$$
U : \mathcal{S} \to \mathcal{A}, \quad (A, E) \mapsto (A \to A[E])
$$
Let $X$ be the simplicial object in 
in the category of functors from $\mathcal{A}$ to $\mathcal{A}$
constructed in Simplicial, Section \ref{simplicial-section-standard}.
It is clear that $P_\bullet = X(A \to B)$ because if we fix
$A$ then.

\medskip\noindent
Set $Y = U \circ V$. Recall that $X$ is constructed from $Y$
and certain maps and has terms $X_n = Y \circ \ldots \circ Y$
with $n + 1$ terms; the construction is given in
Simplicial, Example \ref{simplicial-example-godement} and please see
proof of Simplicial, Lemma \ref{simplicial-lemma-standard-simplicial}
for details.

\medskip\noindent
Let $f : \text{id}_\mathcal{A} \to \text{id}_\mathcal{A}$
be the Frobenius endomorphism of the identity functor.
In other words, we set $f_{A \to B} = (F_A, F_B) : (A \to B) \to (A \to B)$.
Then our two maps on $X(A \to B)$ are given by the natural transformations
$f \star 1_X$ and $1_X \star f$. Details omitted.
Thus we conclude by Simplicial, Lemma
\ref{simplicial-lemma-godement-before-after}.
\end{proof}

\begin{lemma}
\label{lemma-frobenius-acts-as-zero}
Let $p$ be a prime number. Let $A \to B$ be a ring homomorphism
and assume that $p = 0$ in $A$. The map $L_{B/A} \to L_{B/A}$
of Section \ref{section-functoriality} induced by the
Frobenius maps $F_A$ and $F_B$ is homotopic to zero.
\end{lemma}

\begin{proof}
Let $P_\bullet$ be the standard resolution of $B$ over $A$.
By Lemma \ref{lemma-frobenius-homotopy} the map $P_\bullet \to P_\bullet$
induced by $F_A$ and $F_B$ is homotopic to the map
$F_{P_\bullet} : P_\bullet \to P_\bullet$ given by
Frobenius on each term. Hence we obtain what we want as clearly
$F_{P_\bullet}$ induces the zero map $\Omega_{P_n/A} \to \Omega_{P_n/A}$
(since the derivative of a $p$th power is zero).
\end{proof}

\begin{lemma}
\label{lemma-perfect-zero}
Let $p$ be a prime number. Let $A \to B$ be a ring homomorphism
and assume that $p = 0$ in $A$. If $A$ and $B$ are perfect, then
$L_{B/A}$ is zero in $D(B)$.
\end{lemma}

\begin{proof}
The map $(F_A, F_B) : (A \to B) \to (A \to B)$ is an isomorphism
hence induces an isomorphism on $L_{B/A}$ and on the other hand
induces zero on $L_{B/A}$ by Lemma \ref{lemma-frobenius-acts-as-zero}.
\end{proof}





\section{Comparison with the naive cotangent complex}
\label{section-surjections}

\noindent
The naive cotangent complex was introduced in
Algebra, Section \ref{algebra-section-netherlander}.

\begin{remark}
\label{remark-make-map}
Let $A \to B$ be a ring map.  Working on $\mathcal{C}_{B/A}$ as in
Section \ref{section-compute-L-pi-shriek} let
$\mathcal{J} \subset \mathcal{O}$ be the kernel of
$\mathcal{O} \to \underline{B}$. Note that $L\pi_!(\mathcal{J}) = 0$ by
Lemma \ref{lemma-apply-O-B-comparison}. Set
$\Omega =  \Omega_{\mathcal{O}/A} \otimes_\mathcal{O} \underline{B}$
so that
$L_{B/A} = L\pi_!(\Omega)$ by Lemma \ref{lemma-compute-cotangent-complex}.
It follows that $L\pi_!(\mathcal{J} \to \Omega) = L\pi_!(\Omega) = L_{B/A}$.
Thus, for any object $U = (P \to B)$ of $\mathcal{C}_{B/A}$ we obtain a map
\begin{equation}
\label{equation-comparison-map-A}
(J \to \Omega_{P/A} \otimes_P B) \longrightarrow L_{B/A}
\end{equation}
where $J = \Ker(P \to B)$ in $D(A)$, see
Cohomology on Sites, Remark
\ref{sites-cohomology-remark-map-evaluation-to-derived}.
Continuing in this manner, note that
$L\pi_!(\mathcal{J} \otimes_\mathcal{O}^\mathbf{L} \underline{B}) =
L\pi_!(\mathcal{J}) = 0$ by
Lemma \ref{lemma-O-homology-B-homology}.
Since $\text{Tor}_0^\mathcal{O}(\mathcal{J}, \underline{B}) =
\mathcal{J}/\mathcal{J}^2$
the spectral sequence
$$
H_p(\mathcal{C}_{B/A}, \text{Tor}_q^\mathcal{O}(\mathcal{J}, \underline{B}))
\Rightarrow 
H_{p + q}(\mathcal{C}_{B/A},
\mathcal{J} \otimes_\mathcal{O}^\mathbf{L} \underline{B}) = 0
$$
(dual of
Derived Categories, Lemma \ref{derived-lemma-two-ss-complex-functor})
implies that
$H_0(\mathcal{C}_{B/A}, \mathcal{J}/\mathcal{J}^2) = 0$
and $H_1(\mathcal{C}_{B/A}, \mathcal{J}/\mathcal{J}^2) = 0$.
It follows that the complex of $\underline{B}$-modules
$\mathcal{J}/\mathcal{J}^2 \to \Omega$ satisfies
$\tau_{\geq -1}L\pi_!(\mathcal{J}/\mathcal{J}^2 \to \Omega) =
\tau_{\geq -1}L_{B/A}$.
Thus, for any object $U = (P \to B)$ of $\mathcal{C}_{B/A}$ we obtain a map
\begin{equation}
\label{equation-comparison-map}
(J/J^2 \to \Omega_{P/A} \otimes_P B) \longrightarrow \tau_{\geq -1}L_{B/A}
\end{equation}
in $D(B)$, see
Cohomology on Sites, Remark
\ref{sites-cohomology-remark-map-evaluation-to-derived}.
\end{remark}

\noindent
The first case is where we have a surjection of rings.

\begin{lemma}
\label{lemma-surjection}
\begin{slogan}
The cohomology of the cotangent complex of a surjective ring map is trivial in
degree zero; it is the kernel modulo its square in degree $-1$.
\end{slogan}
Let $A \to B$ be a surjective ring map with kernel $I$.
Then $H^0(L_{B/A}) = 0$ and $H^{-1}(L_{B/A}) = I/I^2$.
This isomorphism comes from the map (\ref{equation-comparison-map})
for the object $(A \to B)$ of $\mathcal{C}_{B/A}$.
\end{lemma}

\begin{proof}
We will show below (using the surjectivity of $A \to B$)
that there exists a short exact sequence
$$
0 \to \pi^{-1}(I/I^2) \to \mathcal{J}/\mathcal{J}^2 \to \Omega \to 0
$$
of sheaves on $\mathcal{C}_{B/A}$. Taking $L\pi_!$ and
the associated long exact sequence of homology, and using the
vanishing of $H_1(\mathcal{C}_{B/A}, \mathcal{J}/\mathcal{J}^2)$ and
$H_0(\mathcal{C}_{B/A}, \mathcal{J}/\mathcal{J}^2)$
shown in Remark \ref{remark-make-map} we obtain what we want using
Lemma \ref{lemma-pi-lower-shriek-constant-sheaf}.

\medskip\noindent
What is left is to verify the local statement mentioned above.
For every object $U = (P \to B)$ of $\mathcal{C}_{B/A}$
we can choose an isomorphism $P = A[E]$ such that the map
$P \to B$ maps each $e \in E$ to zero. Then
$J = \mathcal{J}(U) \subset P = \mathcal{O}(U)$
is equal to $J = IP + (e; e \in E)$. The value on $U$ of the short sequence
of sheaves above is the sequence
$$
0 \to I/I^2 \to J/J^2 \to \Omega_{P/A} \otimes_P B \to 0
$$
Verification omitted (hint: the only tricky point is that
$IP \cap J^2 = IJ$; which follows for example from
More on Algebra, Lemma \ref{more-algebra-lemma-conormal-sequence-H1-regular}).
\end{proof}

\begin{lemma}
\label{lemma-relation-with-naive-cotangent-complex}
Let $A \to B$ be a ring map. Then $\tau_{\geq -1}L_{B/A}$
is canonically quasi-isomorphic to the naive cotangent complex.
\end{lemma}

\begin{proof}
Consider $P = A[B] \to B$ with kernel $I$. The naive cotangent
complex $\NL_{B/A}$ of $B$ over $A$ is the complex
$I/I^2 \to \Omega_{P/A} \otimes_P B$,
see Algebra, Definition \ref{algebra-definition-naive-cotangent-complex}.
Observe that in (\ref{equation-comparison-map}) we have already
constructed a canonical map
$$
c : \NL_{B/A} \longrightarrow \tau_{\geq -1}L_{B/A}
$$
Consider the distinguished triangle (\ref{equation-triangle})
$$
L_{P/A} \otimes_P^\mathbf{L} B \to L_{B/A} \to L_{B/P} \to 
(L_{P/A} \otimes_P^\mathbf{L} B)[1]
$$
associated to the ring maps $A \to A[B] \to B$. We know that
$L_{P/A} = \Omega_{P/A}[0] = \NL_{P/A}$ in $D(P)$
(Lemma \ref{lemma-cotangent-complex-polynomial-algebra}
and
Algebra, Lemma \ref{algebra-lemma-NL-polynomial-algebra})
and that
$\tau_{\geq -1}L_{B/P} = I/I^2[1] = \NL_{B/P}$ in $D(B)$
(Lemma \ref{lemma-surjection} and
Algebra, Lemma \ref{algebra-lemma-NL-surjection}).
To show $c$ is a quasi-isomorphism it suffices by
Algebra, Lemma \ref{algebra-lemma-exact-sequence-NL}
and the long exact cohomology sequence associated to the
distinguished triangle
to show that the maps $L_{P/A} \to L_{B/A} \to L_{B/P}$ are compatible
on cohomology groups with the corresponding maps
$\NL_{P/A} \to \NL_{B/A} \to \NL_{B/P}$
of the naive cotangent complex. We omit the verification.
\end{proof}

\begin{remark}
\label{remark-explicit-comparison-map}
We can make the comparison map of
Lemma \ref{lemma-relation-with-naive-cotangent-complex}
explicit in the following way.
Let $P_\bullet$ be the standard resolution of $B$
over $A$.
Let $I = \Ker(A[B] \to B)$.
Recall that $P_0 = A[B]$. The map of the
lemma is given by the commutative diagram
$$
\xymatrix{
L_{B/A} \ar[d] & \ldots \ar[r] &
\Omega_{P_2/A} \otimes_{P_2} B
\ar[r] \ar[d] &
\Omega_{P_1/A} \otimes_{P_1} B
\ar[r] \ar[d] &
\Omega_{P_0/A} \otimes_{P_0} B
\ar[d] \\
\NL_{B/A} & \ldots \ar[r] &
0 \ar[r] & 
I/I^2 \ar[r] &
\Omega_{P_0/A} \otimes_{P_0} B
}
$$
We construct the downward arrow with target $I/I^2$
by sending $\text{d}f \otimes b$ to the class of
$(d_0(f) - d_1(f))b$ in $I/I^2$. Here $d_i : P_1 \to P_0$,
$i = 0, 1$ are the two face maps of the simplicial structure.
This makes sense as $d_0 - d_1$ maps $P_1$ into $I = \Ker(P_0 \to B)$.
We omit the verification that this rule is well defined.
Our map is compatible with the differential
$\Omega_{P_1/A} \otimes_{P_1} B \to \Omega_{P_0/A} \otimes_{P_0} B$
as this differential maps $\text{d}f \otimes b$ to
$\text{d}(d_0(f) - d_1(f)) \otimes b$. Moreover, the differential
$\Omega_{P_2/A} \otimes_{P_2} B \to \Omega_{P_1/A} \otimes_{P_1} B$
maps $\text{d}f \otimes b$ to $\text{d}(d_0(f) - d_1(f) + d_2(f)) \otimes b$
which are annihilated by our downward arrow. Hence a map of complexes.
We omit the verification that this is the same as the map of
Lemma \ref{lemma-relation-with-naive-cotangent-complex}.
\end{remark}

\begin{remark}
\label{remark-surjection}
Adopt notation as in Remark \ref{remark-make-map}. The arguments given
there show that the differential
$$
H_2(\mathcal{C}_{B/A}, \mathcal{J}/\mathcal{J}^2)
\longrightarrow
H_0(\mathcal{C}_{B/A}, \text{Tor}_1^\mathcal{O}(\mathcal{J}, \underline{B}))
$$
of the spectral sequence is an isomorphism. Let $\mathcal{C}'_{B/A}$
denote the full subcategory of $\mathcal{C}_{B/A}$ consisting of surjective
maps $P \to B$. The agreement of the cotangent complex with the naive
cotangent complex (Lemma \ref{lemma-relation-with-naive-cotangent-complex})
shows that we have an exact sequence of sheaves
$$
0 \to \underline{H_1(L_{B/A})} \to
\mathcal{J}/\mathcal{J}^2 \xrightarrow{\text{d}} \Omega \to
\underline{H_2(L_{B/A})} \to 0
$$
on $\mathcal{C}'_{B/A}$. It follows that $\Ker(d)$ and
$\Coker(d)$ on the whole category $\mathcal{C}_{B/A}$ have
vanishing higher homology groups, since
these are computed by the homology groups of constant simplicial abelian
groups by Lemma \ref{lemma-identify-pi-shriek}. Hence we conclude
that
$$
H_n(\mathcal{C}_{B/A}, \mathcal{J}/\mathcal{J}^2) \to H_n(L_{B/A})
$$
is an isomorphism for all $n \geq 2$. Combined with the remark above
we obtain the formula
$H_2(L_{B/A}) =
H_0(\mathcal{C}_{B/A}, \text{Tor}_1^\mathcal{O}(\mathcal{J}, \underline{B}))$.
\end{remark}





\section{A spectral sequence of Quillen}
\label{section-spectral-sequence}

\noindent
In this section we discuss a spectral sequence relating derived
tensor product to the cotangent complex.

\begin{lemma}
\label{lemma-vanishing-symmetric-powers}
Notation and assumptions as in
Cohomology on Sites, Example \ref{sites-cohomology-example-category-to-point}.
Assume $\mathcal{C}$ has a cosimplicial object as in
Cohomology on Sites, Lemma
\ref{sites-cohomology-lemma-compute-by-cosimplicial-resolution}.
Let $\mathcal{F}$ be a flat $\underline{B}$-module such that
$H_0(\mathcal{C}, \mathcal{F}) = 0$.
Then $H_l(\mathcal{C}, \text{Sym}_{\underline{B}}^k(\mathcal{F})) = 0$
for $l < k$.
\end{lemma}

\begin{proof}
We drop the subscript ${}_{\underline{B}}$ from tensor products, wedge powers,
and symmetric powers. We will prove the lemma by induction on $k$.
The cases $k = 0, 1$ follow from the assumptions. If $k > 1$ consider
the exact complex
$$
\ldots \to
\wedge^2\mathcal{F} \otimes \text{Sym}^{k - 2}\mathcal{F} \to
\mathcal{F} \otimes \text{Sym}^{k - 1}\mathcal{F} \to
\text{Sym}^k\mathcal{F} \to 0
$$
with differentials as in the Koszul complex. If we think of this as a
resolution of $\text{Sym}^k\mathcal{F}$, then this gives a first quadrant
spectral sequence
$$
E_1^{p, q} =
H_p(\mathcal{C},
\wedge^{q + 1}\mathcal{F} \otimes \text{Sym}^{k - q - 1}\mathcal{F})
\Rightarrow
H_{p + q}(\mathcal{C}, \text{Sym}^k(\mathcal{F}))
$$
By Cohomology on Sites, Lemma \ref{sites-cohomology-lemma-eilenberg-zilber}
we have
$$
L\pi_!(\wedge^{q + 1}\mathcal{F} \otimes \text{Sym}^{k - q - 1}\mathcal{F}) =
L\pi_!(\wedge^{q + 1}\mathcal{F}) \otimes_B^\mathbf{L}
L\pi_!(\text{Sym}^{k - q - 1}\mathcal{F}))
$$
It follows (from the construction of derived tensor products) that
the induction hypothesis combined with the vanishing
of $H_0(\mathcal{C}, \wedge^{q + 1}(\mathcal{F})) = 0$ will prove what we want.
This is true because $\wedge^{q + 1}(\mathcal{F})$ is a quotient
of $\mathcal{F}^{\otimes q + 1}$ and
$H_0(\mathcal{C}, \mathcal{F}^{\otimes q + 1})$
is a quotient of $H_0(\mathcal{C}, \mathcal{F})^{\otimes q + 1}$
which is zero.
\end{proof}

\begin{remark}
\label{remark-first-homology-symmetric-power}
In the situation of Lemma \ref{lemma-vanishing-symmetric-powers}
one can show that
$H_k(\mathcal{C}, \text{Sym}^k(\mathcal{F})) =
\wedge^k_B(H_1(\mathcal{C}, \mathcal{F}))$.
Namely, it can be deduced from the proof that
$H_k(\mathcal{C}, \text{Sym}^k(\mathcal{F}))$ is the $S_k$-coinvariants
of
$$
H^{-k}(L\pi_!(\mathcal{F}) \otimes_B^\mathbf{L}
L\pi_!(\mathcal{F}) \otimes_B^\mathbf{L}
\ldots \otimes_B^\mathbf{L} L\pi_!(\mathcal{F})) =
H_1(\mathcal{C}, \mathcal{F})^{\otimes k}
$$
Thus our claim is that this action is given by the usual action
of $S_k$ on the tensor product multiplied by the sign character.
To prove this one has to work through the sign conventions
in the definition of the total complex associated to a
multi-complex. We omit the verification.
\end{remark}

\begin{lemma}
\label{lemma-map-tors-zero}
Let $A$ be a ring. Let $P = A[E]$ be a polynomial ring.
Set $I = (e; e \in E) \subset P$. The maps
$\text{Tor}_i^P(A, I^{n + 1}) \to \text{Tor}_i^P(A, I^n)$
are zero for all $i$ and $n$.
\end{lemma}

\begin{proof}
Denote $x_e \in P$ the variable corresponding to $e \in E$.
A free resolution of $A$ over $P$ is given by the Koszul complex
$K_\bullet$ on the $x_e$. Here $K_i$ has basis given by wedges
$e_1 \wedge \ldots \wedge e_i$, $e_1, \ldots, e_i \in E$ and $d(e) = x_e$.
Thus $K_\bullet \otimes_P I^n = I^nK_\bullet$ computes
$\text{Tor}_i^P(A, I^n)$. Observe that everything is graded
with $\deg(x_e) = 1$, $\deg(e) = 1$, and $\deg(a) = 0$ for $a \in A$.
Suppose $\xi \in I^{n + 1}K_i$ is a cocycle homogeneous of degree $m$.
Note that $m \geq i + 1 + n$. Then $\xi = \text{d}\eta$ for some
$\eta \in K_{i + 1}$ as $K_\bullet$ is exact in degrees $ > 0$.
(The case $i = 0$ is left to the reader.)
Now $\deg(\eta) = m \geq i + 1 + n$. Hence writing $\eta$
in terms of the basis we see the coordinates are in $I^n$.
Thus $\xi$ maps to zero in the homology of $I^nK_\bullet$ as desired.
\end{proof}

\begin{theorem}[Quillen spectral sequence]
\label{theorem-quillen-spectral-sequence}
Let $A \to B$ be a surjective ring map. Consider the sheaf
$\Omega = \Omega_{\mathcal{O}/A} \otimes_\mathcal{O} \underline{B}$ of
$\underline{B}$-modules on $\mathcal{C}_{B/A}$, see
Section \ref{section-compute-L-pi-shriek}.
Then there is a spectral sequence with $E_1$-page
$$
E_1^{p, q} =
H_{- p - q}(\mathcal{C}_{B/A}, \text{Sym}^p_{\underline{B}}(\Omega))
\Rightarrow \text{Tor}^A_{- p - q}(B, B)
$$
with $d_r$ of bidegree $(r, -r + 1)$.
Moreover, $H_i(\mathcal{C}_{B/A}, \text{Sym}^k_{\underline{B}}(\Omega)) = 0$
for $i < k$.
\end{theorem}

\begin{proof}
Let $I \subset A$ be the kernel of $A \to B$. Let
$\mathcal{J} \subset \mathcal{O}$
be the kernel of $\mathcal{O} \to \underline{B}$. Then
$I\mathcal{O} \subset \mathcal{J}$. Set
$\mathcal{K} = \mathcal{J}/I\mathcal{O}$ and
$\overline{\mathcal{O}} = \mathcal{O}/I\mathcal{O}$.

\medskip\noindent
For every object $U = (P \to B)$ of $\mathcal{C}_{B/A}$
we can choose an isomorphism $P = A[E]$ such that the map
$P \to B$ maps each $e \in E$ to zero. Then
$J = \mathcal{J}(U) \subset P = \mathcal{O}(U)$
is equal to $J = IP + (e; e \in E)$. Moreover
$\overline{\mathcal{O}}(U) = B[E]$ and $K = \mathcal{K}(U) = (e; e \in E)$
is the ideal generated by the variables in the polynomial ring $B[E]$.
In particular it is clear that
$$
K/K^2 \xrightarrow{\text{d}} \Omega_{P/A} \otimes_P B
$$
is a bijection. In other words, $\Omega = \mathcal{K}/\mathcal{K}^2$
and $\text{Sym}_B^k(\Omega) = \mathcal{K}^k/\mathcal{K}^{k + 1}$.
Note that $\pi_!(\Omega) = \Omega_{B/A} = 0$ (Lemma \ref{lemma-identify-H0})
as $A \to B$ is surjective
(Algebra, Lemma \ref{algebra-lemma-trivial-differential-surjective}).
By Lemma \ref{lemma-vanishing-symmetric-powers} we conclude that
$$
H_i(\mathcal{C}_{B/A}, \mathcal{K}^k/\mathcal{K}^{k + 1}) =
H_i(\mathcal{C}_{B/A}, \text{Sym}^k_{\underline{B}}(\Omega)) = 0
$$
for $i < k$. This proves the final statement of the theorem.

\medskip\noindent
The approach to the theorem is to note that
$$
B \otimes_A^\mathbf{L} B = L\pi_!(\mathcal{O}) \otimes_A^\mathbf{L} B =
L\pi_!(\mathcal{O} \otimes_{\underline{A}}^\mathbf{L} \underline{B}) =
L\pi_!(\overline{\mathcal{O}})
$$
The first equality by Lemma \ref{lemma-apply-O-B-comparison},
the second equality by
Cohomology on Sites, Lemma \ref{sites-cohomology-lemma-change-of-rings}, and
the third equality as $\mathcal{O}$ is flat over $\underline{A}$.
The sheaf $\overline{\mathcal{O}}$ has a filtration
$$
\ldots \subset
\mathcal{K}^3 \subset
\mathcal{K}^2 \subset
\mathcal{K} \subset
\overline{\mathcal{O}}
$$
This induces a filtration $F$ on a complex $C$ representing
$L\pi_!(\overline{\mathcal{O}})$ with $F^pC$ representing
$L\pi_!(\mathcal{K}^p)$ (construction of $C$ and $F$ omitted).
Consider the spectral sequence of
Homology, Section \ref{homology-section-filtered-complex}
associated to $(C, F)$. It has $E_1$-page
$$
E_1^{p, q} = H_{- p - q}(\mathcal{C}_{B/A}, \mathcal{K}^p/\mathcal{K}^{p + 1})
\quad\Rightarrow\quad
H_{- p - q}(\mathcal{C}_{B/A}, \overline{\mathcal{O}}) = 
\text{Tor}_{- p - q}^A(B, B)
$$
and differentials $E_r^{p, q} \to E_r^{p + r, q - r + 1}$. To show convergence
we will show that for every $k$ there exists a $c$ such that
$H_i(\mathcal{C}_{B/A}, \mathcal{K}^n) = 0$
for $i < k$ and $n > c$\footnote{A posteriori
the ``correct'' vanishing $H_i(\mathcal{C}_{B/A}, \mathcal{K}^n) = 0$ for
$i < n$ can be concluded.}.

\medskip\noindent
Given $k \geq 0$ set $c = k^2$. We claim that
$$
H_i(\mathcal{C}_{B/A}, \mathcal{K}^{n + c}) \to
H_i(\mathcal{C}_{B/A}, \mathcal{K}^n)
$$
is zero for $i < k$ and all $n \geq 0$. Note that
$\mathcal{K}^n/\mathcal{K}^{n + c}$ has a finite filtration whose successive
quotients $\mathcal{K}^m/\mathcal{K}^{m + 1}$, $n \leq m < n + c$
have $H_i(\mathcal{C}_{B/A}, \mathcal{K}^m/\mathcal{K}^{m + 1}) = 0$
for $i < n$ (see above). Hence the claim implies
$H_i(\mathcal{C}_{B/A}, \mathcal{K}^{n + c}) = 0$ for $i < k$ and all
$n \geq k$ which is what we need to show.

\medskip\noindent
Proof of the claim. Recall that for any $\mathcal{O}$-module $\mathcal{F}$
the map $\mathcal{F} \to \mathcal{F} \otimes_\mathcal{O}^\mathbf{L} B$
induces an isomorphism on applying $L\pi_!$, see
Lemma \ref{lemma-O-homology-B-homology}.
Consider the map
$$
\mathcal{K}^{n + k} \otimes_\mathcal{O}^\mathbf{L} B
\longrightarrow
\mathcal{K}^n \otimes_\mathcal{O}^\mathbf{L} B
$$
We claim that this map induces the zero map on cohomology sheaves
in degrees $0, -1, \ldots, - k + 1$. If this second claim holds, then
the $k$-fold composition
$$
\mathcal{K}^{n + c} \otimes_\mathcal{O}^\mathbf{L} B
\longrightarrow
\mathcal{K}^n \otimes_\mathcal{O}^\mathbf{L} B
$$
factors through $\tau_{\leq -k}\mathcal{K}^n \otimes_\mathcal{O}^\mathbf{L} B$
hence induces zero on $H_i(\mathcal{C}_{B/A}, -) = L_i\pi_!( - )$
for $i < k$, see
Derived Categories, Lemma \ref{derived-lemma-trick-vanishing-composition}.
By the remark above this means the same thing is true for
$H_i(\mathcal{C}_{B/A}, \mathcal{K}^{n + c}) \to
H_i(\mathcal{C}_{B/A}, \mathcal{K}^n)$
which proves the (first) claim.

\medskip\noindent
Proof of the second claim. The statement is local, hence we may work
over an object $U = (P \to B)$ as above. We have to show
the maps
$$
\text{Tor}_i^P(B, K^{n + k}) \to \text{Tor}_i^P(B, K^n)
$$
are zero for $i < k$. There is a spectral sequence
$$
\text{Tor}_a^P(P/IP, \text{Tor}_b^{P/IP}(B, K^n))
\Rightarrow
\text{Tor}_{a + b}^P(B, K^n),
$$
see More on Algebra, Example \ref{more-algebra-example-tor-change-rings}.
Thus it suffices to prove the maps
$$
\text{Tor}_i^{P/IP}(B, K^{n + 1}) \to \text{Tor}_i^{P/IP}(B, K^n)
$$
are zero for all $i$. This is Lemma \ref{lemma-map-tors-zero}.
\end{proof}

\begin{remark}
\label{remark-elucidate-ss}
In the situation of Theorem \ref{theorem-quillen-spectral-sequence}
let $I = \Ker(A \to B)$. Then
$H^{-1}(L_{B/A}) = H_1(\mathcal{C}_{B/A}, \Omega) = I/I^2$, see
Lemma \ref{lemma-surjection}.
Hence $H_k(\mathcal{C}_{B/A}, \text{Sym}^k(\Omega)) = \wedge^k_B(I/I^2)$ by
Remark \ref{remark-first-homology-symmetric-power}. Thus the
$E_1$-page looks like
$$
\begin{matrix}
B \\
0 \\
0 & I/I^2 \\
0 & H^{-2}(L_{B/A}) \\
0 & H^{-3}(L_{B/A}) & \wedge^2(I/I^2) \\
0 & H^{-4}(L_{B/A}) & H_3(\mathcal{C}_{B/A}, \text{Sym}^2(\Omega)) \\
0 & H^{-5}(L_{B/A}) & H_4(\mathcal{C}_{B/A}, \text{Sym}^2(\Omega)) &
\wedge^3(I/I^2)
\end{matrix}
$$
with horizontal differential. Thus we obtain edge maps
$\text{Tor}_i^A(B, B) \to H^{-i}(L_{B/A})$, $i > 0$ and
$\wedge^i_B(I/I^2) \to \text{Tor}_i^A(B, B)$. Finally, we have
$\text{Tor}_1^A(B, B) = I/I^2$ and there is a
five term exact sequence
$$
\text{Tor}_3^A(B, B) \to H^{-3}(L_{B/A}) \to \wedge^2_B(I/I^2) \to
\text{Tor}_2^A(B, B) \to H^{-2}(L_{B/A}) \to 0
$$
of low degree terms.
\end{remark}

\begin{remark}
\label{remark-elucidate-degree-two}
Let $A \to B$ be a ring map. Let $P_\bullet$ be a resolution of
$B$ over $A$ (Remark \ref{remark-resolution}).
Set $J_n = \Ker(P_n \to B)$. Note that
$$
\text{Tor}_2^{P_n}(B, B) = 
\text{Tor}_1^{P_n}(J_n, B) =
\Ker(J_n \otimes_{P_n} J_n \to J_n^2).
$$
Hence $H_2(L_{B/A})$ is canonically equal to
$$
\Coker(\text{Tor}_2^{P_1}(B, B) \to \text{Tor}_2^{P_0}(B, B))
$$
by Remark \ref{remark-surjection}. To make this more explicit we choose
$P_2$, $P_1$, $P_0$ as in Example \ref{example-resolution-length-two}.
We claim that
$$
\text{Tor}_2^{P_1}(B, B) =
\wedge^2(\bigoplus\nolimits_{t \in T} B)\ \oplus
\ \bigoplus\nolimits_{t \in T} J_0\ \oplus
\ \text{Tor}_2^{P_0}(B, B)
$$
Namely, the basis elements $x_t \wedge x_{t'}$ of the first summand
corresponds to the element $x_t \otimes x_{t'} - x_{t'} \otimes x_t$
of $J_1 \otimes_{P_1} J_1$. For $f \in J_0$ the element $x_t \otimes f$
of the second summand corresponds to the element
$x_t \otimes s_0(f) - s_0(f) \otimes x_t$ of $J_1 \otimes_{P_1} J_1$.
Finally, the map $\text{Tor}_2^{P_0}(B, B) \to \text{Tor}_2^{P_1}(B, B)$
is given by $s_0$. The map
$d_0 - d_1 : \text{Tor}_2^{P_1}(B, B) \to \text{Tor}_2^{P_0}(B, B)$
is zero on the last summand, maps $x_t \otimes f$ to
$f \otimes f_t - f_t \otimes f$, and maps $x_t \wedge x_{t'}$
to $f_t \otimes f_{t'} - f_{t'} \otimes f_t$. All in all we conclude
that there is an exact sequence
$$
\wedge^2_B(J_0/J_0^2) \to \text{Tor}_2^{P_0}(B, B) \to H^{-2}(L_{B/A}) \to 0
$$
In this way we obtain a direct proof of a consequence of Quillen's spectral
sequence discussed in Remark \ref{remark-elucidate-ss}.
\end{remark}






\section{Comparison with Lichtenbaum-Schlessinger}
\label{section-compare-higher}

\noindent
Let $A \to B$ be a ring map. In \cite{Lichtenbaum-Schlessinger}
there is a fairly explicit determination of $\tau_{\geq -2}L_{B/A}$
which is often used in calculations of versal deformation spaces of
singularities. The construction follows.
Choose a polynomial algebra $P$ over $A$
and a surjection $P \to B$ with kernel $I$. Choose generators
$f_t$, $t \in T$ for $I$ which induces a surjection
$F = \bigoplus_{t \in T} P \to I$ with $F$ a free $P$-module.
Let $Rel \subset F$ be the kernel of $F \to I$, in other words
$Rel$ is the set of relations among the $f_t$. Let $TrivRel \subset Rel$
be the submodule of trivial relations, i.e., the submodule of $Rel$
generated by the elements $(\ldots, f_{t'}, 0, \ldots, 0, -f_t, 0, \ldots)$.
Consider the complex of $B$-modules
\begin{equation}
\label{equation-lichtenbaum-schlessinger}
Rel/TrivRel \longrightarrow
F \otimes_P B \longrightarrow
\Omega_{P/A} \otimes_P B
\end{equation}
where the last term is placed in degree $0$. The first map is the obvious
one and the second map sends the basis element corresponding to $t \in T$
to $\text{d}f_t \otimes 1$.

\begin{definition}
\label{definition-biderivation}
Let $A \to B$ be a ring map. Let $M$ be a $(B, B)$-bimodule
over $A$. An {\it $A$-biderivation} is an $A$-linear map $\lambda : B \to M$
such that $\lambda(xy) = x\lambda(y) + \lambda(x)y$.
\end{definition}

\noindent
For a polynomial algebra the biderivations are easy to describe.

\begin{lemma}
\label{lemma-polynomial-ring-unique}
Let $P = A[S]$ be a polynomial ring over $A$. Let $M$ be a $(P, P)$-bimodule
over $A$. Given $m_s \in M$ for $s \in S$, there exists a unique
$A$-biderivation $\lambda : P \to M$ mapping $s$ to $m_s$ for $s \in S$.
\end{lemma}

\begin{proof}
We set
$$
\lambda(s_1 \ldots s_t) =
\sum s_1 \ldots s_{i - 1} m_{s_i} s_{i + 1} \ldots s_t
$$
in $M$. Extending by $A$-linearity we obtain a biderivation.
\end{proof}

\noindent
Here is the comparison statement. The reader may also read about this
in \cite[page 206, Proposition 12]{Andre-Homologie} or in the paper
\cite{Doncel} which extends the complex
(\ref{equation-lichtenbaum-schlessinger}) by one term and the comparison
to $\tau_{\geq -3}$.

\begin{lemma}
\label{lemma-compare-higher}
In the situation above denote $L$ the complex
(\ref{equation-lichtenbaum-schlessinger}).
There is a canonical map $L_{B/A} \to L$ in $D(B)$ which
induces an isomorphism $\tau_{\geq -2}L_{B/A} \to L$ in $D(B)$.
\end{lemma}

\begin{proof}
Let $P_\bullet \to B$ be a resolution of $B$ over $A$
(Remark \ref{remark-resolution}). We will identify $L_{B/A}$ with
$\Omega_{P_\bullet/A} \otimes B$. To construct the map we
make some choices.

\medskip\noindent
Choose an $A$-algebra map $\psi : P_0 \to P$ compatible with the
given maps $P_0 \to B$ and $P \to B$.

\medskip\noindent
Write $P_1 = A[S]$ for some set $S$. For $s \in S$ we may write
$$
\psi(d_0(s) - d_1(s)) = \sum p_{s, t} f_t
$$
for some $p_{s, t} \in P$. Think of $F = \bigoplus_{t \in T} P$
as a $(P_1, P_1)$-bimodule via the maps $(\psi \circ d_0, \psi \circ d_1)$.
By Lemma \ref{lemma-polynomial-ring-unique} we obtain a unique
$A$-biderivation $\lambda : P_1 \to F$ mapping $s$ to the vector with
coordinates $p_{s, t}$. By construction the composition
$$
P_1 \longrightarrow F \longrightarrow P
$$
sends $f \in P_1$ to $\psi(d_0(f) - d_1(f))$ because the map
$f \mapsto \psi(d_0(f) - d_1(f))$ is an $A$-biderivation agreeing with
the composition on generators.

\medskip\noindent
For $g \in P_2$ we claim that $\lambda(d_0(g) - d_1(g) + d_2(g))$
is an element of $Rel$. Namely, by the last remark of the previous
paragraph the image of $\lambda(d_0(g) - d_1(g) + d_2(g))$ in $P$ is
$$
\psi((d_0 - d_1)(d_0(g) - d_1(g) + d_2(g)))
$$
which is zero by Simplicial, Section \ref{simplicial-section-complexes}).

\medskip\noindent
The choice of $\psi$ determines a map
$$
\text{d}\psi \otimes 1 :
\Omega_{P_0/A} \otimes B
\longrightarrow
\Omega_{P/A} \otimes B
$$
Composing $\lambda$ with the map $F \to F \otimes B$ gives a
usual $A$-derivation as the two $P_1$-module structures on
$F \otimes B$ agree. Thus $\lambda$ determines a map
$$
\overline{\lambda} :
\Omega_{P_1/A} \otimes B
\longrightarrow
F \otimes B
$$
Finally, We obtain a $B$-linear map
$$
q :
\Omega_{P_2/A} \otimes B
\longrightarrow
Rel/TrivRel
$$
by mapping $\text{d}g$ to the class of $\lambda(d_0(g) - d_1(g) + d_2(g))$
in the quotient.

\medskip\noindent
The diagram
$$
\xymatrix{
\Omega_{P_3/A} \otimes B \ar[r] \ar[d] &
\Omega_{P_2/A} \otimes B \ar[r] \ar[d]_q &
\Omega_{P_1/A} \otimes B \ar[r] \ar[d]_{\overline{\lambda}} &
\Omega_{P_0/A} \otimes B \ar[d]_{\text{d}\psi \otimes 1} \\
0 \ar[r] &
Rel/TrivRel \ar[r] &
F \otimes B \ar[r] &
\Omega_{P/A} \otimes B
}
$$
commutes (calculation omitted) and we obtain the map of the lemma.
By Remark \ref{remark-explicit-comparison-map} and
Lemma \ref{lemma-relation-with-naive-cotangent-complex} we see that this map
induces isomorphisms $H_1(L_{B/A}) \to H_1(L)$ and $H_0(L_{B/A}) \to H_0(L)$.

\medskip\noindent
It remains to see that our map $L_{B/A} \to L$ induces an isomorphism
$H_2(L_{B/A}) \to H_2(L)$. Choose a resolution of $B$ over $A$ with
$P_0 = P = A[u_i]$ and then $P_1$ and $P_2$ as in
Example \ref{example-resolution-length-two}.
In Remark \ref{remark-elucidate-degree-two} we have constructed an exact
sequence
$$
\wedge^2_B(J_0/J_0^2) \to \text{Tor}_2^{P_0}(B, B) \to H^{-2}(L_{B/A}) \to 0
$$
where $P_0 = P$ and $J_0 = \Ker(P \to B) = I$.
Calculating the Tor group using the short exact sequences
$0 \to I \to P \to B \to 0$ and $0 \to Rel \to F \to I \to 0$
we find that
$\text{Tor}_2^P(B, B) = \Ker(Rel \otimes B \to F \otimes B)$.
The image of the map $\wedge^2_B(I/I^2) \to \text{Tor}_2^P(B, B)$
under this identification is exactly the image of $TrivRel \otimes B$.
Thus we see that $H_2(L_{B/A}) \cong H_2(L)$.

\medskip\noindent
Finally, we have to check that our map $L_{B/A} \to L$ actually induces
this isomorphism. We will use the notation and results discussed in
Example \ref{example-resolution-length-two} and
Remarks \ref{remark-elucidate-degree-two} and \ref{remark-surjection}
without further mention. Pick an element $\xi$ of
$\text{Tor}_2^{P_0}(B, B) = \Ker(I \otimes_P I \to I^2)$.
Write $\xi = \sum h_{t', t}f_{t'} \otimes f_t$ for some
$h_{t', t} \in P$. Tracing through the exact sequences above we
find that $\xi$ corresponds to the image in $Rel \otimes B$
of the element $r \in Rel \subset F = \bigoplus_{t \in T} P$ with
$t$th coordinate $r_t = \sum_{t' \in T} h_{t', t}f_{t'}$.
On the other hand, $\xi$ corresponds to the element of
$H_2(L_{B/A}) = H_2(\Omega)$ which is the image
via $\text{d} : H_2(\mathcal{J}/\mathcal{J}^2) \to H_2(\Omega)$
of the boundary of $\xi$ under the $2$-extension
$$
0 \to
\text{Tor}_2^\mathcal{O}(\underline{B}, \underline{B})
\to 
\mathcal{J} \otimes_\mathcal{O} \mathcal{J} \to \mathcal{J}
\to
\mathcal{J}/\mathcal{J}^2 \to 0
$$
We compute the successive transgressions of our element. First we have
$$
\xi = (d_0 - d_1)(- \sum s_0(h_{t', t} f_{t'}) \otimes x_t)
$$
and next we have
$$
\sum s_0(h_{t', t} f_{t'}) x_t = d_0(v_r) - d_1(v_r) + d_2(v_r)
$$
by our choice of the variables $v$ in
Example \ref{example-resolution-length-two}.
We may choose our map $\lambda$ above such that
$\lambda(u_i) = 0$ and $\lambda(x_t) = - e_t$ where $e_t \in F$
denotes the basis vector corresponding to $t \in T$.
Hence the construction of our map $q$ above sends $\text{d}v_r$ to
$$
\lambda(\sum s_0(h_{t', t} f_{t'}) x_t) =
\sum\nolimits_t \left(\sum\nolimits_{t'} h_{t', t}f_{t'}\right) e_t
$$
matching the image of $\xi$ in $Rel \otimes B$ (the two minus signs
we found above cancel out). This agreement finishes the proof.
\end{proof}

\begin{remark}[Functoriality of the Lichtenbaum-Schlessinger complex]
\label{remark-functoriality-lichtenbaum-schlessinger}
Consider a commutative square
$$
\xymatrix{
A' \ar[r] & B' \\
A \ar[u] \ar[r] & B \ar[u]
}
$$
of ring maps. Choose a factorization
$$
\xymatrix{
A' \ar[r] & P' \ar[r] & B' \\
A \ar[u] \ar[r] & P \ar[u] \ar[r] & B \ar[u]
}
$$
with $P$ a polynomial algebra over $A$ and $P'$ a polynomial algebra over $A'$.
Choose generators $f_t$, $t \in T$ for $\Ker(P \to B)$.
For $t \in T$ denote $f'_t$ the image of $f_t$ in $P'$.
Choose $f'_s \in P'$ such that the elements $f'_t$ for
$t \in T' = T \amalg S$ generate the kernel
of $P' \to B'$. Set $F = \bigoplus_{t \in T} P$ and
$F' = \bigoplus_{t' \in T'} P'$. Let $Rel = \Ker(F \to P)$
and $Rel' = \Ker(F' \to P')$ where the maps are given
by multiplication by $f_t$, resp.\ $f'_t$ on the coordinates.
Finally, set $TrivRel$, resp.\ $TrivRel'$ equal to the submodule
of $Rel$, resp.\ $TrivRel$ generated by the elements
$(\ldots, f_{t'}, 0, \ldots, 0, -f_t, 0, \ldots)$
for $t, t' \in T$, resp.\ $T'$. Having made these choices we obtain a
canonical commutative diagram
$$
\xymatrix{
L' : &
Rel'/TrivRel' \ar[r] &
F' \otimes_{P'} B' \ar[r] &
\Omega_{P'/A'} \otimes_{P'} B' \\
L : \ar[u] &
Rel/TrivRel \ar[r] \ar[u] &
F \otimes_P B \ar[r] \ar[u] &
\Omega_{P/A} \otimes_P B \ar[u]
}
$$
Moreover, tracing through the choices made in the proof of
Lemma \ref{lemma-compare-higher}
the reader sees that one obtains a commutative diagram
$$
\xymatrix{
L_{B'/A'} \ar[r] & L' \\
L_{B/A} \ar[r] \ar[u] & L \ar[u]
}
$$
\end{remark}





\section{The cotangent complex of a local complete intersection}
\label{section-lci}

\noindent
If $A \to B$ is a local complete intersection map, then
$L_{B/A}$ is a perfect complex. The key to proving this is
the following lemma.

\begin{lemma}
\label{lemma-special-case}
Let $A = \mathbf{Z}[x_1, \ldots, x_n] \to B = \mathbf{Z}$
be the ring map which sends $x_i$ to $0$ for $i = 1, \ldots, n$.
Let $I = (x_1, \ldots, x_n) \subset A$. Then $L_{B/A}$ is quasi-isomorphic to
$I/I^2[1]$.
\end{lemma}

\begin{proof}
There are several ways to prove this. For example one can explicitly construct
a resolution of $B$ over $A$ and compute. We will use (\ref{equation-triangle}).
Namely, consider the distinguished triangle
$$
L_{\mathbf{Z}[x_1, \ldots, x_n]/\mathbf{Z}}
\otimes_{\mathbf{Z}[x_1, \ldots, x_n]} \mathbf{Z} \to
L_{\mathbf{Z}/\mathbf{Z}} \to
L_{\mathbf{Z}/\mathbf{Z}[x_1, \ldots, x_n]}\to
L_{\mathbf{Z}[x_1, \ldots, x_n]/\mathbf{Z}}
\otimes_{\mathbf{Z}[x_1, \ldots, x_n]} \mathbf{Z}[1]
$$
The complex $L_{\mathbf{Z}[x_1, \ldots, x_n]/\mathbf{Z}}$
is quasi-isomorphic to $\Omega_{\mathbf{Z}[x_1, \ldots, x_n]/\mathbf{Z}}$ by
Lemma \ref{lemma-cotangent-complex-polynomial-algebra}.
The complex $L_{\mathbf{Z}/\mathbf{Z}}$ is zero in $D(\mathbf{Z})$ by
Lemma \ref{lemma-when-zero}.
Thus we see that $L_{B/A}$ has only one nonzero cohomology group
which is as described in the lemma by Lemma \ref{lemma-surjection}.
\end{proof}

\begin{lemma}
\label{lemma-mod-regular-sequence}
Let $A \to B$ be a surjective ring map whose kernel $I$ is generated
by a Koszul-regular sequence (for example a regular sequence).
Then $L_{B/A}$ is quasi-isomorphic to $I/I^2[1]$.
\end{lemma}

\begin{proof}
Let $f_1, \ldots, f_r \in I$ be a Koszul regular sequence generating $I$.
Consider the ring map $\mathbf{Z}[x_1, \ldots, x_r] \to A$ sending
$x_i$ to $f_i$. Since $x_1, \ldots, x_r$ is a regular sequence in
$\mathbf{Z}[x_1, \ldots, x_r]$ we see that the Koszul complex
on $x_1, \ldots, x_r$ is a free resolution of
$\mathbf{Z} = \mathbf{Z}[x_1, \ldots, x_r]/(x_1, \ldots, x_r)$
over $\mathbf{Z}[x_1, \ldots, x_r]$
(see More on Algebra, Lemma \ref{more-algebra-lemma-regular-koszul-regular}).
Thus the assumption that $f_1, \ldots, f_r$ is Koszul regular
exactly means that
$B = A \otimes_{\mathbf{Z}[x_1, \ldots, x_r]}^\mathbf{L} \mathbf{Z}$.
Hence
$L_{B/A} = L_{\mathbf{Z}/\mathbf{Z}[x_1, \ldots, x_r]}
\otimes_\mathbf{Z}^\mathbf{L} B$ by
Lemmas \ref{lemma-flat-base-change-cotangent-complex} and
\ref{lemma-special-case}.
\end{proof}

\begin{lemma}
\label{lemma-mod-Koszul-regular-ideal}
Let $A \to B$ be a surjective ring map whose kernel $I$ is Koszul.
Then $L_{B/A}$ is quasi-isomorphic to $I/I^2[1]$.
\end{lemma}

\begin{proof}
Locally on $\Spec(A)$ the ideal $I$ is generated by a Koszul regular
sequence, see More on Algebra, Definition
\ref{more-algebra-definition-regular-ideal}.
Hence this follows from
Lemma \ref{lemma-flat-base-change-cotangent-complex}.
\end{proof}

\begin{proposition}
\label{proposition-cotangent-complex-local-complete-intersection}
Let $A \to B$ be a local complete intersection map.
Then $L_{B/A}$ is a perfect complex with tor amplitude in $[-1, 0]$.
\end{proposition}

\begin{proof}
Choose a surjection $P = A[x_1, \ldots, x_n] \to B$ with kernel $J$.
By Lemma \ref{lemma-relation-with-naive-cotangent-complex}
we see that $J/J^2 \to \bigoplus B\text{d}x_i$
is quasi-isomorphic to $\tau_{\geq -1}L_{B/A}$.
Note that $J/J^2$ is finite projective
(More on Algebra, Lemma
\ref{more-algebra-lemma-quasi-regular-ideal-finite-projective}),
hence $\tau_{\geq -1}L_{B/A}$ is a perfect complex with
tor amplitude in $[-1, 0]$.
Thus it suffices to show that $H^i(L_{B/A}) = 0$ for $i \not \in [-1, 0]$.
This follows from (\ref{equation-triangle})
$$
L_{P/A} \otimes_P^\mathbf{L} B \to L_{B/A} \to L_{B/P} \to
L_{P/A} \otimes_P^\mathbf{L} B[1]
$$
and Lemma \ref{lemma-mod-Koszul-regular-ideal}
to see that $H^i(L_{B/P})$ is zero unless $i \in \{-1, 0\}$.
(We also use Lemma \ref{lemma-cotangent-complex-polynomial-algebra}
for the term on the left.)
\end{proof}








\section{Tensor products and the cotangent complex}
\label{section-tensor-product}

\noindent
Let $R$ be a ring and let $A$, $B$ be $R$-algebras. In this section we
discuss $L_{A \otimes_R B/R}$. Most of the information we want is contained
in the following diagram
\begin{equation}
\label{equation-tensor-product}
\vcenter{
\xymatrix{
L_{A/R} \otimes_A^\mathbf{L} (A \otimes_R B) \ar[r] &
L_{A \otimes_R B/B} \ar[r] &
E \\
L_{A/R} \otimes_A^\mathbf{L} (A \otimes_R B) \ar[r] \ar@{=}[u] &
L_{A \otimes_R B/R} \ar[r] \ar[u] &
L_{A \otimes_R B/A} \ar[u] \\
 &
L_{B/R} \otimes_B^\mathbf{L} (A \otimes_R B) \ar[u] \ar@{=}[r] &
L_{B/R} \otimes_B^\mathbf{L} (A \otimes_R B) \ar[u]
}
}
\end{equation}
Explanation: The middle row is the fundamental triangle
(\ref{equation-triangle}) for the ring maps $R \to A \to A \otimes_R B$.
The middle column is the fundamental triangle
(\ref{equation-triangle}) for the ring maps $R \to B \to A \otimes_R B$.
Next, $E$ is an object of $D(A \otimes_R B)$ which ``fits'' into the
upper right corner, i.e., which turns both the top row
and the right column into distinguished triangles. Such an $E$
exists by Derived Categories, Proposition \ref{derived-proposition-9}
applied to the lower left square (with $0$ placed in the missing
spot). To be more explicit, we could for example define $E$ as the cone
(Derived Categories, Definition \ref{derived-definition-cone})
of the map of complexes
$$
L_{A/R} \otimes_A^\mathbf{L} (A \otimes_R B) \oplus
L_{B/R} \otimes_B^\mathbf{L} (A \otimes_R B)
\longrightarrow
L_{A \otimes_R B/R}
$$
and get the two maps with target $E$ by an application of TR3.
In the Tor independent case the object $E$ is zero.

\begin{lemma}
\label{lemma-tensor-product-tor-independent}
If $A$ and $B$ are Tor independent $R$-algebras, then the object $E$
in (\ref{equation-tensor-product}) is zero. In this case we have
$$
L_{A \otimes_R B/R} =
L_{A/R} \otimes_A^\mathbf{L} (A \otimes_R B) \oplus
L_{B/R} \otimes_B^\mathbf{L} (A \otimes_R B)
$$
which is represented by the complex
$L_{A/R} \otimes_R B \oplus L_{B/R} \otimes_R A $
of $A \otimes_R B$-modules.
\end{lemma}

\begin{proof}
The first two statements are immediate from
Lemma \ref{lemma-flat-base-change-cotangent-complex}.
The last statement follows as $L_{A/R}$ is a complex
of free $A$-modules, hence $L_{A/R} \otimes_A^\mathbf{L} (A \otimes_R B)$
is represented by
$L_{A/R} \otimes_A (A \otimes_R B) = L_{A/R} \otimes_R B$
\end{proof}

\noindent
In general we can say this about the object $E$.

\begin{lemma}
\label{lemma-tensor-product}
Let $R$ be a ring and let $A$, $B$ be $R$-algebras. The object $E$
in (\ref{equation-tensor-product}) satisfies
$$
H^i(E) =
\left\{
\begin{matrix}
0 & \text{if} & i \geq -1 \\
\text{Tor}_1^R(A, B) & \text{if} & i = -2
\end{matrix}
\right.
$$
\end{lemma}

\begin{proof}
We use the description of $E$ as the cone on
$L_{B/R} \otimes_B^\mathbf{L} (A \otimes_R B) \to L_{A \otimes_R B/A}$.
By Lemma \ref{lemma-compare-higher} the canonical truncations
$\tau_{\geq -2}L_{B/R}$ and $\tau_{\geq -2}L_{A \otimes_R B/A}$
are computed by the Lichtenbaum-Schlessinger complex
(\ref{equation-lichtenbaum-schlessinger}).
These isomorphisms are compatible with functoriality
(Remark \ref{remark-functoriality-lichtenbaum-schlessinger}).
Thus in this proof we work with the Lichtenbaum-Schlessinger complexes.

\medskip\noindent
Choose a polynomial algebra $P$ over $R$ and a surjection $P \to B$.
Choose generators $f_t \in P$, $t \in T$ of the kernel of this surjection.
Let $Rel \subset F = \bigoplus_{t \in T} P$ be the kernel of the map
$F \to P$ which maps the basis vector corresponding to $t$ to $f_t$.
Set $P_A = A \otimes_R P$ and $F_A = A \otimes_R F = P_A \otimes_P F$.
Let $Rel_A$ be the kernel of the map $F_A \to P_A$. Using the exact sequence
$$
0 \to Rel \to F \to P \to B \to 0
$$
and standard short exact sequences for Tor we obtain an exact sequence
$$
A \otimes_R Rel \to Rel_A \to \text{Tor}_1^R(A, B) \to 0
$$
Note that $P_A \to A \otimes_R B$ is a surjection whose kernel is generated
by the elements $1 \otimes f_t$ in $P_A$. Denote $TrivRel_A \subset Rel_A$
the $P_A$-submodule generated by the elements
$(\ldots, 1 \otimes f_{t'}, 0, \ldots,
0, - 1 \otimes f_t \otimes 1, 0, \ldots)$.
Since $TrivRel \otimes_R A \to TrivRel_A$ is surjective, we find a
canonical exact sequence
$$
A \otimes_R (Rel/TrivRel) \to Rel_A/TrivRel_A \to \text{Tor}_1^R(A, B) \to 0
$$
The map of Lichtenbaum-Schlessinger complexes is given by the diagram
$$
\xymatrix{
Rel_A/TrivRel_A \ar[r] &
F_A \otimes_{P_A} (A \otimes_R B) \ar[r] &
\Omega_{P_A/A \otimes_R B} \otimes_{P_A} (A \otimes_R B) \\
Rel/TrivRel \ar[r] \ar[u]_{-2} &
F \otimes_P B \ar[r] \ar[u]_{-1} &
\Omega_{P/A} \otimes_P B \ar[u]_0
}
$$
Note that vertical maps $-1$ and $-0$ induce an isomorphism after applying
the functor $A \otimes_R - = P_A \otimes_P -$ to the source and the vertical
map $-2$ gives exactly the map whose cokernel is the desired Tor module
as we saw above.
\end{proof}












\section{Deformations of ring maps and the cotangent complex}
\label{section-deformations}

\noindent
This section is the continuation of
Deformation Theory, Section \ref{defos-section-deformations}
which we urge the reader to read first.
We start with a surjective ring map $A' \to A$
whose kernel is an ideal $I$ of square zero. Moreover we assume
given a ring map $A \to B$, a $B$-module $N$, and an $A$-module map
$c : I \to N$. In this section we ask ourselves whether we can find
the question mark fitting into the following diagram
\begin{equation}
\label{equation-to-solve}
\vcenter{
\xymatrix{
0 \ar[r] & N \ar[r] & {?} \ar[r] & B \ar[r] & 0 \\
0 \ar[r] & I \ar[u]^c \ar[r] & A' \ar[u] \ar[r] & A \ar[u] \ar[r] & 0
}
}
\end{equation}
and moreover how unique the solution is (if it exists). More precisely,
we look for a surjection of $A'$-algebras $B' \to B$ whose kernel is
an ideal of square zero and is
identified with $N$ such that $A' \to B'$ induces the given map $c$.
We will say $B'$ is a {\it solution} to (\ref{equation-to-solve}).

\begin{lemma}
\label{lemma-find-obstruction}
In the situation above we have
\begin{enumerate}
\item There is a canonical element $\xi \in \Ext^2_B(L_{B/A}, N)$
whose vanishing is a sufficient and necessary condition for the existence
of a solution to (\ref{equation-to-solve}).
\item If there exists a solution, then the set of
isomorphism classes of solutions is principal homogeneous under
$\Ext^1_B(L_{B/A}, N)$.
\item Given a solution $B'$, the set of automorphisms of $B'$
fitting into (\ref{equation-to-solve}) is canonically isomorphic
to $\Ext^0_B(L_{B/A}, N)$.
\end{enumerate}
\end{lemma}

\begin{proof}
Via the identifications $\NL_{B/A} = \tau_{\geq -1}L_{B/A}$
(Lemma \ref{lemma-relation-with-naive-cotangent-complex}) and
$H^0(L_{B/A}) = \Omega_{B/A}$ (Lemma \ref{lemma-identify-H0})
we have seen parts (2) and (3) in
Deformation Theory, Lemmas \ref{defos-lemma-huge-diagram} and
\ref{defos-lemma-choices}.

\medskip\noindent
Proof of (1). Roughly speaking, this follows from the discussion in 
Deformation Theory, Remark \ref{defos-remark-parametrize-solutions}
by replacing the naive cotangent complex by the full cotangent complex.
Here is a more detailed explanation. By
Deformation Theory, Lemma \ref{defos-lemma-parametrize-solutions}
and Remark \ref{defos-remark-parametrize-solutions}
there exists an element
$$
\xi' \in
\Ext^1_A(\NL_{A/A'}, N) =
\Ext^1_B(\NL_{A/A'} \otimes_A^\mathbf{L} B, N) =
\Ext^1_B(L_{A/A'} \otimes_A^\mathbf{L} B, N)
$$
(for the equalities see Deformation Theory, Remark
\ref{defos-remark-parametrize-solutions} and use that
$\NL_{A'/A} = \tau_{\geq -1} L_{A'/A}$)
such that a solution exists if and only if this element is in
the image of the map
$$
\Ext^1_B(\NL_{B/A'}, N) = \Ext^1_B(L_{B/A'}, N)
\longrightarrow
\Ext^1_B(L_{A/A'} \otimes_A^\mathbf{L} B, N)
$$
The distinguished triangle (\ref{equation-triangle})
for $A' \to A \to B$ gives rise to a long exact sequence
$$
\ldots \to
\Ext^1_B(L_{B/A'}, N) \to
\Ext^1_B(L_{A/A'} \otimes_A^\mathbf{L} B, N) \to
\Ext^2_B(L_{B/A}, N) \to \ldots
$$
Hence taking $\xi$ the image of $\xi'$ works.
\end{proof}





\section{The Atiyah class of a module}
\label{section-atiyah}

\noindent
Let $A \to B$ be a ring map. Let $M$ be a $B$-module.
Let $P \to B$ be an object of $\mathcal{C}_{B/A}$
(Section \ref{section-compute-L-pi-shriek}).
Consider the extension of principal parts
$$
0 \to \Omega_{P/A} \otimes_P M \to P^1_{P/A}(M) \to M \to 0
$$
see Algebra, Lemma \ref{algebra-lemma-sequence-of-principal-parts}.
This sequence is functorial in $P$ by
Algebra, Remark \ref{algebra-remark-functoriality-principal-parts}.
Thus we obtain a short exact sequence of sheaves of $\mathcal{O}$-modules
$$
0 \to \Omega_{\mathcal{O}/\underline{A}} \otimes_\mathcal{O} \underline{M} \to
P^1_{\mathcal{O}/\underline{A}}(M) \to \underline{M} \to 0
$$
on $\mathcal{C}_{B/A}$. We have
$L\pi_!(\Omega_{\mathcal{O}/\underline{A}} \otimes_\mathcal{O} \underline{M})
= L_{B/A} \otimes_B M = L_{B/A} \otimes_B^\mathbf{L} M$
by Lemma \ref{lemma-pi-shriek-standard} and the flatness of
the terms of $L_{B/A}$.
We have $L\pi_!(\underline{M}) = M$ by
Lemma \ref{lemma-pi-lower-shriek-constant-sheaf}.
Thus a distinguished triangle
\begin{equation}
\label{equation-atiyah}
L_{B/A} \otimes_B^\mathbf{L} M \to
L\pi_!\left(P^1_{\mathcal{O}/\underline{A}}(M)\right) \to M
\to L_{B/A} \otimes_B^\mathbf{L} M [1]
\end{equation}
in $D(B)$. Here we use Cohomology on Sites, Remark
\ref{sites-cohomology-remark-O-homology-B-homology-general}
to get a distinguished triangle in $D(B)$ and not just in $D(A)$.

\begin{definition}
\label{definition-atiyah-class}
Let $A \to B$ be a ring map. Let $M$ be a $B$-module.
The map $M \to L_{B/A} \otimes_B^\mathbf{L} M[1]$
in (\ref{equation-atiyah}) is called the {\it Atiyah class} of $M$.
\end{definition}




\section{The cotangent complex}
\label{section-cotangent-complex}

\noindent
In this section we discuss the cotangent complex of a map of sheaves
of rings on a site. In later sections we specialize this to obtain
the cotangent complex of a morphism of ringed topoi, a morphism of
ringed spaces, a morphism of schemes, a morphism of algebraic space, etc.

\medskip\noindent
Let $\mathcal{C}$ be a site and let $\Sh(\mathcal{C})$ denote the
associated topos. Let $\mathcal{A}$ denote a sheaf of rings
on $\mathcal{C}$. Let $\mathcal{A}\textit{-Alg}$ be the category of
$\mathcal{A}$-algebras. Consider the pair of adjoint functors $(U, V)$ where
$V : \mathcal{A}\textit{-Alg} \to \Sh(\mathcal{C})$ is the forgetful functor and
$U : \Sh(\mathcal{C}) \to \mathcal{A}\textit{-Alg}$ assigns to a sheaf of sets
$\mathcal{E}$ the polynomial algebra $\mathcal{A}[\mathcal{E}]$ on
$\mathcal{E}$ over $\mathcal{A}$.
Let $X_\bullet$ be the simplicial object of
$\text{Fun}(\mathcal{A}\textit{-Alg}, \mathcal{A}\textit{-Alg})$
constructed in
Simplicial, Section \ref{simplicial-section-standard}.

\medskip\noindent
Now assume that $\mathcal{A} \to \mathcal{B}$ is a homomorphism of sheaves
of rings. Then $\mathcal{B}$ is an object of the category
$\mathcal{A}\textit{-Alg}$. Denote
$\mathcal{P}_\bullet = X_\bullet(\mathcal{B})$ the resulting
simplicial $\mathcal{A}$-algebra.
Recall that
$\mathcal{P}_0 = \mathcal{A}[\mathcal{B}]$,
$\mathcal{P}_1 = \mathcal{A}[\mathcal{A}[\mathcal{B}]]$, and so on.
Recall also that there is an augmentation
$$
\epsilon : \mathcal{P}_\bullet \longrightarrow \mathcal{B}
$$
where we view $\mathcal{B}$ as a constant simplicial $\mathcal{A}$-algebra.

\begin{definition}
\label{definition-standard-resolution-sheaves-rings}
Let $\mathcal{C}$ be a site.
Let $\mathcal{A} \to \mathcal{B}$ be a homomorphism of sheaves of rings
on $\mathcal{C}$. The {\it standard resolution of $\mathcal{B}$ over
$\mathcal{A}$} is the augmentation
$\epsilon : \mathcal{P}_\bullet \to \mathcal{B}$
with terms
$$
\mathcal{P}_0 = \mathcal{A}[\mathcal{B}],\quad
\mathcal{P}_1 = \mathcal{A}[\mathcal{A}[\mathcal{B}]],\quad \ldots
$$
and maps as constructed above.
\end{definition}

\noindent
With this definition in hand the cotangent complex of a map of sheaves
of rings is defined as follows.
We will use the module of differentials as defined in
Modules on Sites, Section \ref{sites-modules-section-differentials}.

\begin{definition}
\label{definition-cotangent-complex-morphism-sheaves-rings}
Let $\mathcal{C}$ be a site.
Let $\mathcal{A} \to \mathcal{B}$ be a homomorphism of sheaves of rings
on $\mathcal{C}$.
The {\it cotangent complex} $L_{\mathcal{B}/\mathcal{A}}$
is the complex of $\mathcal{B}$-modules associated to the
simplicial module
$$
\Omega_{\mathcal{P}_\bullet/\mathcal{A}}
\otimes_{\mathcal{P}_\bullet, \epsilon} \mathcal{B}
$$
where $\epsilon : \mathcal{P}_\bullet \to \mathcal{B}$
is the standard resolution of $\mathcal{B}$ over
$\mathcal{A}$. We usually think of $L_{\mathcal{B}/\mathcal{A}}$
as an object of $D(\mathcal{B})$.
\end{definition}

\noindent
These constructions satisfy a functoriality similar to that discussed
in Section \ref{section-functoriality}. Namely, given a commutative diagram
\begin{equation}
\label{equation-commutative-square-sheaves}
\vcenter{
\xymatrix{
\mathcal{B} \ar[r] & \mathcal{B}' \\
\mathcal{A} \ar[u] \ar[r] & \mathcal{A}' \ar[u]
}
}
\end{equation}
of sheaves of rings on $\mathcal{C}$ there is a canonical
$\mathcal{B}$-linear map of complexes
$$
L_{\mathcal{B}/\mathcal{A}} \longrightarrow L_{\mathcal{B}'/\mathcal{A}'}
$$
constructed as follows. If $\mathcal{P}_\bullet \to \mathcal{B}$ is the
standard resolution of $\mathcal{B}$ over $\mathcal{A}$ and
$\mathcal{P}'_\bullet \to \mathcal{B}'$ is the
standard resolution of $\mathcal{B}'$ over $\mathcal{A}'$,
then there is a canonical map $\mathcal{P}_\bullet \to \mathcal{P}'_\bullet$
of simplicial $\mathcal{A}$-algebras compatible with the augmentations
$\mathcal{P}_\bullet \to \mathcal{B}$ and
$\mathcal{P}'_\bullet \to \mathcal{B}'$. The maps
$$
\mathcal{P}_0 = \mathcal{A}[\mathcal{B}]
\longrightarrow
\mathcal{A}'[\mathcal{B}'] = \mathcal{P}'_0,
\quad
\mathcal{P}_1 = \mathcal{A}[\mathcal{A}[\mathcal{B}]]
\longrightarrow
\mathcal{A}'[\mathcal{A}'[\mathcal{B}']] = \mathcal{P}'_1
$$
and so on are given by the given maps $\mathcal{A} \to \mathcal{A}'$
and $\mathcal{B} \to \mathcal{B}'$. The desired map
$L_{\mathcal{B}/\mathcal{A}} \to L_{\mathcal{B}'/\mathcal{A}'}$
then comes from the associated maps on sheaves of differentials.

\begin{lemma}
\label{lemma-pullback-cotangent-morphism-topoi}
Let $f : \Sh(\mathcal{D}) \to \Sh(\mathcal{C})$ be a morphism of topoi.
Let $\mathcal{A} \to \mathcal{B}$ be a homomorphism of sheaves of rings
on $\mathcal{C}$. Then
$f^{-1}L_{\mathcal{B}/\mathcal{A}} = L_{f^{-1}\mathcal{B}/f^{-1}\mathcal{A}}$.
\end{lemma}

\begin{proof}
The diagram
$$
\xymatrix{
\mathcal{A}\textit{-Alg} \ar[d]_{f^{-1}} \ar[r] &
\Sh(\mathcal{C}) \ar@<1ex>[l] \ar[d]^{f^{-1}} \\
f^{-1}\mathcal{A}\textit{-Alg} \ar[r] & \Sh(\mathcal{D}) \ar@<1ex>[l]
}
$$
commutes.
\end{proof}

\begin{lemma}
\label{lemma-compute-L-morphism-sheaves-rings}
Let $\mathcal{C}$ be a site. Let $\mathcal{A} \to \mathcal{B}$ be a
homomorphism of sheaves of rings on $\mathcal{C}$. Then
$H^i(L_{\mathcal{B}/\mathcal{A}})$ is the sheaf associated to the
presheaf $U \mapsto H^i(L_{\mathcal{B}(U)/\mathcal{A}(U)})$.
\end{lemma}

\begin{proof}
Let $\mathcal{C}'$ be the site we get by endowing $\mathcal{C}$ with the
chaotic topology (presheaves are sheaves). There is a morphism of topoi
$f : \Sh(\mathcal{C}) \to \Sh(\mathcal{C}')$ where $f_*$ is the inclusion
of sheaves into presheaves and $f^{-1}$ is sheafification.
By Lemma \ref{lemma-pullback-cotangent-morphism-topoi}
it suffices to prove the result for $\mathcal{C}'$, i.e.,
in case $\mathcal{C}$ has the chaotic topology.

\medskip\noindent
If $\mathcal{C}$ carries the chaotic topology, then
$L_{\mathcal{B}/\mathcal{A}}(U)$ is equal to
$L_{\mathcal{B}(U)/\mathcal{A}(U)}$ because
$$
\xymatrix{
\mathcal{A}\textit{-Alg} \ar[d]_{\text{sections over }U} \ar[r] &
\Sh(\mathcal{C}) \ar@<1ex>[l] \ar[d]^{\text{sections over }U} \\
\mathcal{A}(U)\textit{-Alg} \ar[r] & \textit{Sets} \ar@<1ex>[l]
}
$$
commutes.
\end{proof}

\begin{remark}
\label{remark-map-sections-over-U}
It is clear from the proof of
Lemma \ref{lemma-compute-L-morphism-sheaves-rings}
that for any $U \in \Ob(\mathcal{C})$ there is a canonical map
$L_{\mathcal{B}(U)/\mathcal{A}(U)} \to L_{\mathcal{B}/\mathcal{A}}(U)$
of complexes of $\mathcal{B}(U)$-modules. Moreover, these maps
are compatible with restriction maps and the complex
$L_{\mathcal{B}/\mathcal{A}}$
is the sheafification of the rule $U \mapsto L_{\mathcal{B}(U)/\mathcal{A}(U)}$.
\end{remark}

\begin{lemma}
\label{lemma-H0-L-morphism-sheaves-rings}
Let $\mathcal{C}$ be a site. Let $\mathcal{A} \to \mathcal{B}$ be a
homomorphism of sheaves of rings on $\mathcal{C}$. Then
$H^0(L_{\mathcal{B}/\mathcal{A}}) = \Omega_{\mathcal{B}/\mathcal{A}}$.
\end{lemma}

\begin{proof}
Follows from Lemmas \ref{lemma-compute-L-morphism-sheaves-rings}
and \ref{lemma-identify-H0} and
Modules on Sites, Lemma \ref{sites-modules-lemma-differentials-sheafify}.
\end{proof}

\begin{lemma}
\label{lemma-compute-L-product-sheaves-rings}
Let $\mathcal{C}$ be a site. Let $\mathcal{A} \to \mathcal{B}$
and $\mathcal{A} \to \mathcal{B}'$ be homomorphisms of sheaves of rings
on $\mathcal{C}$. Then
$$
L_{\mathcal{B} \times \mathcal{B}'/\mathcal{A}}
\longrightarrow
L_{\mathcal{B}/\mathcal{A}} \oplus L_{\mathcal{B}'/\mathcal{A}}
$$
is an isomorphism in $D(\mathcal{B} \times \mathcal{B}')$.
\end{lemma}

\begin{proof}
By Lemma \ref{lemma-compute-L-morphism-sheaves-rings}
it suffices to prove this for ring maps.
In the case of rings this is
Lemma \ref{lemma-cotangent-complex-product}.
\end{proof}

\noindent
The fundamental triangle for the cotangent complex of sheaves of rings
is an easy consequence of the result for homomorphisms of rings.

\begin{lemma}
\label{lemma-triangle-sheaves-rings}
Let $\mathcal{D}$ be a site. Let $\mathcal{A} \to \mathcal{B} \to \mathcal{C}$
be homomorphisms of sheaves of rings on $\mathcal{D}$.
There is a canonical distinguished triangle
$$
L_{\mathcal{B}/\mathcal{A}} \otimes_\mathcal{B}^\mathbf{L} \mathcal{C}
\to L_{\mathcal{C}/\mathcal{A}} \to L_{\mathcal{C}/\mathcal{B}} \to
L_{\mathcal{B}/\mathcal{A}} \otimes_\mathcal{B}^\mathbf{L} \mathcal{C}[1]
$$
in $D(\mathcal{C})$.
\end{lemma}

\begin{proof}
We will use the method described in
Remarks \ref{remark-triangle} and \ref{remark-explicit-map}
to construct the triangle; we will freely use the results mentioned there.
As in those remarks we first construct the triangle in case
$\mathcal{B} \to \mathcal{C}$ is an injective map of sheaves of rings.
In this case we set
\begin{enumerate}
\item $\mathcal{P}_\bullet$ is the standard resolution of $\mathcal{B}$
over $\mathcal{A}$,
\item $\mathcal{Q}_\bullet$ is the standard resolution of $\mathcal{C}$
over $\mathcal{A}$,
\item $\mathcal{R}_\bullet$ is the standard resolution of $\mathcal{C}$
over $\mathcal{B}$,
\item $\mathcal{S}_\bullet$ is the standard resolution of $\mathcal{B}$
over $\mathcal{B}$,
\item $\overline{\mathcal{Q}}_\bullet =
\mathcal{Q}_\bullet \otimes_{\mathcal{P}_\bullet} \mathcal{B}$, and
\item $\overline{\mathcal{R}}_\bullet =
\mathcal{R}_\bullet \otimes_{\mathcal{S}_\bullet} \mathcal{B}$.
\end{enumerate}
The distinguished triangle is the distinguished triangle associated
to the short exact sequence
of simplicial $\mathcal{C}$-modules
$$
0 \to
\Omega_{\mathcal{P}_\bullet/\mathcal{A}}
\otimes_{\mathcal{P}_\bullet} \mathcal{C} \to
\Omega_{\mathcal{Q}_\bullet/\mathcal{A}}
\otimes_{\mathcal{Q}_\bullet} \mathcal{C} \to
\Omega_{\overline{\mathcal{Q}}_\bullet/\mathcal{B}}
\otimes_{\overline{\mathcal{Q}}_\bullet} \mathcal{C} \to 0
$$
The first two terms are equal to the first two terms of the triangle
of the statement of the lemma. The identification of the last term with
$L_{\mathcal{C}/\mathcal{B}}$ uses the quasi-isomorphisms of complexes
$$
L_{\mathcal{C}/\mathcal{B}} =
\Omega_{\mathcal{R}_\bullet/\mathcal{B}}
\otimes_{\mathcal{R}_\bullet} \mathcal{C}
\longrightarrow
\Omega_{\overline{\mathcal{R}}_\bullet/\mathcal{B}}
\otimes_{\overline{\mathcal{R}}_\bullet} \mathcal{C}
\longleftarrow
\Omega_{\overline{\mathcal{Q}}_\bullet/\mathcal{B}}
\otimes_{\overline{\mathcal{Q}}_\bullet} \mathcal{C}
$$
All the constructions used above can first be done on the level
of presheaves and then sheafified. Hence to prove sequences are exact,
or that map are quasi-isomorphisms it suffices to prove the corresponding
statement for the ring maps
$\mathcal{A}(U) \to \mathcal{B}(U) \to \mathcal{C}(U)$
which are known. This finishes the proof in the case that
$\mathcal{B} \to \mathcal{C}$ is injective.

\medskip\noindent
In general, we reduce to the case where $\mathcal{B} \to \mathcal{C}$ is
injective by replacing $\mathcal{C}$ by $\mathcal{B} \times \mathcal{C}$ if
necessary. This is possible by the argument given in
Remark \ref{remark-triangle} by
Lemma \ref{lemma-compute-L-product-sheaves-rings}.
\end{proof}

\begin{lemma}
\label{lemma-stalk-cotangent-complex}
Let $\mathcal{C}$ be a site. Let $\mathcal{A} \to \mathcal{B}$ be a
homomorphism of sheaves of rings on $\mathcal{C}$. If $p$ is a point
of $\mathcal{C}$, then
$(L_{\mathcal{B}/\mathcal{A}})_p = L_{\mathcal{B}_p/\mathcal{A}_p}$.
\end{lemma}

\begin{proof}
This is a special case of Lemma \ref{lemma-pullback-cotangent-morphism-topoi}.
\end{proof}

\noindent
For the construction of the naive cotangent complex and its properties
we refer to
Modules on Sites, Section \ref{sites-modules-section-netherlander}.

\begin{lemma}
\label{lemma-compare-cotangent-complex-with-naive}
Let $\mathcal{C}$ be a site. Let $\mathcal{A} \to \mathcal{B}$ be a
homomorphism of sheaves of rings on $\mathcal{C}$.
There is a canonical map
$L_{\mathcal{B}/\mathcal{A}} \to \NL_{\mathcal{B}/\mathcal{A}}$
which identifies the naive cotangent complex with the truncation
$\tau_{\geq -1}L_{\mathcal{B}/\mathcal{A}}$.
\end{lemma}

\begin{proof}
Let $\mathcal{P}_\bullet$ be the standard resolution of $\mathcal{B}$
over $\mathcal{A}$.
Let $\mathcal{I} = \Ker(\mathcal{A}[\mathcal{B}] \to \mathcal{B})$.
Recall that $\mathcal{P}_0 = \mathcal{A}[\mathcal{B}]$. The map of the
lemma is given by the commutative diagram
$$
\xymatrix{
L_{\mathcal{B}/\mathcal{A}} \ar[d] & \ldots \ar[r] &
\Omega_{\mathcal{P}_2/\mathcal{A}} \otimes_{\mathcal{P}_2} \mathcal{B}
\ar[r] \ar[d] &
\Omega_{\mathcal{P}_1/\mathcal{A}} \otimes_{\mathcal{P}_1} \mathcal{B}
\ar[r] \ar[d] &
\Omega_{\mathcal{P}_0/\mathcal{A}} \otimes_{\mathcal{P}_0} \mathcal{B}
\ar[d] \\
\NL_{\mathcal{B}/\mathcal{A}} & \ldots \ar[r] &
0 \ar[r] & 
\mathcal{I}/\mathcal{I}^2 \ar[r] &
\Omega_{\mathcal{P}_0/\mathcal{A}} \otimes_{\mathcal{P}_0} \mathcal{B}
}
$$
We construct the downward arrow with target $\mathcal{I}/\mathcal{I}^2$
by sending a local section $\text{d}f \otimes b$ to the class of
$(d_0(f) - d_1(f))b$ in $\mathcal{I}/\mathcal{I}^2$.
Here $d_i : \mathcal{P}_1 \to \mathcal{P}_0$,
$i = 0, 1$ are the two face maps of the simplicial structure.
This makes sense as $d_0 - d_1$ maps $\mathcal{P}_1$ into
$\mathcal{I} = \Ker(\mathcal{P}_0 \to \mathcal{B})$.
We omit the verification that this rule is well defined.
Our map is compatible with the differential
$\Omega_{\mathcal{P}_1/\mathcal{A}} \otimes_{\mathcal{P}_1} \mathcal{B}
\to \Omega_{\mathcal{P}_0/\mathcal{A}} \otimes_{\mathcal{P}_0} \mathcal{B}$
as this differential maps a local section $\text{d}f \otimes b$ to
$\text{d}(d_0(f) - d_1(f)) \otimes b$. Moreover, the differential
$\Omega_{\mathcal{P}_2/\mathcal{A}} \otimes_{\mathcal{P}_2} \mathcal{B}
\to \Omega_{\mathcal{P}_1/\mathcal{A}} \otimes_{\mathcal{P}_1} \mathcal{B}$
maps a local section $\text{d}f \otimes b$ to
$\text{d}(d_0(f) - d_1(f) + d_2(f)) \otimes b$
which are annihilated by our downward arrow. Hence a map of complexes.

\medskip\noindent
To see that our map induces an isomorphism on the cohomology sheaves
$H^0$ and $H^{-1}$ we argue as follows. Let $\mathcal{C}'$ be the site
with the same underlying category as $\mathcal{C}$ but endowed with the
chaotic topology. Let $f : \Sh(\mathcal{C}) \to \Sh(\mathcal{C}')$ be
the morphism of topoi whose pullback functor is sheafification.
Let $\mathcal{A}' \to \mathcal{B}'$ be the given map, but thought of
as a map of sheaves of rings on $\mathcal{C}'$. The construction above
gives a map $L_{\mathcal{B}'/\mathcal{A}'} \to \NL_{\mathcal{B}'/\mathcal{A}'}$
on $\mathcal{C}'$ whose value over any object $U$ of $\mathcal{C}'$
is just the map
$$
L_{\mathcal{B}(U)/\mathcal{A}(U)} \to \NL_{\mathcal{B}(U)/\mathcal{A}(U)}
$$
of Remark \ref{remark-explicit-comparison-map} which induces an isomorphism
on $H^0$ and $H^{-1}$. Since
$f^{-1}L_{\mathcal{B}'/\mathcal{A}'} = L_{\mathcal{B}/\mathcal{A}}$
(Lemma \ref{lemma-pullback-cotangent-morphism-topoi})
and
$f^{-1}\NL_{\mathcal{B}'/\mathcal{A}'} = \NL_{\mathcal{B}/\mathcal{A}}$
(Modules on Sites, Lemma \ref{sites-modules-lemma-pullback-NL})
the lemma is proved.
\end{proof}











\section{The Atiyah class of a sheaf of modules}
\label{section-atiyah-general}

\noindent
Let $\mathcal{C}$ be a site. Let $\mathcal{A} \to \mathcal{B}$ be a
homomorphism of sheaves of rings. Let $\mathcal{F}$ be a
sheaf of $\mathcal{B}$-modules. Let $\mathcal{P}_\bullet \to \mathcal{B}$
be the standard resolution of $\mathcal{B}$ over $\mathcal{A}$
(Section \ref{section-cotangent-complex}).
For every $n \geq 0$ consider the extension of principal parts
\begin{equation}
\label{equation-atiyah-extension}
0 \to
\Omega_{\mathcal{P}_n/\mathcal{A}} \otimes_{\mathcal{P}_n} \mathcal{F} \to
\mathcal{P}^1_{\mathcal{P}_n/\mathcal{A}}(\mathcal{F}) \to
\mathcal{F} \to 0
\end{equation}
see
Modules on Sites, Lemma \ref{sites-modules-lemma-sequence-of-principal-parts}.
The functoriality of this construction
(Modules on Sites, Remark
\ref{sites-modules-remark-functoriality-principal-parts})
tells us (\ref{equation-atiyah-extension}) is the degree $n$ part of
a short exact sequence of simplicial $\mathcal{P}_\bullet$-modules
(Cohomology on Sites, Section
\ref{sites-cohomology-section-simplicial-modules}).
Using the functor $L\pi_! : D(\mathcal{P}_\bullet) \to D(\mathcal{B})$
of Cohomology on Sites, Remark
\ref{sites-cohomology-remark-homology-augmentation}
(here we use that $\mathcal{P}_\bullet \to \mathcal{A}$ is a resolution)
we obtain a distinguished triangle
\begin{equation}
\label{equation-atiyah-general}
L_{\mathcal{B}/\mathcal{A}} \otimes_\mathcal{B}^\mathbf{L} \mathcal{F} \to
L\pi_!\left(\mathcal{P}^1_{\mathcal{P}_\bullet/\mathcal{A}}(\mathcal{F})\right)
\to \mathcal{F} \to
L_{\mathcal{B}/\mathcal{A}} \otimes_\mathcal{B}^\mathbf{L} \mathcal{F} [1]
\end{equation}
in $D(\mathcal{B})$.

\begin{definition}
\label{definition-atiyah-class-general}
Let $\mathcal{C}$ be a site.
Let $\mathcal{A} \to \mathcal{B}$ be a homomorphism of sheaves of rings.
Let $\mathcal{F}$ be a sheaf of $\mathcal{B}$-modules.
The map $\mathcal{F} \to
L_{\mathcal{B}/\mathcal{A}} \otimes_\mathcal{B}^\mathbf{L} \mathcal{F}[1]$
in (\ref{equation-atiyah-general}) is called the {\it Atiyah class} of
$\mathcal{F}$.
\end{definition}









\section{The cotangent complex of a morphism of ringed spaces}
\label{section-cotangent-morphism-ringed-spaces}

\noindent
The cotangent complex of a morphism of ringed spaces is defined
in terms of the cotangent complex we defined above.

\begin{definition}
\label{definition-cotangent-complex-morphism-ringed-spaces}
Let $f : (X, \mathcal{O}_X) \to (S, \mathcal{O}_S)$ be a morphism of
ringed spaces. The {\it cotangent complex} $L_f$ of $f$ is
$L_f = L_{\mathcal{O}_X/f^{-1}\mathcal{O}_S}$.
We will also use the notation
$L_f = L_{X/S} = L_{\mathcal{O}_X/\mathcal{O}_S}$.
\end{definition}

\noindent
More precisely, this means that we consider the cotangent complex
(Definition \ref{definition-cotangent-complex-morphism-sheaves-rings})
of the homomorphism $f^\sharp : f^{-1}\mathcal{O}_S \to \mathcal{O}_X$
of sheaves of rings on the site associated to the topological space $X$
(Sites, Example \ref{sites-example-site-topological}).

\begin{lemma}
\label{lemma-H0-L-morphism-ringed-spaces}
Let $f : (X, \mathcal{O}_X) \to (S, \mathcal{O}_S)$ be a morphism of
ringed spaces. Then $H^0(L_{X/S}) = \Omega_{X/S}$.
\end{lemma}

\begin{proof}
Special case of Lemma \ref{lemma-H0-L-morphism-sheaves-rings}.
\end{proof}

\begin{lemma}
\label{lemma-triangle-ringed-spaces}
Let $f : X \to Y$ and $g : Y \to Z$ be morphisms of ringed spaces.
Then there is a canonical distinguished triangle
$$
Lf^* L_{Y/Z} \to L_{X/Z} \to L_{X/Y} \to Lf^*L_{Y/Z}[1]
$$
in $D(\mathcal{O}_X)$.
\end{lemma}

\begin{proof}
Set $h = g \circ f$ so that $h^{-1}\mathcal{O}_Z = f^{-1}g^{-1}\mathcal{O}_Z$.
By Lemma \ref{lemma-pullback-cotangent-morphism-topoi} we have
$f^{-1}L_{Y/Z} = L_{f^{-1}\mathcal{O}_Y/h^{-1}\mathcal{O}_Z}$
and this is a complex of flat $f^{-1}\mathcal{O}_Y$-modules.
Hence the distinguished triangle above is an example of the
distinguished triangle of
Lemma \ref{lemma-triangle-sheaves-rings}
with $\mathcal{A} = h^{-1}\mathcal{O}_Z$, $\mathcal{B} = f^{-1}\mathcal{O}_Y$,
and $\mathcal{C} = \mathcal{O}_X$.
\end{proof}

\begin{lemma}
\label{lemma-compare-cotangent-complex-with-naive-ringed-spaces}
Let $f : (X, \mathcal{O}_X) \to (Y, \mathcal{O}_Y)$ be a morphism of
ringed spaces. There is a canonical map $L_{X/Y} \to \NL_{X/Y}$ which
identifies the naive cotangent complex with the truncation
$\tau_{\geq -1}L_{X/Y}$.
\end{lemma}

\begin{proof}
Special case of Lemma \ref{lemma-compare-cotangent-complex-with-naive}.
\end{proof}






\section{Deformations of ringed spaces and the cotangent complex}
\label{section-deformations-ringed-spaces}

\noindent
This section is the continuation of
Deformation Theory, Section \ref{defos-section-deformations-ringed-spaces}
which we urge the reader to read first. We briefly recall the setup.
We have a first order thickening
$t : (S, \mathcal{O}_S) \to (S', \mathcal{O}_{S'})$ of ringed spaces
with $\mathcal{J} = \Ker(t^\sharp)$, a morphism of ringed spaces
$f : (X, \mathcal{O}_X) \to (S, \mathcal{O}_S)$, an $\mathcal{O}_X$-module
$\mathcal{G}$, and an $f$-map $c : \mathcal{J} \to \mathcal{G}$
of sheaves of modules. We ask whether we can find
the question mark fitting into the following diagram
\begin{equation}
\label{equation-to-solve-ringed-spaces}
\vcenter{
\xymatrix{
0 \ar[r] & \mathcal{G} \ar[r] & {?} \ar[r] & \mathcal{O}_X \ar[r] & 0 \\
0 \ar[r] & \mathcal{J} \ar[u]^c \ar[r] & \mathcal{O}_{S'} \ar[u] \ar[r] &
\mathcal{O}_S \ar[u] \ar[r] & 0
}
}
\end{equation}
and moreover how unique the solution is (if it exists). More precisely,
we look for a first order thickening
$i : (X, \mathcal{O}_X) \to (X', \mathcal{O}_{X'})$
and a morphism of thickenings $(f, f')$ as in
Deformation Theory, Equation (\ref{defos-equation-morphism-thickenings})
where $\Ker(i^\sharp)$ is identified with $\mathcal{G}$
such that $(f')^\sharp$ induces the given map $c$.
We will say $X'$ is a {\it solution} to
(\ref{equation-to-solve-ringed-spaces}).

\begin{lemma}
\label{lemma-find-obstruction-ringed-spaces}
In the situation above we have
\begin{enumerate}
\item There is a canonical element
$\xi \in \Ext^2_{\mathcal{O}_X}(L_{X/S}, \mathcal{G})$
whose vanishing is a sufficient and necessary condition for the existence
of a solution to (\ref{equation-to-solve-ringed-spaces}).
\item If there exists a solution, then the set of
isomorphism classes of solutions is principal homogeneous under
$\Ext^1_{\mathcal{O}_X}(L_{X/S}, \mathcal{G})$.
\item Given a solution $X'$, the set of automorphisms of $X'$
fitting into (\ref{equation-to-solve-ringed-spaces}) is canonically isomorphic
to $\Ext^0_{\mathcal{O}_X}(L_{X/S}, \mathcal{G})$.
\end{enumerate}
\end{lemma}

\begin{proof}
Via the identifications $\NL_{X/S} = \tau_{\geq -1}L_{X/S}$
(Lemma \ref{lemma-compare-cotangent-complex-with-naive-ringed-spaces})
and
$H^0(L_{X/S}) = \Omega_{X/S}$
(Lemma \ref{lemma-H0-L-morphism-ringed-spaces})
we have seen parts (2) and (3) in
Deformation Theory, Lemmas \ref{defos-lemma-huge-diagram-ringed-spaces} and
\ref{defos-lemma-choices-ringed-spaces}.

\medskip\noindent
Proof of (1). Roughly speaking, this follows from the discussion in
Deformation Theory, Remark
\ref{defos-remark-parametrize-solutions-ringed-spaces}
by replacing the naive cotangent complex by the full cotangent complex.
Here is a more detailed explanation. By
Deformation Theory, Lemma \ref{defos-lemma-parametrize-solutions-ringed-spaces}
there exists an element
$$
\xi' \in
\Ext^1_{\mathcal{O}_X}(Lf^*\NL_{S/S'}, \mathcal{G}) =
\Ext^1_{\mathcal{O}_X}(Lf^*L_{S/S'}, \mathcal{G})
$$
such that a solution exists if and only if this element is in
the image of the map
$$
\Ext^1_{\mathcal{O}_X}(NL_{X/S'}, \mathcal{G}) =
\Ext^1_{\mathcal{O}_X}(L_{X/S'}, \mathcal{G})
\longrightarrow
\Ext^1_{\mathcal{O}_X}(Lf^*L_{S/S'}, \mathcal{G})
$$
The distinguished triangle of Lemma \ref{lemma-triangle-ringed-spaces}
for $X \to S \to S'$ gives rise to a long exact sequence
$$
\ldots \to
\Ext^1_{\mathcal{O}_X}(L_{X/S'}, \mathcal{G}) \to
\Ext^1_{\mathcal{O}_X}(Lf^*L_{S/S'}, \mathcal{G}) \to
\Ext^2_{\mathcal{O}_X}(L_{X/S}, \mathcal{G}) \to \ldots
$$
Hence taking $\xi$ the image of $\xi'$ works.
\end{proof}










\section{The cotangent complex of a morphism of ringed topoi}
\label{section-cotangent-morphism-ringed-topoi}

\noindent
The cotangent complex of a morphism of ringed topoi is defined
in terms of the cotangent complex we defined above.

\begin{definition}
\label{definition-cotangent-complex-morphism-ringed-topoi}
Let $(f, f^\sharp) : (\Sh(\mathcal{C}), \mathcal{O}_\mathcal{C}) \to
(\Sh(\mathcal{D}), \mathcal{O}_\mathcal{D})$ be a morphism of ringed topoi.
The {\it cotangent complex} $L_f$ of $f$ is
$L_f = L_{\mathcal{O}_\mathcal{C}/f^{-1}\mathcal{O}_\mathcal{D}}$.
We sometimes write $L_f = L_{\mathcal{O}_\mathcal{C}/\mathcal{O}_\mathcal{D}}$.
\end{definition}

\noindent
This definition applies to many situations, but it doesn't always produce
the thing one expects. For example, if $f : X \to Y$ is a morphism of
schemes, then $f$ induces a morphism of big \'etale sites
$f_{big} : (\Sch/X)_\etale \to (\Sch/Y)_\etale$
which is a morphism of ringed topoi (Descent, Remark
\ref{descent-remark-change-topologies-ringed}).
However, $L_{f_{big}} = 0$ since $(f_{big})^\sharp$ is an isomorphism.
On the other hand, if we take $L_f$ where we think of $f$ as a morphism
between the underlying Zariski ringed topoi, then $L_f$ does agree with
the cotangent complex $L_{X/Y}$ (as defined below)
whose zeroth cohomology sheaf is $\Omega_{X/Y}$.

\begin{lemma}
\label{lemma-H0-L-morphism-ringed-topoi}
Let $f : (\Sh(\mathcal{C}), \mathcal{O}) \to
(\Sh(\mathcal{B}), \mathcal{O}_\mathcal{B})$ be a morphism of
ringed topoi. Then $H^0(L_f) = \Omega_f$.
\end{lemma}

\begin{proof}
Special case of Lemma \ref{lemma-H0-L-morphism-sheaves-rings}.
\end{proof}

\begin{lemma}
\label{lemma-triangle-ringed-topoi}
Let $f : (\Sh(\mathcal{C}_1), \mathcal{O}_1) \to
(\Sh(\mathcal{C}_2), \mathcal{O}_2)$ and
$g : (\Sh(\mathcal{C}_2), \mathcal{O}_2) \to
(\Sh(\mathcal{C}_3), \mathcal{O}_3)$ be morphisms of ringed topoi.
Then there is a canonical distinguished triangle
$$
Lf^* L_g \to L_{g \circ f} \to L_f \to Lf^*L_g[1]
$$
in $D(\mathcal{O}_1)$.
\end{lemma}

\begin{proof}
Set $h = g \circ f$ so that $h^{-1}\mathcal{O}_3 = f^{-1}g^{-1}\mathcal{O}_3$.
By Lemma \ref{lemma-pullback-cotangent-morphism-topoi} we have
$f^{-1}L_g = L_{f^{-1}\mathcal{O}_2/h^{-1}\mathcal{O}_3}$
and this is a complex of flat $f^{-1}\mathcal{O}_2$-modules.
Hence the distinguished triangle above is an example of the
distinguished triangle of
Lemma \ref{lemma-triangle-sheaves-rings}
with $\mathcal{A} = h^{-1}\mathcal{O}_3$, $\mathcal{B} = f^{-1}\mathcal{O}_2$,
and $\mathcal{C} = \mathcal{O}_1$.
\end{proof}

\begin{lemma}
\label{lemma-compare-cotangent-complex-with-naive-ringed-topoi}
Let $f : (\Sh(\mathcal{C}), \mathcal{O}) \to
(\Sh(\mathcal{B}), \mathcal{O}_\mathcal{B})$ be a morphism of
ringed topoi. There is a canonical map $L_f \to \NL_f$ which
identifies the naive cotangent complex with the truncation
$\tau_{\geq -1}L_f$.
\end{lemma}

\begin{proof}
Special case of Lemma \ref{lemma-compare-cotangent-complex-with-naive}.
\end{proof}








\section{Deformations of ringed topoi and the cotangent complex}
\label{section-deformations-ringed-topoi}

\noindent
This section is the continuation of
Deformation Theory, Section \ref{defos-section-deformations-ringed-topoi}
which we urge the reader to read first. We briefly recall the setup.
We have a first order thickening
$t : (\Sh(\mathcal{B}), \mathcal{O}_\mathcal{B}) \to
(\Sh(\mathcal{B}'), \mathcal{O}_{\mathcal{B}'})$ of ringed topoi
with $\mathcal{J} = \Ker(t^\sharp)$, a morphism of ringed topoi
$f : (\Sh(\mathcal{C}), \mathcal{O}) \to
(\Sh(\mathcal{B}), \mathcal{O}_\mathcal{B})$, an $\mathcal{O}$-module
$\mathcal{G}$, and a map $f^{-1}\mathcal{J} \to \mathcal{G}$
of sheaves of $f^{-1}\mathcal{O}_\mathcal{B}$-modules.
We ask whether we can find
the question mark fitting into the following diagram
\begin{equation}
\label{equation-to-solve-ringed-topoi}
\vcenter{
\xymatrix{
0 \ar[r] & \mathcal{G} \ar[r] & {?} \ar[r] & \mathcal{O} \ar[r] & 0 \\
0 \ar[r] & f^{-1}\mathcal{J} \ar[u]^c \ar[r] &
f^{-1}\mathcal{O}_{\mathcal{B}'} \ar[u] \ar[r] &
f^{-1}\mathcal{O}_\mathcal{B} \ar[u] \ar[r] & 0
}
}
\end{equation}
and moreover how unique the solution is (if it exists). More precisely,
we look for a first order thickening
$i : (\Sh(\mathcal{C}), \mathcal{O}) \to (\Sh(\mathcal{C}'), \mathcal{O}')$
and a morphism of thickenings $(f, f')$ as in
Deformation Theory, Equation
(\ref{defos-equation-morphism-thickenings-ringed-topoi})
where $\Ker(i^\sharp)$ is identified with $\mathcal{G}$
such that $(f')^\sharp$ induces the given map $c$.
We will say $(\Sh(\mathcal{C}'), \mathcal{O}')$ is a {\it solution} to
(\ref{equation-to-solve-ringed-topoi}).

\begin{lemma}
\label{lemma-find-obstruction-ringed-topoi}
In the situation above we have
\begin{enumerate}
\item There is a canonical element
$\xi \in \Ext^2_\mathcal{O}(L_f, \mathcal{G})$
whose vanishing is a sufficient and necessary condition for the existence
of a solution to (\ref{equation-to-solve-ringed-topoi}).
\item If there exists a solution, then the set of
isomorphism classes of solutions is principal homogeneous under
$\Ext^1_\mathcal{O}(L_f, \mathcal{G})$.
\item Given a solution $X'$, the set of automorphisms of $X'$
fitting into (\ref{equation-to-solve-ringed-topoi}) is canonically isomorphic
to $\Ext^0_\mathcal{O}(L_f, \mathcal{G})$.
\end{enumerate}
\end{lemma}

\begin{proof}
Via the identifications $\NL_f = \tau_{\geq -1}L_f$
(Lemma \ref{lemma-compare-cotangent-complex-with-naive-ringed-topoi}) and
$H^0(L_f) = \Omega_f$
(Lemma \ref{lemma-H0-L-morphism-ringed-topoi})
we have seen parts (2) and (3) in
Deformation Theory, Lemmas \ref{defos-lemma-huge-diagram-ringed-topoi} and
\ref{defos-lemma-choices-ringed-topoi}.

\medskip\noindent
Proof of (1). To match notation with Deformation Theory, Section
\ref{defos-section-deformations-ringed-topoi} we will write
$\NL_f = \NL_{\mathcal{O}/\mathcal{O}_\mathcal{B}}$ and
$L_f = L_{\mathcal{O}/\mathcal{O}_\mathcal{B}}$ and similarly
for the morphisms $t$ and $t \circ f$. By
Deformation Theory, Lemma \ref{defos-lemma-parametrize-solutions-ringed-topoi}
there exists an element
$$
\xi' \in
\Ext^1_\mathcal{O}(
Lf^*\NL_{\mathcal{O}_\mathcal{B}/\mathcal{O}_{\mathcal{B}'}}, \mathcal{G}) =
\Ext^1_\mathcal{O}(
Lf^*L_{\mathcal{O}_\mathcal{B}/\mathcal{O}_{\mathcal{B}'}}, \mathcal{G})
$$
such that a solution exists if and only if this element is in
the image of the map
$$
\Ext^1_\mathcal{O}(
\NL_{\mathcal{O}/\mathcal{O}_{\mathcal{B}'}}, \mathcal{G}) =
\Ext^1_\mathcal{O}(
L_{\mathcal{O}/\mathcal{O}_{\mathcal{B}'}}, \mathcal{G})
\longrightarrow
\Ext^1_\mathcal{O}(
Lf^*L_{\mathcal{O}_\mathcal{B}/\mathcal{O}_{\mathcal{B}'}}, \mathcal{G})
$$
The distinguished triangle of Lemma \ref{lemma-triangle-ringed-topoi}
for $f$ and $t$ gives rise to a long exact sequence
$$
\ldots \to
\Ext^1_\mathcal{O}(
L_{\mathcal{O}/\mathcal{O}_{\mathcal{B}'}}, \mathcal{G}) \to
\Ext^1_\mathcal{O}(
Lf^*L_{\mathcal{O}_\mathcal{B}/\mathcal{O}_{\mathcal{B}'}}, \mathcal{G})
\to
\Ext^1_\mathcal{O}(
L_{\mathcal{O}/\mathcal{O}_\mathcal{B}}, \mathcal{G})
$$
Hence taking $\xi$ the image of $\xi'$ works.
\end{proof}












\section{The cotangent complex of a morphism of schemes}
\label{section-cotangent-morphism-schemes}

\noindent
As promised above we define the cotangent complex of a morphism of
schemes as follows.

\begin{definition}
\label{definition-cotangent-morphism-schemes}
Let $f : X \to Y$ be a morphism of schemes. The {\it cotangent complex
$L_{X/Y}$ of $X$ over $Y$} is the cotangent complex of $f$ as a
morphism of ringed spaces
(Definition \ref{definition-cotangent-complex-morphism-ringed-spaces}).
\end{definition}

\noindent
In particular, the results of
Section \ref{section-cotangent-morphism-ringed-spaces} apply
to cotangent complexes of morphisms of schemes.
The next lemma shows this definition is compatible with the definition
for ring maps and it also implies that $L_{X/Y}$ is an
object of $D_\QCoh(\mathcal{O}_X)$.

\begin{lemma}
\label{lemma-morphism-affine-schemes}
Let $f : X \to Y$ be a morphism of schemes. Let $U = \Spec(B) \subset X$
and $V = \Spec(A) \subset Y$ be affine opens such that $f(U) \subset V$.
There is a canonical map
$$
\widetilde{L_{B/A}} \longrightarrow L_{X/Y}|_U
$$
of complexes which is an isomorphism in $D(\mathcal{O}_U)$.
This map is compatible with restricting to smaller affine opens
of $X$ and $Y$.
\end{lemma}

\begin{proof}
By Remark \ref{remark-map-sections-over-U}
there is a canonical map of complexes
$L_{\mathcal{O}_X(U)/f^{-1}\mathcal{O}_Y(U)} \to L_{X/Y}(U)$
of $B = \mathcal{O}_X(U)$-modules, which is compatible
with further restrictions. Using the canonical map
$A \to f^{-1}\mathcal{O}_Y(U)$ we obtain a canonical map
$L_{B/A} \to L_{\mathcal{O}_X(U)/f^{-1}\mathcal{O}_Y(U)}$
of complexes of $B$-modules.
Using the universal property of the $\widetilde{\ }$
functor (see Schemes, Lemma \ref{schemes-lemma-compare-constructions})
we obtain a map as in the statement of the lemma.
We may check this map is an isomorphism on cohomology sheaves
by checking it induces isomorphisms on stalks.
This follows immediately from
Lemmas \ref{lemma-stalk-cotangent-complex} and \ref{lemma-localize}
(and the description of the stalks of
$\mathcal{O}_X$ and $f^{-1}\mathcal{O}_Y$
at a point $\mathfrak p \in \Spec(B)$ as $B_\mathfrak p$ and
$A_\mathfrak q$ where $\mathfrak q = A \cap \mathfrak p$; references
used are Schemes, Lemma \ref{schemes-lemma-spec-sheaves}
and
Sheaves, Lemma \ref{sheaves-lemma-stalk-pullback}).
\end{proof}

\begin{lemma}
\label{lemma-scheme-over-ring}
Let $\Lambda$ be a ring. Let $X$ be a scheme over $\Lambda$.
Then
$$
L_{X/\Spec(\Lambda)} = L_{\mathcal{O}_X/\underline{\Lambda}}
$$
where $\underline{\Lambda}$ is the constant sheaf with value
$\Lambda$ on $X$.
\end{lemma}

\begin{proof}
Let $p : X \to \Spec(\Lambda)$ be the structure morphism.
Let $q : \Spec(\Lambda) \to (*, \Lambda)$ be the obvious morphism.
By the distinguished triangle of Lemma \ref{lemma-triangle-ringed-spaces}
it suffices to show that $L_q = 0$. To see this it suffices to
show for $\mathfrak p \in \Spec(\Lambda)$ that
$$
(L_q)_\mathfrak p =
L_{\mathcal{O}_{\Spec(\Lambda), \mathfrak p}/\Lambda} =
L_{\Lambda_\mathfrak p/\Lambda}
$$
(Lemma \ref{lemma-stalk-cotangent-complex})
is zero which follows from Lemma \ref{lemma-when-zero}.
\end{proof}






\section{The cotangent complex of a scheme over a ring}
\label{section-cotangent-schemes-variant}

\noindent
Let $\Lambda$ be a ring and let $X$ be a scheme over $\Lambda$.
Write $L_{X/\Spec(\Lambda)} = L_{X/\Lambda}$ which is justified
by Lemma \ref{lemma-scheme-over-ring}.
In this section we give a description of $L_{X/\Lambda}$ similar to
Lemma \ref{lemma-compute-cotangent-complex}.
Namely, we construct a category $\mathcal{C}_{X/\Lambda}$
fibred over $X_{Zar}$ and endow it with a sheaf of (polynomial)
$\Lambda$-algebras $\mathcal{O}$ such that
$$
L_{X/\Lambda} =
L\pi_!(\Omega_{\mathcal{O}/\underline{\Lambda}} \otimes_\mathcal{O}
\underline{\mathcal{O}}_X).
$$
We will later use the category $\mathcal{C}_{X/\Lambda}$ to construct
a naive obstruction theory for the stack of coherent sheaves.

\medskip\noindent
Let $\Lambda$ be a ring. Let $X$ be a scheme over $\Lambda$.
Let $\mathcal{C}_{X/\Lambda}$ be the category whose objects are
commutative diagrams
\begin{equation}
\label{equation-object}
\vcenter{
\xymatrix{
X \ar[d] & U \ar[l] \ar[d] \\
\Spec(\Lambda) & \mathbf{A} \ar[l]
}
}
\end{equation}
of schemes where
\begin{enumerate}
\item $U$ is an open subscheme of $X$,
\item there exists an isomorphism $\mathbf{A} = \Spec(P)$
where $P$ is a polynomial algebra over $\Lambda$ (on some set
of variables).
\end{enumerate}
In other words, $\mathbf{A}$ is an (infinite dimensional) affine space over
$\Spec(\Lambda)$. Morphisms are given by commutative diagrams.
Recall that $X_{Zar}$ denotes the small Zariski site $X$.
There is a forgetful functor
$$
u : \mathcal{C}_{X/\Lambda} \to X_{Zar},\ (U \to \mathbf{A}) \mapsto U
$$
Observe that the fibre category over $U$ is canonically equivalent
to the category $\mathcal{C}_{\mathcal{O}_X(U)/\Lambda}$ introduced
in Section \ref{section-compute-L-pi-shriek}.

\begin{lemma}
\label{lemma-category-fibred}
In the situation above the category
$\mathcal{C}_{X/\Lambda}$ is fibred over $X_{Zar}$.
\end{lemma}

\begin{proof}
Given an object $U \to \mathbf{A}$ of $\mathcal{C}_{X/\Lambda}$ and a morphism
$U' \to U$ of $X_{Zar}$ consider the object $U' \to \mathbf{A}$ of
$\mathcal{C}_{X/\Lambda}$ where $U' \to \mathbf{A}$ is the composition of
$U \to \mathbf{A}$ and $U' \to U$. The morphism
$(U' \to \mathbf{A}) \to (U \to \mathbf{A})$ of
$\mathcal{C}_{X/\Lambda}$ is strongly cartesian over $X_{Zar}$.
\end{proof}

\noindent
We endow $\mathcal{C}_{X/\Lambda}$ with the topology inherited from
$X_{Zar}$ (see Stacks, Section \ref{stacks-section-topology}).
The functor $u$ defines a morphism of topoi
$\pi : \Sh(\mathcal{C}_{X/\Lambda}) \to \Sh(X_{Zar})$.
The site $\mathcal{C}_{X/\Lambda}$ comes with several sheaves of rings.
\begin{enumerate}
\item The sheaf $\mathcal{O}$ given by the rule
$(U \to \mathbf{A}) \mapsto \Gamma(\mathbf{A}, \mathcal{O}_\mathbf{A})$.
\item The sheaf $\underline{\mathcal{O}}_X = \pi^{-1}\mathcal{O}_X$ given by
the rule $(U \to \mathbf{A}) \mapsto \mathcal{O}_X(U)$.
\item The constant sheaf $\underline{\Lambda}$.
\end{enumerate}
We obtain morphisms of ringed topoi
\begin{equation}
\label{equation-pi-schemes}
\vcenter{
\xymatrix{
(\Sh(\mathcal{C}_{X/\Lambda}), \underline{\mathcal{O}}_X) \ar[r]_i \ar[d]_\pi &
(\Sh(\mathcal{C}_{X/\Lambda}), \mathcal{O}) \\
(\Sh(X_{Zar}), \mathcal{O}_X)
}
}
\end{equation}
The morphism $i$ is the identity on underlying topoi and
$i^\sharp : \mathcal{O} \to \underline{\mathcal{O}}_X$
is the obvious map.
The map $\pi$ is a special case of Cohomology on Sites, Situation
\ref{sites-cohomology-situation-fibred-category}.
An important role will be played in the following
by the derived functors
$
Li^* : D(\mathcal{O}) \longrightarrow D(\underline{\mathcal{O}}_X)
$
left adjoint to $Ri_* = i_* : D(\underline{\mathcal{O}}_X) \to D(\mathcal{O})$
and
$
L\pi_! : D(\underline{\mathcal{O}}_X) \longrightarrow D(\mathcal{O}_X)
$
left adjoint to
$\pi^* = \pi^{-1} : D(\mathcal{O}_X) \to D(\underline{\mathcal{O}}_X)$.
We can compute $L\pi_!$ thanks to our earlier work.

\begin{remark}
\label{remark-compute-L-pi-shriek}
In the situation above, for every $U \subset X$ open let
$P_{\bullet, U}$ be the standard resolution of $\mathcal{O}_X(U)$
over $\Lambda$. Set $\mathbf{A}_{n, U} = \Spec(P_{n, U})$. Then
$\mathbf{A}_{\bullet, U}$
is a cosimplicial object of the fibre category
$\mathcal{C}_{\mathcal{O}_X(U)/\Lambda}$ of
$\mathcal{C}_{X/\Lambda}$ over $U$. Moreover, as discussed
in Remark \ref{remark-resolution} we have that $\mathbf{A}_{\bullet, U}$
is a cosimplicial object of $\mathcal{C}_{\mathcal{O}_X(U)/\Lambda}$
as in Cohomology on Sites, Lemma
\ref{sites-cohomology-lemma-compute-by-cosimplicial-resolution}.
Since the construction $U \mapsto \mathbf{A}_{\bullet, U}$ is functorial
in $U$, given any (abelian) sheaf $\mathcal{F}$ on $\mathcal{C}_{X/\Lambda}$
we obtain a complex of presheaves
$$
U \longmapsto \mathcal{F}(\mathbf{A}_{\bullet, U})
$$
whose cohomology groups compute the homology of $\mathcal{F}$ on the fibre
category. We conclude by
Cohomology on Sites, Lemma
\ref{sites-cohomology-lemma-compute-left-derived-pi-shriek}
that the sheafification computes $L_n\pi_!(\mathcal{F})$.
In other words, the complex of sheaves whose term in degree $-n$ is
the sheafification of $U \mapsto \mathcal{F}(\mathbf{A}_{n, U})$ computes
$L\pi_!(\mathcal{F})$.
\end{remark}

\noindent
With this remark out of the way we can state the main
result of this section.

\begin{lemma}
\label{lemma-cotangent-morphism-schemes}
In the situation above there is a canonical isomorphism
$$
L_{X/\Lambda} = 
L\pi_!(Li^*\Omega_{\mathcal{O}/\underline{\Lambda}}) =
L\pi_!(i^*\Omega_{\mathcal{O}/\underline{\Lambda}}) =
L\pi_!(\Omega_{\mathcal{O}/\underline{\Lambda}}
\otimes_\mathcal{O} \underline{\mathcal{O}}_X)
$$
in $D(\mathcal{O}_X)$.
\end{lemma}

\begin{proof}
We first observe that for any object $(U \to \mathbf{A})$ of
$\mathcal{C}_{X/\Lambda}$
the value of the sheaf $\mathcal{O}$ is a polynomial algebra over $\Lambda$.
Hence $\Omega_{\mathcal{O}/\underline{\Lambda}}$ is a flat $\mathcal{O}$-module
and we conclude the second and third equalities of the statement of the
lemma hold.

\medskip\noindent
By Remark \ref{remark-compute-L-pi-shriek} the object
$L\pi_!(\Omega_{\mathcal{O}/\underline{\Lambda}}
\otimes_\mathcal{O} \underline{\mathcal{O}}_X)$
is computed as the sheafification of the complex of presheaves
$$
U \mapsto
\left(\Omega_{\mathcal{O}/\underline{\Lambda}}
\otimes_\mathcal{O} \underline{\mathcal{O}}_X\right)(\mathbf{A}_{\bullet, U})
=
\Omega_{P_{\bullet, U}/\Lambda} \otimes_{P_{\bullet, U}} \mathcal{O}_X(U) =
L_{\mathcal{O}_X(U)/\Lambda}
$$
using notation as in Remark \ref{remark-compute-L-pi-shriek}.
Now Remark \ref{remark-map-sections-over-U} shows that
$L\pi_!(\Omega_{\mathcal{O}/\underline{\Lambda}}
\otimes_\mathcal{O} \underline{\mathcal{O}}_X)$
computes the cotangent complex of the map of rings
$\underline{\Lambda} \to \mathcal{O}_X$ on $X$.
This is what we want by Lemma \ref{lemma-scheme-over-ring}.
\end{proof}








\section{The cotangent complex of a morphism of algebraic spaces}
\label{section-cotangent-morphism-spaces}

\noindent
We define the cotangent complex of a morphism of algebraic spaces
using the associated morphism between the small \'etale sites.

\begin{definition}
\label{definition-cotangent-morphism-spaces}
Let $S$ be a scheme. Let $f : X \to Y$ be a morphism of algebraic spaces
over $S$. The {\it cotangent complex $L_{X/Y}$ of $X$ over $Y$} is the
cotangent complex of the morphism of ringed topoi $f_{small}$
between the small \'etale sites of $X$ and $Y$
(see
Properties of Spaces, Lemma
\ref{spaces-properties-lemma-morphism-ringed-topoi}
and
Definition \ref{definition-cotangent-complex-morphism-ringed-topoi}).
\end{definition}

\noindent
In particular, the results of
Section \ref{section-cotangent-morphism-ringed-topoi} apply
to cotangent complexes of morphisms of algebraic spaces.
The next lemmas show this definition is compatible with the definition
for ring maps and for schemes and that $L_{X/Y}$ is an
object of $D_\QCoh(\mathcal{O}_X)$.

\begin{lemma}
\label{lemma-etale-localization}
Let $S$ be a scheme. Consider a commutative diagram
$$
\xymatrix{
U \ar[d]_p \ar[r]_g & V \ar[d]^q \\
X \ar[r]^f & Y
}
$$
of algebraic spaces over $S$ with $p$ and $q$ \'etale.
Then there is a canonical identification
$L_{X/Y}|_{U_\etale} = L_{U/V}$ in $D(\mathcal{O}_U)$.
\end{lemma}

\begin{proof}
Formation of the cotangent complex commutes with pullback
(Lemma \ref{lemma-pullback-cotangent-morphism-topoi}) and
we have $p_{small}^{-1}\mathcal{O}_X = \mathcal{O}_U$ and
$g_{small}^{-1}\mathcal{O}_{V_\etale} =
p_{small}^{-1}f_{small}^{-1}\mathcal{O}_{Y_\etale}$
because $q_{small}^{-1}\mathcal{O}_{Y_\etale} =
\mathcal{O}_{V_\etale}$
(Properties of Spaces, Lemma
\ref{spaces-properties-lemma-etale-exact-pullback}).
Tracing through the definitions we conclude that
$L_{X/Y}|_{U_\etale} = L_{U/V}$.
\end{proof}

\begin{lemma}
\label{lemma-compare-spaces-schemes}
Let $S$ be a scheme. Let $f : X \to Y$ be a morphism of algebraic spaces
over $S$. Assume $X$ and $Y$ representable by schemes $X_0$ and $Y_0$.
Then there is a canonical identification
$L_{X/Y} = \epsilon^*L_{X_0/Y_0}$ in $D(\mathcal{O}_X)$
where $\epsilon$ is as in Derived Categories of Spaces, Section
\ref{spaces-perfect-section-derived-quasi-coherent-etale}
and $L_{X_0/Y_0}$ is as in
 Definition \ref{definition-cotangent-morphism-schemes}.
\end{lemma}

\begin{proof}
Let $f_0 : X_0 \to Y_0$ be the morphism of schemes corresponding to $f$.
There is a canonical map
$\epsilon^{-1}f_0^{-1}\mathcal{O}_{Y_0} \to f_{small}^{-1}\mathcal{O}_Y$
compatible with
$\epsilon^\sharp : \epsilon^{-1}\mathcal{O}_{X_0} \to \mathcal{O}_X$
because there is a commutative diagram
$$
\xymatrix{
X_{0, Zar} \ar[d]_{f_0} & X_\etale \ar[l]^\epsilon \ar[d]^f \\
Y_{0, Zar} & Y_\etale \ar[l]_\epsilon
}
$$
see Derived Categories of Spaces, Remark
\ref{spaces-perfect-remark-match-total-direct-images}.
Thus we obtain a canonical map
$$
\epsilon^{-1}L_{X_0/Y_0} =
\epsilon^{-1}L_{\mathcal{O}_{X_0}/f_0^{-1}\mathcal{O}_{Y_0}} =
L_{\epsilon^{-1}\mathcal{O}_{X_0}/\epsilon^{-1}f_0^{-1}\mathcal{O}_{Y_0}}
\longrightarrow
L_{\mathcal{O}_X/f^{-1}_{small}\mathcal{O}_Y} = L_{X/Y}
$$
by the functoriality discussed in Section \ref{section-cotangent-complex}
and Lemma \ref{lemma-pullback-cotangent-morphism-topoi}.
To see that the induced map $\epsilon^*L_{X_0/Y_0} \to L_{X/Y}$ is an
isomorphism we may check on stalks at geometric points
(Properties of Spaces, Theorem
\ref{spaces-properties-theorem-exactness-stalks}).
We will use Lemma \ref{lemma-stalk-cotangent-complex}
to compute the stalks. Let $\overline{x} : \Spec(k) \to X_0$
be a geometric point lying over $x \in X_0$, with
$\overline{y} = f \circ \overline{x}$ lying over $y \in Y_0$. Then
$$
L_{X/Y, \overline{x}} =
L_{\mathcal{O}_{X, \overline{x}}/\mathcal{O}_{Y, \overline{y}}}
$$
and
$$
(\epsilon^*L_{X_0/Y_0})_{\overline{x}} =
L_{X_0/Y_0, x} \otimes_{\mathcal{O}_{X_0, x}}
\mathcal{O}_{X, \overline{x}} =
L_{\mathcal{O}_{X_0, x}/\mathcal{O}_{Y_0, y}}
\otimes_{\mathcal{O}_{X_0, x}} \mathcal{O}_{X, \overline{x}}
$$
Some details omitted (hint: use that the stalk of a pullback
is the stalk at the image point, see
Sites, Lemma \ref{sites-lemma-point-morphism-sites},
as well as the corresponding result for modules, see
Modules on Sites, Lemma \ref{sites-modules-lemma-pullback-stalk}).
Observe that $\mathcal{O}_{X, \overline{x}}$ is the strict
henselization of $\mathcal{O}_{X_0, x}$ and similarly
for $\mathcal{O}_{Y, \overline{y}}$
(Properties of Spaces, Lemma
\ref{spaces-properties-lemma-describe-etale-local-ring}).
Thus the result follows from
Lemma \ref{lemma-cotangent-complex-henselization}.
\end{proof}

\begin{lemma}
\label{lemma-space-over-ring}
Let $\Lambda$ be a ring. Let $X$ be an algebraic space over $\Lambda$.
Then
$$
L_{X/\Spec(\Lambda)} = L_{\mathcal{O}_X/\underline{\Lambda}}
$$
where $\underline{\Lambda}$ is the constant sheaf with value
$\Lambda$ on $X_\etale$.
\end{lemma}

\begin{proof}
Let $p : X \to \Spec(\Lambda)$ be the structure morphism.
Let $q : \Spec(\Lambda)_\etale \to (*, \Lambda)$
be the obvious morphism. By the distinguished triangle of
Lemma \ref{lemma-triangle-ringed-topoi}
it suffices to show that $L_q = 0$. To see this it suffices to
show
(Properties of Spaces, Theorem
\ref{spaces-properties-theorem-exactness-stalks})
for a geometric point $\overline{t} : \Spec(k) \to \Spec(\Lambda)$ that
$$
(L_q)_{\overline{t}} =
L_{\mathcal{O}_{\Spec(\Lambda)_\etale, \overline{t}}/\Lambda}
$$
(Lemma \ref{lemma-stalk-cotangent-complex})
is zero. Since $\mathcal{O}_{\Spec(\Lambda)_\etale, \overline{t}}$
is a strict henselization of a local ring of $\Lambda$
(Properties of Spaces, Lemma
\ref{spaces-properties-lemma-describe-etale-local-ring})
this follows from Lemma \ref{lemma-when-zero}.
\end{proof}










\section{The cotangent complex of an algebraic space over a ring}
\label{section-cotangent-spaces-variant}

\noindent
Let $\Lambda$ be a ring and let $X$ be an algebraic space over $\Lambda$.
Write $L_{X/\Spec(\Lambda)} = L_{X/\Lambda}$ which is justified
by Lemma \ref{lemma-space-over-ring}.
In this section we give a description of $L_{X/\Lambda}$ similar to
Lemma \ref{lemma-compute-cotangent-complex}.
Namely, we construct a category $\mathcal{C}_{X/\Lambda}$
fibred over $X_\etale$ and endow it with a sheaf of (polynomial)
$\Lambda$-algebras $\mathcal{O}$ such that
$$
L_{X/\Lambda} =
L\pi_!(\Omega_{\mathcal{O}/\underline{\Lambda}} \otimes_\mathcal{O}
\underline{\mathcal{O}}_X).
$$
We will later use the category $\mathcal{C}_{X/\Lambda}$ to construct
a naive obstruction theory for the stack of coherent sheaves.

\medskip\noindent
Let $\Lambda$ be a ring. Let $X$ be an algebraic space over $\Lambda$.
Let $\mathcal{C}_{X/\Lambda}$ be the category whose objects are
commutative diagrams
\begin{equation}
\label{equation-object-space}
\vcenter{
\xymatrix{
X \ar[d] & U \ar[l] \ar[d] \\
\Spec(\Lambda) & \mathbf{A} \ar[l]
}
}
\end{equation}
of schemes where
\begin{enumerate}
\item $U$ is a scheme,
\item $U \to X$ is \'etale,
\item there exists an isomorphism $\mathbf{A} = \Spec(P)$
where $P$ is a polynomial algebra over $\Lambda$ (on some set
of variables).
\end{enumerate}
In other words, $\mathbf{A}$ is an (infinite dimensional) affine space over
$\Spec(\Lambda)$. Morphisms are given by commutative diagrams.
Recall that $X_\etale$ denotes the small \'etale site of $X$
whose objects are schemes \'etale over $X$.
There is a forgetful functor
$$
u : \mathcal{C}_{X/\Lambda} \to X_\etale,
\quad
(U \to \mathbf{A}) \mapsto U
$$
Observe that the fibre category over $U$ is canonically equivalent
to the category $\mathcal{C}_{\mathcal{O}_X(U)/\Lambda}$ introduced
in Section \ref{section-compute-L-pi-shriek}.

\begin{lemma}
\label{lemma-category-fibred-space}
In the situation above the category
$\mathcal{C}_{X/\Lambda}$ is fibred over $X_\etale$.
\end{lemma}

\begin{proof}
Given an object $U \to \mathbf{A}$ of $\mathcal{C}_{X/\Lambda}$ and a morphism
$U' \to U$ of $X_\etale$ consider the object $U' \to \mathbf{A}$ of
$\mathcal{C}_{X/\Lambda}$ where $U' \to \mathbf{A}$ is the composition of
$U \to \mathbf{A}$
and $U' \to U$. The morphism $(U' \to \mathbf{A}) \to (U \to \mathbf{A})$ of
$\mathcal{C}_{X/\Lambda}$ is strongly cartesian over $X_\etale$.
\end{proof}

\noindent
We endow $\mathcal{C}_{X/\Lambda}$ with the topology inherited from
$X_\etale$ (see Stacks, Section \ref{stacks-section-topology}).
The functor $u$ defines a morphism of topoi
$\pi : \Sh(\mathcal{C}_{X/\Lambda}) \to \Sh(X_\etale)$.
The site $\mathcal{C}_{X/\Lambda}$ comes with several sheaves of rings.
\begin{enumerate}
\item The sheaf $\mathcal{O}$ given by the rule
$(U \to \mathbf{A}) \mapsto \Gamma(\mathbf{A}, \mathcal{O}_\mathbf{A})$.
\item The sheaf $\underline{\mathcal{O}}_X = \pi^{-1}\mathcal{O}_X$ given by
the rule $(U \to \mathbf{A}) \mapsto \mathcal{O}_X(U)$.
\item The constant sheaf $\underline{\Lambda}$.
\end{enumerate}
We obtain morphisms of ringed topoi
\begin{equation}
\label{equation-pi-spaces}
\vcenter{
\xymatrix{
(\Sh(\mathcal{C}_{X/\Lambda}), \underline{\mathcal{O}}_X) \ar[r]_i \ar[d]_\pi &
(\Sh(\mathcal{C}_{X/\Lambda}), \mathcal{O}) \\
(\Sh(X_\etale), \mathcal{O}_X)
}
}
\end{equation}
The morphism $i$ is the identity on underlying topoi and
$i^\sharp : \mathcal{O} \to \underline{\mathcal{O}}_X$
is the obvious map.
The map $\pi$ is a special case of Cohomology on Sites, Situation
\ref{sites-cohomology-situation-fibred-category}.
An important role will be played in the following
by the derived functors
$
Li^* : D(\mathcal{O}) \longrightarrow D(\underline{\mathcal{O}}_X)
$
left adjoint to $Ri_* = i_* : D(\underline{\mathcal{O}}_X) \to D(\mathcal{O})$
and
$
L\pi_! : D(\underline{\mathcal{O}}_X) \longrightarrow D(\mathcal{O}_X)
$
left adjoint to
$\pi^* = \pi^{-1} : D(\mathcal{O}_X) \to D(\underline{\mathcal{O}}_X)$.
We can compute $L\pi_!$ thanks to our earlier work.

\begin{remark}
\label{remark-compute-L-pi-shriek-spaces}
In the situation above, for every object $U \to X$ of $X_\etale$
let $P_{\bullet, U}$ be the standard resolution of $\mathcal{O}_X(U)$
over $\Lambda$. Set $\mathbf{A}_{n, U} = \Spec(P_{n, U})$.
Then $\mathbf{A}_{\bullet, U}$
is a cosimplicial object of the fibre category
$\mathcal{C}_{\mathcal{O}_X(U)/\Lambda}$ of
$\mathcal{C}_{X/\Lambda}$ over $U$. Moreover, as discussed
in Remark \ref{remark-resolution} we have that $\mathbf{A}_{\bullet, U}$
is a cosimplicial object of $\mathcal{C}_{\mathcal{O}_X(U)/\Lambda}$
as in Cohomology on Sites, Lemma
\ref{sites-cohomology-lemma-compute-by-cosimplicial-resolution}.
Since the construction $U \mapsto \mathbf{A}_{\bullet, U}$ is functorial
in $U$, given any (abelian) sheaf $\mathcal{F}$ on $\mathcal{C}_{X/\Lambda}$
we obtain a complex of presheaves
$$
U \longmapsto \mathcal{F}(\mathbf{A}_{\bullet, U})
$$
whose cohomology groups compute the homology of $\mathcal{F}$ on the fibre
category. We conclude by
Cohomology on Sites, Lemma
\ref{sites-cohomology-lemma-compute-left-derived-pi-shriek}
that the sheafification computes $L_n\pi_!(\mathcal{F})$.
In other words, the complex of sheaves whose term in degree $-n$ is
the sheafification of $U \mapsto \mathcal{F}(\mathbf{A}_{n, U})$ computes
$L\pi_!(\mathcal{F})$.
\end{remark}

\noindent
With this remark out of the way we can state the main
result of this section.

\begin{lemma}
\label{lemma-cotangent-morphism-spaces}
In the situation above there is a canonical isomorphism
$$
L_{X/\Lambda} = 
L\pi_!(Li^*\Omega_{\mathcal{O}/\underline{\Lambda}}) =
L\pi_!(i^*\Omega_{\mathcal{O}/\underline{\Lambda}}) =
L\pi_!(\Omega_{\mathcal{O}/\underline{\Lambda}}
\otimes_\mathcal{O} \underline{\mathcal{O}}_X)
$$
in $D(\mathcal{O}_X)$.
\end{lemma}

\begin{proof}
We first observe that for any object $(U \to \mathbf{A})$ of
$\mathcal{C}_{X/\Lambda}$
the value of the sheaf $\mathcal{O}$ is a polynomial algebra over $\Lambda$.
Hence $\Omega_{\mathcal{O}/\underline{\Lambda}}$ is a flat $\mathcal{O}$-module
and we conclude the second and third equalities of the statement of the
lemma hold.

\medskip\noindent
By Remark \ref{remark-compute-L-pi-shriek-spaces} the object
$L\pi_!(\Omega_{\mathcal{O}/\underline{\Lambda}}
\otimes_\mathcal{O} \underline{\mathcal{O}}_X)$
is computed as the sheafification of the complex of presheaves
$$
U \mapsto
\left(\Omega_{\mathcal{O}/\underline{\Lambda}}
\otimes_\mathcal{O} \underline{\mathcal{O}}_X\right)(\mathbf{A}_{\bullet, U})
=
\Omega_{P_{\bullet, U}/\Lambda} \otimes_{P_{\bullet, U}} \mathcal{O}_X(U) =
L_{\mathcal{O}_X(U)/\Lambda}
$$
using notation as in Remark \ref{remark-compute-L-pi-shriek-spaces}.
Now Remark \ref{remark-map-sections-over-U} shows that
$L\pi_!(\Omega_{\mathcal{O}/\underline{\Lambda}}
\otimes_\mathcal{O} \underline{\mathcal{O}}_X)$
computes the cotangent complex of the map of rings
$\underline{\Lambda} \to \mathcal{O}_X$ on $X_\etale$.
This is what we want by Lemma \ref{lemma-space-over-ring}.
\end{proof}






\section{Fibre products of algebraic spaces and the cotangent complex}
\label{section-fibre-product}

\noindent
Let $S$ be a scheme. Let $X \to B$ and $Y \to B$ be morphisms of algebraic
spaces over $S$. Consider the fibre product $X \times_B Y$ with projection
morphisms $p : X \times_B Y \to X$ and $q : X \times_B Y \to Y$.
In this section we discuss $L_{X \times_B Y/B}$. Most of the
information we want is contained in the following diagram
\begin{equation}
\label{equation-fibre-product}
\vcenter{
\xymatrix{
Lp^*L_{X/B} \ar[r] &
L_{X \times_B Y/Y} \ar[r] &
E \\
Lp^*L_{X/B} \ar[r] \ar@{=}[u] &
L_{X \times_B Y/B} \ar[r] \ar[u] &
L_{X \times_B Y/X} \ar[u] \\
 &
Lq^*L_{Y/B} \ar[u] \ar@{=}[r] &
Lq^*L_{Y/B} \ar[u]
}
}
\end{equation}
Explanation: The middle row is the fundamental triangle of
Lemma \ref{lemma-triangle-ringed-topoi} for the morphisms
$X \times_B Y \to X \to B$. The middle column is the fundamental triangle
for the morphisms $X \times_B Y \to Y \to B$.
Next, $E$ is an object of $D(\mathcal{O}_{X \times_B Y})$ which ``fits''
into the upper right corner, i.e., which turns both the top row
and the right column into distinguished triangles. Such an $E$
exists by Derived Categories, Proposition \ref{derived-proposition-9}
applied to the lower left square (with $0$ placed in the missing
spot). To be more explicit, we could for example define $E$ as the cone
(Derived Categories, Definition \ref{derived-definition-cone})
of the map of complexes
$$
Lp^*L_{X/B} \oplus Lq^*L_{Y/B} \longrightarrow L_{X \times_B Y/B}
$$
and get the two maps with target $E$ by an application of TR3.
In the Tor independent case the object $E$ is zero.

\begin{lemma}
\label{lemma-fibre-product-tor-independent}
In the situation above, if $X$ and $Y$ are Tor independent over $B$, then
the object $E$ in (\ref{equation-fibre-product}) is zero. In this case we
have
$$
L_{X \times_B Y/B} = Lp^*L_{X/B} \oplus Lq^*L_{Y/B}
$$
\end{lemma}

\begin{proof}
Choose a scheme $W$ and a surjective \'etale morphism $W \to B$.
Choose a scheme $U$ and a surjective \'etale morphism $U \to X \times_B W$.
Choose a scheme $V$ and a surjective \'etale morphism $V \to Y \times_B W$.
Then $U \times_W V \to X \times_B Y$ is surjective \'etale too.
Hence it suffices to prove that the restriction of $E$ to $U \times_W V$
is zero. By Lemma \ref{lemma-compare-spaces-schemes} and
Derived Categories of Spaces, Lemma \ref{spaces-perfect-lemma-tor-independent}
this reduces us to the case of schemes.
Taking suitable affine opens we reduce to the case of affine schemes.
Using 
Lemma \ref{lemma-morphism-affine-schemes}
we reduce to the case of a tensor product of rings, i.e., to
Lemma \ref{lemma-tensor-product-tor-independent}.
\end{proof}

\noindent
In general we can say the following about the object $E$.

\begin{lemma}
\label{lemma-fibre-product}
Let $S$ be a scheme. Let $X \to B$ and $Y \to B$ be morphisms of algebraic
spaces over $S$. The object $E$ in (\ref{equation-fibre-product}) satisfies
$H^i(E) = 0$ for $i = 0, -1$ and for a geometric point
$(\overline{x}, \overline{y}) : \Spec(k) \to X \times_B Y$ we have
$$
H^{-2}(E)_{(\overline{x}, \overline{y})} =
\text{Tor}_1^R(A, B) \otimes_{A \otimes_R B} C
$$
where $R = \mathcal{O}_{B, \overline{b}}$, $A = \mathcal{O}_{X, \overline{x}}$,
$B = \mathcal{O}_{Y, \overline{y}}$, and
$C = \mathcal{O}_{X \times_B Y, (\overline{x}, \overline{y})}$.
\end{lemma}

\begin{proof}
The formation of the cotangent complex commutes with taking stalks
and pullbacks, see
Lemmas \ref{lemma-stalk-cotangent-complex} and
\ref{lemma-pullback-cotangent-morphism-topoi}.
Note that $C$ is a henselization of $A \otimes_R B$.
$L_{C/R} = L_{A \otimes_R B/R} \otimes_{A \otimes_R B} C$
by the results of Section \ref{section-localization}.
Thus the stalk of $E$ at our geometric point is the cone of the
map $L_{A/R} \otimes C \to L_{A \otimes_R B/R} \otimes C$.
Therefore the results of the lemma follow from
the case of rings, i.e., Lemma \ref{lemma-tensor-product}.
\end{proof}








\begin{multicols}{2}[\section{Other chapters}]
\noindent
Preliminaries
\begin{enumerate}
\item \hyperref[introduction-section-phantom]{Introduction}
\item \hyperref[conventions-section-phantom]{Conventions}
\item \hyperref[sets-section-phantom]{Set Theory}
\item \hyperref[categories-section-phantom]{Categories}
\item \hyperref[topology-section-phantom]{Topology}
\item \hyperref[sheaves-section-phantom]{Sheaves on Spaces}
\item \hyperref[sites-section-phantom]{Sites and Sheaves}
\item \hyperref[stacks-section-phantom]{Stacks}
\item \hyperref[fields-section-phantom]{Fields}
\item \hyperref[algebra-section-phantom]{Commutative Algebra}
\item \hyperref[brauer-section-phantom]{Brauer Groups}
\item \hyperref[homology-section-phantom]{Homological Algebra}
\item \hyperref[derived-section-phantom]{Derived Categories}
\item \hyperref[simplicial-section-phantom]{Simplicial Methods}
\item \hyperref[more-algebra-section-phantom]{More on Algebra}
\item \hyperref[smoothing-section-phantom]{Smoothing Ring Maps}
\item \hyperref[modules-section-phantom]{Sheaves of Modules}
\item \hyperref[sites-modules-section-phantom]{Modules on Sites}
\item \hyperref[injectives-section-phantom]{Injectives}
\item \hyperref[cohomology-section-phantom]{Cohomology of Sheaves}
\item \hyperref[sites-cohomology-section-phantom]{Cohomology on Sites}
\item \hyperref[dga-section-phantom]{Differential Graded Algebra}
\item \hyperref[dpa-section-phantom]{Divided Power Algebra}
\item \hyperref[sdga-section-phantom]{Differential Graded Sheaves}
\item \hyperref[hypercovering-section-phantom]{Hypercoverings}
\end{enumerate}
Schemes
\begin{enumerate}
\setcounter{enumi}{25}
\item \hyperref[schemes-section-phantom]{Schemes}
\item \hyperref[constructions-section-phantom]{Constructions of Schemes}
\item \hyperref[properties-section-phantom]{Properties of Schemes}
\item \hyperref[morphisms-section-phantom]{Morphisms of Schemes}
\item \hyperref[coherent-section-phantom]{Cohomology of Schemes}
\item \hyperref[divisors-section-phantom]{Divisors}
\item \hyperref[limits-section-phantom]{Limits of Schemes}
\item \hyperref[varieties-section-phantom]{Varieties}
\item \hyperref[topologies-section-phantom]{Topologies on Schemes}
\item \hyperref[descent-section-phantom]{Descent}
\item \hyperref[perfect-section-phantom]{Derived Categories of Schemes}
\item \hyperref[more-morphisms-section-phantom]{More on Morphisms}
\item \hyperref[flat-section-phantom]{More on Flatness}
\item \hyperref[groupoids-section-phantom]{Groupoid Schemes}
\item \hyperref[more-groupoids-section-phantom]{More on Groupoid Schemes}
\item \hyperref[etale-section-phantom]{\'Etale Morphisms of Schemes}
\end{enumerate}
Topics in Scheme Theory
\begin{enumerate}
\setcounter{enumi}{41}
\item \hyperref[chow-section-phantom]{Chow Homology}
\item \hyperref[intersection-section-phantom]{Intersection Theory}
\item \hyperref[pic-section-phantom]{Picard Schemes of Curves}
\item \hyperref[weil-section-phantom]{Weil Cohomology Theories}
\item \hyperref[adequate-section-phantom]{Adequate Modules}
\item \hyperref[dualizing-section-phantom]{Dualizing Complexes}
\item \hyperref[duality-section-phantom]{Duality for Schemes}
\item \hyperref[discriminant-section-phantom]{Discriminants and Differents}
\item \hyperref[derham-section-phantom]{de Rham Cohomology}
\item \hyperref[local-cohomology-section-phantom]{Local Cohomology}
\item \hyperref[algebraization-section-phantom]{Algebraic and Formal Geometry}
\item \hyperref[curves-section-phantom]{Algebraic Curves}
\item \hyperref[resolve-section-phantom]{Resolution of Surfaces}
\item \hyperref[models-section-phantom]{Semistable Reduction}
\item \hyperref[functors-section-phantom]{Functors and Morphisms}
\item \hyperref[equiv-section-phantom]{Derived Categories of Varieties}
\item \hyperref[pione-section-phantom]{Fundamental Groups of Schemes}
\item \hyperref[etale-cohomology-section-phantom]{\'Etale Cohomology}
\item \hyperref[crystalline-section-phantom]{Crystalline Cohomology}
\item \hyperref[proetale-section-phantom]{Pro-\'etale Cohomology}
\item \hyperref[relative-cycles-section-phantom]{Relative Cycles}
\item \hyperref[more-etale-section-phantom]{More \'Etale Cohomology}
\item \hyperref[trace-section-phantom]{The Trace Formula}
\end{enumerate}
Algebraic Spaces
\begin{enumerate}
\setcounter{enumi}{64}
\item \hyperref[spaces-section-phantom]{Algebraic Spaces}
\item \hyperref[spaces-properties-section-phantom]{Properties of Algebraic Spaces}
\item \hyperref[spaces-morphisms-section-phantom]{Morphisms of Algebraic Spaces}
\item \hyperref[decent-spaces-section-phantom]{Decent Algebraic Spaces}
\item \hyperref[spaces-cohomology-section-phantom]{Cohomology of Algebraic Spaces}
\item \hyperref[spaces-limits-section-phantom]{Limits of Algebraic Spaces}
\item \hyperref[spaces-divisors-section-phantom]{Divisors on Algebraic Spaces}
\item \hyperref[spaces-over-fields-section-phantom]{Algebraic Spaces over Fields}
\item \hyperref[spaces-topologies-section-phantom]{Topologies on Algebraic Spaces}
\item \hyperref[spaces-descent-section-phantom]{Descent and Algebraic Spaces}
\item \hyperref[spaces-perfect-section-phantom]{Derived Categories of Spaces}
\item \hyperref[spaces-more-morphisms-section-phantom]{More on Morphisms of Spaces}
\item \hyperref[spaces-flat-section-phantom]{Flatness on Algebraic Spaces}
\item \hyperref[spaces-groupoids-section-phantom]{Groupoids in Algebraic Spaces}
\item \hyperref[spaces-more-groupoids-section-phantom]{More on Groupoids in Spaces}
\item \hyperref[bootstrap-section-phantom]{Bootstrap}
\item \hyperref[spaces-pushouts-section-phantom]{Pushouts of Algebraic Spaces}
\end{enumerate}
Topics in Geometry
\begin{enumerate}
\setcounter{enumi}{81}
\item \hyperref[spaces-chow-section-phantom]{Chow Groups of Spaces}
\item \hyperref[groupoids-quotients-section-phantom]{Quotients of Groupoids}
\item \hyperref[spaces-more-cohomology-section-phantom]{More on Cohomology of Spaces}
\item \hyperref[spaces-simplicial-section-phantom]{Simplicial Spaces}
\item \hyperref[spaces-duality-section-phantom]{Duality for Spaces}
\item \hyperref[formal-spaces-section-phantom]{Formal Algebraic Spaces}
\item \hyperref[restricted-section-phantom]{Algebraization of Formal Spaces}
\item \hyperref[spaces-resolve-section-phantom]{Resolution of Surfaces Revisited}
\end{enumerate}
Deformation Theory
\begin{enumerate}
\setcounter{enumi}{89}
\item \hyperref[formal-defos-section-phantom]{Formal Deformation Theory}
\item \hyperref[defos-section-phantom]{Deformation Theory}
\item \hyperref[cotangent-section-phantom]{The Cotangent Complex}
\item \hyperref[examples-defos-section-phantom]{Deformation Problems}
\end{enumerate}
Algebraic Stacks
\begin{enumerate}
\setcounter{enumi}{93}
\item \hyperref[algebraic-section-phantom]{Algebraic Stacks}
\item \hyperref[examples-stacks-section-phantom]{Examples of Stacks}
\item \hyperref[stacks-sheaves-section-phantom]{Sheaves on Algebraic Stacks}
\item \hyperref[criteria-section-phantom]{Criteria for Representability}
\item \hyperref[artin-section-phantom]{Artin's Axioms}
\item \hyperref[quot-section-phantom]{Quot and Hilbert Spaces}
\item \hyperref[stacks-properties-section-phantom]{Properties of Algebraic Stacks}
\item \hyperref[stacks-morphisms-section-phantom]{Morphisms of Algebraic Stacks}
\item \hyperref[stacks-limits-section-phantom]{Limits of Algebraic Stacks}
\item \hyperref[stacks-cohomology-section-phantom]{Cohomology of Algebraic Stacks}
\item \hyperref[stacks-perfect-section-phantom]{Derived Categories of Stacks}
\item \hyperref[stacks-introduction-section-phantom]{Introducing Algebraic Stacks}
\item \hyperref[stacks-more-morphisms-section-phantom]{More on Morphisms of Stacks}
\item \hyperref[stacks-geometry-section-phantom]{The Geometry of Stacks}
\end{enumerate}
Topics in Moduli Theory
\begin{enumerate}
\setcounter{enumi}{107}
\item \hyperref[moduli-section-phantom]{Moduli Stacks}
\item \hyperref[moduli-curves-section-phantom]{Moduli of Curves}
\end{enumerate}
Miscellany
\begin{enumerate}
\setcounter{enumi}{109}
\item \hyperref[examples-section-phantom]{Examples}
\item \hyperref[exercises-section-phantom]{Exercises}
\item \hyperref[guide-section-phantom]{Guide to Literature}
\item \hyperref[desirables-section-phantom]{Desirables}
\item \hyperref[coding-section-phantom]{Coding Style}
\item \hyperref[obsolete-section-phantom]{Obsolete}
\item \hyperref[fdl-section-phantom]{GNU Free Documentation License}
\item \hyperref[index-section-phantom]{Auto Generated Index}
\end{enumerate}
\end{multicols}


\bibliography{my}
\bibliographystyle{amsalpha}

\end{document}
