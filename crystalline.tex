\IfFileExists{stacks-project.cls}{%
\documentclass{stacks-project}
}{%
\documentclass{amsart}
}

% For dealing with references we use the comment environment
\usepackage{verbatim}
\newenvironment{reference}{\comment}{\endcomment}
%\newenvironment{reference}{}{}
\newenvironment{slogan}{\comment}{\endcomment}
\newenvironment{history}{\comment}{\endcomment}

% For commutative diagrams we use Xy-pic
\usepackage[all]{xy}

% We use 2cell for 2-commutative diagrams.
\xyoption{2cell}
\UseAllTwocells

% We use multicol for the list of chapters between chapters
\usepackage{multicol}

% This is generall recommended for better output
\usepackage{lmodern}
\usepackage[T1]{fontenc}

% For cross-file-references
\usepackage{xr-hyper}

% Package for hypertext links:
\usepackage{hyperref}

% For any local file, say "hello.tex" you want to link to please
% use \externaldocument[hello-]{hello}
\externaldocument[introduction-]{introduction}
\externaldocument[conventions-]{conventions}
\externaldocument[sets-]{sets}
\externaldocument[categories-]{categories}
\externaldocument[topology-]{topology}
\externaldocument[sheaves-]{sheaves}
\externaldocument[sites-]{sites}
\externaldocument[stacks-]{stacks}
\externaldocument[fields-]{fields}
\externaldocument[algebra-]{algebra}
\externaldocument[brauer-]{brauer}
\externaldocument[homology-]{homology}
\externaldocument[derived-]{derived}
\externaldocument[simplicial-]{simplicial}
\externaldocument[more-algebra-]{more-algebra}
\externaldocument[smoothing-]{smoothing}
\externaldocument[modules-]{modules}
\externaldocument[sites-modules-]{sites-modules}
\externaldocument[injectives-]{injectives}
\externaldocument[cohomology-]{cohomology}
\externaldocument[sites-cohomology-]{sites-cohomology}
\externaldocument[dga-]{dga}
\externaldocument[dpa-]{dpa}
\externaldocument[sdga-]{sdga}
\externaldocument[hypercovering-]{hypercovering}
\externaldocument[schemes-]{schemes}
\externaldocument[constructions-]{constructions}
\externaldocument[properties-]{properties}
\externaldocument[morphisms-]{morphisms}
\externaldocument[coherent-]{coherent}
\externaldocument[divisors-]{divisors}
\externaldocument[limits-]{limits}
\externaldocument[varieties-]{varieties}
\externaldocument[topologies-]{topologies}
\externaldocument[descent-]{descent}
\externaldocument[perfect-]{perfect}
\externaldocument[more-morphisms-]{more-morphisms}
\externaldocument[flat-]{flat}
\externaldocument[groupoids-]{groupoids}
\externaldocument[more-groupoids-]{more-groupoids}
\externaldocument[etale-]{etale}
\externaldocument[chow-]{chow}
\externaldocument[intersection-]{intersection}
\externaldocument[pic-]{pic}
\externaldocument[weil-]{weil}
\externaldocument[adequate-]{adequate}
\externaldocument[dualizing-]{dualizing}
\externaldocument[duality-]{duality}
\externaldocument[discriminant-]{discriminant}
\externaldocument[derham-]{derham}
\externaldocument[local-cohomology-]{local-cohomology}
\externaldocument[algebraization-]{algebraization}
\externaldocument[curves-]{curves}
\externaldocument[resolve-]{resolve}
\externaldocument[models-]{models}
\externaldocument[functors-]{functors}
\externaldocument[equiv-]{equiv}
\externaldocument[pione-]{pione}
\externaldocument[etale-cohomology-]{etale-cohomology}
\externaldocument[proetale-]{proetale}
\externaldocument[relative-cycles-]{relative-cycles}
\externaldocument[more-etale-]{more-etale}
\externaldocument[trace-]{trace}
\externaldocument[crystalline-]{crystalline}
\externaldocument[spaces-]{spaces}
\externaldocument[spaces-properties-]{spaces-properties}
\externaldocument[spaces-morphisms-]{spaces-morphisms}
\externaldocument[decent-spaces-]{decent-spaces}
\externaldocument[spaces-cohomology-]{spaces-cohomology}
\externaldocument[spaces-limits-]{spaces-limits}
\externaldocument[spaces-divisors-]{spaces-divisors}
\externaldocument[spaces-over-fields-]{spaces-over-fields}
\externaldocument[spaces-topologies-]{spaces-topologies}
\externaldocument[spaces-descent-]{spaces-descent}
\externaldocument[spaces-perfect-]{spaces-perfect}
\externaldocument[spaces-more-morphisms-]{spaces-more-morphisms}
\externaldocument[spaces-flat-]{spaces-flat}
\externaldocument[spaces-groupoids-]{spaces-groupoids}
\externaldocument[spaces-more-groupoids-]{spaces-more-groupoids}
\externaldocument[bootstrap-]{bootstrap}
\externaldocument[spaces-pushouts-]{spaces-pushouts}
\externaldocument[spaces-chow-]{spaces-chow}
\externaldocument[groupoids-quotients-]{groupoids-quotients}
\externaldocument[spaces-more-cohomology-]{spaces-more-cohomology}
\externaldocument[spaces-simplicial-]{spaces-simplicial}
\externaldocument[spaces-duality-]{spaces-duality}
\externaldocument[formal-spaces-]{formal-spaces}
\externaldocument[restricted-]{restricted}
\externaldocument[spaces-resolve-]{spaces-resolve}
\externaldocument[formal-defos-]{formal-defos}
\externaldocument[defos-]{defos}
\externaldocument[cotangent-]{cotangent}
\externaldocument[examples-defos-]{examples-defos}
\externaldocument[algebraic-]{algebraic}
\externaldocument[examples-stacks-]{examples-stacks}
\externaldocument[stacks-sheaves-]{stacks-sheaves}
\externaldocument[criteria-]{criteria}
\externaldocument[artin-]{artin}
\externaldocument[quot-]{quot}
\externaldocument[stacks-properties-]{stacks-properties}
\externaldocument[stacks-morphisms-]{stacks-morphisms}
\externaldocument[stacks-limits-]{stacks-limits}
\externaldocument[stacks-cohomology-]{stacks-cohomology}
\externaldocument[stacks-perfect-]{stacks-perfect}
\externaldocument[stacks-introduction-]{stacks-introduction}
\externaldocument[stacks-more-morphisms-]{stacks-more-morphisms}
\externaldocument[stacks-geometry-]{stacks-geometry}
\externaldocument[moduli-]{moduli}
\externaldocument[moduli-curves-]{moduli-curves}
\externaldocument[examples-]{examples}
\externaldocument[exercises-]{exercises}
\externaldocument[guide-]{guide}
\externaldocument[desirables-]{desirables}
\externaldocument[coding-]{coding}
\externaldocument[obsolete-]{obsolete}
\externaldocument[fdl-]{fdl}
\externaldocument[index-]{index}

% Theorem environments.
%
\theoremstyle{plain}
\newtheorem{theorem}[subsection]{Theorem}
\newtheorem{proposition}[subsection]{Proposition}
\newtheorem{lemma}[subsection]{Lemma}

\theoremstyle{definition}
\newtheorem{definition}[subsection]{Definition}
\newtheorem{example}[subsection]{Example}
\newtheorem{exercise}[subsection]{Exercise}
\newtheorem{situation}[subsection]{Situation}

\theoremstyle{remark}
\newtheorem{remark}[subsection]{Remark}
\newtheorem{remarks}[subsection]{Remarks}

\numberwithin{equation}{subsection}

% Macros
%
\def\lim{\mathop{\mathrm{lim}}\nolimits}
\def\colim{\mathop{\mathrm{colim}}\nolimits}
\def\Spec{\mathop{\mathrm{Spec}}}
\def\Hom{\mathop{\mathrm{Hom}}\nolimits}
\def\Ext{\mathop{\mathrm{Ext}}\nolimits}
\def\SheafHom{\mathop{\mathcal{H}\!\mathit{om}}\nolimits}
\def\SheafExt{\mathop{\mathcal{E}\!\mathit{xt}}\nolimits}
\def\Sch{\mathit{Sch}}
\def\Mor{\mathop{\mathrm{Mor}}\nolimits}
\def\Ob{\mathop{\mathrm{Ob}}\nolimits}
\def\Sh{\mathop{\mathit{Sh}}\nolimits}
\def\NL{\mathop{N\!L}\nolimits}
\def\CH{\mathop{\mathrm{CH}}\nolimits}
\def\proetale{{pro\text{-}\acute{e}tale}}
\def\etale{{\acute{e}tale}}
\def\QCoh{\mathit{QCoh}}
\def\Ker{\mathop{\mathrm{Ker}}}
\def\Im{\mathop{\mathrm{Im}}}
\def\Coker{\mathop{\mathrm{Coker}}}
\def\Coim{\mathop{\mathrm{Coim}}}

% Boxtimes
%
\DeclareMathSymbol{\boxtimes}{\mathbin}{AMSa}{"02}

%
% Macros for moduli stacks/spaces
%
\def\QCohstack{\mathcal{QC}\!\mathit{oh}}
\def\Cohstack{\mathcal{C}\!\mathit{oh}}
\def\Spacesstack{\mathcal{S}\!\mathit{paces}}
\def\Quotfunctor{\mathrm{Quot}}
\def\Hilbfunctor{\mathrm{Hilb}}
\def\Curvesstack{\mathcal{C}\!\mathit{urves}}
\def\Polarizedstack{\mathcal{P}\!\mathit{olarized}}
\def\Complexesstack{\mathcal{C}\!\mathit{omplexes}}
% \Pic is the operator that assigns to X its picard group, usage \Pic(X)
% \Picardstack_{X/B} denotes the Picard stack of X over B
% \Picardfunctor_{X/B} denotes the Picard functor of X over B
\def\Pic{\mathop{\mathrm{Pic}}\nolimits}
\def\Picardstack{\mathcal{P}\!\mathit{ic}}
\def\Picardfunctor{\mathrm{Pic}}
\def\Deformationcategory{\mathcal{D}\!\mathit{ef}}


% OK, start here.
%
\begin{document}

\title{Crystalline Cohomology}


\maketitle

\phantomsection
\label{section-phantom}

\tableofcontents



\section{Introduction}
\label{section-introduction}

\noindent
This chapter is based on a lecture series given by Johan de Jong
held in 2012 at Columbia University.
The goals of this chapter are to give a quick introduction to
crystalline cohomology. A reference is the book \cite{Berthelot}.

\medskip\noindent
We have moved the more elementary purely algebraic discussion of divided
power rings to a preliminary chapter as it is also useful
in discussing Tate resolutions in commutative algebra.
Please see Divided Power Algebra, Section \ref{dpa-section-introduction}.













\section{Divided power envelope}
\label{section-divided-power-envelope}

\noindent
The construction of the following lemma will be dubbed the
divided power envelope. It will play an important role later.

\begin{lemma}
\label{lemma-divided-power-envelope}
Let $(A, I, \gamma)$ be a divided power ring.
Let $A \to B$ be a ring map. Let $J \subset B$ be an ideal
with $IB \subset J$. There exists a homomorphism of
divided power rings
$$
(A, I, \gamma) \longrightarrow (D, \bar J, \bar \gamma)
$$
such that
$$
\Hom_{(A, I, \gamma)}((D, \bar J, \bar \gamma), (C, K, \delta)) =
\Hom_{(A, I)}((B, J), (C, K))
$$
functorially in the divided power algebra $(C, K, \delta)$ over
$(A, I, \gamma)$. Here the LHS is morphisms of divided
power rings over $(A, I, \gamma)$ and the RHS is morphisms of
(ring, ideal) pairs over $(A, I)$.
\end{lemma}

\begin{proof}
Denote $\mathcal{C}$ the category of divided power rings
$(C, K, \delta)$. Consider the functor
$F : \mathcal{C} \longrightarrow \textit{Sets}$ defined by
$$
F(C, K, \delta) =
\left\{
(\varphi, \psi)
\middle|
\begin{matrix}
\varphi : (A, I, \gamma) \to (C, K, \delta)
\text{ homomorphism of divided power rings} \\
\psi : (B, J) \to (C, K)\text{ an }
A\text{-algebra homomorphism with }\psi(J) \subset K
\end{matrix}
\right\}
$$
We will show that
Divided Power Algebra, Lemma \ref{dpa-lemma-a-version-of-brown}
applies to this functor which will
prove the lemma. Suppose that $(\varphi, \psi) \in F(C, K, \delta)$.
Let $C' \subset C$ be the subring generated by $\varphi(A)$,
$\psi(B)$, and $\delta_n(\psi(f))$ for all $f \in J$.
Let $K' \subset K \cap C'$ be the ideal of $C'$ generated by
$\varphi(I)$ and $\delta_n(\psi(f))$ for $f \in J$.
Then $(C', K', \delta|_{K'})$ is a divided power ring and
$C'$ has cardinality bounded by the cardinal
$\kappa = |A| \otimes |B|^{\aleph_0}$.
Moreover, $\varphi$ factors as $A \to C' \to C$ and $\psi$ factors
as $B \to C' \to C$. This proves assumption (1) of
Divided Power Algebra, Lemma \ref{dpa-lemma-a-version-of-brown}
holds. Assumption (2) is clear
as limits in the category of divided power rings commute with the
forgetful functor $(C, K, \delta) \mapsto (C, K)$, see
Divided Power Algebra, Lemma \ref{dpa-lemma-limits} and its proof.
\end{proof}

\begin{definition}
\label{definition-divided-power-envelope}
Let $(A, I, \gamma)$ be a divided power ring.
Let $A \to B$ be a ring map. Let $J \subset B$ be an ideal
with $IB \subset J$. The divided power algebra $(D, \bar J, \bar\gamma)$
constructed in Lemma \ref{lemma-divided-power-envelope}
is called the {\it divided power envelope of $J$ in $B$
relative to $(A, I, \gamma)$} and is denoted $D_B(J)$ or $D_{B, \gamma}(J)$.
\end{definition}

\noindent
Let $(A, I, \gamma) \to (C, K, \delta)$ be a homomorphism of divided
power rings. The universal property of
$D_{B, \gamma}(J) = (D, \bar J, \bar \gamma)$ is
$$
\begin{matrix}
\text{ring maps }B \to C \\
\text{ which map }J\text{ into }K
\end{matrix}
\longleftrightarrow
\begin{matrix}
\text{divided power homomorphisms} \\
(D, \bar J, \bar \gamma) \to (C, K, \delta)
\end{matrix}
$$
and the correspondence is given by precomposing with the map $B \to D$
which corresponds to $\text{id}_D$. Here are some properties of
$(D, \bar J, \bar \gamma)$ which follow directly from the universal
property. There are $A$-algebra maps
\begin{equation}
\label{equation-divided-power-envelope}
B \longrightarrow D \longrightarrow B/J
\end{equation}
The first arrow maps $J$ into $\bar J$ and $\bar J$ is the kernel
of the second arrow. The elements $\bar\gamma_n(x)$ where $n > 0$
and $x$ is an element in the image of $J \to D$ generate $\bar J$
as an ideal in $D$ and generate $D$ as a $B$-algebra.

\begin{lemma}
\label{lemma-divided-power-envelop-quotient}
Let $(A, I, \gamma)$ be a divided power ring.
Let $\varphi : B' \to B$ be a surjection of $A$-algebras with kernel $K$.
Let $IB \subset J \subset B$ be an ideal. Let $J' \subset B'$
be the inverse image of $J$. Write
$D_{B', \gamma}(J') = (D', \bar J', \bar\gamma)$.
Then $D_{B, \gamma}(J) = (D'/K', \bar J'/K', \bar\gamma)$
where $K'$ is the ideal generated by the elements $\bar\gamma_n(k)$
for $n \geq 1$ and $k \in K$.
\end{lemma}

\begin{proof}
Write $D_{B, \gamma}(J) = (D, \bar J, \bar \gamma)$.
The universal property of $D'$ gives us a homomorphism $D' \to D$
of divided power algebras. As $B' \to B$ and $J' \to J$ are surjective, we
see that $D' \to D$ is surjective (see remarks above). It is clear that
$\bar\gamma_n(k)$ is in the kernel for $n \geq 1$ and $k \in K$, i.e.,
we obtain a homomorphism $D'/K' \to D$. Conversely, there exists a divided
power structure on $\bar J'/K' \subset D'/K'$, see
Divided Power Algebra, Lemma \ref{dpa-lemma-kernel}.
Hence the universal property of $D$ gives an inverse $D \to D'/K'$ and we win.
\end{proof}

\noindent
In the situation of Definition \ref{definition-divided-power-envelope}
we can choose a surjection $P \to B$ where $P$ is a polynomial
algebra over $A$ and let $J' \subset P$ be the inverse image of $J$.
The previous lemma describes $D_{B, \gamma}(J)$ in terms of
$D_{P, \gamma}(J')$. Note that $\gamma$ extends to a divided power
structure $\gamma'$ on $IP$ by
Divided Power Algebra, Lemma \ref{dpa-lemma-gamma-extends}. Hence
$D_{P, \gamma}(J') = D_{P, \gamma'}(J')$ is an example of a special
case of divided power envelopes we describe in the following lemma.

\begin{lemma}
\label{lemma-describe-divided-power-envelope}
Let $(B, I, \gamma)$ be a divided power algebra. Let $I \subset J \subset B$
be an ideal. Let $(D, \bar J, \bar \gamma)$ be the divided power envelope
of $J$ relative to $\gamma$. Choose elements $f_t \in J$, $t \in T$ such
that $J = I + (f_t)$. Then there exists a surjection
$$
\Psi : B\langle x_t \rangle \longrightarrow D
$$
of divided power rings mapping $x_t$ to the image of $f_t$ in $D$.
The kernel of $\Psi$ is generated by the elements $x_t - f_t$ and
all
$$
\delta_n\left(\sum r_t x_t - r_0\right)
$$
whenever $\sum r_t f_t = r_0$ in $B$ for some $r_t \in B$, $r_0 \in I$.
\end{lemma}

\begin{proof}
In the statement of the lemma we think of $B\langle x_t \rangle$
as a divided power ring with ideal
$J' = IB\langle x_t \rangle + B\langle x_t \rangle_{+}$, see
Divided Power Algebra, Remark \ref{dpa-remark-divided-power-polynomial-algebra}.
The existence of $\Psi$ follows from the universal property of
divided power polynomial rings. Surjectivity of $\Psi$ follows from
the fact that its image is a divided power subring of $D$, hence equal to $D$
by the universal property of $D$. It is clear that
$x_t - f_t$ is in the kernel. Set
$$
\mathcal{R} = \{(r_0, r_t) \in I \oplus \bigoplus\nolimits_{t \in T} B
\mid \sum r_t f_t = r_0 \text{ in }B\}
$$
If $(r_0, r_t) \in \mathcal{R}$ then it is clear that
$\sum r_t x_t - r_0$ is in the kernel.
As $\Psi$ is a homomorphism of divided power rings
and $\sum r_tx_t - r_0 \in J'$
it follows that $\delta_n(\sum r_t x_t - r_0)$ is in the kernel as well.
Let $K \subset B\langle x_t \rangle$ be the ideal generated by
$x_t - f_t$ and the elements $\delta_n(\sum r_t x_t - r_0)$ for
$(r_0, r_t) \in \mathcal{R}$.
To show that $K = \Ker(\Psi)$ it suffices to show that
$\delta$ extends to $B\langle x_t \rangle/K$. Namely, if so the universal
property of $D$ gives a map $D \to B\langle x_t \rangle/K$
inverse to $\Psi$. Hence we have to show that $K \cap J'$ is
preserved by $\delta_n$, see
Divided Power Algebra, Lemma \ref{dpa-lemma-kernel}.
Let $K' \subset B\langle x_t \rangle$ be the ideal
generated by the elements
\begin{enumerate}
\item $\delta_m(\sum r_t x_t - r_0)$ where $m > 0$ and
$(r_0, r_t) \in \mathcal{R}$,
\item $x_{t'}^{[m]}(x_t - f_t)$ where $m > 0$ and $t', t \in I$.
\end{enumerate}
We claim that $K' = K \cap J'$. The claim proves that $K \cap J'$
is preserved by $\delta_n$, $n > 0$ by the criterion of
Divided Power Algebra, Lemma \ref{dpa-lemma-kernel} (2)(c)
and a computation of $\delta_n$
of the elements listed which we leave to the reader.
To prove the claim note that $K' \subset K \cap J'$.
Conversely, if $h \in K \cap J'$ then, modulo $K'$ we can write
$$
h = \sum r_t (x_t - f_t)
$$
for some $r_t \in B$. As $h \in K \cap J' \subset J'$
we see that $r_0 = \sum r_t f_t \in I$. Hence $(r_0, r_t) \in \mathcal{R}$
and we see that
$$
h = \sum r_t x_t - r_0
$$
is in $K'$ as desired.
\end{proof}

\begin{lemma}
\label{lemma-divided-power-envelope-add-variables}
Let $(A, I, \gamma)$ be a divided power ring.
Let $B$ be an $A$-algebra and $IB \subset J \subset B$ an ideal.
Let $x_i$ be a set of variables. Then
$$
D_{B[x_i], \gamma}(JB[x_i] + (x_i)) = D_{B, \gamma}(J) \langle x_i \rangle
$$
\end{lemma}

\begin{proof}
One possible proof is to deduce this from
Lemma \ref{lemma-describe-divided-power-envelope}
as any relation between $x_i$ in $B[x_i]$ is trivial.
On the other hand, the lemma follows from the universal property
of the divided power polynomial algebra and the universal property of
divided power envelopes.
\end{proof}

\noindent
Conditions (1) and (2) of the following lemma hold if $B \to B'$ is flat
at all primes of $V(IB') \subset \Spec(B')$ and is very closely related
to that condition, see
Algebra, Lemma \ref{algebra-lemma-what-does-it-mean}.
It in particular says that taking the divided power
envelope commutes with localization.

\begin{lemma}
\label{lemma-flat-base-change-divided-power-envelope}
Let $(A, I, \gamma)$ be a divided power ring.
Let $B \to B'$ be a homomorphism of $A$-algebras.
Assume that
\begin{enumerate}
\item $B/IB \to B'/IB'$ is flat, and
\item $\text{Tor}_1^B(B', B/IB) = 0$.
\end{enumerate}
Then for any ideal $IB \subset J \subset B$ the canonical map
$$
D_B(J) \otimes_B B' \longrightarrow D_{B'}(JB')
$$
is an isomorphism.
\end{lemma}

\begin{proof}
Set $D = D_B(J)$ and denote $\bar J \subset D$ its divided power ideal
with divided power structure $\bar\gamma$. The universal property of
$D$ produces a $B$-algebra map $D \to D_{B'}(JB')$, whence a map as in
the lemma. It suffices to show that
the divided powers $\bar\gamma$ extend to $D \otimes_B B'$ since then
the universal property of $D_{B'}(JB')$ will produce a map
$D_{B'}(JB') \to D \otimes_B B'$ inverse to the one in the lemma.

\medskip\noindent
Choose a surjection $P \to B'$ where $P$ is a polynomial algebra over $B$.
In particular $B \to P$ is flat, hence $D \to D \otimes_B P$ is flat by
Algebra, Lemma \ref{algebra-lemma-flat-base-change}.
Then $\bar\gamma$ extends to $D \otimes_B P$ by
Divided Power Algebra, Lemma \ref{dpa-lemma-gamma-extends}; we
will denote this extension
$\bar\gamma$ also. Set $\mathfrak a = \Ker(P \to B')$ so that
we have the short exact sequence
$$
0 \to \mathfrak a \to P \to B' \to 0
$$
Thus $\text{Tor}_1^B(B', B/IB) = 0$ implies that
$\mathfrak a \cap IP = I\mathfrak a$.
Now we have the following commutative diagram
$$
\xymatrix{
B/J \otimes_B \mathfrak a \ar[r]_\beta &
B/J \otimes_B P \ar[r] &
B/J \otimes_B B' \\
D \otimes_B \mathfrak a \ar[r]^\alpha \ar[u] &
D \otimes_B P \ar[r] \ar[u] &
D \otimes_B B' \ar[u] \\
\bar J \otimes_B \mathfrak a \ar[r] \ar[u] &
\bar J \otimes_B P \ar[r] \ar[u] &
\bar J \otimes_B B' \ar[u]
}
$$
This diagram is exact even with $0$'s added at the top and the right.
We have to show the divided powers on the ideal
$\bar J \otimes_B P$ preserve the ideal
$\Im(\alpha) \cap \bar J \otimes_B P$, see
Divided Power Algebra, Lemma \ref{dpa-lemma-kernel}.
Consider the exact sequence
$$
0 \to \mathfrak a/I\mathfrak a \to P/IP \to B'/IB' \to 0
$$
(which uses that $\mathfrak a \cap IP = I\mathfrak a$ as seen above).
As $B'/IB'$ is flat over $B/IB$ this sequence remains exact after
applying $B/J \otimes_{B/IB} -$, see
Algebra, Lemma \ref{algebra-lemma-flat-tor-zero}. Hence
$$
\Ker(B/J \otimes_{B/IB} \mathfrak a/I\mathfrak a \to
B/J \otimes_{B/IB} P/IP) =
\Ker(\mathfrak a/J\mathfrak a \to P/JP)
$$
is zero. Thus $\beta$ is injective. It follows that
$\Im(\alpha) \cap \bar J \otimes_B P$ is the
image of $\bar J \otimes \mathfrak a$. Now if
$f \in \bar J$ and $a \in \mathfrak a$, then
$\bar\gamma_n(f \otimes a) = \bar\gamma_n(f) \otimes a^n$
hence the result is clear.
\end{proof}

\noindent
The following lemma is a special case of
\cite[Proposition 2.1.7]{dJ-crystalline} which in turn is a
generalization of \cite[Proposition 2.8.2]{Berthelot}.

\begin{lemma}
\label{lemma-flat-extension-divided-power-envelope}
Let $(B, I, \gamma) \to (B', I', \gamma')$ be a homomorphism of
divided power rings. Let $I \subset J \subset B$ and
$I' \subset J' \subset B'$ be ideals. Assume
\begin{enumerate}
\item $B/I \to B'/I'$ is flat, and
\item $J' = JB' + I'$.
\end{enumerate}
Then the canonical map
$$
D_{B, \gamma}(J) \otimes_B B' \longrightarrow D_{B', \gamma'}(J')
$$
is an isomorphism.
\end{lemma}

\begin{proof}
Set $D = D_{B, \gamma}(J)$. Choose elements $f_t \in J$ which generate $J/I$.
Set $\mathcal{R} = \{(r_0, r_t) \in I \oplus \bigoplus\nolimits_{t \in T} B
\mid \sum r_t f_t = r_0 \text{ in }B\}$ as in the proof of
Lemma \ref{lemma-describe-divided-power-envelope}. This lemma shows that
$$
D = B\langle x_t \rangle/ K
$$
where $K$ is generated by the elements $x_t - f_t$ and
$\delta_n(\sum r_t x_t - r_0)$ for $(r_0, r_t) \in \mathcal{R}$.
Thus we see that
\begin{equation}
\label{equation-base-change}
D \otimes_B B' = B'\langle x_t \rangle/K'
\end{equation}
where $K'$ is generated by the images in $B'\langle x_t \rangle$
of the generators of $K$ listed above. Let $f'_t \in B'$ be the image
of $f_t$. By assumption (1) we see that the elements $f'_t \in J'$
generate $J'/I'$ and we see that $x_t - f'_t \in K'$. Set
$$
\mathcal{R}' =
\{(r'_0, r'_t) \in I' \oplus \bigoplus\nolimits_{t \in T} B'
\mid \sum r'_t f'_t = r'_0 \text{ in }B'\}
$$
To finish the proof we have to show that
$\delta'_n(\sum r'_t x_t - r'_0) \in K'$ for
$(r'_0, r'_t) \in \mathcal{R}'$, because then the presentation
(\ref{equation-base-change}) of $D \otimes_B B'$ is identical
to the presentation of $D_{B', \gamma'}(J')$ obtain in
Lemma \ref{lemma-describe-divided-power-envelope} from the generators $f'_t$.
Suppose that $(r'_0, r'_t) \in \mathcal{R}'$. Then
$\sum r'_t f'_t = 0$ in $B'/I'$. As $B/I \to B'/I'$ is flat by
assumption (1) we can apply the equational criterion of flatness
(Algebra, Lemma \ref{algebra-lemma-flat-eq}) to see
that there exist an $m > 0$ and
$r_{jt} \in B$ and $c_j \in B'$, $j = 1, \ldots, m$ such
that
$$
r_{j0} = \sum\nolimits_t r_{jt} f_t  \in I \text{ for } j = 1, \ldots, m
$$
and
$$
i'_t  = r'_t - \sum\nolimits_j c_j r_{jt} \in I' \text{ for all }t
$$
Note that this also implies that
$r'_0 = \sum_t i'_t f_t + \sum_j c_j r_{j0}$.
Then we have
\begin{align*}
\delta'_n(\sum\nolimits_t r'_t x_t  - r'_0)
& =
\delta'_n(
\sum\nolimits_t i'_t x_t +
\sum\nolimits_{t, j} c_j r_{jt} x_t -
\sum\nolimits_t i'_t f_t -
\sum\nolimits_j c_j r_{j0}) \\
& =
\delta'_n(
\sum\nolimits_t i'_t(x_t - f_t) +
\sum\nolimits_j c_j (\sum\nolimits_t r_{jt} x_t  - r_{j0}))
\end{align*}
Since $\delta_n(a + b) = \sum_{m = 0, \ldots, n} \delta_m(a) \delta_{n - m}(b)$
and since $\delta_m(\sum i'_t(x_t - f_t))$ is in the ideal
generated by $x_t - f_t \in K'$ for $m > 0$, it suffices to prove that
$\delta_n(\sum c_j (\sum r_{jt} x_t  - r_{j0}))$ is in $K'$.
For this we use
$$
\delta_n(\sum\nolimits_j c_j (\sum\nolimits_t r_{jt} x_t  - r_{j0}))
=
\sum c_1^{n_1} \ldots c_m^{n_m}
\delta_{n_1}(\sum r_{1t} x_t  - r_{10}) \ldots
\delta_{n_m}(\sum r_{mt} x_t  - r_{m0})
$$
where the sum is over $n_1 + \ldots + n_m = n$. This proves what we want.
\end{proof}





\section{Some explicit divided power thickenings}
\label{section-explicit-thickenings}

\noindent
The constructions in this section will help us to define the connection
on a crystal in modules on the crystalline site.

\begin{lemma}
\label{lemma-divided-power-first-order-thickening}
Let $(A, I, \gamma)$ be a divided power ring. Let $M$ be an $A$-module.
Let $B = A \oplus M$ as an $A$-algebra where $M$ is an ideal of square zero
and set $J = I \oplus M$. Set
$$
\delta_n(x + z) = \gamma_n(x) + \gamma_{n - 1}(x)z
$$
for $x \in I$ and $z \in M$.
Then $\delta$ is a divided power structure and
$A \to B$ is a homomorphism of divided power rings from
$(A, I, \gamma)$ to $(B, J, \delta)$.
\end{lemma}

\begin{proof}
We have to check conditions (1) -- (5) of
Divided Power Algebra, Definition \ref{dpa-definition-divided-powers}.
We will prove this directly for this case, but please see the proof of
the next lemma for a method which avoids calculations.
Conditions (1) and (3) are clear. Condition (2) follows from
\begin{align*}
\delta_n(x + z)\delta_m(x + z)
& =
(\gamma_n(x) + \gamma_{n - 1}(x)z)(\gamma_m(x) + \gamma_{m - 1}(x)z) \\
& = \gamma_n(x)\gamma_m(x) + \gamma_n(x)\gamma_{m - 1}(x)z +
\gamma_{n - 1}(x)\gamma_m(x)z \\
& =
\frac{(n + m)!}{n!m!} \gamma_{n + m}(x) +
\left(\frac{(n + m - 1)!}{n!(m - 1)!} +
\frac{(n + m - 1)!}{(n - 1)!m!}\right)
\gamma_{n + m - 1}(x) z \\
& =
\frac{(n + m)!}{n!m!}\delta_{n + m}(x + z)
\end{align*}
Condition (5) follows from
\begin{align*}
\delta_n(\delta_m(x + z))
& =
\delta_n(\gamma_m(x) + \gamma_{m - 1}(x)z) \\
& =
\gamma_n(\gamma_m(x)) + \gamma_{n - 1}(\gamma_m(x))\gamma_{m - 1}(x)z \\
& =
\frac{(nm)!}{n! (m!)^n} \gamma_{nm}(x) +
\frac{((n - 1)m)!}{(n - 1)! (m!)^{n - 1}}
\gamma_{(n - 1)m}(x) \gamma_{m - 1}(x) z \\
& = \frac{(nm)!}{n! (m!)^n}(\gamma_{nm}(x) + \gamma_{nm - 1}(x) z)
\end{align*}
by elementary number theory. To prove (4) we have to see that
$$
\delta_n(x + x' + z + z')
=
\gamma_n(x + x') + \gamma_{n - 1}(x + x')(z + z')
$$
is equal to
$$
\sum\nolimits_{i = 0}^n
(\gamma_i(x) + \gamma_{i - 1}(x)z)
(\gamma_{n - i}(x') + \gamma_{n - i - 1}(x')z')
$$
This follows easily on collecting the coefficients of
$1$, $z$, and $z'$ and using condition (4) for $\gamma$.
\end{proof}

\begin{lemma}
\label{lemma-divided-power-second-order-thickening}
Let $(A, I, \gamma)$ be a divided power ring. Let $M$, $N$ be $A$-modules.
Let $q : M \times M \to N$ be an $A$-bilinear map.
Let $B = A \oplus M \oplus N$ as an $A$-algebra with multiplication
$$
(x, z, w)\cdot (x', z', w') = (xx', xz' + x'z, xw' + x'w + q(z, z') + q(z', z))
$$
and set $J = I \oplus M \oplus N$. Set
$$
\delta_n(x, z, w) = (\gamma_n(x), \gamma_{n - 1}(x)z,
\gamma_{n - 1}(x)w + \gamma_{n - 2}(x)q(z, z))
$$
for $(x, z, w) \in J$.
Then $\delta$ is a divided power structure and
$A \to B$ is a homomorphism of divided power rings from
$(A, I, \gamma)$ to $(B, J, \delta)$.
\end{lemma}

\begin{proof}
Suppose we want to prove that property (4) of
Divided Power Algebra, Definition \ref{dpa-definition-divided-powers}
is satisfied. Pick $(x, z, w)$ and $(x', z', w')$ in $J$.
Pick a map
$$
A_0 = \mathbf{Z}\langle s, s'\rangle \longrightarrow A,\quad
s \longmapsto x,
s' \longmapsto x'
$$
which is possible by the universal property of divided power
polynomial rings. Set $M_0 = A_0 \oplus A_0$ and
$N_0 = A_0 \oplus A_0 \oplus M_0 \otimes_{A_0} M_0$.
Let $q_0 : M_0 \times M_0 \to N_0$ be the obvious map.
Define $M_0 \to M$ as the $A_0$-linear map which sends
the basis vectors of $M_0$ to $z$ and $z'$. Define $N_0 \to N$
as the $A_0$ linear map which sends the first two basis vectors
of $N_0$ to $w$ and $w'$ and uses
$M_0 \otimes_{A_0} M_0 \to M \otimes_A M \xrightarrow{q} N$
on the last summand. Then we see that it suffices to prove the
identity (4) for the situation $(A_0, M_0, N_0, q_0)$.
Similarly for the other identities. This reduces us to the case
of a $\mathbf{Z}$-torsion free ring $A$ and $A$-torsion free modules.
In this case all we have to do is show that
$$
n! \delta_n(x, z, w) = (x, z, w)^n
$$
in the ring $A$, see Divided Power Algebra, Lemma \ref{dpa-lemma-silly}.
To see this note that
$$
(x, z, w)^2 = (x^2, 2xz, 2xw + 2q(z, z))
$$
and by induction
$$
(x, z, w)^n = (x^n, nx^{n - 1}z, nx^{n - 1}w + n(n - 1)x^{n - 2}q(z, z))
$$
On the other hand,
$$
n! \delta_n(x, z, w) = (n!\gamma_n(x), n!\gamma_{n - 1}(x)z,
n!\gamma_{n - 1}(x)w + n!\gamma_{n - 2}(x) q(z, z))
$$
which matches. This finishes the proof.
\end{proof}




\section{Compatibility}
\label{section-compatibility}

\noindent
This section isn't required reading; it explains how our discussion
fits with that of \cite{Berthelot}.
Consider the following technical notion.

\begin{definition}
\label{definition-compatible}
Let $(A, I, \gamma)$ and $(B, J, \delta)$ be divided power rings.
Let $A \to B$ be a ring map. We say
{\it $\delta$ is compatible with $\gamma$}
if there exists a divided power structure $\bar\gamma$ on
$J + IB$ such that
$$
(A, I, \gamma) \to (B, J + IB, \bar \gamma)\quad\text{and}\quad
(B, J, \delta) \to (B, J + IB, \bar \gamma)
$$
are homomorphisms of divided power rings.
\end{definition}

\noindent
Let $p$ be a prime number. Let $(A, I, \gamma)$ be a divided power ring.
Let $A \to C$ be a ring map with $p$ nilpotent in $C$.
Assume that $\gamma$ extends to $IC$ (see
Divided Power Algebra, Lemma \ref{dpa-lemma-gamma-extends}).
In this situation, the (big affine) crystalline site of
$\Spec(C)$ over $\Spec(A)$
as defined in \cite{Berthelot} 
is the opposite of the category of systems
$$
(B, J, \delta, A \to B, C \to B/J)
$$
where
\begin{enumerate}
\item $(B, J, \delta)$ is a divided power ring with $p$ nilpotent in $B$,
\item $\delta$ is compatible with $\gamma$, and
\item the diagram
$$
\xymatrix{
B \ar[r] & B/J \\
A \ar[u] \ar[r] & C \ar[u]
}
$$
is commutative.
\end{enumerate}
The conditions
``$\gamma$ extends to $C$ and $\delta$ compatible with $\gamma$''
are used in \cite{Berthelot} to ensure that
the crystalline cohomology of $\Spec(C)$ is the same as the crystalline
cohomology of $\Spec(C/IC)$. We will avoid this issue
by working exclusively with $C$ such that $IC = 0$\footnote{Of course there
will be a price to pay.}. In this case,
for a system $(B, J, \delta, A \to B, C \to B/J)$ as above,
the commutativity of the displayed diagram above implies $IB \subset J$ and
compatibility is equivalent to the condition that
$(A, I, \gamma) \to (B, J, \delta)$ is a homomorphism of divided
power rings.




\section{Affine crystalline site}
\label{section-affine-site}

\noindent
In this section we discuss the algebraic variant of the crystalline site.
Our basic situation in which we discuss this material will be as
follows.

\begin{situation}
\label{situation-affine}
Here $p$ is a prime number, $(A, I, \gamma)$ is a divided power
ring such that $A$ is a $\mathbf{Z}_{(p)}$-algebra, and $A \to C$ is a
ring map such that $IC = 0$ and such that $p$ is nilpotent in $C$.
\end{situation}

\noindent
Usually the prime number $p$ will be contained in the
divided power ideal $I$.

\begin{definition}
\label{definition-affine-thickening}
In Situation \ref{situation-affine}.
\begin{enumerate}
\item A {\it divided power thickening} of $C$ over $(A, I, \gamma)$
is a homomorphism of divided power algebras $(A, I, \gamma) \to (B, J, \delta)$
such that $p$ is nilpotent in $B$ and a ring map $C \to B/J$ such that
$$
\xymatrix{
B \ar[r] & B/J \\
& C \ar[u] \\
A \ar[uu] \ar[r] & A/I \ar[u]
}
$$
is commutative.
\item A {\it homomorphism of divided power thickenings}
$$
(B, J, \delta, C \to B/J) \longrightarrow (B', J', \delta', C \to B'/J')
$$
is a homomorphism $\varphi : B \to B'$ of divided power $A$-algebras such
that $C \to B/J \to B'/J'$ is the given map $C \to B'/J'$.
\item We denote $\text{CRIS}(C/A, I, \gamma)$ or simply $\text{CRIS}(C/A)$
the category of divided power thickenings of $C$ over $(A, I, \gamma)$.
\item We denote $\text{Cris}(C/A, I, \gamma)$ or simply $\text{Cris}(C/A)$
the full subcategory consisting of $(B, J, \delta, C \to B/J)$ such that
$C \to B/J$ is an isomorphism. We often denote such an object
$(B \to C, \delta)$ with $J = \Ker(B \to C)$ being understood.
\end{enumerate}
\end{definition}

\noindent
Note that for a divided power thickening $(B, J, \delta)$ as above
the ideal $J$ is locally nilpotent, see
Divided Power Algebra, Lemma \ref{dpa-lemma-nil}.
There is a canonical functor
\begin{equation}
\label{equation-forget-affine}
\text{CRIS}(C/A) \longrightarrow C\text{-algebras},\quad
(B, J, \delta) \longmapsto B/J
\end{equation}
This category does not have equalizers or fibre products in general.
It also doesn't have an initial object ($=$ empty colimit) in general.

\begin{lemma}
\label{lemma-affine-thickenings-colimits}
In Situation \ref{situation-affine}.
\begin{enumerate}
\item $\text{CRIS}(C/A)$ has finite products (but not infinite ones),
\item $\text{CRIS}(C/A)$ has all finite nonempty colimits and
(\ref{equation-forget-affine}) commutes with these, and
\item $\text{Cris}(C/A)$ has all finite nonempty colimits and
$\text{Cris}(C/A) \to \text{CRIS}(C/A)$ commutes with them.
\end{enumerate}
\end{lemma}

\begin{proof}
The empty product, i.e., the final object in the category of divided
power thickenings of $C$ over $(A, I, \gamma)$, is the zero ring viewed
as an $A$-algebra endowed with the zero ideal and the unique divided powers
on the zero ideal and finally endowed with the unique homomorphism of $C$ to
the zero ring. If $(B_t, J_t, \delta_t)_{t \in T}$ is a family of objects of
$\text{CRIS}(C/A)$ then we can form the product
$(\prod_t B_t, \prod_t J_t, \prod_t \delta_t)$ as in
Divided Power Algebra, Lemma \ref{dpa-lemma-limits}.
The map $C \to \prod B_t/\prod J_t = \prod B_t/J_t$ is clear.
However, we are only guaranteed that $p$ is nilpotent in $\prod_t B_t$
if $T$ is finite.

\medskip\noindent
Given two objects $(B, J, \gamma)$ and $(B', J', \gamma')$ of
$\text{CRIS}(C/A)$ we can form a cocartesian diagram
$$
\xymatrix{
(B, J, \delta) \ar[r] & (B'', J'', \delta'') \\
(A, I, \gamma) \ar[r] \ar[u] & (B', J', \delta') \ar[u]
}
$$
in the category of divided power rings. Then we see that we have
$$
B''/J'' = B/J \otimes_{A/I} B'/J' \longleftarrow C \otimes_{A/I} C
$$
see Divided Power Algebra, Remark \ref{dpa-remark-forgetful}.
Denote $J'' \subset K \subset B''$
the ideal such that
$$
\xymatrix{
B''/J'' \ar[r] & B''/K \\
C \otimes_{A/I} C \ar[r] \ar[u] & C \ar[u]
}
$$
is a pushout, i.e., $B''/K \cong B/J \otimes_C B'/J'$.
Let $D_{B''}(K) = (D, \bar K, \bar \delta)$
be the divided power envelope of $K$ in $B''$ relative to
$(B'', J'', \delta'')$. Then it is easily verified that
$(D, \bar K, \bar \delta)$ is a coproduct of $(B, J, \delta)$ and
$(B', J', \delta')$ in $\text{CRIS}(C/A)$.

\medskip\noindent
Next, we come to coequalizers. Let
$\alpha, \beta : (B, J, \delta) \to (B', J', \delta')$ be morphisms of
$\text{CRIS}(C/A)$. Consider $B'' = B'/ (\alpha(b) - \beta(b))$. Let
$J'' \subset B''$ be the image of $J'$. Let
$D_{B''}(J'') = (D, \bar J, \bar\delta)$ be the divided power envelope of
$J''$ in $B''$ relative to $(B', J', \delta')$. Then it is easily verified
that $(D, \bar J, \bar \delta)$ is the coequalizer of $(B, J, \delta)$ and
$(B', J', \delta')$ in $\text{CRIS}(C/A)$.

\medskip\noindent
By Categories, Lemma \ref{categories-lemma-almost-finite-colimits-exist}
we have all finite nonempty colimits in $\text{CRIS}(C/A)$. The constructions
above shows that (\ref{equation-forget-affine}) commutes with them.
This formally implies part (3) as $\text{Cris}(C/A)$ is the fibre category
of (\ref{equation-forget-affine}) over $C$.
\end{proof}

\begin{remark}
\label{remark-completed-affine-site}
In Situation \ref{situation-affine} we denote
$\text{Cris}^\wedge(C/A)$ the category whose objects are
pairs $(B \to C, \delta)$ such that
\begin{enumerate}
\item $B$ is a $p$-adically complete $A$-algebra,
\item $B \to C$ is a surjection of $A$-algebras,
\item $\delta$ is a divided power structure on $\Ker(B \to C)$,
\item $A \to B$ is a homomorphism of divided power rings.
\end{enumerate}
Morphisms are defined as in Definition \ref{definition-affine-thickening}.
Then $\text{Cris}(C/A) \subset \text{Cris}^\wedge(C/A)$ is the full
subcategory consisting of those $B$ such that $p$ is nilpotent in $B$.
Conversely, any object $(B \to C, \delta)$ of $\text{Cris}^\wedge(C/A)$
is equal to the limit
$$
(B \to C, \delta) = \lim_e (B/p^eB \to C, \delta)
$$
where for $e \gg 0$ the object $(B/p^eB \to C, \delta)$ lies
in $\text{Cris}(C/A)$, see
Divided Power Algebra, Lemma \ref{dpa-lemma-extend-to-completion}.
In particular, we see that $\text{Cris}^\wedge(C/A)$ is a full subcategory
of the category of pro-objects of $\text{Cris}(C/A)$, see
Categories, Remark \ref{categories-remark-pro-category}.
\end{remark}

\begin{lemma}
\label{lemma-list-properties}
In Situation \ref{situation-affine}.
Let $P \to C$ be a surjection of $A$-algebras with kernel $J$.
Write $D_{P, \gamma}(J) = (D, \bar J, \bar\gamma)$.
Let $(D^\wedge, J^\wedge, \bar\gamma^\wedge)$ be the
$p$-adic completion of $D$, see
Divided Power Algebra, Lemma \ref{dpa-lemma-extend-to-completion}.
For every $e \geq 1$ set $P_e = P/p^eP$ and $J_e \subset P_e$
the image of $J$ and write
$D_{P_e, \gamma}(J_e) = (D_e, \bar J_e, \bar\gamma)$.
Then for all $e$ large enough we have
\begin{enumerate}
\item $p^eD \subset \bar J$ and $p^eD^\wedge \subset \bar J^\wedge$
are preserved by divided powers,
\item $D^\wedge/p^eD^\wedge = D/p^eD = D_e$ as divided power rings,
\item $(D_e, \bar J_e, \bar\gamma)$ is an object of $\text{Cris}(C/A)$,
\item $(D^\wedge, \bar J^\wedge, \bar\gamma^\wedge)$ is equal to
$\lim_e (D_e, \bar J_e, \bar\gamma)$, and
\item $(D^\wedge, \bar J^\wedge, \bar\gamma^\wedge)$ is an object of
$\text{Cris}^\wedge(C/A)$.
\end{enumerate}
\end{lemma}

\begin{proof}
Part (1) follows from
Divided Power Algebra, Lemma \ref{dpa-lemma-extend-to-completion}.
It is a general property of $p$-adic completion that
$D/p^eD = D^\wedge/p^eD^\wedge$. Since $D/p^eD$ is a divided power ring
and since $P \to D/p^eD$ factors through $P_e$, the universal property of
$D_e$ produces a map $D_e \to D/p^eD$. Conversely, the universal property
of $D$ produces a map $D \to D_e$ which factors through $D/p^eD$. We omit
the verification that these maps are mutually inverse. This proves (2).
If $e$ is large enough, then $p^eC = 0$, hence we see (3) holds.
Part (4) follows from
Divided Power Algebra, Lemma \ref{dpa-lemma-extend-to-completion}.
Part (5) is clear from the definitions.
\end{proof}

\begin{lemma}
\label{lemma-set-generators}
In Situation \ref{situation-affine}.
Let $P$ be a polynomial algebra over $A$ and let
$P \to C$ be a surjection of $A$-algebras with kernel $J$.
With $(D_e, \bar J_e, \bar\gamma)$ as in Lemma \ref{lemma-list-properties}:
for every object $(B, J_B, \delta)$ of $\text{CRIS}(C/A)$ there
exists an $e$ and a morphism $D_e \to B$ of $\text{CRIS}(C/A)$.
\end{lemma}

\begin{proof}
We can find an $A$-algebra homomorphism $P \to B$
lifting the map $C \to B/J_B$. By our definition of
$\text{CRIS}(C/A)$ we see that $p^eB = 0$ for
some $e$ hence $P \to B$ factors as $P \to P_e \to B$.
By the universal property of the divided power envelope we
conclude that $P_e \to B$ factors through $D_e$.
\end{proof}

\begin{lemma}
\label{lemma-generator-completion}
In Situation \ref{situation-affine}.
Let $P$ be a polynomial algebra over $A$ and let
$P \to C$ be a surjection of $A$-algebras with kernel $J$.
Let $(D, \bar J, \bar\gamma)$ be the $p$-adic completion of
$D_{P, \gamma}(J)$. For every object $(B \to C, \delta)$ of
$\text{Cris}^\wedge(C/A)$ there
exists a morphism $D \to B$ of $\text{Cris}^\wedge(C/A)$.
\end{lemma}

\begin{proof}
We can find an $A$-algebra homomorphism $P \to B$ compatible
with maps to $C$. By our definition of
$\text{Cris}(C/A)$ we see that $P \to B$ factors as
$P \to D_{P, \gamma}(J) \to B$. As $B$ is $p$-adically complete
we can factor this map through $D$.
\end{proof}



\section{Module of differentials}
\label{section-differentials}

\noindent
In this section we develop a theory of modules of differentials
for divided power rings.

\begin{definition}
\label{definition-derivation}
Let $A$ be a ring. Let $(B, J, \delta)$ be a divided power ring.
Let $A \to B$ be a ring map. Let $M$ be an $B$-module.
A {\it divided power $A$-derivation} into $M$ is a map
$\theta : B \to M$ which is additive, annihilates the elements
of $A$, satisfies the Leibniz rule
$\theta(bb') = b\theta(b') + b'\theta(b)$ and satisfies
$$
\theta(\delta_n(x)) = \delta_{n - 1}(x)\theta(x)
$$
for all $n \geq 1$ and all $x \in J$.
\end{definition}

\noindent
In the situation of the definition, just as in the case of usual
derivations, there exists a {\it universal divided power $A$-derivation}
$$
\text{d}_{B/A, \delta} : B \to \Omega_{B/A, \delta}
$$
such that any divided power $A$-derivation $\theta : B \to M$ is equal to
$\theta = \xi \circ d_{B/A, \delta}$ for some unique $B$-linear map
$\xi : \Omega_{B/A, \delta} \to M$. If $(A, I, \gamma) \to (B, J, \delta)$
is a homomorphism of divided power rings, then we can forget the
divided powers on $A$ and consider the divided power derivations of
$B$ over $A$. Here are some basic properties of the universal
module of (divided power) differentials.

\begin{lemma}
\label{lemma-omega}
Let $A$ be a ring. Let $(B, J, \delta)$ be a divided power ring and
$A \to B$ a ring map. 
\begin{enumerate}
\item Consider $B[x]$ with divided power ideal $(JB[x], \delta')$
where $\delta'$ is the extension of $\delta$ to $B[x]$. Then
$$
\Omega_{B[x]/A, \delta'} =
\Omega_{B/A, \delta} \otimes_B B[x] \oplus B[x]\text{d}x.
$$
\item Consider $B\langle x \rangle$ with divided power ideal
$(JB\langle x \rangle + B\langle x \rangle_{+}, \delta')$. Then
$$
\Omega_{B\langle x\rangle/A, \delta'} =
\Omega_{B/A, \delta} \otimes_B B\langle x \rangle \oplus
B\langle x\rangle \text{d}x.
$$
\item Let $K \subset J$ be an ideal preserved by $\delta_n$ for
all $n > 0$. Set $B' = B/K$ and denote $\delta'$ the induced
divided power on $J/K$. Then $\Omega_{B'/A, \delta'}$ is the quotient
of $\Omega_{B/A, \delta} \otimes_B B'$ by the $B'$-submodule generated
by $\text{d}k$ for $k \in K$.
\end{enumerate}
\end{lemma}

\begin{proof}
These are proved directly from the construction of $\Omega_{B/A, \delta}$
as the free $B$-module on the elements $\text{d}b$ modulo the relations
\begin{enumerate}
\item $\text{d}(b + b') = \text{d}b + \text{d}b'$, $b, b' \in B$,
\item $\text{d}a = 0$, $a \in A$,
\item $\text{d}(bb') = b \text{d}b' + b' \text{d}b$, $b, b' \in B$,
\item $\text{d}\delta_n(f) = \delta_{n - 1}(f)\text{d}f$, $f \in J$, $n > 1$.
\end{enumerate}
Note that the last relation explains why we get ``the same'' answer for
the divided power polynomial algebra and the usual polynomial algebra:
in the first case $x$ is an element of the divided power ideal and hence
$\text{d}x^{[n]} = x^{[n - 1]}\text{d}x$.
\end{proof}

\noindent
Let $(A, I, \gamma)$ be a divided power ring. In this setting the
correct version of the powers of $I$ is given by the divided powers
$$
I^{[n]} = \text{ideal generated by }
\gamma_{e_1}(x_1) \ldots \gamma_{e_t}(x_t)
\text{ with }\sum e_j \geq n\text{ and }x_j \in I.
$$
Of course we have $I^n \subset I^{[n]}$. Note that $I^{[1]} = I$.
Sometimes we also set $I^{[0]} = A$.

\begin{lemma}
\label{lemma-diagonal-and-differentials}
Let $(A, I, \gamma) \to (B, J, \delta)$ be a homomorphism
of divided power rings. Let $(B(1), J(1), \delta(1))$ be the coproduct
of $(B, J, \delta)$ with itself over $(A, I, \gamma)$, i.e.,
such that
$$
\xymatrix{
(B, J, \delta) \ar[r] & (B(1), J(1), \delta(1)) \\
(A, I, \gamma) \ar[r] \ar[u] & (B, J, \delta) \ar[u]
}
$$
is cocartesian. Denote $K = \Ker(B(1) \to B)$.
Then $K \cap J(1) \subset J(1)$ is preserved by the divided power
structure and
$$
\Omega_{B/A, \delta} = K/ \left(K^2 + (K \cap J(1))^{[2]}\right)
$$
canonically.
\end{lemma}

\begin{proof}
The fact that $K \cap J(1) \subset J(1)$ is preserved by the divided power
structure follows from the fact that $B(1) \to B$ is a homomorphism of
divided power rings.

\medskip\noindent
Recall that $K/K^2$ has a canonical $B$-module structure.
Denote $s_0, s_1 : B \to B(1)$ the two coprojections and consider
the map $\text{d} : B \to K/K^2 +(K \cap J(1))^{[2]}$ given by
$b \mapsto s_1(b) - s_0(b)$. It is clear that $\text{d}$ is additive,
annihilates $A$, and satisfies the Leibniz rule.
We claim that $\text{d}$ is a divided power $A$-derivation.
Let $x \in J$. Set $y = s_1(x)$ and $z = s_0(x)$.
Denote $\delta$ the divided power structure on $J(1)$.
We have to show that $\delta_n(y) - \delta_n(z) = \delta_{n - 1}(y)(y - z)$
modulo $K^2 +(K \cap J(1))^{[2]}$ for $n \geq 1$.
The equality holds for $n = 1$. Assume $n > 1$.
Note that $\delta_i(y - z)$ lies in $(K \cap J(1))^{[2]}$ for $i > 1$.
Calculating modulo $K^2 + (K \cap J(1))^{[2]}$ we have
$$
\delta_n(z) = \delta_n(z - y + y) =
\sum\nolimits_{i = 0}^n \delta_i(z - y)\delta_{n - i}(y) =
\delta_{n - 1}(y) \delta_1(z - y) + \delta_n(y)
$$
This proves the desired equality.

\medskip\noindent
Let $M$ be a $B$-module. Let $\theta : B \to M$ be a divided power
$A$-derivation.
Set $D = B \oplus M$ where $M$ is an ideal of square zero. Define a
divided power structure on $J \oplus M \subset D$ by setting
$\delta_n(x + m) = \delta_n(x) + \delta_{n - 1}(x)m$ for $n > 1$, see
Lemma \ref{lemma-divided-power-first-order-thickening}.
There are two divided power algebra homomorphisms $B \to D$: the first
is given by the inclusion and the second by the map $b \mapsto b + \theta(b)$.
Hence we get a canonical homomorphism $B(1) \to D$ of divided power
algebras over $(A, I, \gamma)$. This induces a map $K \to M$
which annihilates $K^2$ (as $M$ is an ideal of square zero) and
$(K \cap J(1))^{[2]}$ as $M^{[2]} = 0$. The composition
$B \to K/K^2 + (K \cap J(1))^{[2]} \to M$ equals $\theta$ by construction.
It follows that $\text{d}$
is a universal divided power $A$-derivation and we win.
\end{proof}

\begin{remark}
\label{remark-filtration-differentials}
Let $A \to B$ be a ring map and let $(J, \delta)$ be a divided
power structure on $B$. The universal module $\Omega_{B/A, \delta}$
comes with a little bit of extra structure, namely the $B$-submodule
$N$ of $\Omega_{B/A, \delta}$ generated by $\text{d}_{B/A, \delta}(J)$.
In terms of the isomorphism given in
Lemma \ref{lemma-diagonal-and-differentials}
this corresponds to the image of
$K \cap J(1)$ in $\Omega_{B/A, \delta}$. Consider the $A$-algebra
$D = B \oplus \Omega^1_{B/A, \delta}$ with ideal $\bar J = J \oplus N$
and divided powers $\bar \delta$ as in the proof of the lemma.
Then $(D, \bar J, \bar \delta)$ is a divided power ring
and the two maps $B \to D$ given by $b \mapsto b$ and
$b \mapsto b + \text{d}_{B/A, \delta}(b)$
are homomorphisms of divided power rings over $A$. Moreover, $N$
is the smallest submodule of $\Omega_{B/A, \delta}$ such that this is true.
\end{remark}

\begin{lemma}
\label{lemma-diagonal-and-differentials-affine-site}
In Situation \ref{situation-affine}.
Let $(B, J, \delta)$ be an object of $\text{CRIS}(C/A)$.
Let $(B(1), J(1), \delta(1))$ be the coproduct of $(B, J, \delta)$
with itself in $\text{CRIS}(C/A)$. Denote
$K = \Ker(B(1) \to B)$. Then $K \cap J(1) \subset J(1)$
is preserved by the divided power structure and
$$
\Omega_{B/A, \delta} = K/ \left(K^2 + (K \cap J(1))^{[2]}\right)
$$
canonically.
\end{lemma}

\begin{proof}
Word for word the same as the proof of
Lemma \ref{lemma-diagonal-and-differentials}.
The only point that has to be checked is that the
divided power ring $D = B \oplus M$ is an object of $\text{CRIS}(C/A)$
and that the two maps $B \to C$ are morphisms of $\text{CRIS}(C/A)$.
Since $D/(J \oplus M) = B/J$ we can use $C \to B/J$ to view
$D$ as an object of $\text{CRIS}(C/A)$
and the statement on morphisms is clear from the construction.
\end{proof}

\begin{lemma}
\label{lemma-module-differentials-divided-power-envelope}
Let $(A, I, \gamma)$ be a divided power ring. Let $A \to B$ be a ring
map and let $IB \subset J \subset B$ be an ideal. Let
$D_{B, \gamma}(J) = (D, \bar J, \bar \gamma)$ be the divided power envelope.
Then we have
$$
\Omega_{D/A, \bar\gamma} = \Omega_{B/A} \otimes_B D
$$
\end{lemma}

\begin{proof}[First proof]
Let $M$ be a $D$-module. We claim that an $A$-derivation
$\vartheta : B \to M$ is the same thing as a divided power
$A$-derivation $\theta : D \to M$. The claim implies the
statement by the Yoneda lemma.

\medskip\noindent
Consider the square zero thickening $D \oplus M$ of $D$.
There is a divided power structure $\delta$ on $\bar J \oplus M$
if we set the higher divided power operations zero on $M$.
In other words, we set
$\delta_n(x + m) = \bar\gamma_n(x) + \bar\gamma_{n - 1}(x)m$ for
any $x \in \bar J$ and $m \in M$, see
Lemma \ref{lemma-divided-power-first-order-thickening}.
Consider the $A$-algebra map $B \to D \oplus M$ whose first
component is given by the map $B \to D$ and whose second component
is $\vartheta$. By the universal property we get a corresponding
homomorphism $D \to D \oplus M$ of divided power algebras
whose second component is the divided power
$A$-derivation $\theta$ corresponding to $\vartheta$.
\end{proof}

\begin{proof}[Second proof]
We will prove this first when $B$ is flat over $A$. In this case $\gamma$
extends to a divided power structure $\gamma'$ on $IB$, see
Divided Power Algebra, Lemma \ref{dpa-lemma-gamma-extends}.
Hence $D = D_{B, \gamma'}(J)$ is equal to a quotient of
the divided power ring $(D', J', \delta)$ where $D' =  B\langle x_t \rangle$
and $J' = IB\langle x_t \rangle + B\langle x_t \rangle_{+}$
by the elements $x_t - f_t$ and $\delta_n(\sum r_t x_t - r_0)$, see
Lemma \ref{lemma-describe-divided-power-envelope} for notation
and explanation. Write $\text{d} : D' \to \Omega_{D'/A, \delta}$
for the universal derivation. Note that
$$
\Omega_{D'/A, \delta} =
\Omega_{B/A} \otimes_B D' \oplus \bigoplus D' \text{d}x_t,
$$
see Lemma \ref{lemma-omega}. We conclude that $\Omega_{D/A, \bar\gamma}$
is the quotient of $\Omega_{D'/A, \delta} \otimes_{D'} D$ by the submodule
generated by $\text{d}$ applied to the generators of the
kernel of $D' \to D$ listed above, see Lemma \ref{lemma-omega}.
Since $\text{d}(x_t - f_t) = - \text{d}f_t + \text{d}x_t$
we see that we have $\text{d}x_t = \text{d}f_t$ in the quotient.
In particular we see that $\Omega_{B/A} \otimes_B D \to \Omega_{D/A, \gamma}$
is surjective with kernel given by the images of $\text{d}$
applied to the elements $\delta_n(\sum r_t x_t - r_0)$.
However, given a relation $\sum r_tf_t - r_0 = 0$ in $B$ with
$r_t \in B$ and $r_0 \in IB$ we see that
\begin{align*}
\text{d}\delta_n(\sum r_t x_t - r_0)
& =
\delta_{n - 1}(\sum r_t x_t - r_0)\text{d}(\sum r_t x_t - r_0)
\\
& =
\delta_{n - 1}(\sum r_t x_t - r_0)
\left(
\sum r_t\text{d}(x_t - f_t) + \sum (x_t - f_t)\text{d}r_t
\right)
\end{align*}
because $\sum r_tf_t - r_0 = 0$ in $B$. Hence this is already zero in
$\Omega_{B/A} \otimes_A D$ and we win in the case that $B$ is flat over $A$.

\medskip\noindent
In the general case we write $B$ as a quotient of a polynomial ring
$P \to B$ and let $J' \subset P$ be the inverse image of $J$. Then
$D = D'/K'$ with notation as in
Lemma \ref{lemma-divided-power-envelop-quotient}.
By the case handled in the first paragraph of the proof we have
$\Omega_{D'/A, \bar\gamma'} = \Omega_{P/A} \otimes_P D'$. Then
$\Omega_{D/A, \bar \gamma}$ is the quotient of $\Omega_{P/A} \otimes_P D$
by the submodule generated by $\text{d}\bar\gamma_n'(k)$ where $k$
is an element of the kernel of $P \to B$, see
Lemma \ref{lemma-omega} and the description of $K'$ from
Lemma \ref{lemma-divided-power-envelop-quotient}. Since
$\text{d}\bar\gamma_n'(k) = \bar\gamma'_{n - 1}(k)\text{d}k$ we see
again that it suffices to divided by the submodule generated by
$\text{d}k$ with $k \in \Ker(P \to B)$ and since $\Omega_{B/A}$
is the quotient of $\Omega_{P/A} \otimes_A B$ by these elements
(Algebra, Lemma \ref{algebra-lemma-differential-seq}) we win.
\end{proof}

\begin{remark}
\label{remark-divided-powers-de-rham-complex}
Let $A \to B$ be a ring map and let $(J, \delta)$ be a divided power
structure on $B$. Set
$\Omega_{B/A, \delta}^i = \wedge^i_B \Omega_{B/A, \delta}$
where $\Omega_{B/A, \delta}$ is the target of the universal divided power
$A$-derivation $\text{d} = \text{d}_{B/A} : B \to \Omega_{B/A, \delta}$.
Note that $\Omega_{B/A, \delta}$ is the quotient of $\Omega_{B/A}$ by the
$B$-submodule generated by the elements
$\text{d}\delta_n(x) - \delta_{n - 1}(x)\text{d}x$ for $x \in J$.
We claim Algebra, Lemma \ref{algebra-lemma-de-rham-complex} applies.
To see this it suffices to verify the elements
$\text{d}\delta_n(x) - \delta_{n - 1}(x)\text{d}x$
of $\Omega_B$ are mapped to zero in $\Omega^2_{B/A, \delta}$.
We observe that
$$
\text{d}(\delta_{n - 1}(x)) \wedge \text{d}x
= \delta_{n - 2}(x) \text{d}x \wedge \text{d}x = 0
$$
in $\Omega^2_{B/A, \delta}$ as desired. Hence we obtain a
{\it divided power de Rham complex}
$$
\Omega^0_{B/A, \delta} \to \Omega^1_{B/A, \delta} \to
\Omega^2_{B/A, \delta} \to \ldots
$$
which will play an important role in the sequel.
\end{remark}

\begin{remark}
\label{remark-connection}
Let $A \to B$ be a ring map. Let $\Omega_{B/A} \to \Omega$
be a quotient satisfying the assumptions of
Algebra, Lemma \ref{algebra-lemma-de-rham-complex}.
Let $M$ be a $B$-module. A {\it connection} is an additive map
$$
\nabla : M \longrightarrow M \otimes_B \Omega
$$
such that $\nabla(bm) = b \nabla(m) + m \otimes \text{d}b$
for $b \in B$ and $m \in M$. In this situation we can define maps
$$
\nabla : M \otimes_B \Omega^i \longrightarrow M \otimes_B \Omega^{i + 1}
$$
by the rule $\nabla(m \otimes \omega) = \nabla(m) \wedge \omega +
m \otimes \text{d}\omega$. This works because if $b \in B$, then
\begin{align*}
\nabla(bm \otimes \omega) - \nabla(m \otimes b\omega)
& =
\nabla(bm) \wedge \omega + bm \otimes \text{d}\omega
- \nabla(m) \wedge b\omega - m \otimes \text{d}(b\omega) \\
& =
b\nabla(m) \wedge \omega + m \otimes \text{d}b \wedge \omega
+ bm \otimes \text{d}\omega \\
& \ \ \ \ \ \ - b\nabla(m) \wedge \omega - bm \otimes \text{d}(\omega)
- m \otimes \text{d}b \wedge \omega = 0
\end{align*}
As is customary we say the connection is {\it integrable} if and
only if the composition
$$
M \xrightarrow{\nabla} M \otimes_B \Omega^1
\xrightarrow{\nabla} M \otimes_B \Omega^2
$$
is zero. In this case we obtain a complex
$$
M \xrightarrow{\nabla} M \otimes_B \Omega^1
\xrightarrow{\nabla} M \otimes_B \Omega^2
\xrightarrow{\nabla} M \otimes_B \Omega^3
\xrightarrow{\nabla} M \otimes_B \Omega^4 \to \ldots
$$
which is called the de Rham complex of the connection.
\end{remark}

\begin{remark}
\label{remark-base-change-connection}
Consider a commutative diagram of rings
$$
\xymatrix{
B \ar[r]_\varphi & B' \\
A \ar[u] \ar[r] & A' \ar[u]
}
$$
Let $\Omega_{B/A} \to \Omega$ and $\Omega_{B'/A'} \to \Omega'$
be quotients satisfying the assumptions of
Algebra, Lemma \ref{algebra-lemma-de-rham-complex}.
Assume there is a map $\varphi : \Omega \to \Omega'$ which
fits into a commutative diagram
$$
\xymatrix{
\Omega_{B/A} \ar[r] \ar[d] &
\Omega_{B'/A'} \ar[d] \\
\Omega \ar[r]^{\varphi} &
\Omega'
}
$$
where the top horizontal arrow is the canonical map
$\Omega_{B/A} \to \Omega_{B'/A'}$ induced by $\varphi : B \to B'$.
In this situation, given any pair $(M, \nabla)$ where $M$ is a $B$-module
and $\nabla : M \to M \otimes_B \Omega$ is a connection
we obtain a {\it base change} $(M \otimes_B B', \nabla')$ where
$$
\nabla' :
M \otimes_B B'
\longrightarrow
(M \otimes_B B') \otimes_{B'} \Omega' = M \otimes_B \Omega'
$$
is defined by the rule
$$
\nabla'(m \otimes b') =
\sum m_i \otimes b'\text{d}\varphi(b_i) + m \otimes \text{d}b' 
$$
if $\nabla(m) = \sum m_i \otimes \text{d}b_i$. If $\nabla$ is integrable,
then so is $\nabla'$, and in this case there is a canonical map of
de Rham complexes (Remark \ref{remark-connection})
\begin{equation}
\label{equation-base-change-map-complexes}
M \otimes_B \Omega^\bullet
\longrightarrow
(M \otimes_B B') \otimes_{B'} (\Omega')^\bullet =
M \otimes_B (\Omega')^\bullet
\end{equation}
which maps $m \otimes \eta$ to $m \otimes \varphi(\eta)$.
\end{remark}

\begin{lemma}
\label{lemma-differentials-completion}
Let $A \to B$ be a ring map and let $(J, \delta)$ be a divided power
structure on $B$. Let $p$ be a prime number. Assume that $A$ is a
$\mathbf{Z}_{(p)}$-algebra and that $p$ is nilpotent in $B/J$. Then
we have
$$
\lim_e \Omega_{B_e/A, \bar\delta} =
\lim_e \Omega_{B/A, \delta}/p^e\Omega_{B/A, \delta} =
\lim_e \Omega_{B^\wedge/A, \delta^\wedge}/p^e \Omega_{B^\wedge/A, \delta^\wedge}
$$
see proof for notation and explanation.
\end{lemma}

\begin{proof}
By Divided Power Algebra, Lemma \ref{dpa-lemma-extend-to-completion}
we see that $\delta$ extends
to $B_e = B/p^eB$ for all sufficiently large $e$. Hence the first limit
make sense. The lemma also produces a divided power structure $\delta^\wedge$
on the completion $B^\wedge = \lim_e B_e$, hence the last limit makes
sense. By Lemma \ref{lemma-omega}
and the fact that $\text{d}p^e = 0$ (always)
we see that the surjection
$\Omega_{B/A, \delta} \to \Omega_{B_e/A, \bar\delta}$ has kernel
$p^e\Omega_{B/A, \delta}$. Similarly for the kernel of 
$\Omega_{B^\wedge/A, \delta^\wedge} \to \Omega_{B_e/A, \bar\delta}$.
Hence the lemma is clear.
\end{proof}



\section{Divided power schemes}
\label{section-divided-power-schemes}

\noindent
Some remarks on how to globalize the previous notions.

\begin{definition}
\label{definition-divided-power-structure}
Let $\mathcal{C}$ be a site. Let $\mathcal{O}$ be a sheaf of rings
on $\mathcal{C}$. Let $\mathcal{I} \subset \mathcal{O}$ be a
sheaf of ideals. A {\it divided power structure $\gamma$} on $\mathcal{I}$
is a sequence of maps $\gamma_n : \mathcal{I} \to \mathcal{I}$, $n \geq 1$
such that for any object $U$ of $\mathcal{C}$ the triple
$$
(\mathcal{O}(U), \mathcal{I}(U), \gamma)
$$
is a divided power ring.
\end{definition}

\noindent
To be sure this applies in particular to sheaves of rings on
topological spaces. But it's good to be a little bit more general
as the structure sheaf of the crystalline site lives on a... site!
A triple $(\mathcal{C}, \mathcal{I}, \gamma)$ as in the
definition above is sometimes called a {\it divided power topos}
in this chapter. Given a second $(\mathcal{C}', \mathcal{I}', \gamma')$ and
given a morphism of ringed topoi
$(f, f^\sharp) : (\Sh(\mathcal{C}), \mathcal{O}) \to
(\Sh(\mathcal{C}'), \mathcal{O}')$
we say that $(f, f^\sharp)$ induces a {\it morphism of divided
power topoi} if $f^\sharp(f^{-1}\mathcal{I}') \subset \mathcal{I}$
and the diagrams
$$
\xymatrix{
f^{-1}\mathcal{I}' \ar[d]_{f^{-1}\gamma'_n} \ar[r]_{f^\sharp} &
\mathcal{I} \ar[d]^{\gamma_n} \\
f^{-1}\mathcal{I}' \ar[r]^{f^\sharp} & \mathcal{I}
}
$$
are commutative for all $n \geq 1$. If $f$ comes from a morphism of
sites induced by a functor $u : \mathcal{C}' \to \mathcal{C}$ then
this just means that
$$
(\mathcal{O}'(U'), \mathcal{I}'(U'), \gamma')
\longrightarrow
(\mathcal{O}(u(U')), \mathcal{I}(u(U')), \gamma)
$$
is a homomorphism of divided power rings for all $U' \in \Ob(\mathcal{C}')$.

\medskip\noindent
In the case of schemes we require the divided power ideal to be
{\bf quasi-coherent}. But apart from this the definition is exactly
the same as in the case of topoi. Here it is.

\begin{definition}
\label{definition-divided-power-scheme}
A {\it divided power scheme} is a triple $(S, \mathcal{I}, \gamma)$
where $S$ is a scheme, $\mathcal{I}$ is a quasi-coherent sheaf of
ideals, and $\gamma$ is a divided power structure on $\mathcal{I}$.
A {\it morphism of divided power schemes}
$(S, \mathcal{I}, \gamma) \to (S', \mathcal{I}', \gamma')$ is
a morphism of schemes $f : S \to S'$ such that
$f^{-1}\mathcal{I}'\mathcal{O}_S \subset \mathcal{I}$ and such that
$$
(\mathcal{O}_{S'}(U'), \mathcal{I}'(U'), \gamma')
\longrightarrow
(\mathcal{O}_S(f^{-1}U'), \mathcal{I}(f^{-1}U'), \gamma)
$$
is a homomorphism of divided power rings for all $U' \subset S'$ open.
\end{definition}

\noindent
Recall that there is a 1-to-1 correspondence between quasi-coherent
sheaves of ideals and closed immersions, see
Morphisms, Section \ref{morphisms-section-closed-immersions}.
Thus given a divided power scheme $(T, \mathcal{J}, \gamma)$ we
get a canonical closed immersion $U \to T$ defined by $\mathcal{J}$.
Conversely, given a closed immersion $U \to T$ and a divided power
structure $\gamma$ on the sheaf of ideals $\mathcal{J}$ associated
to $U \to T$ we obtain a divided power scheme $(T, \mathcal{J}, \gamma)$.
In many situations we only want to consider such triples
$(U, T, \gamma)$ when the morphism $U \to T$ is a thickening, see
More on Morphisms, Definition \ref{more-morphisms-definition-thickening}.

\begin{definition}
\label{definition-divided-power-thickening}
A triple $(U, T, \gamma)$ as above is called a {\it divided power thickening}
if $U \to T$ is a thickening.
\end{definition}

\noindent
Fibre products of divided power schemes exist when one of the
three is a divided power thickening. Here is a formal statement.

\begin{lemma}
\label{lemma-fibre-product}
Let $(U', T', \delta') \to (S'_0, S', \gamma')$ and
$(S_0, S, \gamma) \to (S'_0, S', \gamma')$ be morphisms of
divided power schemes. If $(U', T', \delta')$ is a divided power
thickening, then there exists a divided power scheme $(T_0, T, \delta)$
and
$$
\xymatrix{
T \ar[r] \ar[d] & T' \ar[d] \\
S \ar[r] & S'
}
$$
which is a cartesian diagram in the category of divided power schemes.
\end{lemma}

\begin{proof}
Omitted. Hints: If $T$ exists, then $T_0 = S_0 \times_{S'_0} U'$
(argue as in Divided Power Algebra, Remark \ref{dpa-remark-forgetful}).
Since $T'$ is a divided power thickening, we see that $T$
(if it exists) will be a divided power thickening too.
Hence we can define $T$ as the scheme with underlying topological
space the underlying topological space of $T_0 = S_0 \times_{S'_0} U'$
and as structure sheaf on affine pieces the ring given
by Lemma \ref{lemma-affine-thickenings-colimits}.
\end{proof}

\noindent
We make the following observation. Suppose that $(U, T, \gamma)$
is triple as above. Assume that $T$ is a scheme over $\mathbf{Z}_{(p)}$
and that $p$ is locally nilpotent on $U$. Then
\begin{enumerate}
\item $p$ locally nilpotent on $T \Leftrightarrow U \to T$
is a thickening (see Divided Power Algebra, Lemma \ref{dpa-lemma-nil}), and
\item $p^e\mathcal{O}_T$ is locally on $T$ preserved by $\gamma$
for $e \gg 0$ (see
Divided Power Algebra, Lemma \ref{dpa-lemma-extend-to-completion}).
\end{enumerate}
This suggest that good results on divided power thickenings will be
available under the following hypotheses.

\begin{situation}
\label{situation-global}
Here $p$ is a prime number and $(S, \mathcal{I}, \gamma)$ is a divided power
scheme over $\mathbf{Z}_{(p)}$. We set $S_0 = V(\mathcal{I}) \subset S$.
Finally, $X \to S_0$ is a morphism of schemes such that $p$ is
locally nilpotent on $X$.
\end{situation}

\noindent
It is in this situation that we will define the big and small
crystalline sites.









\section{The big crystalline site}
\label{section-big-site}

\noindent
We first define the big site. Given a divided power scheme
$(S, \mathcal{I}, \gamma)$ we say $(T, \mathcal{J}, \delta)$ is
a divided power scheme over $(S, \mathcal{I}, \gamma)$ if
$T$ comes endowed with a morphism $T \to S$ of divided power
schemes. Similarly, we say a divided power thickening $(U, T, \delta)$
is a divided power thickening over $(S, \mathcal{I}, \gamma)$
if $T$ comes endowed with a morphism $T \to S$ of divided power
schemes.

\begin{definition}
\label{definition-divided-power-thickening-X}
In Situation \ref{situation-global}.
\begin{enumerate}
\item A {\it divided power thickening of $X$ relative to
$(S, \mathcal{I}, \gamma)$} is given by a divided power thickening
$(U, T, \delta)$ over $(S, \mathcal{I}, \gamma)$
and an $S$-morphism $U \to X$.
\item A {\it morphism of divided power thickenings of $X$
relative to $(S, \mathcal{I}, \gamma)$} is defined in the obvious
manner.
\end{enumerate}
The category of divided power thickenings of $X$ relative to
$(S, \mathcal{I}, \gamma)$ is denoted $\text{CRIS}(X/S, \mathcal{I}, \gamma)$
or simply $\text{CRIS}(X/S)$.
\end{definition}

\noindent
For any $(U, T, \delta)$ in $\text{CRIS}(X/S)$
we have that $p$ is locally nilpotent on $T$, see discussion preceding
Situation \ref{situation-global}.
A good way to visualize all the data associated to $(U, T, \delta)$
is the commutative diagram
$$
\xymatrix{
T \ar[dd] & U \ar[l] \ar[d] \\
& X \ar[d] \\
S & S_0 \ar[l]
}
$$
where $S_0 = V(\mathcal{I}) \subset S$. Morphisms of $\text{CRIS}(X/S)$
can be similarly visualized as huge commutative diagrams. In particular,
there is a canonical forgetful functor
\begin{equation}
\label{equation-forget}
\text{CRIS}(X/S) \longrightarrow \Sch/X,\quad
(U, T, \delta) \longmapsto U
\end{equation}
as well as its one sided inverse (and left adjoint)
\begin{equation}
\label{equation-endow-trivial}
\Sch/X \longrightarrow \text{CRIS}(X/S),\quad
U \longmapsto (U, U, \emptyset)
\end{equation}
which is sometimes useful.

\begin{lemma}
\label{lemma-divided-power-thickening-fibre-products}
In Situation \ref{situation-global}.
The category $\text{CRIS}(X/S)$ has all finite nonempty limits,
in particular products of pairs and fibre products.
The functor (\ref{equation-forget}) commutes with limits.
\end{lemma}

\begin{proof}
Omitted. Hint: See Lemma \ref{lemma-affine-thickenings-colimits}
for the affine case. See also
Divided Power Algebra, Remark \ref{dpa-remark-forgetful}.
\end{proof}

\begin{lemma}
\label{lemma-divided-power-thickening-base-change-flat}
In Situation \ref{situation-global}. Let
$$
\xymatrix{
(U_3, T_3, \delta_3) \ar[d] \ar[r] & (U_2, T_2, \delta_2) \ar[d] \\
(U_1, T_1, \delta_1) \ar[r] & (U, T, \delta)
}
$$
be a fibre square in the category of divided power thickenings of
$X$ relative to $(S, \mathcal{I}, \gamma)$. If $T_2 \to T$ is
flat and $U_2 = T_2 \times_T U$, then $T_3 = T_1 \times_T T_2$ (as schemes).
\end{lemma}

\begin{proof}
This is true because a divided power structure extends uniquely
along a flat ring map. See
Divided Power Algebra, Lemma \ref{dpa-lemma-gamma-extends}.
\end{proof}

\noindent
The lemma above means that the base change of a flat morphism
of divided power thickenings is another flat morphism, and in
fact is the ``usual'' base change of the morphism. This implies
that the following definition makes sense.

\begin{definition}
\label{definition-big-crystalline-site}
In Situation \ref{situation-global}.
\begin{enumerate}
\item A family of morphisms $\{(U_i, T_i, \delta_i) \to (U, T, \delta)\}$
of divided power thickenings of $X/S$ is a
{\it Zariski, \'etale, smooth, syntomic, or fppf covering}
if and only if
\begin{enumerate}
\item $U_i = U \times_T T_i$ for all $i$ and
\item $\{T_i \to T\}$ is a Zariski, \'etale, smooth, syntomic, or fppf covering.
\end{enumerate}
\item The {\it big crystalline site} of $X$ over $(S, \mathcal{I}, \gamma)$,
is the category $\text{CRIS}(X/S)$ endowed with the Zariski topology.
\item The topos of sheaves on $\text{CRIS}(X/S)$ is denoted
$(X/S)_{\text{CRIS}}$ or sometimes
$(X/S, \mathcal{I}, \gamma)_{\text{CRIS}}$\footnote{This clashes with
our convention to denote the topos associated to a site $\mathcal{C}$
by $\Sh(\mathcal{C})$.}.
\end{enumerate}
\end{definition}

\noindent
There are some obvious functorialities concerning these topoi.

\begin{remark}[Functoriality]
\label{remark-functoriality-big-cris}
Let $p$ be a prime number.
Let $(S, \mathcal{I}, \gamma) \to (S', \mathcal{I}', \gamma')$ be a
morphism of divided power schemes over $\mathbf{Z}_{(p)}$.
Set $S_0 = V(\mathcal{I})$ and $S'_0 = V(\mathcal{I}')$.
Let
$$
\xymatrix{
X \ar[r]_f \ar[d] & Y \ar[d] \\
S_0 \ar[r] & S'_0
}
$$
be a commutative diagram of morphisms of schemes and assume $p$ is
locally nilpotent on $X$ and $Y$. Then we get a continuous and
cocontinuous functor
$$
\text{CRIS}(X/S) \longrightarrow \text{CRIS}(Y/S')
$$
by letting $(U, T, \delta)$ correspond to $(U, T, \delta)$
with $U \to X \to Y$ as the $S'$-morphism from $U$ to $Y$.
Hence we get a morphism of topoi
$$
f_{\text{CRIS}} : (X/S)_{\text{CRIS}} \longrightarrow (Y/S')_{\text{CRIS}}
$$
see Sites, Section \ref{sites-section-cocontinuous-morphism-topoi}.
\end{remark}

\begin{remark}[Comparison with Zariski site]
\label{remark-compare-big-zariski}
In Situation \ref{situation-global}.
The functor (\ref{equation-forget}) is cocontinuous (details omitted) and
commutes with products and fibred products
(Lemma \ref{lemma-divided-power-thickening-fibre-products}).
Hence we obtain a morphism of topoi
$$
U_{X/S} : (X/S)_{\text{CRIS}} \longrightarrow \Sh((\Sch/X)_{Zar})
$$
from the big crystalline topos of $X/S$ to the big Zariski topos of $X$.
See Sites, Section \ref{sites-section-cocontinuous-morphism-topoi}.
\end{remark}

\begin{remark}[Structure morphism]
\label{remark-big-structure-morphism}
In Situation \ref{situation-global}.
Consider the closed subscheme $S_0 = V(\mathcal{I}) \subset S$.
If we assume that $p$ is locally nilpotent on $S_0$ (which is always
the case in practice) then we obtain a situation as in
Definition \ref{definition-divided-power-thickening-X} with $S_0$ instead
of $X$. Hence we get a site $\text{CRIS}(S_0/S)$. If $f : X \to S_0$ is
the structure morphism of $X$ over $S$, then we get a commutative diagram
of morphisms of ringed topoi
$$
\xymatrix{
(X/S)_{\text{CRIS}}
\ar[r]_{f_{\text{CRIS}}} \ar[d]_{U_{X/S}} &
(S_0/S)_{\text{CRIS}} \ar[d]^{U_{S_0/S}} \\
\Sh((\Sch/X)_{Zar}) \ar[r]^{f_{big}} & \Sh((\Sch/S_0)_{Zar}) \ar[rd] \\
& & \Sh((\Sch/S)_{Zar})
}
$$
by Remark \ref{remark-functoriality-big-cris}. We think of the composition
$(X/S)_{\text{CRIS}} \to \Sh((\Sch/S)_{Zar})$ as the structure morphism of
the big crystalline site. Even if $p$ is not locally nilpotent on $S_0$
the structure morphism
$$
(X/S)_{\text{CRIS}} \longrightarrow \Sh((\Sch/S)_{Zar})
$$
is defined as we can take the lower route through the diagram above. Thus it
is the morphism of topoi corresponding to the cocontinuous
functor $\text{CRIS}(X/S) \to (\Sch/S)_{Zar}$ given by the rule
$(U, T, \delta)/S \mapsto U/S$, see
Sites, Section \ref{sites-section-cocontinuous-morphism-topoi}.
\end{remark}

\begin{remark}[Compatibilities]
\label{remark-compatibilities-big-cris}
The morphisms defined above satisfy numerous compatibilities. For example,
in the situation of Remark \ref{remark-functoriality-big-cris}
we obtain a commutative diagram of ringed topoi
$$
\xymatrix{
(X/S)_{\text{CRIS}} \ar[d] \ar[r] & (Y/S')_{\text{CRIS}} \ar[d] \\
\Sh((\Sch/S)_{Zar}) \ar[r] & \Sh((\Sch/S')_{Zar})
}
$$
where the vertical arrows are the structure morphisms.
\end{remark}




\section{The crystalline site}
\label{section-site}

\noindent
Since (\ref{equation-forget}) commutes with products and fibre
products, we see that looking at those $(U, T, \delta)$ such that
$U \to X$ is an open immersion defines a full
subcategory preserved under fibre products (and more generally
finite nonempty limits). Hence the following
definition makes sense.

\begin{definition}
\label{definition-crystalline-site}
In Situation \ref{situation-global}.
\begin{enumerate}
\item The (small) {\it crystalline site} of $X$ over
$(S, \mathcal{I}, \gamma)$, denoted $\text{Cris}(X/S, \mathcal{I}, \gamma)$
or simply $\text{Cris}(X/S)$ is the full subcategory of $\text{CRIS}(X/S)$
consisting of those $(U, T, \delta)$ in $\text{CRIS}(X/S)$ such that
$U \to X$ is an open immersion. It comes endowed with the Zariski topology.
\item The topos of sheaves on $\text{Cris}(X/S)$ is denoted
$(X/S)_{\text{cris}}$ or sometimes
$(X/S, \mathcal{I}, \gamma)_{\text{cris}}$\footnote{This clashes with
our convention to denote the topos associated to a site $\mathcal{C}$
by $\Sh(\mathcal{C})$.}.
\end{enumerate}
\end{definition}

\noindent
For any $(U, T, \delta)$ in $\text{Cris}(X/S)$ the morphism $U \to X$
defines an object of the small Zariski site $X_{Zar}$ of $X$. Hence
a canonical forgetful functor
\begin{equation}
\label{equation-forget-small}
\text{Cris}(X/S) \longrightarrow X_{Zar},\quad
(U, T, \delta) \longmapsto U
\end{equation}
and a left adjoint
\begin{equation}
\label{equation-endow-trivial-small}
X_{Zar} \longrightarrow \text{Cris}(X/S),\quad
U \longmapsto (U, U, \emptyset)
\end{equation}
which is sometimes useful.

\medskip\noindent
We can compare the small and big crystalline sites, just like
we can compare the small and big Zariski sites of a scheme, see
Topologies, Lemma \ref{topologies-lemma-at-the-bottom}.

\begin{lemma}
\label{lemma-compare-big-small}
Assumptions as in Definition \ref{definition-divided-power-thickening-X}.
The inclusion functor
$$
\text{Cris}(X/S) \to \text{CRIS}(X/S)
$$
commutes with finite nonempty limits, is fully faithful, continuous,
and cocontinuous. There are morphisms of topoi
$$
(X/S)_{\text{cris}} \xrightarrow{i} (X/S)_{\text{CRIS}}
\xrightarrow{\pi} (X/S)_{\text{cris}}
$$
whose composition is the identity and of which the first is induced
by the inclusion functor. Moreover, $\pi_* = i^{-1}$.
\end{lemma}

\begin{proof}
For the first assertion see
Lemma \ref{lemma-divided-power-thickening-fibre-products}.
This gives us a morphism of topoi
$i : (X/S)_{\text{cris}} \to (X/S)_{\text{CRIS}}$ and a left adjoint
$i_!$ such that $i^{-1}i_! = i^{-1}i_* = \text{id}$, see
Sites, Lemmas \ref{sites-lemma-when-shriek},
\ref{sites-lemma-preserve-equalizers}, and
\ref{sites-lemma-back-and-forth}.
We claim that $i_!$ is exact. If this is true, then we can define
$\pi$ by the rules $\pi^{-1} = i_!$ and $\pi_* = i^{-1}$
and everything is clear. To prove the claim, note that we already know
that $i_!$ is right exact and preserves fibre products (see references
given). Hence it suffices to show that $i_! * = *$ where $*$ indicates
the final object in the category of sheaves of sets. 
To see this it suffices to produce a set of objects
$(U_i, T_i, \delta_i)$, $i \in I$ of $\text{Cris}(X/S)$ such that
$$
\coprod\nolimits_{i \in I} h_{(U_i, T_i, \delta_i)} \to *
$$
is surjective in $(X/S)_{\text{CRIS}}$ (details omitted; hint: use that
$\text{Cris}(X/S)$ has products and that the functor
$\text{Cris}(X/S) \to \text{CRIS}(X/S)$ commutes with them).
In the affine case this
follows from Lemma \ref{lemma-set-generators}. We omit the proof
in general.
\end{proof}

\begin{remark}[Functoriality]
\label{remark-functoriality-cris}
Let $p$ be a prime number.
Let $(S, \mathcal{I}, \gamma) \to (S', \mathcal{I}', \gamma')$
be a morphism of divided power schemes over $\mathbf{Z}_{(p)}$.
Let
$$
\xymatrix{
X \ar[r]_f \ar[d] & Y \ar[d] \\
S_0 \ar[r] & S'_0
}
$$
be a commutative diagram of morphisms of schemes and assume $p$ is
locally nilpotent on $X$ and $Y$. By analogy with
Topologies, Lemma \ref{topologies-lemma-morphism-big-small} we define
$$
f_{\text{cris}} : (X/S)_{\text{cris}} \longrightarrow (Y/S')_{\text{cris}}
$$
by the formula $f_{\text{cris}} = \pi_Y \circ f_{\text{CRIS}} \circ i_X$
where $i_X$ and $\pi_Y$ are as in Lemma \ref{lemma-compare-big-small}
for $X$ and $Y$ and where $f_{\text{CRIS}}$ is as in
Remark \ref{remark-functoriality-big-cris}.
\end{remark}

\begin{remark}[Comparison with Zariski site]
\label{remark-compare-zariski}
In Situation \ref{situation-global}.
The functor (\ref{equation-forget-small}) is continuous, cocontinuous, and
commutes with products and fibred products.
Hence we obtain a morphism of topoi
$$
u_{X/S} : (X/S)_{\text{cris}} \longrightarrow \Sh(X_{Zar})
$$
relating the small crystalline topos of $X/S$ with
the small Zariski topos of $X$.
See Sites, Section \ref{sites-section-cocontinuous-morphism-topoi}.
\end{remark}

\begin{lemma}
\label{lemma-localize}
In Situation \ref{situation-global}.
Let $X' \subset X$ and $S' \subset S$ be open subschemes such that
$X'$ maps into $S'$. Then there is a fully faithful functor
$\text{Cris}(X'/S') \to \text{Cris}(X/S)$
which gives rise to a morphism of topoi fitting into the commutative
diagram
$$
\xymatrix{
(X'/S')_{\text{cris}} \ar[r] \ar[d]_{u_{X'/S'}} &
(X/S)_{\text{cris}} \ar[d]^{u_{X/S}} \\
\Sh(X'_{Zar}) \ar[r] & \Sh(X_{Zar})
}
$$
Moreover, this diagram is an example of localization of morphisms of
topoi as in Sites, Lemma \ref{sites-lemma-localize-morphism-topoi}.
\end{lemma}

\begin{proof}
The fully faithful functor comes from thinking of
objects of $\text{Cris}(X'/S')$ as divided power
thickenings $(U, T, \delta)$ of $X$ where $U \to X$
factors through $X' \subset X$ (since then automatically $T \to S$
will factor through $S'$). This functor is clearly cocontinuous
hence we obtain a morphism of topoi as indicated.
Let $h_{X'} \in \Sh(X_{Zar})$ be the representable sheaf associated
to $X'$ viewed as an object of $X_{Zar}$. It is clear that
$\Sh(X'_{Zar})$ is the localization $\Sh(X_{Zar})/h_{X'}$.
On the other hand, the category $\text{Cris}(X/S)/u_{X/S}^{-1}h_{X'}$
(see Sites, Lemma \ref{sites-lemma-localize-topos-site})
is canonically identified with $\text{Cris}(X'/S')$ by the functor above.
This finishes the proof.
\end{proof}

\begin{remark}[Structure morphism]
\label{remark-structure-morphism}
In Situation \ref{situation-global}.
Consider the closed subscheme $S_0 = V(\mathcal{I}) \subset S$.
If we assume that $p$ is locally nilpotent on $S_0$ (which is always
the case in practice) then we obtain a situation as in
Definition \ref{definition-divided-power-thickening-X} with $S_0$ instead
of $X$. Hence we get a site $\text{Cris}(S_0/S)$. If $f : X \to S_0$
is the structure morphism of $X$ over $S$, then we get a
commutative diagram of ringed topoi
$$
\xymatrix{
(X/S)_{\text{cris}}
\ar[r]_{f_{\text{cris}}} \ar[d]_{u_{X/S}} &
(S_0/S)_{\text{cris}} \ar[d]^{u_{S_0/S}} \\
\Sh(X_{Zar}) \ar[r]^{f_{small}} & \Sh(S_{0, Zar}) \ar[rd] \\
& & \Sh(S_{Zar})
}
$$
see Remark \ref{remark-functoriality-cris}. We think of the composition
$(X/S)_{\text{cris}} \to \Sh(S_{Zar})$ as the structure morphism of the
crystalline site. Even if $p$ is not locally nilpotent on $S_0$
the structure morphism
$$
\tau_{X/S} : (X/S)_{\text{cris}} \longrightarrow \Sh(S_{Zar})
$$
is defined as we can take the lower route through the diagram above.
\end{remark}

\begin{remark}[Compatibilities]
\label{remark-compatibilities}
The morphisms defined above satisfy numerous compatibilities. For example,
in the situation of Remark \ref{remark-functoriality-cris}
we obtain a commutative diagram of ringed topoi
$$
\xymatrix{
(X/S)_{\text{cris}} \ar[d] \ar[r] & (Y/S')_{\text{cris}} \ar[d] \\
\Sh((\Sch/S)_{Zar}) \ar[r] & \Sh((\Sch/S')_{Zar})
}
$$
where the vertical arrows are the structure morphisms.
\end{remark}



\section{Sheaves on the crystalline site}
\label{section-sheaves}

\noindent
Notation and assumptions as in Situation \ref{situation-global}.
In order to discuss the small and big crystalline sites of $X/S$
simultaneously in this section we let
$$
\mathcal{C} = \text{CRIS}(X/S)
\quad\text{or}\quad
\mathcal{C} = \text{Cris}(X/S).
$$
A sheaf $\mathcal{F}$ on $\mathcal{C}$ gives rise to
a {\it restriction} $\mathcal{F}_T$ for every object $(U, T, \delta)$
of $\mathcal{C}$. Namely, $\mathcal{F}_T$ is the Zariski sheaf on
the scheme $T$ defined by the rule
$$
\mathcal{F}_T(W) = \mathcal{F}(U \cap W, W, \delta|_W)
$$
for $W \subset T$ is open. Moreover, if $f : T \to T'$ is a morphism
between objects
$(U, T, \delta)$ and $(U', T', \delta')$ of $\mathcal{C}$, then there
is a canonical {\it comparison} map
\begin{equation}
\label{equation-comparison}
c_f : f^{-1}\mathcal{F}_{T'} \longrightarrow \mathcal{F}_T.
\end{equation}
Namely, if $W' \subset T'$ is open then $f$ induces a morphism
$$
f|_{f^{-1}W'} :
(U \cap f^{-1}(W'), f^{-1}W', \delta|_{f^{-1}W'})
\longrightarrow
(U' \cap W', W', \delta|_{W'})
$$
of $\mathcal{C}$, hence we can use the restriction mapping
$(f|_{f^{-1}W'})^*$ of $\mathcal{F}$ to define a map
$\mathcal{F}_{T'}(W') \to \mathcal{F}_T(f^{-1}W')$.
These maps are clearly compatible with further restriction, hence
define an $f$-map from $\mathcal{F}_{T'}$ to $\mathcal{F}_T$ (see
Sheaves, Section \ref{sheaves-section-presheaves-functorial}
and especially
Sheaves, Definition \ref{sheaves-definition-f-map}).
Thus a map $c_f$ as in (\ref{equation-comparison}).
Note that if $f$ is an open immersion, then $c_f$ is an
isomorphism, because in that case $\mathcal{F}_T$ is just
the restriction of $\mathcal{F}_{T'}$ to $T$.

\medskip\noindent
Conversely, given Zariski sheaves $\mathcal{F}_T$ for every object
$(U, T, \delta)$ of $\mathcal{C}$ and comparison maps
$c_f$ as above which (a) are isomorphisms for open immersions, and (b)
satisfy a suitable cocycle condition, we obtain a sheaf on
$\mathcal{C}$. This is proved exactly as in
Topologies, Lemma \ref{topologies-lemma-characterize-sheaf-big}.

\medskip\noindent
The {\it structure sheaf} on $\mathcal{C}$ is the sheaf
$\mathcal{O}_{X/S}$ defined by the rule
$$
\mathcal{O}_{X/S} :
(U, T, \delta)
\longmapsto
\Gamma(T, \mathcal{O}_T)
$$
This is a sheaf by the definition of coverings in $\mathcal{C}$.
Suppose that $\mathcal{F}$ is a sheaf of $\mathcal{O}_{X/S}$-modules.
In this case the comparison mappings (\ref{equation-comparison})
define a comparison map
\begin{equation}
\label{equation-comparison-modules}
c_f : f^*\mathcal{F}_{T'} \longrightarrow \mathcal{F}_T
\end{equation}
of $\mathcal{O}_T$-modules.

\medskip\noindent
Another type of example comes by starting with a sheaf
$\mathcal{G}$ on $(\Sch/X)_{Zar}$ or $X_{Zar}$ (depending on whether
$\mathcal{C} = \text{CRIS}(X/S)$ or $\mathcal{C} = \text{Cris}(X/S)$).
Then $\underline{\mathcal{G}}$ defined by the rule
$$
\underline{\mathcal{G}} :
(U, T, \delta)
\longmapsto
\mathcal{G}(U)
$$
is a sheaf on $\mathcal{C}$. In particular, if we take
$\mathcal{G} = \mathbf{G}_a = \mathcal{O}_X$, then we obtain
$$
\underline{\mathbf{G}_a} :
(U, T, \delta)
\longmapsto
\Gamma(U, \mathcal{O}_U)
$$
There is a surjective map of sheaves
$\mathcal{O}_{X/S} \to \underline{\mathbf{G}_a}$ defined by the
canonical maps $\Gamma(T, \mathcal{O}_T) \to \Gamma(U, \mathcal{O}_U)$
for objects $(U, T, \delta)$. The kernel of this map is denoted
$\mathcal{J}_{X/S}$, hence a short exact sequence
$$
0 \to
\mathcal{J}_{X/S} \to
\mathcal{O}_{X/S} \to
\underline{\mathbf{G}_a} \to 0
$$
Note that $\mathcal{J}_{X/S}$ comes equipped with a canonical
divided power structure. After all, for each object $(U, T, \delta)$
the third component $\delta$ {\it is} a divided power structure on the
kernel of $\mathcal{O}_T \to \mathcal{O}_U$. Hence the (big)
crystalline topos is a divided power topos.





\section{Crystals in modules}
\label{section-crystals}

\noindent
It turns out that a crystal is a very general gadget. However, the
definition may be a bit hard to parse, so we first give the definition
in the case of modules on the crystalline sites.

\begin{definition}
\label{definition-modules}
In Situation \ref{situation-global}.
Let $\mathcal{C} = \text{CRIS}(X/S)$ or $\mathcal{C} = \text{Cris}(X/S)$.
Let $\mathcal{F}$ be a sheaf of $\mathcal{O}_{X/S}$-modules on $\mathcal{C}$.
\begin{enumerate}
\item We say $\mathcal{F}$ is {\it locally quasi-coherent} if for every
object $(U, T, \delta)$ of $\mathcal{C}$ the restriction $\mathcal{F}_T$
is a quasi-coherent $\mathcal{O}_T$-module.
\item We say $\mathcal{F}$ is {\it quasi-coherent} if it is quasi-coherent
in the sense of
Modules on Sites, Definition \ref{sites-modules-definition-site-local}.
\item We say $\mathcal{F}$ is a {\it crystal in $\mathcal{O}_{X/S}$-modules}
if all the comparison maps (\ref{equation-comparison-modules}) are
isomorphisms.
\end{enumerate}
\end{definition}

\noindent
It turns out that we can relate these notions as follows.

\begin{lemma}
\label{lemma-crystal-quasi-coherent-modules}
With notation $X/S, \mathcal{I}, \gamma, \mathcal{C}, \mathcal{F}$
as in Definition \ref{definition-modules}. The following are equivalent
\begin{enumerate}
\item $\mathcal{F}$ is quasi-coherent, and
\item $\mathcal{F}$ is locally quasi-coherent and a crystal in
$\mathcal{O}_{X/S}$-modules.
\end{enumerate}
\end{lemma}

\begin{proof}
Assume (1). Let $f : (U', T', \delta') \to (U, T, \delta)$ be an object of
$\mathcal{C}$. We have to prove (a) $\mathcal{F}_T$ is a quasi-coherent
$\mathcal{O}_T$-module and (b) $c_f : f^*\mathcal{F}_T \to \mathcal{F}_{T'}$
is an isomorphism. The assumption means that we can find a covering
$\{(T_i, U_i, \delta_i) \to (T, U, \delta)\}$ and for each $i$
the restriction of $\mathcal{F}$ to $\mathcal{C}/(T_i, U_i, \delta_i)$
has a global presentation. Since it suffices to prove (a) and (b)
Zariski locally, we may replace $f : (T', U', \delta') \to (T, U, \delta)$
by the base change to $(T_i, U_i, \delta_i)$ and assume that $\mathcal{F}$
restricted to $\mathcal{C}/(T, U, \delta)$ has a global
presentation
$$
\bigoplus\nolimits_{j \in J}
\mathcal{O}_{X/S}|_{\mathcal{C}/(U, T, \delta)} \longrightarrow
\bigoplus\nolimits_{i \in I}
\mathcal{O}_{X/S}|_{\mathcal{C}/(U, T, \delta)} \longrightarrow
\mathcal{F}|_{\mathcal{C}/(U, T, \delta)}
\longrightarrow 0
$$
It is clear that this gives a presentation
$$
\bigoplus\nolimits_{j \in J} \mathcal{O}_T \longrightarrow
\bigoplus\nolimits_{i \in I} \mathcal{O}_T \longrightarrow
\mathcal{F}_T
\longrightarrow 0
$$
and hence (a) holds. Moreover, the presentation restricts to $T'$
to give a similar presentation of $\mathcal{F}_{T'}$, whence (b) holds.

\medskip\noindent
Assume (2). Let $(U, T, \delta)$ be an object of $\mathcal{C}$.
We have to find a covering of $(U, T, \delta)$ such that $\mathcal{F}$ has a
global presentation when we restrict to the localization of $\mathcal{C}$
at the members of the covering. Thus we may assume that $T$ is affine.
In this case we can choose a presentation
$$
\bigoplus\nolimits_{j \in J} \mathcal{O}_T \longrightarrow
\bigoplus\nolimits_{i \in I} \mathcal{O}_T \longrightarrow
\mathcal{F}_T
\longrightarrow 0
$$
as $\mathcal{F}_T$ is assumed to be a quasi-coherent $\mathcal{O}_T$-module.
Then by the crystal property of $\mathcal{F}$ we see that this pulls back
to a presentation of $\mathcal{F}_{T'}$ for any morphism
$f : (U', T', \delta') \to (U, T, \delta)$ of $\mathcal{C}$.
Thus the desired presentation of $\mathcal{F}|_{\mathcal{C}/(U, T, \delta)}$.
\end{proof}

\begin{definition}
\label{definition-crystal-quasi-coherent-modules}
If $\mathcal{F}$ satisfies the equivalent conditions of
Lemma \ref{lemma-crystal-quasi-coherent-modules}, then
we say that $\mathcal{F}$ is a
{\it crystal in quasi-coherent modules}.
We say that $\mathcal{F}$ is a {\it crystal in finite locally free modules}
if, in addition, $\mathcal{F}$ is finite locally free.
\end{definition}

\noindent
Of course, as Lemma \ref{lemma-crystal-quasi-coherent-modules} shows, this
notation is somewhat heavy since a quasi-coherent module is always a crystal.
But it is standard terminology in the literature.

\begin{remark}
\label{remark-crystal}
To formulate the general notion of a crystal we use the language
of stacks and strongly cartesian morphisms, see
Stacks, Definition \ref{stacks-definition-stack} and
Categories, Definition \ref{categories-definition-cartesian-over-C}.
In Situation \ref{situation-global} let
$p : \mathcal{C} \to \text{Cris}(X/S)$ be a stack.
A {\it crystal in objects of $\mathcal{C}$ on $X$ relative to $S$}
is a {\it cartesian section} $\sigma : \text{Cris}(X/S) \to \mathcal{C}$,
i.e., a functor $\sigma$ such that $p \circ \sigma = \text{id}$
and such that $\sigma(f)$ is strongly cartesian for all
morphisms $f$ of $\text{Cris}(X/S)$. Similarly for the big crystalline site.
\end{remark}





\section{Sheaf of differentials}
\label{section-differentials-sheaf}

\noindent
In this section we will stick with the (small) crystalline site
as it seems more natural. We globalize
Definition \ref{definition-derivation} as follows.

\begin{definition}
\label{definition-global-derivation}
In Situation \ref{situation-global} let
$\mathcal{F}$ be a sheaf of $\mathcal{O}_{X/S}$-modules on
$\text{Cris}(X/S)$. An
{\it $S$-derivation $D : \mathcal{O}_{X/S} \to \mathcal{F}$}
is a map of sheaves such that for every object $(U, T, \delta)$ of
$\text{Cris}(X/S)$ the map
$$
D : \Gamma(T, \mathcal{O}_T) \longrightarrow \Gamma(T, \mathcal{F})
$$
is a divided power $\Gamma(V, \mathcal{O}_V)$-derivation where $V \subset S$
is any open such that $T \to S$ factors through $V$.
\end{definition}

\noindent
This means that $D$ is additive, satisfies the Leibniz rule, annihilates
functions coming from $S$, and satisfies $D(f^{[n]}) = f^{[n - 1]}D(f)$
for a local section $f$ of the divided power ideal $\mathcal{J}_{X/S}$.
This is a special case of a very general notion which we now describe.

\medskip\noindent
Please compare the following discussion with
Modules on Sites, Section \ref{sites-modules-section-differentials}. Let
$\mathcal{C}$ be a site, let $\mathcal{A} \to \mathcal{B}$ be a
map of sheaves of rings on $\mathcal{C}$, let $\mathcal{J} \subset \mathcal{B}$
be a sheaf of ideals, let $\delta$ be a divided power structure on
$\mathcal{J}$, and let $\mathcal{F}$ be a sheaf of $\mathcal{B}$-modules.
Then there is a notion of a {\it divided power $\mathcal{A}$-derivation}
$D : \mathcal{B} \to \mathcal{F}$. This means that $D$ is $\mathcal{A}$-linear,
satisfies the Leibniz rule, and satisfies
$D(\delta_n(x)) = \delta_{n - 1}(x)D(x)$ for local sections $x$ of
$\mathcal{J}$. In this situation there exists a
{\it universal divided power $\mathcal{A}$-derivation}
$$
\text{d}_{\mathcal{B}/\mathcal{A}, \delta} :
\mathcal{B}
\longrightarrow
\Omega_{\mathcal{B}/\mathcal{A}, \delta}
$$
Moreover, $\text{d}_{\mathcal{B}/\mathcal{A}, \delta}$ is the composition
$$
\mathcal{B}
\longrightarrow
\Omega_{\mathcal{B}/\mathcal{A}}
\longrightarrow
\Omega_{\mathcal{B}/\mathcal{A}, \delta}
$$
where the first map is the universal derivation constructed in the proof
of Modules on Sites, Lemma \ref{sites-modules-lemma-universal-module}
and the second arrow is the quotient by the submodule generated by
the local sections
$\text{d}_{\mathcal{B}/\mathcal{A}}(\delta_n(x)) -
\delta_{n - 1}(x)\text{d}_{\mathcal{B}/\mathcal{A}}(x)$.

\medskip\noindent
We translate this into a relative notion as follows. Suppose
$(f, f^\sharp) : (\Sh(\mathcal{C}), \mathcal{O}) \to
(\Sh(\mathcal{C}'), \mathcal{O}')$ is a morphism of ringed topoi,
$\mathcal{J} \subset \mathcal{O}$ a sheaf of ideals, $\delta$ a
divided power structure on $\mathcal{J}$, and $\mathcal{F}$ a sheaf
of $\mathcal{O}$-modules. In this situation we say
$D : \mathcal{O} \to \mathcal{F}$ is a divided power $\mathcal{O}'$-derivation
if $D$ is a divided power $f^{-1}\mathcal{O}'$-derivation as defined above.
Moreover, we write
$$
\Omega_{\mathcal{O}/\mathcal{O}', \delta} =
\Omega_{\mathcal{O}/f^{-1}\mathcal{O}', \delta}
$$
which is the receptacle of the universal divided power
$\mathcal{O}'$-derivation.

\medskip\noindent
Applying this to the structure morphism
$$
(X/S)_{\text{Cris}} \longrightarrow \Sh(S_{Zar})
$$
(see Remark \ref{remark-structure-morphism}) we recover the notion of
Definition \ref{definition-global-derivation} above.
In particular, there is a universal divided power derivation
$$
d_{X/S} : \mathcal{O}_{X/S} \to \Omega_{X/S}
$$
Note that we omit from the notation the decoration indicating the
module of differentials is compatible with divided powers (it seems
unlikely anybody would ever consider the usual module of differentials
of the structure sheaf on the crystalline site).

\begin{lemma}
\label{lemma-module-differentials-divided-power-scheme}
Let $(T, \mathcal{J}, \delta)$ be a divided power scheme.
Let $T \to S$ be a morphism of schemes.
The quotient $\Omega_{T/S} \to \Omega_{T/S, \delta}$
described above is a quasi-coherent $\mathcal{O}_T$-module.
For $W \subset T$ affine open mapping into $V \subset S$ affine open
we have
$$
\Gamma(W, \Omega_{T/S, \delta}) =
\Omega_{\Gamma(W, \mathcal{O}_W)/\Gamma(V, \mathcal{O}_V), \delta}
$$
where the right hand side is
as constructed in Section \ref{section-differentials}.
\end{lemma}

\begin{proof}
Omitted.
\end{proof}

\begin{lemma}
\label{lemma-module-of-differentials}
In Situation \ref{situation-global}.
For $(U, T, \delta)$ in $\text{Cris}(X/S)$ the restriction
$(\Omega_{X/S})_T$ to $T$ is $\Omega_{T/S, \delta}$ and the restriction
$\text{d}_{X/S}|_T$ is equal to $\text{d}_{T/S, \delta}$.
\end{lemma}

\begin{proof}
Omitted.
\end{proof}

\begin{lemma}
\label{lemma-module-of-differentials-on-affine}
In Situation \ref{situation-global}.
For any affine object $(U, T, \delta)$ of $\text{Cris}(X/S)$
mapping into an affine open $V \subset S$ we have
$$
\Gamma((U, T, \delta), \Omega_{X/S}) =
\Omega_{\Gamma(T, \mathcal{O}_T)/\Gamma(V, \mathcal{O}_V), \delta}
$$
where the right hand side is
as constructed in Section \ref{section-differentials}.
\end{lemma}

\begin{proof}
Combine Lemmas \ref{lemma-module-differentials-divided-power-scheme} and
\ref{lemma-module-of-differentials}.
\end{proof}

\begin{lemma}
\label{lemma-describe-omega-small}
In Situation \ref{situation-global}.
Let $(U, T, \delta)$ be an object of $\text{Cris}(X/S)$.
Let
$$
(U(1), T(1), \delta(1)) = (U, T, \delta) \times (U, T, \delta)
$$
in $\text{Cris}(X/S)$. Let $\mathcal{K} \subset \mathcal{O}_{T(1)}$
be the quasi-coherent sheaf of ideals corresponding to the closed
immersion $\Delta : T \to T(1)$. Then
$\mathcal{K} \subset \mathcal{J}_{T(1)}$ is preserved by the
divided structure on $\mathcal{J}_{T(1)}$ and we have
$$
(\Omega_{X/S})_T = \mathcal{K}/\mathcal{K}^{[2]}
$$
\end{lemma}

\begin{proof}
Note that $U = U(1)$ as $U \to X$ is an open immersion and as
(\ref{equation-forget-small}) commutes with products. Hence we see that
$\mathcal{K} \subset \mathcal{J}_{T(1)}$. Given this fact the lemma follows
by working affine locally on $T$ and using
Lemmas \ref{lemma-module-of-differentials-on-affine} and
\ref{lemma-diagonal-and-differentials-affine-site}.
\end{proof}

\noindent
It turns out that $\Omega_{X/S}$ is not a crystal in quasi-coherent
$\mathcal{O}_{X/S}$-modules. But it does satisfy two closely
related properties (compare with
Lemma \ref{lemma-crystal-quasi-coherent-modules}).

\begin{lemma}
\label{lemma-omega-locally-quasi-coherent}
In Situation \ref{situation-global}.
The sheaf of differentials $\Omega_{X/S}$ has the following two
properties:
\begin{enumerate}
\item $\Omega_{X/S}$ is locally quasi-coherent, and
\item for any morphism $(U, T, \delta) \to (U', T', \delta')$
of $\text{Cris}(X/S)$ where $f : T \to T'$ is a closed immersion
the map $c_f : f^*(\Omega_{X/S})_{T'} \to (\Omega_{X/S})_T$ is surjective.
\end{enumerate}
\end{lemma}

\begin{proof}
Part (1) follows from a combination of
Lemmas \ref{lemma-module-differentials-divided-power-scheme} and
\ref{lemma-module-of-differentials}.
Part (2) follows from the fact that
$(\Omega_{X/S})_T = \Omega_{T/S, \delta}$
is a quotient of $\Omega_{T/S}$ and that $f^*\Omega_{T'/S} \to \Omega_{T/S}$
is surjective.
\end{proof}







\section{Two universal thickenings}
\label{section-universal-thickenings}

\noindent
The constructions in this section will help us define a connection on
a crystal in modules on the crystalline site. In some sense the constructions
here are the ``sheafified, universal'' versions of the constructions in
Section \ref{section-explicit-thickenings}.

\begin{remark}
\label{remark-first-order-thickening}
In Situation \ref{situation-global}.
Let $(U, T, \delta)$ be an object of $\text{Cris}(X/S)$.
Write $\Omega_{T/S, \delta} = (\Omega_{X/S})_T$, see
Lemma \ref{lemma-module-of-differentials}.
We explicitly describe a first order thickening $T'$ of
$T$. Namely, set
$$
\mathcal{O}_{T'} = \mathcal{O}_T \oplus \Omega_{T/S, \delta}
$$
with algebra structure such that $\Omega_{T/S, \delta}$ is an
ideal of square zero. Let $\mathcal{J} \subset \mathcal{O}_T$
be the ideal sheaf of the closed immersion $U \to T$. Set
$\mathcal{J}' = \mathcal{J} \oplus \Omega_{T/S, \delta}$.
Define a divided power structure on $\mathcal{J}'$ by setting
$$
\delta_n'(f, \omega) = (\delta_n(f), \delta_{n - 1}(f)\omega),
$$
see Lemma \ref{lemma-divided-power-first-order-thickening}.
There are two ring maps
$$
p_0, p_1 : \mathcal{O}_T \to \mathcal{O}_{T'}
$$
The first is given by $f \mapsto (f, 0)$ and the second by
$f \mapsto (f, \text{d}_{T/S, \delta}f)$. Note that both are compatible
with the divided power structures on $\mathcal{J}$ and $\mathcal{J}'$
and so is the quotient map $\mathcal{O}_{T'} \to \mathcal{O}_T$.
Thus we get an object $(U, T', \delta')$ of $\text{Cris}(X/S)$
and a commutative diagram
$$
\xymatrix{
& T \ar[ld]_{\text{id}} \ar[d]^i \ar[rd]^{\text{id}} \\
T & T' \ar[l]_{p_0} \ar[r]^{p_1} & T
}
$$
of $\text{Cris}(X/S)$ such that $i$ is a first order thickening whose ideal
sheaf is identified with $\Omega_{T/S, \delta}$ and such that
$p_1^* - p_0^* : \mathcal{O}_T \to \mathcal{O}_{T'}$
is identified with the universal derivation $\text{d}_{T/S, \delta}$
composed with the inclusion $\Omega_{T/S, \delta} \to \mathcal{O}_{T'}$.
\end{remark}

\begin{remark}
\label{remark-second-order-thickening}
In Situation \ref{situation-global}.
Let $(U, T, \delta)$ be an object of $\text{Cris}(X/S)$.
Write $\Omega_{T/S, \delta} = (\Omega_{X/S})_T$, see
Lemma \ref{lemma-module-of-differentials}.
We also write $\Omega^2_{T/S, \delta}$ for its second exterior
power. We explicitly describe a second order thickening $T''$ of $T$.
Namely, set
$$
\mathcal{O}_{T''} =
\mathcal{O}_T \oplus \Omega_{T/S, \delta} \oplus \Omega_{T/S, \delta}
\oplus \Omega^2_{T/S, \delta}
$$
with algebra structure defined in the following way
$$
(f, \omega_1, \omega_2, \eta) \cdot
(f', \omega_1', \omega_2', \eta') =
(ff', f\omega_1' + f'\omega_1, f\omega_2' + f'\omega_2,
f\eta' + f'\eta + \omega_1 \wedge \omega_2' + \omega_1' \wedge \omega_2).
$$
Let $\mathcal{J} \subset \mathcal{O}_T$
be the ideal sheaf of the closed immersion $U \to T$. Let
$\mathcal{J}''$ be the inverse image of $\mathcal{J}$ under the
projection $\mathcal{O}_{T''} \to \mathcal{O}_T$.
Define a divided power structure on $\mathcal{J}''$ by setting
$$
\delta_n''(f, \omega_1, \omega_2, \eta) =
(\delta_n(f), \delta_{n - 1}(f)\omega_1, \delta_{n - 1}(f)\omega_2,
\delta_{n - 1}(f)\eta + \delta_{n - 2}(f)\omega_1 \wedge \omega_2)
$$
see Lemma \ref{lemma-divided-power-second-order-thickening}.
There are three ring maps
$q_0, q_1, q_2 : \mathcal{O}_T \to \mathcal{O}_{T''}$
given by
\begin{align*}
q_0(f) & = (f, 0, 0, 0), \\
q_1(f) & = (f, \text{d}f, 0, 0), \\
q_2(f) & = (f, \text{d}f, \text{d}f, 0)
\end{align*}
where $\text{d} = \text{d}_{T/S, \delta}$.
Note that all three are compatible with the divided power structures
on $\mathcal{J}$ and $\mathcal{J}''$. There are three ring maps
$q_{01}, q_{12}, q_{02} : \mathcal{O}_{T'} \to \mathcal{O}_{T''}$
where $\mathcal{O}_{T'}$ is as in Remark \ref{remark-first-order-thickening}.
Namely, set
\begin{align*}
q_{01}(f, \omega) & = (f, \omega, 0, 0), \\
q_{12}(f, \omega) & =
(f, \text{d}f, \omega, \text{d}\omega), \\
q_{02}(f, \omega) & = (f, \omega, \omega, 0)
\end{align*}
These are also compatible with the given divided power
structures. Let's do the verifications for $q_{12}$: Note
that $q_{12}$ is a ring homomorphism as
\begin{align*}
q_{12}(f, \omega)q_{12}(g, \eta) & =
(f, \text{d}f, \omega, \text{d}\omega)(g, \text{d}g, \eta, \text{d}\eta) \\
& =
(fg, f\text{d}g + g \text{d}f, f\eta + g\omega,
f\text{d}\eta + g\text{d}\omega + \text{d}f \wedge \eta +
\text{d}g \wedge \omega) \\
& = q_{12}(fg, f\eta + g\omega) = q_{12}((f, \omega)(g, \eta))
\end{align*}
Note that $q_{12}$ is compatible with divided powers because
\begin{align*}
\delta_n''(q_{12}(f, \omega)) & =
\delta_n''((f, \text{d}f, \omega, \text{d}\omega)) \\
& =
(\delta_n(f), \delta_{n - 1}(f)\text{d}f, \delta_{n - 1}(f)\omega,
\delta_{n - 1}(f)\text{d}\omega + \delta_{n - 2}(f)\text{d}(f) \wedge \omega)
\\
& = q_{12}((\delta_n(f), \delta_{n - 1}(f)\omega)) =
q_{12}(\delta'_n(f, \omega))
\end{align*}
The verifications for $q_{01}$ and $q_{02}$ are easier.
Note that $q_0 = q_{01} \circ p_0$, $q_1 = q_{01} \circ p_1$,
$q_1 = q_{12} \circ p_0$, $q_2 = q_{12} \circ p_1$,
$q_0 = q_{02} \circ p_0$, and $q_2 = q_{02} \circ p_1$.
Thus $(U, T'', \delta'')$ is an object of $\text{Cris}(X/S)$
and we get morphisms
$$
\xymatrix{
T''
\ar@<2ex>[r]
\ar@<0ex>[r]
\ar@<-2ex>[r]
&
T'
\ar@<1ex>[r]
\ar@<-1ex>[r]
&
T
}
$$
of $\text{Cris}(X/S)$ satisfying the relations described above.
In applications we will use $q_i : T'' \to T$ and
$q_{ij} : T'' \to T'$ to denote the morphisms associated to the
ring maps described above.
\end{remark}






\section{The de Rham complex}
\label{section-de-Rham}

\noindent
In Situation \ref{situation-global}.
Working on the (small) crystalline site, we define
$\Omega^i_{X/S} = \wedge^i_{\mathcal{O}_{X/S}} \Omega_{X/S}$
for $i \geq 0$. The universal $S$-derivation $\text{d}_{X/S}$ gives
rise to the {\it de Rham complex}
$$
\mathcal{O}_{X/S} \to \Omega^1_{X/S} \to \Omega^2_{X/S} \to \ldots
$$
on $\text{Cris}(X/S)$, see
Lemma \ref{lemma-module-of-differentials-on-affine} and
Remark \ref{remark-divided-powers-de-rham-complex}.


\section{Connections}
\label{section-connections}

\noindent
In Situation \ref{situation-global}.
Given an $\mathcal{O}_{X/S}$-module $\mathcal{F}$ on $\text{Cris}(X/S)$
a {\it connection} is a map of abelian sheaves
$$
\nabla :
\mathcal{F}
\longrightarrow
\mathcal{F} \otimes_{\mathcal{O}_{X/S}} \Omega_{X/S}
$$
such that $\nabla(f s) = f\nabla(s) + s \otimes \text{d}f$
for local sections $s, f$ of $\mathcal{F}$ and $\mathcal{O}_{X/S}$.
Given a connection there are canonical maps
$
\nabla :
\mathcal{F} \otimes_{\mathcal{O}_{X/S}} \Omega^i_{X/S}
\longrightarrow
\mathcal{F} \otimes_{\mathcal{O}_{X/S}} \Omega^{i + 1}_{X/S}
$
defined by the rule $\nabla(s \otimes \omega) =
\nabla(s) \wedge \omega + s \otimes \text{d}\omega$
as in Remark \ref{remark-connection}. We say the connection is
{\it integrable} if $\nabla \circ \nabla = 0$. If $\nabla$ is integrable
we obtain the {\it de Rham complex}
$$
\mathcal{F} \to
\mathcal{F} \otimes_{\mathcal{O}_{X/S}} \Omega^1_{X/S} \to
\mathcal{F} \otimes_{\mathcal{O}_{X/S}} \Omega^2_{X/S} \to \ldots
$$
on $\text{Cris}(X/S)$. It turns out that any crystal in
$\mathcal{O}_{X/S}$-modules comes equipped with a canonical
integrable connection.

\begin{lemma}
\label{lemma-automatic-connection}
In Situation \ref{situation-global}.
Let $\mathcal{F}$ be a crystal in $\mathcal{O}_{X/S}$-modules
on $\text{Cris}(X/S)$. Then $\mathcal{F}$ comes equipped with a
canonical integrable connection.
\end{lemma}

\begin{proof}
Say $(U, T, \delta)$ is an object of $\text{Cris}(X/S)$.
Let $(U, T', \delta')$ be the infinitesimal thickening of $T$
by $(\Omega_{X/S})_T = \Omega_{T/S, \delta}$
constructed in Remark \ref{remark-first-order-thickening}.
It comes with projections $p_0, p_1 : T' \to T$
and a diagonal $i : T \to T'$. By assumption we get
isomorphisms
$$
p_0^*\mathcal{F}_T \xrightarrow{c_0}
\mathcal{F}_{T'} \xleftarrow{c_1}
p_1^*\mathcal{F}_T
$$
of $\mathcal{O}_{T'}$-modules. Pulling $c = c_1^{-1} \circ c_0$
back to $T$ by $i$ we obtain the identity map
of $\mathcal{F}_T$. Hence if $s \in \Gamma(T, \mathcal{F}_T)$
then $\nabla(s) = p_1^*s - c(p_0^*s)$ is a section of
$p_1^*\mathcal{F}_T$ which vanishes on pulling back by $i$. Hence
$\nabla(s)$ is a section of
$$
\mathcal{F}_T
\otimes_{\mathcal{O}_T}
\Omega_{T/S, \delta}
$$
because this is the kernel of $p_1^*\mathcal{F}_T \to \mathcal{F}_T$
as $\mathcal{O}_{T'} = \mathcal{O}_T \oplus \Omega_{T/S, \delta}$
by construction. It is easily verified that $\nabla(fs) =
f\nabla(s) + s \otimes \text{d}(f)$ using the description of
$\text{d}$ in Remark \ref{remark-first-order-thickening}.

\medskip\noindent
The collection of maps
$$
\nabla : \Gamma(T, \mathcal{F}_T) \to
\Gamma(T, \mathcal{F}_T \otimes_{\mathcal{O}_T} \Omega_{T/S, \delta})
$$
so obtained is functorial in $T$ because the construction of $T'$
is functorial in $T$. Hence we obtain a connection.

\medskip\noindent
To show that the connection is integrable we consider the
object $(U, T'', \delta'')$ constructed in
Remark \ref{remark-second-order-thickening}.
Because $\mathcal{F}$ is a sheaf we see that
$$
\xymatrix{
q_0^*\mathcal{F}_T \ar[rr]_{q_{01}^*c} \ar[rd]_{q_{02}^*c} & &
q_1^*\mathcal{F}_T \ar[ld]^{q_{12}^*c} \\
& q_2^*\mathcal{F}_T
}
$$
is a commutative diagram of $\mathcal{O}_{T''}$-modules. For
$s \in \Gamma(T, \mathcal{F}_T)$ we have
$c(p_0^*s) = p_1^*s - \nabla(s)$. Write
$\nabla(s) = \sum p_1^*s_i \cdot \omega_i$ where $s_i$ is a local section
of $\mathcal{F}_T$ and $\omega_i$ is a local section of $\Omega_{T/S, \delta}$.
We think of $\omega_i$ as a local section of the structure
sheaf of $\mathcal{O}_{T'}$ and hence we write product instead of tensor
product. On the one hand
\begin{align*}
q_{12}^*c \circ q_{01}^*c(q_0^*s) & = 
q_{12}^*c(q_1^*s - \sum q_1^*s_i \cdot q_{01}^*\omega_i) \\
& =
q_2^*s - \sum q_2^*s_i \cdot q_{12}^*\omega_i -
\sum q_2^*s_i \cdot q_{01}^*\omega_i +
\sum q_{12}^*\nabla(s_i) \cdot q_{01}^*\omega_i
\end{align*}
and on the other hand
$$
q_{02}^*c(q_0^*s) = q_2^*s - \sum q_2^*s_i \cdot q_{02}^*\omega_i.
$$
From the formulae of Remark \ref{remark-second-order-thickening} we see
that
$q_{01}^*\omega_i + q_{12}^*\omega_i - q_{02}^*\omega_i = \text{d}\omega_i$.
Hence the difference of the two expressions above is
$$
\sum q_2^*s_i \cdot \text{d}\omega_i -
\sum q_{12}^*\nabla(s_i) \cdot q_{01}^*\omega_i
$$
Note that
$q_{12}^*\omega \cdot q_{01}^*\omega' = \omega' \wedge \omega =
- \omega \wedge \omega'$ by the definition of the multiplication on
$\mathcal{O}_{T''}$. Thus the expression above is $\nabla^2(s)$ viewed
as a section of the subsheaf $\mathcal{F}_T \otimes \Omega^2_{T/S, \delta}$ of
$q_2^*\mathcal{F}$. Hence we get the integrability condition.
\end{proof}




\section{Cosimplicial algebra}
\label{section-cosimplicial}

\noindent
This section should be moved somewhere else. A
{\it cosimplicial ring} is a cosimplicial object
in the category of rings. Given a ring $R$, a
{\it cosimplicial $R$-algebra} is a cosimplicial object in the
category of $R$-algebras. A {\it cosimplicial ideal} in a cosimplicial
ring $A_*$ is given by an ideal $I_n \subset A_n$ for all $n$ such
that $A(f)(I_n) \subset I_m$ for all $f : [n] \to [m]$ in $\Delta$.

\medskip\noindent
Let $A_*$ be a cosimplicial ring. Let $\mathcal{C}$ be the category
of pairs $(A, M)$ where $A$ is a ring and $M$ is a module over $A$.
A morphism $(A, M) \to (A', M')$ consists of a ring map $A \to A'$ and
an $A$-module map $M \to M'$ where $M'$ is viewed as an $A$-module
via $A \to A'$ and the $A'$-module structure on $M'$. Having said this
we can define a {\it cosimplicial module $M_*$ over $A_*$} as a cosimplicial
object $(A_*, M_*)$ of $\mathcal{C}$ whose first entry is equal to $A_*$.
A {\it homomorphism $\varphi_* : M_* \to N_*$ of cosimplicial modules over
$A_*$} is a morphism $(A_*, M_*) \to (A_*, N_*)$ of cosimplicial objects
in $\mathcal{C}$ whose first component is $1_{A_*}$.

\medskip\noindent
A {\it homotopy} between homomorphisms $\varphi_*, \psi_* : M_* \to N_*$
of cosimplicial modules over $A_*$ is a homotopy between the associated
maps $(A_*, M_*) \to (A_*, N_*)$ whose first component is the
trivial homotopy (dual to
Simplicial, Example \ref{simplicial-example-trivial-homotopy}).
We spell out what this means. Such a homotopy is a homotopy
$$
h : M_* \longrightarrow \Hom(\Delta[1], N_*)
$$
between $\varphi_*$ and $\psi_*$ as homomorphisms of cosimplicial abelian
groups such that for each $n$ the map
$h_n : M_n \to \prod_{\alpha \in \Delta[1]_n} N_n$ is $A_n$-linear.
The following lemma is a version of
Simplicial, Lemma \ref{simplicial-lemma-functorial-homotopy}
for cosimplicial modules.

\begin{lemma}
\label{lemma-homotopy-tensor}
Let $A_*$ be a cosimplicial ring. Let $\varphi_*, \psi_* : K_* \to M_*$
be homomorphisms of cosimplicial $A_*$-modules.
\begin{enumerate}
\item
\label{item-tensor}
If $\varphi_*$ and $\psi_*$ are homotopic, then
$$
\varphi_* \otimes 1, \psi_* \otimes 1 :
K_* \otimes_{A_*} L_* \longrightarrow M_* \otimes_{A_*} L_*
$$
are homotopic for any cosimplicial $A_*$-module $L_*$.
\item
\label{item-wedge}
If $\varphi_*$ and $\psi_*$ are homotopic, then
$$
\wedge^i(\varphi_*), \wedge^i(\psi_*) :
\wedge^i(K_*) \longrightarrow \wedge^i(M_*)
$$
are homotopic.
\item
\label{item-base-change}
If $\varphi_*$ and $\psi_*$ are homotopic, and $A_* \to B_*$
is a homomorphism of cosimplicial rings, then
$$
\varphi_* \otimes 1, \psi_* \otimes 1 :
K_* \otimes_{A_*} B_* \longrightarrow M_* \otimes_{A_*} B_*
$$
are homotopic as homomorphisms of cosimplicial $B_*$-modules.
\item
\label{item-completion}
If $I_* \subset A_*$ is a cosimplicial ideal, then the induced
maps
$$
\varphi^\wedge_*, \psi^\wedge_* :
K_*^\wedge \longrightarrow M_*^\wedge
$$
between completions are homotopic.
\item Add more here as needed, for example symmetric powers.
\end{enumerate}
\end{lemma}

\begin{proof}
Let $h : M_* \longrightarrow \Hom(\Delta[1], N_*)$ be the given
homotopy. In degree $n$ we have
$$
h_n = (h_{n, \alpha}) :
K_n \longrightarrow
\prod\nolimits_{\alpha \in \Delta[1]_n} K_n
$$
see Simplicial, Section \ref{simplicial-section-homotopy-cosimplicial}.
In order for a collection of $h_{n, \alpha}$ to form a homotopy,
it is necessary and sufficient if for every $f : [n] \to [m]$ we
have
$$
h_{m, \alpha} \circ M_*(f) = N_*(f) \circ h_{n, \alpha \circ f}
$$
see
Simplicial, Equation (\ref{simplicial-equation-property-homotopy-cosimplicial}).
We also should have that $\psi_n = h_{n, 0 : [n] \to [1]}$ and
$\varphi_n = h_{n, 1 : [n] \to [1]}$.

\medskip\noindent
In each of the cases of the lemma we can produce the corresponding maps.
Case (\ref{item-tensor}). We can use the homotopy $h \otimes 1$ defined
in degree $n$ by setting
$$
(h \otimes 1)_{n, \alpha} = h_{n, \alpha} \otimes 1_{L_n} :
K_n \otimes_{A_n} L_n
\longrightarrow
M_n \otimes_{A_n} L_n.
$$
Case (\ref{item-wedge}). We can use the homotopy $\wedge^ih$ defined
in degree $n$ by setting
$$
\wedge^i(h)_{n, \alpha} = \wedge^i(h_{n, \alpha}) :
\wedge_{A_n}(K_n)
\longrightarrow
\wedge^i_{A_n}(M_n).
$$
Case (\ref{item-base-change}). We can use the homotopy $h \otimes 1$ defined
in degree $n$ by setting
$$
(h \otimes 1)_{n, \alpha} = h_{n, \alpha} \otimes 1 :
K_n \otimes_{A_n} B_n
\longrightarrow
M_n \otimes_{A_n} B_n.
$$
Case (\ref{item-completion}). We can use the homotopy $h^\wedge$ defined
in degree $n$ by setting
$$
(h^\wedge)_{n, \alpha} = h_{n, \alpha}^\wedge :
K_n^\wedge
\longrightarrow
M_n^\wedge.
$$
This works because each $h_{n, \alpha}$ is $A_n$-linear.
\end{proof}






\section{Crystals in quasi-coherent modules}
\label{section-quasi-coherent-crystals}

\noindent
In Situation \ref{situation-affine}.
Set $X = \Spec(C)$ and $S = \Spec(A)$. We are going to
classify crystals in quasi-coherent modules on $\text{Cris}(X/S)$.
Before we do so we fix some notation.

\medskip\noindent
Choose a polynomial ring $P = A[x_i]$ over $A$ and a surjection $P \to C$
of $A$-algebras with kernel $J = \Ker(P \to C)$. Set
\begin{equation}
\label{equation-D}
D = \lim_e D_{P, \gamma}(J) / p^eD_{P, \gamma}(J)
\end{equation}
for the $p$-adically completed divided power envelope.
This ring comes with a divided power ideal $\bar J$ and divided power
structure $\bar \gamma$, see Lemma \ref{lemma-list-properties}.
Set $D_e = D/p^eD$ and denote $\bar J_e$ the image of $\bar J$ in $D_e$.
We will use the short hand
\begin{equation}
\label{equation-omega-D}
\Omega_D = \lim_e \Omega_{D_e/A, \bar\gamma} =
\lim_e \Omega_{D/A, \bar\gamma}/p^e\Omega_{D/A, \bar\gamma}
\end{equation}
for the $p$-adic completion of the module of divided power differentials,
see Lemma \ref{lemma-differentials-completion}.
It is also the $p$-adic completion of
$\Omega_{D_{P, \gamma}(J)/A, \bar\gamma}$
which is free on $\text{d}x_i$, see
Lemma \ref{lemma-module-differentials-divided-power-envelope}.
Hence any element of $\Omega_D$ can be written uniquely as a sum
$\sum f_i\text{d}x_i$ with for all $e$ only finitely many $f_i$
not in $p^eD$. Moreover, the maps
$\text{d}_{D_e/A, \bar\gamma} : D_e \to \Omega_{D_e/A, \bar\gamma}$
fit together to define a divided power $A$-derivation
\begin{equation}
\label{equation-derivation-D}
\text{d} : D \longrightarrow \Omega_D
\end{equation}
on $p$-adic completions.

\medskip\noindent
We will also need the ``products $\Spec(D(n))$ of $\Spec(D)$'', see
Proposition \ref{proposition-compute-cohomology} and its proof for an
explanation. Formally these are defined as follows. For $n \geq 0$ let
$J(n) = \Ker(P \otimes_A \ldots \otimes_A P \to C)$ where
the tensor product has $n + 1$ factors. We set
\begin{equation}
\label{equation-Dn}
D(n) = \lim_e
D_{P \otimes_A \ldots \otimes_A P, \gamma}(J(n))/
p^eD_{P \otimes_A \ldots \otimes_A P, \gamma}(J(n))
\end{equation}
equal to the $p$-adic completion of the divided power envelope.
We denote $\bar J(n)$ its divided power ideal and $\bar \gamma(n)$
its divided powers. We also introduce $D(n)_e = D(n)/p^eD(n)$ as well
as the $p$-adically completed module of differentials
\begin{equation}
\label{equation-omega-Dn}
\Omega_{D(n)} = \lim_e \Omega_{D(n)_e/A, \bar\gamma} =
\lim_e \Omega_{D(n)/A, \bar\gamma}/p^e\Omega_{D(n)/A, \bar\gamma}
\end{equation}
and derivation
\begin{equation}
\label{equation-derivation-Dn}
\text{d} : D(n) \longrightarrow \Omega_{D(n)}
\end{equation}
Of course we have $D = D(0)$. Note that the rings $D(0), D(1), D(2), \ldots$
form a cosimplicial object in the category of divided power rings.

\begin{lemma}
\label{lemma-structure-Dn}
Let $D$ and $D(n)$ be as in (\ref{equation-D}) and (\ref{equation-Dn}).
The coprojection $P \to P \otimes_A \ldots \otimes_A P$,
$f \mapsto f \otimes 1 \otimes \ldots \otimes 1$
induces an isomorphism
\begin{equation}
\label{equation-structure-Dn}
D(n) = \lim_e D\langle \xi_i(j) \rangle/p^eD\langle \xi_i(j) \rangle
\end{equation}
of algebras over $D$ with
$$
\xi_i(j) = x_i \otimes 1 \otimes \ldots \otimes 1 -
1 \otimes \ldots \otimes 1 \otimes x_i \otimes 1 \otimes \ldots \otimes 1
$$
for $j = 1, \ldots, n$ where the second $x_i$ is placed in the $j + 1$st
slot; recall that $D(n)$ is constructed starting with the
$n + 1$-fold tensor product of $P$ over $A$.
\end{lemma}

\begin{proof}
We have
$$
P \otimes_A \ldots \otimes_A P = P[\xi_i(j)]
$$
and $J(n)$ is generated by $J$ and the elements $\xi_i(j)$.
Hence the lemma follows from
Lemma \ref{lemma-divided-power-envelope-add-variables}.
\end{proof}

\begin{lemma}
\label{lemma-property-Dn}
Let $D$ and $D(n)$ be as in (\ref{equation-D}) and (\ref{equation-Dn}).
Then $(D, \bar J, \bar\gamma)$ and $(D(n), \bar J(n), \bar\gamma(n))$
are objects of $\text{Cris}^\wedge(C/A)$, see
Remark \ref{remark-completed-affine-site}, and
$$
D(n) = \coprod\nolimits_{j = 0, \ldots, n} D
$$
in $\text{Cris}^\wedge(C/A)$.
\end{lemma}

\begin{proof}
The first assertion is clear. For the second, if $(B \to C, \delta)$ is an
object of $\text{Cris}^\wedge(C/A)$, then we have
$$
\Mor_{\text{Cris}^\wedge(C/A)}(D, B) = 
\Hom_A((P, J), (B, \Ker(B \to C)))
$$
and similarly for $D(n)$ replacing $(P, J)$ by
$(P \otimes_A \ldots \otimes_A P, J(n))$. The property on coproducts follows
as $P \otimes_A \ldots \otimes_A P$ is a coproduct.
\end{proof}

\noindent
In the lemma below we will consider pairs $(M, \nabla)$ satisfying the
following conditions
\begin{enumerate}
\item
\label{item-complete}
$M$ is a $p$-adically complete $D$-module,
\item
\label{item-connection}
$\nabla : M \to M \otimes^\wedge_D \Omega_D$ is a connection, i.e.,
$\nabla(fm) = m \otimes \text{d}f + f\nabla(m)$,
\item
\label{item-integrable}
$\nabla$ is integrable
(see Remark \ref{remark-connection}), and
\item
\label{item-topologically-quasi-nilpotent}
$\nabla$ is {\it topologically quasi-nilpotent}: If we write
$\nabla(m) = \sum \theta_i(m)\text{d}x_i$ for some operators
$\theta_i : M \to M$, then for any $m \in M$ there are only finitely
many pairs $(i, k)$ such that $\theta_i^k(m) \not \in pM$.
\end{enumerate}
The operators $\theta_i$ are sometimes denoted
$\nabla_{\partial/\partial x_i}$ in the literature.
In the following lemma we construct a functor from crystals in quasi-coherent
modules on $\text{Cris}(X/S)$ to the category of such pairs. We will show
this functor is an equivalence in
Proposition \ref{proposition-crystals-on-affine}.

\begin{lemma}
\label{lemma-crystals-on-affine}
In the situation above there is a functor
$$
\begin{matrix}
\text{crystals in quasi-coherent} \\
\mathcal{O}_{X/S}\text{-modules on }\text{Cris}(X/S)
\end{matrix}
\longrightarrow
\begin{matrix}
\text{pairs }(M, \nabla)\text{ satisfying} \\
\text{(\ref{item-complete}), (\ref{item-connection}),
(\ref{item-integrable}), and (\ref{item-topologically-quasi-nilpotent})}
\end{matrix}
$$
\end{lemma}

\begin{proof}
Let $\mathcal{F}$ be a crystal in quasi-coherent modules on $X/S$.
Set $T_e = \Spec(D_e)$ so that $(X, T_e, \bar\gamma)$ is an object
of $\text{Cris}(X/S)$ for $e \gg 0$. We have morphisms
$$
(X, T_e, \bar\gamma) \to (X, T_{e + 1}, \bar\gamma) \to \ldots
$$
which are closed immersions. We set
$$
M =
\lim_e \Gamma((X, T_e, \bar\gamma), \mathcal{F}) =
\lim_e \Gamma(T_e, \mathcal{F}_{T_e}) = \lim_e M_e
$$
Note that since $\mathcal{F}$ is locally quasi-coherent we have
$\mathcal{F}_{T_e} = \widetilde{M_e}$. Since $\mathcal{F}$ is a
crystal we have $M_e = M_{e + 1}/p^eM_{e + 1}$. Hence we see that
$M_e = M/p^eM$ and that $M$ is $p$-adically complete, see
Algebra, Lemma \ref{algebra-lemma-limit-complete}.

\medskip\noindent
By Lemma \ref{lemma-automatic-connection} we know that $\mathcal{F}$
comes endowed with a canonical integrable connection
$\nabla : \mathcal{F} \to \mathcal{F} \otimes \Omega_{X/S}$.
If we evaluate this connection on the objects $T_e$ constructed above
we obtain a canonical integrable connection
$$
\nabla : M \longrightarrow M \otimes^\wedge_D \Omega_D
$$
To see that this is topologically nilpotent we work out what this means.

\medskip\noindent
Now we can do the same procedure for the rings $D(n)$.
This produces a $p$-adically complete $D(n)$-module $M(n)$. Again using
the crystal property of $\mathcal{F}$ we obtain isomorphisms
$$
M \otimes^\wedge_{D, p_0} D(1) \rightarrow M(1)
\leftarrow M \otimes^\wedge_{D, p_1} D(1)
$$
compare with the proof of Lemma \ref{lemma-automatic-connection}.
Denote $c$ the composition from left to right. Pick $m \in M$.
Write $\xi_i = x_i \otimes 1 - 1 \otimes x_i$.
Using (\ref{equation-structure-Dn}) we can write uniquely
$$
c(m \otimes 1) = \sum\nolimits_K \theta_K(m) \otimes \prod \xi_i^{[k_i]}
$$
for some $\theta_K(m) \in M$ where the sum is over multi-indices
$K = (k_i)$ with $k_i \geq 0$ and $\sum k_i < \infty$. Set
$\theta_i = \theta_K$ where $K$ has a $1$ in the $i$th spot and
zeros elsewhere. We have
$$
\nabla(m) = \sum \theta_i(m) \text{d}x_i.
$$
as can be seen by comparing with the definition of
$\nabla$. Namely, the defining equation is
$p_1^*m = \nabla(m) - c(p_0^*m)$ in Lemma \ref{lemma-automatic-connection}
but the sign works out because in the Stacks project we consistently use
$\text{d}f = p_1(f) - p_0(f)$ modulo the ideal of the diagonal squared,
and hence $\xi_i = x_i \otimes 1 - 1 \otimes x_i$ maps to $-\text{d}x_i$
modulo the ideal of the diagonal squared.

\medskip\noindent
Denote $q_i : D \to D(2)$ and $q_{ij} : D(1) \to D(2)$ the coprojections
corresponding to the indices $i, j$. As in the last paragraph of the proof of
Lemma \ref{lemma-automatic-connection}
we see that
$$
q_{02}^*c = q_{12}^*c \circ q_{01}^*c.
$$
This means that
$$
\sum\nolimits_{K''} \theta_{K''}(m) \otimes \prod {\zeta''_i}^{[k''_i]}
=
\sum\nolimits_{K', K} \theta_{K'}(\theta_K(m))
\otimes \prod {\zeta'_i}^{[k'_i]} \prod \zeta_i^{[k_i]}
$$
in $M \otimes^\wedge_{D, q_2} D(2)$ where
\begin{align*}
\zeta_i & = x_i \otimes 1 \otimes 1 - 1 \otimes x_i \otimes 1,\\
\zeta'_i & = 1 \otimes x_i \otimes 1 - 1 \otimes 1 \otimes x_i,\\
\zeta''_i & = x_i \otimes 1 \otimes 1 - 1 \otimes 1 \otimes x_i.
\end{align*}
In particular $\zeta''_i = \zeta_i + \zeta'_i$ and we have that
$D(2)$ is the $p$-adic completion of the divided power polynomial
ring in $\zeta_i, \zeta'_i$ over $q_2(D)$, see Lemma \ref{lemma-structure-Dn}.
Comparing coefficients in the expression above it follows immediately that
$\theta_i \circ \theta_j = \theta_j \circ \theta_i$
(this provides an alternative proof of the integrability of $\nabla$) and that
$$
\theta_K(m) = (\prod \theta_i^{k_i})(m).
$$
In particular, as the sum expressing $c(m \otimes 1)$ above has to converge
$p$-adically we conclude that for each $i$ and each $m \in M$ only a finite
number of $\theta_i^k(m)$ are allowed to be nonzero modulo $p$.
\end{proof}

\begin{proposition}
\label{proposition-crystals-on-affine}
The functor
$$
\begin{matrix}
\text{crystals in quasi-coherent} \\
\mathcal{O}_{X/S}\text{-modules on }\text{Cris}(X/S)
\end{matrix}
\longrightarrow
\begin{matrix}
\text{pairs }(M, \nabla)\text{ satisfying} \\
\text{(\ref{item-complete}), (\ref{item-connection}),
(\ref{item-integrable}), and (\ref{item-topologically-quasi-nilpotent})}
\end{matrix}
$$
of Lemma \ref{lemma-crystals-on-affine}
is an equivalence of categories.
\end{proposition}

\begin{proof}
Let $(M, \nabla)$ be given. We are going to construct
a crystal in quasi-coherent modules $\mathcal{F}$.
Write $\nabla(m) = \sum \theta_i(m)\text{d}x_i$.
Then $\theta_i \circ \theta_j = \theta_j \circ \theta_i$ and we
can set $\theta_K(m) = (\prod \theta_i^{k_i})(m)$ for any multi-index
$K = (k_i)$ with $k_i \geq 0$ and $\sum k_i < \infty$.

\medskip\noindent
Let $(U, T, \delta)$ be any object of $\text{Cris}(X/S)$ with $T$ affine.
Say $T = \Spec(B)$ and the ideal of $U \to T$ is $J_B \subset B$.
By Lemma \ref{lemma-set-generators} there exists an integer $e$ and a morphism
$$
f : (U, T, \delta) \longrightarrow (X, T_e, \bar\gamma)
$$
where $T_e = \Spec(D_e)$ as in the proof of
Lemma \ref{lemma-crystals-on-affine}.
Choose such an $e$ and $f$; denote $f : D \to B$ also the corresponding
divided power $A$-algebra map. We will set $\mathcal{F}_T$ equal to the
quasi-coherent sheaf of $\mathcal{O}_T$-modules associated to the $B$-module
$$
M \otimes_{D, f} B.
$$
However, we have to show that this is independent of the choice of $f$.
Suppose that $g : D \to B$ is a second such morphism. Since $f$ and $g$
are morphisms in $\text{Cris}(X/S)$ we see that the image of
$f - g : D \to B$ is contained in the divided power ideal $J_B$.
Write $\xi_i = f(x_i) - g(x_i) \in J_B$. By analogy with the proof
of Lemma \ref{lemma-crystals-on-affine} we define an isomorphism
$$
c_{f, g} : M \otimes_{D, f} B \longrightarrow M \otimes_{D, g} B
$$
by the formula
$$
m \otimes 1 \longmapsto 
\sum\nolimits_K \theta_K(m) \otimes \prod \xi_i^{[k_i]}
$$
which makes sense by our remarks above and the fact that $\nabla$
is topologically quasi-nilpotent (so the sum is finite!).
A computation shows that
$$
c_{g, h} \circ c_{f, g} = c_{f, h}
$$
if given a third morphism
$h : (U, T, \delta) \longrightarrow (X, T_e, \bar\gamma)$.
It is also true that $c_{f, f} = 1$.
Hence these maps are all isomorphisms and we see that
the module $\mathcal{F}_T$ is independent of the choice of $f$.

\medskip\noindent
If $a : (U', T', \delta') \to (U, T, \delta)$ is a morphism of affine objects
of $\text{Cris}(X/S)$, then choosing $f' = f \circ a$ it is clear
that there exists a canonical isomorphism
$a^*\mathcal{F}_T \to \mathcal{F}_{T'}$. We omit the verification that this
map is independent of the choice of $f$. Using these maps as the restriction
maps it is clear that we obtain a crystal in quasi-coherent modules
on the full subcategory of $\text{Cris}(X/S)$ consisting of affine objects.
We omit the proof that this extends to a crystal on all of
$\text{Cris}(X/S)$. We also omit the proof that this procedure is a functor
and that it is quasi-inverse to the functor constructed in
Lemma \ref{lemma-crystals-on-affine}.
\end{proof}

\begin{lemma}
\label{lemma-crystals-on-affine-smooth}
In Situation \ref{situation-affine}.
Let $A \to P' \to C$ be ring maps with $A \to P'$ smooth and $P' \to C$
surjective with kernel $J'$. Let $D'$ be the $p$-adic completion of
$D_{P', \gamma}(J')$. There are homomorphisms of divided power $A$-algebras
$$
a : D \longrightarrow D',\quad b : D' \longrightarrow D
$$
compatible with the maps $D \to C$ and $D' \to C$ such that
$a \circ b = \text{id}_{D'}$. These maps induce
an equivalence of categories of pairs $(M, \nabla)$ satisfying
(\ref{item-complete}), (\ref{item-connection}),
(\ref{item-integrable}), and (\ref{item-topologically-quasi-nilpotent})
over $D$ and pairs $(M', \nabla')$  satisfying
(\ref{item-complete}), (\ref{item-connection}),
(\ref{item-integrable}), and
(\ref{item-topologically-quasi-nilpotent})\footnote{This condition
is tricky to formulate for $(M', \nabla')$ over $D'$. See proof.} over $D'$.
In particular, the equivalence of categories of
Proposition \ref{proposition-crystals-on-affine}
also holds for the corresponding functor towards pairs over $D'$.
\end{lemma}

\begin{proof}
First, suppose that $P' = A[y_1, \ldots, y_m]$ is a polynomial algebra
over $A$. In this case, we can find ring maps $P \to P'$ and $P' \to P$
compatible with the maps to $C$ which induce maps $a : D \to D'$ and
$b : D' \to D$ as in the lemma. Using completed base change along $a$
and $b$ we obtain functors between the categories of modules with connection
satisfying properties (\ref{item-complete}), (\ref{item-connection}),
(\ref{item-integrable}), and (\ref{item-topologically-quasi-nilpotent})
simply because these these categories are equivalent to the category
of quasi-coherent crystals by Proposition \ref{proposition-crystals-on-affine}
(and this equivalence is compatible with the base change operation as shown
in the proof of the proposition).

\medskip\noindent
Proof for general smooth $P'$.
By the first paragraph of the proof we may assume $P = A[y_1, \ldots, y_m]$
which gives us a surjection $P \to P'$ compatible with the map to $C$.
Hence we obtain a surjective map $a : D \to D'$ by functoriality of
divided power envelopes and completion. Pick $e$ large enough so that
$D_e$ is a divided power
thickening of $C$ over $A$. Then $D_e \to C$ is a surjection whose kernel
is locally nilpotent, see Divided Power Algebra, Lemma \ref{dpa-lemma-nil}.
Setting $D'_e = D'/p^eD'$
we see that the kernel of $D_e \to D'_e$ is locally nilpotent.
Hence by Algebra, Lemma \ref{algebra-lemma-smooth-strong-lift}
we can find a lift $\beta_e : P' \to D_e$ of the map $P' \to D'_e$.
Note that $D_{e + i + 1} \to D_{e + i} \times_{D'_{e + i}} D'_{e + i + 1}$
is surjective with square zero kernel for any $i \geq 0$ because
$p^{e + i}D \to p^{e + i}D'$ is surjective. Applying the usual lifting
property (Algebra, Proposition \ref{algebra-proposition-smooth-formally-smooth})
successively to the diagrams
$$
\xymatrix{
P' \ar[r] & D_{e + i} \times_{D'_{e + i}} D'_{e + i + 1} \\
A \ar[u] \ar[r] & D_{e + i + 1} \ar[u]
}
$$
we see that we can find an $A$-algebra map $\beta : P' \to D$ whose
composition with $a$ is the given map $P' \to D'$.
By the universal property of the divided power envelope we obtain a
map $D_{P', \gamma}(J') \to D$. As $D$ is $p$-adically complete we
obtain $b : D' \to D$ such that $a \circ b = \text{id}_{D'}$.

\medskip\noindent
Consider the base change functors
$$
F : (M, \nabla) \longmapsto
(M \otimes^\wedge_{D, a} D', \nabla')
\quad\text{and}\quad
G : (M', \nabla') \longmapsto
(M' \otimes^\wedge_{D', b} D, \nabla)
$$
on modules with connections satisfying (\ref{item-complete}),
(\ref{item-connection}), and (\ref{item-integrable}).
See Remark \ref{remark-base-change-connection}.
Since $a \circ b = \text{id}_{D'}$ we see that
$F \circ G$ is the identity functor. Let us say that $(M', \nabla')$
has property (\ref{item-topologically-quasi-nilpotent}) if this
is true for $G(M', \nabla')$. A formal argument now shows that to finish
the proof it suffices to show that $G(F(M, \nabla))$ is isomorphic
to $(M, \nabla)$ in the case that $(M, \nabla)$ satisfies all four
conditions (\ref{item-complete}), (\ref{item-connection}),
(\ref{item-integrable}), and (\ref{item-topologically-quasi-nilpotent}).
For this we use the functorial isomorphism
$$
c_{\text{id}_D, b \circ a} :
M \otimes_{D, \text{id}_D} D
\longrightarrow
M \otimes_{D, b \circ a} D
$$
of the proof of Proposition \ref{proposition-crystals-on-affine}
(which requires the topological quasi-nilpotency of $\nabla$
which we have assumed).
It remains to prove that this map is horizontal, i.e.,
compatible with connections, which we omit.

\medskip\noindent
The last statement of the proof now follows.
\end{proof}

\begin{remark}
\label{remark-equivalence-more-general}
The equivalence of Proposition \ref{proposition-crystals-on-affine}
holds if we start with a surjection $P \to C$ where $P/A$ satisfies the
strong lifting property of
Algebra, Lemma \ref{algebra-lemma-smooth-strong-lift}.
To prove this we can argue as in the proof of
Lemma \ref{lemma-crystals-on-affine-smooth}.
(Details will be added here if we ever need this.)
Presumably there is also a direct proof of this result, but the advantage
of using polynomial rings is that the rings $D(n)$ are $p$-adic completions
of divided power polynomial rings and the algebra is simplified.
\end{remark}



\section{General remarks on cohomology}
\label{section-cohomology-lqc}

\noindent
In this section we do a bit of work to translate the cohomology
of modules on the cristalline site of an affine scheme into
an algebraic question.

\begin{lemma}
\label{lemma-vanishing-lqc}
In Situation \ref{situation-global}.
Let $\mathcal{F}$ be a locally quasi-coherent $\mathcal{O}_{X/S}$-module
on $\text{Cris}(X/S)$. Then we have
$$
H^p((U, T, \delta), \mathcal{F}) = 0
$$
for all $p > 0$ and all $(U, T, \delta)$ with $T$ or $U$ affine.
\end{lemma}

\begin{proof}
As $U \to T$ is a thickening we see that $U$ is affine if and only if $T$
is affine, see Limits, Lemma \ref{limits-lemma-affine}.
Having said this, let us apply
Cohomology on Sites, Lemma \ref{sites-cohomology-lemma-cech-vanish-collection}
to the collection $\mathcal{B}$ of affine objects $(U, T, \delta)$ and the
collection $\text{Cov}$ of affine open coverings
$\mathcal{U} = \{(U_i, T_i, \delta_i) \to (U, T, \delta)\}$. The
{\v C}ech complex
${\check C}^*(\mathcal{U}, \mathcal{F})$ for such a covering is simply
the {\v C}ech complex of the quasi-coherent $\mathcal{O}_T$-module
$\mathcal{F}_T$
(here we are using the assumption that $\mathcal{F}$ is locally quasi-coherent)
with respect to the affine open covering $\{T_i \to T\}$ of the
affine scheme $T$. Hence the {\v C}ech cohomology is zero by
Cohomology of Schemes, Lemma
\ref{coherent-lemma-cech-cohomology-quasi-coherent} and
\ref{coherent-lemma-quasi-coherent-affine-cohomology-zero}.
Thus the hypothesis of
Cohomology on Sites, Lemma \ref{sites-cohomology-lemma-cech-vanish-collection}
are satisfied and we win.
\end{proof}

\begin{lemma}
\label{lemma-compare}
In Situation \ref{situation-global}.
Assume moreover $X$ and $S$ are affine schemes.
Consider the full subcategory $\mathcal{C} \subset \text{Cris}(X/S)$
consisting of divided power thickenings $(X, T, \delta)$
endowed with the chaotic topology (see
Sites, Example \ref{sites-example-indiscrete}).
For any locally quasi-coherent $\mathcal{O}_{X/S}$-module $\mathcal{F}$
we have
$$
R\Gamma(\mathcal{C}, \mathcal{F}|_\mathcal{C}) =
R\Gamma(\text{Cris}(X/S), \mathcal{F})
$$
\end{lemma}

\begin{proof}
Denote $\text{AffineCris}(X/S)$ the fully subcategory of $\text{Cris}(X/S)$
consisting of those objects $(U, T, \delta)$ with $U$ and $T$ affine.
We turn this into a site by saying a family of morphisms
$\{(U_i, T_i, \delta_i) \to (U, T, \delta)\}_{i \in I}$ of
$\text{AffineCris}(X/S)$ is a covering if and only if it is a covering
of $\text{Cris}(X/S)$. With this definition the inclusion functor
$$
\text{AffineCris}(X/S) \longrightarrow \text{Cris}(X/S)
$$
is a special cocontinuous functor as defined in
Sites, Definition \ref{sites-definition-special-cocontinuous-functor}.
The proof of this is exactly the same as the proof of
Topologies, Lemma \ref{topologies-lemma-affine-big-site-Zariski}.
Thus we see that the topos of sheaves on $\text{Cris}(X/S)$
is the same as the topos of sheaves on $\text{AffineCris}(X/S)$
via restriction by the displayed inclusion functor.
Therefore we have to prove the corresponding statement for the
inclusion $\mathcal{C} \subset \text{AffineCris}(X/S)$.

\medskip\noindent
We will use without further mention that $\mathcal{C}$ and
$\text{AffineCris}(X/S)$ have products and fibre products
(details omitted, see
Lemma \ref{lemma-divided-power-thickening-fibre-products}).
The inclusion functor $u : \mathcal{C} \to \text{AffineCris}(X/S)$
is fully faithful, continuous, and commutes with products and fibre products.
We claim it defines a morphism of ringed sites
$$
f :
(\text{AffineCris}(X/S), \mathcal{O}_{X/S})
\longrightarrow
(\Sh(\mathcal{C}), \mathcal{O}_{X/S}|_\mathcal{C})
$$
To see this we will use Sites, Lemma \ref{sites-lemma-directed-morphism}.
Note that $\mathcal{C}$ has fibre products and $u$ commutes with them
so the categories $\mathcal{I}^u_{(U, T, \delta)}$ are disjoint unions
of directed categories (by Sites, Lemma \ref{sites-lemma-almost-directed} and
Categories, Lemma \ref{categories-lemma-split-into-directed}). Hence it
suffices to show that $\mathcal{I}^u_{(U, T, \delta)}$ is connected.
Nonempty follows from Lemma \ref{lemma-set-generators}: since $U$ and $T$
are affine that lemma says there is at least one object
$(X, T', \delta')$ of $\mathcal{C}$ and a morphism
$(U, T, \delta) \to (X, T', \delta')$ of divided power thickenings.
Connectedness follows from the fact that $\mathcal{C}$ has products
and that $u$ commutes with them (compare with the proof of
Sites, Lemma \ref{sites-lemma-directed}).

\medskip\noindent
Note that $f_*\mathcal{F} = \mathcal{F}|_\mathcal{C}$. Hence the lemma
follows if $R^pf_*\mathcal{F} = 0$ for $p > 0$, see
Cohomology on Sites, Lemma \ref{sites-cohomology-lemma-apply-Leray}. By
Cohomology on Sites, Lemma \ref{sites-cohomology-lemma-higher-direct-images}
it suffices to show that
$H^p(\text{AffineCris}(X/S)/(X, T, \delta), \mathcal{F}) = 0$
for all $(X, T, \delta)$.
This follows from Lemma \ref{lemma-vanishing-lqc} because the
topos of the site $\text{AffineCris}(X/S)/(X, T, \delta)$
is equivalent to the topos of the site
$\text{Cris}(X/S)/(X, T, \delta)$ used in the lemma.
\end{proof}

\begin{lemma}
\label{lemma-complete}
In Situation \ref{situation-affine}.
Set $\mathcal{C} = (\text{Cris}(C/A))^{opp}$ and
$\mathcal{C}^\wedge = (\text{Cris}^\wedge(C/A))^{opp}$
endowed with the chaotic topology, see
Remark \ref{remark-completed-affine-site} for notation.
There is a morphism of topoi
$$
g : \Sh(\mathcal{C}) \longrightarrow \Sh(\mathcal{C}^\wedge)
$$
such that if $\mathcal{F}$ is a sheaf of abelian groups on
$\mathcal{C}$, then
$$
R^pg_*\mathcal{F}(B \to C, \delta) =
\left\{
\begin{matrix}
\lim_e \mathcal{F}(B_e \to C, \delta) & \text{if }p = 0 \\
R^1\lim_e \mathcal{F}(B_e \to C, \delta) & \text{if }p = 1 \\
0 & \text{else}
\end{matrix}
\right.
$$
where $B_e = B/p^eB$ for $e \gg 0$.
\end{lemma}

\begin{proof}
Any functor between categories defines a morphism between chaotic
topoi in the same direction, for example because such a functor
can be considered as a cocontinuous functor between sites, see
Sites, Section \ref{sites-section-cocontinuous-morphism-topoi}.
Proof of the description of $g_*\mathcal{F}$ is omitted.
Note that in the statement we take $(B_e \to C, \delta)$
is an object of $\text{Cris}(C/A)$  only for $e$ large enough.
Let $\mathcal{I}$ be an injective abelian sheaf on $\mathcal{C}$.
Then the transition maps
$$
\mathcal{I}(B_e \to C, \delta) \leftarrow
\mathcal{I}(B_{e + 1} \to C, \delta)
$$
are surjective as the morphisms
$$
(B_e \to C, \delta)
\longrightarrow
(B_{e + 1} \to C, \delta)
$$
are monomorphisms in the category $\mathcal{C}$. Hence for an injective
abelian sheaf both sides of the displayed formula of the lemma agree.
Taking an injective resolution of $\mathcal{F}$ one easily obtains
the result (sheaves are presheaves, so exactness is measured on the
level of groups of sections over objects).
\end{proof}

\begin{lemma}
\label{lemma-category-with-covering}
Let $\mathcal{C}$ be a category endowed with the chaotic topology.
Let $X$ be an object of $\mathcal{C}$ such that every object of
$\mathcal{C}$ has a morphism towards $X$. Assume that $\mathcal{C}$
has products of pairs.
Then for every abelian sheaf $\mathcal{F}$ on $\mathcal{C}$
the total cohomology $R\Gamma(\mathcal{C}, \mathcal{F})$ is represented
by the complex
$$
\mathcal{F}(X) \to \mathcal{F}(X \times X) \to
\mathcal{F}(X \times X \times X) \to \ldots
$$
associated to the cosimplicial abelian group $[n] \mapsto \mathcal{F}(X^n)$.
\end{lemma}

\begin{proof}
Note that $H^q(X^p, \mathcal{F}) = 0$ for all $q > 0$ as any presheaf is a
sheaf on $\mathcal{C}$. The assumption on $X$ is that $h_X \to *$
is surjective. Using that $H^q(X, \mathcal{F}) = H^q(h_X, \mathcal{F})$ and
$H^q(\mathcal{C}, \mathcal{F}) = H^q(*, \mathcal{F})$ we see that our
statement is a special case of
Cohomology on Sites,
Lemma \ref{sites-cohomology-lemma-cech-to-cohomology-sheaf-sets}.
\end{proof}









\section{Cosimplicial preparations}
\label{section-cohomology}

\noindent
In this section we compare crystalline cohomology with de Rham
cohomology. We follow \cite{Bhatt}.

\begin{example}
\label{example-cosimplicial-module}
Suppose that $A_*$ is any cosimplicial ring.
Consider the cosimplicial module $M_*$ defined by the rule
$$
M_n = \bigoplus\nolimits_{i = 0, ..., n} A_n e_i
$$
For a map $f : [n] \to [m]$ define $M_*(f) : M_n \to M_m$
to be the unique $A_*(f)$-linear map which maps $e_i$ to $e_{f(i)}$.
We claim the identity on $M_*$ is homotopic to $0$.
Namely, a homotopy is given by a map of cosimplicial modules
$$
h : M_* \longrightarrow \Hom(\Delta[1], M_*)
$$
see Section \ref{section-cosimplicial}.
For $j \in \{0, \ldots, n + 1\}$ we let $\alpha^n_j : [n] \to [1]$ be the map
defined by $\alpha^n_j(i) = 0 \Leftrightarrow i < j$. Then
$\Delta[1]_n = \{\alpha^n_0, \ldots, \alpha^n_{n + 1}\}$ and correspondingly
$\Hom(\Delta[1], M_*)_n = \prod_{j = 0, \ldots, n + 1} M_n$, see
Simplicial, Sections \ref{simplicial-section-homotopy} and
\ref{simplicial-section-homotopy-cosimplicial}. Instead of using
this product representation, we think of an element
in $\Hom(\Delta[1], M_*)_n$ as a function $\Delta[1]_n \to M_n$.
Using this notation, we define $h$ in degree $n$ by the rule
$$
h_n(e_i)(\alpha^n_j) =
\left\{
\begin{matrix}
e_{i} & \text{if} & i < j \\
0 & \text{else} 
\end{matrix}
\right.
$$
We first check $h$ is a morphism of cosimplicial modules. Namely, for
$f : [n] \to [m]$ we will show that
\begin{equation}
\label{equation-cosimplicial-morphism}
h_m \circ M_*(f) = \Hom(\Delta[1], M_*)(f) \circ h_n
\end{equation}
The left hand side of (\ref{equation-cosimplicial-morphism}) evaluated at
$e_i$ and then in turn evaluated at $\alpha^m_j$ is
$$
h_m(e_{f(i)})(\alpha^m_j) =
\left\{
\begin{matrix}
e_{f(i)} & \text{if} & f(i) < j \\
0 & \text{else}
\end{matrix}
\right.
$$
Note that $\alpha^m_j \circ f = \alpha^n_{j'}$ where
$0 \leq j' \leq n + 1$ is the unique index such that $f(i) < j$
if and only if $i < j'$. Thus the right hand side of
(\ref{equation-cosimplicial-morphism}) evaluated at $e_i$
and then in turn evaluated at $\alpha^m_j$ is
$$
M_*(f)(h_n(e_i)(\alpha^m_j \circ f) =
M_*(f)(h_n(e_i)(\alpha^n_{j'})) =
\left\{
\begin{matrix}
e_{f(i)} & \text{if} & i < j' \\
0 & \text{else} 
\end{matrix}
\right.
$$
It follows from our description of $j'$ that the two answers are equal.
Hence $h$ is a map of cosimplicial modules.
Let $0 : \Delta[0] \to \Delta[1]$ and
$1 : \Delta[0] \to \Delta[1]$ be the obvious maps, and denote
$ev_0, ev_1 : \Hom(\Delta[1], M_*) \to M_*$ the corresponding
evaluation maps. The reader verifies readily that
the compositions
$$
ev_0 \circ h, ev_1 \circ h : M_* \longrightarrow M_*
$$
are $0$ and $1$ respectively, whence $h$ is the desired homotopy between
$0$ and $1$.
\end{example}

\begin{lemma}
\label{lemma-vanishing-omega-1}
With notation as in (\ref{equation-omega-Dn}) the complex
$$
\Omega_{D(0)} \to \Omega_{D(1)} \to \Omega_{D(2)} \to \ldots
$$
is homotopic to zero as a $D(*)$-cosimplicial module.
\end{lemma}

\begin{proof}
We are going to use the principle of
Simplicial, Lemma \ref{simplicial-lemma-functorial-homotopy}
and more specifically
Lemma \ref{lemma-homotopy-tensor}
which tells us that homotopic maps between (co)simplicial objects
are transformed by any functor into homotopic maps.
The complex of the lemma is equal to the $p$-adic completion of the
base change of the cosimplicial module
$$
M_* = \left(
\Omega_{P/A} \to
\Omega_{P \otimes_A P/A} \to
\Omega_{P \otimes_A P \otimes_A P/A} \to \ldots
\right)
$$
via the cosimplicial ring map $P\otimes_A \ldots \otimes_A P \to D(n)$. This
follows from Lemma \ref{lemma-module-differentials-divided-power-envelope},
see comments following (\ref{equation-omega-D}). Hence it
suffices to show that the cosimplicial module $M_*$ is homotopic to zero
(uses base change and $p$-adic completion).
We can even assume $A = \mathbf{Z}$ and $P = \mathbf{Z}[\{x_i\}_{i \in I}]$
as we can use base change with $\mathbf{Z} \to A$.
In this case $P^{\otimes n + 1}$ is the polynomial algebra on the elements
$$
x_i(e) = 1 \otimes \ldots \otimes x_i \otimes \ldots \otimes 1
$$
with $x_i$ in the $e$th slot. The modules of the complex are free on the
generators $\text{d}x_i(e)$. Note that if $f : [n] \to [m]$ is a
map then we see that
$$
M_*(f)(\text{d}x_i(e)) = \text{d}x_i(f(e))
$$
Hence we see that $M_*$ is a direct sum over $I$ of copies of the module
studied in Example \ref{example-cosimplicial-module} and we win.
\end{proof}

\begin{lemma}
\label{lemma-vanishing}
With notation as in (\ref{equation-Dn}) and (\ref{equation-omega-Dn}),
given any cosimplicial module $M_*$ over $D(*)$ and
$i > 0$ the cosimplicial module
$$
M_0 \otimes^\wedge_{D(0)} \Omega^i_{D(0)} \to
M_1 \otimes^\wedge_{D(1)} \Omega^i_{D(1)} \to
M_2 \otimes^\wedge_{D(2)} \Omega^i_{D(2)} \to \ldots
$$
is homotopic to zero, where $\Omega^i_{D(n)}$ is the $p$-adic completion
of the $i$th exterior power of $\Omega_{D(n)}$.
\end{lemma}

\begin{proof}
By Lemma \ref{lemma-vanishing-omega-1} the endomorphisms $0$ and $1$
of $\Omega_{D(*)}$ are homotopic.
If we apply the functor $\wedge^i$ we see that
the same is true for the cosimplicial module $\wedge^i\Omega_{D(*)}$, see
Lemma \ref{lemma-homotopy-tensor}.
Another application of the same lemma shows the $p$-adic completion
$\Omega^i_{D(*)}$ is homotopy equivalent to zero.
Tensoring with $M_*$ we see that $M_* \otimes_{D(*)} \Omega^i_{D(*)}$
is homotopic to zero, see Lemma \ref{lemma-homotopy-tensor} again.
A final application of the $p$-adic completion functor finishes the proof.
\end{proof}



\section{Divided power Poincar\'e lemma}
\label{section-poincare}

\noindent
Just the simplest possible version.

\begin{lemma}
\label{lemma-poincare}
Let $A$ be a ring. Let $P = A\langle x_i \rangle$ be a divided
power polynomial ring over $A$. For any $A$-module $M$ the complex
$$
0 \to M \to
M \otimes_A P \to
M \otimes_A \Omega^1_{P/A, \delta} \to
M \otimes_A \Omega^2_{P/A, \delta} \to \ldots
$$
is exact. Let $D$ be the $p$-adic completion of $P$.
Let $\Omega^i_D$ be the $p$-adic completion of the $i$th exterior
power of $\Omega_{D/A, \delta}$. For any $p$-adically complete
$A$-module $M$ the complex
$$
0 \to M \to
M \otimes^\wedge_A D \to
M \otimes^\wedge_A \Omega^1_D \to
M \otimes^\wedge_A \Omega^2_D \to \ldots
$$
is exact.
\end{lemma}

\begin{proof}
It suffices to show that the complex
$$
E :
(0 \to A \to P \to \Omega^1_{P/A, \delta} \to
\Omega^2_{P/A, \delta} \to \ldots)
$$
is homotopy equivalent to zero as a complex of $A$-modules.
For every multi-index $K = (k_i)$ we can consider the subcomplex $E(K)$
which in degree $j$ consists of
$$
\bigoplus\nolimits_{I = \{i_1, \ldots, i_j\} \subset \text{Supp}(K)}
A
\prod\nolimits_{i \not \in I} x_i^{[k_i]}
\prod\nolimits_{i \in I} x_i^{[k_i - 1]}
\text{d}x_{i_1} \wedge \ldots \wedge \text{d}x_{i_j}
$$
Since $E = \bigoplus E(K)$ we see that it suffices to prove each of the
complexes $E(K)$ is homotopic to zero. If $K = 0$, then
$E(K) : (A \to A)$ is homotopic to zero. If $K$ has nonempty (finite)
support $S$, then the complex $E(K)$ is isomorphic to the complex
$$
0 \to A \to
\bigoplus\nolimits_{s \in S} A \to
\wedge^2(\bigoplus\nolimits_{s \in S} A) \to
\ldots \to \wedge^{\# S}(\bigoplus\nolimits_{s \in S} A) \to 0
$$
which is homotopic to zero, for example by
More on Algebra, Lemma \ref{more-algebra-lemma-homotopy-koszul-abstract}.
\end{proof}

\noindent
An alternative (more direct) approach to the following lemma is
explained in Example \ref{example-integrate}.

\begin{lemma}
\label{lemma-relative-poincare}
Let $A$ be a ring. Let $(B, I, \delta)$ be a divided power ring.
Let $P = B\langle x_i \rangle$ be a divided power polynomial
ring over $B$ with divided power ideal $J = IP + B\langle x_i \rangle_{+}$
as usual. Let $M$ be a $B$-module endowed with an integrable connection
$\nabla : M \to M \otimes_B \Omega^1_{B/A, \delta}$. Then the map of
de Rham complexes
$$
M \otimes_B \Omega^*_{B/A, \delta}
\longrightarrow
M \otimes_P \Omega^*_{P/A, \delta}
$$
is a quasi-isomorphism. Let $D$, resp.\ $D'$ be the $p$-adic completion of
$B$, resp.\ $P$ and let $\Omega^i_D$, resp.\ $\Omega^i_{D'}$ be the $p$-adic
completion of $\Omega^i_{B/A, \delta}$,
resp.\ $\Omega^i_{P/A, \delta}$. Let $M$ be a $p$-adically complete
$D$-module endowed with an integral connection
$\nabla : M \to M \otimes^\wedge_D \Omega^1_D$.
Then the map of de Rham complexes
$$
M \otimes^\wedge_D \Omega^*_D
\longrightarrow
M \otimes^\wedge_D \Omega^*_{D'}
$$
is a quasi-isomorphism.
\end{lemma}

\begin{proof}
Consider the decreasing filtration $F^*$ on $\Omega^*_{B/A, \delta}$
given by the subcomplexes
$F^i(\Omega^*_{B/A, \delta}) = \sigma_{\geq i}\Omega^*_{B/A, \delta}$.
See Homology, Section \ref{homology-section-truncations}.
This induces a decreasing filtration $F^*$ on $\Omega^*_{P/A, \delta}$
by setting
$$
F^i(\Omega^*_{P/A, \delta}) =
F^i(\Omega^*_{B/A, \delta}) \wedge \Omega^*_{P/A, \delta}.
$$
We have a split short exact sequence
$$
0 \to \Omega^1_{B/A, \delta} \otimes_B P \to
\Omega^1_{P/A, \delta} \to
\Omega^1_{P/B, \delta} \to 0
$$
and the last module is free on $\text{d}x_i$. It follows from this that
$F^i(\Omega^*_{P/A, \delta}) \to \Omega^*_{P/A, \delta}$ is a termwise
split injection and that
$$
\text{gr}^i_F(\Omega^*_{P/A, \delta}) =
\Omega^i_{B/A, \delta} \otimes_B \Omega^*_{P/B, \delta}
$$
as complexes. Thus we can define a filtration $F^*$ on
$M \otimes_B \Omega^*_{B/A, \delta}$ by setting
$$
F^i(M \otimes_B \Omega^*_{P/A, \delta}) =
M \otimes_B F^i(\Omega^*_{P/A, \delta})
$$
and we have
$$
\text{gr}^i_F(M \otimes_B \Omega^*_{P/A, \delta}) =
M \otimes_B \Omega^i_{B/A, \delta} \otimes_B \Omega^*_{P/B, \delta}
$$
as complexes.
By Lemma \ref{lemma-poincare} each of these complexes is
quasi-isomorphic to $M \otimes_B \Omega^i_{B/A, \delta}$ placed in degree $0$.
Hence we see that the first displayed map of the lemma is a morphism of
filtered complexes which induces a quasi-isomorphism on graded pieces. This
implies that it is a quasi-isomorphism, for example by the spectral sequence
associated to a filtered complex, see
Homology, Section \ref{homology-section-filtered-complex}.

\medskip\noindent
The proof of the second quasi-isomorphism is exactly the same.
\end{proof}



\section{Cohomology in the affine case}
\label{section-cohomology-affine}

\noindent
Let's go back to the situation studied in
Section \ref{section-quasi-coherent-crystals}. We
start with $(A, I, \gamma)$ and $A/I \to C$ and set
$X = \Spec(C)$ and $S = \Spec(A)$. Then we choose
a polynomial ring $P$ over $A$ and a surjection $P \to C$ with
kernel $J$. We obtain $D$ and $D(n)$ see
(\ref{equation-D}) and (\ref{equation-Dn}).
Set $T(n)_e = \Spec(D(n)/p^eD(n))$ so that
$(X, T(n)_e, \delta(n))$ is an object of $\text{Cris}(X/S)$.
Let $\mathcal{F}$ be a sheaf of $\mathcal{O}_{X/S}$-modules and set
$$
M(n) = \lim_e \Gamma((X, T(n)_e, \delta(n)), \mathcal{F})
$$
for $n = 0, 1, 2, 3, \ldots$. This forms a cosimplicial module
over the cosimplicial ring $D(0), D(1), D(2), \ldots$.

\begin{proposition}
\label{proposition-compute-cohomology}
With notations as above assume that
\begin{enumerate}
\item $\mathcal{F}$ is locally quasi-coherent, and
\item for any morphism $(U, T, \delta) \to (U', T', \delta')$
of $\text{Cris}(X/S)$ where $f : T \to T'$ is a closed immersion
the map $c_f : f^*\mathcal{F}_{T'} \to \mathcal{F}_T$ is surjective.
\end{enumerate}
Then the complex
$$
M(0) \to M(1) \to M(2) \to \ldots
$$
computes $R\Gamma(\text{Cris}(X/S), \mathcal{F})$.
\end{proposition}

\begin{proof}
Using assumption (1) and Lemma \ref{lemma-compare} we see that
$R\Gamma(\text{Cris}(X/S), \mathcal{F})$ is isomorphic to
$R\Gamma(\mathcal{C}, \mathcal{F})$. Note that the categories
$\mathcal{C}$ used in Lemmas \ref{lemma-compare} and \ref{lemma-complete}
agree. Let $f : T \to T'$ be a closed immersion as in (2). Surjectivity
of $c_f : f^*\mathcal{F}_{T'} \to \mathcal{F}_T$ is equivalent to
surjectivity of $\mathcal{F}_{T'} \to f_*\mathcal{F}_T$. Hence, if
$\mathcal{F}$ satisfies (1) and (2), then we obtain a short exact sequence
$$
0 \to \mathcal{K} \to \mathcal{F}_{T'} \to f_*\mathcal{F}_T \to 0
$$
of quasi-coherent $\mathcal{O}_{T'}$-modules on $T'$, see
Schemes, Section \ref{schemes-section-quasi-coherent} and in particular
Lemma \ref{schemes-lemma-push-forward-quasi-coherent}.
Thus, if $T'$ is affine, then we conclude that the restriction map
$\mathcal{F}(U', T', \delta') \to \mathcal{F}(U, T, \delta)$
is surjective by the vanishing of $H^1(T', \mathcal{K})$, see
Cohomology of Schemes, Lemma
\ref{coherent-lemma-quasi-coherent-affine-cohomology-zero}.
Hence the transition maps of the inverse systems in Lemma \ref{lemma-complete}
are surjective. We conclude that
$R^pg_*(\mathcal{F}|_\mathcal{C}) = 0$ for all $p \geq 1$
where $g$ is as in Lemma \ref{lemma-complete}.
The object $D$ of the category $\mathcal{C}^\wedge$
satisfies the assumption of Lemma \ref{lemma-category-with-covering} by
Lemma \ref{lemma-generator-completion}
with
$$
D \times \ldots \times D = D(n)
$$
in $\mathcal{C}$ because $D(n)$ is the $n + 1$-fold coproduct of
$D$ in $\text{Cris}^\wedge(C/A)$, see Lemma \ref{lemma-property-Dn}.
Thus we win.
\end{proof}

\begin{lemma}
\label{lemma-cohomology-is-zero}
Assumptions and notation as in
Proposition \ref{proposition-compute-cohomology}.
Then
$$
H^j(\text{Cris}(X/S), \mathcal{F} \otimes_{\mathcal{O}_{X/S}} \Omega^i_{X/S})
= 0
$$
for all $i > 0$ and all $j \geq 0$.
\end{lemma}

\begin{proof}
Using Lemma \ref{lemma-omega-locally-quasi-coherent} it follows that
$\mathcal{H} = \mathcal{F} \otimes_{\mathcal{O}_{X/S}} \Omega^i_{X/S}$
also satisfies assumptions (1) and (2) of
Proposition \ref{proposition-compute-cohomology}.
Write $M(n)_e = \Gamma((X, T(n)_e, \delta(n)), \mathcal{F})$
so that $M(n) = \lim_e M(n)_e$. Then
\begin{align*}
\lim_e \Gamma((X, T(n)_e, \delta(n)), \mathcal{H}) & =
\lim_e M(n)_e \otimes_{D(n)_e} \Omega_{D(n)}/p^e\Omega_{D(n)} \\
& = \lim_e M(n)_e \otimes_{D(n)} \Omega_{D(n)}
\end{align*}
By
Lemma \ref{lemma-vanishing}
the cosimplicial modules
$$
M(0)_e \otimes_{D(0)} \Omega^i_{D(0)} \to
M(1)_e \otimes_{D(1)} \Omega^i_{D(1)} \to
M(2)_e \otimes_{D(2)} \Omega^i_{D(2)} \to \ldots
$$
are homotopic to zero. Because the transition maps
$M(n)_{e + 1} \to M(n)_e$ are surjective, we see that
the inverse limit of the associated complexes are acyclic\footnote{Actually,
they are even homotopic to zero as the homotopies fit together, but we don't
need this. The reason for this roundabout argument is that
the limit $\lim_e M(n)_e \otimes_{D(n)} \Omega^i_{D(n)}$ isn't the
$p$-adic completion of $M(n) \otimes_{D(n)} \Omega^i_{D(n)}$ as with
the assumptions of the lemma we don't know that
$M(n)_e = M(n)_{e + 1}/p^eM(n)_{e + 1}$. If $\mathcal{F}$ is a crystal
then this does hold.}.
Hence the vanishing of cohomology of $\mathcal{H}$ by
Proposition \ref{proposition-compute-cohomology}.
\end{proof}

\begin{proposition}
\label{proposition-compute-cohomology-crystal}
Assumptions as in Proposition \ref{proposition-compute-cohomology}
but now assume that $\mathcal{F}$ is a crystal in quasi-coherent modules.
Let $(M, \nabla)$ be the corresponding module with connection over $D$, see
Proposition \ref{proposition-crystals-on-affine}. Then the complex
$$
M \otimes^\wedge_D \Omega^*_D
$$
computes $R\Gamma(\text{Cris}(X/S), \mathcal{F})$.
\end{proposition}

\begin{proof}
We will prove this using the two spectral sequences associated to the
double complex $K^{*, *}$ with terms
$$
K^{a, b} = M \otimes_D^\wedge \Omega^a_{D(b)}
$$
What do we know so far? Well, Lemma \ref{lemma-vanishing}
tells us that each column $K^{a, *}$, $a > 0$ is acyclic.
Proposition \ref{proposition-compute-cohomology} tells us that
the first column $K^{0, *}$ is quasi-isomorphic to
$R\Gamma(\text{Cris}(X/S), \mathcal{F})$.
Hence the first spectral sequence associated to the double complex
shows that there is a canonical quasi-isomorphism of
$R\Gamma(\text{Cris}(X/S), \mathcal{F})$ with
$\text{Tot}(K^{*, *})$.

\medskip\noindent
Next, let's consider the rows $K^{*, b}$. By
Lemma \ref{lemma-structure-Dn}
each of the $b + 1$ maps $D \to D(b)$ presents $D(b)$ as the $p$-adic
completion of a divided power polynomial algebra over $D$.
Hence Lemma \ref{lemma-relative-poincare} shows that the map
$$
M \otimes^\wedge_D\Omega^*_D
\longrightarrow
M \otimes^\wedge_{D(b)} \Omega^*_{D(b)} = K^{*, b}
$$
is a quasi-isomorphism. Note that each of these maps defines the {\it same}
map on cohomology (and even the same map in the derived category) as
the inverse is given by the co-diagonal map $D(b) \to D$ (corresponding
to the multiplication map $P \otimes_A \ldots \otimes_A P \to P$).
Hence if we look at the $E_1$ page of the second spectral sequence
we obtain
$$
E_1^{a, b} = H^a(M \otimes^\wedge_D\Omega^*_D)
$$
with differentials
$$
E_1^{a, 0} \xrightarrow{0}
E_1^{a, 1} \xrightarrow{1}
E_1^{a, 2} \xrightarrow{0}
E_1^{a, 3} \xrightarrow{1} \ldots
$$
as each of these is the alternation sum of the given identifications
$H^a(M \otimes^\wedge_D\Omega^*_D) = E_1^{a, 0} = E_1^{a, 1} = \ldots$.
Thus we see that the $E_2$ page is equal $H^a(M \otimes^\wedge_D\Omega^*_D)$
on the first row and zero elsewhere. It follows that the identification
of $M \otimes^\wedge_D\Omega^*_D$ with the first row induces a
quasi-isomorphism of $M \otimes^\wedge_D\Omega^*_D$ with
$\text{Tot}(K^{*, *})$.
\end{proof}

\begin{lemma}
\label{lemma-compute-cohomology-crystal-smooth}
Assumptions as in Proposition \ref{proposition-compute-cohomology-crystal}.
Let $A \to P' \to C$ be ring maps with $A \to P'$ smooth and $P' \to C$
surjective with kernel $J'$. Let $D'$ be the $p$-adic completion of
$D_{P', \gamma}(J')$. Let $(M', \nabla')$ be the pair over $D'$
corresponding to $\mathcal{F}$, see
Lemma \ref{lemma-crystals-on-affine-smooth}. Then the complex
$$
M' \otimes^\wedge_{D'} \Omega^*_{D'}
$$
computes $R\Gamma(\text{Cris}(X/S), \mathcal{F})$.
\end{lemma}

\begin{proof}
Choose $a : D \to D'$ and $b : D' \to D$ as in
Lemma \ref{lemma-crystals-on-affine-smooth}.
Note that the base change $M = M' \otimes_{D', b} D$ with its
connection $\nabla$ corresponds to $\mathcal{F}$. Hence we know
that $M \otimes^\wedge_D \Omega_D^*$ computes the crystalline
cohomology of $\mathcal{F}$, see
Proposition \ref{proposition-compute-cohomology-crystal}.
Hence it suffices to show that the base change maps (induced
by $a$ and $b$)
$$
M' \otimes^\wedge_{D'} \Omega^*_{D'}
\longrightarrow
M \otimes^\wedge_D \Omega^*_D
\quad\text{and}\quad
M \otimes^\wedge_D \Omega^*_D
\longrightarrow
M' \otimes^\wedge_{D'} \Omega^*_{D'}
$$
are quasi-isomorphisms. Since $a \circ b = \text{id}_{D'}$ we see
that the composition one way around is the identity on the complex
$M' \otimes^\wedge_{D'} \Omega^*_{D'}$. Hence it suffices to show that
the map
$$
M \otimes^\wedge_D \Omega^*_D
\longrightarrow
M \otimes^\wedge_D \Omega^*_D
$$
induced by $b \circ a : D \to D$ is a quasi-isomorphism. (Note that we
have the same complex on both sides as $M = M' \otimes^\wedge_{D', b} D$,
hence $M \otimes^\wedge_{D, b \circ a} D =
M' \otimes^\wedge_{D', b \circ a \circ b} D =
M' \otimes^\wedge_{D', b} D = M$.) In fact, we claim that for any
divided power $A$-algebra homomorphism $\rho : D \to D$ compatible
with the augmentation to $C$ the induced map
$M \otimes^\wedge_D \Omega^*_D \to M \otimes^\wedge_{D, \rho} \Omega^*_D$
is a quasi-isomorphism.

\medskip\noindent
Write $\rho(x_i) = x_i + z_i$. The elements $z_i$ are in the
divided power ideal of $D$ because $\rho$ is compatible with the
augmentation to $C$. Hence we can factor the map $\rho$
as a composition
$$
D \xrightarrow{\sigma} D\langle \xi_i \rangle^\wedge \xrightarrow{\tau} D
$$
where the first map is given by $x_i \mapsto x_i + \xi_i$ and the
second map is the divided power $D$-algebra map which maps $\xi_i$ to $z_i$.
(This uses the universal properties of polynomial algebra, divided
power polynomial algebras, divided power envelopes, and $p$-adic completion.)
Note that there exists an {\it automorphism} $\alpha$ of
$D\langle \xi_i \rangle^\wedge$ with $\alpha(x_i) = x_i - \xi_i$
and $\alpha(\xi_i) = \xi_i$. Applying Lemma \ref{lemma-relative-poincare}
to $\alpha \circ \sigma$ (which maps $x_i$ to $x_i$) and using that
$\alpha$ is an isomorphism we conclude that $\sigma$ induces a
quasi-isomorphism of $M \otimes^\wedge_D \Omega^*_D$ with
$M \otimes^\wedge_{D, \sigma} \Omega^*_{D\langle x_i \rangle^\wedge}$.
On the other hand the map $\tau$ has as a left inverse the
map $D \to D\langle x_i \rangle^\wedge$, $x_i \mapsto x_i$
and we conclude (using Lemma \ref{lemma-relative-poincare} once more)
that $\tau$ induces a quasi-isomorphism of
$M \otimes^\wedge_{D, \sigma} \Omega^*_{D\langle x_i \rangle^\wedge}$
with $M \otimes^\wedge_{D, \tau \circ \sigma} \Omega^*_D$. Composing these
two quasi-isomorphisms we obtain that $\rho$ induces a quasi-isomorphism
$M \otimes^\wedge_D \Omega^*_D \to M \otimes^\wedge_{D, \rho} \Omega^*_D$
as desired.
\end{proof}










\section{Two counter examples}
\label{section-examples}

\noindent
Before we turn to some of the successes of crystalline cohomology,
let us give two examples which explain why crystalline cohomology
does not work very well if the schemes in question are either not
proper over the base, or singular. The first example can be found
in \cite{BO}.

\begin{example}
\label{example-torsion}
Let $A = \mathbf{Z}_p$ with divided power ideal $(p)$ endowed with
its unique divided powers $\gamma$. Let
$C = \mathbf{F}_p[x, y]/(x^2, xy, y^2)$. We choose the presentation
$$
C = P/J = \mathbf{Z}_p[x, y]/(x^2, xy, y^2, p)
$$
Let $D = D_{P, \gamma}(J)^\wedge$ with divided power ideal
$(\bar J, \bar \gamma)$ as in Section \ref{section-quasi-coherent-crystals}.
We will denote $x, y$ also the images of $x$ and $y$ in $D$.
Consider the element
$$
\tau = \bar\gamma_p(x^2)\bar\gamma_p(y^2) - \bar\gamma_p(xy)^2 \in D
$$
We note that $p\tau = 0$ as
$$
p! \bar\gamma_p(x^2) \bar\gamma_p(y^2) =
x^{2p} \bar\gamma_p(y^2) = \bar\gamma_p(x^2y^2) =
x^py^p \bar\gamma_p(xy) = p! \bar\gamma_p(xy)^2
$$
in $D$. We also note that $\text{d}\tau = 0$ in $\Omega_D$ as
\begin{align*}
\text{d}(\bar\gamma_p(x^2) \bar\gamma_p(y^2))
& =
\bar\gamma_{p - 1}(x^2)\bar\gamma_p(y^2)\text{d}x^2 +
\bar\gamma_p(x^2)\bar\gamma_{p - 1}(y^2)\text{d}y^2 \\
& =
2 x \bar\gamma_{p - 1}(x^2)\bar\gamma_p(y^2)\text{d}x +
2 y \bar\gamma_p(x^2)\bar\gamma_{p - 1}(y^2)\text{d}y \\
& =
2/(p - 1)!( x^{2p - 1} \bar\gamma_p(y^2)\text{d}x +
y^{2p - 1} \bar\gamma_p(x^2)\text{d}y ) \\
& =
2/(p - 1)!
(x^{p - 1} \bar\gamma_p(xy^2)\text{d}x +
y^{p - 1} \bar\gamma_p(x^2y)\text{d}y) \\
& =
2/(p - 1)!
(x^{p - 1}y^p \bar\gamma_p(xy)\text{d}x +
x^py^{p - 1} \bar\gamma_p(xy)\text{d}y) \\
& =
2 \bar\gamma_{p - 1}(xy) \bar\gamma_p(xy)(y\text{d}x + x \text{d}y) \\
& = 
\text{d}(\bar\gamma_p(xy)^2)
\end{align*}
Finally, we claim that $\tau \not = 0$ in $D$. To see this it suffices to
produce an object $(B \to \mathbf{F}_p[x, y]/(x^2, xy, y^2), \delta)$
of $\text{Cris}(C/S)$ such that $\tau$ does not map to zero in $B$.
To do this take
$$
B = \mathbf{F}_p[x, y, u, v]/(x^3, x^2y, xy^2, y^3, xu, yu, xv, yv, u^2, v^2)
$$
with the obvious surjection to $C$. Let $K = \Ker(B \to C)$ and
consider the map
$$
\delta_p : K \longrightarrow K,\quad
ax^2 + bxy + cy^2 + du + ev + fuv \longmapsto a^pu + c^pv
$$
One checks this satisfies the assumptions (1), (2), (3) of
Divided Power Algebra, Lemma \ref{dpa-lemma-need-only-gamma-p}
and hence defines a divided power structure. Moreover,
we see that $\tau$ maps to $uv$ which is not zero in $B$.
Set $X = \Spec(C)$ and $S = \Spec(A)$.
We draw the following conclusions
\begin{enumerate}
\item $H^0(\text{Cris}(X/S), \mathcal{O}_{X/S})$ has $p$-torsion, and
\item pulling back by Frobenius $F^* : H^0(\text{Cris}(X/S), \mathcal{O}_{X/S})
\to H^0(\text{Cris}(X/S), \mathcal{O}_{X/S})$ is not injective.
\end{enumerate}
Namely, $\tau$ defines a nonzero torsion element of
$H^0(\text{Cris}(X/S), \mathcal{O}_{X/S})$ by
Proposition \ref{proposition-compute-cohomology-crystal}.
Similarly, $F^*(\tau) = \sigma(\tau)$ where $\sigma : D \to D$ is the
map induced by any lift of Frobenius on $P$. If we choose $\sigma(x) = x^p$
and $\sigma(y) = y^p$, then an easy computation shows that $F^*(\tau) = 0$.
\end{example}

\noindent
The next example shows that even for affine $n$-space crystalline
cohomology does not give the correct thing.

\begin{example}
\label{example-affine-n-space}
Let $A = \mathbf{Z}_p$ with divided power ideal $(p)$ endowed with
its unique divided powers $\gamma$. Let
$C = \mathbf{F}_p[x_1, \ldots, x_r]$. We choose the presentation
$$
C = P/J = P/pP\quad\text{with}\quad P = \mathbf{Z}_p[x_1, \ldots, x_r]
$$
Note that $pP$ has divided powers by
Divided Power Algebra, Lemma \ref{dpa-lemma-gamma-extends}.
Hence setting $D = P^\wedge$ with divided power ideal $(p)$ we obtain a
situation as in Section \ref{section-quasi-coherent-crystals}.
We conclude that $R\Gamma(\text{Cris}(X/S), \mathcal{O}_{X/S})$
is represented by the complex
$$
D \to \Omega^1_D \to \Omega^2_D \to \ldots \to \Omega^r_D
$$
see Proposition \ref{proposition-compute-cohomology-crystal}.
Assuming $r > 0$ we conclude the following
\begin{enumerate}
\item The cristalline cohomology of the cristalline structure sheaf
of $X = \mathbf{A}^r_{\mathbf{F}_p}$ over $S = \Spec(\mathbf{Z}_p)$
is zero except in degrees $0, \ldots, r$.
\item We have $H^0(\text{Cris}(X/S), \mathcal{O}_{X/S}) = \mathbf{Z}_p$.
\item The cohomology group $H^r(\text{Cris}(X/S), \mathcal{O}_{X/S})$
is infinite and is not a torsion abelian group.
\item The cohomology group $H^r(\text{Cris}(X/S), \mathcal{O}_{X/S})$
is not separated for the $p$-adic topology.
\end{enumerate}
While the first two statements are reasonable, parts (3) and (4) are
disconcerting! The truth of these statements follows immediately from
working out what the complex displayed above looks like. Let's just do
this in case $r = 1$. Then we are just looking at the two term complex
of $p$-adically complete modules
$$
\text{d} :
D = \left(
\bigoplus\nolimits_{n \geq 0} \mathbf{Z}_p x^n
\right)^\wedge
\longrightarrow
\Omega^1_D = \left(
\bigoplus\nolimits_{n \geq 1} \mathbf{Z}_p x^{n - 1}\text{d}x
\right)^\wedge
$$
The map is given by $\text{diag}(0, 1, 2, 3, 4, \ldots)$ except that
the first summand is missing on the right hand side. Now it is clear
that $\bigoplus_{n > 0} \mathbf{Z}_p/n\mathbf{Z}_p$ is a subgroup
of the cokernel, hence the cokernel is infinite. In fact, the element
$$
\omega = \sum\nolimits_{e > 0} p^e x^{p^{2e} - 1}\text{d}x
$$
is clearly not a torsion element of the cokernel. But it gets worse.
Namely, consider the element
$$
\eta = \sum\nolimits_{e > 0} p^e x^{p^e - 1}\text{d}x
$$
For every $t > 0$ the element $\eta$ is congruent to
$\sum_{e > t} p^e x^{p^e - 1}\text{d}x$ modulo the image of
$\text{d}$ which is divisible by $p^t$. But $\eta$ is not in the image of
$\text{d}$ because it would have to be the image of
$a + \sum_{e > 0} x^{p^e}$ for some $a \in \mathbf{Z}_p$
which is not an element of the left hand side. In fact, $p^N\eta$
is similarly not in the image of $\text{d}$ for any integer $N$.
This implies that $\eta$ ``generates'' a copy of $\mathbf{Q}_p$ inside
of $H^1_{\text{cris}}(\mathbf{A}_{\mathbf{F}_p}^1/\Spec(\mathbf{Z}_p))$.
\end{example}








\section{Applications}
\label{section-applications}

\noindent
In this section we collect some applications of the material in
the previous sections.

\begin{proposition}
\label{proposition-compare-with-de-Rham}
In Situation \ref{situation-global}.
Let $\mathcal{F}$ be a crystal in quasi-coherent modules on
$\text{Cris}(X/S)$. The truncation map of complexes
$$
(\mathcal{F} \to
\mathcal{F} \otimes_{\mathcal{O}_{X/S}} \Omega^1_{X/S} \to
\mathcal{F} \otimes_{\mathcal{O}_{X/S}} \Omega^2_{X/S} \to \ldots)
\longrightarrow \mathcal{F}[0],
$$
while not a quasi-isomorphism, becomes a quasi-isomorphism after applying
$Ru_{X/S, *}$. In fact, for any $i > 0$, we have 
$$
Ru_{X/S, *}(\mathcal{F} \otimes_{\mathcal{O}_{X/S}} \Omega^i_{X/S}) = 0.
$$
\end{proposition}

\begin{proof}
By Lemma \ref{lemma-automatic-connection} we get a de Rham complex
as indicated in the lemma. We abbreviate
$\mathcal{H} = \mathcal{F} \otimes \Omega^i_{X/S}$.
Let $X' \subset X$ be an affine open
subscheme which maps into an affine open subscheme $S' \subset S$.
Then
$$
(Ru_{X/S, *}\mathcal{H})|_{X'_{Zar}} =
Ru_{X'/S', *}(\mathcal{H}|_{\text{Cris}(X'/S')}),
$$
see Lemma \ref{lemma-localize}. Thus
Lemma \ref{lemma-cohomology-is-zero}
shows that $Ru_{X/S, *}\mathcal{H}$ is a complex of sheaves on
$X_{Zar}$ whose cohomology on any affine open is trivial.
As $X$ has a basis for its topology consisting of affine opens
this implies that $Ru_{X/S, *}\mathcal{H}$ is quasi-isomorphic to zero.
\end{proof}

\begin{remark}
\label{remark-vanishing}
The proof of Proposition \ref{proposition-compare-with-de-Rham}
shows that the conclusion
$$
Ru_{X/S, *}(\mathcal{F} \otimes_{\mathcal{O}_{X/S}} \Omega^i_{X/S}) = 0
$$
for $i > 0$ is true for any $\mathcal{O}_{X/S}$-module
$\mathcal{F}$ which satisfies conditions (1) and (2) of
Proposition \ref{proposition-compute-cohomology}.
This applies to the following non-crystals:
$\Omega^i_{X/S}$ for all $i$, and any sheaf of the form
$\underline{\mathcal{F}}$, where $\mathcal{F}$ is a quasi-coherent
$\mathcal{O}_X$-module. In particular, it applies to the
sheaf $\underline{\mathcal{O}_X} = \underline{\mathbf{G}_a}$.
But note that we need something like Lemma \ref{lemma-automatic-connection}
to produce a de Rham complex which requires $\mathcal{F}$ to be a crystal.
Hence (currently) the collection of sheaves of modules for which the full
statement of Proposition \ref{proposition-compare-with-de-Rham} holds
is exactly the category of crystals in quasi-coherent modules.
\end{remark}

\noindent
In Situation \ref{situation-global}.
Let $\mathcal{F}$ be a crystal in quasi-coherent modules on
$\text{Cris}(X/S)$. Let $(U, T, \delta)$ be an object of
$\text{Cris}(X/S)$. Proposition \ref{proposition-compare-with-de-Rham}
allows us to construct a canonical map
\begin{equation}
\label{equation-restriction}
R\Gamma(\text{Cris}(X/S), \mathcal{F})
\longrightarrow
R\Gamma(T, \mathcal{F}_T \otimes_{\mathcal{O}_T} \Omega^*_{T/S, \delta})
\end{equation}
Namely, we have $R\Gamma(\text{Cris}(X/S), \mathcal{F}) =
R\Gamma(\text{Cris}(X/S), \mathcal{F} \otimes \Omega^*_{X/S})$,
we can restrict global cohomology classes to $T$, and $\Omega_{X/S}$
restricts to $\Omega_{T/S, \delta}$ by
Lemma \ref{lemma-module-of-differentials}.

















\section{Some further results}
\label{section-missing}

\noindent
In this section we mention some results whose proof is missing.
We will formulate these as a series of remarks and we will convert
them into actual lemmas and propositions only when we add detailed
proofs.

\begin{remark}[Higher direct images]
\label{remark-compute-direct-image}
Let $p$ be a prime number. Let
$(S, \mathcal{I}, \gamma) \to (S', \mathcal{I}', \gamma')$ be
a morphism of divided power schemes over $\mathbf{Z}_{(p)}$. Let
$$
\xymatrix{
X \ar[r]_f \ar[d] & X' \ar[d] \\
S_0 \ar[r] & S'_0
}
$$
be a commutative diagram of morphisms of schemes and assume $p$ is
locally nilpotent on $X$ and $X'$. Let $\mathcal{F}$ be an
$\mathcal{O}_{X/S}$-module on $\text{Cris}(X/S)$. Then
$Rf_{\text{cris}, *}\mathcal{F}$ can be computed as follows.

\medskip\noindent
Given an object $(U', T', \delta')$ of $\text{Cris}(X'/S')$ set
$U = X \times_{X'} U' = f^{-1}(U')$ (an open subscheme of $X$). Denote
$(T_0, T, \delta)$ the divided power scheme over $S$ such that
$$
\xymatrix{
T \ar[r] \ar[d] & T' \ar[d] \\
S \ar[r] & S'
}
$$
is cartesian in the category of divided power schemes, see
Lemma \ref{lemma-fibre-product}. There is an
induced morphism $U \to T_0$ and we obtain a morphism
$(U/T)_{\text{cris}} \to (X/S)_{\text{cris}}$, see
Remark \ref{remark-functoriality-cris}.
Let $\mathcal{F}_U$ be the pullback of $\mathcal{F}$.
Let $\tau_{U/T} : (U/T)_{\text{cris}} \to T_{Zar}$ be the structure morphism.
Then we have
\begin{equation}
\label{equation-identify-pushforward}
\left(Rf_{\text{cris}, *}\mathcal{F}\right)_{T'} =
R(T \to T')_*\left(R\tau_{U/T, *} \mathcal{F}_U \right)
\end{equation}
where the left hand side is the restriction (see
Section \ref{section-sheaves}).

\medskip\noindent
Hints: First, show that $\text{Cris}(U/T)$ is the localization (in the sense
of Sites, Lemma \ref{sites-lemma-localize-topos-site}) of $\text{Cris}(X/S)$
at the sheaf of sets $f_{\text{cris}}^{-1}h_{(U', T', \delta')}$. Next, reduce
the statement to the case where $\mathcal{F}$ is an injective module
and pushforward of modules using that the pullback of an injective
$\mathcal{O}_{X/S}$-module is an injective $\mathcal{O}_{U/T}$-module on
$\text{Cris}(U/T)$. Finally, check the result holds for plain pushforward.
\end{remark}

\begin{remark}[Mayer-Vietoris]
\label{remark-mayer-vietoris}
In the situation of Remark \ref{remark-compute-direct-image}
suppose we have an open covering $X = X' \cup X''$. Denote
$X''' = X' \cap X''$. Let $f'$, $f''$, and $f''$ be the restriction of $f$
to $X'$, $X''$, and $X'''$. Moreover, let $\mathcal{F}'$, $\mathcal{F}''$,
and $\mathcal{F}'''$ be the restriction of $\mathcal{F}$ to the crystalline
sites of $X'$, $X''$, and $X'''$. Then there exists a distinguished triangle
$$
Rf_{\text{cris}, *}\mathcal{F}
\longrightarrow
Rf'_{\text{cris}, *}\mathcal{F}' \oplus Rf''_{\text{cris}, *}\mathcal{F}''
\longrightarrow
Rf'''_{\text{cris}, *}\mathcal{F}'''
\longrightarrow
Rf_{\text{cris}, *}\mathcal{F}[1]
$$
in $D(\mathcal{O}_{X'/S'})$.

\medskip\noindent
Hints: This is a formal consequence of the fact that the subcategories
$\text{Cris}(X'/S)$, $\text{Cris}(X''/S)$, $\text{Cris}(X'''/S)$ correspond
to open subobjects of the final sheaf on $\text{Cris}(X/S)$ and that the
last is the intersection of the first two.
\end{remark}

\begin{remark}[{\v C}ech complex]
\label{remark-cech-complex}
Let $p$ be a prime number. Let $(A, I, \gamma)$ be a divided power
ring with $A$ a $\mathbf{Z}_{(p)}$-algebra. Set $S = \Spec(A)$ and
$S_0 = \Spec(A/I)$. Let $X$ be a separated\footnote{This assumption is
not strictly necessary, as using hypercoverings the construction of the
remark can be extended to the general case.} scheme over
$S_0$ such that $p$ is locally nilpotent on $X$. Let $\mathcal{F}$ be a
crystal in quasi-coherent $\mathcal{O}_{X/S}$-modules.

\medskip\noindent
Choose an affine open covering
$X = \bigcup_{\lambda \in \Lambda} U_\lambda$ of $X$.
Write $U_\lambda = \Spec(C_\lambda)$. Choose a polynomial algebra
$P_\lambda$ over $A$ and a surjection $P_\lambda \to C_\lambda$.
Having fixed these choices we can construct a {\v C}ech complex which
computes $R\Gamma(\text{Cris}(X/S), \mathcal{F})$.

\medskip\noindent
Given $n \geq 0$ and $\lambda_0, \ldots, \lambda_n \in \Lambda$
write $U_{\lambda_0 \ldots \lambda_n} = U_{\lambda_0} \cap \ldots
\cap U_{\lambda_n}$. This is an affine scheme by assumption. Write
$U_{\lambda_0 \ldots \lambda_n} = \Spec(C_{\lambda_0 \ldots \lambda_n})$.
Set
$$
P_{\lambda_0 \ldots \lambda_n} =
P_{\lambda_0} \otimes_A \ldots \otimes_A P_{\lambda_n}
$$
which comes with a canonical surjection onto $C_{\lambda_0 \ldots \lambda_n}$.
Denote the kernel $J_{\lambda_0 \ldots \lambda_n}$ and set
$D_{\lambda_0 \ldots \lambda_n}$
the $p$-adically completed divided power envelope of
$J_{\lambda_0 \ldots \lambda_n}$ in $P_{\lambda_0 \ldots \lambda_n}$
relative to $\gamma$. Let $M_{\lambda_0 \ldots \lambda_n}$ be the
$P_{\lambda_0 \ldots \lambda_n}$-module corresponding
to the restriction of $\mathcal{F}$ to
$\text{Cris}(U_{\lambda_0 \ldots \lambda_n}/S)$ via
Proposition \ref{proposition-crystals-on-affine}.
By construction we obtain a cosimplicial divided power ring $D(*)$
having in degree $n$ the ring
$$
D(n) =
\prod\nolimits_{\lambda_0 \ldots \lambda_n}
D_{\lambda_0 \ldots \lambda_n}
$$
(use that divided power envelopes are functorial and the trivial
cosimplicial structure on the ring $P(*)$ defined similarly).
Since $M_{\lambda_0 \ldots \lambda_n}$ is the ``value'' of $\mathcal{F}$
on the objects $\Spec(D_{\lambda_0 \ldots \lambda_n})$ we see that
$M(*)$ defined by the rule
$$
M(n) = \prod\nolimits_{\lambda_0 \ldots \lambda_n}
M_{\lambda_0 \ldots \lambda_n}
$$
forms a cosimplicial $D(*)$-module. Now we claim that we have
$$
R\Gamma(\text{Cris}(X/S), \mathcal{F}) = s(M(*))
$$
Here $s(-)$ denotes the cochain complex associated to a cosimplicial
module (see
Simplicial, Section \ref{simplicial-section-dold-kan-cosimplicial}).

\medskip\noindent
Hints: The proof of this is similar to the proof of
Proposition \ref{proposition-compute-cohomology} (in particular
the result holds for any module satisfying the assumptions of
that proposition).
\end{remark}

\begin{remark}[Alternating {\v C}ech complex]
\label{remark-alternating-cech-complex}
Let $p$ be a prime number. Let $(A, I, \gamma)$ be a divided power
ring with $A$ a $\mathbf{Z}_{(p)}$-algebra. Set $S = \Spec(A)$ and
$S_0 = \Spec(A/I)$. Let $X$ be a separated quasi-compact scheme
over $S_0$ such that $p$ is locally nilpotent on $X$. Let
$\mathcal{F}$ be a crystal in quasi-coherent $\mathcal{O}_{X/S}$-modules.

\medskip\noindent
Choose a finite affine open covering
$X = \bigcup_{\lambda \in \Lambda} U_\lambda$ of $X$
and a total ordering on $\Lambda$.
Write $U_\lambda = \Spec(C_\lambda)$. Choose a polynomial algebra
$P_\lambda$ over $A$ and a surjection $P_\lambda \to C_\lambda$.
Having fixed these choices we can construct an alternating
{\v C}ech complex which computes $R\Gamma(\text{Cris}(X/S), \mathcal{F})$.

\medskip\noindent
We are going to use the notation introduced in
Remark \ref{remark-cech-complex}.
Denote $\Omega_{\lambda_0 \ldots \lambda_n}$
the $p$-adically completed module of differentials of
$D_{\lambda_0 \ldots \lambda_n}$ over $A$ compatible with the divided power
structure. Let $\nabla$ be the integrable connection on
$M_{\lambda_0 \ldots \lambda_n}$ coming from
Proposition \ref{proposition-crystals-on-affine}.
Consider the double complex $M^{\bullet, \bullet}$ with
terms
$$
M^{n, m} =
\bigoplus\nolimits_{\lambda_0 < \ldots < \lambda_n}
M_{\lambda_0 \ldots \lambda_n}
\otimes^\wedge_{D_{\lambda_0 \ldots \lambda_n}}
\Omega^m_{D_{\lambda_0 \ldots \lambda_n}}.
$$
For the differential $d_1$ (increasing $n$) we use the usual
{\v C}ech differential and for the differential $d_2$ we use
the connection, i.e., the differential of the de Rham complex.
We claim that
$$
R\Gamma(\text{Cris}(X/S), \mathcal{F}) = \text{Tot}(M^{\bullet, \bullet})
$$
Here $\text{Tot}(-)$ denotes the total complex associated to a
double complex, see
Homology, Definition \ref{homology-definition-associated-simple-complex}.

\medskip\noindent
Hints: We have
$$
R\Gamma(\text{Cris}(X/S), \mathcal{F}) = R\Gamma(\text{Cris}(X/S),
\mathcal{F} \otimes_{\mathcal{O}_{X/S}} \Omega_{X/S}^\bullet)
$$
by Proposition \ref{proposition-compare-with-de-Rham}.
The right hand side of the formula is simply the alternating {\v C}ech complex
for the covering $X = \bigcup_{\lambda \in \Lambda} U_\lambda$
(which induces an open covering of the final sheaf of $\text{Cris}(X/S)$)
and the complex $\mathcal{F} \otimes_{\mathcal{O}_{X/S}} \Omega_{X/S}^\bullet$,
see Proposition \ref{proposition-compute-cohomology-crystal}.
Now the result follows from a general result in cohomology on sites,
namely that the alternating {\v C}ech complex computes the cohomology
provided it gives the correct answer on all the pieces (insert future
reference here).
\end{remark}

\begin{remark}[Quasi-coherence]
\label{remark-quasi-coherent}
In the situation of Remark \ref{remark-compute-direct-image}
assume that $S \to S'$ is quasi-compact and quasi-separated and
that $X \to S_0$ is quasi-compact and quasi-separated. Then for a crystal
in quasi-coherent $\mathcal{O}_{X/S}$-modules $\mathcal{F}$
the sheaves $R^if_{\text{cris}, *}\mathcal{F}$ are locally quasi-coherent.

\medskip\noindent
Hints: We have to show that the restrictions to $T'$ are quasi-coherent
$\mathcal{O}_{T'}$-modules, where $(U', T', \delta')$ is any object of
$\text{Cris}(X'/S')$. It suffices to do this when $T'$ is affine.
We use the formula (\ref{equation-identify-pushforward}),
the fact that $T \to T'$ is quasi-compact and quasi-separated (as $T$
is affine over the base change of $T'$ by $S \to S'$), and
Cohomology of Schemes, Lemma
\ref{coherent-lemma-quasi-coherence-higher-direct-images}
to see that it suffices to show that the sheaves
$R^i\tau_{U/T, *}\mathcal{F}_U$ are quasi-coherent.
Note that $U \to T_0$ is also quasi-compact and quasi-separated, see
Schemes, Lemmas \ref{schemes-lemma-quasi-compact-permanence} and
\ref{schemes-lemma-quasi-compact-permanence}.

\medskip\noindent
This reduces us to proving that $R^i\tau_{X/S, *}\mathcal{F}$
is quasi-coherent on $S$ in the case that $p$ locally nilpotent on $S$. Here
$\tau_{X/S}$ is the structure morphism, see
Remark \ref{remark-structure-morphism}.
We may work locally on $S$, hence we may assume $S$ affine
(see Lemma \ref{lemma-localize}). Induction on the number
of affines covering $X$ and Mayer-Vietoris
(Remark \ref{remark-mayer-vietoris}) reduces the question to
the case where $X$ is also affine (as in the proof of
Cohomology of Schemes, Lemma
\ref{coherent-lemma-quasi-coherence-higher-direct-images}).
Say $X = \Spec(C)$ and $S = \Spec(A)$ so that $(A, I, \gamma)$ and
$A \to C$ are as
in Situation \ref{situation-affine}. Choose a polynomial algebra
$P$ over $A$ and a surjection $P \to C$ as in
Section \ref{section-quasi-coherent-crystals}.
Let $(M, \nabla)$ be the module corresponding to $\mathcal{F}$, see
Proposition \ref{proposition-crystals-on-affine}.
Applying 
Proposition \ref{proposition-compute-cohomology-crystal}
we see that $R\Gamma(\text{Cris}(X/S), \mathcal{F})$ is represented by
$M \otimes_D \Omega_D^*$. Note that completion isn't necessary
as $p$ is nilpotent in $A$! We have to show that this is compatible
with taking principal opens in $S = \Spec(A)$. Suppose that $g \in A$.
Then we conclude that similarly $R\Gamma(\text{Cris}(X_g/S_g), \mathcal{F})$
is computed by $M_g \otimes_{D_g} \Omega_{D_g}^*$ (again this uses that
$p$-adic completion isn't necessary). Hence we conclude because localization
is an exact functor on $A$-modules.
\end{remark}

\begin{remark}[Boundedness]
\label{remark-bounded-cohomology}
In the situation of Remark \ref{remark-compute-direct-image}
assume that $S \to S'$ is quasi-compact and quasi-separated and
that $X \to S_0$ is of finite type and quasi-separated. Then there exists
an integer $i_0$ such that for any crystal
in quasi-coherent $\mathcal{O}_{X/S}$-modules $\mathcal{F}$
we have $R^if_{\text{cris}, *}\mathcal{F} = 0$ for all $i > i_0$.

\medskip\noindent
Hints: Arguing as in Remark \ref{remark-quasi-coherent} (using
Cohomology of Schemes, Lemma
\ref{coherent-lemma-quasi-coherence-higher-direct-images})
we reduce to proving that $H^i(\text{Cris}(X/S), \mathcal{F}) = 0$ for $i \gg 0$
in the situation of Proposition \ref{proposition-compute-cohomology-crystal}
when $C$ is a finite type algebra over $A$. This is clear as we can
choose a finite polynomial algebra and we see that $\Omega^i_D = 0$
for $i \gg 0$.
\end{remark}

\begin{remark}[Specific boundedness]
\label{remark-bounded-cohomology-over-point}
In Situation \ref{situation-global} let $\mathcal{F}$ be a crystal in
quasi-coherent $\mathcal{O}_{X/S}$-modules. Assume that $S_0$
has a unique point and that $X \to S_0$ is of finite presentation.
\begin{enumerate}
\item If $\dim X = d$ and $X/S_0$ has embedding dimension $e$, then
$H^i(\text{Cris}(X/S), \mathcal{F}) = 0$ for $i > d + e$.
\item If $X$ is separated and can be covered by $q$ affines, and
$X/S_0$ has embedding dimension $e$, then
$H^i(\text{Cris}(X/S), \mathcal{F}) = 0$ for $i > q + e$.
\end{enumerate}
Hints: In case (1) we can use that
$$
H^i(\text{Cris}(X/S), \mathcal{F}) = H^i(X_{Zar}, Ru_{X/S, *}\mathcal{F})
$$
and that $Ru_{X/S, *}\mathcal{F}$ is locally calculated by a de Rham
complex constructed using an embedding of $X$ into a smooth scheme
of dimension $e$ over $S$
(see Lemma \ref{lemma-compute-cohomology-crystal-smooth}).
These de Rham complexes are zero in all degrees $> e$. Hence (1)
follows from Cohomology, Proposition
\ref{cohomology-proposition-vanishing-Noetherian}.
In case (2) we use the alternating {\v C}ech complex (see
Remark \ref{remark-alternating-cech-complex}) to reduce to the case
$X$ affine. In the affine case we prove the result using the de Rham complex
associated to an embedding of $X$ into a smooth scheme of dimension $e$
over $S$ (it takes some work to construct such a thing).
\end{remark}

\begin{remark}[Base change map]
\label{remark-base-change}
In the situation of Remark \ref{remark-compute-direct-image}
assume $S = \Spec(A)$ and $S' = \Spec(A')$ are affine.
Let $\mathcal{F}'$ be an $\mathcal{O}_{X'/S'}$-module.
Let $\mathcal{F}$ be the pullback of $\mathcal{F}'$.
Then there is a canonical base change map
$$
L(S' \to S)^*R\tau_{X'/S', *}\mathcal{F}'
\longrightarrow
R\tau_{X/S, *}\mathcal{F}
$$
where $\tau_{X/S}$ and $\tau_{X'/S'}$ are the structure morphisms, see
Remark \ref{remark-structure-morphism}. On global sections this
gives a base change map
\begin{equation}
\label{equation-base-change-map}
R\Gamma(\text{Cris}(X'/S'), \mathcal{F}') \otimes^\mathbf{L}_{A'} A
\longrightarrow
R\Gamma(\text{Cris}(X/S), \mathcal{F})
\end{equation}
in $D(A)$.

\medskip\noindent
Hint: Compose the very general base change map of
Cohomology on Sites, Remark \ref{sites-cohomology-remark-base-change}
with the canonical map
$Lf_{\text{cris}}^*\mathcal{F}' \to
f_{\text{cris}}^*\mathcal{F}' = \mathcal{F}$.
\end{remark}

\begin{remark}[Base change isomorphism]
\label{remark-base-change-isomorphism}
The map (\ref{equation-base-change-map}) is an isomorphism provided
all of the following conditions are satisfied:
\begin{enumerate}
\item $p$ is nilpotent in $A'$,
\item $\mathcal{F}'$ is a crystal in quasi-coherent
$\mathcal{O}_{X'/S'}$-modules,
\item $X' \to S'_0$ is a quasi-compact, quasi-separated morphism,
\item $X = X' \times_{S'_0} S_0$,
\item $\mathcal{F}'$ is a flat $\mathcal{O}_{X'/S'}$-module,
\item $X' \to S'_0$ is a local complete intersection morphism (see
More on Morphisms, Definition \ref{more-morphisms-definition-lci}; this
holds for example if $X' \to S'_0$ is syntomic or smooth),
\item $X'$ and $S_0$ are Tor independent over $S'_0$ (see
More on Algebra, Definition \ref{more-algebra-definition-tor-independent};
this holds for example if either $S_0 \to S'_0$ or $X' \to S'_0$ is flat).
\end{enumerate}
Hints: Condition (1) means that in the arguments below $p$-adic completion
does nothing and can be ignored.
Using condition (3) and Mayer Vietoris (see
Remark \ref{remark-mayer-vietoris}) this reduces to the case
where $X'$ is affine. In fact by condition (6), after shrinking
further, we can assume that $X' = \Spec(C')$ and we are given a presentation
$C' = A'/I'[x_1, \ldots, x_n]/(\bar f'_1, \ldots, \bar f'_c)$
where $\bar f'_1, \ldots, \bar f'_c$ is a Koszul-regular sequence in $A'/I'$.
(This means that smooth locally $\bar f'_1, \ldots, \bar f'_c$ forms
a regular sequence, see More on Algebra,
Lemma \ref{more-algebra-lemma-Koszul-regular-flat-locally-regular}.)
We choose a lift of
$\bar f'_i$ to an element $f'_i \in A'[x_1, \ldots, x_n]$. By (4) we see that
$X = \Spec(C)$ with $C = A/I[x_1, \ldots, x_n]/(\bar f_1, \ldots, \bar f_c)$
where $f_i \in A[x_1, \ldots, x_n]$ is the image of $f'_i$.
By property (7) we see that $\bar f_1, \ldots, \bar f_c$ is a Koszul-regular
sequence in $A/I[x_1, \ldots, x_n]$. The divided power envelope of
$I'A'[x_1, \ldots, x_n] + (f'_1, \ldots, f'_c)$ in $A'[x_1, \ldots, x_n]$
relative to $\gamma'$ is
$$
D' = A'[x_1, \ldots, x_n]\langle \xi_1, \ldots, \xi_c \rangle/(\xi_i - f'_i)
$$
see Lemma \ref{lemma-describe-divided-power-envelope}. Then you check that
$\xi_1 - f'_1, \ldots, \xi_n - f'_n$ is a Koszul-regular sequence in the
ring $A'[x_1, \ldots, x_n]\langle \xi_1, \ldots, \xi_c\rangle$.
Similarly the divided power envelope of
$IA[x_1, \ldots, x_n] + (f_1, \ldots, f_c)$ in $A[x_1, \ldots, x_n]$
relative to $\gamma$ is
$$
D = A[x_1, \ldots, x_n]\langle \xi_1, \ldots, \xi_c\rangle/(\xi_i - f_i)
$$
and $\xi_1 - f_1, \ldots, \xi_n - f_n$ is a Koszul-regular sequence in the
ring $A[x_1, \ldots, x_n]\langle \xi_1, \ldots, \xi_c\rangle$.
It follows that $D' \otimes_{A'}^\mathbf{L} A = D$. Condition (2)
implies $\mathcal{F}'$ corresponds to a pair $(M', \nabla)$
consisting of a $D'$-module with connection, see
Proposition \ref{proposition-crystals-on-affine}.
Then $M = M' \otimes_{D'} D$ corresponds to the pullback $\mathcal{F}$.
By assumption (5) we see that $M'$ is a flat $D'$-module, hence
$$
M = M' \otimes_{D'} D = M' \otimes_{D'} D' \otimes_{A'}^\mathbf{L} A
= M' \otimes_{A'}^\mathbf{L} A
$$
Since the modules of differentials $\Omega_{D'}$ and $\Omega_D$
(as defined in Section \ref{section-quasi-coherent-crystals})
are free $D'$-modules on the same generators we see that
$$
M \otimes_D \Omega^\bullet_D =
M' \otimes_{D'} \Omega^\bullet_{D'} \otimes_{D'} D =
M' \otimes_{D'} \Omega^\bullet_{D'} \otimes_{A'}^\mathbf{L} A
$$
which proves what we want by
Proposition \ref{proposition-compute-cohomology-crystal}.
\end{remark}

\begin{remark}[Rlim]
\label{remark-rlim}
Let $p$ be a prime number. Let $(A, I, \gamma)$ be a divided power
ring with $A$ an algebra over $\mathbf{Z}_{(p)}$ with $p$ nilpotent
in $A/I$. Set $S = \Spec(A)$ and $S_0 = \Spec(A/I)$.
Let $X$ be a scheme over $S_0$ with $p$ locally
nilpotent on $X$. Let $\mathcal{F}$ be any
$\mathcal{O}_{X/S}$-module. For $e \gg 0$ we have $(p^e) \subset I$
is preserved by $\gamma$, see
Divided Power Algebra, Lemma \ref{dpa-lemma-extend-to-completion}.
Set $S_e = \Spec(A/p^eA)$ for $e \gg 0$.
Then $\text{Cris}(X/S_e)$ is a full subcategory of $\text{Cris}(X/S)$
and we denote $\mathcal{F}_e$ the restriction of $\mathcal{F}$ to
$\text{Cris}(X/S_e)$. Then
$$
R\Gamma(\text{Cris}(X/S), \mathcal{F}) =
R\lim_e R\Gamma(\text{Cris}(X/S_e), \mathcal{F}_e)
$$

\medskip\noindent
Hints: Suffices to prove this for $\mathcal{F}$ injective.
In this case the sheaves $\mathcal{F}_e$ are injective
modules too, the transition maps
$\Gamma(\mathcal{F}_{e + 1}) \to \Gamma(\mathcal{F}_e)$ are
surjective, and we have
$\Gamma(\mathcal{F}) = \lim_e \Gamma(\mathcal{F}_e)$ because
any object of $\text{Cris}(X/S)$ is locally an object of one
of the categories $\text{Cris}(X/S_e)$ by definition of
$\text{Cris}(X/S)$.
\end{remark}

\begin{remark}[Comparison]
\label{remark-comparison}
Let $p$ be a prime number. Let $(A, I, \gamma)$ be a divided power
ring with $p$ nilpotent in $A$. Set $S = \Spec(A)$ and
$S_0 = \Spec(A/I)$. Let $Y$ be a smooth scheme over $S$ and set
$X = Y \times_S S_0$. Let
$\mathcal{F}$ be a crystal in quasi-coherent $\mathcal{O}_{X/S}$-modules.
Then
\begin{enumerate}
\item $\gamma$ extends to a divided power structure on the ideal
of $X$ in $Y$ so that $(X, Y, \gamma)$ is an object of $\text{Cris}(X/S)$,
\item the restriction $\mathcal{F}_Y$ (see Section \ref{section-sheaves})
comes endowed with a canonical integrable connection
$\nabla : \mathcal{F}_Y \to
\mathcal{F}_Y \otimes_{\mathcal{O}_Y} \Omega_{Y/S}$, and
\item we have
$$
R\Gamma(\text{Cris}(X/S), \mathcal{F}) =
R\Gamma(Y, \mathcal{F}_Y \otimes_{\mathcal{O}_Y} \Omega^\bullet_{Y/S})
$$
in $D(A)$.
\end{enumerate}
Hints: See Divided Power Algebra, Lemma \ref{dpa-lemma-gamma-extends} for (1).
See Lemma \ref{lemma-automatic-connection} for (2).
For Part (3) note that there is a map, see
(\ref{equation-restriction}). This map is an isomorphism when
$X$ is affine, see
Lemma \ref{lemma-compute-cohomology-crystal-smooth}.
This shows that $Ru_{X/S, *}\mathcal{F}$ and
$\mathcal{F}_Y \otimes \Omega^\bullet_{Y/S}$ are quasi-isomorphic
as complexes on $Y_{Zar} = X_{Zar}$.
Since $R\Gamma(\text{Cris}(X/S), \mathcal{F}) =
R\Gamma(X_{Zar}, Ru_{X/S, *}\mathcal{F})$ the result follows.
\end{remark}

\begin{remark}[Perfectness]
\label{remark-perfect}
Let $p$ be a prime number. Let $(A, I, \gamma)$ be a divided power
ring with $p$ nilpotent in $A$. Set $S = \Spec(A)$ and
$S_0 = \Spec(A/I)$. Let $X$ be a proper smooth scheme over $S_0$.
Let $\mathcal{F}$ be a crystal in finite locally free
quasi-coherent $\mathcal{O}_{X/S}$-modules.
Then $R\Gamma(\text{Cris}(X/S), \mathcal{F})$ is a
perfect object of $D(A)$.

\medskip\noindent
Hints: By Remark \ref{remark-base-change-isomorphism} we have
$$
R\Gamma(\text{Cris}(X/S), \mathcal{F}) \otimes_A^\mathbf{L} A/I
\cong
R\Gamma(\text{Cris}(X/S_0), \mathcal{F}|_{\text{Cris}(X/S_0)})
$$
By Remark \ref{remark-comparison} we have
$$
R\Gamma(\text{Cris}(X/S_0), \mathcal{F}|_{\text{Cris}(X/S_0)}) =
R\Gamma(X, \mathcal{F}_X \otimes \Omega^\bullet_{X/S_0})
$$
Using the stupid filtration on the de Rham complex we see that
the last displayed complex is perfect in $D(A/I)$ as soon as the complexes
$$
R\Gamma(X, \mathcal{F}_X \otimes \Omega^q_{X/S_0})
$$
are perfect complexes in $D(A/I)$, see
More on Algebra, Lemma \ref{more-algebra-lemma-two-out-of-three-perfect}.
This is true by standard arguments
in coherent cohomology using that $\mathcal{F}_X \otimes \Omega^q_{X/S_0}$
is a finite locally free sheaf and $X \to S_0$ is proper and flat
(insert future reference here). Applying
More on Algebra, Lemma \ref{more-algebra-lemma-perfect-modulo-nilpotent-ideal}
we see that
$$
R\Gamma(\text{Cris}(X/S), \mathcal{F}) \otimes_A^\mathbf{L} A/I^n
$$
is a perfect object of $D(A/I^n)$ for all $n$. This isn't quite enough
unless $A$ is Noetherian. Namely, even though $I$ is locally nilpotent
by our assumption that $p$ is nilpotent, see
Divided Power Algebra, Lemma \ref{dpa-lemma-nil},
we cannot conclude that $I^n = 0$ for some $n$. A counter example
is $\mathbf{F}_p\langle x \rangle$. To prove it in general when
$\mathcal{F} = \mathcal{O}_{X/S}$ the argument of
\url{https://math.columbia.edu/~dejong/wordpress/?p=2227}
works. When the coefficients $\mathcal{F}$ are non-trivial the
argument of \cite{Faltings-very} seems to be as follows. Reduce to the
case $pA = 0$ by More on Algebra, Lemma
\ref{more-algebra-lemma-perfect-modulo-nilpotent-ideal}.
In this case the Frobenius map $A \to A$, $a \mapsto a^p$ factors
as $A \to A/I \xrightarrow{\varphi} A$ (as $x^p = 0$ for $x \in I$). Set
$X^{(1)} = X \otimes_{A/I, \varphi} A$. The absolute Frobenius morphism
of $X$ factors through a morphism $F_X : X \to X^{(1)}$ (a kind of
relative Frobenius). Affine locally if $X = \Spec(C)$ then
$X^{(1)} = \Spec( C \otimes_{A/I, \varphi} A)$
and $F_X$ corresponds to $C \otimes_{A/I, \varphi} A \to C$,
$c \otimes a \mapsto c^pa$. This defines morphisms of ringed topoi
$$
(X/S)_{\text{cris}}
\xrightarrow{(F_X)_{\text{cris}}}
(X^{(1)}/S)_{\text{cris}}
\xrightarrow{u_{X^{(1)}/S}}
\Sh(X^{(1)}_{Zar})
$$
whose composition is denoted $\text{Frob}_X$. One then shows that
$R\text{Frob}_{X, *}\mathcal{F}$ is representable by a
perfect complex of $\mathcal{O}_{X^{(1)}}$-modules(!)
by a local calculation.
\end{remark}

\begin{remark}[Complete perfectness]
\label{remark-complete-perfect}
Let $p$ be a prime number. Let $(A, I, \gamma)$ be a divided power
ring with $A$ a $p$-adically complete ring and $p$ nilpotent in $A/I$. Set
$S = \Spec(A)$ and $S_0 = \Spec(A/I)$. Let $X$ be a proper
smooth scheme over $S_0$. Let $\mathcal{F}$ be a crystal in
finite locally free quasi-coherent $\mathcal{O}_{X/S}$-modules.
Then $R\Gamma(\text{Cris}(X/S), \mathcal{F})$ is a
perfect object of $D(A)$.

\medskip\noindent
Hints: We know that $K = R\Gamma(\text{Cris}(X/S), \mathcal{F})$
is the derived limit $K = R\lim K_e$ of the cohomologies over $A/p^eA$,
see Remark \ref{remark-rlim}.
Each $K_e$ is a perfect complex of $D(A/p^eA)$ by
Remark \ref{remark-perfect}.
Since $A$ is $p$-adically complete the result
follows from
More on Algebra, Lemma \ref{more-algebra-lemma-Rlim-perfect-gives-complete}.
\end{remark}

\begin{remark}[Complete comparison]
\label{remark-complete-comparison}
Let $p$ be a prime number. Let $(A, I, \gamma)$ be a divided power
ring with $A$ a Noetherian $p$-adically complete ring and $p$ nilpotent
in $A/I$. Set $S = \Spec(A)$ and
$S_0 = \Spec(A/I)$. Let $Y$ be a proper smooth scheme over $S$ and set
$X = Y \times_S S_0$. Let $\mathcal{F}$ be a finite type crystal in
quasi-coherent $\mathcal{O}_{X/S}$-modules. Then
\begin{enumerate}
\item there exists a coherent $\mathcal{O}_Y$-module $\mathcal{F}_Y$
endowed with integrable connection
$$
\nabla :
\mathcal{F}_Y
\longrightarrow
\mathcal{F}_Y \otimes_{\mathcal{O}_Y} \Omega_{Y/S}
$$
such that $\mathcal{F}_Y/p^e\mathcal{F}_Y$ is the module with connection
over $A/p^eA$ found in Remark \ref{remark-comparison}, and
\item we have
$$
R\Gamma(\text{Cris}(X/S), \mathcal{F}) =
R\Gamma(Y, \mathcal{F}_Y \otimes_{\mathcal{O}_Y} \Omega^\bullet_{Y/S})
$$
in $D(A)$.
\end{enumerate}
Hints: The existence of $\mathcal{F}_Y$ is Grothendieck's existence theorem
(insert future reference here). The isomorphism of cohomologies follows
as both sides are computed as $R\lim$ of the versions modulo $p^e$
(see Remark \ref{remark-rlim} for the left hand side; use the theorem
on formal functions, see
Cohomology of Schemes, Theorem \ref{coherent-theorem-formal-functions}
for the right hand side).
Each of the versions modulo $p^e$ are isomorphic by
Remark \ref{remark-comparison}.
\end{remark}




\section{Pulling back along purely inseparable maps}
\label{section-pull-back-along-pth-root}

\noindent
By an $\alpha_p$-cover we mean a morphism of the form
$$
X' = \Spec(C[z]/(z^p - c)) \longrightarrow \Spec(C) = X
$$
where $C$ is an $\mathbf{F}_p$-algebra and $c \in C$. Equivalently,
$X'$ is an $\alpha_p$-torsor over $X$. An {\it iterated
$\alpha_p$-cover}\footnote{This is nonstandard notation.}
is a morphism of schemes in characteristic
$p$ which is locally on the target a composition of finitely many
$\alpha_p$-covers. In this section we prove that pullback along
such a morphism induces a quasi-isomorphism on crystalline cohomology
after inverting the prime $p$. In fact, we prove a precise version
of this result. We begin with a preliminary lemma whose formulation
needs some notation.

\medskip\noindent
Assume we have a ring map $B \to B'$ and quotients $\Omega_B \to \Omega$ and
$\Omega_{B'} \to \Omega'$ satisfying the assumptions of
Remark \ref{remark-base-change-connection}.
Thus (\ref{equation-base-change-map-complexes}) provides a
canonical map of complexes
$$
c_M^\bullet :
M \otimes_B \Omega^\bullet
\longrightarrow
M \otimes_B (\Omega')^\bullet
$$
for all $B$-modules $M$ endowed with integrable connection
$\nabla : M \to M \otimes_B \Omega_B$.

\medskip\noindent
Suppose we have $a \in B$, $z \in B'$, and a map $\theta : B' \to B'$
satisfying the following assumptions
\begin{enumerate}
\item
\label{item-d-a-zero}
$\text{d}(a) = 0$,
\item
\label{item-direct-sum}
$\Omega' = B' \otimes_B \Omega \oplus B'\text{d}z$; we write
$\text{d}(f) = \text{d}_1(f) + \partial_z(f) \text{d}z$
with $\text{d}_1(f) \in B' \otimes \Omega$ and $\partial_z(f) \in B'$
for all $f \in B'$,
\item
\label{item-theta-linear}
$\theta : B' \to B'$ is $B$-linear,
\item
\label{item-integrate}
$\partial_z \circ \theta = a$,
\item
\label{item-injective}
$B \to B'$ is universally injective (and hence $\Omega \to \Omega'$
is injective),
\item
\label{item-factor}
$af - \theta(\partial_z(f)) \in B$ for all $f \in B'$,
\item
\label{item-horizontal}
$(\theta \otimes 1)(\text{d}_1(f)) - \text{d}_1(\theta(f)) \in \Omega$
for all $f \in B'$ where
$\theta \otimes 1 : B' \otimes \Omega \to B' \otimes \Omega$
\end{enumerate}
These conditions are not logically independent.
For example, assumption (\ref{item-integrate}) implies
that $\partial_z(af - \theta(\partial_z(f))) = 0$.
Hence if the image of $B \to B'$ is the collection of
elements annihilated by $\partial_z$, then (\ref{item-factor})
follows. A similar argument can be made for condition (\ref{item-horizontal}).

\begin{lemma}
\label{lemma-find-homotopy}
In the situation above there exists a map of complexes
$$
e_M^\bullet :
M \otimes_B (\Omega')^\bullet
\longrightarrow
M \otimes_B \Omega^\bullet
$$
such that $c_M^\bullet \circ e_M^\bullet$
and $e_M^\bullet \circ c_M^\bullet$ are homotopic to
multiplication by $a$.
\end{lemma}

\begin{proof}
In this proof all tensor products are over $B$.
Assumption (\ref{item-direct-sum}) implies that
$$
M \otimes (\Omega')^i =
(B' \otimes M \otimes \Omega^i)
\oplus
(B' \text{d}z \otimes M \otimes \Omega^{i - 1})
$$
for all $i \geq 0$. A collection of additive generators for
$M \otimes (\Omega')^i$ is formed by elements of the form
$f \omega$ and elements of the form $f \text{d}z \wedge \eta$
where $f \in B'$, $\omega \in M \otimes \Omega^i$, and
$\eta \in M \otimes \Omega^{i - 1}$.

\medskip\noindent
For $f \in B'$ we write
$$
\epsilon(f) = af - \theta(\partial_z(f))
\quad\text{and}\quad
\epsilon'(f) = (\theta \otimes 1)(\text{d}_1(f)) - \text{d}_1(\theta(f))
$$
so that $\epsilon(f) \in B$ and $\epsilon'(f) \in \Omega$ by
assumptions (\ref{item-factor}) and (\ref{item-horizontal}).
We define $e_M^\bullet$ by the rules
$e^i_M(f\omega) = \epsilon(f) \omega$ and
$e^i_M(f \text{d}z \wedge \eta) = \epsilon'(f) \wedge \eta$.
We will see below that the collection of maps $e^i_M$ is a map of complexes.

\medskip\noindent
We define
$$
h^i : M \otimes_B (\Omega')^i \longrightarrow M \otimes_B (\Omega')^{i - 1}
$$
by the rules $h^i(f \omega) = 0$ and
$h^i(f \text{d}z \wedge \eta) = \theta(f) \eta$
for elements as above.  We claim that
$$
\text{d} \circ h + h \circ \text{d} = a - c_M^\bullet \circ e_M^\bullet
$$
Note that multiplication by $a$ is a map of complexes
by (\ref{item-d-a-zero}). Hence, since $c_M^\bullet$ is an injective map
of complexes by assumption (\ref{item-injective}), we conclude that
$e_M^\bullet$ is a map of complexes. To prove the claim we compute
\begin{align*}
(\text{d} \circ h + h \circ \text{d})(f \omega)
& =
h\left(\text{d}(f) \wedge \omega + f \nabla(\omega)\right)
\\
& =
\theta(\partial_z(f)) \omega
\\
& =
a f\omega - \epsilon(f)\omega 
\\
& =
a f \omega - c^i_M(e^i_M(f\omega))
\end{align*}
The second equality because $\text{d}z$ does not occur in $\nabla(\omega)$
and the third equality by assumption (6). Similarly, we have
\begin{align*}
(\text{d} \circ h + h \circ \text{d})(f \text{d}z \wedge \eta)
& =
\text{d}(\theta(f) \eta) +
h\left(\text{d}(f) \wedge \text{d}z \wedge \eta -
f \text{d}z \wedge \nabla(\eta)\right)
\\
& =
\text{d}(\theta(f)) \wedge \eta + \theta(f) \nabla(\eta)
- (\theta \otimes 1)(\text{d}_1(f)) \wedge \eta
- \theta(f) \nabla(\eta)
\\
& =
\text{d}_1(\theta(f)) \wedge \eta +
\partial_z(\theta(f)) \text{d}z \wedge \eta -
(\theta \otimes 1)(\text{d}_1(f)) \wedge \eta
\\
& =
a f \text{d}z \wedge \eta - \epsilon'(f) \wedge \eta \\
& = a f \text{d}z \wedge \eta - c^i_M(e^i_M(f \text{d}z \wedge \eta))
\end{align*}
The second equality because
$\text{d}(f) \wedge \text{d}z \wedge \eta =
- \text{d}z \wedge \text{d}_1(f) \wedge \eta$.
The fourth equality by assumption (\ref{item-integrate}).
On the other hand it is immediate from the definitions
that $e^i_M(c^i_M(\omega)) = \epsilon(1) \omega = a \omega$.
This proves the lemma.
\end{proof}

\begin{example}
\label{example-integrate}
A standard example of the situation above occurs when
$B' = B\langle z \rangle$ is the divided power polynomial ring
over a divided power ring $(B, J, \delta)$ with divided powers
$\delta'$ on $J' = B'_{+} + JB' \subset B'$. Namely, we take
$\Omega = \Omega_{B, \delta}$ and $\Omega' = \Omega_{B', \delta'}$.
In this case we can take $a = 1$ and
$$
\theta( \sum b_m z^{[m]} ) = \sum b_m z^{[m + 1]}
$$
Note that
$$
f - \theta(\partial_z(f)) = f(0)
$$
equals the constant term. It follows that in this case
Lemma \ref{lemma-find-homotopy}
recovers the crystalline Poincar\'e lemma
(Lemma \ref{lemma-relative-poincare}).
\end{example}

\begin{lemma}
\label{lemma-computation}
In Situation \ref{situation-affine}. Assume $D$ and $\Omega_D$ are as in
(\ref{equation-D}) and (\ref{equation-omega-D}).
Let $\lambda \in D$. Let $D'$ be the $p$-adic completion of
$$
D[z]\langle \xi \rangle/(\xi - (z^p - \lambda))
$$
and let $\Omega_{D'}$ be the $p$-adic completion of the module of
divided power differentials of $D'$ over $A$. For any pair $(M, \nabla)$
over $D$ satisfying (\ref{item-complete}), (\ref{item-connection}),
(\ref{item-integrable}), and (\ref{item-topologically-quasi-nilpotent})
the canonical map of complexes (\ref{equation-base-change-map-complexes})
$$
c_M^\bullet : M \otimes_D^\wedge \Omega^\bullet_D
\longrightarrow
M \otimes_D^\wedge \Omega^\bullet_{D'}
$$
has the following property: There exists a map $e_M^\bullet$
in the opposite direction such that both $c_M^\bullet \circ e_M^\bullet$
and $e_M^\bullet \circ c_M^\bullet$ are homotopic to multiplication by $p$.
\end{lemma}

\begin{proof}
We will prove this using Lemma \ref{lemma-find-homotopy} with $a = p$.
Thus we have to find $\theta : D' \to D'$ and prove
(\ref{item-d-a-zero}), (\ref{item-direct-sum}), (\ref{item-theta-linear}),
(\ref{item-integrate}), (\ref{item-injective}), (\ref{item-factor}),
(\ref{item-horizontal}). We first collect some information about the rings
$D$ and $D'$ and the modules $\Omega_D$ and $\Omega_{D'}$.

\medskip\noindent
Writing
$$
D[z]\langle \xi \rangle/(\xi - (z^p - \lambda))
=
D\langle \xi \rangle[z]/(z^p - \xi - \lambda)
$$
we see that $D'$ is the $p$-adic completion of the free $D$-module
$$
\bigoplus\nolimits_{i = 0, \ldots, p - 1}
\bigoplus\nolimits_{n \geq 0}
z^i \xi^{[n]} D
$$
where $\xi^{[0]} = 1$.
It follows that $D \to D'$ has a continuous $D$-linear section, in particular
$D \to D'$ is universally injective, i.e., (\ref{item-injective}) holds.
We think of $D'$ as a divided power algebra
over $A$ with divided power ideal $\overline{J}' = \overline{J}D' + (\xi)$.
Then $D'$ is also the $p$-adic completion of the divided power envelope
of the ideal generated by $z^p - \lambda$ in $D$, see
Lemma \ref{lemma-describe-divided-power-envelope}. Hence
$$
\Omega_{D'} = \Omega_D \otimes_D^\wedge D' \oplus D'\text{d}z
$$
by Lemma \ref{lemma-module-differentials-divided-power-envelope}.
This proves (\ref{item-direct-sum}). Note that (\ref{item-d-a-zero})
is obvious.

\medskip\noindent
At this point we construct $\theta$. (We wrote a PARI/gp script theta.gp
verifying some of the formulas in this proof which can be found in the
scripts subdirectory of the Stacks project.) Before we do so we compute
the derivative of the elements $z^i \xi^{[n]}$. We have
$\text{d}z^i = i z^{i - 1} \text{d}z$. For $n \geq 1$ we have
$$
\text{d}\xi^{[n]} =
\xi^{[n - 1]} \text{d}\xi =
- \xi^{[n - 1]}\text{d}\lambda + p z^{p - 1} \xi^{[n - 1]}\text{d}z
$$
because $\xi = z^p - \lambda$. For $0 < i < p$ and $n \geq 1$ we have
\begin{align*}
\text{d}(z^i\xi^{[n]})
& =
iz^{i - 1}\xi^{[n]}\text{d}z + z^i\xi^{[n - 1]}\text{d}\xi \\
& =
iz^{i - 1}\xi^{[n]}\text{d}z + z^i\xi^{[n - 1]}\text{d}(z^p - \lambda) \\
& =
- z^i\xi^{[n - 1]}\text{d}\lambda +
(iz^{i - 1}\xi^{[n]} + pz^{i + p - 1}\xi^{[n - 1]})\text{d}z \\
& =
- z^i\xi^{[n - 1]}\text{d}\lambda +
(iz^{i - 1}\xi^{[n]} + pz^{i - 1}(\xi + \lambda)\xi^{[n - 1]})\text{d}z \\
& =
- z^i\xi^{[n - 1]}\text{d}\lambda +
((i + pn)z^{i - 1}\xi^{[n]} + p\lambda z^{i - 1}\xi^{[n - 1]})\text{d}z
\end{align*}
the last equality because $\xi \xi^{[n - 1]} = n\xi^{[n]}$.
Thus we see that
\begin{align*}
\partial_z(z^i) & = i z^{i - 1} \\
\partial_z(\xi^{[n]}) & = p z^{p - 1} \xi^{[n - 1]} \\
\partial_z(z^i\xi^{[n]}) & =
(i + pn) z^{i - 1} \xi^{[n]} + p \lambda z^{i - 1}\xi^{[n - 1]}
\end{align*}
Motivated by these formulas we define $\theta$ by the rules
$$
\begin{matrix}
\theta(z^j)
& = & p\frac{z^{j + 1}}{j + 1}
& j = 0, \ldots p - 1, \\
\theta(z^{p - 1}\xi^{[m]})
& = & \xi^{[m + 1]}
& m \geq 1, \\
\theta(z^j \xi^{[m]})
& = &
\frac{p z^{j + 1} \xi^{[m]} - \theta(p\lambda z^j \xi^{[m - 1]})}{(j + 1 + pm)}
& 0 \leq j < p - 1, m \geq 1
\end{matrix}
$$
where in the last line we use induction on $m$ to define our choice of
$\theta$. Working this out we get (for $0 \leq j < p - 1$ and $1 \leq m$)
$$
\theta(z^j \xi^{[m]}) =
\textstyle{\frac{p z^{j + 1} \xi^{[m]}}{(j + 1 + pm)} -
\frac{p^2 \lambda z^{j + 1} \xi^{[m - 1]}}{(j + 1 + pm)(j + 1 + p(m - 1))} +
\ldots +
\frac{(-1)^m p^{m + 1} \lambda^m z^{j + 1}}
{(j + 1 + pm) \ldots (j + 1)}}
$$
although we will not use this expression below. It is clear that $\theta$
extends uniquely to a $p$-adically continuous $D$-linear map on $D'$.
By construction we have (\ref{item-theta-linear}) and (\ref{item-integrate}).
It remains to prove (\ref{item-factor}) and (\ref{item-horizontal}).

\medskip\noindent
Proof of (\ref{item-factor}) and (\ref{item-horizontal}).
As $\theta$ is $D$-linear and continuous it suffices to prove that
$p - \theta \circ \partial_z$,
resp.\ $(\theta \otimes 1) \circ \text{d}_1 - \text{d}_1 \circ \theta$
gives an element of $D$, resp.\ $\Omega_D$ when evaluated on the
elements $z^i\xi^{[n]}$\footnote{This can be done by direct computation:
It turns out that $p - \theta \circ \partial_z$ evaluated on
$z^i\xi^{[n]}$ gives zero except for $1$ which is mapped to $p$ and
$\xi$ which is mapped to $-p\lambda$. It turns out that 
$(\theta \otimes 1) \circ \text{d}_1 - \text{d}_1 \circ \theta$
evaluated on $z^i\xi^{[n]}$ gives zero except for $z^{p - 1}\xi$
which is mapped to $-\lambda$.}.
Set $D_0 = \mathbf{Z}_{(p)}[\lambda]$ and
$D_0' = \mathbf{Z}_{(p)}[z, \lambda]\langle \xi \rangle/(\xi - z^p + \lambda)$.
Observe that each of the expressions above is an element of
$D_0'$ or $\Omega_{D_0'}$. Hence it suffices to prove the result
in the case of $D_0 \to D_0'$. Note that $D_0$ and $D_0'$
are torsion free rings and that $D_0 \otimes \mathbf{Q} = \mathbf{Q}[\lambda]$
and $D'_0 \otimes \mathbf{Q} = \mathbf{Q}[z, \lambda]$.
Hence $D_0 \subset D'_0$ is the subring of elements annihilated
by $\partial_z$ and (\ref{item-factor})
follows from (\ref{item-integrate}), see the discussion directly preceding
Lemma \ref{lemma-find-homotopy}. Similarly, we have
$\text{d}_1(f) = \partial_\lambda(f)\text{d}\lambda$ hence
$$
\left((\theta \otimes 1) \circ \text{d}_1 - \text{d}_1 \circ \theta\right)(f)
=
\left(\theta(\partial_\lambda(f)) - \partial_\lambda(\theta(f))\right)
\text{d}\lambda
$$
Applying $\partial_z$ to the coefficient we obtain
\begin{align*}
\partial_z\left(
\theta(\partial_\lambda(f)) - \partial_\lambda(\theta(f))
\right)
& =
p \partial_\lambda(f) - \partial_z(\partial_\lambda(\theta(f))) \\
& =
p \partial_\lambda(f) - \partial_\lambda(\partial_z(\theta(f))) \\
& =
p \partial_\lambda(f) - \partial_\lambda(p f) = 0
\end{align*}
whence the coefficient does not depend on $z$ as desired.
This finishes the proof of the lemma.
\end{proof}

\noindent
Note that an iterated $\alpha_p$-cover $X' \to X$ (as defined in the
introduction to this section) is finite locally free. Hence if $X$ is
connected the degree of $X' \to X$ is constant and is a power of $p$.

\begin{lemma}
\label{lemma-pullback-along-p-power-cover}
Let $p$ be a prime number. Let $(S, \mathcal{I}, \gamma)$ be a divided power
scheme over $\mathbf{Z}_{(p)}$ with $p \in \mathcal{I}$. We set
$S_0 = V(\mathcal{I}) \subset S$. Let $f : X' \to X$ be an iterated
$\alpha_p$-cover of schemes over $S_0$ with constant degree $q$. Let
$\mathcal{F}$ be any crystal in quasi-coherent sheaves on $X$ and set
$\mathcal{F}' = f_{\text{cris}}^*\mathcal{F}$.
In the distinguished triangle
$$
Ru_{X/S, *}\mathcal{F}
\longrightarrow
f_*Ru_{X'/S, *}\mathcal{F}'
\longrightarrow
E
\longrightarrow
Ru_{X/S, *}\mathcal{F}[1]
$$
the object $E$ has cohomology sheaves annihilated by $q$.
\end{lemma}

\begin{proof}
Note that $X' \to X$ is a homeomorphism hence we can identify the underlying
topological spaces of $X$ and $X'$. The question is clearly local on $X$,
hence we may assume $X$, $X'$, and $S$ affine and $X' \to X$ given as a
composition
$$
X' = X_n \to X_{n - 1} \to X_{n - 2} \to \ldots \to X_0 = X
$$
where each morphism $X_{i + 1} \to X_i$ is an $\alpha_p$-cover.
Denote $\mathcal{F}_i$ the pullback of $\mathcal{F}$ to $X_i$.
It suffices to prove that each of the maps
$$
R\Gamma(\text{Cris}(X_i/S), \mathcal{F}_i)
\longrightarrow
R\Gamma(\text{Cris}(X_{i + 1}/S), \mathcal{F}_{i + 1})
$$
fits into a triangle whose third member has cohomology groups annihilated
by $p$. (This uses axiom TR4 for the triangulated category $D(X)$. Details
omitted.)

\medskip\noindent
Hence we may assume that $S = \Spec(A)$, $X = \Spec(C)$, $X' = \Spec(C')$
and $C' = C[z]/(z^p - c)$ for some $c \in C$. Choose a polynomial algebra
$P$ over $A$ and a surjection $P \to C$. Let $D$ be the $p$-adically completed
divided power envelop of $\Ker(P \to C)$ in $P$ as in (\ref{equation-D}).
Set $P' = P[z]$ with surjection $P' \to C'$ mapping $z$ to the class of $z$
in $C'$. Choose a lift $\lambda \in D$ of $c \in C$. Then we see that
the $p$-adically completed divided power envelope $D'$ of
$\Ker(P' \to C')$ in $P'$ is isomorphic to the $p$-adic completion of
$D[z]\langle \xi \rangle/(\xi - (z^p - \lambda))$, see
Lemma \ref{lemma-computation} and its proof.
Thus we see that the result follows from this lemma
by the computation of cohomology of crystals in quasi-coherent modules in
Proposition \ref{proposition-compute-cohomology-crystal}.
\end{proof}

\noindent
The bound in the following lemma is probably not optimal.

\begin{lemma}
\label{lemma-pullback-along-p-power-cover-cohomology}
With notations and assumptions as in
Lemma \ref{lemma-pullback-along-p-power-cover}
the map
$$
f^* :
H^i(\text{Cris}(X/S), \mathcal{F})
\longrightarrow
H^i(\text{Cris}(X'/S), \mathcal{F}')
$$
has kernel and cokernel annihilated by $q^{i + 1}$.
\end{lemma}

\begin{proof}
This follows from the fact that $E$ has nonzero cohomology sheaves in
degrees $-1$ and up, so that the spectral sequence
$H^a(\mathcal{H}^b(E)) \Rightarrow H^{a + b}(E)$ converges.
This combined with the long exact cohomology sequence associated
to a distinguished triangle gives the bound.
\end{proof}

\noindent
In Situation \ref{situation-global} assume that $p \in \mathcal{I}$.
Set
$$
X^{(1)} = X \times_{S_0, F_{S_0}} S_0.
$$
Denote $F_{X/S_0} : X \to X^{(1)}$ the relative Frobenius morphism.

\begin{lemma}
\label{lemma-pullback-relative-frobenius}
In the situation above, assume that $X \to S_0$ is smooth of relative
dimension $d$. Then $F_{X/S_0}$ is an iterated $\alpha_p$-cover
of degree $p^d$. Hence Lemmas \ref{lemma-pullback-along-p-power-cover} and
\ref{lemma-pullback-along-p-power-cover-cohomology} apply to this
situation. In particular, for any crystal in quasi-coherent modules
$\mathcal{G}$ on $\text{Cris}(X^{(1)}/S)$ the map
$$
F_{X/S_0}^* : H^i(\text{Cris}(X^{(1)}/S), \mathcal{G})
\longrightarrow
H^i(\text{Cris}(X/S), F_{X/S_0, \text{cris}}^*\mathcal{G})
$$
has kernel and cokernel annihilated by $p^{d(i + 1)}$.
\end{lemma}

\begin{proof}
It suffices to prove the first statement. To see this we may assume
that $X$ is \'etale over $\mathbf{A}^d_{S_0}$, see
Morphisms, Lemma \ref{morphisms-lemma-smooth-etale-over-affine-space}.
Denote $\varphi : X \to \mathbf{A}^d_{S_0}$ this \'etale morphism.
In this case the relative Frobenius of $X/S_0$ fits into a diagram
$$
\xymatrix{
X \ar[d] \ar[r] & X^{(1)} \ar[d] \\
\mathbf{A}^d_{S_0} \ar[r] & \mathbf{A}^d_{S_0}
}
$$
where the lower horizontal arrow is the relative frobenius morphism
of $\mathbf{A}^d_{S_0}$ over $S_0$. This is the morphism which raises
all the coordinates to the $p$th power, hence it is an iterated
$\alpha_p$-cover. The proof is finished by observing that the diagram
is a fibre square, see
\'Etale Morphisms, Lemma \ref{etale-lemma-relative-frobenius-etale}.
\end{proof}













\section{Frobenius action on crystalline cohomology}
\label{section-frobenius}

\noindent
In this section we prove that Frobenius pullback induces a quasi-isomorphism
on crystalline cohomology after inverting the prime $p$. But in order to
even formulate this we need to work in a special situation.

\begin{situation}
\label{situation-F-crystal}
In Situation \ref{situation-global} assume the following
\begin{enumerate}
\item $S = \Spec(A)$ for some divided power ring $(A, I, \gamma)$
with $p \in I$,
\item there is given a homomorphism of divided power rings $\sigma : A \to A$
such that $\sigma(x) = x^p \bmod pA$ for all $x \in A$.
\end{enumerate}
\end{situation}

\noindent
In Situation \ref{situation-F-crystal} the morphism
$\Spec(\sigma) : S \to S$ is a lift of the absolute Frobenius
$F_{S_0} : S_0 \to S_0$ and since the diagram
$$
\xymatrix{
X \ar[d] \ar[r]_{F_X} & X \ar[d] \\
S_0 \ar[r]^{F_{S_0}} & S_0
}
$$
is commutative where $F_X : X \to X$ is the absolute Frobenius morphism
of $X$. Thus we obtain a morphism of crystalline topoi
$$
(F_X)_{\text{cris}} :
(X/S)_{\text{cris}}
\longrightarrow
(X/S)_{\text{cris}}
$$
see Remark \ref{remark-functoriality-cris}. Here is the terminology concerning
$F$-crystals following the notation of Saavedra, see
\cite{Saavedra}.

\begin{definition}
\label{definition-F-crystal}
In Situation \ref{situation-F-crystal} an {\it $F$-crystal on $X/S$
(relative to $\sigma$)} is a pair $(\mathcal{E}, F_\mathcal{E})$
given by a crystal in finite locally free $\mathcal{O}_{X/S}$-modules
$\mathcal{E}$ together with a map
$$
F_\mathcal{E} : (F_X)_{\text{cris}}^*\mathcal{E} \longrightarrow \mathcal{E}
$$
An $F$-crystal is called {\it nondegenerate} if there exists an integer
$i \geq 0$ a map $V : \mathcal{E} \to (F_X)_{\text{cris}}^*\mathcal{E}$
such that $V \circ F_{\mathcal{E}} = p^i \text{id}$.
\end{definition}

\begin{remark}
\label{remark-F-crystal-variants}
Let $(\mathcal{E}, F)$ be an $F$-crystal as in
Definition \ref{definition-F-crystal}.
In the literature the nondegeneracy condition is often part of the
definition of an $F$-crystal. Moreover, often it is also assumed that
$F \circ V = p^n\text{id}$. What is needed for the result below is
that there exists an integer $j \geq 0$ such that $\Ker(F)$ and
$\Coker(F)$ are killed by $p^j$. If the rank of $\mathcal{E}$
is bounded (for example if $X$ is quasi-compact), then both of these
conditions follow from the nondegeneracy condition as formulated in
the definition. Namely, suppose $R$ is a ring, $r \geq 1$ is an integer and
$K, L \in \text{Mat}(r \times r, R)$ are matrices with
$K L = p^i 1_{r \times r}$. Then $\det(K)\det(L) = p^{ri}$. 
Let $L'$ be the adjugate matrix of $L$, i.e.,
$L' L = L L' = \det(L)$. Set $K' = p^{ri} K$ and $j = ri + i$.
Then we have $K' L = p^j 1_{r \times r}$ as $K L = p^i$ and
$$
L K' = L K \det(L) \det(M) = L K L L' \det(M) = L p^i L' \det(M) =
p^j 1_{r \times r}
$$
It follows that if $V$ is as in Definition \ref{definition-F-crystal}
then setting $V' = p^N V$ where $N > i \cdot \text{rank}(\mathcal{E})$
we get $V' \circ F = p^{N + i}$ and $F \circ V' = p^{N + i}$.
\end{remark}

\begin{theorem}
\label{theorem-cohomology-F-crystal}
In Situation \ref{situation-F-crystal} let $(\mathcal{E}, F_\mathcal{E})$
be a nondegenerate $F$-crystal. Assume $A$ is a $p$-adically complete
Noetherian ring and that $X \to S_0$ is proper smooth. Then
the canonical map
$$
F_\mathcal{E} \circ (F_X)_{\text{cris}}^* :
R\Gamma(\text{Cris}(X/S), \mathcal{E}) \otimes^\mathbf{L}_{A, \sigma} A
\longrightarrow
R\Gamma(\text{Cris}(X/S), \mathcal{E})
$$
becomes an isomorphism after inverting $p$.
\end{theorem}

\begin{proof}
We first write the arrow as a composition of three arrows.
Namely, set
$$
X^{(1)} = X \times_{S_0, F_{S_0}} S_0
$$
and denote $F_{X/S_0} : X \to X^{(1)}$ the relative Frobenius morphism.
Denote $\mathcal{E}^{(1)}$ the base change of $\mathcal{E}$
by $\Spec(\sigma)$, in other words the pullback of $\mathcal{E}$
to $\text{Cris}(X^{(1)}/S)$ by the morphism of crystalline topoi
associated to the commutative diagram
$$
\xymatrix{
X^{(1)} \ar[r] \ar[d] & X \ar[d] \\
S \ar[r]^{\Spec(\sigma)} & S
}
$$
Then we have the base change map
\begin{equation}
\label{equation-base-change-sigma}
R\Gamma(\text{Cris}(X/S), \mathcal{E}) \otimes^\mathbf{L}_{A, \sigma} A
\longrightarrow
R\Gamma(\text{Cris}(X^{(1)}/S), \mathcal{E}^{(1)})
\end{equation}
see Remark \ref{remark-base-change}. Note that the composition
of $F_{X/S_0} : X \to X^{(1)}$ with the projection $X^{(1)} \to X$
is the absolute Frobenius morphism $F_X$. Hence we see that
$F_{X/S_0}^*\mathcal{E}^{(1)} = (F_X)_{\text{cris}}^*\mathcal{E}$.
Thus pullback by $F_{X/S_0}$ is a map
\begin{equation}
\label{equation-to-prove}
F_{X/S_0}^* :
R\Gamma(\text{Cris}(X^{(1)}/S), \mathcal{E}^{(1)})
\longrightarrow
R\Gamma(\text{Cris}(X/S), (F_X)^*_{\text{cris}}\mathcal{E})
\end{equation}
Finally we can use $F_\mathcal{E}$ to get a map
\begin{equation}
\label{equation-F-E}
R\Gamma(\text{Cris}(X/S), (F_X)^*_{\text{cris}}\mathcal{E})
\longrightarrow
R\Gamma(\text{Cris}(X/S), \mathcal{E})
\end{equation}
The map of the theorem is the composition of the three maps
(\ref{equation-base-change-sigma}), (\ref{equation-to-prove}), and
(\ref{equation-F-E}) above. The first is a
quasi-isomorphism modulo all powers of $p$ by
Remark \ref{remark-base-change-isomorphism}.
Hence it is a quasi-isomorphism since the complexes involved are perfect
in $D(A)$ see Remark \ref{remark-complete-perfect}.
The third map is a quasi-isomorphism after inverting $p$ simply
because $F_\mathcal{E}$ has an inverse up to a power of $p$, see
Remark \ref{remark-F-crystal-variants}.
Finally, the second is an isomorphism after inverting $p$
by Lemma \ref{lemma-pullback-relative-frobenius}.
\end{proof}






\begin{multicols}{2}[\section{Other chapters}]
\noindent
Preliminaries
\begin{enumerate}
\item \hyperref[introduction-section-phantom]{Introduction}
\item \hyperref[conventions-section-phantom]{Conventions}
\item \hyperref[sets-section-phantom]{Set Theory}
\item \hyperref[categories-section-phantom]{Categories}
\item \hyperref[topology-section-phantom]{Topology}
\item \hyperref[sheaves-section-phantom]{Sheaves on Spaces}
\item \hyperref[sites-section-phantom]{Sites and Sheaves}
\item \hyperref[stacks-section-phantom]{Stacks}
\item \hyperref[fields-section-phantom]{Fields}
\item \hyperref[algebra-section-phantom]{Commutative Algebra}
\item \hyperref[brauer-section-phantom]{Brauer Groups}
\item \hyperref[homology-section-phantom]{Homological Algebra}
\item \hyperref[derived-section-phantom]{Derived Categories}
\item \hyperref[simplicial-section-phantom]{Simplicial Methods}
\item \hyperref[more-algebra-section-phantom]{More on Algebra}
\item \hyperref[smoothing-section-phantom]{Smoothing Ring Maps}
\item \hyperref[modules-section-phantom]{Sheaves of Modules}
\item \hyperref[sites-modules-section-phantom]{Modules on Sites}
\item \hyperref[injectives-section-phantom]{Injectives}
\item \hyperref[cohomology-section-phantom]{Cohomology of Sheaves}
\item \hyperref[sites-cohomology-section-phantom]{Cohomology on Sites}
\item \hyperref[dga-section-phantom]{Differential Graded Algebra}
\item \hyperref[dpa-section-phantom]{Divided Power Algebra}
\item \hyperref[sdga-section-phantom]{Differential Graded Sheaves}
\item \hyperref[hypercovering-section-phantom]{Hypercoverings}
\end{enumerate}
Schemes
\begin{enumerate}
\setcounter{enumi}{25}
\item \hyperref[schemes-section-phantom]{Schemes}
\item \hyperref[constructions-section-phantom]{Constructions of Schemes}
\item \hyperref[properties-section-phantom]{Properties of Schemes}
\item \hyperref[morphisms-section-phantom]{Morphisms of Schemes}
\item \hyperref[coherent-section-phantom]{Cohomology of Schemes}
\item \hyperref[divisors-section-phantom]{Divisors}
\item \hyperref[limits-section-phantom]{Limits of Schemes}
\item \hyperref[varieties-section-phantom]{Varieties}
\item \hyperref[topologies-section-phantom]{Topologies on Schemes}
\item \hyperref[descent-section-phantom]{Descent}
\item \hyperref[perfect-section-phantom]{Derived Categories of Schemes}
\item \hyperref[more-morphisms-section-phantom]{More on Morphisms}
\item \hyperref[flat-section-phantom]{More on Flatness}
\item \hyperref[groupoids-section-phantom]{Groupoid Schemes}
\item \hyperref[more-groupoids-section-phantom]{More on Groupoid Schemes}
\item \hyperref[etale-section-phantom]{\'Etale Morphisms of Schemes}
\end{enumerate}
Topics in Scheme Theory
\begin{enumerate}
\setcounter{enumi}{41}
\item \hyperref[chow-section-phantom]{Chow Homology}
\item \hyperref[intersection-section-phantom]{Intersection Theory}
\item \hyperref[pic-section-phantom]{Picard Schemes of Curves}
\item \hyperref[weil-section-phantom]{Weil Cohomology Theories}
\item \hyperref[adequate-section-phantom]{Adequate Modules}
\item \hyperref[dualizing-section-phantom]{Dualizing Complexes}
\item \hyperref[duality-section-phantom]{Duality for Schemes}
\item \hyperref[discriminant-section-phantom]{Discriminants and Differents}
\item \hyperref[derham-section-phantom]{de Rham Cohomology}
\item \hyperref[local-cohomology-section-phantom]{Local Cohomology}
\item \hyperref[algebraization-section-phantom]{Algebraic and Formal Geometry}
\item \hyperref[curves-section-phantom]{Algebraic Curves}
\item \hyperref[resolve-section-phantom]{Resolution of Surfaces}
\item \hyperref[models-section-phantom]{Semistable Reduction}
\item \hyperref[functors-section-phantom]{Functors and Morphisms}
\item \hyperref[equiv-section-phantom]{Derived Categories of Varieties}
\item \hyperref[pione-section-phantom]{Fundamental Groups of Schemes}
\item \hyperref[etale-cohomology-section-phantom]{\'Etale Cohomology}
\item \hyperref[crystalline-section-phantom]{Crystalline Cohomology}
\item \hyperref[proetale-section-phantom]{Pro-\'etale Cohomology}
\item \hyperref[relative-cycles-section-phantom]{Relative Cycles}
\item \hyperref[more-etale-section-phantom]{More \'Etale Cohomology}
\item \hyperref[trace-section-phantom]{The Trace Formula}
\end{enumerate}
Algebraic Spaces
\begin{enumerate}
\setcounter{enumi}{64}
\item \hyperref[spaces-section-phantom]{Algebraic Spaces}
\item \hyperref[spaces-properties-section-phantom]{Properties of Algebraic Spaces}
\item \hyperref[spaces-morphisms-section-phantom]{Morphisms of Algebraic Spaces}
\item \hyperref[decent-spaces-section-phantom]{Decent Algebraic Spaces}
\item \hyperref[spaces-cohomology-section-phantom]{Cohomology of Algebraic Spaces}
\item \hyperref[spaces-limits-section-phantom]{Limits of Algebraic Spaces}
\item \hyperref[spaces-divisors-section-phantom]{Divisors on Algebraic Spaces}
\item \hyperref[spaces-over-fields-section-phantom]{Algebraic Spaces over Fields}
\item \hyperref[spaces-topologies-section-phantom]{Topologies on Algebraic Spaces}
\item \hyperref[spaces-descent-section-phantom]{Descent and Algebraic Spaces}
\item \hyperref[spaces-perfect-section-phantom]{Derived Categories of Spaces}
\item \hyperref[spaces-more-morphisms-section-phantom]{More on Morphisms of Spaces}
\item \hyperref[spaces-flat-section-phantom]{Flatness on Algebraic Spaces}
\item \hyperref[spaces-groupoids-section-phantom]{Groupoids in Algebraic Spaces}
\item \hyperref[spaces-more-groupoids-section-phantom]{More on Groupoids in Spaces}
\item \hyperref[bootstrap-section-phantom]{Bootstrap}
\item \hyperref[spaces-pushouts-section-phantom]{Pushouts of Algebraic Spaces}
\end{enumerate}
Topics in Geometry
\begin{enumerate}
\setcounter{enumi}{81}
\item \hyperref[spaces-chow-section-phantom]{Chow Groups of Spaces}
\item \hyperref[groupoids-quotients-section-phantom]{Quotients of Groupoids}
\item \hyperref[spaces-more-cohomology-section-phantom]{More on Cohomology of Spaces}
\item \hyperref[spaces-simplicial-section-phantom]{Simplicial Spaces}
\item \hyperref[spaces-duality-section-phantom]{Duality for Spaces}
\item \hyperref[formal-spaces-section-phantom]{Formal Algebraic Spaces}
\item \hyperref[restricted-section-phantom]{Algebraization of Formal Spaces}
\item \hyperref[spaces-resolve-section-phantom]{Resolution of Surfaces Revisited}
\end{enumerate}
Deformation Theory
\begin{enumerate}
\setcounter{enumi}{89}
\item \hyperref[formal-defos-section-phantom]{Formal Deformation Theory}
\item \hyperref[defos-section-phantom]{Deformation Theory}
\item \hyperref[cotangent-section-phantom]{The Cotangent Complex}
\item \hyperref[examples-defos-section-phantom]{Deformation Problems}
\end{enumerate}
Algebraic Stacks
\begin{enumerate}
\setcounter{enumi}{93}
\item \hyperref[algebraic-section-phantom]{Algebraic Stacks}
\item \hyperref[examples-stacks-section-phantom]{Examples of Stacks}
\item \hyperref[stacks-sheaves-section-phantom]{Sheaves on Algebraic Stacks}
\item \hyperref[criteria-section-phantom]{Criteria for Representability}
\item \hyperref[artin-section-phantom]{Artin's Axioms}
\item \hyperref[quot-section-phantom]{Quot and Hilbert Spaces}
\item \hyperref[stacks-properties-section-phantom]{Properties of Algebraic Stacks}
\item \hyperref[stacks-morphisms-section-phantom]{Morphisms of Algebraic Stacks}
\item \hyperref[stacks-limits-section-phantom]{Limits of Algebraic Stacks}
\item \hyperref[stacks-cohomology-section-phantom]{Cohomology of Algebraic Stacks}
\item \hyperref[stacks-perfect-section-phantom]{Derived Categories of Stacks}
\item \hyperref[stacks-introduction-section-phantom]{Introducing Algebraic Stacks}
\item \hyperref[stacks-more-morphisms-section-phantom]{More on Morphisms of Stacks}
\item \hyperref[stacks-geometry-section-phantom]{The Geometry of Stacks}
\end{enumerate}
Topics in Moduli Theory
\begin{enumerate}
\setcounter{enumi}{107}
\item \hyperref[moduli-section-phantom]{Moduli Stacks}
\item \hyperref[moduli-curves-section-phantom]{Moduli of Curves}
\end{enumerate}
Miscellany
\begin{enumerate}
\setcounter{enumi}{109}
\item \hyperref[examples-section-phantom]{Examples}
\item \hyperref[exercises-section-phantom]{Exercises}
\item \hyperref[guide-section-phantom]{Guide to Literature}
\item \hyperref[desirables-section-phantom]{Desirables}
\item \hyperref[coding-section-phantom]{Coding Style}
\item \hyperref[obsolete-section-phantom]{Obsolete}
\item \hyperref[fdl-section-phantom]{GNU Free Documentation License}
\item \hyperref[index-section-phantom]{Auto Generated Index}
\end{enumerate}
\end{multicols}


\bibliography{my}
\bibliographystyle{amsalpha}

\end{document}
