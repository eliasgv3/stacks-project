\IfFileExists{stacks-project.cls}{%
\documentclass{stacks-project}
}{%
\documentclass{amsart}
}

% For dealing with references we use the comment environment
\usepackage{verbatim}
\newenvironment{reference}{\comment}{\endcomment}
%\newenvironment{reference}{}{}
\newenvironment{slogan}{\comment}{\endcomment}
\newenvironment{history}{\comment}{\endcomment}

% For commutative diagrams we use Xy-pic
\usepackage[all]{xy}

% We use 2cell for 2-commutative diagrams.
\xyoption{2cell}
\UseAllTwocells

% We use multicol for the list of chapters between chapters
\usepackage{multicol}

% This is generall recommended for better output
\usepackage{lmodern}
\usepackage[T1]{fontenc}

% For cross-file-references
\usepackage{xr-hyper}

% Package for hypertext links:
\usepackage{hyperref}

% For any local file, say "hello.tex" you want to link to please
% use \externaldocument[hello-]{hello}
\externaldocument[introduction-]{introduction}
\externaldocument[conventions-]{conventions}
\externaldocument[sets-]{sets}
\externaldocument[categories-]{categories}
\externaldocument[topology-]{topology}
\externaldocument[sheaves-]{sheaves}
\externaldocument[sites-]{sites}
\externaldocument[stacks-]{stacks}
\externaldocument[fields-]{fields}
\externaldocument[algebra-]{algebra}
\externaldocument[brauer-]{brauer}
\externaldocument[homology-]{homology}
\externaldocument[derived-]{derived}
\externaldocument[simplicial-]{simplicial}
\externaldocument[more-algebra-]{more-algebra}
\externaldocument[smoothing-]{smoothing}
\externaldocument[modules-]{modules}
\externaldocument[sites-modules-]{sites-modules}
\externaldocument[injectives-]{injectives}
\externaldocument[cohomology-]{cohomology}
\externaldocument[sites-cohomology-]{sites-cohomology}
\externaldocument[dga-]{dga}
\externaldocument[dpa-]{dpa}
\externaldocument[sdga-]{sdga}
\externaldocument[hypercovering-]{hypercovering}
\externaldocument[schemes-]{schemes}
\externaldocument[constructions-]{constructions}
\externaldocument[properties-]{properties}
\externaldocument[morphisms-]{morphisms}
\externaldocument[coherent-]{coherent}
\externaldocument[divisors-]{divisors}
\externaldocument[limits-]{limits}
\externaldocument[varieties-]{varieties}
\externaldocument[topologies-]{topologies}
\externaldocument[descent-]{descent}
\externaldocument[perfect-]{perfect}
\externaldocument[more-morphisms-]{more-morphisms}
\externaldocument[flat-]{flat}
\externaldocument[groupoids-]{groupoids}
\externaldocument[more-groupoids-]{more-groupoids}
\externaldocument[etale-]{etale}
\externaldocument[chow-]{chow}
\externaldocument[intersection-]{intersection}
\externaldocument[pic-]{pic}
\externaldocument[weil-]{weil}
\externaldocument[adequate-]{adequate}
\externaldocument[dualizing-]{dualizing}
\externaldocument[duality-]{duality}
\externaldocument[discriminant-]{discriminant}
\externaldocument[derham-]{derham}
\externaldocument[local-cohomology-]{local-cohomology}
\externaldocument[algebraization-]{algebraization}
\externaldocument[curves-]{curves}
\externaldocument[resolve-]{resolve}
\externaldocument[models-]{models}
\externaldocument[functors-]{functors}
\externaldocument[equiv-]{equiv}
\externaldocument[pione-]{pione}
\externaldocument[etale-cohomology-]{etale-cohomology}
\externaldocument[proetale-]{proetale}
\externaldocument[relative-cycles-]{relative-cycles}
\externaldocument[more-etale-]{more-etale}
\externaldocument[trace-]{trace}
\externaldocument[crystalline-]{crystalline}
\externaldocument[spaces-]{spaces}
\externaldocument[spaces-properties-]{spaces-properties}
\externaldocument[spaces-morphisms-]{spaces-morphisms}
\externaldocument[decent-spaces-]{decent-spaces}
\externaldocument[spaces-cohomology-]{spaces-cohomology}
\externaldocument[spaces-limits-]{spaces-limits}
\externaldocument[spaces-divisors-]{spaces-divisors}
\externaldocument[spaces-over-fields-]{spaces-over-fields}
\externaldocument[spaces-topologies-]{spaces-topologies}
\externaldocument[spaces-descent-]{spaces-descent}
\externaldocument[spaces-perfect-]{spaces-perfect}
\externaldocument[spaces-more-morphisms-]{spaces-more-morphisms}
\externaldocument[spaces-flat-]{spaces-flat}
\externaldocument[spaces-groupoids-]{spaces-groupoids}
\externaldocument[spaces-more-groupoids-]{spaces-more-groupoids}
\externaldocument[bootstrap-]{bootstrap}
\externaldocument[spaces-pushouts-]{spaces-pushouts}
\externaldocument[spaces-chow-]{spaces-chow}
\externaldocument[groupoids-quotients-]{groupoids-quotients}
\externaldocument[spaces-more-cohomology-]{spaces-more-cohomology}
\externaldocument[spaces-simplicial-]{spaces-simplicial}
\externaldocument[spaces-duality-]{spaces-duality}
\externaldocument[formal-spaces-]{formal-spaces}
\externaldocument[restricted-]{restricted}
\externaldocument[spaces-resolve-]{spaces-resolve}
\externaldocument[formal-defos-]{formal-defos}
\externaldocument[defos-]{defos}
\externaldocument[cotangent-]{cotangent}
\externaldocument[examples-defos-]{examples-defos}
\externaldocument[algebraic-]{algebraic}
\externaldocument[examples-stacks-]{examples-stacks}
\externaldocument[stacks-sheaves-]{stacks-sheaves}
\externaldocument[criteria-]{criteria}
\externaldocument[artin-]{artin}
\externaldocument[quot-]{quot}
\externaldocument[stacks-properties-]{stacks-properties}
\externaldocument[stacks-morphisms-]{stacks-morphisms}
\externaldocument[stacks-limits-]{stacks-limits}
\externaldocument[stacks-cohomology-]{stacks-cohomology}
\externaldocument[stacks-perfect-]{stacks-perfect}
\externaldocument[stacks-introduction-]{stacks-introduction}
\externaldocument[stacks-more-morphisms-]{stacks-more-morphisms}
\externaldocument[stacks-geometry-]{stacks-geometry}
\externaldocument[moduli-]{moduli}
\externaldocument[moduli-curves-]{moduli-curves}
\externaldocument[examples-]{examples}
\externaldocument[exercises-]{exercises}
\externaldocument[guide-]{guide}
\externaldocument[desirables-]{desirables}
\externaldocument[coding-]{coding}
\externaldocument[obsolete-]{obsolete}
\externaldocument[fdl-]{fdl}
\externaldocument[index-]{index}

% Theorem environments.
%
\theoremstyle{plain}
\newtheorem{theorem}[subsection]{Theorem}
\newtheorem{proposition}[subsection]{Proposition}
\newtheorem{lemma}[subsection]{Lemma}

\theoremstyle{definition}
\newtheorem{definition}[subsection]{Definition}
\newtheorem{example}[subsection]{Example}
\newtheorem{exercise}[subsection]{Exercise}
\newtheorem{situation}[subsection]{Situation}

\theoremstyle{remark}
\newtheorem{remark}[subsection]{Remark}
\newtheorem{remarks}[subsection]{Remarks}

\numberwithin{equation}{subsection}

% Macros
%
\def\lim{\mathop{\mathrm{lim}}\nolimits}
\def\colim{\mathop{\mathrm{colim}}\nolimits}
\def\Spec{\mathop{\mathrm{Spec}}}
\def\Hom{\mathop{\mathrm{Hom}}\nolimits}
\def\Ext{\mathop{\mathrm{Ext}}\nolimits}
\def\SheafHom{\mathop{\mathcal{H}\!\mathit{om}}\nolimits}
\def\SheafExt{\mathop{\mathcal{E}\!\mathit{xt}}\nolimits}
\def\Sch{\mathit{Sch}}
\def\Mor{\mathop{\mathrm{Mor}}\nolimits}
\def\Ob{\mathop{\mathrm{Ob}}\nolimits}
\def\Sh{\mathop{\mathit{Sh}}\nolimits}
\def\NL{\mathop{N\!L}\nolimits}
\def\CH{\mathop{\mathrm{CH}}\nolimits}
\def\proetale{{pro\text{-}\acute{e}tale}}
\def\etale{{\acute{e}tale}}
\def\QCoh{\mathit{QCoh}}
\def\Ker{\mathop{\mathrm{Ker}}}
\def\Im{\mathop{\mathrm{Im}}}
\def\Coker{\mathop{\mathrm{Coker}}}
\def\Coim{\mathop{\mathrm{Coim}}}

% Boxtimes
%
\DeclareMathSymbol{\boxtimes}{\mathbin}{AMSa}{"02}

%
% Macros for moduli stacks/spaces
%
\def\QCohstack{\mathcal{QC}\!\mathit{oh}}
\def\Cohstack{\mathcal{C}\!\mathit{oh}}
\def\Spacesstack{\mathcal{S}\!\mathit{paces}}
\def\Quotfunctor{\mathrm{Quot}}
\def\Hilbfunctor{\mathrm{Hilb}}
\def\Curvesstack{\mathcal{C}\!\mathit{urves}}
\def\Polarizedstack{\mathcal{P}\!\mathit{olarized}}
\def\Complexesstack{\mathcal{C}\!\mathit{omplexes}}
% \Pic is the operator that assigns to X its picard group, usage \Pic(X)
% \Picardstack_{X/B} denotes the Picard stack of X over B
% \Picardfunctor_{X/B} denotes the Picard functor of X over B
\def\Pic{\mathop{\mathrm{Pic}}\nolimits}
\def\Picardstack{\mathcal{P}\!\mathit{ic}}
\def\Picardfunctor{\mathrm{Pic}}
\def\Deformationcategory{\mathcal{D}\!\mathit{ef}}


% OK, start here.
%
\begin{document}

\title{Divided Power Algebra}


\maketitle

\phantomsection
\label{section-phantom}

\tableofcontents

\section{Introduction}
\label{section-introduction}

\noindent
In this chapter we talk about divided power algebras and what
you can do with them. A reference is the book \cite{Berthelot}.





\section{Divided powers}
\label{section-divided-powers}

\noindent
In this section we collect some results on divided power rings.
We will use the convention $0! = 1$ (as empty products should give $1$).

\begin{definition}
\label{definition-divided-powers}
Let $A$ be a ring. Let $I$ be an ideal of $A$. A collection of maps
$\gamma_n : I \to I$, $n > 0$ is called a {\it divided power structure}
on $I$ if for all $n \geq 0$, $m > 0$, $x, y \in I$, and $a \in A$ we have
\begin{enumerate}
\item $\gamma_1(x) = x$, we also set $\gamma_0(x) = 1$,
\item $\gamma_n(x)\gamma_m(x) = \frac{(n + m)!}{n! m!} \gamma_{n + m}(x)$,
\item $\gamma_n(ax) = a^n \gamma_n(x)$,
\item $\gamma_n(x + y) = \sum_{i = 0, \ldots, n} \gamma_i(x)\gamma_{n - i}(y)$,
\item $\gamma_n(\gamma_m(x)) = \frac{(nm)!}{n! (m!)^n} \gamma_{nm}(x)$.
\end{enumerate}
\end{definition}

\noindent
Note that the rational numbers $\frac{(n + m)!}{n! m!}$
and $\frac{(nm)!}{n! (m!)^n}$ occurring in the definition are in fact integers;
the first is the number of ways to choose $n$ out of $n + m$ and
the second counts the number of ways to divide a group of $nm$
objects into $n$ groups of $m$.
We make some remarks about the definition which show that
$\gamma_n(x)$ is a replacement for $x^n/n!$ in $I$.

\begin{lemma}
\label{lemma-silly}
Let $A$ be a ring. Let $I$ be an ideal of $A$.
\begin{enumerate}
\item If $\gamma$ is a divided power structure\footnote{Here
and in the following, $\gamma$ stands short for a sequence
of maps $\gamma_1, \gamma_2, \gamma_3, \ldots$ from $I$ to $I$.}
on $I$, then
$n! \gamma_n(x) = x^n$ for $n \geq 1$, $x \in I$.
\end{enumerate}
Assume $A$ is torsion free as a $\mathbf{Z}$-module.
\begin{enumerate}
\item[(2)] A divided power structure on $I$, if it exists, is unique.
\item[(3)] If $\gamma_n : I \to I$ are maps then
$$
\gamma\text{ is a divided power structure}
\Leftrightarrow
n! \gamma_n(x) = x^n\ \forall x \in I, n \geq 1.
$$
\item[(4)] The ideal $I$ has a divided power structure
if and only if there exists
a set of generators $x_i$ of $I$ as an ideal such that
for all $n \geq 1$ we have $x_i^n \in (n!)I$.
\end{enumerate}
\end{lemma}

\begin{proof}
Proof of (1). If $\gamma$ is a divided power structure, then condition
(2) (applied to $1$ and $n-1$ instead of $n$ and $m$)
implies that $n \gamma_n(x) = \gamma_1(x)\gamma_{n - 1}(x)$. Hence
by induction and condition (1) we get $n! \gamma_n(x) = x^n$.

\medskip\noindent
Assume $A$ is torsion free as a $\mathbf{Z}$-module.
Proof of (2). This is clear from (1).

\medskip\noindent
Proof of (3). Assume that $n! \gamma_n(x) = x^n$ for all $x \in I$ and
$n \geq 1$. Since $A \subset A \otimes_{\mathbf{Z}} \mathbf{Q}$ it suffices
to prove the axioms (1) -- (5) of Definition
\ref{definition-divided-powers} in case $A$ is a $\mathbf{Q}$-algebra.
In this case $\gamma_n(x) = x^n/n!$ and it is straightforward
to verify (1) -- (5); for example, (4) corresponds to the binomial
formula
$$
(x + y)^n = \sum_{i = 0, \ldots, n} \frac{n!}{i!(n - i)!} x^iy^{n - i}
$$
We encourage the reader to do the verifications
to make sure that we have the coefficients correct.

\medskip\noindent
Proof of (4). Assume we have generators $x_i$ of $I$ as an ideal
such that $x_i^n \in (n!)I$ for all $n \geq 1$. We claim that
for all $x \in I$ we have $x^n \in (n!)I$. If the claim holds then
we can set $\gamma_n(x) = x^n/n!$ which is a divided power structure by (3).
To prove the claim we note that it holds for $x = ax_i$. Hence we see
that the claim holds for a set of generators of $I$ as an abelian group.
By induction on the length of an expression in terms of these, it suffices
to prove the claim for $x + y$ if it holds for $x$ and $y$. This
follows immediately from the binomial theorem.
\end{proof}

\begin{example}
\label{example-ideal-generated-by-p}
Let $p$ be a prime number.
Let $A$ be a ring such that every integer $n$ not divisible by $p$
is invertible, i.e., $A$ is a $\mathbf{Z}_{(p)}$-algebra. Then
$I = pA$ has a canonical divided power structure. Namely, given
$x = pa \in I$ we set
$$
\gamma_n(x) = \frac{p^n}{n!} a^n
$$
The reader verifies immediately that $p^n/n! \in p\mathbf{Z}_{(p)}$
for $n \geq 1$ (for instance, this can be derived from the fact
that the exponent of $p$ in the prime factorization of $n!$ is
$\left\lfloor n/p \right\rfloor + \left\lfloor n/p^2 \right\rfloor
+ \left\lfloor n/p^3 \right\rfloor + \ldots$),
so that the definition makes sense and gives us a sequence of
maps $\gamma_n : I \to I$. It is a straightforward exercise to
verify that conditions (1) -- (5) of
Definition \ref{definition-divided-powers} are satisfied.
Alternatively, it is clear that the definition works for
$A_0 = \mathbf{Z}_{(p)}$ and then the result follows from
Lemma \ref{lemma-gamma-extends}.
\end{example}

\noindent
We notice that $\gamma_n\left(0\right) = 0$ for any ideal $I$ of
$A$ and any divided power structure $\gamma$ on $I$. (This follows
from axiom (3) in Definition \ref{definition-divided-powers},
applied to $a=0$.)

\begin{lemma}
\label{lemma-check-on-generators}
Let $A$ be a ring. Let $I$ be an ideal of $A$. Let $\gamma_n : I \to I$,
$n \geq 1$ be a sequence of maps. Assume
\begin{enumerate}
\item[(a)] (1), (3), and (4) of Definition \ref{definition-divided-powers}
hold for all $x, y \in I$, and
\item[(b)] properties (2) and (5) hold for $x$ in
some set of generators of $I$ as an ideal.
\end{enumerate}
Then $\gamma$ is a divided power structure on $I$.
\end{lemma}

\begin{proof}
The numbers (1), (2), (3), (4), (5) in this proof refer to the
conditions listed in Definition \ref{definition-divided-powers}.
Applying (3) we see that if (2) and (5) hold for $x$ then (2) and (5)
hold for $ax$ for all $a \in A$. Hence we see (b) implies
(2) and (5) hold for a set of generators
of $I$ as an abelian group. Hence, by induction of the length
of an expression in terms of these it suffices to prove that, given
$x, y \in I$ such that (2) and (5) hold for $x$ and $y$, then (2) and (5) hold
for $x + y$.

\medskip\noindent
Proof of (2) for $x + y$. By (4) we have
$$
\gamma_n(x + y)\gamma_m(x + y) =
\sum\nolimits_{i + j = n,\ k + l = m}
\gamma_i(x)\gamma_k(x)\gamma_j(y)\gamma_l(y)
$$
Using (2) for $x$ and $y$ this equals
$$
\sum \frac{(i + k)!}{i!k!}\frac{(j + l)!}{j!l!}
\gamma_{i + k}(x)\gamma_{j + l}(y)
$$
Comparing this with the expansion
$$
\gamma_{n + m}(x + y) = \sum \gamma_a(x)\gamma_b(y)
$$
we see that we have to prove that given $a + b = n + m$ we have
$$
\sum\nolimits_{i + k = a,\ j + l = b,\ i + j = n,\ k + l = m}
\frac{(i + k)!}{i!k!}\frac{(j + l)!}{j!l!}
=
\frac{(n + m)!}{n!m!}.
$$
Instead of arguing this directly, we note that the result is true
for the ideal $I = (x, y)$ in the polynomial ring $\mathbf{Q}[x, y]$
because $\gamma_n(f) = f^n/n!$, $f \in I$ defines a divided power
structure on $I$. Hence the equality of rational numbers above is true.

\medskip\noindent
Proof of (5) for $x + y$ given that (1) -- (4) hold and that (5)
holds for $x$ and $y$. We will again reduce the proof to an equality
of rational numbers. Namely, using (4) we can write
$\gamma_n(\gamma_m(x + y)) = \gamma_n(\sum \gamma_i(x)\gamma_j(y))$.
Using (4) we can write
$\gamma_n(\gamma_m(x + y))$ as a sum of terms which are products of
factors of the form $\gamma_k(\gamma_i(x)\gamma_j(y))$.
If $i > 0$ then
\begin{align*}
\gamma_k(\gamma_i(x)\gamma_j(y)) & =
\gamma_j(y)^k\gamma_k(\gamma_i(x)) \\
& = \frac{(ki)!}{k!(i!)^k} \gamma_j(y)^k \gamma_{ki}(x) \\
& =
\frac{(ki)!}{k!(i!)^k} \frac{(kj)!}{(j!)^k} \gamma_{ki}(x) \gamma_{kj}(y)
\end{align*}
using (3) in the first equality, (5) for $x$ in the second, and
(2) exactly $k$ times in the third. Using (5) for $y$ we see the
same equality holds when $i = 0$. Continuing like this using all
axioms but (5) we see that we can write
$$
\gamma_n(\gamma_m(x + y)) =
\sum\nolimits_{i + j = nm} c_{ij}\gamma_i(x)\gamma_j(y)
$$
for certain universal constants $c_{ij} \in \mathbf{Z}$. Again the fact
that the equality is valid in the polynomial ring $\mathbf{Q}[x, y]$
implies that the coefficients $c_{ij}$ are all equal to $(nm)!/n!(m!)^n$
as desired.
\end{proof}

\begin{lemma}
\label{lemma-two-ideals}
Let $A$ be a ring with two ideals $I, J \subset A$.
Let $\gamma$ be a divided power structure on $I$ and let
$\delta$ be a divided power structure on $J$.
Then
\begin{enumerate}
\item $\gamma$ and $\delta$ agree on $IJ$,
\item if $\gamma$ and $\delta$ agree on $I \cap J$ then they are
the restriction of a unique divided power structure $\epsilon$
on $I + J$.
\end{enumerate}
\end{lemma}

\begin{proof}
Let $x \in I$ and $y \in J$. Then
$$
\gamma_n(xy) = y^n\gamma_n(x) = n! \delta_n(y) \gamma_n(x) =
\delta_n(y) x^n = \delta_n(xy).
$$
Hence $\gamma$ and $\delta$ agree on a set of (additive) generators
of $IJ$. By property (4) of Definition \ref{definition-divided-powers}
it follows that they agree on all of $IJ$.

\medskip\noindent
Assume $\gamma$ and $\delta$ agree on $I \cap J$.
Let $z \in I + J$. Write $z = x + y$ with $x \in I$ and $y \in J$.
Then we set
$$
\epsilon_n(z) = \sum \gamma_i(x)\delta_{n - i}(y)
$$
for all $n \geq 1$.
To see that this is well defined, suppose that $z = x' + y'$ is another
representation with $x' \in I$ and $y' \in J$. Then
$w = x - x' = y' - y \in I \cap J$. Hence
\begin{align*}
\sum\nolimits_{i + j = n} \gamma_i(x)\delta_j(y)
& =
\sum\nolimits_{i + j = n} \gamma_i(x' + w)\delta_j(y) \\
& =
\sum\nolimits_{i' + l + j = n} \gamma_{i'}(x')\gamma_l(w)\delta_j(y) \\
& =
\sum\nolimits_{i' + l + j = n} \gamma_{i'}(x')\delta_l(w)\delta_j(y) \\
& =
\sum\nolimits_{i' + j' = n} \gamma_{i'}(x')\delta_{j'}(y + w) \\
& =
\sum\nolimits_{i' + j' = n} \gamma_{i'}(x')\delta_{j'}(y')
\end{align*}
as desired. Hence, we have defined maps
$\epsilon_n : I + J \to I + J$ for all $n \geq 1$; it is easy
to see that $\epsilon_n \mid_{I} = \gamma_n$ and
$\epsilon_n \mid_{J} = \delta_n$.
Next, we prove conditions (1) -- (5) of
Definition \ref{definition-divided-powers} for the collection
of maps $\epsilon_n$.
Properties (1) and (3) are clear. To see (4), suppose
that $z = x + y$ and $z' = x' + y'$ with $x, x' \in I$ and $y, y' \in J$
and compute
\begin{align*}
\epsilon_n(z + z') & =
\sum\nolimits_{a + b = n} \gamma_a(x + x')\delta_b(y + y') \\
& =
\sum\nolimits_{i + i' + j + j' = n}
\gamma_i(x) \gamma_{i'}(x')\delta_j(y)\delta_{j'}(y') \\
& =
\sum\nolimits_{k = 0, \ldots, n}
\sum\nolimits_{i+j=k} \gamma_i(x)\delta_j(y)
\sum\nolimits_{i'+j'=n-k} \gamma_{i'}(x')\delta_{j'}(y') \\
& =
\sum\nolimits_{k = 0, \ldots, n}\epsilon_k(z)\epsilon_{n-k}(z')
\end{align*}
as desired. Now we see that it suffices to prove (2) and (5) for
elements of $I$ or $J$, see Lemma \ref{lemma-check-on-generators}.
This is clear because $\gamma$ and $\delta$ are divided power
structures.

\medskip\noindent
The existence of a divided power structure $\epsilon$ on $I+J$
whose restrictions to $I$ and $J$ are $\gamma$ and $\delta$ is
thus proven; its uniqueness is rather clear.
\end{proof}

\begin{lemma}
\label{lemma-nil}
Let $p$ be a prime number. Let $A$ be a ring, let $I \subset A$ be an ideal,
and let $\gamma$ be a divided power structure on $I$. Assume $p$ is nilpotent
in $A/I$. Then $I$ is locally nilpotent if and only if $p$ is nilpotent in $A$.
\end{lemma}

\begin{proof}
If $p^N = 0$ in $A$, then for $x \in I$ we have
$x^{pN} = (pN)!\gamma_{pN}(x) = 0$ because $(pN)!$ is
divisible by $p^N$. Conversely, assume $I$ is locally nilpotent.
We've also assumed that $p$ is nilpotent in $A/I$, hence
$p^r \in I$ for some $r$, hence $p^r$ nilpotent, hence $p$ nilpotent.
\end{proof}








\section{Divided power rings}
\label{section-divided-power-rings}

\noindent
There is a category of divided power rings.
Here is the definition.

\begin{definition}
\label{definition-divided-power-ring}
A {\it divided power ring} is a triple $(A, I, \gamma)$ where
$A$ is a ring, $I \subset A$ is an ideal, and $\gamma = (\gamma_n)_{n \geq 1}$
is a divided power structure on $I$.
A {\it homomorphism of divided power rings}
$\varphi : (A, I, \gamma) \to (B, J, \delta)$ is a ring homomorphism
$\varphi : A \to B$ such that $\varphi(I) \subset J$ and such that
$\delta_n(\varphi(x)) = \varphi(\gamma_n(x))$ for all $x \in I$ and
$n \geq 1$.
\end{definition}

\noindent
We sometimes say ``let $(B, J, \delta)$ be a divided power algebra over
$(A, I, \gamma)$'' to indicate that $(B, J, \delta)$ is a divided power ring
which comes equipped with a homomorphism of divided power rings
$(A, I, \gamma) \to (B, J, \delta)$.

\begin{lemma}
\label{lemma-limits}
The category of divided power rings has all limits and they agree with
limits in the category of rings.
\end{lemma}

\begin{proof}
The empty limit is the zero ring (that's weird but we need it).
The product of a collection of divided power rings $(A_t, I_t, \gamma_t)$,
$t \in T$ is given by $(\prod A_t, \prod I_t, \gamma)$ where
$\gamma_n((x_t)) = (\gamma_{t, n}(x_t))$.
The equalizer of $\alpha, \beta : (A, I, \gamma) \to (B, J, \delta)$
is just $C = \{a \in A \mid \alpha(a) = \beta(a)\}$ with ideal $C \cap I$
and induced divided powers. It follows that all limits exist, see
Categories, Lemma \ref{categories-lemma-limits-products-equalizers}.
\end{proof}

\noindent
The following lemma illustrates a very general category theoretic
phenomenon in the case of divided power algebras.

\begin{lemma}
\label{lemma-a-version-of-brown}
Let $\mathcal{C}$ be the category of divided power rings. Let
$F : \mathcal{C} \to \textit{Sets}$ be a functor.
Assume that
\begin{enumerate}
\item there exists a cardinal $\kappa$ such that for every
$f \in F(A, I, \gamma)$ there exists a morphism
$(A', I', \gamma') \to (A, I, \gamma)$ of $\mathcal{C}$ such that $f$
is the image of $f' \in F(A', I', \gamma')$ and $|A'| \leq \kappa$, and
\item $F$ commutes with limits.
\end{enumerate}
Then $F$ is representable, i.e., there exists an object $(B, J, \delta)$
of $\mathcal{C}$ such that
$$
F(A, I, \gamma) = \Hom_\mathcal{C}((B, J, \delta), (A, I, \gamma))
$$
functorially in $(A, I, \gamma)$.
\end{lemma}

\begin{proof}
This is a special case of
Categories, Lemma \ref{categories-lemma-a-version-of-brown}.
\end{proof}

\begin{lemma}
\label{lemma-colimits}
The category of divided power rings has all colimits.
\end{lemma}

\begin{proof}
The empty colimit is $\mathbf{Z}$ with divided power ideal $(0)$.
Let's discuss general colimits. Let $\mathcal{C}$ be a category and let
$c \mapsto (A_c, I_c, \gamma_c)$ be a diagram. Consider the functor
$$
F(B, J, \delta) = \lim_{c \in \mathcal{C}}
Hom((A_c, I_c, \gamma_c), (B, J, \delta))
$$
Note that any $f = (f_c)_{c \in C} \in F(B, J, \delta)$ has the property
that all the images $f_c(A_c)$ generate a subring $B'$ of $B$ of bounded
cardinality $\kappa$ and that all the images $f_c(I_c)$ generate a
divided power sub ideal $J'$ of $B'$. And we get a factorization of
$f$ as a $f'$ in $F(B')$ followed by the inclusion $B' \to B$. Also,
$F$ commutes with limits. Hence we may apply
Lemma \ref{lemma-a-version-of-brown}
to see that $F$ is representable and we win.
\end{proof}

\begin{remark}
\label{remark-forgetful}
The forgetful functor $(A, I, \gamma) \mapsto A$ does not commute with
colimits. For example, let
$$
\xymatrix{
(B, J, \delta) \ar[r] & (B'', J'', \delta'') \\
(A, I, \gamma) \ar[r] \ar[u] & (B', J', \delta') \ar[u]
}
$$
be a pushout in the category of divided power rings.
Then in general the map $B \otimes_A B' \to B''$ isn't an
isomorphism. (It is always surjective.)
An explicit example is given by
$(A, I, \gamma) = (\mathbf{Z}, (0), \emptyset)$,
$(B, J, \delta) = (\mathbf{Z}/4\mathbf{Z}, 2\mathbf{Z}/4\mathbf{Z}, \delta)$,
and
$(B', J', \delta') =
(\mathbf{Z}/4\mathbf{Z}, 2\mathbf{Z}/4\mathbf{Z}, \delta')$
where $\delta_2(2) = 2$ and $\delta'_2(2) = 0$.
More precisely, using Lemma \ref{lemma-need-only-gamma-p}
we let $\delta$, resp.\ $\delta'$ be the unique
divided power structure on $J$, resp.\ $J'$ such that
$\delta_2 : J \to J$, resp.\ $\delta'_2 : J' \to J'$
is the map $0 \mapsto 0, 2 \mapsto 2$, resp.\ $0 \mapsto 0, 2 \mapsto 0$.
Then $(B'', J'', \delta'') = (\mathbf{F}_2, (0), \emptyset)$
which doesn't agree with the tensor product. However, note that it is always
true that
$$
B''/J'' = B/J \otimes_{A/I} B'/J'
$$
as can be seen from the universal property of the pushout by considering
maps into divided power algebras of the form $(C, (0), \emptyset)$.
\end{remark}


\section{Extending divided powers}
\label{section-extend}

\noindent
Here is the definition.

\begin{definition}
\label{definition-extends}
Given a divided power ring $(A, I, \gamma)$ and a ring map
$A \to B$ we say $\gamma$ {\it extends} to $B$ if there exists a
divided power structure $\bar \gamma$ on $IB$ such that
$(A, I, \gamma) \to (B, IB, \bar\gamma)$ is a homomorphism of
divided power rings.
\end{definition}

\begin{lemma}
\label{lemma-gamma-extends}
Let $(A, I, \gamma)$ be a divided power ring.
Let $A \to B$ be a ring map.
If $\gamma$ extends to $B$ then it extends uniquely.
Assume (at least) one of the following conditions holds
\begin{enumerate}
\item $IB = 0$,
\item $I$ is principal, or
\item $A \to B$ is flat.
\end{enumerate}
Then $\gamma$ extends to $B$.
\end{lemma}

\begin{proof}
Any element of $IB$ can be written as a finite sum
$\sum\nolimits_{i=1}^t b_ix_i$ with
$b_i \in B$ and $x_i \in I$. If $\gamma$ extends to $\bar\gamma$ on $IB$
then $\bar\gamma_n(x_i) = \gamma_n(x_i)$.
Thus, conditions (3) and (4) in
Definition \ref{definition-divided-powers} imply that
$$
\bar\gamma_n(\sum\nolimits_{i=1}^t b_ix_i) =
\sum\nolimits_{n_1 + \ldots + n_t = n}
\prod\nolimits_{i = 1}^t b_i^{n_i}\gamma_{n_i}(x_i)
$$
Thus we see that $\bar\gamma$ is unique if it exists.

\medskip\noindent
If $IB = 0$ then setting $\bar\gamma_n(0) = 0$ works. If $I = (x)$
then we define $\bar\gamma_n(bx) = b^n\gamma_n(x)$. This is well defined:
if $b'x = bx$, i.e., $(b - b')x = 0$ then
\begin{align*}
b^n\gamma_n(x) - (b')^n\gamma_n(x)
& =
(b^n - (b')^n)\gamma_n(x) \\
& =
(b^{n - 1} + \ldots + (b')^{n - 1})(b - b')\gamma_n(x) = 0
\end{align*}
because $\gamma_n(x)$ is divisible by $x$ (since
$\gamma_n(I) \subset I$) and hence annihilated by $b - b'$.
Next, we prove conditions (1) -- (5) of
Definition \ref{definition-divided-powers}.
Parts (1), (2), (3), (5) are obvious from the construction.
For (4) suppose that $y, z \in IB$, say $y = bx$ and $z = cx$. Then
$y + z = (b + c)x$ hence
\begin{align*}
\bar\gamma_n(y + z)
& =
(b + c)^n\gamma_n(x) \\
& =
\sum \frac{n!}{i!(n - i)!}b^ic^{n -i}\gamma_n(x) \\
& =
\sum b^ic^{n - i}\gamma_i(x)\gamma_{n - i}(x) \\
& =
\sum \bar\gamma_i(y)\bar\gamma_{n -i}(z)
\end{align*}
as desired.

\medskip\noindent
Assume $A \to B$ is flat. Suppose that $b_1, \ldots, b_r \in B$ and
$x_1, \ldots, x_r \in I$. Then
$$
\bar\gamma_n(\sum b_ix_i) =
\sum b_1^{e_1} \ldots b_r^{e_r} \gamma_{e_1}(x_1) \ldots \gamma_{e_r}(x_r)
$$
where the sum is over $e_1 + \ldots + e_r = n$
if $\bar\gamma_n$ exists. Next suppose that we have $c_1, \ldots, c_s \in B$
and $a_{ij} \in A$ such that $b_i = \sum a_{ij}c_j$.
Setting $y_j = \sum a_{ij}x_i$ we claim that
$$
\sum b_1^{e_1} \ldots b_r^{e_r} \gamma_{e_1}(x_1) \ldots \gamma_{e_r}(x_r) =
\sum c_1^{d_1} \ldots c_s^{d_s} \gamma_{d_1}(y_1) \ldots \gamma_{d_s}(y_s)
$$
in $B$ where on the right hand side we are summing over
$d_1 + \ldots + d_s = n$. Namely, using the axioms of a divided power
structure we can expand both sides into a sum with coefficients
in $\mathbf{Z}[a_{ij}]$ of terms of the form
$c_1^{d_1} \ldots c_s^{d_s}\gamma_{e_1}(x_1) \ldots \gamma_{e_r}(x_r)$.
To see that the coefficients agree we note that the result is true
in $\mathbf{Q}[x_1, \ldots, x_r, c_1, \ldots, c_s, a_{ij}]$ with
$\gamma$ the unique divided power structure on $(x_1, \ldots, x_r)$.
By Lazard's theorem (Algebra, Theorem \ref{algebra-theorem-lazard})
we can write $B$ as a directed colimit of finite free $A$-modules.
In particular, if $z \in IB$ is written as $z = \sum x_ib_i$ and
$z = \sum x'_{i'}b'_{i'}$, then we can find $c_1, \ldots, c_s \in B$
and $a_{ij}, a'_{i'j} \in A$ such that $b_i = \sum a_{ij}c_j$
and $b'_{i'} = \sum a'_{i'j}c_j$ such that
$y_j = \sum x_ia_{ij} = \sum x'_{i'}a'_{i'j}$ holds\footnote{This
can also be proven without recourse to
Algebra, Theorem \ref{algebra-theorem-lazard}. Indeed, if
$z = \sum x_ib_i$ and $z = \sum x'_{i'}b'_{i'}$, then
$\sum x_ib_i - \sum x'_{i'}b'_{i'} = 0$ is a relation in the
$A$-module $B$. Thus, Algebra, Lemma \ref{algebra-lemma-flat-eq}
(applied to the $x_i$ and $x'_{i'}$ taking the place of the $f_i$,
and the $b_i$ and $b'_{i'}$ taking the role of the $x_i$) yields
the existence of the $c_1, \ldots, c_s \in B$
and $a_{ij}, a'_{i'j} \in A$ as required.}.
Hence the procedure above gives a well defined map $\bar\gamma_n$
on $IB$. By construction $\bar\gamma$ satisfies conditions (1), (3), and
(4). Moreover, for $x \in I$ we have $\bar\gamma_n(x) = \gamma_n(x)$. Hence
it follows from Lemma \ref{lemma-check-on-generators} that $\bar\gamma$
is a divided power structure on $IB$.
\end{proof}

\begin{lemma}
\label{lemma-kernel}
Let $(A, I, \gamma)$ be a divided power ring.
\begin{enumerate}
\item If $\varphi : (A, I, \gamma) \to (B, J, \delta)$ is a
homomorphism of divided power rings, then $\Ker(\varphi) \cap I$
is preserved by $\gamma_n$ for all $n \geq 1$.
\item Let $\mathfrak a \subset A$ be an ideal and set
$I' = I \cap \mathfrak a$. The following are equivalent
\begin{enumerate}
\item $I'$ is preserved by $\gamma_n$ for all $n > 0$,
\item $\gamma$ extends to $A/\mathfrak a$, and
\item there exist a set of generators $x_i$ of $I'$ as an ideal
such that $\gamma_n(x_i) \in I'$ for all $n > 0$.
\end{enumerate}
\end{enumerate}
\end{lemma}

\begin{proof}
Proof of (1). This is clear. Assume (2)(a). Define
$\bar\gamma_n(x \bmod I') = \gamma_n(x) \bmod I'$ for $x \in I$.
This is well defined since $\gamma_n(x + y) = \gamma_n(x) \bmod I'$
for $y \in I'$ by Definition \ref{definition-divided-powers} (4) and
the fact that $\gamma_j(y) \in I'$ by assumption. It is clear that
$\bar\gamma$ is a divided power structure as $\gamma$ is one.
Hence (2)(b) holds. Also, (2)(b) implies (2)(a) by part (1).
It is clear that (2)(a) implies (2)(c). Assume (2)(c).
Note that $\gamma_n(x) = a^n\gamma_n(x_i) \in I'$ for $x = ax_i$.
Hence we see that $\gamma_n(x) \in I'$ for a set of generators of $I'$
as an abelian group. By induction on the length of an expression in
terms of these, it suffices to prove $\forall n : \gamma_n(x + y) \in I'$
if $\forall n : \gamma_n(x), \gamma_n(y) \in I'$. This
follows immediately from the fourth axiom of a divided power structure.
\end{proof}

\begin{lemma}
\label{lemma-sub-dp-ideal}
Let $(A, I, \gamma)$ be a divided power ring.
Let $E \subset I$ be a subset.
Then the smallest ideal $J \subset I$ preserved by $\gamma$
and containing all $f \in E$ is the ideal $J$ generated by
$\gamma_n(f)$, $n \geq 1$, $f \in E$.
\end{lemma}

\begin{proof}
Follows immediately from Lemma \ref{lemma-kernel}.
\end{proof}

\begin{lemma}
\label{lemma-extend-to-completion}
Let $(A, I, \gamma)$ be a divided power ring. Let $p$ be a prime.
If $p$ is nilpotent in $A/I$, then
\begin{enumerate}
\item the $p$-adic completion $A^\wedge = \lim_e A/p^eA$ surjects onto $A/I$,
\item the kernel of this map is the $p$-adic completion $I^\wedge$ of $I$, and
\item each $\gamma_n$ is continuous for the $p$-adic topology and extends
to $\gamma_n^\wedge : I^\wedge \to I^\wedge$ defining a divided power
structure on $I^\wedge$.
\end{enumerate}
If moreover $A$ is a $\mathbf{Z}_{(p)}$-algebra, then
\begin{enumerate}
\item[(4)] for $e$ large enough the ideal $p^eA \subset I$ is preserved by the
divided power structure $\gamma$ and
$$
(A^\wedge, I^\wedge, \gamma^\wedge) = \lim_e (A/p^eA, I/p^eA, \bar\gamma)
$$
in the category of divided power rings.
\end{enumerate}
\end{lemma}

\begin{proof}
Let $t \geq 1$ be an integer such that $p^tA/I = 0$, i.e., $p^tA \subset I$.
The map $A^\wedge \to A/I$ is the composition $A^\wedge \to A/p^tA \to A/I$
which is surjective (for example by
Algebra, Lemma \ref{algebra-lemma-completion-generalities}).
As $p^eI \subset p^eA \cap I \subset p^{e - t}I$ for $e \geq t$ we see
that the kernel of the composition $A^\wedge \to A/I$ is the $p$-adic
completion of $I$. The map $\gamma_n$ is continuous because
$$
\gamma_n(x + p^ey) =
\sum\nolimits_{i + j = n} p^{je}\gamma_i(x)\gamma_j(y) =
\gamma_n(x) \bmod p^eI
$$
by the axioms of a divided power structure. It is clear that the axioms
for divided power structures are inherited by the maps $\gamma_n^\wedge$
from the maps $\gamma_n$. Finally, to see the last statement say $e > t$.
Then $p^eA \subset I$ and $\gamma_1(p^eA) \subset p^eA$ and for $n > 1$
we have
$$
\gamma_n(p^ea) = p^n \gamma_n(p^{e - 1}a) = \frac{p^n}{n!} p^{n(e - 1)}a^n
\in p^e A
$$
as $p^n/n! \in \mathbf{Z}_{(p)}$ and as $n \geq 2$ and $e \geq 2$ so
$n(e - 1) \geq e$.
This proves that $\gamma$ extends to $A/p^eA$, see Lemma \ref{lemma-kernel}.
The statement on limits is clear from the construction of limits in
the proof of Lemma \ref{lemma-limits}.
\end{proof}




\section{Divided power polynomial algebras}
\label{section-divided-power-polynomial-ring}

\noindent
A very useful example is the {\it divided power polynomial algebra}.
Let $A$ be a ring. Let $t \geq 1$. We will denote
$A\langle x_1, \ldots, x_t \rangle$ the following $A$-algebra:
As an $A$-module we set
$$
A\langle x_1, \ldots, x_t \rangle =
\bigoplus\nolimits_{n_1, \ldots, n_t \geq 0} A x_1^{[n_1]} \ldots x_t^{[n_t]}
$$
with multiplication given by
$$
x_i^{[n]}x_i^{[m]} = \frac{(n + m)!}{n!m!}x_i^{[n + m]}.
$$
We also set $x_i = x_i^{[1]}$. Note that
$1 = x_1^{[0]} \ldots x_t^{[0]}$. There is a similar construction
which gives the divided power polynomial algebra in infinitely many
variables. There is an canonical $A$-algebra map
$A\langle x_1, \ldots, x_t \rangle \to A$ sending $x_i^{[n]}$ to zero
for $n > 0$. The kernel of this map is denoted
$A\langle x_1, \ldots, x_t \rangle_{+}$.

\begin{lemma}
\label{lemma-divided-power-polynomial-algebra}
Let $(A, I, \gamma)$ be a divided power ring.
There exists a unique divided power structure $\delta$ on
$$
J = IA\langle x_1, \ldots, x_t \rangle + A\langle x_1, \ldots, x_t \rangle_{+}
$$
such that
\begin{enumerate}
\item $\delta_n(x_i) = x_i^{[n]}$, and
\item $(A, I, \gamma) \to (A\langle x_1, \ldots, x_t \rangle, J, \delta)$
is a homomorphism of divided power rings.
\end{enumerate}
Moreover, $(A\langle x_1, \ldots, x_t \rangle, J, \delta)$ has the
following universal property: A homomorphism of divided power rings
$\varphi : (A\langle x_1, \ldots, x_t \rangle, J, \delta) \to
(C, K, \epsilon)$ is
the same thing as a homomorphism of divided power rings
$A \to C$ and elements $k_1, \ldots, k_t \in K$.
\end{lemma}

\begin{proof}
We will prove the lemma in case of a divided power polynomial algebra
in one variable. The result for the general case can be argued in exactly
the same way, or by noting that $A\langle x_1, \ldots, x_t\rangle$ is
isomorphic to the ring obtained by adjoining the divided power variables
$x_1, \ldots, x_t$ one by one.

\medskip\noindent
Let $A\langle x \rangle_{+}$ be the ideal generated by
$x, x^{[2]}, x^{[3]}, \ldots$.
Note that $J = IA\langle x \rangle + A\langle x \rangle_{+}$
and that
$$
IA\langle x \rangle \cap A\langle x \rangle_{+} =
IA\langle x \rangle \cdot A\langle x \rangle_{+}
$$
Hence by Lemma \ref{lemma-two-ideals} it suffices to show that there
exist divided power structures on the ideals $IA\langle x \rangle$ and
$A\langle x \rangle_{+}$. The existence of the first follows from
Lemma \ref{lemma-gamma-extends} as $A \to A\langle x \rangle$ is flat.
For the second, note that if $A$ is torsion free, then we can apply
Lemma \ref{lemma-silly} (4) to see that $\delta$ exists. Namely, choosing
as generators the elements $x^{[m]}$ we see that
$(x^{[m]})^n = \frac{(nm)!}{(m!)^n} x^{[nm]}$
and $n!$ divides the integer $\frac{(nm)!}{(m!)^n}$.
In general write $A = R/\mathfrak a$ for some torsion free ring $R$
(e.g., a polynomial ring over $\mathbf{Z}$). The kernel of
$R\langle x \rangle \to A\langle x \rangle$ is
$\bigoplus \mathfrak a x^{[m]}$. Applying criterion (2)(c) of
Lemma \ref{lemma-kernel} we see that the divided power structure
on $R\langle x \rangle_{+}$ extends to $A\langle x \rangle$ as
desired.

\medskip\noindent
Proof of the universal property. Given a homomorphism $\varphi : A \to C$
of divided power rings and $k_1, \ldots, k_t \in K$ we consider
$$
A\langle x_1, \ldots, x_t \rangle \to C,\quad
x_1^{[n_1]} \ldots x_t^{[n_t]} \longmapsto
\epsilon_{n_1}(k_1) \ldots \epsilon_{n_t}(k_t)
$$
using $\varphi$ on coefficients. The only thing to check is that
this is an $A$-algebra homomorphism (details omitted). The inverse
construction is clear.
\end{proof}

\begin{remark}
\label{remark-divided-power-polynomial-algebra}
Let $(A, I, \gamma)$ be a divided power ring.
There is a variant of Lemma \ref{lemma-divided-power-polynomial-algebra}
for infinitely many variables. First note that if $s < t$ then there
is a canonical map
$$
A\langle x_1, \ldots, x_s \rangle \to A\langle x_1, \ldots, x_t\rangle
$$
Hence if $W$ is any set, then we set
$$
A\langle x_w: w \in W\rangle =
\colim_{E \subset W} A\langle x_e:e \in E\rangle
$$
(colimit over $E$ finite subset of $W$)
with transition maps as above. By the definition of a colimit we see
that the universal mapping property of $A\langle x_w: w \in W\rangle$ is
completely analogous to the mapping property stated in
Lemma \ref{lemma-divided-power-polynomial-algebra}.
\end{remark}

\noindent
The following lemma can be found in \cite{BO}.

\begin{lemma}
\label{lemma-need-only-gamma-p}
Let $p$ be a prime number. Let $A$ be a ring such that every integer $n$
not divisible by $p$ is invertible, i.e., $A$ is a $\mathbf{Z}_{(p)}$-algebra.
Let $I \subset A$ be an ideal. Two divided power structures
$\gamma, \gamma'$ on $I$ are equal if and only if $\gamma_p = \gamma'_p$.
Moreover, given a map $\delta : I \to I$ such that
\begin{enumerate}
\item $p!\delta(x) = x^p$ for all $x \in I$,
\item $\delta(ax) = a^p\delta(x)$ for all $a \in A$, $x \in I$, and
\item
$\delta(x + y) =
\delta(x) +
\sum\nolimits_{i + j = p, i,j \geq 1} \frac{1}{i!j!} x^i y^j +
\delta(y)$ for all $x, y \in I$,
\end{enumerate}
then there exists a unique divided power structure $\gamma$ on $I$ such
that $\gamma_p = \delta$.
\end{lemma}

\begin{proof}
If $n$ is not divisible by $p$, then $\gamma_n(x) = c x \gamma_{n - 1}(x)$
where $c$ is a unit in $\mathbf{Z}_{(p)}$. Moreover,
$$
\gamma_{pm}(x) = c \gamma_m(\gamma_p(x))
$$
where $c$ is a unit in $\mathbf{Z}_{(p)}$. Thus the first assertion is clear.
For the second assertion, we can, working backwards, use these equalities
to define all $\gamma_n$. More precisely, if
$n = a_0 + a_1p + \ldots + a_e p^e$ with $a_i \in \{0, \ldots, p - 1\}$ then
we set
$$
\gamma_n(x) = c_n x^{a_0} \delta(x)^{a_1} \ldots \delta^e(x)^{a_e}
$$
for $c_n \in \mathbf{Z}_{(p)}$ defined by
$$
c_n =
{(p!)^{a_1 + a_2(1 + p) + \ldots + a_e(1 + \ldots + p^{e - 1})}}/{n!}.
$$
Now we have to show the axioms (1) -- (5) of a divided power structure, see
Definition \ref{definition-divided-powers}. We observe that (1) and (3) are
immediate. Verification of (2) and (5) is by a direct calculation which
we omit. Let $x, y \in I$. We claim there is a ring map
$$
\varphi : \mathbf{Z}_{(p)}\langle u, v \rangle \longrightarrow A
$$
which maps $u^{[n]}$ to $\gamma_n(x)$ and $v^{[n]}$ to $\gamma_n(y)$.
By construction of $\mathbf{Z}_{(p)}\langle u, v \rangle$ this means
we have to check that
$$
\gamma_n(x)\gamma_m(x) = \frac{(n + m)!}{n!m!} \gamma_{n + m}(x)
$$
in $A$ and similarly for $y$. This is true because (2) holds for $\gamma$.
Let $\epsilon$ denote the divided power structure on the
ideal $\mathbf{Z}_{(p)}\langle u, v\rangle_{+}$ of
$\mathbf{Z}_{(p)}\langle u, v\rangle$.
Next, we claim that $\varphi(\epsilon_n(f)) = \gamma_n(\varphi(f))$
for $f \in \mathbf{Z}_{(p)}\langle u, v\rangle_{+}$ and all $n$.
This is clear for $n = 0, 1, \ldots, p - 1$. For $n = p$ it suffices
to prove it for a set of generators of the ideal
$\mathbf{Z}_{(p)}\langle u, v\rangle_{+}$ because both $\epsilon_p$
and $\gamma_p = \delta$ satisfy properties (1) and (3) of the lemma.
Hence it suffices to prove that
$\gamma_p(\gamma_n(x)) = \frac{(pn)!}{p!(n!)^p}\gamma_{pn}(x)$ and
similarly for $y$, which follows as (5) holds for $\gamma$.
Now, if $n = a_0 + a_1p + \ldots + a_e p^e$
is an arbitrary integer written in $p$-adic expansion as above, then
$$
\epsilon_n(f) =
c_n f^{a_0} \gamma_p(f)^{a_1} \ldots \gamma_p^e(f)^{a_e}
$$
because $\epsilon$ is a divided power structure. Hence we see that
$\varphi(\epsilon_n(f)) = \gamma_n(\varphi(f))$ holds for all $n$.
Applying this for $f = u + v$ we see that axiom (4) for $\gamma$
follows from the fact that $\epsilon$ is a divided power structure.
\end{proof}













\section{Tate resolutions}
\label{section-tate}

\noindent
In this section we briefly discuss the resolutions constructed in
\cite{Tate-homology} and \cite{AH}
which combine divided power structures with
differential graded algebras.
In this section we will use {\it homological notation} for
differential graded algebras.
Our differential graded algebras will sit in nonnegative homological
degrees. Thus our differential graded algebras $(A, \text{d})$
will be given as chain complexes
$$
\ldots \to A_2 \to A_1 \to A_0 \to 0 \to \ldots
$$
endowed with a multiplication.

\medskip\noindent
Let $R$ be a ring (commutative, as usual).
In this section we will often consider graded
$R$-algebras $A = \bigoplus_{d \geq 0} A_d$ whose components are
zero in negative degrees. We will set $A_+ = \bigoplus_{d > 0} A_d$.
We will write $A_{even} = \bigoplus_{d \geq 0} A_{2d}$ and
$A_{odd} = \bigoplus_{d \geq 0} A_{2d + 1}$.
Recall that $A$ is graded commutative if
$x y = (-1)^{\deg(x)\deg(y)} y x$ for homogeneous elements $x, y$.
Recall that $A$ is strictly graded commutative if in addition
$x^2 = 0$ for homogeneous elements $x$ of odd degree. Finally, to understand
the following definition, keep in mind that $\gamma_n(x) = x^n/n!$
if $A$ is a $\mathbf{Q}$-algebra.

\begin{definition}
\label{definition-divided-powers-graded}
Let $R$ be a ring. Let $A = \bigoplus_{d \geq 0} A_d$ be a graded
$R$-algebra which is strictly graded commutative. A collection of maps
$\gamma_n : A_{even, +} \to A_{even, +}$ defined for all $n > 0$ is called
a {\it divided power structure} on $A$ if we have
\begin{enumerate}
\item $\gamma_n(x) \in A_{2nd}$ if $x \in A_{2d}$,
\item $\gamma_1(x) = x$ for any $x$, we also set $\gamma_0(x) = 1$,
\item $\gamma_n(x)\gamma_m(x) = \frac{(n + m)!}{n! m!} \gamma_{n + m}(x)$,
\item $\gamma_n(xy) = x^n \gamma_n(y)$ for all $x \in A_{even}$ and
$y \in A_{even, +}$,
\item $\gamma_n(xy) = 0$ if $x, y \in A_{odd}$ homogeneous and $n > 1$
\item if $x, y \in A_{even, +}$ then
$\gamma_n(x + y) = \sum_{i = 0, \ldots, n} \gamma_i(x)\gamma_{n - i}(y)$,
\item $\gamma_n(\gamma_m(x)) =
\frac{(nm)!}{n! (m!)^n} \gamma_{nm}(x)$ for $x \in A_{even, +}$.
\end{enumerate}
\end{definition}

\noindent
Observe that conditions (2), (3), (4), (6), and (7) imply that
$\gamma$ is a ``usual'' divided power structure on the ideal
$A_{even, +}$ of the (commutative) ring $A_{even}$, see
Sections \ref{section-divided-powers},
\ref{section-divided-power-rings},
\ref{section-extend}, and
\ref{section-divided-power-polynomial-ring}.
In particular, we have $n! \gamma_n(x) = x^n$ for all $x \in A_{even, +}$.
Condition (1) states that $\gamma$ is compatible with grading and condition
(5) tells us $\gamma_n$ for $n > 1$ vanishes on products
of homogeneous elements of odd degree. But note that it may happen
that
$$
\gamma_2(z_1 z_2 + z_3 z_4) = z_1z_2z_3z_4
$$
is nonzero if $z_1, z_2, z_3, z_4$ are homogeneous elements of odd degree.

\begin{example}[Adjoining odd variable]
\label{example-adjoining-odd}
Let $R$ be a ring. Let $(A, \gamma)$ be a strictly graded commutative
graded $R$-algebra endowed with a divided power structure as in the
definition above. Let $d > 0$ be an odd integer.
In this setting we can adjoin a variable $T$ of degree $d$ to $A$.
Namely, set
$$
A\langle T \rangle = A \oplus AT
$$
with grading given by $A\langle T \rangle_m = A_m \oplus A_{m - d}T$.
We claim there is a unique divided power structure on
$A\langle T \rangle$ compatible with the given divided power
structure on $A$. Namely, we set
$$
\gamma_n(x + yT) = \gamma_n(x) + \gamma_{n - 1}(x)yT
$$
for $x \in A_{even, +}$ and $y \in A_{odd}$.
\end{example}

\begin{example}[Adjoining even variable]
\label{example-adjoining-even}
Let $R$ be a ring. Let $(A, \gamma)$ be a strictly graded commutative
graded $R$-algebra endowed with a divided power structure as in the
definition above. Let $d > 0$ be an even integer.
In this setting we can adjoin a variable $T$ of degree $d$ to $A$.
Namely, set
$$
A\langle T \rangle = A \oplus AT \oplus AT^{(2)} \oplus AT^{(3)} \oplus \ldots
$$
with multiplication given by
$$
T^{(n)} T^{(m)} = \frac{(n + m)!}{n!m!} T^{(n + m)}
$$
and with grading given by
$$
A\langle T \rangle_m =
A_m \oplus A_{m - d}T \oplus A_{m - 2d}T^{(2)} \oplus \ldots
$$
We claim there is a unique divided power structure on
$A\langle T \rangle$ compatible with the given divided power
structure on $A$ such that $\gamma_n(T^{(i)}) = T^{(ni)}$.
To define the divided power structure we first set
$$
\gamma_n\left(\sum\nolimits_{i > 0} x_i T^{(i)}\right) =
\sum \prod\nolimits_{n = \sum e_i} x_i^{e_i} T^{(ie_i)}
$$
if $x_i$ is in $A_{even}$. If $x_0 \in A_{even, +}$
then we take
$$
\gamma_n\left(\sum\nolimits_{i \geq 0} x_i T^{(i)}\right) =
\sum\nolimits_{a + b = n}
\gamma_a(x_0)\gamma_b\left(\sum\nolimits_{i > 0} x_iT^{(i)}\right)
$$
where $\gamma_b$ is as defined above.
\end{example}

\begin{remark}
\label{remark-adjoining-set-of-variables}
We can also adjoin a set (possibly infinite) of exterior or divided
power generators in a given degree $d > 0$, rather than just one
as in Examples \ref{example-adjoining-odd}
and \ref{example-adjoining-even}. Namely, 
following Remark \ref{remark-divided-power-polynomial-algebra}:
for $(A,\gamma)$
as above and a set $J$, let $A\langle
T_j:j\in J\rangle$ be the directed colimit of the algebras
$A\langle T_j:j\in S\rangle$ over all finite subsets $S$
of $J$. It is immediate that this algebra has a unique divided power
structure, compatible with the given structure on $A$ and on
each generator $T_j$.
\end{remark}

\noindent
At this point we tie in the definition of divided power structures
with differentials. To understand the definition note that
$\text{d}(x^n/n!) = \text{d}(x) x^{n - 1}/(n - 1)!$ if $A$
is a $\mathbf{Q}$-algebra and $x \in A_{even, +}$.

\begin{definition}
\label{definition-divided-powers-dga}
Let $R$ be a ring. Let $A = \bigoplus_{d \geq 0} A_d$ be a
differential graded $R$-algebra which is strictly graded commutative.
A divided power structure $\gamma$ on $A$ is {\it compatible with
the differential graded structure} if
$\text{d}(\gamma_n(x)) = \text{d}(x) \gamma_{n - 1}(x)$ for
all $x \in A_{even, +}$.
\end{definition}

\noindent
Warning: Let $(A, \text{d}, \gamma)$ be as in
Definition \ref{definition-divided-powers-dga}.
It may not be true that $\gamma_n(x)$ is a boundary, if
$x$ is a boundary. Thus $\gamma$ in general does not induce
a divided power structure on the homology algebra $H(A)$.
In some papers the authors put an additional compatibility
condition in order to ensure that this is the case, but we elect
not to do so.

\begin{lemma}
\label{lemma-dpdga-good}
Let $(A, \text{d}, \gamma)$ and $(B, \text{d}, \gamma)$ be as in
Definition \ref{definition-divided-powers-dga}. Let $f : A \to B$
be a map of differential graded algebras compatible with divided
power structures. Assume
\begin{enumerate}
\item $H_k(A) = 0$ for $k > 0$, and
\item $f$ is surjective.
\end{enumerate}
Then $\gamma$ induces a divided power structure on the graded
$R$-algebra $H(B)$.
\end{lemma}

\begin{proof}
Suppose that $x$ and $x'$ are homogeneous of the same degree $2d$
and define the same cohomology class in $H(B)$. Say $x' - x = \text{d}(w)$.
Choose a lift $y \in A_{2d}$ of $x$ and a lift $z \in A_{2d + 1}$
of $w$. Then $y' = y + \text{d}(z)$ is a lift of $x'$.
Hence
$$
\gamma_n(y') = \sum \gamma_i(y) \gamma_{n - i}(\text{d}(z))
= \gamma_n(y) +
\sum\nolimits_{i < n} \gamma_i(y) \gamma_{n - i}(\text{d}(z))
$$
Since $A$ is acyclic in positive degrees and since
$\text{d}(\gamma_j(\text{d}(z))) = 0$ for all $j$ we can write
this as
$$
\gamma_n(y') = \gamma_n(y) +
\sum\nolimits_{i < n} \gamma_i(y) \text{d}(z_i)
$$
for some $z_i$ in $A$. Moreover, for $0 < i < n$ we have
$$
\text{d}(\gamma_i(y) z_i) =
\text{d}(\gamma_i(y))z_i + \gamma_i(y)\text{d}(z_i) =
\text{d}(y) \gamma_{i - 1}(y) z_i + \gamma_i(y)\text{d}(z_i)
$$
and the first term maps to zero in $B$ as $\text{d}(y)$ maps to zero in $B$.
Hence $\gamma_n(x')$ and $\gamma_n(x)$ map to the same element of $H(B)$.
Thus we obtain a well defined map $\gamma_n : H_{2d}(B) \to H_{2nd}(B)$
for all $d > 0$ and $n > 0$. We omit the verification that this
defines a divided power structure on $H(B)$.
\end{proof}

\begin{lemma}
\label{lemma-base-change-div}
Let $(A, \text{d}, \gamma)$ be as in
Definition \ref{definition-divided-powers-dga}.
Let $R \to R'$ be a ring map.
Then $\text{d}$ and $\gamma$ induce similar structures on
$A' = A \otimes_R R'$ such that $(A', \text{d}, \gamma)$ is as in
Definition \ref{definition-divided-powers-dga}.
\end{lemma}

\begin{proof}
Observe that $A'_{even} = A_{even} \otimes_R R'$ and
$A'_{even, +} = A_{even, +} \otimes_R R'$. Hence we are trying to
show that the divided powers $\gamma$ extend to $A'_{even}$
(terminology as in Definition \ref{definition-extends}).
Once we have shown $\gamma$ extends it follows easily that this
extension has all the desired properties.

\medskip\noindent
Choose a polynomial $R$-algebra $P$ (on any set of generators)
and a surjection of $R$-algebras
$P \to R'$. The ring map $A_{even} \to A_{even} \otimes_R P$ is flat,
hence the divided powers $\gamma$ extend to $A_{even} \otimes_R P$
uniquely by Lemma \ref{lemma-gamma-extends}.
Let $J = \Ker(P \to R')$. To show that $\gamma$ extends
to $A \otimes_R R'$ it suffices to show that
$I' = \Ker(A_{even, +} \otimes_R P \to A_{even, +} \otimes_R R')$
is generated by elements $z$ such that $\gamma_n(z) \in I'$
for all $n > 0$. This is clear as $I'$ is generated by elements
of the form $x \otimes f$ with
$x \in A_{even, +}$ and $f \in \Ker(P \to R')$.
\end{proof}

\begin{lemma}
\label{lemma-extend-differential}
Let $(A, \text{d}, \gamma)$ be as in
Definition \ref{definition-divided-powers-dga}.
Let $d \geq 1$ be an integer.
Let $A\langle T \rangle$ be the graded divided power polynomial algebra
on $T$ with $\deg(T) = d$
constructed in Example \ref{example-adjoining-odd} or
\ref{example-adjoining-even}.
Let $f \in A_{d - 1}$ be an element with $\text{d}(f) = 0$.
There exists a unique differential $\text{d}$
on $A\langle T\rangle$ such that $\text{d}(T) = f$ and
such that $\text{d}$ is compatible with the divided power
structure on $A\langle T \rangle$.
\end{lemma}

\begin{proof}
This is proved by a direct computation which is omitted.
\end{proof}

\noindent
In Lemma \ref{lemma-compute-cohomology-adjoin-variable}
we will compute the cohomology of $A\langle T \rangle$ in some special cases.
Here is Tate's construction, as extended
by Avramov and Halperin.

\begin{lemma}
\label{lemma-tate-resolution}
Let $R \to S$ be a homomorphism of commutative rings.
There exists a factorization
$$
R \to A \to S
$$
with the following properties:
\begin{enumerate}
\item $(A, \text{d}, \gamma)$ is as in
Definition \ref{definition-divided-powers-dga},
\item $A \to S$ is a quasi-isomorphism (if we endow $S$ with
the zero differential),
\item $A_0 = R[x_j: j\in J] \to S$ is any surjection of a polynomial
ring onto $S$, and
\item $A$ is a graded divided power polynomial algebra over $R$.
\end{enumerate}
The last condition means that $A$ is constructed out of $A_0$ by
successively adjoining a set of variables $T$ in each degree $> 0$ as in
Example \ref{example-adjoining-odd} or \ref{example-adjoining-even}.
Moreover, if $R$ is Noetherian and $R\to S$ is of finite type,
then $A$ can be taken to have only finitely many generators in
each degree.
\end{lemma}

\begin{proof}
We write out the construction for the case that $R$ is Noetherian
and $R\to S$ is of finite type. Without those assumptions, the proof
is the same, except that we have to use some set (possibly
infinite) of generators in each degree.

\medskip\noindent
Start of the construction: Let $A(0) = R[x_1, \ldots, x_n]$ be
a (usual) polynomial ring and let $A(0) \to S$ be a surjection.
As grading we take $A(0)_0 = A(0)$ and $A(0)_d = 0$ for $d \not = 0$.
Thus $\text{d} = 0$ and $\gamma_n$, $n > 0$, is zero as well.

\medskip\noindent
Choose generators $f_1, \ldots, f_m \in R[x_1, \ldots, x_n]$
for the kernel of the given map $A(0) = R[x_1, \ldots, x_n] \to S$.
We apply Example \ref{example-adjoining-odd} $m$ times to get
$$
A(1) = A(0)\langle T_1, \ldots, T_m\rangle
$$
with $\deg(T_i) = 1$ as a graded divided power polynomial algebra.
We set $\text{d}(T_i) = f_i$. Since $A(1)$ is a divided power polynomial
algebra over $A(0)$ and since $\text{d}(f_i) = 0$
this extends uniquely to a differential on $A(1)$ by
Lemma \ref{lemma-extend-differential}.

\medskip\noindent
Induction hypothesis: Assume we are given factorizations
$$
R \to A(0) \to A(1) \to \ldots \to A(m) \to S
$$
where $A(0)$ and $A(1)$ are as above and each $R \to A(m') \to S$
for $2 \leq m' \leq m$ satisfies properties (1) and (4)
of the statement of the lemma and (2) replaced by the condition that
$H_i(A(m')) \to H_i(S)$ is an isomorphism for
$m' > i \geq 0$. The base case is $m = 1$.

\medskip\noindent
Induction step: Assume we have $R \to A(m) \to S$
as in the induction hypothesis. Consider the
group $H_m(A(m))$. This is a module over $H_0(A(m)) = S$.
In fact, it is a subquotient of $A(m)_m$ which is a finite
type module over $A(m)_0 = R[x_1, \ldots, x_n]$.
Thus we can pick finitely many elements
$$
e_1, \ldots, e_t \in \Ker(\text{d} : A(m)_m \to A(m)_{m - 1})
$$
which map to generators of this module. Applying
Example \ref{example-adjoining-odd} or
\ref{example-adjoining-even} $t$ times we get
$$
A(m + 1) = A(m)\langle T_1, \ldots, T_t\rangle
$$
with $\deg(T_i) = m + 1$ as a graded divided power algebra. We set
$\text{d}(T_i) = e_i$. Since $A(m+1)$ is a divided power polynomial
algebra over $A(m)$ and since $\text{d}(e_i) = 0$
this extends uniquely to a differential on $A(m + 1)$
compatible with the divided power structure.
Since we've added only material in degree $m + 1$ and higher we see
that $H_i(A(m + 1)) = H_i(A(m))$ for $i < m$. Moreover, it is
clear that $H_m(A(m + 1)) = 0$ by construction.

\medskip\noindent
To finish the proof we observe that we have shown there exists
a sequence of maps
$$
R \to A(0) \to A(1) \to \ldots \to A(m) \to A(m + 1) \to \ldots \to S
$$
and to finish the proof we set $A = \colim A(m)$.
\end{proof}

\begin{lemma}
\label{lemma-tate-resoluton-pseudo-coherent-ring-map}
Let $R \to S$ be a pseudo-coherent ring map (More on Algebra, Definition
\ref{more-algebra-definition-pseudo-coherent-perfect}). Then
Lemma \ref{lemma-tate-resolution} holds, with the resolution $A$ of $S$
having finitely many generators in each degree.
\end{lemma}

\begin{proof}
This is proved in exactly the same way as Lemma \ref{lemma-tate-resolution}.
The only additional twist is that, given $A(m) \to S$ we have to
show that $H_m = H_m(A(m))$ is a finite $R[x_1, \ldots, x_m]$-module
(so that in the next step we need only add finitely many variables).
Consider the complex
$$
\ldots \to A(m)_{m - 1} \to A(m)_m \to A(m)_{m - 1} \to
\ldots \to A(m)_0 \to S \to 0
$$
Since $S$ is a pseudo-coherent $R[x_1, \ldots, x_n]$-module
and since $A(m)_i$ is a finite free $R[x_1, \ldots, x_n]$-module
we conclude that this is a pseudo-coherent complex, see
More on Algebra, Lemma \ref{more-algebra-lemma-complex-pseudo-coherent-modules}.
Since the complex is exact in (homological) degrees $> m$
we conclude that $H_m$ is a finite $R$-module by
More on Algebra, Lemma \ref{more-algebra-lemma-finite-cohomology}.
\end{proof}

\begin{lemma}
\label{lemma-uniqueness-tate-resolution}
Let $R$ be a commutative ring. Suppose that $(A, \text{d}, \gamma)$ and
$(B, \text{d}, \gamma)$ are as in
Definition \ref{definition-divided-powers-dga}.
Let $\overline{\varphi} : H_0(A) \to H_0(B)$ be an $R$-algebra map.
Assume
\begin{enumerate}
\item $A$ is a graded divided power polynomial algebra over $R$.
\item $H_k(B) = 0$ for $k > 0$.
\end{enumerate}
Then there exists a map $\varphi : A \to B$ of differential
graded $R$-algebras compatible with divided powers
that lifts $\overline{\varphi}$.
\end{lemma}

\begin{proof}
The assumption means that $A$ is obtained from $R$ by successively adjoining
some set of polynomial generators in degree zero, exterior generators
in positive odd degrees, and divided power generators
in positive even degrees. So we have a filtration
$R \subset A(0) \subset A(1) \subset \ldots$
of $A$ such that $A(m + 1)$ is obtained from $A(m)$ by adjoining
generators of the appropriate type (which we simply call
``divided power generators'') in degree $m + 1$.
In particular, $A(0) \to H_0(A)$ is a surjection from a (usual) polynomial
algebra over $R$ onto $H_0(A)$. Thus we can lift $\overline{\varphi}$
to an $R$-algebra map $\varphi(0) : A(0) \to B_0$.

\medskip\noindent
Write $A(1) = A(0)\langle T_j:j\in J\rangle$ for some
set $J$ of divided power variables $T_j$ of degree $1$. Let $f_j \in B_0$
be $f_j = \varphi(0)(\text{d}(T_j))$. Observe that $f_j$
maps to zero in $H_0(B)$ as $\text{d}T_j$ maps to zero in $H_0(A)$.
Thus we can find $b_j \in B_1$ with $\text{d}(b_j) = f_j$.
By the universal property of divided power polynomial algebras from
Lemma \ref{lemma-divided-power-polynomial-algebra},
we find a lift $\varphi(1) : A(1) \to B$ of $\varphi(0)$
mapping $T_j$ to $f_j$.

\medskip\noindent
Having constructed $\varphi(m)$ for some $m \geq 1$ we can construct
$\varphi(m + 1) : A(m + 1) \to B$ in exactly the same manner.
We omit the details.
\end{proof}

\begin{lemma}
\label{lemma-divided-powers-on-tor}
Let $R$ be a commutative ring. Let $S$ and $T$ be commutative $R$-algebras.
Then there is a canonical structure
of a strictly graded commutative $R$-algebra with divided powers on
$$
\operatorname{Tor}_*^R(S, T).
$$
\end{lemma}

\begin{proof}
Choose a factorization $R \to A \to S$ as above. Since $A \to S$
is a quasi-isomorphism and since $A_d$ is a free $R$-module,
we see that the differential graded algebra $B = A \otimes_R T$ computes
the Tor groups displayed in the lemma. Choose a surjection
$R[y_j:j\in J] \to T$. Then we see that
$B$ is a quotient of the differential graded algebra
$A[y_j:j\in J]$ whose homology sits in degree $0$ (it is equal
to $S[y_j:j\in J]$).
By Lemma \ref{lemma-base-change-div} the differential graded algebras $B$ and
$A[y_j:j\in J]$ have divided power structures compatible
with the differentials. Hence we obtain our divided
power structure on $H(B)$ by Lemma \ref{lemma-dpdga-good}.

\medskip\noindent
The divided power algebra structure constructed in this way is independent
of the choice of $A$. Namely, if $A'$ is a second choice, then
Lemma \ref{lemma-uniqueness-tate-resolution}
implies there is a map $A \to A'$ preserving all structure and the
augmentations towards $S$. Then the induced map
$B = A \otimes_R T \to A' \otimes_R T' = B'$ also preserves
all structure
and is a quasi-isomorphism. The induced isomorphism of
Tor algebras is therefore compatible with products
and divided powers.
\end{proof}





\section{Application to complete intersections}
\label{section-application-ci}

\noindent
Let $R$ be a ring. Let $(A, \text{d}, \gamma)$ be as in
Definition \ref{definition-divided-powers-dga}.
A {\it derivation} of degree $2$ is an $R$-linear
map $\theta : A \to A$ with the following
properties
\begin{enumerate}
\item $\theta(A_d) \subset A_{d - 2}$,
\item $\theta(xy) = \theta(x)y + x\theta(y)$,
\item $\theta$ commutes with $\text{d}$,
\item $\theta(\gamma_n(x)) = \theta(x) \gamma_{n - 1}(x)$
for all $x \in A_{2d}$ all $d$.
\end{enumerate}
In the following lemma we construct a derivation.

\begin{lemma}
\label{lemma-get-derivation}
Let $R$ be a ring. Let $(A, \text{d}, \gamma)$ be as in
Definition \ref{definition-divided-powers-dga}.
Let $R' \to R$ be a surjection of rings whose kernel
has square zero and is generated by one element $f$.
If $A$ is a graded divided power polynomial algebra over $R$
with finitely many variables in each degree,
then we obtain a derivation
$\theta : A/IA \to A/IA$ where $I$ is the annihilator
of $f$ in $R$.
\end{lemma}

\begin{proof}
Since $A$ is a divided power polynomial algebra, we can find a divided
power polynomial algebra $A'$ over $R'$ such that $A = A' \otimes_{R'} R$.
Moreover, we can lift $\text{d}$ to an $R$-linear
operator $\text{d}$ on $A'$ such that
\begin{enumerate}
\item  $\text{d}(xy) = \text{d}(x)y + (-1)^{\deg(x)}x \text{d}(y)$
for $x, y \in A'$ homogeneous, and
\item  $\text{d}(\gamma_n(x)) = \text{d}(x) \gamma_{n - 1}(x)$ for
$x \in A'_{even, +}$.
\end{enumerate}
We omit the details (hint: proceed one variable at the time).
However, it may not be the case that $\text{d}^2$
is zero on $A'$. It is clear that $\text{d}^2$ maps $A'$ into
$fA' \cong A/IA$. Hence $\text{d}^2$ annihilates $fA'$ and factors
as a map $A \to A/IA$. Since $\text{d}^2$ is $R$-linear we obtain
our map $\theta : A/IA \to A/IA$. The verification of the properties
of a derivation is immediate.
\end{proof}

\begin{lemma}
\label{lemma-compute-theta}
Assumption and notation as in Lemma \ref{lemma-get-derivation}.
Suppose $S = H_0(A)$ is isomorphic to
$R[x_1, \ldots, x_n]/(f_1, \ldots, f_m)$
for some $n$, $m$, and $f_j \in R[x_1, \ldots, x_n]$.
Moreover, suppose given a relation
$$
\sum r_j f_j = 0
$$
with $r_j \in R[x_1, \ldots, x_n]$.
Choose $r'_j, f'_j \in R'[x_1, \ldots, x_n]$ lifting $r_j, f_j$.
Write $\sum r'_j f'_j = gf$ for some $g \in R/I[x_1, \ldots, x_n]$.
If $H_1(A) = 0$ and all the coefficients of each $r_j$ are in $I$, then
there exists an element $\xi \in H_2(A/IA)$ such that
$\theta(\xi) = g$ in $S/IS$.
\end{lemma}

\begin{proof}
Let $A(0) \subset A(1) \subset A(2) \subset \ldots$ be the filtration
of $A$ such that $A(m)$ is gotten from $A(m - 1)$ by adjoining divided
power variables of degree $m$. Then $A(0)$ is a polynomial algebra
over $R$ equipped with an $R$-algebra surjection $A(0) \to S$.
Thus we can choose a map
$$
\varphi : R[x_1, \ldots, x_n] \to A(0)
$$
lifting the augmentations to $S$.
Next, $A(1) = A(0)\langle T_1, \ldots, T_t \rangle$ for some divided
power variables $T_i$ of degree $1$. Since $H_0(A) = S$ we
can pick $\xi_j \in \sum A(0)T_i$ with $\text{d}(\xi_j) = \varphi(f_j)$.
Then
$$
\text{d}\left(\sum \varphi(r_j) \xi_j\right) =
\sum  \varphi(r_j) \varphi(f_j) =  \sum \varphi(r_jf_j) = 0
$$
Since $H_1(A) = 0$ we can pick $\xi \in A_2$ with
$\text{d}(\xi) = \sum \varphi(r_j) \xi_j$.
If the coefficients of $r_j$ are in $I$, then the same
is true for $\varphi(r_j)$. In this case
$\text{d}(\xi)$ dies in $A_1/IA_1$ and
hence $\xi$ defines a class in $H_2(A/IA)$.

\medskip\noindent
The construction of $\theta$ in the proof of Lemma \ref{lemma-get-derivation}
proceeds by successively lifting $A(i)$ to $A'(i)$ and lifting the
differential $\text{d}$. We lift $\varphi$
to $\varphi' : R'[x_1, \ldots, x_n] \to A'(0)$.
Next, we have $A'(1) = A'(0)\langle T_1, \ldots, T_t\rangle$.
Moreover, we can lift $\xi_j$ to $\xi'_j \in \sum A'(0)T_i$.
Then $\text{d}(\xi'_j) = \varphi'(f'_j) + f a_j$ for some
$a_j \in A'(0)$.
Consider a lift $\xi' \in A'_2$ of $\xi$.
Then we know that
$$
\text{d}(\xi') = \sum \varphi'(r'_j)\xi'_j + \sum fb_iT_i
$$
for some $b_i \in A(0)$. Applying $\text{d}$ again we find
$$
\theta(\xi) = \sum \varphi'(r'_j)\varphi'(f'_j) +
\sum f \varphi'(r'_j) a_j + \sum fb_i \text{d}(T_i)
$$
The first term gives us what we want. The second term is zero
because the coefficients of $r_j$ are in $I$ and hence are
annihilated by $f$. The third term maps to zero in $H_0$
because $\text{d}(T_i)$ maps to zero.
\end{proof}

\noindent
The method of proof of the following lemma is apparently due to Gulliksen.

\begin{lemma}
\label{lemma-not-finite-pd}
Let $R' \to R$ be a surjection of Noetherian rings whose kernel has square
zero and is generated by one element $f$. Let
$S = R[x_1, \ldots, x_n]/(f_1, \ldots, f_m)$.
Let $\sum r_j f_j = 0$ be a relation in $R[x_1, \ldots, x_n]$.
Assume that
\begin{enumerate}
\item each $r_j$ has coefficients in the annihilator $I$ of $f$ in $R$,
\item for some lifts $r'_j, f'_j \in R'[x_1, \ldots, x_n]$ we have
$\sum r'_j f'_j = gf$ where $g$ is not nilpotent in $S/IS$.
\end{enumerate}
Then $S$ does not have finite tor dimension over $R$ (i.e., $S$ is not
a perfect $R$-algebra).
\end{lemma}

\begin{proof}
Choose a Tate resolution $R \to A \to S$ as in
Lemma \ref{lemma-tate-resolution}.
Let $\xi \in H_2(A/IA)$ and $\theta : A/IA \to A/IA$ be the element
and derivation found in Lemmas \ref{lemma-get-derivation} and
\ref{lemma-compute-theta}.
Observe that
$$
\theta^n(\gamma_n(\xi)) = g^n
$$
in $H_0(A/IA) = S/IS$.
Hence if $g$ is not nilpotent in $S/IS$, then $\xi^n$ is nonzero in
$H_{2n}(A/IA)$ for all $n > 0$. Since
$H_{2n}(A/IA) = \text{Tor}^R_{2n}(S, R/I)$ we conclude.
\end{proof}

\noindent
The following result can be found in \cite{Rodicio}.

\begin{lemma}
\label{lemma-injective}
Let $(A, \mathfrak m)$ be a Noetherian local ring. Let
$I \subset J \subset A$ be proper ideals. If $A/J$ has finite
tor dimension over $A/I$, then $I/\mathfrak m I \to J/\mathfrak m J$
is injective.
\end{lemma}

\begin{proof}
Let $f \in I$ be an element mapping to a nonzero element of $I/\mathfrak m I$
which is mapped to zero in $J/\mathfrak mJ$. We can choose an ideal $I'$
with $\mathfrak mI \subset I' \subset I$ such that $I/I'$ is generated by
the image of $f$. Set $R = A/I$ and $R' = A/I'$. Let $J = (a_1, \ldots, a_m)$
for some $a_j \in A$. Then $f = \sum b_j a_j$ for some $b_j \in \mathfrak m$.
Let $r_j, f_j \in R$ resp.\ $r'_j, f'_j \in R'$ be the image of $b_j, a_j$.
Then we see we are
in the situation of Lemma \ref{lemma-not-finite-pd}
(with the ideal $I$ of that lemma equal to $\mathfrak m_R$)
and the lemma is proved.
\end{proof}

\begin{lemma}
\label{lemma-regular-sequence}
Let $(A, \mathfrak m)$ be a Noetherian local ring. Let
$I \subset J \subset A$ be proper ideals. Assume
\begin{enumerate}
\item $A/J$ has finite tor dimension over $A/I$, and
\item $J$ is generated by a regular sequence.
\end{enumerate}
Then $I$ is generated by a regular sequence and $J/I$
is generated by a regular sequence.
\end{lemma}

\begin{proof}
By Lemma \ref{lemma-injective} we see that
$I/\mathfrak m I \to J/\mathfrak m J$
is injective. Thus we can find $s \leq r$ and a minimal system of
generators $f_1, \ldots, f_r$ of $J$ such that $f_1, \ldots, f_s$ are in $I$
and form a minimal system of generators of $I$.
The lemma follows as any minimal system of generators of $J$
is a regular sequence by
More on Algebra, Lemmas
\ref{more-algebra-lemma-independence-of-generators} and
\ref{more-algebra-lemma-noetherian-finite-all-equivalent}.
\end{proof}

\begin{lemma}
\label{lemma-perfect-map-ci}
Let $R \to S$ be a local ring map of Noetherian local rings.
Let $I \subset R$ and $J \subset S$ be ideals with
$IS \subset J$. If $R \to S$ is flat and $S/\mathfrak m_RS$ is
regular, then the following are equivalent
\begin{enumerate}
\item $J$ is generated by a regular sequence and
$S/J$ has finite tor dimension as a module over $R/I$,
\item $J$ is generated by a regular sequence and
$\text{Tor}^{R/I}_p(S/J, R/\mathfrak m_R)$ is nonzero
for only finitely many $p$,
\item $I$ is generated by a regular sequence
and $J/IS$ is generated by a regular sequence in $S/IS$.
\end{enumerate}
\end{lemma}

\begin{proof}
If (3) holds, then $J$ is generated by a regular sequence, see for example
More on Algebra, Lemmas
\ref{more-algebra-lemma-join-koszul-regular-sequences} and
\ref{more-algebra-lemma-noetherian-finite-all-equivalent}.
Moreover, if (3) holds, then $S/J = (S/I)/(J/I)$
has finite projective dimension over $S/IS$ because the Koszul
complex will be a finite free resolution of $S/J$ over $S/IS$.
Since $R/I \to S/IS$ is flat, it then follows that $S/J$ has finite
tor dimension over $R/I$ by
More on Algebra, Lemma \ref{more-algebra-lemma-flat-push-tor-amplitude}.
Thus (3) implies (1).

\medskip\noindent
The implication (1) $\Rightarrow$ (2) is trivial.
Assume (2). By
More on Algebra, Lemma \ref{more-algebra-lemma-perfect-over-regular-local-ring}
we find that $S/J$ has finite tor dimension over $S/IS$.
Thus we can apply Lemma \ref{lemma-regular-sequence}
to conclude that $IS$ and $J/IS$ are generated by regular sequences.
Let $f_1, \ldots, f_r \in I$ be a minimal system of generators of $I$.
Since $R \to S$ is flat, we see that $f_1, \ldots, f_r$ form a minimal
system of generators for $IS$ in $S$. Thus $f_1, \ldots, f_r \in R$
is a sequence of elements whose images in $S$ form a regular sequence
by More on Algebra, Lemmas
\ref{more-algebra-lemma-independence-of-generators} and
\ref{more-algebra-lemma-noetherian-finite-all-equivalent}.
Thus $f_1, \ldots, f_r$ is a regular sequence in $R$ by
Algebra, Lemma \ref{algebra-lemma-flat-increases-depth}.
\end{proof}





\section{Local complete intersection rings}
\label{section-lci}

\noindent
Let $(A, \mathfrak m)$ be a Noetherian complete local ring.
By the Cohen structure theorem (see
Algebra, Theorem \ref{algebra-theorem-cohen-structure-theorem})
we can write $A$ as the quotient of a regular Noetherian
complete local ring $R$. Let us say that $A$ is a
{\it complete intersection}
if there exists some surjection $R \to A$
with $R$ a regular local ring such that the kernel
is generated by a regular sequence.
The following lemma shows this notion is independent of
the choice of the surjection.

\begin{lemma}
\label{lemma-ci-well-defined}
Let $(A, \mathfrak m)$ be a Noetherian complete local ring.
The following are equivalent
\begin{enumerate}
\item for every surjection of local rings $R \to A$ with $R$
a regular local ring, the kernel of $R \to A$ is generated
by a regular sequence, and
\item for some surjection of local rings $R \to A$ with $R$
a regular local ring, the kernel of $R \to A$ is generated
by a regular sequence.
\end{enumerate}
\end{lemma}

\begin{proof}
Let $k$ be the residue field of $A$. If the characteristic of
$k$ is $p > 0$, then we denote $\Lambda$ a Cohen ring
(Algebra, Definition \ref{algebra-definition-cohen-ring})
with residue field $k$ (Algebra, Lemma \ref{algebra-lemma-cohen-rings-exist}).
If the characteristic of $k$ is $0$ we set $\Lambda = k$.
Recall that $\Lambda[[x_1, \ldots, x_n]]$ for any $n$
is formally smooth over $\mathbf{Z}$, resp.\ $\mathbf{Q}$
in the $\mathfrak m$-adic topology, see
More on Algebra, Lemma
\ref{more-algebra-lemma-power-series-ring-over-Cohen-fs}.
Fix a surjection $\Lambda[[x_1, \ldots, x_n]] \to A$ as in
the Cohen structure theorem
(Algebra, Theorem \ref{algebra-theorem-cohen-structure-theorem}).

\medskip\noindent
Let $R \to A$ be a surjection from a regular local ring $R$.
Let $f_1, \ldots, f_r$ be a minimal sequence of generators
of $\Ker(R \to A)$. We will use without further mention
that an ideal in a Noetherian local ring is generated by a regular
sequence if and only if any minimal set of generators is a
regular sequence. Observe that $f_1, \ldots, f_r$
is a regular sequence in $R$ if and only if $f_1, \ldots, f_r$
is a regular sequence in the completion $R^\wedge$ by
Algebra, Lemmas \ref{algebra-lemma-flat-increases-depth} and
\ref{algebra-lemma-completion-flat}.
Moreover, we have
$$
R^\wedge/(f_1, \ldots, f_r)R^\wedge =
(R/(f_1, \ldots, f_n))^\wedge = A^\wedge = A
$$
because $A$ is $\mathfrak m_A$-adically complete (first equality by
Algebra, Lemma \ref{algebra-lemma-completion-tensor}). Finally,
the ring $R^\wedge$ is regular since $R$ is regular
(More on Algebra, Lemma \ref{more-algebra-lemma-completion-regular}).
Hence we may assume $R$ is complete.

\medskip\noindent
If $R$ is complete we can choose a map
$\Lambda[[x_1, \ldots, x_n]] \to R$ lifting the given map
$\Lambda[[x_1, \ldots, x_n]] \to A$, see
More on Algebra, Lemma \ref{more-algebra-lemma-lift-continuous}.
By adding some more variables $y_1, \ldots, y_m$ mapping
to generators of the kernel of $R \to A$ we may assume that
$\Lambda[[x_1, \ldots, x_n, y_1, \ldots, y_m]] \to R$ is surjective
(some details omitted). Then we can consider the commutative diagram
$$
\xymatrix{
\Lambda[[x_1, \ldots, x_n, y_1, \ldots, y_m]] \ar[r] \ar[d] & R \ar[d] \\
\Lambda[[x_1, \ldots, x_n]] \ar[r] & A
}
$$
By Algebra, Lemma \ref{algebra-lemma-ci-well-defined} we see that
the condition for $R \to A$ is equivalent to the condition for
the fixed chosen map
$\Lambda[[x_1, \ldots, x_n]] \to A$. This finishes the proof of the lemma.
\end{proof}

\noindent
The following two lemmas are sanity checks on the definition given above.

\begin{lemma}
\label{lemma-quotient-regular-ring-by-regular-sequence}
Let $R$ be a regular ring. Let $\mathfrak p \subset R$ be a prime.
Let $f_1, \ldots, f_r \in \mathfrak p$ be a regular sequence.
Then the completion of
$$
A = (R/(f_1, \ldots, f_r))_\mathfrak p =
R_\mathfrak p/(f_1, \ldots, f_r)R_\mathfrak p
$$
is a complete intersection in the sense defined above.
\end{lemma}

\begin{proof}
The completion of $A$ is equal to
$A^\wedge = R_\mathfrak p^\wedge/(f_1, \ldots, f_r)R_\mathfrak p^\wedge$
because completion for finite modules over the Noetherian ring
$R_\mathfrak p$ is exact
(Algebra, Lemma \ref{algebra-lemma-completion-tensor}).
The image of the sequence $f_1, \ldots, f_r$ in $R_\mathfrak p$
is a regular sequence by
Algebra, Lemmas \ref{algebra-lemma-completion-flat} and
\ref{algebra-lemma-flat-increases-depth}.
Moreover, $R_\mathfrak p^\wedge$ is a regular local ring by
More on Algebra, Lemma \ref{more-algebra-lemma-completion-regular}.
Hence the result holds by our definition of complete
intersection for complete local rings.
\end{proof}

\noindent
The following lemma is the analogue of Algebra, Lemma \ref{algebra-lemma-lci}.

\begin{lemma}
\label{lemma-quotient-regular-ring}
Let $R$ be a regular ring. Let $\mathfrak p \subset R$ be a prime.
Let $I \subset \mathfrak p$ be an ideal.
Set $A = (R/I)_\mathfrak p = R_\mathfrak p/I_\mathfrak p$.
The following are equivalent
\begin{enumerate}
\item the completion of $A$
is a complete intersection in the sense above,
\item $I_\mathfrak p \subset R_\mathfrak p$ is generated
by a regular sequence,
\item the module $(I/I^2)_\mathfrak p$ can be generated by
$\dim(R_\mathfrak p) - \dim(A)$ elements,
\item add more here.
\end{enumerate}
\end{lemma}

\begin{proof}
We may and do replace $R$ by its localization at $\mathfrak p$.
Then $\mathfrak p = \mathfrak m$ is the maximal ideal of $R$
and $A = R/I$. Let $f_1, \ldots, f_r \in I$ be a minimal sequence
of generators.  The completion of $A$ is equal to
$A^\wedge = R^\wedge/(f_1, \ldots, f_r)R^\wedge$
because completion for finite modules over the Noetherian ring
$R_\mathfrak p$ is exact
(Algebra, Lemma \ref{algebra-lemma-completion-tensor}).

\medskip\noindent
If (1) holds, then the image of the sequence $f_1, \ldots, f_r$ in $R^\wedge$
is a regular sequence by assumption. Hence it is a regular sequence
in $R$ by Algebra, Lemmas \ref{algebra-lemma-completion-flat} and
\ref{algebra-lemma-flat-increases-depth}. Thus (1) implies (2).

\medskip\noindent
Assume (3) holds. Set $c = \dim(R) - \dim(A)$ and let $f_1, \ldots, f_c \in I$
map to generators of $I/I^2$. by Nakayama's lemma
(Algebra, Lemma \ref{algebra-lemma-NAK})
we see that $I = (f_1, \ldots, f_c)$. Since $R$ is regular and hence
Cohen-Macaulay (Algebra, Proposition \ref{algebra-proposition-CM-module})
we see that $f_1, \ldots, f_c$ is a regular sequence by
Algebra, Proposition \ref{algebra-proposition-CM-module}.
Thus (3) implies (2).
Finally, (2) implies (1) by
Lemma \ref{lemma-quotient-regular-ring-by-regular-sequence}.
\end{proof}

\noindent
The following result is due to Avramov, see \cite{Avramov}.

\begin{proposition}
\label{proposition-avramov}
Let $A \to B$ be a flat local homomorphism of Noetherian local rings.
Then the following are equivalent
\begin{enumerate}
\item $B^\wedge$ is a complete intersection,
\item $A^\wedge$ and $(B/\mathfrak m_A B)^\wedge$ are complete intersections.
\end{enumerate}
\end{proposition}

\begin{proof}
Consider the diagram
$$
\xymatrix{
B \ar[r] & B^\wedge \\
A \ar[u] \ar[r] & A^\wedge \ar[u]
}
$$
Since the horizontal maps are faithfully flat
(Algebra, Lemma \ref{algebra-lemma-completion-faithfully-flat})
we conclude that the right vertical arrow is flat
(for example by Algebra, Lemma
\ref{algebra-lemma-criterion-flatness-fibre-Noetherian}).
Moreover, we have
$(B/\mathfrak m_A B)^\wedge = B^\wedge/\mathfrak m_{A^\wedge} B^\wedge$
by Algebra, Lemma \ref{algebra-lemma-completion-tensor}.
Thus we may assume $A$ and $B$ are complete local Noetherian rings.

\medskip\noindent
Assume $A$ and $B$ are complete local Noetherian rings.
Choose a diagram
$$
\xymatrix{
S \ar[r] & B \\
R \ar[u] \ar[r] & A \ar[u]
}
$$
as in More on Algebra, Lemma
\ref{more-algebra-lemma-embed-map-Noetherian-complete-local-rings}.
Let $I = \Ker(R \to A)$ and $J = \Ker(S \to B)$.
Note that since $R/I = A \to B = S/J$ is flat the map
$J/IS \otimes_R R/\mathfrak m_R \to J/J \cap \mathfrak m_R S$
is an isomorphism. Hence a minimal system of generators of $J/IS$
maps to a minimal system of generators of
$\Ker(S/\mathfrak m_R S \to B/\mathfrak m_A B)$.
Finally, $S/\mathfrak m_R S$ is a regular local ring.

\medskip\noindent
Assume (1) holds, i.e., $J$ is generated by a regular sequence.
Since $A = R/I \to B = S/J$ is flat we see
Lemma \ref{lemma-perfect-map-ci} applies and we deduce
that $I$ and $J/IS$ are generated by regular sequences.
We have $\dim(B) = \dim(A) + \dim(B/\mathfrak m_A B)$ and
$\dim(S/IS) = \dim(A) + \dim(S/\mathfrak m_R S)$
(Algebra, Lemma \ref{algebra-lemma-dimension-base-fibre-equals-total}).
Thus $J/IS$ is generated by
$$
\dim(S/IS) - \dim(S/J) = \dim(S/\mathfrak m_R S) - \dim(B/\mathfrak m_A B)
$$
elements (Algebra, Lemma \ref{algebra-lemma-one-equation}).
It follows that $\Ker(S/\mathfrak m_R S \to B/\mathfrak m_A B)$
is generated by the same number of elements (see above).
Hence $\Ker(S/\mathfrak m_R S \to B/\mathfrak m_A B)$
is generated by a regular sequence, see for example
Lemma \ref{lemma-quotient-regular-ring}.
In this way we see that (2) holds.

\medskip\noindent
If (2) holds, then $I$ and $J/J \cap \mathfrak m_RS$
are generated by regular sequences. Lifting these generators
(see above), using flatness of $R/I \to S/IS$,
and using Grothendieck's lemma
(Algebra, Lemma \ref{algebra-lemma-grothendieck-regular-sequence})
we find that $J/IS$ is generated by a regular sequence in $S/IS$.
Thus Lemma \ref{lemma-perfect-map-ci} tells us that $J$
is generated by a regular sequence, whence (1) holds.
\end{proof}

\begin{definition}
\label{definition-lci}
Let $A$ be a Noetherian ring.
\begin{enumerate}
\item If $A$ is local, then we say $A$ is a {\it complete intersection}
if its completion is a complete intersection in the sense above.
\item In general we say $A$ is a {\it local complete intersection}
if all of its local rings are complete intersections.
\end{enumerate}
\end{definition}

\noindent
We will check below that this does not conflict with the terminology
introduced in
Algebra, Definitions \ref{algebra-definition-lci-field} and
\ref{algebra-definition-lci-local-ring}.
But first, we show this ``makes sense'' by showing
that if $A$ is a Noetherian
local complete intersection, then $A$ is a local complete intersection,
i.e., all of its local rings are complete intersections.

\begin{lemma}
\label{lemma-ci-good}
Let $(A, \mathfrak m)$ be a Noetherian local ring. Let
$\mathfrak p \subset A$ be a prime ideal. If $A$ is a complete
intersection, then $A_\mathfrak p$ is a complete intersection too.
\end{lemma}

\begin{proof}
Choose a prime $\mathfrak q$ of $A^\wedge$ lying over $\mathfrak p$
(this is possible as $A \to A^\wedge$ is faithfully flat by
Algebra, Lemma \ref{algebra-lemma-completion-faithfully-flat}).
Then $A_\mathfrak p \to (A^\wedge)_\mathfrak q$ is a flat local
ring homomorphism. Thus by Proposition \ref{proposition-avramov}
we see that $A_\mathfrak p$ is a complete intersection if and only if
$(A^\wedge)_\mathfrak q$ is a complete intersection. Thus it suffices
to prove the lemma in case $A$ is complete (this is the key step
of the proof).

\medskip\noindent
Assume $A$ is complete. By definition we may write
$A = R/(f_1, \ldots, f_r)$ for some regular sequence
$f_1, \ldots, f_r$ in a regular local ring $R$.
Let $\mathfrak q \subset R$ be the prime corresponding to $\mathfrak p$.
Observe that $f_1, \ldots, f_r \in \mathfrak q$ and that
$A_\mathfrak p = R_\mathfrak q/(f_1, \ldots, f_r)R_\mathfrak q$.
Hence $A_\mathfrak p$ is a complete intersection by
Lemma \ref{lemma-quotient-regular-ring-by-regular-sequence}.
\end{proof}

\begin{lemma}
\label{lemma-check-lci-at-maximal-ideals}
Let $A$ be a Noetherian ring. Then $A$ is a local complete intersection
if and only if $A_\mathfrak m$ is a complete intersection for every
maximal ideal $\mathfrak m$ of $A$.
\end{lemma}

\begin{proof}
This follows immediately from Lemma \ref{lemma-ci-good} and the definitions.
\end{proof}

\begin{lemma}
\label{lemma-check-lci-agrees}
Let $S$ be a finite type algebra over a field $k$.
\begin{enumerate}
\item for a prime $\mathfrak q \subset S$ the local ring $S_\mathfrak q$
is a complete intersection in the sense of
Algebra, Definition \ref{algebra-definition-lci-local-ring}
if and only if $S_\mathfrak q$ is a complete
intersection in the sense of Definition \ref{definition-lci}, and
\item $S$ is a local complete intersection in the sense of
Algebra, Definition \ref{algebra-definition-lci-field}
if and only if $S$ is a local complete
intersection in the sense of Definition \ref{definition-lci}.
\end{enumerate}
\end{lemma}

\begin{proof}
Proof of (1). Let $k[x_1, \ldots, x_n] \to S$ be a surjection.
Let $\mathfrak p \subset k[x_1, \ldots, x_n]$ be the prime ideal
corresponding to $\mathfrak q$.
Let $I \subset k[x_1, \ldots, x_n]$ be the kernel of our surjection.
Note that $k[x_1, \ldots, x_n]_\mathfrak p \to S_\mathfrak q$
is surjective with kernel $I_\mathfrak p$. Observe that
$k[x_1, \ldots, x_n]$ is a regular ring by
Algebra, Proposition \ref{algebra-proposition-finite-gl-dim-polynomial-ring}.
Hence the equivalence of the two notions in (1) follows by
combining
Lemma \ref{lemma-quotient-regular-ring}
with Algebra, Lemma \ref{algebra-lemma-lci-local}.

\medskip\noindent
Having proved (1) the equivalence in (2) follows from the
definition and Algebra, Lemma \ref{algebra-lemma-lci-global}.
\end{proof}

\begin{lemma}
\label{lemma-avramov}
Let $A \to B$ be a flat local homomorphism of Noetherian local rings.
Then the following are equivalent
\begin{enumerate}
\item $B$ is a complete intersection,
\item $A$ and $B/\mathfrak m_A B$ are complete intersections.
\end{enumerate}
\end{lemma}

\begin{proof}
Now that the definition makes sense this is a trivial reformulation
of the (nontrivial) Proposition \ref{proposition-avramov}.
\end{proof}









\section{Local complete intersection maps}
\label{section-lci-homomorphisms}

\noindent
Let $A \to B$ be a local homomorphism of Noetherian complete local rings.
A consequence of the Cohen structure theorem is that we can find a
commutative diagram
$$
\xymatrix{
S \ar[r] & B \\
& A \ar[lu] \ar[u]
}
$$
of Noetherian complete local rings with
$S \to B$ surjective, $A \to S$ flat, and
$S/\mathfrak m_A S$ a regular local ring. This follows from
More on Algebra, Lemma
\ref{more-algebra-lemma-embed-map-Noetherian-complete-local-rings}.
Let us (temporarily) say $A \to S \to B$ is a {\it good factorization}
of $A \to B$ if $S$ is a Noetherian local ring,
$A \to S \to B$ are local ring maps,
$S \to B$ surjective, $A \to S$ flat, and $S/\mathfrak m_AS$ regular.
Let us say that $A \to B$ is a
{\it complete intersection homomorphism}
if there exists some good factorization $A \to S \to B$
such that the kernel of $S \to B$ is generated by a regular sequence.
The following lemma shows this notion is independent of
the choice of the diagram.

\begin{lemma}
\label{lemma-ci-map-well-defined}
Let $A \to B$ be a local homomorphism of Noetherian complete local rings.
The following are equivalent
\begin{enumerate}
\item for some good factorization $A \to S \to B$ the kernel of
$S \to B$ is generated by a regular sequence, and
\item for every good factorization $A \to S \to B$ the kernel of
$S \to B$ is generated by a regular sequence.
\end{enumerate}
\end{lemma}

\begin{proof}
Let $A \to S \to B$ be a good factorization.
As $B$ is complete we obtain a factorization
$A \to S^\wedge \to B$ where $S^\wedge$ is the completion of $S$.
Note that this is also a good factorization:
The ring map $S \to S^\wedge$ is flat
(Algebra, Lemma \ref{algebra-lemma-completion-flat}),
hence $A \to S^\wedge$ is flat.
The ring $S^\wedge/\mathfrak m_A S^\wedge = (S/\mathfrak m_A S)^\wedge$
is regular since $S/\mathfrak m_A S$ is regular
(More on Algebra, Lemma \ref{more-algebra-lemma-completion-regular}).
Let $f_1, \ldots, f_r$ be a minimal sequence of generators
of $\Ker(S \to B)$. We will use without further mention
that an ideal in a Noetherian local ring is generated by a regular
sequence if and only if any minimal set of generators is a
regular sequence. Observe that $f_1, \ldots, f_r$
is a regular sequence in $S$ if and only if $f_1, \ldots, f_r$
is a regular sequence in the completion $S^\wedge$ by
Algebra, Lemma \ref{algebra-lemma-flat-increases-depth}.
Moreover, we have
$$
S^\wedge/(f_1, \ldots, f_r)R^\wedge =
(S/(f_1, \ldots, f_n))^\wedge = B^\wedge = B
$$
because $B$ is $\mathfrak m_B$-adically complete (first equality by
Algebra, Lemma \ref{algebra-lemma-completion-tensor}).
Thus the kernel of $S \to B$ is generated by a regular sequence
if and only if the kernel of $S^\wedge \to B$ is generated by a
regular sequence.
Hence it suffices to consider good factorizations where $S$ is complete.

\medskip\noindent
Assume we have two factorizations $A \to S \to B$ and
$A \to S' \to B$ with $S$ and $S'$ complete. By
More on Algebra, Lemma \ref{more-algebra-lemma-dominate-two-surjections}
the ring $S \times_B S'$ is a Noetherian complete local ring.
Hence, using More on Algebra, Lemma
\ref{more-algebra-lemma-embed-map-Noetherian-complete-local-rings}
we can choose a good factorization $A \to S'' \to S \times_B S'$
with $S''$ complete. Thus it suffices to show:
If $A \to S' \to S \to B$ are comparable good factorizations,
then $\Ker(S \to B)$ is generated by a regular sequence
if and only if $\Ker(S' \to B)$ is generated by a regular sequence.

\medskip\noindent
Let $A \to S' \to S \to B$ be comparable good factorizations.
First, since $S'/\mathfrak m_R S' \to S/\mathfrak m_R S$ is
a surjection of regular local rings, the kernel is generated
by a regular sequence
$\overline{x}_1, \ldots, \overline{x}_c \in
\mathfrak m_{S'}/\mathfrak m_R S'$
which can be extended to a regular system of parameters for
the regular local ring $S'/\mathfrak m_R S'$, see
(Algebra, Lemma \ref{algebra-lemma-regular-quotient-regular}).
Set $I = \Ker(S' \to S)$. By flatness of $S$ over $R$ we have
$$
I/\mathfrak m_R I =
\Ker(S'/\mathfrak m_R S' \to S/\mathfrak m_R S) =
(\overline{x}_1, \ldots, \overline{x}_c).
$$
Choose lifts $x_1, \ldots, x_c \in I$. These lifts form a regular sequence
generating $I$ as $S'$ is flat over $R$, see
Algebra, Lemma \ref{algebra-lemma-grothendieck-regular-sequence}.

\medskip\noindent
We conclude that if also $\Ker(S \to B)$ is generated by a
regular sequence, then so is $\Ker(S' \to B)$, see
More on Algebra, Lemmas
\ref{more-algebra-lemma-join-koszul-regular-sequences} and
\ref{more-algebra-lemma-noetherian-finite-all-equivalent}.

\medskip\noindent
Conversely, assume that $J = \Ker(S' \to B)$ is generated
by a regular sequence. Because the generators $x_1, \ldots, x_c$
of $I$ map to linearly independent elements of
$\mathfrak m_{S'}/\mathfrak m_{S'}^2$ we see that
$I/\mathfrak m_{S'}I \to J/\mathfrak m_{S'}J$ is injective.
Hence there exists a minimal system of generators
$x_1, \ldots, x_c, y_1, \ldots, y_d$ for $J$.
Then $x_1, \ldots, x_c, y_1, \ldots, y_d$ is a regular sequence
and it follows that the images of $y_1, \ldots, y_d$ in $S$
form a regular sequence generating $\Ker(S \to B)$.
This finishes the proof of the lemma.
\end{proof}

\noindent
In the following proposition observe that the condition on vanishing of
Tor's applies in particular if $B$ has finite tor dimension over $A$ and
thus in particular if $B$ is flat over $A$.

\begin{proposition}
\label{proposition-avramov-map}
Let $A \to B$ be a local homomorphism of Noetherian local rings.
Then the following are equivalent
\begin{enumerate}
\item $B$ is a complete intersection and
$\text{Tor}^A_p(B, A/\mathfrak m_A)$ is nonzero for only finitely many $p$,
\item $A$ is a complete intersection and
$A^\wedge \to B^\wedge$ is a complete intersection homomorphism
in the sense defined above.
\end{enumerate}
\end{proposition}

\begin{proof}
Let $F_\bullet \to A/\mathfrak m_A$ be a resolution by finite
free $A$-modules. Observe that
$\text{Tor}^A_p(B, A/\mathfrak m_A)$
is the $p$th homology of the complex $F_\bullet \otimes_A B$.
Let $F_\bullet^\wedge = F_\bullet \otimes_A A^\wedge$ be the completion.
Then $F_\bullet^\wedge$ is a resolution of $A^\wedge/\mathfrak m_{A^\wedge}$
by finite free $A^\wedge$-modules (as $A \to A^\wedge$ is flat and completion
on finite modules is exact, see
Algebra, Lemmas \ref{algebra-lemma-completion-tensor} and
\ref{algebra-lemma-completion-flat}).
It follows that
$$
F_\bullet^\wedge \otimes_{A^\wedge} B^\wedge =
F_\bullet \otimes_A B \otimes_B B^\wedge
$$
By flatness of $B \to B^\wedge$ we conclude that
$$
\text{Tor}^{A^\wedge}_p(B^\wedge, A^\wedge/\mathfrak m_{A^\wedge}) =
\text{Tor}^A_p(B, A/\mathfrak m_A) \otimes_B B^\wedge
$$
In this way we see that the condition in (1) on the local ring map $A \to B$
is equivalent to the same condition for the local ring map
$A^\wedge \to B^\wedge$.
Thus we may assume $A$ and $B$ are complete local Noetherian rings
(since the other conditions are formulated in terms of the completions
in any case).

\medskip\noindent
Assume $A$ and $B$ are complete local Noetherian rings.
Choose a diagram
$$
\xymatrix{
S \ar[r] & B \\
R \ar[u] \ar[r] & A \ar[u]
}
$$
as in More on Algebra, Lemma
\ref{more-algebra-lemma-embed-map-Noetherian-complete-local-rings}.
Let $I = \Ker(R \to A)$ and $J = \Ker(S \to B)$.
The proposition now follows from Lemma \ref{lemma-perfect-map-ci}.
\end{proof}

\begin{remark}
\label{remark-no-good-ci-map}
It appears difficult to define an good notion of ``local complete
intersection homomorphisms'' for maps between general Noetherian rings.
The reason is that, for a local Noetherian ring $A$, the fibres of
$A \to A^\wedge$ are not local complete intersection rings.
Thus, if $A \to B$ is a local homomorphism of local Noetherian rings,
and the map of completions $A^\wedge \to B^\wedge$ is a
complete intersection homomorphism in the sense defined above,
then $(A_\mathfrak p)^\wedge \to (B_\mathfrak q)^\wedge$ is in general
{\bf not} a complete intersection homomorphism in the sense
defined above. A solution can be had by working exclusively with
excellent Noetherian rings. More generally, one could work with
those Noetherian rings whose formal fibres are complete
intersections, see \cite{Rodicio-ci}.
We will develop this theory in
Dualizing Complexes, Section \ref{dualizing-section-formal-fibres}.
\end{remark}

\noindent
To finish of this section we compare the notion defined above
with the notion introduced in
More on Algebra, Section \ref{section-lci}.

\begin{lemma}
\label{lemma-well-defined-if-you-can-find-good-factorization}
Consider a commutative diagram
$$
\xymatrix{
S \ar[r] & B \\
& A \ar[lu] \ar[u]
}
$$
of Noetherian local rings with $S \to B$ surjective, $A \to S$ flat, and
$S/\mathfrak m_A S$ a regular local ring. The following are equivalent
\begin{enumerate}
\item $\Ker(S \to B)$ is generated by a regular sequence, and
\item $A^\wedge \to B^\wedge$ is a complete intersection homomorphism
as defined above.
\end{enumerate}
\end{lemma}

\begin{proof}
Omitted. Hint: the proof is identical to the argument given in
the first paragraph of the proof of Lemma \ref{lemma-ci-map-well-defined}.
\end{proof}

\begin{lemma}
\label{lemma-finite-type-lci-map}
Let $A$ be a Noetherian ring.
Let $A \to B$ be a finite type ring map.
The following are equivalent
\begin{enumerate}
\item $A \to B$ is a local complete intersection in the sense of
More on Algebra, Definition
\ref{more-algebra-definition-local-complete-intersection},
\item for every prime $\mathfrak q \subset B$ and with
$\mathfrak p = A \cap \mathfrak q$ the ring map
$(A_\mathfrak p)^\wedge \to (B_\mathfrak q)^\wedge$ is
a complete intersection homomorphism in the sense defined above.
\end{enumerate}
\end{lemma}

\begin{proof}
Choose a surjection $R = A[x_1, \ldots, x_n] \to B$.
Observe that $A \to R$ is flat with regular fibres.
Let $I$ be the kernel of $R \to B$.
Assume (2). Then we see that
$I$ is locally generated by a regular sequence
by
Lemma \ref{lemma-well-defined-if-you-can-find-good-factorization}
and
Algebra, Lemma \ref{algebra-lemma-regular-sequence-in-neighbourhood}.
In other words, (1) holds.
Conversely, assume (1). Then after localizing on $R$ and $B$
we can assume that $I$ is generated by a Koszul regular sequence.
By More on Algebra, Lemma
\ref{more-algebra-lemma-noetherian-finite-all-equivalent}
we find that $I$ is locally generated by a regular sequence.
Hence (2) hold by
Lemma \ref{lemma-well-defined-if-you-can-find-good-factorization}.
Some details omitted.
\end{proof}

\begin{lemma}
\label{lemma-avramov-map-finite-type}
Let $A$ be a Noetherian ring. Let $A \to B$ be a finite type ring map
such that the image of $\Spec(B) \to \Spec(A)$ contains all closed
points of $\Spec(A)$. Then the following are equivalent
\begin{enumerate}
\item $B$ is a complete intersection and $A \to B$ has finite
tor dimension,
\item $A$ is a complete intersection and $A \to B$ is a local complete
intersection in the sense of More on Algebra, Definition
\ref{more-algebra-definition-local-complete-intersection}.
\end{enumerate}
\end{lemma}

\begin{proof}
This is a reformulation of Proposition \ref{proposition-avramov-map}
via Lemma \ref{lemma-finite-type-lci-map}.
We omit the details.
\end{proof}








\section{Smooth ring maps and diagonals}
\label{section-smooth-diagonal-perfect}

\noindent
In this section we use the material above to characterize smooth ring maps as
those whose diagonal is perfect.

\begin{lemma}
\label{lemma-local-perfect-diagonal}
Let $A \to B$ be a local ring homomorphism of Noetherian local rings such that
$B$ is flat and essentially of finite type over $A$. If
$$
B \otimes_A B \longrightarrow B
$$
is a perfect ring map, i.e., if $B$ has finite tor dimension over
$B \otimes_A B$, then $B$ is the localization of a smooth $A$-algebra.
\end{lemma}

\begin{proof}
As $B$ is essentially of finite type over $A$, so is $B \otimes_A B$ and
in particular $B \otimes_A B$ is Noetherian. Hence the quotient $B$ of
$B \otimes_A B$ is pseudo-coherent over $B \otimes_A B$
(More on Algebra, Lemma \ref{more-algebra-lemma-Noetherian-pseudo-coherent})
which explains why perfectness of the ring map (More on Algebra, Definition
\ref{more-algebra-definition-pseudo-coherent-perfect}) agrees with the
condition of finite tor dimension.

\medskip\noindent
We may write $B = R/K$ where $R$ is the localization of $A[x_1, \ldots, x_n]$
at a prime ideal and $K \subset R$ is an ideal. Denote
$\mathfrak m \subset R \otimes_A R$ the maximal ideal which is the inverse
image of the maximal ideal of $B$ via the surjection
$R \otimes_A R \to B \otimes_A B \to B$. Then we have surjections
$$
(R \otimes_A R)_\mathfrak m \to (B \otimes_A B)_\mathfrak m \to B
$$
and hence ideals $I \subset J \subset (R \otimes_A R)_\mathfrak m$
as in Lemma \ref{lemma-injective}. We conclude that
$I/\mathfrak m I \to J/\mathfrak m J$ is injective.

\medskip\noindent
Let $K = (f_1, \ldots, f_r)$ with $r$ minimal. We may and do assume that
$f_i \in R$ is the image of an element of $A[x_1, \ldots, x_n]$ which we
also denote $f_i$. Observe that $I$ is generated
by $f_1 \otimes 1, \ldots, f_r \otimes 1$ and
$1 \otimes f_1, \ldots, 1 \otimes f_r$. We claim that this is a minimal
set of generators of $I$. Namely, if $\kappa$ is the common residue field
of $R$, $B$, $(R \otimes_A R)_\mathfrak m$, and $(B \otimes_A B)_\mathfrak m$
then we have a map
$R \otimes_A R \to R \otimes_A \kappa \oplus \kappa \otimes_A R$
which factors through $(R \otimes_A R)_\mathfrak m$. Since $B$ is
flat over $A$ and since we have the short exact sequence
$0 \to K \to R \to B \to 0$ we see that
$K \otimes_A \kappa \subset R \otimes_A \kappa$, see
Algebra, Lemma \ref{algebra-lemma-flat-tor-zero}.
Thus restricting the map
$(R \otimes_A R)_\mathfrak m \to R \otimes_A \kappa \oplus \kappa \otimes_A R$
to $I$ we obtain a map
$$
I \to K \otimes_A \kappa \oplus \kappa \otimes_A K \to
K \otimes_B \kappa \oplus \kappa \otimes_B K.
$$
The elements
$f_1 \otimes 1, \ldots, f_r \otimes 1, 1 \otimes f_1, \ldots, 1 \otimes f_r$
map to a basis of the target of this map, since by Nakayama's lemma
(Algebra, Lemma \ref{algebra-lemma-NAK})
$f_1, \ldots, f_r$ map to a basis of $K \otimes_B \kappa$.
This proves our claim.

\medskip\noindent
The ideal $J$ is generated by $f_1 \otimes 1, \ldots, f_r \otimes 1$
and the elements $x_1 \otimes 1 - 1 \otimes x_1, \ldots,
x_n \otimes 1 - 1 \otimes x_n$ (for the proof it suffices to
see that these elements are contained in the ideal $J$).
Now we can write
$$
f_i \otimes 1 - 1 \otimes f_i =
\sum g_{ij} (x_j \otimes 1 - 1 \otimes x_j)
$$
for some $g_{ij}$ in $(R \otimes_A R)_\mathfrak m$. This is a general
fact about elements of $A[x_1, \ldots, x_n]$ whose proof we omit.
Denote $a_{ij} \in \kappa$ the image of $g_{ij}$. Another computation
shows that $a_{ij}$ is the image of $\partial f_i / \partial x_j$ in $\kappa$.
The injectivity of $I/\mathfrak m I \to J/\mathfrak m J$ and the remarks
made above forces the matrix $(a_{ij})$ to have maximal rank $r$.
Set
$$
C = A[x_1, \ldots, x_n]/(f_1, \ldots, f_r)
$$
and consider the naive cotangent complex
$\NL_{C/A} \cong (C^{\oplus r} \to C^{\oplus n})$
where the map is given by the matrix of partial derivatives.
Thus $\NL_{C/A} \otimes_A B$
is isomorphic to a free $B$-module of rank $n - r$ placed in degree $0$.
Hence $C_g$ is smooth over $A$ for some $g \in C$ mapping to a unit
in $B$, see Algebra, Lemma \ref{algebra-lemma-smooth-at-point}.
This finishes the proof.
\end{proof}

\begin{lemma}
\label{lemma-perfect-diagonal}
Let $A \to B$ be a flat finite type ring map of Noetherian rings. If
$$
B \otimes_A B \longrightarrow B
$$
is a perfect ring map, i.e., if $B$ has finite tor dimension over
$B \otimes_A B$, then $B$ is a smooth $A$-algebra.
\end{lemma}

\begin{proof}
This follows from Lemma \ref{lemma-local-perfect-diagonal}
and general facts about smooth ring maps, see
Algebra, Lemmas \ref{algebra-lemma-smooth-at-point} and
\ref{algebra-lemma-locally-smooth}.
Alternatively, the reader can slightly modify the proof of
Lemma \ref{lemma-local-perfect-diagonal} to prove
this lemma.
\end{proof}






\section{Freeness of the conormal module}
\label{section-freeness-conormal}

\noindent
Tate resolutions and derivations on them can be used to prove
(stronger) versions of the results in this section, see \cite{Iyengar}.
Two more elementary references are
\cite{Vasconcelos} and \cite{Ferrand-lci}.

\begin{lemma}
\label{lemma-free-summand-in-ideal-finite-proj-dim}
\begin{reference}
\cite{Vasconcelos}
\end{reference}
Let $R$ be a Noetherian local ring. Let $I \subset R$ be an ideal
of finite projective dimension over $R$. If $F \subset I/I^2$ is a
direct summand isomorphic to $R/I$, then there exists a nonzerodivisor
$x \in I$ such that the image of $x$ in $I/I^2$ generates $F$.
\end{lemma}

\begin{proof}
By assumption we may choose a finite free resolution
$$
0 \to R^{\oplus n_e} \to R^{\oplus n_{e-1}} \to \ldots \to
R^{\oplus n_1} \to R \to R/I \to 0
$$
Then $\varphi_1 : R^{\oplus n_1} \to R$ has rank $1$ and
we see that $I$ contains a nonzerodivisor $y$ by
Algebra, Proposition \ref{algebra-proposition-what-exact}.
Let $\mathfrak p_1, \ldots, \mathfrak p_n$ be the associated
primes of $R$, see Algebra, Lemma \ref{algebra-lemma-finite-ass}.
Let $I^2 \subset J \subset I$ be an ideal such that $J/I^2 = F$.
Then $J \not \subset \mathfrak p_i$ for all $i$
as $y^2 \in J$ and $y^2 \not \in \mathfrak p_i$, see
Algebra, Lemma \ref{algebra-lemma-ass-zero-divisors}.
By Nakayama's lemma (Algebra, Lemma \ref{algebra-lemma-NAK})
we have $J \not \subset \mathfrak m J + I^2$.
By Algebra, Lemma \ref{algebra-lemma-silly}
we can pick $x \in J$, $x \not \in \mathfrak m J + I^2$ and
$x \not \in \mathfrak p_i$ for $i = 1, \ldots, n$.
Then $x$ is a nonzerodivisor and the image
of $x$ in $I/I^2$ generates (by Nakayama's lemma)
the summand $J/I^2 \cong R/I$.
\end{proof}

\begin{lemma}
\label{lemma-vasconcelos}
\begin{reference}
Local version of \cite[Theorem 1.1]{Vasconcelos}
\end{reference}
Let $R$ be a Noetherian local ring. Let $I \subset R$ be an ideal
of finite projective dimension over $R$. If $F \subset I/I^2$
is a direct summand free of rank $r$, then there exists a regular sequence
$x_1, \ldots, x_r \in I$ such that $x_1 \bmod I^2, \ldots, x_r \bmod I^2$
generate $F$.
\end{lemma}

\begin{proof}
If $r = 0$ there is nothing to prove. Assume $r > 0$. We may pick
$x \in I$ such that $x$ is a nonzerodivisor and $x \bmod I^2$
generates a summand of $F$ isomorphic to $R/I$, see
Lemma \ref{lemma-free-summand-in-ideal-finite-proj-dim}.
Consider the ring $R' = R/(x)$ and the ideal $I' = I/(x)$.
Of course $R'/I' = R/I$. The short exact sequence
$$
0 \to R/I \xrightarrow{x} I/xI \to I' \to 0
$$
splits because the map $I/xI \to I/I^2$ sends $xR/xI$
to a direct summand. Now $I/xI = I \otimes_R^\mathbf{L} R'$ has
finite projective dimension over $R'$, see
More on Algebra, Lemmas \ref{more-algebra-lemma-perfect-module} and
\ref{more-algebra-lemma-pull-perfect}.
Hence the summand $I'$ has finite projective dimension over $R'$.
On the other hand, we have the short exact sequence
$0 \to xR/xI \to I/I^2 \to I'/(I')^2 \to 0$ and we conclude
$I'/(I')^2$ has the free direct summand $F' = F/(R/I \cdot x)$
of rank $r - 1$. By induction on $r$ we may
we pick a regular sequence $x'_2, \ldots, x'_r \in I'$
such that there congruence classes freely generate $F'$.
If $x_1 = x$ and $x_2, \ldots, x_r$ are any elements lifting
$x'_1, \ldots, x'_r$ in $R$, then we see that the lemma holds.
\end{proof}

\begin{proposition}
\label{proposition-regular-ideal}
\begin{reference}
Variant of \cite[Corollary 1]{Vasconcelos}. See also
\cite{Iyengar} and \cite{Ferrand-lci}.
\end{reference}
Let $R$ be a Noetherian ring. Let $I \subset R$ be an ideal
which has finite projective dimension and such that $I/I^2$ is
finite locally free over $R/I$. Then $I$ is a regular ideal
(More on Algebra, Definition \ref{more-algebra-definition-regular-ideal}).
\end{proposition}

\begin{proof}
By Algebra, Lemma \ref{algebra-lemma-regular-sequence-in-neighbourhood}
it suffices to show that $I_\mathfrak p \subset R_\mathfrak p$ is generated
by a regular sequence for every $\mathfrak p \supset I$. Thus we may
assume $R$ is local. If $I/I^2$ has rank $r$, then by
Lemma \ref{lemma-vasconcelos} we find a regular sequence
$x_1, \ldots, x_r \in I$ generating $I/I^2$. By
Nakayama (Algebra, Lemma \ref{algebra-lemma-NAK})
we conclude that $I$ is generated by $x_1, \ldots, x_r$.
\end{proof}

\noindent
For any local complete intersection homomorphism $A \to B$
of rings, the naive cotangent complex $\NL_{B/A}$ is perfect
of tor-amplitude in $[-1, 0]$, see
More on Algebra, Lemma \ref{more-algebra-lemma-lci-NL}.
Using the above, we can show that this sometimes
characterizes local complete intersection homomorphisms.

\begin{lemma}
\label{lemma-perfect-NL-lci}
Let $A \to B$ be a perfect (More on Algebra, Definition
\ref{more-algebra-definition-pseudo-coherent-perfect})
ring homomorphism of Noetherian rings. Then the following are equivalent
\begin{enumerate}
\item $\NL_{B/A}$ has tor-amplitude in $[-1, 0]$,
\item $\NL_{B/A}$ is a perfect object of $D(B)$
with tor-amplitude in $[-1, 0]$, and
\item $A \to B$ is a local complete intersection
(More on Algebra, Definition
\ref{more-algebra-definition-local-complete-intersection}).
\end{enumerate}
\end{lemma}

\begin{proof}
Write $B = A[x_1, \ldots, x_n]/I$. Then $\NL_{B/A}$ is represented by
the complex
$$
I/I^2 \longrightarrow \bigoplus B \text{d}x_i
$$
of $B$-modules with $I/I^2$ placed in degree $-1$. Since the term in
degree $0$ is finite free, this complex has tor-amplitude in $[-1, 0]$ if and
only if $I/I^2$ is a flat $B$-module, see
More on Algebra, Lemma \ref{more-algebra-lemma-last-one-flat}.
Since $I/I^2$ is a finite $B$-module and $B$ is Noetherian, this is true
if and only if $I/I^2$ is a finite locally free $B$-module
(Algebra, Lemma \ref{algebra-lemma-finite-projective}).
Thus the equivalence of (1) and (2) is clear. Moreover, the equivalence
of (1) and (3) also follows if we apply
Proposition \ref{proposition-regular-ideal}
(and the observation that a regular ideal is a Koszul regular
ideal as well as a quasi-regular ideal, see
More on Algebra, Section \ref{more-algebra-section-ideals}).
\end{proof}

\begin{lemma}
\label{lemma-flat-fp-NL-lci}
Let $A \to B$ be a flat ring map of finite presentation.
Then the following are equivalent
\begin{enumerate}
\item $\NL_{B/A}$ has tor-amplitude in $[-1, 0]$,
\item $\NL_{B/A}$ is a perfect object of $D(B)$
with tor-amplitude in $[-1, 0]$,
\item $A \to B$ is syntomic
(Algebra, Definition \ref{algebra-definition-lci}), and
\item $A \to B$ is a local complete intersection
(More on Algebra, Definition
\ref{more-algebra-definition-local-complete-intersection}).
\end{enumerate}
\end{lemma}

\begin{proof}
The equivalence of (3) and (4) is More on Algebra, Lemma
\ref{more-algebra-lemma-syntomic-lci}.

\medskip\noindent
If $A \to B$ is syntomic, then we can find a cocartesian diagram
$$
\xymatrix{
B_0 \ar[r] & B \\
A_0 \ar[r] \ar[u] & A \ar[u]
}
$$
such that $A_0 \to B_0$ is syntomic and $A_0$ is Noetherian, see
Algebra, Lemmas \ref{algebra-lemma-limit-module-finite-presentation} and
\ref{algebra-lemma-colimit-lci}. By Lemma \ref{lemma-perfect-NL-lci}
we see that $\NL_{B_0/A_0}$ is perfect of tor-amplitude in $[-1, 0]$.
By More on Algebra, Lemma \ref{more-algebra-lemma-base-change-NL-flat}
we conclude the same thing is true for
$\NL_{B/A} = \NL_{B_0/A_0} \otimes_{B_0}^\mathbf{L} B$ (see
also More on Algebra, Lemmas \ref{more-algebra-lemma-pull-tor-amplitude} and
\ref{more-algebra-lemma-pull-perfect}).
This proves that (3) implies (2).

\medskip\noindent
Assume (1). By More on Algebra, Lemma
\ref{more-algebra-lemma-base-change-NL-flat}
for every ring map $A \to k$ where
$k$ is a field, we see that $\NL_{B \otimes_A k/k}$ has
tor-amplitude in $[-1, 0]$ (see
More on Algebra, Lemma \ref{more-algebra-lemma-pull-tor-amplitude}).
Hence by Lemma \ref{lemma-perfect-NL-lci} we see that $k \to B \otimes_A k$ is
a local complete intersection homomorphism. Thus $A \to B$
is syntomic by definition. This proves (1) implies (3)
and finishes the proof.
\end{proof}





\section{Koszul complexes and Tate resolutions}
\label{section-koszul-vs-tate}

\noindent
In this section we ``lift'' the result of
More on Algebra, Lemma \ref{more-algebra-lemma-sequence-Koszul-complexes}
to the category of differential graded algebras endowed with divided
powers compatible with the differential graded structure (beware
that in this section we represent Koszul complexes as chain complexes
whereas in locus citatus we use cochain complexes).

\medskip\noindent
Let $R$ be a ring. Let $I \subset R$ be an ideal generated
by $f_1, \ldots, f_r \in R$. For $n \geq 1$ we denote
$$
K_n = K_{n, \bullet} = R\langle \xi_1, \ldots, \xi_r\rangle
$$
the differential graded Koszul algebra with $\xi_i$ in degree $1$ and
$\text{d}(\xi_i) = f_i$. There exists a unique divided power structure on this
(as in Definition \ref{definition-divided-powers-dga}), see
Example \ref{example-adjoining-odd}. For $m > n$ the transition map
$K_m \to K_n$ is the differential graded algebra map compatible with
divided powers given by sending $\xi_i$ to $f_i^{m - n}\xi_i$.

\begin{lemma}
\label{lemma-lift-tate-to-koszul}
In the situation above, if $R$ is Noetherian, then
for every $n$ there exists an $N \geq n$ and maps
$$
K_N \to A \to R/(f_1^N, \ldots, f_r^N)\quad\text{and}\quad A \to K_n
$$
with the following properties
\begin{enumerate}
\item $(A, \text{d}, \gamma)$ is as in
Definition \ref{definition-divided-powers-dga},
\item $A \to R/(f_1^N, \ldots, f_r^N)$ is a quasi-isomorphism,
\item the composition $K_N \to A \to R/(f_1^N, \ldots, f_r^N)$
is the canonical map,
\item the composition $K_N \to A \to K_n$ is the transition map,
\item $A_0 = R \to R/(f_1^N, \ldots, f_r^N)$ is the canonical
surjection,
\item $A$ is a graded divided power polynomial algebra over $R$
with finitely many generators in each degree, and
\item $A \to K_n$ is a homomorphism of differential graded $R$-algebras
compatible with divided powers which induces the canonical map
$R/(f_1^N, \ldots, f_r^N) \to R/(f_1^n, \ldots, f_r^n)$ on
homology in degree $0$.
\end{enumerate}
Condition (4) means that $A$ is constructed out of $A_0$ by
successively adjoining a finite set of variables $T$ in each degree
$> 0$ as in Example \ref{example-adjoining-odd} or \ref{example-adjoining-even}.
\end{lemma}

\begin{proof}
Fix $n$. If $r = 0$, then we can just pick $A = R$. Assume $r > 0$. By
More on Algebra, Lemma \ref{more-algebra-lemma-sequence-Koszul-complexes}
(translated into the language of chain complexes) we can choose
$$
n_{r} > n_{r - 1} > \ldots > n_1 > n_0 = n
$$
such that the transition maps $K_{n_{i + 1}} \to K_{n_i}$ on Koszul
algebras (see above) induce the zero map on homology in degrees $> 0$.
We will prove the lemma with $N = n_r$.

\medskip\noindent
We will construct $A$ exactly as in the statement and proof of
Lemma \ref{lemma-tate-resolution}. Thus we will have
$$
A = \colim A(m),\quad\text{and}\quad
A(0) \to A(1) \to A(2) \to \ldots \to R/(f_1^N, \ldots, f_r^N)
$$
This will immediately give us properties (1), (2), (5), and (6).
To finish the proof we will construct the $R$-algebra maps
$K_N \to A \to K_n$. To do this we will construct
\begin{enumerate}
\item an isomorphism $A(1) \to K_N = K_{n_r}$,
\item a map $A(2) \to K_{n_{r - 1}}$,
\item $\ldots$
\item a map $A(r) \to K_{n_1}$,
\item a map $A(r + 1) \to K_{n_0} = K_n$, and
\item a map $A \to K_n$.
\end{enumerate}
In each of these steps the map constructed will be between
differential graded algebras compatibly endowed with divided powers
and each of the maps will be compatible with the
previous one via the transition maps between the Koszul algebras
and each of the maps will induce the obvious canonical map
on homology in degree $0$.

\medskip\noindent
Recall that $A(0) = R$. For $m = 1$, the proof of
Lemma \ref{lemma-tate-resolution}
chooses $A(1) = R\langle T_1, \ldots, T_r\rangle$ with
$T_i$ of degree $1$ and with $\text{d}(T_i) = f_i^N$.
Namely, the $f_i^N$ are generators of the kernel of
$A(0) \to R/(f_1^N, \ldots, f_r^N)$.
Thus for $A(1) \to K_N = K_{n_r}$ we use the map
$$
\varphi_1 : A(1) \longrightarrow K_{n_r},\quad T_i \longmapsto \xi_i
$$
which is an isomorphism.

\medskip\noindent
For $m = 2$, the construction in the proof of Lemma \ref{lemma-tate-resolution}
chooses generators $e_1, \ldots, e_t \in \Ker(\text{d} : A(1)_1 \to A(1)_0)$.
The construction proceeds by taking
$A(2) = A(1)\langle T_1, \ldots, T_t\rangle$
as a divided power polynomial algebra with $T_i$ of degree $2$
and with $\text{d}(T_i) = e_i$.
Since $\varphi_1(e_i)$ is a cocycle in $K_{n_r}$
we see that its image in $K_{n_{r - 1}}$ is a coboundary by
our choice of $n_r$ and $n_{r - 1}$ above.
Hence we can construct the following commutative diagram
$$
\xymatrix{
A(1) \ar[d] \ar[r]_{\varphi_1} & K_{n_r} \ar[d] \\
A(2) \ar[r]^{\varphi_2} & K_{n_{r - 1}}
}
$$
by sending $T_i$ to an element in degree $2$ whose boundary is the
image of $\varphi_1(e_i)$. The map $\varphi_2$ exists and is compatible
with the differential and the divided powers by the universal
of the divided power polynomial algebra.

\medskip\noindent
The algebra $A(m)$ and the map $\varphi_m : A(m) \to K_{n_{r + 1 - m}}$
are constructed in exactly the same manner for $m = 2, \ldots, r$.

\medskip\noindent
Given the map $A(r) \to K_{n_1}$ we see that the composition
$H_r(A(r)) \to H_r(K_{n_1}) \to H_r(K_{n_0}) \subset (K_{n_0})_r$
is zero, hence we can extend this to $A(r + 1) \to K_{n_0} = K_n$
by sending the new polynomial generators of $A(r + 1)$ to zero.

\medskip\noindent
Having constructed $A(r + 1) \to K_{n_0} = K_n$ we can simply
extend to $A(r + 2), A(r + 3), \ldots$ in the only possible way
by sending the new polynomial generators to zero.
This finishes the proof.
\end{proof}

\begin{remark}
\label{remark-pro-system-koszul}
In the situation above, if $R$ is Noetherian,
we can inductively choose a sequence
of integers $1 = n_0 < n_1 < n_2 < \ldots $ such that
for $i = 1, 2, 3, \ldots$ we have maps
$K_{n_i} \to A_i \to R/(f_1^{n_i}, \ldots, f_r^{n_i})$
and $A_i \to K_{n_{i - 1}}$ as in Lemma \ref{lemma-lift-tate-to-koszul}.
Denote $A_{i + 1} \to A_i$ the composition $A_{i + 1} \to K_{n_i} \to A_i$.
Then the diagram
$$
\xymatrix{
K_{n_1} \ar[d] &
K_{n_2} \ar[d] \ar[l] &
K_{n_3} \ar[d] \ar[l] &
\ldots \ar[l] \\
A_1 \ar[d] &
A_2 \ar[l] \ar[d] &
A_3 \ar[l] \ar[d] &
\ldots \ar[l] \\
K_1 &
K_{n_1} \ar[l] &
K_{n_2} \ar[l] &
\ldots \ar[l]
}
$$
commutes. In this way we see that the inverse systems
$(K_n)$ and $(A_n)$ are pro-isomorphic in the category of
differential graded $R$-algebras with compatible divided powers.
\end{remark}

\begin{lemma}
\label{lemma-compute-cohomology-adjoin-variable}
Let $(A, \text{d}, \gamma)$, $d \geq 1$, $f \in A_{d - 1}$,
and $A\langle T \rangle$ be as in Lemma \ref{lemma-extend-differential}.
\begin{enumerate}
\item If $d = 1$, then there is a long exact sequences
$$
\ldots \to H_0(A) \xrightarrow{f} H_0(A) \to H_0(A\langle T \rangle) \to 0
$$
\item For $d = 2$ there is a bounded spectral sequence
$(E_1)_{i, j} = H_{j - i}(A) \cdot T^{[i]}$
converging to $H_{i + j}(A\langle T \rangle)$. The differential
$(d_1)_{i, j} : H_{j - i}(A) \cdot T^{[i]} \to
H_{j - i + 1}(A) \cdot T^{[i - 1]}$
sends $\xi \cdot T^{[i]}$ to the class of $f \xi \cdot T^{[i - 1]}$.
\item Add more here for other degrees as needed.
\end{enumerate}
\end{lemma}

\begin{proof}
For $d = 1$, we have a short exact sequence of complexes
$$
0 \to A \to A\langle T \rangle \to A \cdot T \to 0
$$
and the result (1) follows easily from this. For $d = 2$ we view
$A\langle T \rangle$ as a filtered chain complex with subcomplexes
$$
F^pA\langle T \rangle = \bigoplus\nolimits_{i \leq p} A \cdot T^{[i]}
$$
Applying the spectral sequence of
Homology, Section \ref{homology-section-filtered-complex}
(translated into chain complexes) we obtain (2).
\end{proof}

\noindent
The following lemma will be needed later.

\begin{lemma}
\label{lemma-construct-some-maps}
In the situation above, for all $n \geq t \geq 1$ there exists an $N > n$
and a map
$$
K_t \longrightarrow K_n \otimes_R K_t
$$
in the derived category of left differential graded $K_N$-modules
whose composition with the multiplication map is the transition map
(in either direction).
\end{lemma}

\begin{proof}
We first prove this for $r = 1$. Set $f = f_1$.
Write $K_t = R\langle x \rangle$,
$K_n = R\langle y \rangle$, and $K_N = R\langle z \rangle$
with $x$, $y$, $z$ of degree $1$ and
$\text{d}(x) = f^t$, $\text{d}(y) = f^n$, and
$\text{d}(z) = f^N$. For all $N > t$ we claim there is a quasi-isomorphism
$$
B_{N, t} = R\langle x, z, u \rangle
\longrightarrow
K_t,\quad
x \mapsto x,\quad
z \mapsto f^{N - t}x,\quad
u \mapsto 0
$$
Here the left hand side denotes the divided power polynomial algebra
in variables $x$ and $z$ of degree $1$ and $u$ of degree $2$ with
$\text{d}(x) = f^t$, $\text{d}(z) = f^N$, and
$\text{d}(u) = z - f^{N - t}x$. To prove the claim,
we observe that the following three submodules of
$H_*(R\langle x, z\rangle)$ are the same
\begin{enumerate}
\item the kernel of $H_*(R\langle x, z\rangle) \to H_*(K_t)$,
\item the image of
$z - f^{N - t}x : H_*(R\langle x, z\rangle) \to H_*(R\langle x, z\rangle)$, and
\item the kernel of
$z - f^{N - t}x : H_*(R\langle x, z\rangle) \to H_*(R\langle x, z\rangle)$.
\end{enumerate}
This observation is proved by a direct computation\footnote{Hint: setting
$z' = z - f^{N - t}x$ we see that
$R\langle x, z\rangle = R\langle x, z'\rangle$ with $\text{d}(z') = 0$
and moreover the map $R\langle x, z'\rangle \to K_t$ is
the map killing $z'$.} which we omit. Then we can
apply Lemma \ref{lemma-compute-cohomology-adjoin-variable} part (2)
to see that the claim is true.

\medskip\noindent
Via the homomorphism $K_N \to B_{N, t}$ of differential graded $R$-algebras
sending $z$ to $z$, we may view $B_{N, t} \to K_t$ as a quasi-isomorphism of
left differential graded $K_N$-modules. To define the arrow in the statement
of the lemma we use the homomorphism
$$
B_{N, t} = R\langle x, z, u \rangle \to K_n \otimes_R K_t,\quad
x \mapsto 1 \otimes x,\quad
z \mapsto f^{N - n}y \otimes 1,\quad
u \mapsto - f^{N - n - t}y \otimes x
$$
This makes sense as long as we assume $N \geq n + t$. It is a
pleasant computation to show that the (pre or post) composition with
the multiplication map is the transition map.

\medskip\noindent
For $r > 1$ we proceed by writing each of the Koszul algebras as a
tensor product of Koszul algebras in $1$ variable and we apply the
previous construction. In other words, we write
$$
K_t = R\langle x_1, \ldots, x_r\rangle =
R\langle x_1\rangle \otimes_R \ldots \otimes_R R\langle x_r\rangle
$$
where $x_i$ is in degree $1$ and $\text{d}(x_i) = f_i^t$.
In the case $r > 1$ we then use
$$
B_{N, t} =
R\langle x_1, z_1, u_1 \rangle
\otimes_R \ldots \otimes_R
R\langle x_r, z_r, u_r \rangle
$$
where $x_i, z_i$ have degree $1$ and $u_i$ has degree $2$ and we have
$\text{d}(x_i) = f_i^t$, $\text{d}(z_i) = f_i^N$, and
$\text{d}(u_i) = z_i - f_i^{N - t}x_i$.
The tensor product map $B_{N, t} \to K_t$ will be a quasi-isomorphism
as it is a tensor product of quasi-isomorphisms between bounded above
complexes of free $R$-modules. Finally, we define the map
$$
B_{N, t}
\to
K_n \otimes_R K_t =
R\langle y_1, \ldots, y_r\rangle \otimes_R
R\langle x_1, \ldots, x_r\rangle
$$
as the tensor product of the maps constructed in the case of $r = 1$
or simply by the rules $x_i \mapsto 1 \otimes x_i$,
$z_i \mapsto f_i^{N - n}y_i \otimes 1$, and
$u_i \mapsto - f_i^{N - n - t}y_i \otimes x_i$ which makes
sense as long as $N \geq n + t$. We omit the details.
\end{proof}








\begin{multicols}{2}[\section{Other chapters}]
\noindent
Preliminaries
\begin{enumerate}
\item \hyperref[introduction-section-phantom]{Introduction}
\item \hyperref[conventions-section-phantom]{Conventions}
\item \hyperref[sets-section-phantom]{Set Theory}
\item \hyperref[categories-section-phantom]{Categories}
\item \hyperref[topology-section-phantom]{Topology}
\item \hyperref[sheaves-section-phantom]{Sheaves on Spaces}
\item \hyperref[sites-section-phantom]{Sites and Sheaves}
\item \hyperref[stacks-section-phantom]{Stacks}
\item \hyperref[fields-section-phantom]{Fields}
\item \hyperref[algebra-section-phantom]{Commutative Algebra}
\item \hyperref[brauer-section-phantom]{Brauer Groups}
\item \hyperref[homology-section-phantom]{Homological Algebra}
\item \hyperref[derived-section-phantom]{Derived Categories}
\item \hyperref[simplicial-section-phantom]{Simplicial Methods}
\item \hyperref[more-algebra-section-phantom]{More on Algebra}
\item \hyperref[smoothing-section-phantom]{Smoothing Ring Maps}
\item \hyperref[modules-section-phantom]{Sheaves of Modules}
\item \hyperref[sites-modules-section-phantom]{Modules on Sites}
\item \hyperref[injectives-section-phantom]{Injectives}
\item \hyperref[cohomology-section-phantom]{Cohomology of Sheaves}
\item \hyperref[sites-cohomology-section-phantom]{Cohomology on Sites}
\item \hyperref[dga-section-phantom]{Differential Graded Algebra}
\item \hyperref[dpa-section-phantom]{Divided Power Algebra}
\item \hyperref[sdga-section-phantom]{Differential Graded Sheaves}
\item \hyperref[hypercovering-section-phantom]{Hypercoverings}
\end{enumerate}
Schemes
\begin{enumerate}
\setcounter{enumi}{25}
\item \hyperref[schemes-section-phantom]{Schemes}
\item \hyperref[constructions-section-phantom]{Constructions of Schemes}
\item \hyperref[properties-section-phantom]{Properties of Schemes}
\item \hyperref[morphisms-section-phantom]{Morphisms of Schemes}
\item \hyperref[coherent-section-phantom]{Cohomology of Schemes}
\item \hyperref[divisors-section-phantom]{Divisors}
\item \hyperref[limits-section-phantom]{Limits of Schemes}
\item \hyperref[varieties-section-phantom]{Varieties}
\item \hyperref[topologies-section-phantom]{Topologies on Schemes}
\item \hyperref[descent-section-phantom]{Descent}
\item \hyperref[perfect-section-phantom]{Derived Categories of Schemes}
\item \hyperref[more-morphisms-section-phantom]{More on Morphisms}
\item \hyperref[flat-section-phantom]{More on Flatness}
\item \hyperref[groupoids-section-phantom]{Groupoid Schemes}
\item \hyperref[more-groupoids-section-phantom]{More on Groupoid Schemes}
\item \hyperref[etale-section-phantom]{\'Etale Morphisms of Schemes}
\end{enumerate}
Topics in Scheme Theory
\begin{enumerate}
\setcounter{enumi}{41}
\item \hyperref[chow-section-phantom]{Chow Homology}
\item \hyperref[intersection-section-phantom]{Intersection Theory}
\item \hyperref[pic-section-phantom]{Picard Schemes of Curves}
\item \hyperref[weil-section-phantom]{Weil Cohomology Theories}
\item \hyperref[adequate-section-phantom]{Adequate Modules}
\item \hyperref[dualizing-section-phantom]{Dualizing Complexes}
\item \hyperref[duality-section-phantom]{Duality for Schemes}
\item \hyperref[discriminant-section-phantom]{Discriminants and Differents}
\item \hyperref[derham-section-phantom]{de Rham Cohomology}
\item \hyperref[local-cohomology-section-phantom]{Local Cohomology}
\item \hyperref[algebraization-section-phantom]{Algebraic and Formal Geometry}
\item \hyperref[curves-section-phantom]{Algebraic Curves}
\item \hyperref[resolve-section-phantom]{Resolution of Surfaces}
\item \hyperref[models-section-phantom]{Semistable Reduction}
\item \hyperref[functors-section-phantom]{Functors and Morphisms}
\item \hyperref[equiv-section-phantom]{Derived Categories of Varieties}
\item \hyperref[pione-section-phantom]{Fundamental Groups of Schemes}
\item \hyperref[etale-cohomology-section-phantom]{\'Etale Cohomology}
\item \hyperref[crystalline-section-phantom]{Crystalline Cohomology}
\item \hyperref[proetale-section-phantom]{Pro-\'etale Cohomology}
\item \hyperref[relative-cycles-section-phantom]{Relative Cycles}
\item \hyperref[more-etale-section-phantom]{More \'Etale Cohomology}
\item \hyperref[trace-section-phantom]{The Trace Formula}
\end{enumerate}
Algebraic Spaces
\begin{enumerate}
\setcounter{enumi}{64}
\item \hyperref[spaces-section-phantom]{Algebraic Spaces}
\item \hyperref[spaces-properties-section-phantom]{Properties of Algebraic Spaces}
\item \hyperref[spaces-morphisms-section-phantom]{Morphisms of Algebraic Spaces}
\item \hyperref[decent-spaces-section-phantom]{Decent Algebraic Spaces}
\item \hyperref[spaces-cohomology-section-phantom]{Cohomology of Algebraic Spaces}
\item \hyperref[spaces-limits-section-phantom]{Limits of Algebraic Spaces}
\item \hyperref[spaces-divisors-section-phantom]{Divisors on Algebraic Spaces}
\item \hyperref[spaces-over-fields-section-phantom]{Algebraic Spaces over Fields}
\item \hyperref[spaces-topologies-section-phantom]{Topologies on Algebraic Spaces}
\item \hyperref[spaces-descent-section-phantom]{Descent and Algebraic Spaces}
\item \hyperref[spaces-perfect-section-phantom]{Derived Categories of Spaces}
\item \hyperref[spaces-more-morphisms-section-phantom]{More on Morphisms of Spaces}
\item \hyperref[spaces-flat-section-phantom]{Flatness on Algebraic Spaces}
\item \hyperref[spaces-groupoids-section-phantom]{Groupoids in Algebraic Spaces}
\item \hyperref[spaces-more-groupoids-section-phantom]{More on Groupoids in Spaces}
\item \hyperref[bootstrap-section-phantom]{Bootstrap}
\item \hyperref[spaces-pushouts-section-phantom]{Pushouts of Algebraic Spaces}
\end{enumerate}
Topics in Geometry
\begin{enumerate}
\setcounter{enumi}{81}
\item \hyperref[spaces-chow-section-phantom]{Chow Groups of Spaces}
\item \hyperref[groupoids-quotients-section-phantom]{Quotients of Groupoids}
\item \hyperref[spaces-more-cohomology-section-phantom]{More on Cohomology of Spaces}
\item \hyperref[spaces-simplicial-section-phantom]{Simplicial Spaces}
\item \hyperref[spaces-duality-section-phantom]{Duality for Spaces}
\item \hyperref[formal-spaces-section-phantom]{Formal Algebraic Spaces}
\item \hyperref[restricted-section-phantom]{Algebraization of Formal Spaces}
\item \hyperref[spaces-resolve-section-phantom]{Resolution of Surfaces Revisited}
\end{enumerate}
Deformation Theory
\begin{enumerate}
\setcounter{enumi}{89}
\item \hyperref[formal-defos-section-phantom]{Formal Deformation Theory}
\item \hyperref[defos-section-phantom]{Deformation Theory}
\item \hyperref[cotangent-section-phantom]{The Cotangent Complex}
\item \hyperref[examples-defos-section-phantom]{Deformation Problems}
\end{enumerate}
Algebraic Stacks
\begin{enumerate}
\setcounter{enumi}{93}
\item \hyperref[algebraic-section-phantom]{Algebraic Stacks}
\item \hyperref[examples-stacks-section-phantom]{Examples of Stacks}
\item \hyperref[stacks-sheaves-section-phantom]{Sheaves on Algebraic Stacks}
\item \hyperref[criteria-section-phantom]{Criteria for Representability}
\item \hyperref[artin-section-phantom]{Artin's Axioms}
\item \hyperref[quot-section-phantom]{Quot and Hilbert Spaces}
\item \hyperref[stacks-properties-section-phantom]{Properties of Algebraic Stacks}
\item \hyperref[stacks-morphisms-section-phantom]{Morphisms of Algebraic Stacks}
\item \hyperref[stacks-limits-section-phantom]{Limits of Algebraic Stacks}
\item \hyperref[stacks-cohomology-section-phantom]{Cohomology of Algebraic Stacks}
\item \hyperref[stacks-perfect-section-phantom]{Derived Categories of Stacks}
\item \hyperref[stacks-introduction-section-phantom]{Introducing Algebraic Stacks}
\item \hyperref[stacks-more-morphisms-section-phantom]{More on Morphisms of Stacks}
\item \hyperref[stacks-geometry-section-phantom]{The Geometry of Stacks}
\end{enumerate}
Topics in Moduli Theory
\begin{enumerate}
\setcounter{enumi}{107}
\item \hyperref[moduli-section-phantom]{Moduli Stacks}
\item \hyperref[moduli-curves-section-phantom]{Moduli of Curves}
\end{enumerate}
Miscellany
\begin{enumerate}
\setcounter{enumi}{109}
\item \hyperref[examples-section-phantom]{Examples}
\item \hyperref[exercises-section-phantom]{Exercises}
\item \hyperref[guide-section-phantom]{Guide to Literature}
\item \hyperref[desirables-section-phantom]{Desirables}
\item \hyperref[coding-section-phantom]{Coding Style}
\item \hyperref[obsolete-section-phantom]{Obsolete}
\item \hyperref[fdl-section-phantom]{GNU Free Documentation License}
\item \hyperref[index-section-phantom]{Auto Generated Index}
\end{enumerate}
\end{multicols}


\bibliography{my}
\bibliographystyle{amsalpha}

\end{document}
