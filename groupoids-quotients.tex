\IfFileExists{stacks-project.cls}{%
\documentclass{stacks-project}
}{%
\documentclass{amsart}
}

% For dealing with references we use the comment environment
\usepackage{verbatim}
\newenvironment{reference}{\comment}{\endcomment}
%\newenvironment{reference}{}{}
\newenvironment{slogan}{\comment}{\endcomment}
\newenvironment{history}{\comment}{\endcomment}

% For commutative diagrams we use Xy-pic
\usepackage[all]{xy}

% We use 2cell for 2-commutative diagrams.
\xyoption{2cell}
\UseAllTwocells

% We use multicol for the list of chapters between chapters
\usepackage{multicol}

% This is generall recommended for better output
\usepackage{lmodern}
\usepackage[T1]{fontenc}

% For cross-file-references
\usepackage{xr-hyper}

% Package for hypertext links:
\usepackage{hyperref}

% For any local file, say "hello.tex" you want to link to please
% use \externaldocument[hello-]{hello}
\externaldocument[introduction-]{introduction}
\externaldocument[conventions-]{conventions}
\externaldocument[sets-]{sets}
\externaldocument[categories-]{categories}
\externaldocument[topology-]{topology}
\externaldocument[sheaves-]{sheaves}
\externaldocument[sites-]{sites}
\externaldocument[stacks-]{stacks}
\externaldocument[fields-]{fields}
\externaldocument[algebra-]{algebra}
\externaldocument[brauer-]{brauer}
\externaldocument[homology-]{homology}
\externaldocument[derived-]{derived}
\externaldocument[simplicial-]{simplicial}
\externaldocument[more-algebra-]{more-algebra}
\externaldocument[smoothing-]{smoothing}
\externaldocument[modules-]{modules}
\externaldocument[sites-modules-]{sites-modules}
\externaldocument[injectives-]{injectives}
\externaldocument[cohomology-]{cohomology}
\externaldocument[sites-cohomology-]{sites-cohomology}
\externaldocument[dga-]{dga}
\externaldocument[dpa-]{dpa}
\externaldocument[sdga-]{sdga}
\externaldocument[hypercovering-]{hypercovering}
\externaldocument[schemes-]{schemes}
\externaldocument[constructions-]{constructions}
\externaldocument[properties-]{properties}
\externaldocument[morphisms-]{morphisms}
\externaldocument[coherent-]{coherent}
\externaldocument[divisors-]{divisors}
\externaldocument[limits-]{limits}
\externaldocument[varieties-]{varieties}
\externaldocument[topologies-]{topologies}
\externaldocument[descent-]{descent}
\externaldocument[perfect-]{perfect}
\externaldocument[more-morphisms-]{more-morphisms}
\externaldocument[flat-]{flat}
\externaldocument[groupoids-]{groupoids}
\externaldocument[more-groupoids-]{more-groupoids}
\externaldocument[etale-]{etale}
\externaldocument[chow-]{chow}
\externaldocument[intersection-]{intersection}
\externaldocument[pic-]{pic}
\externaldocument[weil-]{weil}
\externaldocument[adequate-]{adequate}
\externaldocument[dualizing-]{dualizing}
\externaldocument[duality-]{duality}
\externaldocument[discriminant-]{discriminant}
\externaldocument[derham-]{derham}
\externaldocument[local-cohomology-]{local-cohomology}
\externaldocument[algebraization-]{algebraization}
\externaldocument[curves-]{curves}
\externaldocument[resolve-]{resolve}
\externaldocument[models-]{models}
\externaldocument[functors-]{functors}
\externaldocument[equiv-]{equiv}
\externaldocument[pione-]{pione}
\externaldocument[etale-cohomology-]{etale-cohomology}
\externaldocument[proetale-]{proetale}
\externaldocument[relative-cycles-]{relative-cycles}
\externaldocument[more-etale-]{more-etale}
\externaldocument[trace-]{trace}
\externaldocument[crystalline-]{crystalline}
\externaldocument[spaces-]{spaces}
\externaldocument[spaces-properties-]{spaces-properties}
\externaldocument[spaces-morphisms-]{spaces-morphisms}
\externaldocument[decent-spaces-]{decent-spaces}
\externaldocument[spaces-cohomology-]{spaces-cohomology}
\externaldocument[spaces-limits-]{spaces-limits}
\externaldocument[spaces-divisors-]{spaces-divisors}
\externaldocument[spaces-over-fields-]{spaces-over-fields}
\externaldocument[spaces-topologies-]{spaces-topologies}
\externaldocument[spaces-descent-]{spaces-descent}
\externaldocument[spaces-perfect-]{spaces-perfect}
\externaldocument[spaces-more-morphisms-]{spaces-more-morphisms}
\externaldocument[spaces-flat-]{spaces-flat}
\externaldocument[spaces-groupoids-]{spaces-groupoids}
\externaldocument[spaces-more-groupoids-]{spaces-more-groupoids}
\externaldocument[bootstrap-]{bootstrap}
\externaldocument[spaces-pushouts-]{spaces-pushouts}
\externaldocument[spaces-chow-]{spaces-chow}
\externaldocument[groupoids-quotients-]{groupoids-quotients}
\externaldocument[spaces-more-cohomology-]{spaces-more-cohomology}
\externaldocument[spaces-simplicial-]{spaces-simplicial}
\externaldocument[spaces-duality-]{spaces-duality}
\externaldocument[formal-spaces-]{formal-spaces}
\externaldocument[restricted-]{restricted}
\externaldocument[spaces-resolve-]{spaces-resolve}
\externaldocument[formal-defos-]{formal-defos}
\externaldocument[defos-]{defos}
\externaldocument[cotangent-]{cotangent}
\externaldocument[examples-defos-]{examples-defos}
\externaldocument[algebraic-]{algebraic}
\externaldocument[examples-stacks-]{examples-stacks}
\externaldocument[stacks-sheaves-]{stacks-sheaves}
\externaldocument[criteria-]{criteria}
\externaldocument[artin-]{artin}
\externaldocument[quot-]{quot}
\externaldocument[stacks-properties-]{stacks-properties}
\externaldocument[stacks-morphisms-]{stacks-morphisms}
\externaldocument[stacks-limits-]{stacks-limits}
\externaldocument[stacks-cohomology-]{stacks-cohomology}
\externaldocument[stacks-perfect-]{stacks-perfect}
\externaldocument[stacks-introduction-]{stacks-introduction}
\externaldocument[stacks-more-morphisms-]{stacks-more-morphisms}
\externaldocument[stacks-geometry-]{stacks-geometry}
\externaldocument[moduli-]{moduli}
\externaldocument[moduli-curves-]{moduli-curves}
\externaldocument[examples-]{examples}
\externaldocument[exercises-]{exercises}
\externaldocument[guide-]{guide}
\externaldocument[desirables-]{desirables}
\externaldocument[coding-]{coding}
\externaldocument[obsolete-]{obsolete}
\externaldocument[fdl-]{fdl}
\externaldocument[index-]{index}

% Theorem environments.
%
\theoremstyle{plain}
\newtheorem{theorem}[subsection]{Theorem}
\newtheorem{proposition}[subsection]{Proposition}
\newtheorem{lemma}[subsection]{Lemma}

\theoremstyle{definition}
\newtheorem{definition}[subsection]{Definition}
\newtheorem{example}[subsection]{Example}
\newtheorem{exercise}[subsection]{Exercise}
\newtheorem{situation}[subsection]{Situation}

\theoremstyle{remark}
\newtheorem{remark}[subsection]{Remark}
\newtheorem{remarks}[subsection]{Remarks}

\numberwithin{equation}{subsection}

% Macros
%
\def\lim{\mathop{\mathrm{lim}}\nolimits}
\def\colim{\mathop{\mathrm{colim}}\nolimits}
\def\Spec{\mathop{\mathrm{Spec}}}
\def\Hom{\mathop{\mathrm{Hom}}\nolimits}
\def\Ext{\mathop{\mathrm{Ext}}\nolimits}
\def\SheafHom{\mathop{\mathcal{H}\!\mathit{om}}\nolimits}
\def\SheafExt{\mathop{\mathcal{E}\!\mathit{xt}}\nolimits}
\def\Sch{\mathit{Sch}}
\def\Mor{\mathop{\mathrm{Mor}}\nolimits}
\def\Ob{\mathop{\mathrm{Ob}}\nolimits}
\def\Sh{\mathop{\mathit{Sh}}\nolimits}
\def\NL{\mathop{N\!L}\nolimits}
\def\CH{\mathop{\mathrm{CH}}\nolimits}
\def\proetale{{pro\text{-}\acute{e}tale}}
\def\etale{{\acute{e}tale}}
\def\QCoh{\mathit{QCoh}}
\def\Ker{\mathop{\mathrm{Ker}}}
\def\Im{\mathop{\mathrm{Im}}}
\def\Coker{\mathop{\mathrm{Coker}}}
\def\Coim{\mathop{\mathrm{Coim}}}

% Boxtimes
%
\DeclareMathSymbol{\boxtimes}{\mathbin}{AMSa}{"02}

%
% Macros for moduli stacks/spaces
%
\def\QCohstack{\mathcal{QC}\!\mathit{oh}}
\def\Cohstack{\mathcal{C}\!\mathit{oh}}
\def\Spacesstack{\mathcal{S}\!\mathit{paces}}
\def\Quotfunctor{\mathrm{Quot}}
\def\Hilbfunctor{\mathrm{Hilb}}
\def\Curvesstack{\mathcal{C}\!\mathit{urves}}
\def\Polarizedstack{\mathcal{P}\!\mathit{olarized}}
\def\Complexesstack{\mathcal{C}\!\mathit{omplexes}}
% \Pic is the operator that assigns to X its picard group, usage \Pic(X)
% \Picardstack_{X/B} denotes the Picard stack of X over B
% \Picardfunctor_{X/B} denotes the Picard functor of X over B
\def\Pic{\mathop{\mathrm{Pic}}\nolimits}
\def\Picardstack{\mathcal{P}\!\mathit{ic}}
\def\Picardfunctor{\mathrm{Pic}}
\def\Deformationcategory{\mathcal{D}\!\mathit{ef}}


% OK, start here.
%
\begin{document}

\title{Quotients of Groupoids}


\maketitle

\phantomsection
\label{section-phantom}

\tableofcontents

\section{Introduction}
\label{section-introduction}

\noindent
This chapter is devoted to generalities concerning groupoids and their
quotients (as far as they exist).
There is a lot of literature on this subject, see for example
\cite{GIT}, \cite{seshadri_quotients}, \cite{KollarQuotients},
\cite{K-M}, \cite{KollarFinite} and many more.





\section{Conventions and notation}
\label{section-conventions-notation}

\noindent
In this chapter the conventions and notation are those introduced in
Groupoids in Spaces, Sections \ref{spaces-groupoids-section-conventions}
and \ref{spaces-groupoids-section-notation}.


\section{Invariant morphisms}
\label{section-invariant}

\begin{definition}
\label{definition-invariant}
Let $S$ be a scheme, and let $B$ be an algebraic space over $S$.
Let $j = (t, s) : R \to U \times_B U$ be a pre-relation of algebraic
spaces over $B$. We say a morphism $\phi : U \to X$ of algebraic spaces
over $B$ is {\it $R$-invariant} if the diagram
$$
\xymatrix{
R \ar[r]_s \ar[d]_t & U \ar[d]^\phi \\
U \ar[r]^\phi & X
}
$$
is commutative. If $j : R \to U \times_B U$ comes from the action
of a group algebraic space $G$ on $U$ over $B$ as in
Groupoids in Spaces, Lemma \ref{spaces-groupoids-lemma-groupoid-from-action},
then we say that $\phi$ is {\it $G$-invariant}.
\end{definition}

\noindent
In other words, a morphism $U \to X$ is $R$-invariant if it equalizes
$s$ and $t$.  We can reformulate this in terms of associated quotient
sheaves as follows.

\begin{lemma}
\label{lemma-invariant}
Let $S$ be a scheme, and let $B$ be an algebraic space over $S$.
Let $j = (t, s) : R \to U \times_B U$ be a pre-relation of algebraic
spaces over $B$. A morphism of algebraic spaces $\phi : U \to X$ is
$R$-invariant if and only if it factors as
$U \to U/R \to X$.
\end{lemma}

\begin{proof}
This is clear from the definition of the quotient sheaf in
Groupoids in Spaces, Section \ref{spaces-groupoids-section-quotient-sheaves}.
\end{proof}

\begin{lemma}
\label{lemma-base-change-on-invariant}
Let $S$ be a scheme, and let $B$ be an algebraic space over $S$.
Let $j = (t, s) : R \to U \times_B U$ be a pre-relation of algebraic
spaces over $B$. Let $U \to X$ be an $R$-invariant morphism of algebraic
spaces over $B$. Let $X' \to X$ be any morphism of algebraic spaces.
\begin{enumerate}
\item Setting $U' = X' \times_X U$, $R' = X' \times_X R$ we obtain
a pre-relation $j' : R' \to U' \times_B U'$.
\item If $j$ is a relation, then $j'$ is a relation.
\item If $j$ is a pre-equivalence relation, then $j'$ is a
pre-equivalence relation.
\item If $j$ is an equivalence relation, then $j'$ is an equivalence
relation.
\item If $j$ comes from a groupoid in algebraic spaces
$(U, R, s, t, c)$ over $B$, then
\begin{enumerate}
\item $(U, R, s, t, c)$ is a groupoid in algebraic spaces over $X$, and
\item $j'$ comes from the base change $(U', R', s', t', c')$
of this groupoid to $X'$, see
Groupoids in Spaces, Lemma
\ref{spaces-groupoids-lemma-base-change-groupoid}.
\end{enumerate}
\item If $j$ comes from the action of a group algebraic space $G/B$ on $U$
as in Groupoids in Spaces, Lemma
\ref{spaces-groupoids-lemma-groupoid-from-action}
then $j'$ comes from the induced action of $G$ on $U'$.
\end{enumerate}
\end{lemma}

\begin{proof}
Omitted. Hint: Functorial point of view combined with the picture:
$$
\xymatrix{
R' = X' \times_X R \ar[dd] \ar[rr] \ar[rd] & &
X' \times_X U = U' \ar'[d][dd] \ar[rd] \\
& R \ar[dd] \ar[rr] & & U \ar[dd] \\
U' = X' \times_X U \ar'[r][rr] \ar[rd] & & X' \ar[rd] \\
& U \ar[rr] & & X
}
$$
\end{proof}

\begin{definition}
\label{definition-base-change}
In the situation of Lemma \ref{lemma-base-change-on-invariant}
we call $j' : R' \to U' \times_B U'$ the {\it base change} of the pre-relation
$j$ to $X'$. We say it is a {\it flat base change} if $X' \to X$ is a flat
morphism of algebraic spaces.
\end{definition}

\noindent
This kind of base change interacts well with taking quotient sheaves
and quotient stacks.

\begin{lemma}
\label{lemma-base-change-quotient-sheaf}
In the situation of Lemma \ref{lemma-base-change-on-invariant}
there is an isomorphism of sheaves
$$
U'/R' = X' \times_X U/R
$$
For the construction of quotient sheaves, see
Groupoids in Spaces, Section \ref{spaces-groupoids-section-quotient-sheaves}.
\end{lemma}

\begin{proof}
Since $U \to X$ is $R$-invariant, it is clear that the map
$U \to X$ factors through the quotient sheaf $U/R$.
Recall that by definition
$$
\xymatrix{
R \ar@<1ex>[r] \ar@<-1ex>[r] &
U \ar[r] &
U/R
}
$$
is a coequalizer diagram in the category $\Sh$ of sheaves of sets on
$(\Sch/S)_{fppf}$. In fact, this is a coequalizer diagram in the
comma category $\Sh/X$. Since the base change functor
$X' \times_X - : \Sh/X \to \Sh/X'$ is exact (true in any topos),
we conclude.
\end{proof}

\begin{lemma}
\label{lemma-base-change-quotient-stack}
Let $S$ be a scheme. Let $B$ be an algebraic space over $S$.
Let $(U, R, s, t, c)$ be a groupoid in algebraic spaces over $B$.
Let $U \to X$ be an $R$-invariant morphism of algebraic spaces over
$B$. Let $g : X' \to X$ be a morphism of algebraic spaces over $B$
and let $(U', R', s', t', c')$ be the base change as in
Lemma \ref{lemma-base-change-on-invariant}. Then
$$
\xymatrix{
[U'/R'] \ar[r] \ar[d] & [U/R] \ar[d] \\
\mathcal{S}_{X'} \ar[r] & \mathcal{S}_X
}
$$
is a $2$-fibre product of stacks in groupoids over $(\Sch/S)_{fppf}$.
For the construction of quotient stacks and the morphisms in this
diagram, see
Groupoids in Spaces, Section \ref{spaces-groupoids-section-stacks}.
\end{lemma}

\begin{proof}
We will prove this by using the explicit
description of the quotient stacks given in
Groupoids in Spaces, Lemma
\ref{spaces-groupoids-lemma-quotient-stack-objects}.
However, we strongly urge the reader to find their own proof.
First, we may view $(U, R, s, t, c)$ as a groupoid in
algebraic spaces over $X$, hence we obtain a map
$f : [U/R] \to \mathcal{S}_X$, see
Groupoids in Spaces, Lemma \ref{spaces-groupoids-lemma-quotient-stack-arrows}.
Similarly, we have $f' : [U'/R'] \to X'$.

\medskip\noindent
An object of the $2$-fibre product
$\mathcal{S}_{X'} \times_{\mathcal{S}_X} [U/R]$ over a scheme $T$ over $S$
is the same as a morphism $x' : T \to X'$ and an object $y$ of $[U/R]$ over $T$
such that such that the composition $g \circ x'$ is equal to $f(y)$.
This makes sense because objects of $\mathcal{S}_X$ over $T$
are morphisms $T \to X$. By Groupoids in Spaces, Lemma
\ref{spaces-groupoids-lemma-quotient-stack-objects}
we may assume $y$ is given by a $[U/R]$-descent datum $(u_i, r_{ij})$
relative to an fppf covering $\{T_i \to T\}$.
The agreement of $g \circ x' = f(y)$ means that the diagrams
$$
\vcenter{
\xymatrix{
T_i \ar[rr]_{u_i} \ar[d] & & U \ar[d] \\
T \ar[r]^{x'} & X' \ar[r]^g & X
}
}
\quad\text{and}\quad
\vcenter{
\xymatrix{
T_i \times_T T_j \ar[rr]_{r_{ij}} \ar[d] & & R \ar[d] \\
T \ar[r]^{x'} & X' \ar[r]^g & X
}
}
$$
are commutative.

\medskip\noindent
On the other hand, an object $y'$ of $[U'/R']$ over a scheme $T$ over $S$
by Groupoids in Spaces, Lemma
\ref{spaces-groupoids-lemma-quotient-stack-objects}
is given by a $[U'/R']$-descent datum $(u'_i, r'_{ij})$
relative to an fppf covering $\{T_i \to T\}$.
Setting $f'(y') = x' : T \to X'$ we see that
the diagrams
$$
\vcenter{
\xymatrix{
T_i \ar[r]_{u'_i} \ar[d] & U' \ar[d] \\
T \ar[r]^{x'} & X'
}
}
\quad\text{and}\quad
\vcenter{
\xymatrix{
T_i \times_T T_j \ar[r]_{r'_{ij}} \ar[d] & U' \ar[d] \\
T \ar[r]^{x'} & X'
}
}
$$
are commutative.

\medskip\noindent
With this notation in place, we define a functor
$$
[U'/R'] \longrightarrow \mathcal{S}_{X'} \times_{\mathcal{S}_X} [U/R]
$$
by sending $y' = (u'_i, r'_{ij})$ as above to the object
$(x', (u_i, r_{ij}))$ where $x' = f'(y')$, where
$u_i$ is the composition $T_i \to U' \to U$, and where
$r_{ij}$ is the composition $T_i \times_T T_j \to R' \to R$.
Conversely, given an object $(x', (u_i, r_{ij})$
of the right hand side we can send this to the object
$((x', u_i), (x', r_{ij}))$ of the left hand side.
We omit the discussion of what to do with morphisms (works
in exactly the same manner).
\end{proof}









\section{Categorical quotients}
\label{section-categorical}

\noindent
This is the most basic kind of quotient one can consider.

\begin{definition}
\label{definition-categorical}
Let $S$ be a scheme, and let $B$ be an algebraic space over $S$.
Let $j = (t, s) : R \to U \times_B U$ be pre-relation in algebraic spaces
over $B$.
\begin{enumerate}
\item We say a morphism $\phi : U \to X$ of algebraic spaces over $B$
is a {\it categorical quotient} if it is $R$-invariant, and
for every $R$-invariant morphism $\psi : U \to Y$ of algebraic spaces over $B$
there exists a unique morphism $\chi : X \to Y$ such that
$\psi = \phi \circ \chi$.
\item Let $\mathcal{C}$ be a full subcategory of the category of algebraic
spaces over $B$. Assume $U$, $R$ are objects of $\mathcal{C}$.
In this situation we say
a morphism $\phi : U \to X$ of algebraic spaces over $B$
is a {\it categorical quotient in $\mathcal{C}$}
if $X \in \Ob(\mathcal{C})$, and $\phi$ is $R$-invariant,
and for every $R$-invariant morphism
$\psi : U \to Y$ with $Y \in \Ob(\mathcal{C})$
there exists a unique morphism $\chi : X \to Y$ such
that $\psi = \phi \circ \chi$.
\item If $B = S$ and $\mathcal{C}$ is the category of schemes over $S$,
then we say $U \to X$ is a
{\it categorical quotient in the category of schemes}, or simply a
{\it categorical quotient in schemes}.
\end{enumerate}
\end{definition}

\noindent
We often single out a category $\mathcal{C}$ of algebraic spaces over $B$
by some separation axiom, see
Example \ref{example-categories}
for some standard cases.
Note that $\phi : U \to X$ is a categorical quotient if and only
if $U \to X$ is a coequalizer for the
morphisms $t, s : R \to U$ in the category. Hence we immediately
deduce the following lemma.

\begin{lemma}
\label{lemma-categorical-unique}
Let $S$ be a scheme, and let $B$ be an algebraic space over $S$.
Let $j : R \to U \times_B U$ be a pre-relation in algebraic spaces over $B$.
If a categorical quotient in the category of algebraic spaces
over $B$ exists, then it is unique up to unique isomorphism.
Similarly for categorical quotients in full subcategories of
$\textit{Spaces}/B$.
\end{lemma}

\begin{proof}
See Categories, Section \ref{categories-section-coequalizers}.
\end{proof}

\begin{example}
\label{example-categories}
Let $S$ be a scheme, and let $B$ be an algebraic space over $S$.
Here are some standard examples of categories $\mathcal{C}$
that we often come up when applying
Definition \ref{definition-categorical}:
\begin{enumerate}
\item $\mathcal{C}$ is the category of all algebraic spaces over $B$,
\item $B$ is separated and $\mathcal{C}$ is the category of all separated
algebraic spaces over $B$,
\item $B$ is quasi-separated and $\mathcal{C}$ is the category of all
quasi-separated algebraic spaces over $B$,
\item $B$ is locally separated and $\mathcal{C}$ is the category of all
locally separated algebraic spaces over $B$,
\item $B$ is decent and $\mathcal{C}$ is the category of all decent algebraic
spaces over $B$, and
\item $S = B$ and $\mathcal{C}$ is the category of schemes over $S$.
\end{enumerate}
In this case, if $\phi : U \to X$ is a categorical quotient then we say
$U \to X$ is
(1) a {\it categorical quotient},
(2) a {\it categorical quotient in separated algebraic spaces},
(3) a {\it categorical quotient in quasi-separated algebraic spaces},
(4) a {\it categorical quotient in locally separated algebraic spaces},
(5) a {\it categorical quotient in decent algebraic spaces},
(6) a {\it categorical quotient in schemes}.
\end{example}

\begin{definition}
\label{definition-universal-categorical}
Let $S$ be a scheme, and let $B$ be an algebraic space over $S$.
Let $\mathcal{C}$ be a full subcategory of the category of algebraic
spaces over $B$ closed under fibre products.
Let $j = (t, s) : R \to U \times_B U$ be pre-relation in
$\mathcal{C}$, and let $U \to X$ be an $R$-invariant morphism with
$X \in \Ob(\mathcal{C})$.
\begin{enumerate}
\item We say $U \to X$ is a {\it universal categorical quotient}
in $\mathcal{C}$ if for every morphism $X' \to X$ in $\mathcal{C}$
the morphism $U' = X' \times_X U \to X'$ is the categorical quotient in
$\mathcal{C}$ of the base change $j' : R' \to U'$ of $j$.
\item We say $U \to X$ is a {\it uniform categorical quotient}
in $\mathcal{C}$ if for every flat morphism $X' \to X$ in $\mathcal{C}$
the morphism $U' = X' \times_X U \to X'$ is the categorical quotient in
$\mathcal{C}$ of the base change $j' : R' \to U'$ of $j$.
\end{enumerate}
\end{definition}

\begin{lemma}
\label{lemma-categorical-reduced}
In the situation of
Definition \ref{definition-categorical}.
If $\phi : U \to X$ is a categorical quotient and $U$ is reduced,
then $X$ is reduced. The same holds for categorical quotients in
a category of spaces $\mathcal{C}$ listed in
Example \ref{example-categories}.
\end{lemma}

\begin{proof}
Let $X_{red}$ be the reduction of the algebraic space $X$.
Since $U$ is reduced the morphism $\phi : U \to X$ factors through
$i : X_{red} \to X$ (Properties of Spaces, Lemma
\ref{spaces-properties-lemma-map-into-reduction}). Denote this morphism
by $\phi_{red} : U \to X_{red}$. Since $\phi \circ s = \phi \circ t$ we
see that also $\phi_{red} \circ s = \phi_{red} \circ t$ (as
$i : X_{red} \to X$ is a monomorphism). Hence by the universal property
of $\phi$ there exists a morphism $\chi : X \to X_{red}$ such that
$\phi_{red} = \phi \circ \chi$. By uniqueness we see that
$i \circ \chi = \text{id}_X$ and $\chi \circ i = \text{id}_{X_{red}}$.
Hence $i$ is an isomorphism and $X$ is reduced.

\medskip\noindent
To show that this argument works in a category $\mathcal{C}$ one
just needs to show that the reduction of an object of $\mathcal{C}$
is an object of $\mathcal{C}$. We omit the verification that this
holds for each of the standard examples.
\end{proof}





\section{Quotients as orbit spaces}
\label{section-orbits}

\noindent
Let $j = (t, s) : R \to U \times_B U$ be a pre-relation.
If $j$ is a pre-equivalence relation, then loosely speaking
the ``orbits'' of $R$ on $U$
are the subsets $t(s^{-1}(\{u\}))$ of $U$. However, if $j$ is just a
pre-relation, then we need to take the equivalence relation generated
by $R$.

\begin{definition}
\label{definition-orbit}
Let $S$ be a scheme, and let $B$ be an algebraic space over $S$.
Let $j : R \to U \times_B U$ be a pre-relation over $B$.
If $u \in |U|$, then the {\it orbit}, or more precisely the
{\it $R$-orbit} of $u$ is
$$
O_u =
\left\{
u' \in |U|\ :
\begin{matrix}
\exists n \geq 1, \ \exists u_0, \ldots, u_n \in |U|\text{ such that }
u_0 = u \text{ and } u_n = u' \\
\text{and for all }i \in \{0, \ldots, n - 1\}\text{ either }
u_i = u_{i + 1}\text{ or } \\
\exists r \in |R|, \ s(r) = u_i, t(r) = u_{i + 1}
\text{ or } \\
\exists r \in |R|, \ t(r) = u_i, s(r) = u_{i + 1}
\end{matrix}
\right\}
$$
\end{definition}

\noindent
It is clear that these are the equivalence classes of an equivalence relation,
i.e., we have $u' \in O_u$ if and only if $u \in O_{u'}$. The following lemma
is a reformulation of
Groupoids in Spaces,
Lemma \ref{spaces-groupoids-lemma-pre-equivalence-equivalence-relation-points}.

\begin{lemma}
\label{lemma-pre-equivalence-equivalence-relation-points}
Let $B \to S$ as in Section \ref{section-conventions-notation}.
Let $j : R \to U \times_B U$ be a pre-equivalence relation
of algebraic spaces over $B$. Then
$$
O_u =
\{u' \in |U| \text{ such that } \exists r \in |R|, \ s(r) = u, \ t(r) = u'\}.
$$
\end{lemma}

\begin{proof}
By the aforementioned
Groupoids in Spaces,
Lemma \ref{spaces-groupoids-lemma-pre-equivalence-equivalence-relation-points}
we see that the orbits $O_u$ as defined in the lemma give a disjoint
union decomposition of $|U|$. Thus we see they are equal to the
orbits as defined in Definition \ref{definition-orbit}.
\end{proof}

\begin{lemma}
\label{lemma-invariant-map-constant-on-orbit}
In the situation of Definition \ref{definition-orbit}.
Let $\phi : U \to X$ be an $R$-invariant morphism of algebraic spaces over
$B$. Then $|\phi| : |U| \to |X|$ is constant on the orbits.
\end{lemma}

\begin{proof}
To see this we just have to show that $\phi(u) = \phi(u')$
for all $u, u' \in |U|$ such that
there exists an $r \in |R|$ such that $s(r) = u$ and $t(r) = u'$.
And this is clear since $\phi$ equalizes $s$ and $t$.
\end{proof}

\noindent
There are several problems with considering the orbits $O_u \subset |U|$
as a tool for singling out properties of quotient maps. One issue is the
following. Suppose that $\Spec(k) \to B$
is a geometric point of $B$. Consider the canonical map
$$
U(k) \longrightarrow |U|.
$$
Then it is usually not the case that the equivalence classes
of the equivalence relation generated by $j(R(k)) \subset U(k) \times U(k)$
are the inverse images of the orbits $O_u \subset |U|$.
A silly example is to take $S = B = \Spec(\mathbf{Z})$,
$U = R = \Spec(k)$ with $s = t = \text{id}_k$. Then $|U| = |R|$ is
a single point but $U(k)/R(k)$ is enormous.
A more interesting example is to take $S = B = \Spec(\mathbf{Q})$,
choose some of number fields $K \subset L$, and set $U = \Spec(L)$
and $R = \Spec(L \otimes_K L)$ with obvious maps $s, t : R \to U$.
In this case $|U|$ still has just one point, but the quotient
$$
U(k)/R(k) = \Hom(K, k)
$$
consists of more than one element. We conclude from both examples
that if $U \to X$ is an $R$-invariant map and if we want it to
``separate orbits'' we get a much stronger and interesting notion by
considering the induced maps $U(k) \to X(k)$ and ask that
those maps separate orbits.

\medskip\noindent
There is an issue with this too. Namely, suppose that
$S = B = \Spec(\mathbf{R})$,
$U = \Spec(\mathbf{C})$, and
$R = \Spec(\mathbf{C}) \amalg \Spec(K)$
for some field extension $\sigma : \mathbf{C} \to K$.
Let the maps $s, t$ be given by the identity on the component
$\Spec(\mathbf{C})$, but by $\sigma, \sigma \circ \tau$ on the
second component where $\tau$ is complex conjugation. If
$K$ is a nontrivial extension of $\mathbf{C}$, then the two points
$1, \tau \in U(\mathbf{C})$ are not equivalent under
$j(R(\mathbf{C}))$. But after choosing an extension $\mathbf{C} \subset \Omega$
of sufficiently large cardinality (for example larger than the cardinality
of $K$) then the images of $1, \tau \in U(\mathbf{C})$ in
$U(\Omega)$ do become equivalent! It seems intuitively clear that
this happens either because $s, t : R \to U$ are not locally of finite type
or because the cardinality of the field $k$ is not large enough.

\medskip\noindent
Keeping this in mind we make the following definition.

\begin{definition}
\label{definition-geometric-orbits}
Let $S$ be a scheme, and let $B$ be an algebraic space over $S$.
Let $j : R \to U \times_B U$ be a pre-relation over $B$.
Let $\Spec(k) \to B$ be a geometric point of $B$.
\begin{enumerate}
\item We say $\overline{u}, \overline{u}' \in U(k)$ are
{\it weakly $R$-equivalent} if they are in the same equivalence class
for the equivalence relation generated by the relation
$j(R(k)) \subset U(k) \times U(k)$.
\item We say $\overline{u}, \overline{u}' \in U(k)$ are
{\it $R$-equivalent} if for some overfield $k \subset \Omega$
the images in $U(\Omega)$ are weakly $R$-equivalent.
\item The {\it weak orbit}, or more precisely the {\it weak $R$-orbit}
of $\overline{u} \in U(k)$ is set of all
elements of $U(k)$ which are weakly $R$-equivalent to $\overline{u}$.
\item The {\it orbit}, or more precisely the {\it $R$-orbit}
of $\overline{u} \in U(k)$ is set of all
elements of $U(k)$ which are $R$-equivalent to $\overline{u}$.
\end{enumerate}
\end{definition}

\noindent
It turns out that in good cases orbits and weak orbits agree, see
Lemma \ref{lemma-geometric-orbits}. The following lemma illustrates
the difference in the special case of a pre-equivalence relation.

\begin{lemma}
\label{lemma-weak-orbit-pre-equivalence}
Let $S$ be a scheme, and let $B$ be an algebraic space over $S$.
Let $\Spec(k) \to B$ be a geometric point of $B$.
Let $j : R \to U \times_B U$ be a pre-equivalence relation over $B$.
In this case the weak orbit of $\overline{u} \in U(k)$ is simply
$$
\{
\overline{u}' \in U(k)
\text{ such that }
\exists \overline{r} \in R(k),
\ s(\overline{r}) = \overline{u},
\ t(\overline{r}) = \overline{u}'
\}
$$
and the orbit of $\overline{u} \in U(k)$ is
$$
\{
\overline{u}' \in U(k) :
\exists\text{ field extension }K/k, \ \exists\ r \in R(K),
\ s(r) = \overline{u}, \ t(r) = \overline{u}'\}
$$
\end{lemma}

\begin{proof}
This is true because by definition of a pre-equivalence relation the image
$j(R(k)) \subset U(k) \times U(k)$ is an equivalence relation.
\end{proof}

\noindent
Let us describe the recipe for turning any pre-relation into a
pre-equivalence relation. We will use the morphisms
\begin{equation}
\label{equation-list}
\begin{matrix}
j_{diag} &
: &
U &
\longrightarrow &
U \times_B U, &
u &
\longmapsto &
(u, u) \\
j_{flip} &
: &
R &
\longrightarrow &
U \times_B U, &
r &
\longmapsto &
(s(r), t(r)) \\
j_{comp} &
: &
R \times_{s, U, t} R &
\longrightarrow &
U \times_B U, &
(r, r') &
\longmapsto &
(t(r), s(r'))
\end{matrix}
\end{equation}
We define $j_1 = (t_1, s_1) : R_1 \to U \times_B U$ to be the morphism
$$
j \amalg j_{diag} \amalg j_{flip} :
R \amalg U \amalg R
\longrightarrow
U \times_B U
$$
with notation as in
Equation (\ref{equation-list}).
For $n > 1$ we set
$$
j_n = (t_n, s_n) :
R_n = R_1 \times_{s_1, U, t_{n - 1}} R_{n - 1} \longrightarrow U \times_B U
$$
where $t_n$ comes from $t_1$ precomposed with projection onto $R_1$ and
$s_n$ comes from $s_{n - 1}$ precomposed with projection onto $R_{n - 1}$.
Finally, we denote
$$
j_\infty = (t_\infty, s_\infty) :
R_\infty = \coprod\nolimits_{n \geq 1} R_n
\longrightarrow
U \times_B U.
$$

\begin{lemma}
\label{lemma-make-pre-equivalence}
Let $S$ be a scheme, and let $B$ be an algebraic space over $S$.
Let $j : R \to U \times_B U$ be a pre-relation over $B$.
Then $j_\infty : R_\infty \to U \times_B U$ is a
pre-equivalence relation over $B$. Moreover
\begin{enumerate}
\item $\phi : U \to X$ is $R$-invariant if and only if it is
$R_\infty$-invariant,
\item the canonical map of quotient sheaves $U/R \to U/R_\infty$ (see
Groupoids in Spaces, Section \ref{spaces-groupoids-section-quotient-sheaves})
is an isomorphism,
\item weak $R$-orbits agree with weak $R_\infty$-orbits,
\item $R$-orbits agree with $R_\infty$-orbits,
\item if $s, t$ are locally of finite type, then $s_\infty$, $t_\infty$
are locally of finite type,
\item add more here as needed.
\end{enumerate}
\end{lemma}

\begin{proof}
Omitted. Hint for (5): Any property of $s, t$ which is stable under composition
and stable under base change, and Zariski local on the source
will be inherited by $s_\infty, t_\infty$.
\end{proof}

\begin{lemma}
\label{lemma-geometric-orbits}
Let $S$ be a scheme, and let $B$ be an algebraic space over $S$.
Let $j : R \to U \times_B U$ be a pre-relation over $B$.
Let $\Spec(k) \to B$ be a geometric point of $B$.
\begin{enumerate}
\item If $s, t : R \to U$ are locally of finite type
then weak $R$-equivalence on $U(k)$ agrees with $R$-equivalence, and
weak $R$-orbits agree with $R$-orbits on $U(k)$.
\item If $k$ has sufficiently large cardinality then weak $R$-equivalence
on $U(k)$ agrees with $R$-equivalence, and weak $R$-orbits agree
with $R$-orbits on $U(k)$.
\end{enumerate}
\end{lemma}

\begin{proof}
We first prove (1). Assume $s, t$ locally of finite type. By
Lemma \ref{lemma-make-pre-equivalence}
we may assume that $R$ is a pre-equivalence relation.
Let $k$ be an algebraically closed field over $B$.
Suppose $\overline{u}, \overline{u}' \in U(k)$ are $R$-equivalent.
Then for some extension field $\Omega/k$ there exists
a point $\overline{r} \in R(\Omega)$ mapping to
$(\overline{u}, \overline{u}') \in (U \times_B U)(\Omega)$, see
Lemma \ref{lemma-weak-orbit-pre-equivalence}.
Hence
$$
Z = R \times_{j, U \times_B U, (\overline{u}, \overline{u}')} \Spec(k)
$$
is nonempty. As $s$ is locally of finite type we see that
also $j$ is locally of finite type, see
Morphisms of Spaces, Lemma \ref{spaces-morphisms-lemma-permanence-finite-type}.
This implies $Z$ is a nonempty algebraic space locally of finite type
over the algebraically closed field $k$ (use
Morphisms of Spaces,
Lemma \ref{spaces-morphisms-lemma-base-change-finite-type}).
Thus $Z$ has a $k$-valued point, see
Morphisms of Spaces, Lemma
\ref{spaces-morphisms-lemma-locally-finite-type-surjective-geometric-points}.
Hence we conclude there exists a $\overline{r} \in R(k)$ with
$j(\overline{r}) = (\overline{u}, \overline{u}')$, and we conclude that
$\overline{u}, \overline{u}'$ are $R$-equivalent as desired.

\medskip\noindent
The proof of part (2) is the same, except that it uses
Morphisms of Spaces, Lemma
\ref{spaces-morphisms-lemma-large-enough}
instead of
Morphisms of Spaces, Lemma
\ref{spaces-morphisms-lemma-locally-finite-type-surjective-geometric-points}.
This shows that the assertion holds as soon as $|k| > \lambda(R)$ with
$\lambda(R)$ as introduced just above
Morphisms of Spaces, Lemma
\ref{spaces-morphisms-lemma-locally-finite-type-surjective-geometric-points}.
\end{proof}

\noindent
In the following definition we use the terminology ``$k$ is a field
over $B$'' to mean that $\Spec(k)$ comes equipped with a morphism
$\Spec(k) \to B$.

\begin{definition}
\label{definition-set-theoretically-invariant}
Let $S$ be a scheme, and let $B$ be an algebraic space over $S$.
Let $j : R \to U \times_B U$ be a pre-relation over $B$.
\begin{enumerate}
\item We say $\phi : U \to X$ is {\it set-theoretically $R$-invariant}
if and only if the map $U(k) \to X(k)$ equalizes the two maps
$s, t : R(k) \to U(k)$ for every algebraically closed field $k$
over $B$.
\item We say $\phi : U \to X$ {\it separates orbits}, or
{\it separates $R$-orbits} if it is set-theoretically $R$-invariant and
$\phi(\overline{u}) = \phi(\overline{u}')$ in $X(k)$ implies that
$\overline{u}, \overline{u}' \in U(k)$ are in the same orbit
for every algebraically closed field $k$ over $B$.
\end{enumerate}
\end{definition}

\noindent
In
Example \ref{example-not-invariant}
we show that being set-theoretically invariant is ``too weak'' a notion in
the category of algebraic spaces. A more geometric reformulation
of what it means to be set-theoretically invariant or to separate orbits is in
Lemma \ref{lemma-separates-orbits}.

\begin{lemma}
\label{lemma-set-theoretic-invariant}
In the situation of Definition \ref{definition-set-theoretically-invariant}.
A morphism $\phi : U \to X$ is set-theoretically $R$-invariant if and
only if for any algebraically closed field $k$ over $B$ the map
$U(k) \to X(k)$ is constant on orbits.
\end{lemma}

\begin{proof}
This is true because the condition is supposed to hold for all algebraically
closed fields over $B$.
\end{proof}

\begin{lemma}
\label{lemma-invariant-set-theoretically-invariant}
In the situation of Definition \ref{definition-set-theoretically-invariant}.
An invariant morphism is set-theoretically invariant.
\end{lemma}

\begin{proof}
This is immediate from the definitions.
\end{proof}

\begin{lemma}
\label{lemma-set-theoretically-invariant-invariant-when-reduced}
In the situation of Definition \ref{definition-set-theoretically-invariant}.
Let $\phi : U \to X$ be a morphism of algebraic spaces over $B$.
Assume
\begin{enumerate}
\item $\phi$ is set-theoretically $R$-invariant,
\item $R$ is reduced, and
\item $X$ is locally separated over $B$.
\end{enumerate}
Then $\phi$ is $R$-invariant.
\end{lemma}

\begin{proof}
Consider the equalizer
$$
Z = R \times_{(\phi, \phi) \circ j, X \times_B X, \Delta_{X/B}} X
$$
algebraic space. Then $Z \to R$ is an immersion by assumption (3).
By assumption (1) $|Z| \to |R|$ is surjective. This implies that
$Z \to R$ is a bijective closed immersion (use
Schemes, Lemma \ref{schemes-lemma-immersion-when-closed})
and by assumption (2) we conclude that $Z = R$.
\end{proof}

\begin{example}
\label{example-not-invariant}
There exist reduced quasi-separated algebraic spaces $X$, $Y$ and a pair of
morphisms $a, b : Y \to X$ which agree on all $k$-valued points but are not
equal. To get an example take $Y = \Spec(k[[x]])$ and
$$
X = \mathbf{A}^1_k \Big/ \big(\Delta \amalg \{(x, -x) \mid x \not = 0\}\big)
$$
the algebraic space of
Spaces, Example \ref{spaces-example-affine-line-involution}.
The two morphisms $a, b : Y \to X$
come from the two maps $x \mapsto x$ and $x \mapsto -x$
from $Y$ to $\mathbf{A}^1_k = \Spec(k[x])$. On the generic point
the two maps are the same because on the open part $x \not = 0$ of the
space $X$ the functions $x$ and $-x$ are equal. On the closed point
the maps are obviously the same. It is also true that $a \not = b$.
This implies that
Lemma \ref{lemma-set-theoretically-invariant-invariant-when-reduced}
does not hold with assumption (3) replaced by the assumption that $X$
be quasi-separated. Namely, consider the diagram
$$
\xymatrix{
Y \ar[d]_{-1} \ar[r]_1 & Y \ar[d]^a \\
Y \ar[r]^a & X
}
$$
then the composition $a \circ (-1) = b$. Hence we can set $R = Y$,
$U = Y$, $s = 1$, $t = -1$, $\phi = a$ to get an example of a set-theoretically
invariant morphism which is not invariant.
\end{example}

\noindent
The example above is instructive because the map $Y \to X$ even separates
orbits. It shows that in the category of algebraic spaces there are simply
too many set-theoretically invariant morphisms lying around. Next, let us
define what it means for $R$ to be a set-theoretic equivalence relation, while
remembering that we need to allow for field extensions to make this work
correctly.

\begin{definition}
\label{definition-set-theoretic-equivalence}
Let $S$ be a scheme, and let $B$ be an algebraic space over $S$.
Let $j : R \to U \times_B U$ be a pre-relation over $B$.
\begin{enumerate}
\item We say $j$ is a {\it set-theoretic pre-equivalence relation} if
for all algebraically closed fields $k$ over $B$ the relation
$\sim_R$ on $U(k)$ defined by
$$
\overline{u} \sim_R \overline{u}'
\Leftrightarrow
\begin{matrix}
\exists\text{ field extension }K/k, \ \exists\ r \in R(K), \\
s(r) = \overline{u}, \ t(r) = \overline{u}'
\end{matrix}
$$
is an equivalence relation.
\item We say $j$ is a {\it set-theoretic equivalence relation}
if $j$ is universally injective and a set-theoretic pre-equivalence
relation.
\end{enumerate}
\end{definition}

\noindent
Let us reformulate this in more geometric terms.

\begin{lemma}
\label{lemma-set-theoretic-pre-equivalence-geometric}
In the situation of Definition \ref{definition-set-theoretic-equivalence}.
The following are equivalent:
\begin{enumerate}
\item The morphism $j$ is a set-theoretic pre-equivalence relation.
\item The subset $j(|R|) \subset |U \times_B U|$ contains the image of
$|j'|$ for any of the morphisms $j'$ as in Equation (\ref{equation-list}).
\item For every algebraically closed field $k$ over $B$ of sufficiently large
cardinality the subset $j(R(k)) \subset U(k) \times U(k)$ is an equivalence
relation.
\end{enumerate}
If $s, t$ are locally of finite type these are also equivalent to
\begin{enumerate}
\item[(4)] For every algebraically closed field $k$ over $B$
the subset $j(R(k)) \subset U(k) \times U(k)$ is an equivalence relation.
\end{enumerate}
\end{lemma}

\begin{proof}
Assume (2). Let $k$ be an algebraically closed field over $B$.
We are going to show that $\sim_R$ is an equivalence relation.
Suppose that $\overline{u}_i : \Spec(k) \to U$, $i = 1, 2$
are $k$-valued points of $U$. Suppose that $(\overline{u}_1, \overline{u}_2)$
is the image of a $K$-valued point $r \in R(K)$. Consider the
solid commutative diagram
$$
\xymatrix{
\Spec(K') \ar@{..>}[r] \ar@{..>}[d]
&
\Spec(k) \ar[d]_{(\overline{u}_2, \overline{u}_1)} &
\Spec(K) \ar[d] \ar[l] \\
R \ar[r]^-j &
U \times_B U &
R \ar[l]_-{j_{flip}}
}
$$
We also denote $r \in |R|$ the image of $r$.
By assumption the image of $|j_{flip}|$ is contained in the image of
$|j|$, in other words there exists a $r' \in |R|$ such that
$|j|(r') = |j_{flip}|(r)$. But note that $(\overline{u}_2, \overline{u}_1)$
is in the equivalence class that defines $|j|(r')$ (by the commutativity
of the solid part of the diagram). This means there exists a field
extension $K'/k$ and a morphism $r' : \Spec(K) \to R$
(abusively denoted $r'$ as well) with
$j \circ r' = (\overline{u}_2, \overline{u}_1) \circ i$
where $i : \Spec(K') \to \Spec(K)$ is the obvious map.
In other words the dotted part of the diagram commutes.
This proves that $\sim_R$ is a symmetric relation on $U(k)$.
In the similar way, using that the image of $|j_{diag}|$ is contained
in the image of $|j|$ we see that $\sim_R$ is reflexive (details omitted).

\medskip\noindent
To show that $\sim_R$ is transitive assume given
$\overline{u}_i : \Spec(k) \to U$, $i = 1, 2, 3$
and field extensions $K_i/k$ and points
$r_i : \Spec(K_i) \to R$, $i = 1, 2$ such that
$j(r_1) = (\overline{u}_1, \overline{u}_2)$ and
$j(r_1) = (\overline{u}_2, \overline{u}_3)$. Then we may choose a
commutative diagram of fields
$$
\xymatrix{
K & K_2 \ar[l] \\
K_1 \ar[u] & k \ar[l] \ar[u]
}
$$
and we may think of $r_1, r_2 \in R(K)$. We consider the
commutative solid diagram
$$
\xymatrix{
\Spec(K') \ar@{..>}[r] \ar@{..>}[d]
&
\Spec(k) \ar[d]_{(\overline{u}_1, \overline{u}_3)} &
\Spec(K) \ar[d]^{(r_1, r_2)} \ar[l]
\\
R \ar[r]^-j &
U \times_B U &
R \times_{s, U, t} R \ar[l]_-{j_{comp}}
}
$$
By exactly the same reasoning as in the first part of the proof, but
this time using that $|j_{comp}|((r_1, r_2))$ is in the image of $|j|$,
we conclude that a field $K'$ and dotted arrows exist making the
diagram commute. This proves that $\sim_R$ is transitive and concludes
the proof that (2) implies (1).

\medskip\noindent
Assume (1) and let $k$ be an algebraically closed field over $B$ whose
cardinality is larger than $\lambda(R)$, see
Morphisms of Spaces, Lemma \ref{spaces-morphisms-lemma-large-enough}.
Suppose that $\overline{u} \sim_R \overline{u}'$ with
$\overline{u}, \overline{u}' \in U(k)$. By assumption there exists
a point in $|R|$ mapping to $(\overline{u}, \overline{u}') \in |U \times_B U|$.
Hence by
Morphisms of Spaces, Lemma \ref{spaces-morphisms-lemma-large-enough}
we conclude there exists an $\overline{r} \in R(k)$ with
$j(\overline{r}) = (\overline{u}, \overline{u}')$. In this way we see
that (1) implies (3).

\medskip\noindent
Assume (3). Let us show that
$\Im(|j_{comp}|) \subset \Im(|j|)$. Pick any point
$c \in |R \times_{s, U, t} R|$. We may represent this by a morphism
$\overline{c} : \Spec(k) \to R \times_{s, U, t} R$, with $k$ over $B$
having sufficiently large cardinality. By assumption we see that
$j_{comp}(\overline{c}) \in U(k) \times U(k) = (U \times_B U)(k)$
is also the image $j(\overline{r})$ for some $\overline{r} \in R(k)$.
Hence $j_{comp}(c) = j(r)$ in $|U \times_B U|$ as desired (with
$r \in |R|$ the equivalence class of $\overline{r}$). The same argument
shows also that $\Im(|j_{diag}|) \subset \Im(|j|)$ and
$\Im(|j_{flip}|) \subset \Im(|j|)$ (details omitted).
In this way we see that (3) implies (2). At this point we have
shown that (1), (2) and (3) are all equivalent.

\medskip\noindent
It is clear that (4) implies (3) (without any assumptions on $s$, $t$).
To finish the proof of the lemma we show that (1) implies (4) if $s, t$
are locally of finite type. Namely, let $k$ be an algebraically closed
field over $B$. Suppose that $\overline{u} \sim_R \overline{u}'$ with
$\overline{u}, \overline{u}' \in U(k)$. By assumption the algebraic space
$Z = R \times_{j, U \times_B U, (\overline{u}, \overline{u}')} \Spec(k)$
is nonempty. On the other hand, since $j = (t, s)$ is locally of finite type
the morphism $Z \to \Spec(k)$ is locally of finite type as well (use
Morphisms of Spaces, Lemmas \ref{spaces-morphisms-lemma-permanence-finite-type}
and \ref{spaces-morphisms-lemma-base-change-finite-type}).
Hence $Z$ has a $k$ point by
Morphisms of Spaces, Lemma
\ref{spaces-morphisms-lemma-locally-finite-type-surjective-geometric-points}
and we conclude that $(\overline{u}, \overline{u}') \in j(R(k))$
as desired. This finishes the proof of the lemma.
\end{proof}

\begin{lemma}
\label{lemma-set-theoretic-equivalence-geometric}
In the situation of Definition \ref{definition-set-theoretic-equivalence}.
The following are equivalent:
\begin{enumerate}
\item The morphism $j$ is a set-theoretic equivalence relation.
\item The morphism $j$ is universally injective and
$j(|R|) \subset |U \times_B U|$ contains the image of
$|j'|$ for any of the morphisms $j'$ as in Equation (\ref{equation-list}).
\item For every algebraically closed field $k$ over $B$ of sufficiently large
cardinality the map $j : R(k) \to U(k) \times U(k)$ is injective and
its image is an equivalence relation.
\end{enumerate}
If $j$ is decent, or locally separated, or quasi-separated
these are also equivalent to
\begin{enumerate}
\item[(4)] For every algebraically closed field $k$ over $B$
the map $j : R(k) \to U(k) \times U(k)$ is injective and its image
is an equivalence relation.
\end{enumerate}
\end{lemma}

\begin{proof}
The implications (1) $\Rightarrow$ (2) and (2) $\Rightarrow$ (3) follow from
Lemma \ref{lemma-set-theoretic-pre-equivalence-geometric}
and the definitions. The same lemma shows that (3) implies
$j$ is a set-theoretic pre-equivalence relation. But of course condition
(3) also implies that $j$ is universally injective, see
Morphisms of Spaces, Lemma \ref{spaces-morphisms-lemma-universally-injective},
so that $j$ is indeed a set-theoretic equivalence relation.
At this point we know that (1), (2), (3) are all equivalent.

\medskip\noindent
Condition (4) implies (3) without any further hypotheses on $j$. Assume $j$
is decent, or locally separated, or quasi-separated and the equivalent
conditions (1), (2), (3) hold. By
More on Morphisms of Spaces,
Lemma \ref{spaces-more-morphisms-lemma-when-universally-injective-radicial}
we see that $j$ is radicial.
Let $k$ be any algebraically closed field over $B$. Let
$\overline{u}, \overline{u}' \in U(k)$ with
$\overline{u} \sim_R \overline{u}'$. We see that
$R \times_{U \times_B U, (\overline{u}, \overline{u}')} \Spec(k)$
is nonempty. Hence, as $j$ is radicial, its reduction is the spectrum of a
field purely inseparable over $k$. As $k = \overline{k}$ we see that
it is the spectrum of $k$. Whence a point $\overline{r} \in R(k)$
with $t(\overline{r}) = \overline{u}$ and $s(\overline{r}) = \overline{u}'$
as desired.
\end{proof}

\begin{lemma}
\label{lemma-set-theoretic-equivalence}
Let $S$ be a scheme, and let $B$ be an algebraic space over $S$.
Let $j : R \to U \times_B U$ be a pre-relation over $B$.
\begin{enumerate}
\item If $j$ is a pre-equivalence relation, then $j$ is a
set-theoretic pre-equivalence relation. This holds in particular
when $j$ comes from a groupoid in algebraic spaces, or from an
action of a group algebraic space on $U$.
\item If $j$ is an equivalence relation, then $j$ is a
set-theoretic equivalence relation.
\end{enumerate}
\end{lemma}

\begin{proof}
Omitted.
\end{proof}

\begin{lemma}
\label{lemma-separates-orbits}
Let $B \to S$ be as in Section \ref{section-conventions-notation}.
Let $j : R \to U \times_B U$ be a pre-relation.
Let $\phi : U \to X$ be a morphism of algebraic spaces over $B$.
Consider the diagram
$$
\xymatrix{
(U \times_X U) \times_{(U \times_B U)} R \ar[d]^q \ar[r]_-p & R \ar[d]^j \\
U \times_X U \ar[r]^c & U \times_B U
}
$$
Then we have:
\begin{enumerate}
\item The morphism $\phi$ is set-theoretically invariant if and only
if $p$ is surjective.
\item If $j$ is a set-theoretic pre-equivalence relation then
$\phi$ separates orbits if and only if $p$ and $q$ are surjective.
\item If $p$ and $q$ are surjective, then $j$ is a set-theoretic
pre-equivalence relation (and $\phi$ separates orbits).
\item If $\phi$ is $R$-invariant and $j$ is a set-theoretic pre-equivalence
relation, then $\phi$ separates orbits if and only if the induced morphism
$R \to U \times_X U$ is surjective.
\end{enumerate}
\end{lemma}

\begin{proof}
Assume $\phi$ is set-theoretically invariant. This means that for any
algebraically closed field $k$ over $B$ and any $\overline{r} \in R(k)$
we have $\phi(s(\overline{r})) = \phi(t(\overline{r}))$. Hence
$((\phi(t(\overline{r})), \phi(s(\overline{r}))), \overline{r})$
defines a point in the fibre product mapping to $\overline{r}$ via
$p$. This shows that $p$ is surjective. Conversely, assume $p$ is
surjective. Pick $\overline{r} \in R(k)$. As $p$ is surjective, we
can find a field extension $K/k$ and a $K$-valued point
$\tilde r$ of the fibre product with $p(\tilde r) = \overline{r}$.
Then $q(\tilde r) \in U \times_X U$ maps to
$(t(\overline{r}), s(\overline{r}))$ in $U \times_B U$ and we conclude
that $\phi(s(\overline{r})) = \phi(t(\overline{r}))$. This proves
that $\phi$ is set-theoretically invariant.

\medskip\noindent
The proofs of (2), (3), and (4) are omitted. Hint: Assume $k$ is an
algebraically closed field over $B$ of large cardinality. Consider the
associated diagram of sets
$$
\xymatrix{
(U(k) \times_{X(k)} U(k)) \times_{U(k) \times U(k)} R(k) \ar[d]^q \ar[r]_-p &
R(k) \ar[d]^j \\
U(k) \times_{X(k)} U(k) \ar[r]^c & U(k) \times U(k)
}
$$
By the lemmas above the equivalences posed in (2), (3), and (4) become
set-theoretic questions related to the diagram we just displayed, using
that surjectivity translates into surjectivity on $k$-valued points by
Morphisms of Spaces, Lemma \ref{spaces-morphisms-lemma-large-enough}.
\end{proof}

\noindent
Because we have seen above that the notion of a set-theoretically
invariant morphism is a rather weak one in the category of algebraic
spaces, we define an orbit space for a pre-relation as follows.

\begin{definition}
\label{definition-orbit-space}
Let $B \to S$ as in Section \ref{section-conventions-notation}.
Let $j : R \to U \times_B U$ be a pre-relation.
We say $\phi : U \to X$ is an {\it orbit space for $R$} if
\begin{enumerate}
\item $\phi$ is $R$-invariant,
\item $\phi$ separates $R$-orbits, and
\item $\phi$ is surjective.
\end{enumerate}
\end{definition}

\noindent
The definition of separating $R$-orbits involves a discussion of
points with values in algebraically closed fields. But as we've seen
in many cases this just corresponds to the surjectivity of certain
canonically associated morphisms of algebraic spaces.
We summarize some of the discussion above in the following characterization
of orbit spaces.

\begin{lemma}
\label{lemma-orbit-space}
Let $B \to S$ as in Section \ref{section-conventions-notation}.
Let $j : R \to U \times_B U$ be a set-theoretic pre-equivalence
relation. A morphism $\phi : U \to X$ is an orbit space for $R$ if and only if
\begin{enumerate}
\item $\phi \circ s = \phi \circ t$, i.e., $\phi$ is invariant,
\item the induced morphism $(t, s) : R \to U \times_X U$ is surjective, and
\item the morphism $\phi : U \to X$ is surjective.
\end{enumerate}
This characterization applies for example if $j$ is a pre-equivalence relation,
or comes from a groupoid in algebraic spaces over $B$, or comes from the action
of a group algebraic space over $B$ on $U$.
\end{lemma}

\begin{proof}
Follows immediately from Lemma \ref{lemma-separates-orbits} part (4).
\end{proof}

\noindent
In the following lemma it is (probably) not good enough to assume just that
the morphisms $s, t$ are locally of finite type. The reason is that
it may happen that some map $\phi : U \to X$ is an orbit space, yet is
not locally of finite type. In that case $U(k) \to X(k)$ may not
be surjective for all algebraically closed fields $k$ over $B$.

\begin{lemma}
\label{lemma-orbit-space-locally-finite-type-over-base}
Let $B \to S$ as in Section \ref{section-conventions-notation}.
Let $j = (t, s) : R \to U \times_B U$ be a pre-relation.
Assume $R, U$ are locally of finite type over $B$.
Let $\phi : U \to X$ be an $R$-invariant morphism of algebraic spaces over $B$.
Then $\phi$ is an orbit space for $R$ if and only if the natural map
$$
U(k)/\big(\text{equivalence relation generated by }j(R(k))\big)
\longrightarrow
X(k)
$$
is bijective for all algebraically closed fields $k$ over $B$.
\end{lemma}

\begin{proof}
Note that since $U$, $R$ are locally of finite type over $B$ all of the
morphisms $s, t, j, \phi$ are locally of finite type, see
Morphisms of Spaces, Lemma \ref{spaces-morphisms-lemma-permanence-finite-type}.
We will also use without further mention
Morphisms of Spaces, Lemma
\ref{spaces-morphisms-lemma-locally-finite-type-surjective-geometric-points}.
Assume $\phi$ is an orbit space. Let $k$ be any algebraically closed
field over $B$. Let $\overline{x} \in X(k)$. Consider
$U \times_{\phi, X, \overline{x}} \Spec(k)$.
This is a nonempty algebraic space
which is locally of finite type over $k$. Hence it has a $k$-valued point.
This shows the displayed map of the lemma is surjective.
Suppose that $\overline{u}, \overline{u}' \in U(k)$ map to the same
element of $X(k)$. By
Definition \ref{definition-set-theoretically-invariant}
this means that $\overline{u}, \overline{u}'$ are in the same
$R$-orbit. By Lemma \ref{lemma-geometric-orbits} this means that
they are equivalent under the equivalence relation generated by
$j(R(k))$. Thus the displayed morphism is injective.

\medskip\noindent
Conversely, assume the displayed map is bijective for all algebraically
closed fields $k$ over $B$. This condition clearly implies that $\phi$
is surjective. We have already assumed that $\phi$ is $R$-invariant.
Finally, the injectivity of all the displayed maps implies that
$\phi$ separates orbits. Hence $\phi$ is an orbit space.
\end{proof}










\section{Coarse quotients}
\label{section-coarse}

\noindent
We only add this here so that we can later say that coarse quotients
correspond to coarse moduli spaces (or moduli schemes).

\begin{definition}
\label{definition-coarse}
Let $S$ be a scheme and $B$ an algebraic space over $S$.
Let $j : R \to U \times_B U$ be a pre-relation.
A morphism $\phi : U \to X$ of algebraic spaces over $B$
is called a {\it coarse quotient} if
\begin{enumerate}
\item $\phi$ is a categorical quotient, and
\item $\phi$ is an orbit space.
\end{enumerate}
If $S = B$, $U$, $R$ are all schemes, then we say a morphism of schemes
$\phi : U \to X$ is a {\it coarse quotient in schemes} if
\begin{enumerate}
\item $\phi$ is a categorical quotient in schemes, and
\item $\phi$ is an orbit space.
\end{enumerate}
\end{definition}

\noindent
In many situations the algebraic spaces $R$ and $U$ are locally of finite type
over $B$ and the orbit space condition simply means that
$$
U(k)/\big(\text{equivalence relation generated by }j(R(k))\big)
\cong
X(k)
$$
for all algebraically closed fields $k$. See
Lemma \ref{lemma-orbit-space-locally-finite-type-over-base}.
If $j$ is also a (set-theoretic) pre-equivalence relation, then the condition
is simply equivalent to $U(k)/j(R(k)) \to X(k)$ being bijective for all
algebraically closed fields $k$.









\section{Topological properties}
\label{section-topological}

\noindent
Let $S$ be a scheme and $B$ an algebraic space over $S$.
Let $j : R \to U \times_B U$ be a pre-relation.
We say a subset $T \subset |U|$ is {\it $R$-invariant} if
$s^{-1}(T) = t^{-1}(T)$ as subsets of $|R|$.
Note that if $T$ is closed, then it may not be the case that
the corresponding reduced closed subspace of $U$ is $R$-invariant
(as in
Groupoids in Spaces, Definition
\ref{spaces-groupoids-definition-invariant-open})
because the pullbacks $s^{-1}(T)$, $t^{-1}(T)$ may not be reduced.
Here are some conditions that we can consider for an
invariant morphism $\phi : U \to X$.

\begin{definition}
\label{definition-topological}
Let $S$ be a scheme and $B$ an algebraic space over $S$.
Let $j : R \to U \times_B U$ be a pre-relation.
Let $\phi : U \to X$ be an $R$-invariant morphism of algebraic spaces over $B$.
\begin{enumerate}
\item
\label{item-submersive}
The morphism $\phi$ is submersive.
\item
\label{item-invariant-closed}
For any $R$-invariant closed subset $Z \subset |U|$ the image
$\phi(Z)$ is closed in $|X|$.
\item
\label{item-intersect-invariant-closed}
Condition (\ref{item-invariant-closed}) holds and for any pair of
$R$-invariant closed subsets $Z_1, Z_2 \subset |U|$ we have
$$
\phi(Z_1 \cap Z_2) = \phi(Z_1) \cap \phi(Z_2)
$$
\item The morphism $(t, s) : R \to U \times_X U$ is universally submersive.
\label{item-strong}
\end{enumerate}
For each of these properties we can also require them to hold after any
flat base change, or after any base change, see
Definition \ref{definition-base-change}. In this case we say condition
(\ref{item-submersive}),
(\ref{item-invariant-closed}),
(\ref{item-intersect-invariant-closed}), or
(\ref{item-strong}) holds {\it uniformly} or {\it universally}.
\end{definition}








\section{Invariant functions}
\label{section-functions}

\noindent
In some cases it is convenient to pin down the structure sheaf
of a quotient by requiring any invariant function to be a local
section of the structure sheaf of the quotient.

\begin{definition}
\label{definition-functions}
Let $S$ be a scheme and $B$ an algebraic space over $S$.
Let $j : R \to U \times_B U$ be a pre-relation.
Let $\phi : U \to X$ be an $R$-invariant morphism.
Denote $\phi' = \phi \circ s = \phi \circ t : R \to X$.
\begin{enumerate}
\item We denote $(\phi_*\mathcal{O}_U)^R$ the $\mathcal{O}_X$-sub-algebra
of $\phi_*\mathcal{O}_U$ which is the equalizer of the two maps
$$
\xymatrix{
\phi_*\mathcal{O}_U
\ar@<1ex>[rr]^{\phi_*s^\sharp}
\ar@<-1ex>[rr]_{\phi_*t^\sharp}
& &
\phi'_*\mathcal{O}_R
}
$$
on $X_\etale$. We sometimes call this the
{\it sheaf of $R$-invariant functions on $X$}.
\item We say {\it the functions on $X$ are the $R$-invariant functions on
$U$} if the natural map $\mathcal{O}_X \to (\phi_*\mathcal{O}_U)^R$
is an isomorphism.
\end{enumerate}
\end{definition}

\noindent
Of course we can require this property holds after any
(flat or any) base change, leading to a (uniform or) universal notion.
This condition is often thrown in
with other conditions in order to obtain a (more) unique quotient. And of
course a good deal of motivation for the whole subject comes from the following
special case: $U = \Spec(A)$ is an affine scheme over a field
$S = B = \Spec(k)$ and where $R = G \times U$, with
$G$ an affine group scheme over $k$. In this case
you have the option of taking for the quotient:
$$
X = \Spec(A^G)
$$
so that at least the condition of the definition above is satisfied.
Even though this is a nice thing you can do it is often not the right
quotient; for example if $U = \text{GL}_{n, k}$ and $G$ is the group of
upper triangular matrices, then the above gives $X = \Spec(k)$, whereas
a much better quotient (namely the flag variety) exists.








\section{Good quotients}
\label{section-good}

\noindent
Especially when taking quotients by group actions the following definition
is useful.

\begin{definition}
\label{definition-good}
Let $S$ be a scheme and $B$ an algebraic space over $S$.
Let $j : R \to U \times_B U$ be a pre-relation.
A morphism $\phi : U \to X$ of algebraic spaces over $B$
is called a {\it good quotient} if
\begin{enumerate}
\item $\phi$ is invariant,
\item $\phi$ is affine,
\item $\phi$ is surjective,
\item condition (\ref{item-intersect-invariant-closed}) holds universally, and
\item the functions on $X$ are the $R$-invariant functions on $U$.
\end{enumerate}
\end{definition}

\noindent
In \cite{seshadri_quotients} Seshadri gives almost the same definition,
except that instead of (4) he simply requires the condition
(\ref{item-intersect-invariant-closed}) to hold -- he does not require
it to hold universally.







\section{Geometric quotients}
\label{section-geometric}

\noindent
This is Mumford's definition of a geometric quotient (at least the definition
from the first edition of GIT; as far as we can tell later editions
changed ``universally submersive'' to ``submersive'').

\begin{definition}
\label{definition-geometric}
Let $S$ be a scheme and $B$ an algebraic space over $S$.
Let $j : R \to U \times_B U$ be a pre-relation.
A morphism $\phi : U \to X$ of algebraic spaces over $B$
is called a {\it geometric quotient} if
\begin{enumerate}
\item $\phi$ is an orbit space,
\item condition (\ref{item-submersive}) holds universally, i.e.,
$\phi$ is universally submersive, and
\item the functions on $X$ are the $R$-invariant functions on $U$.
\end{enumerate}
\end{definition}










\begin{multicols}{2}[\section{Other chapters}]
\noindent
Preliminaries
\begin{enumerate}
\item \hyperref[introduction-section-phantom]{Introduction}
\item \hyperref[conventions-section-phantom]{Conventions}
\item \hyperref[sets-section-phantom]{Set Theory}
\item \hyperref[categories-section-phantom]{Categories}
\item \hyperref[topology-section-phantom]{Topology}
\item \hyperref[sheaves-section-phantom]{Sheaves on Spaces}
\item \hyperref[sites-section-phantom]{Sites and Sheaves}
\item \hyperref[stacks-section-phantom]{Stacks}
\item \hyperref[fields-section-phantom]{Fields}
\item \hyperref[algebra-section-phantom]{Commutative Algebra}
\item \hyperref[brauer-section-phantom]{Brauer Groups}
\item \hyperref[homology-section-phantom]{Homological Algebra}
\item \hyperref[derived-section-phantom]{Derived Categories}
\item \hyperref[simplicial-section-phantom]{Simplicial Methods}
\item \hyperref[more-algebra-section-phantom]{More on Algebra}
\item \hyperref[smoothing-section-phantom]{Smoothing Ring Maps}
\item \hyperref[modules-section-phantom]{Sheaves of Modules}
\item \hyperref[sites-modules-section-phantom]{Modules on Sites}
\item \hyperref[injectives-section-phantom]{Injectives}
\item \hyperref[cohomology-section-phantom]{Cohomology of Sheaves}
\item \hyperref[sites-cohomology-section-phantom]{Cohomology on Sites}
\item \hyperref[dga-section-phantom]{Differential Graded Algebra}
\item \hyperref[dpa-section-phantom]{Divided Power Algebra}
\item \hyperref[sdga-section-phantom]{Differential Graded Sheaves}
\item \hyperref[hypercovering-section-phantom]{Hypercoverings}
\end{enumerate}
Schemes
\begin{enumerate}
\setcounter{enumi}{25}
\item \hyperref[schemes-section-phantom]{Schemes}
\item \hyperref[constructions-section-phantom]{Constructions of Schemes}
\item \hyperref[properties-section-phantom]{Properties of Schemes}
\item \hyperref[morphisms-section-phantom]{Morphisms of Schemes}
\item \hyperref[coherent-section-phantom]{Cohomology of Schemes}
\item \hyperref[divisors-section-phantom]{Divisors}
\item \hyperref[limits-section-phantom]{Limits of Schemes}
\item \hyperref[varieties-section-phantom]{Varieties}
\item \hyperref[topologies-section-phantom]{Topologies on Schemes}
\item \hyperref[descent-section-phantom]{Descent}
\item \hyperref[perfect-section-phantom]{Derived Categories of Schemes}
\item \hyperref[more-morphisms-section-phantom]{More on Morphisms}
\item \hyperref[flat-section-phantom]{More on Flatness}
\item \hyperref[groupoids-section-phantom]{Groupoid Schemes}
\item \hyperref[more-groupoids-section-phantom]{More on Groupoid Schemes}
\item \hyperref[etale-section-phantom]{\'Etale Morphisms of Schemes}
\end{enumerate}
Topics in Scheme Theory
\begin{enumerate}
\setcounter{enumi}{41}
\item \hyperref[chow-section-phantom]{Chow Homology}
\item \hyperref[intersection-section-phantom]{Intersection Theory}
\item \hyperref[pic-section-phantom]{Picard Schemes of Curves}
\item \hyperref[weil-section-phantom]{Weil Cohomology Theories}
\item \hyperref[adequate-section-phantom]{Adequate Modules}
\item \hyperref[dualizing-section-phantom]{Dualizing Complexes}
\item \hyperref[duality-section-phantom]{Duality for Schemes}
\item \hyperref[discriminant-section-phantom]{Discriminants and Differents}
\item \hyperref[derham-section-phantom]{de Rham Cohomology}
\item \hyperref[local-cohomology-section-phantom]{Local Cohomology}
\item \hyperref[algebraization-section-phantom]{Algebraic and Formal Geometry}
\item \hyperref[curves-section-phantom]{Algebraic Curves}
\item \hyperref[resolve-section-phantom]{Resolution of Surfaces}
\item \hyperref[models-section-phantom]{Semistable Reduction}
\item \hyperref[functors-section-phantom]{Functors and Morphisms}
\item \hyperref[equiv-section-phantom]{Derived Categories of Varieties}
\item \hyperref[pione-section-phantom]{Fundamental Groups of Schemes}
\item \hyperref[etale-cohomology-section-phantom]{\'Etale Cohomology}
\item \hyperref[crystalline-section-phantom]{Crystalline Cohomology}
\item \hyperref[proetale-section-phantom]{Pro-\'etale Cohomology}
\item \hyperref[relative-cycles-section-phantom]{Relative Cycles}
\item \hyperref[more-etale-section-phantom]{More \'Etale Cohomology}
\item \hyperref[trace-section-phantom]{The Trace Formula}
\end{enumerate}
Algebraic Spaces
\begin{enumerate}
\setcounter{enumi}{64}
\item \hyperref[spaces-section-phantom]{Algebraic Spaces}
\item \hyperref[spaces-properties-section-phantom]{Properties of Algebraic Spaces}
\item \hyperref[spaces-morphisms-section-phantom]{Morphisms of Algebraic Spaces}
\item \hyperref[decent-spaces-section-phantom]{Decent Algebraic Spaces}
\item \hyperref[spaces-cohomology-section-phantom]{Cohomology of Algebraic Spaces}
\item \hyperref[spaces-limits-section-phantom]{Limits of Algebraic Spaces}
\item \hyperref[spaces-divisors-section-phantom]{Divisors on Algebraic Spaces}
\item \hyperref[spaces-over-fields-section-phantom]{Algebraic Spaces over Fields}
\item \hyperref[spaces-topologies-section-phantom]{Topologies on Algebraic Spaces}
\item \hyperref[spaces-descent-section-phantom]{Descent and Algebraic Spaces}
\item \hyperref[spaces-perfect-section-phantom]{Derived Categories of Spaces}
\item \hyperref[spaces-more-morphisms-section-phantom]{More on Morphisms of Spaces}
\item \hyperref[spaces-flat-section-phantom]{Flatness on Algebraic Spaces}
\item \hyperref[spaces-groupoids-section-phantom]{Groupoids in Algebraic Spaces}
\item \hyperref[spaces-more-groupoids-section-phantom]{More on Groupoids in Spaces}
\item \hyperref[bootstrap-section-phantom]{Bootstrap}
\item \hyperref[spaces-pushouts-section-phantom]{Pushouts of Algebraic Spaces}
\end{enumerate}
Topics in Geometry
\begin{enumerate}
\setcounter{enumi}{81}
\item \hyperref[spaces-chow-section-phantom]{Chow Groups of Spaces}
\item \hyperref[groupoids-quotients-section-phantom]{Quotients of Groupoids}
\item \hyperref[spaces-more-cohomology-section-phantom]{More on Cohomology of Spaces}
\item \hyperref[spaces-simplicial-section-phantom]{Simplicial Spaces}
\item \hyperref[spaces-duality-section-phantom]{Duality for Spaces}
\item \hyperref[formal-spaces-section-phantom]{Formal Algebraic Spaces}
\item \hyperref[restricted-section-phantom]{Algebraization of Formal Spaces}
\item \hyperref[spaces-resolve-section-phantom]{Resolution of Surfaces Revisited}
\end{enumerate}
Deformation Theory
\begin{enumerate}
\setcounter{enumi}{89}
\item \hyperref[formal-defos-section-phantom]{Formal Deformation Theory}
\item \hyperref[defos-section-phantom]{Deformation Theory}
\item \hyperref[cotangent-section-phantom]{The Cotangent Complex}
\item \hyperref[examples-defos-section-phantom]{Deformation Problems}
\end{enumerate}
Algebraic Stacks
\begin{enumerate}
\setcounter{enumi}{93}
\item \hyperref[algebraic-section-phantom]{Algebraic Stacks}
\item \hyperref[examples-stacks-section-phantom]{Examples of Stacks}
\item \hyperref[stacks-sheaves-section-phantom]{Sheaves on Algebraic Stacks}
\item \hyperref[criteria-section-phantom]{Criteria for Representability}
\item \hyperref[artin-section-phantom]{Artin's Axioms}
\item \hyperref[quot-section-phantom]{Quot and Hilbert Spaces}
\item \hyperref[stacks-properties-section-phantom]{Properties of Algebraic Stacks}
\item \hyperref[stacks-morphisms-section-phantom]{Morphisms of Algebraic Stacks}
\item \hyperref[stacks-limits-section-phantom]{Limits of Algebraic Stacks}
\item \hyperref[stacks-cohomology-section-phantom]{Cohomology of Algebraic Stacks}
\item \hyperref[stacks-perfect-section-phantom]{Derived Categories of Stacks}
\item \hyperref[stacks-introduction-section-phantom]{Introducing Algebraic Stacks}
\item \hyperref[stacks-more-morphisms-section-phantom]{More on Morphisms of Stacks}
\item \hyperref[stacks-geometry-section-phantom]{The Geometry of Stacks}
\end{enumerate}
Topics in Moduli Theory
\begin{enumerate}
\setcounter{enumi}{107}
\item \hyperref[moduli-section-phantom]{Moduli Stacks}
\item \hyperref[moduli-curves-section-phantom]{Moduli of Curves}
\end{enumerate}
Miscellany
\begin{enumerate}
\setcounter{enumi}{109}
\item \hyperref[examples-section-phantom]{Examples}
\item \hyperref[exercises-section-phantom]{Exercises}
\item \hyperref[guide-section-phantom]{Guide to Literature}
\item \hyperref[desirables-section-phantom]{Desirables}
\item \hyperref[coding-section-phantom]{Coding Style}
\item \hyperref[obsolete-section-phantom]{Obsolete}
\item \hyperref[fdl-section-phantom]{GNU Free Documentation License}
\item \hyperref[index-section-phantom]{Auto Generated Index}
\end{enumerate}
\end{multicols}


\bibliography{my}
\bibliographystyle{amsalpha}

\end{document}
