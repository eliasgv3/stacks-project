\IfFileExists{stacks-project.cls}{%
\documentclass{stacks-project}
}{%
\documentclass{amsart}
}

% For dealing with references we use the comment environment
\usepackage{verbatim}
\newenvironment{reference}{\comment}{\endcomment}
%\newenvironment{reference}{}{}
\newenvironment{slogan}{\comment}{\endcomment}
\newenvironment{history}{\comment}{\endcomment}

% For commutative diagrams we use Xy-pic
\usepackage[all]{xy}

% We use 2cell for 2-commutative diagrams.
\xyoption{2cell}
\UseAllTwocells

% We use multicol for the list of chapters between chapters
\usepackage{multicol}

% This is generall recommended for better output
\usepackage{lmodern}
\usepackage[T1]{fontenc}

% For cross-file-references
\usepackage{xr-hyper}

% Package for hypertext links:
\usepackage{hyperref}

% For any local file, say "hello.tex" you want to link to please
% use \externaldocument[hello-]{hello}
\externaldocument[introduction-]{introduction}
\externaldocument[conventions-]{conventions}
\externaldocument[sets-]{sets}
\externaldocument[categories-]{categories}
\externaldocument[topology-]{topology}
\externaldocument[sheaves-]{sheaves}
\externaldocument[sites-]{sites}
\externaldocument[stacks-]{stacks}
\externaldocument[fields-]{fields}
\externaldocument[algebra-]{algebra}
\externaldocument[brauer-]{brauer}
\externaldocument[homology-]{homology}
\externaldocument[derived-]{derived}
\externaldocument[simplicial-]{simplicial}
\externaldocument[more-algebra-]{more-algebra}
\externaldocument[smoothing-]{smoothing}
\externaldocument[modules-]{modules}
\externaldocument[sites-modules-]{sites-modules}
\externaldocument[injectives-]{injectives}
\externaldocument[cohomology-]{cohomology}
\externaldocument[sites-cohomology-]{sites-cohomology}
\externaldocument[dga-]{dga}
\externaldocument[dpa-]{dpa}
\externaldocument[sdga-]{sdga}
\externaldocument[hypercovering-]{hypercovering}
\externaldocument[schemes-]{schemes}
\externaldocument[constructions-]{constructions}
\externaldocument[properties-]{properties}
\externaldocument[morphisms-]{morphisms}
\externaldocument[coherent-]{coherent}
\externaldocument[divisors-]{divisors}
\externaldocument[limits-]{limits}
\externaldocument[varieties-]{varieties}
\externaldocument[topologies-]{topologies}
\externaldocument[descent-]{descent}
\externaldocument[perfect-]{perfect}
\externaldocument[more-morphisms-]{more-morphisms}
\externaldocument[flat-]{flat}
\externaldocument[groupoids-]{groupoids}
\externaldocument[more-groupoids-]{more-groupoids}
\externaldocument[etale-]{etale}
\externaldocument[chow-]{chow}
\externaldocument[intersection-]{intersection}
\externaldocument[pic-]{pic}
\externaldocument[weil-]{weil}
\externaldocument[adequate-]{adequate}
\externaldocument[dualizing-]{dualizing}
\externaldocument[duality-]{duality}
\externaldocument[discriminant-]{discriminant}
\externaldocument[derham-]{derham}
\externaldocument[local-cohomology-]{local-cohomology}
\externaldocument[algebraization-]{algebraization}
\externaldocument[curves-]{curves}
\externaldocument[resolve-]{resolve}
\externaldocument[models-]{models}
\externaldocument[functors-]{functors}
\externaldocument[equiv-]{equiv}
\externaldocument[pione-]{pione}
\externaldocument[etale-cohomology-]{etale-cohomology}
\externaldocument[proetale-]{proetale}
\externaldocument[relative-cycles-]{relative-cycles}
\externaldocument[more-etale-]{more-etale}
\externaldocument[trace-]{trace}
\externaldocument[crystalline-]{crystalline}
\externaldocument[spaces-]{spaces}
\externaldocument[spaces-properties-]{spaces-properties}
\externaldocument[spaces-morphisms-]{spaces-morphisms}
\externaldocument[decent-spaces-]{decent-spaces}
\externaldocument[spaces-cohomology-]{spaces-cohomology}
\externaldocument[spaces-limits-]{spaces-limits}
\externaldocument[spaces-divisors-]{spaces-divisors}
\externaldocument[spaces-over-fields-]{spaces-over-fields}
\externaldocument[spaces-topologies-]{spaces-topologies}
\externaldocument[spaces-descent-]{spaces-descent}
\externaldocument[spaces-perfect-]{spaces-perfect}
\externaldocument[spaces-more-morphisms-]{spaces-more-morphisms}
\externaldocument[spaces-flat-]{spaces-flat}
\externaldocument[spaces-groupoids-]{spaces-groupoids}
\externaldocument[spaces-more-groupoids-]{spaces-more-groupoids}
\externaldocument[bootstrap-]{bootstrap}
\externaldocument[spaces-pushouts-]{spaces-pushouts}
\externaldocument[spaces-chow-]{spaces-chow}
\externaldocument[groupoids-quotients-]{groupoids-quotients}
\externaldocument[spaces-more-cohomology-]{spaces-more-cohomology}
\externaldocument[spaces-simplicial-]{spaces-simplicial}
\externaldocument[spaces-duality-]{spaces-duality}
\externaldocument[formal-spaces-]{formal-spaces}
\externaldocument[restricted-]{restricted}
\externaldocument[spaces-resolve-]{spaces-resolve}
\externaldocument[formal-defos-]{formal-defos}
\externaldocument[defos-]{defos}
\externaldocument[cotangent-]{cotangent}
\externaldocument[examples-defos-]{examples-defos}
\externaldocument[algebraic-]{algebraic}
\externaldocument[examples-stacks-]{examples-stacks}
\externaldocument[stacks-sheaves-]{stacks-sheaves}
\externaldocument[criteria-]{criteria}
\externaldocument[artin-]{artin}
\externaldocument[quot-]{quot}
\externaldocument[stacks-properties-]{stacks-properties}
\externaldocument[stacks-morphisms-]{stacks-morphisms}
\externaldocument[stacks-limits-]{stacks-limits}
\externaldocument[stacks-cohomology-]{stacks-cohomology}
\externaldocument[stacks-perfect-]{stacks-perfect}
\externaldocument[stacks-introduction-]{stacks-introduction}
\externaldocument[stacks-more-morphisms-]{stacks-more-morphisms}
\externaldocument[stacks-geometry-]{stacks-geometry}
\externaldocument[moduli-]{moduli}
\externaldocument[moduli-curves-]{moduli-curves}
\externaldocument[examples-]{examples}
\externaldocument[exercises-]{exercises}
\externaldocument[guide-]{guide}
\externaldocument[desirables-]{desirables}
\externaldocument[coding-]{coding}
\externaldocument[obsolete-]{obsolete}
\externaldocument[fdl-]{fdl}
\externaldocument[index-]{index}

% Theorem environments.
%
\theoremstyle{plain}
\newtheorem{theorem}[subsection]{Theorem}
\newtheorem{proposition}[subsection]{Proposition}
\newtheorem{lemma}[subsection]{Lemma}

\theoremstyle{definition}
\newtheorem{definition}[subsection]{Definition}
\newtheorem{example}[subsection]{Example}
\newtheorem{exercise}[subsection]{Exercise}
\newtheorem{situation}[subsection]{Situation}

\theoremstyle{remark}
\newtheorem{remark}[subsection]{Remark}
\newtheorem{remarks}[subsection]{Remarks}

\numberwithin{equation}{subsection}

% Macros
%
\def\lim{\mathop{\mathrm{lim}}\nolimits}
\def\colim{\mathop{\mathrm{colim}}\nolimits}
\def\Spec{\mathop{\mathrm{Spec}}}
\def\Hom{\mathop{\mathrm{Hom}}\nolimits}
\def\Ext{\mathop{\mathrm{Ext}}\nolimits}
\def\SheafHom{\mathop{\mathcal{H}\!\mathit{om}}\nolimits}
\def\SheafExt{\mathop{\mathcal{E}\!\mathit{xt}}\nolimits}
\def\Sch{\mathit{Sch}}
\def\Mor{\mathop{\mathrm{Mor}}\nolimits}
\def\Ob{\mathop{\mathrm{Ob}}\nolimits}
\def\Sh{\mathop{\mathit{Sh}}\nolimits}
\def\NL{\mathop{N\!L}\nolimits}
\def\CH{\mathop{\mathrm{CH}}\nolimits}
\def\proetale{{pro\text{-}\acute{e}tale}}
\def\etale{{\acute{e}tale}}
\def\QCoh{\mathit{QCoh}}
\def\Ker{\mathop{\mathrm{Ker}}}
\def\Im{\mathop{\mathrm{Im}}}
\def\Coker{\mathop{\mathrm{Coker}}}
\def\Coim{\mathop{\mathrm{Coim}}}

% Boxtimes
%
\DeclareMathSymbol{\boxtimes}{\mathbin}{AMSa}{"02}

%
% Macros for moduli stacks/spaces
%
\def\QCohstack{\mathcal{QC}\!\mathit{oh}}
\def\Cohstack{\mathcal{C}\!\mathit{oh}}
\def\Spacesstack{\mathcal{S}\!\mathit{paces}}
\def\Quotfunctor{\mathrm{Quot}}
\def\Hilbfunctor{\mathrm{Hilb}}
\def\Curvesstack{\mathcal{C}\!\mathit{urves}}
\def\Polarizedstack{\mathcal{P}\!\mathit{olarized}}
\def\Complexesstack{\mathcal{C}\!\mathit{omplexes}}
% \Pic is the operator that assigns to X its picard group, usage \Pic(X)
% \Picardstack_{X/B} denotes the Picard stack of X over B
% \Picardfunctor_{X/B} denotes the Picard functor of X over B
\def\Pic{\mathop{\mathrm{Pic}}\nolimits}
\def\Picardstack{\mathcal{P}\!\mathit{ic}}
\def\Picardfunctor{\mathrm{Pic}}
\def\Deformationcategory{\mathcal{D}\!\mathit{ef}}


% OK, start here.
%
\begin{document}

\title{Cohomology of Algebraic Stacks}

\maketitle

\phantomsection
\label{section-phantom}

\tableofcontents




\section{Introduction}
\label{section-introduction}

\noindent
In this chapter we write about cohomology of algebraic stacks.
This means in particular cohomology of quasi-coherent sheaves, i.e.,
we prove analogues of the results in the chapters entitled
``Cohomology of Schemes'' and ``Cohomology of Algebraic Spaces''.
The results in this chapter are different
from those in \cite{LM-B} mainly because we consistently use the
``big sites''. Before reading this chapter please take a quick look at
the chapter ``Sheaves on Algebraic Stacks'' in order to become
familiar with the terminology introduced there, see
Sheaves on Stacks, Section \ref{stacks-sheaves-section-introduction}.



\section{Conventions and abuse of language}
\label{section-conventions}

\noindent
We continue to use the conventions and the abuse of language
introduced in
Properties of Stacks, Section \ref{stacks-properties-section-conventions}.











\section{Notation}
\label{section-notation}

\noindent
Different topologies. If we indicate an algebraic stack by a calligraphic
letter, such as $\mathcal{X}, \mathcal{Y}, \mathcal{Z}$, then the notation
$\mathcal{X}_{Zar}, \mathcal{X}_\etale, \mathcal{X}_{smooth},
\mathcal{X}_{syntomic}, \mathcal{X}_{fppf}$ indicates the site introduced
in
Sheaves on Stacks, Definition
\ref{stacks-sheaves-definition-inherited-topologies}.
(Think ``big site''.) Correspondingly the structure sheaf of
$\mathcal{X}$ is a sheaf on $\mathcal{X}_{fppf}$.
On the other hand, algebraic spaces and schemes
are usually indicated by roman capitals, such as $X, Y, Z$, and in this case
$X_\etale$ indicates the small \'etale site of $X$ (as
defined in
Topologies, Definition
\ref{topologies-definition-big-small-etale}
or
Properties of Spaces, Definition
\ref{spaces-properties-definition-etale-site}).
It seems that the distinction should be clear enough.

\medskip\noindent
The default topology is the fppf topology. Hence we will sometimes
say ``sheaf on $\mathcal{X}$'' or ``sheaf of $\mathcal{O}_\mathcal{X}$-modules''
when we mean sheaf on $\mathcal{X}_{fppf}$ or object of
$\textit{Mod}(\mathcal{X}_{fppf}, \mathcal{O}_\mathcal{X})$.

\medskip\noindent
If $f : \mathcal{X} \to \mathcal{Y}$ is a morphism of algebraic
stacks, then the functors $f_*$ and $f^{-1}$ defined on presheaves
preserves sheaves for any of the topologies mentioned above. In particular
when we discuss the pushforward or pullback of a sheaf we don't have to
mention which topology we are working with. The same isn't true
when we compute cohomology groups and/or higher direct images. In this
case we will always mention which topology we are working with.

\medskip\noindent
Suppose that $f : X \to \mathcal{Y}$ is a morphism from an algebraic
space $X$ to an algebraic stack $\mathcal{Y}$. Let $\mathcal{G}$ be
a sheaf on $\mathcal{Y}_\tau$ for some topology $\tau$. In this case
$f^{-1}\mathcal{G}$ is a sheaf for the $\tau$ topology on $\mathcal{S}_X$
(the algebraic stack associated to $X$) because (by our conventions) $f$
really is a $1$-morphism $f : \mathcal{S}_X \to \mathcal{Y}$.
If $\tau = \etale$ or stronger, then we write
$f^{-1}\mathcal{G}|_{X_\etale}$
to denote the restriction to the \'etale site of $X$, see
Sheaves on Stacks, Section \ref{stacks-sheaves-section-compare}.
If $\mathcal{G}$ is an $\mathcal{O}_\mathcal{X}$-module we sometimes
write $f^*\mathcal{G}$ and $f^*\mathcal{G}|_{X_\etale}$
instead.




\section{Pullback of quasi-coherent modules}
\label{section-pullback}

\noindent
Let $f : \mathcal{X} \to \mathcal{Y}$ be a morphism of algebraic stacks.
It is a very general fact that quasi-coherent modules on ringed topoi
are compatible with pullbacks. In particular the pullback $f^*$ preserves
quasi-coherent modules and we obtain a functor
$$
f^* :
\QCoh(\mathcal{O}_\mathcal{Y})
\longrightarrow
\QCoh(\mathcal{O}_\mathcal{X}),
$$
see Sheaves on Stacks, Lemma
\ref{stacks-sheaves-lemma-pullback-quasi-coherent}.
In general this functor isn't exact, but if $f$ is flat then it is.

\begin{lemma}
\label{lemma-flat-pullback-quasi-coherent}
If $f : \mathcal{X} \to \mathcal{Y}$ is a flat morphism of algebraic stacks
then $f^* : \QCoh(\mathcal{O}_\mathcal{Y}) \to
\QCoh(\mathcal{O}_\mathcal{X})$ is an exact functor.
\end{lemma}

\begin{proof}
Choose a scheme $V$ and a surjective smooth morphism $V \to \mathcal{Y}$.
Choose a scheme $U$ and a surjective smooth morphism
$U \to V \times_\mathcal{Y} \mathcal{X}$. Then $U \to \mathcal{X}$ is
still smooth and surjective as a composition of two such morphisms.
From the commutative diagram
$$
\xymatrix{
U \ar[d] \ar[r]_{f'} & V \ar[d] \\
\mathcal{X} \ar[r]^f & \mathcal{Y}
}
$$
we obtain a commutative diagram
$$
\xymatrix{
\QCoh(\mathcal{O}_U) & \QCoh(\mathcal{O}_V) \ar[l] \\
\QCoh(\mathcal{O}_\mathcal{X}) \ar[u] &
\QCoh(\mathcal{O}_\mathcal{Y}) \ar[l] \ar[u]
}
$$
of abelian categories. Our proof that the bottom two categories in this
diagram are abelian showed that the vertical functors are faithful
exact functors (see proof of
Sheaves on Stacks, Lemma
\ref{stacks-sheaves-lemma-quasi-coherent-algebraic-stack}).
Since $f'$ is a flat morphism of schemes (by our definition of
flat morphisms of algebraic stacks) we see that $(f')^*$ is an
exact functor on quasi-coherent sheaves on $V$. Thus we win.
\end{proof}

\begin{lemma}
\label{lemma-quasi-coherent-check-exact}
Let $\mathcal{X}$ be an algebraic stack. Let $I$ be a set and
for $i \in I$ let $x_i : U_i \to \mathcal{X}$ be an object
of $\mathcal{X}$. Assume that $x_i$ is flat and
$\coprod x_i : \coprod U_i \to \mathcal{X}$ is surjective.
Let $\varphi : \mathcal{F} \to \mathcal{G}$ be an arrow of
$\QCoh(\mathcal{O}_\mathcal{X})$. Denote $\varphi_i$
the restriction of $\varphi$ to $(U_i)_\etale$.
Then $\varphi$ is injective, resp.\ surjective, resp.\ an isomorphism
if and only if each $\varphi_i$ is so.
\end{lemma}

\begin{proof}
Choose a scheme $U$ and a surjective smooth morphism
$x : U \to \mathcal{X}$. We may and do think of $x$
as an object of $\mathcal{X}$.
This produces a presentation $\mathcal{X} = [U/R]$ for some groupoid
in spaces $(U, R, s, t, c)$ and correspondingly an equivalence
$$
\QCoh(\mathcal{O}_\mathcal{X}) = \QCoh(U, R, s, t, c)
$$
See discussion in Sheaves on Stacks, Section
\ref{stacks-sheaves-section-quasi-coherent-algebraic-stacks}.
The structure of abelian category on the right hand is such that $\varphi$
is injective, resp.\ surjective, resp.\ an isomorphism
if and only if the restriction $\varphi|_{U_\etale}$ is so, see
Groupoids in Spaces, Lemma \ref{spaces-groupoids-lemma-abelian}.

\medskip\noindent
For each $i$ we choose an \'etale covering
$\{W_{i, j} \to V \times_\mathcal{X} U_i\}_{j \in J_i}$
by schemes. Denote $g_{i, j} : W_{i, j} \to V$ and
$h_{i, j} : W_{i, j} \to U_i$ the obvious arrows.
Each of the morphisms of schemes $g_{i, j} : W_{i, j} \to U$ is flat
and they are jointly surjective.
Similarly, for each fixed $i$ the morphisms of schemes
$h_{i, j} : W_{i, j} \to U_i$ are flat and jointly surjective.
By Sheaves on Stacks, Lemma \ref{stacks-sheaves-lemma-quasi-coherent}
the pullback by $(g_{i, j})_{small}$ of the restriction
$\varphi|_{U_\etale}$ is the restriction $\varphi|_{(W_{i, j})_\etale}$
and the pullback by $(h_{i, j})_{small}$ of the restriction
$\varphi|_{(U_i)_\etale}$ is the restriction $\varphi|_{(W_{i, j})_\etale}$.
Pullback of quasi-coherent modules by a flat morphism of schemes is exact
and pullback by a jointly surjective family of flat morphisms of schemes
reflects injective, resp.\ surjective, resp.\ bijective maps
of quasi-coherent modules (in fact this holds for all modules
as we can check exactness at stalks). Thus we see
$$
\varphi|_{U_\etale} \text{ injective}
\Leftrightarrow
\varphi|_{(W_{i, j})_\etale} \text{ injective for all }i, j
\Leftrightarrow
\varphi|_{(U_i)_\etale} \text{ injective for all }i
$$
This finishes the proof.
\end{proof}




\section{Higher direct images of types of modules}
\label{section-key}

\noindent
The following lemma is the basis for our understanding of
higher direct images of certain types of sheaves of modules.
There are two versions: one for the \'etale topology and
one for the fppf topology.

\begin{lemma}
\label{lemma-general-pushforward}
Let $\mathcal{M}$ be a rule which associates to every algebraic stack
$\mathcal{X}$ a subcategory $\mathcal{M}_\mathcal{X}$ of
$\textit{Mod}(\mathcal{X}_\etale, \mathcal{O}_\mathcal{X})$
such that
\begin{enumerate}
\item $\mathcal{M}_\mathcal{X}$ is a weak Serre subcategory
of $\textit{Mod}(\mathcal{X}_\etale, \mathcal{O}_\mathcal{X})$
(see Homology, Definition \ref{homology-definition-serre-subcategory})
for all algebraic stacks $\mathcal{X}$,
\item for a smooth morphism of algebraic stacks
$f : \mathcal{Y} \to \mathcal{X}$ the functor $f^*$ maps
$\mathcal{M}_\mathcal{X}$ into $\mathcal{M}_\mathcal{Y}$,
\item if $f_i : \mathcal{X}_i \to \mathcal{X}$ is a family of smooth
morphisms of algebraic stacks with
$|\mathcal{X}| = \bigcup |f_i|(|\mathcal{X}_i|)$, then an object
$\mathcal{F}$ of
$\textit{Mod}(\mathcal{X}_\etale, \mathcal{O}_\mathcal{X})$
is in $\mathcal{M}_\mathcal{X}$ if and only if
$f_i^*\mathcal{F}$ is in $\mathcal{M}_{\mathcal{X}_i}$ for all $i$, and
\item if $f : \mathcal{Y} \to \mathcal{X}$ is a morphism of algebraic
stacks such that $\mathcal{X}$ and $\mathcal{Y}$ are representable
by affine schemes, then $R^if_*$ maps $\mathcal{M}_\mathcal{Y}$
into $\mathcal{M}_\mathcal{X}$.
\end{enumerate}
Then for any quasi-compact and quasi-separated morphism 
$f : \mathcal{Y} \to \mathcal{X}$ of algebraic stacks
$R^if_*$ maps $\mathcal{M}_\mathcal{Y}$
into $\mathcal{M}_\mathcal{X}$. (Higher direct images computed in \'etale
topology.)
\end{lemma}

\begin{proof}
Let $f : \mathcal{Y} \to \mathcal{X}$ be a quasi-compact and quasi-separated
morphism of algebraic stacks and let $\mathcal{F}$ be an object of
$\mathcal{M}_\mathcal{Y}$. Choose a surjective smooth morphism
$\mathcal{U} \to \mathcal{X}$ where $\mathcal{U}$ is representable by
a scheme. By
Sheaves on Stacks, Lemma
\ref{stacks-sheaves-lemma-base-change-higher-direct-images}
taking higher direct images commutes with base change.
Assumption (2) shows that the pullback of $\mathcal{F}$
to $\mathcal{U} \times_\mathcal{X} \mathcal{Y}$ is in
$\mathcal{M}_{\mathcal{U} \times_\mathcal{X} \mathcal{Y}}$
because the projection
$\mathcal{U} \times_\mathcal{X} \mathcal{Y} \to \mathcal{Y}$
is smooth as a base change of a smooth morphism. Hence (3) shows we may
replace $\mathcal{Y} \to \mathcal{X}$ by the projection
$\mathcal{U} \times_\mathcal{X} \mathcal{Y} \to \mathcal{U}$.
In other words, we may assume that $\mathcal{X}$
is representable by a scheme.
Using (3) once more, we see that the question is Zariski local on
$\mathcal{X}$, hence we may assume that $\mathcal{X}$ is representable by
an affine scheme. Since $f$ is quasi-compact this implies that also
$\mathcal{Y}$ is quasi-compact. Thus we may choose a surjective smooth
morphism $g : \mathcal{V} \to \mathcal{Y}$ where $\mathcal{V}$ is representable
by an affine scheme.

\medskip\noindent
In this situation we have the spectral sequence
$$
E_2^{p, q} = R^q(f \circ g_p)_*g_p^*\mathcal{F}
\Rightarrow
R^{p + q}f_*\mathcal{F}
$$
of
Sheaves on Stacks, Proposition
\ref{stacks-sheaves-proposition-smooth-covering-compute-direct-image}.
Recall that this is a first quadrant spectral sequence hence we may
use the last part of Homology, Lemma \ref{homology-lemma-first-quadrant-ss}.
Note that the morphisms
$$
g_p : \mathcal{V}_p =
\mathcal{V} \times_\mathcal{Y} \ldots \times_\mathcal{Y} \mathcal{V}
\longrightarrow
\mathcal{Y}
$$
are smooth as compositions of base changes of the smooth morphism $g$.
Thus the sheaves $g_p^*\mathcal{F}$ are in
$\mathcal{M}_{\mathcal{V}_p}$ by (2). Hence it suffices to prove that the
higher direct images of objects of $\mathcal{M}_{\mathcal{V}_p}$ under
the morphisms
$$
\mathcal{V}_p =
\mathcal{V} \times_\mathcal{Y} \ldots \times_\mathcal{Y} \mathcal{V}
\longrightarrow
\mathcal{X}
$$
are in $\mathcal{M}_\mathcal{X}$. The algebraic stacks $\mathcal{V}_p$
are quasi-compact and quasi-separated by
Morphisms of Stacks, Lemma
\ref{stacks-morphisms-lemma-quasi-compact-quasi-separated-permanence}.
Of course each $\mathcal{V}_p$ is representable by an algebraic space
(the diagonal of the algebraic stack $\mathcal{Y}$ is representable
by algebraic spaces). This reduces us to the case where
$\mathcal{Y}$ is representable by an algebraic space and $\mathcal{X}$
is representable by an affine scheme.

\medskip\noindent
In the situation where $\mathcal{Y}$ is representable by an algebraic
space and $\mathcal{X}$ is representable by an affine scheme, we choose
anew a surjective smooth morphism $\mathcal{V} \to \mathcal{Y}$ where
$\mathcal{V}$ is representable by an affine scheme. Going through the
argument above once again we once again reduce to the morphisms
$\mathcal{V}_p \to \mathcal{X}$. But in the current situation the algebraic
stacks $\mathcal{V}_p$ are representable by quasi-compact and quasi-separated
schemes (because the diagonal of an algebraic space is representable by
schemes).

\medskip\noindent
Thus we may assume $\mathcal{Y}$ is representable by a scheme and
$\mathcal{X}$ is representable by an affine scheme. Choose (again)
a surjective smooth morphism $\mathcal{V} \to \mathcal{Y}$ where
$\mathcal{V}$ is representable by an affine scheme. In this case all
the algebraic stacks $\mathcal{V}_p$ are representable by separated
schemes (because the diagonal of a scheme is separated).

\medskip\noindent
Thus we may assume $\mathcal{Y}$ is representable by a separated scheme and
$\mathcal{X}$ is representable by an affine scheme. Choose (yet again)
a surjective smooth morphism $\mathcal{V} \to \mathcal{Y}$ where
$\mathcal{V}$ is representable by an affine scheme. In this case all
the algebraic stacks $\mathcal{V}_p$ are representable by affine schemes
(because the diagonal of a separated scheme is a closed immersion hence affine)
and this case is handled by assumption (4).
This finishes the proof.
\end{proof}

\noindent
Here is the version for the fppf topology.

\begin{lemma}
\label{lemma-general-pushforward-fppf}
Let $\mathcal{M}$ be a rule which associates to every algebraic stack
$\mathcal{X}$ a subcategory $\mathcal{M}_\mathcal{X}$ of
$\textit{Mod}(\mathcal{O}_\mathcal{X})$
such that
\begin{enumerate}
\item $\mathcal{O}_\mathcal{X}$ is a weak Serre subcategory
of $\textit{Mod}(\mathcal{O}_\mathcal{X})$
for all algebraic stacks $\mathcal{X}$,
\item for a smooth morphism of algebraic stacks
$f : \mathcal{Y} \to \mathcal{X}$ the functor $f^*$ maps
$\mathcal{M}_\mathcal{X}$ into $\mathcal{M}_\mathcal{Y}$,
\item if $f_i : \mathcal{X}_i \to \mathcal{X}$ is a family of smooth
morphisms of algebraic stacks with
$|\mathcal{X}| = \bigcup |f_i|(|\mathcal{X}_i|)$, then an object
$\mathcal{F}$ of $\textit{Mod}(\mathcal{O}_\mathcal{X})$
is in $\mathcal{M}_\mathcal{X}$ if and only if
$f_i^*\mathcal{F}$ is in $\mathcal{M}_{\mathcal{X}_i}$ for all $i$, and
\item if $f : \mathcal{Y} \to \mathcal{X}$ is a morphism of algebraic
stacks and $\mathcal{X}$ and $\mathcal{Y}$ are representable
by affine schemes, then $R^if_*$ maps $\mathcal{M}_\mathcal{Y}$
into $\mathcal{M}_\mathcal{X}$.
\end{enumerate}
Then for any quasi-compact and quasi-separated morphism 
$f : \mathcal{Y} \to \mathcal{X}$ of algebraic stacks
$R^if_*$ maps $\mathcal{M}_\mathcal{Y}$
into $\mathcal{M}_\mathcal{X}$. (Higher direct images computed in fppf
topology.)
\end{lemma}

\begin{proof}
Identical to the proof of Lemma \ref{lemma-general-pushforward}.
\end{proof}


\section{Locally quasi-coherent modules}
\label{section-locally-quasi-coherent}

\noindent
Let $\mathcal{X}$ be an algebraic stack. Let $\mathcal{F}$ be a presheaf
of $\mathcal{O}_\mathcal{X}$-modules. We can ask whether $\mathcal{F}$
is {\it locally quasi-coherent}, see
Sheaves on Stacks, Definition
\ref{stacks-sheaves-definition-locally-quasi-coherent}.
Briefly, this means $\mathcal{F}$ is an $\mathcal{O}_\mathcal{X}$-module
for the \'etale topology such that for any morphism $f : U \to \mathcal{X}$
the restriction $f^*\mathcal{F}|_{U_\etale}$ is quasi-coherent
on $U_\etale$. (The actual definition is slightly different, but
equivalent.) A useful fact is that
$$
\textit{LQCoh}(\mathcal{O}_\mathcal{X}) \subset
\textit{Mod}(\mathcal{X}_\etale, \mathcal{O}_\mathcal{X})
$$
is a weak Serre subcategory, see
Sheaves on Stacks, Lemma \ref{stacks-sheaves-lemma-lqc-colimits}.

\begin{lemma}
\label{lemma-check-lqc-on-etale-covering}
Let $\mathcal{X}$ be an algebraic stack. Let
$f_j : \mathcal{X}_j \to \mathcal{X}$ be a family of smooth
morphisms of algebraic stacks with
$|\mathcal{X}| =\bigcup |f_j|(|\mathcal{X}_j|)$.
Let $\mathcal{F}$ be a sheaf of $\mathcal{O}_\mathcal{X}$-modules
on $\mathcal{X}_\etale$. If each $f_j^{-1}\mathcal{F}$
is locally quasi-coherent, then so is $\mathcal{F}$.
\end{lemma}

\begin{proof}
We may replace each of the algebraic stacks $\mathcal{X}_j$ by
a scheme $U_j$ (using that any algebraic stack has a smooth covering by
a scheme and that compositions of smooth morphisms are smooth, see
Morphisms of Stacks, Lemma \ref{stacks-morphisms-lemma-composition-smooth}).
The pullback of $\mathcal{F}$ to $(\Sch/U_j)_\etale$ is still
locally quasi-coherent, see
Sheaves on Stacks, Lemma \ref{stacks-sheaves-lemma-pullback-lqc}.
Then $f = \coprod f_j : U = \coprod U_j \to \mathcal{X}$ is a surjective
smooth morphism. Let $x$ be an object of $\mathcal{X}$. By
Sheaves on Stacks, Lemma
\ref{stacks-sheaves-lemma-surjective-flat-locally-finite-presentation}
there exists an \'etale covering $\{x_i \to x\}_{i \in I}$
such that each $x_i$ lifts to an object $u_i$ of $(\Sch/U)_\etale$.
This just means that $x$, $x_i$ live over schemes $V$, $V_i$, that
$\{V_i \to V\}$ is an \'etale covering, and that $x_i$ comes from
a morphism $u_i : V_i \to U$. The restriction
$x_i^*\mathcal{F}|_{V_{i, \etale}}$ is equal to the restriction
of $f^*\mathcal{F}$ to $V_{i, \etale}$, see
Sheaves on Stacks, Lemma \ref{stacks-sheaves-lemma-comparison}.
Hence $x^*\mathcal{F}|_{V_\etale}$
is a sheaf on the small \'etale site of $V$ which is quasi-coherent
when restricted to $V_{i, \etale}$ for each $i$.
This implies that it is quasi-coherent (as desired), for example by
Properties of Spaces, Lemma
\ref{spaces-properties-lemma-characterize-quasi-coherent}.
\end{proof}

\begin{lemma}
\label{lemma-pushforward-locally-quasi-coherent}
Let $f : \mathcal{X} \to \mathcal{Y}$ be a quasi-compact and
quasi-separated morphism of algebraic stacks. Let 
$\mathcal{F}$ be a locally quasi-coherent
$\mathcal{O}_\mathcal{X}$-module on $\mathcal{X}_\etale$.
Then $R^if_*\mathcal{F}$ (computed in the \'etale topology) is
locally quasi-coherent on $\mathcal{Y}_\etale$.
\end{lemma}

\begin{proof}
We will use
Lemma \ref{lemma-general-pushforward}
to prove this. We will check its assumptions (1) -- (4).
Parts (1) and (2) follows from
Sheaves on Stacks, Lemma \ref{stacks-sheaves-lemma-lqc-colimits}.
Part (3) follows from
Lemma \ref{lemma-check-lqc-on-etale-covering}.
Thus it suffices to show (4).

\medskip\noindent
Suppose $f : \mathcal{X} \to \mathcal{Y}$ is a morphism of algebraic stacks
such that $\mathcal{X}$ and $\mathcal{Y}$ are representable by affine
schemes $X$ and $Y$. Choose any object $y$ of $\mathcal{Y}$ lying over a
scheme $V$. For clarity, denote $\mathcal{V} = (\Sch/V)_{fppf}$ the
algebraic stack corresponding to $V$. Consider the cartesian diagram
$$
\xymatrix{
\mathcal{Z} \ar[d] \ar[r]_g \ar[d]_{f'} & \mathcal{X} \ar[d]^f \\
\mathcal{V} \ar[r]^y & \mathcal{Y}
}
$$
Thus $\mathcal{Z}$ is representable by the scheme $Z = V \times_Y X$
and $f'$ is quasi-compact and separated (even affine). By
Sheaves on Stacks, Lemma
\ref{stacks-sheaves-lemma-compare-representable-morphism-cohomology}
we have
$$
R^if_*\mathcal{F}|_{V_\etale} =
R^if'_{small, *}\big(g^*\mathcal{F}|_{Z_\etale}\big)
$$
The right hand side is a quasi-coherent sheaf on $V_\etale$ by
Cohomology of Spaces, Lemma
\ref{spaces-cohomology-lemma-higher-direct-image}.
This implies the left hand side is quasi-coherent which is what
we had to prove.
\end{proof}

\begin{lemma}
\label{lemma-check-lqc-on-flat-covering}
Let $\mathcal{X}$ be an algebraic stack. Let
$f_j : \mathcal{X}_j \to \mathcal{X}$ be a family of flat
and locally finitely presented morphisms of algebraic stacks with
$|\mathcal{X}| =\bigcup |f_j|(|\mathcal{X}_j|)$.
Let $\mathcal{F}$ be a sheaf of $\mathcal{O}_\mathcal{X}$-modules
on $\mathcal{X}_{fppf}$. If each $f_j^{-1}\mathcal{F}$
is locally quasi-coherent, then so is $\mathcal{F}$.
\end{lemma}

\begin{proof}
First, suppose there is a morphism $a : \mathcal{U} \to \mathcal{X}$
which is surjective, flat, locally of finite presentation, quasi-compact,
and quasi-separated such that $a^*\mathcal{F}$ is locally quasi-coherent.
Then there is an exact sequence
$$
0 \to \mathcal{F} \to a_*a^*\mathcal{F} \to b_*b^*\mathcal{F}
$$
where $b$ is the morphism
$b : \mathcal{U} \times_\mathcal{X} \mathcal{U} \to \mathcal{X}$, see
Sheaves on Stacks, Proposition
\ref{stacks-sheaves-proposition-exactness-cech-complex} and
Lemma \ref{stacks-sheaves-lemma-surjective-flat-locally-finite-presentation}.
Moreover, the pullback $b^*\mathcal{F}$ is the pullback of $a^*\mathcal{F}$
via one of the projection morphisms, hence is locally quasi-coherent
(Sheaves on Stacks, Lemma \ref{stacks-sheaves-lemma-pullback-lqc}).
The modules $a_*a^*\mathcal{F}$ and $b_*b^*\mathcal{F}$ are locally
quasi-coherent by Lemma \ref{lemma-pushforward-locally-quasi-coherent}.
(Note that $a_*$ and $b_*$ don't care about which topology is
used to calculate them.)
We conclude that $\mathcal{F}$ is locally quasi-coherent, see
Sheaves on Stacks, Lemma \ref{stacks-sheaves-lemma-lqc-colimits}.

\medskip\noindent
We are going to reduce the proof of the general case the
situation in the first paragraph. Let $x$ be an object of $\mathcal{X}$
lying over the scheme $U$. We have to show that
$\mathcal{F}|_{U_\etale}$ is a quasi-coherent $\mathcal{O}_U$-module.
It suffices to do this (Zariski) locally on $U$, hence we may
assume that $U$ is affine. By
Morphisms of Stacks, Lemma
\ref{stacks-morphisms-lemma-surjective-family-flat-locally-finite-presentation}
there exists an fppf covering $\{a_i : U_i \to U\}$ such that
each $x \circ a_i$ factors through some $f_j$. Hence $a_i^*\mathcal{F}$
is locally quasi-coherent on $(\Sch/U_i)_{fppf}$. After refining
the covering we may assume $\{U_i \to U\}_{i = 1, \ldots, n}$
is a standard fppf covering. Then $x^*\mathcal{F}$ is an fppf
module on $(\Sch/U)_{fppf}$ whose pullback by the morphism
$a : U_1 \amalg \ldots \amalg U_n \to U$ is locally quasi-coherent.
Hence by the first paragraph we see that $x^*\mathcal{F}$ is locally
quasi-coherent, which certainly implies that $\mathcal{F}|_{U_\etale}$
is quasi-coherent.
\end{proof}






\section{Flat comparison maps}
\label{section-flat-comparison}

\noindent
Let $\mathcal{X}$ be an algebraic stack and let $\mathcal{F}$ be an object
of $\textit{Mod}(\mathcal{X}_\etale, \mathcal{O}_\mathcal{X})$.
Given an object $x$ of $\mathcal{X}$ lying over the scheme $U$ the
restriction $\mathcal{F}|_{U_\etale}$ is the restriction of
$x^{-1}\mathcal{F}$ to the small \'etale site of $U$, see
Sheaves on Stacks, Definition \ref{stacks-sheaves-definition-pullback}.
Next, let $\varphi : x \to x'$ be a morphism of $\mathcal{X}$ lying
over a morphism of schemes $f : U \to U'$. Thus a $2$-commutative diagram
$$
\xymatrix{
U \ar[rd]_x \ar[rr]_f & & U' \ar[ld]^{x'} \\
& \mathcal{X}
}
$$
Associated to $\varphi$ we obtain a comparison map between restrictions
\begin{equation}
\label{equation-comparison-modules}
c_\varphi :
f_{small}^*(\mathcal{F}|_{U'_\etale})
\longrightarrow
\mathcal{F}|_{U_\etale}
\end{equation}
see Sheaves on Stacks, Equation
(\ref{stacks-sheaves-equation-comparison-modules}).
In this situation we can consider the following property
of $\mathcal{F}$.

\begin{definition}
\label{definition-flat-base-change}
Let $\mathcal{X}$ be an algebraic stack and let $\mathcal{F}$ in
$\textit{Mod}(\mathcal{X}_\etale, \mathcal{O}_\mathcal{X})$.
We say $\mathcal{F}$ has the {\it flat base change property}\footnote{This
may be nonstandard notation.}
if and only if $c_\varphi$ is an isomorphism whenever $f$ is flat.
\end{definition}

\noindent
Here is a lemma with some properties of this notion.

\begin{lemma}
\label{lemma-check-flat-comparison-on-etale-covering}
Let $\mathcal{X}$ be an algebraic stack. Let $\mathcal{F}$
be an $\mathcal{O}_\mathcal{X}$-module on $\mathcal{X}_\etale$.
\begin{enumerate}
\item If $\mathcal{F}$ has the flat base change property then for any morphism
$g : \mathcal{Y} \to \mathcal{X}$ of algebraic stacks, the
pullback $g^*\mathcal{F}$ does too.
\item The full subcategory of
$\textit{Mod}(\mathcal{X}_\etale, \mathcal{O}_\mathcal{X})$
consisting of modules with the flat base change property
is a weak Serre subcategory.
\item  Let $f_i : \mathcal{X}_i \to \mathcal{X}$ be a family of
smooth morphisms of algebraic stacks such that
$|\mathcal{X}| = \bigcup_i |f_i|(|\mathcal{X}_i|)$. If each
$f_i^*\mathcal{F}$ has the flat base change property then so does
$\mathcal{F}$.
\item The category of $\mathcal{O}_\mathcal{X}$-modules
on $\mathcal{X}_\etale$ with the flat base change property
has colimits and they agree with colimits in
$\textit{Mod}(\mathcal{X}_\etale, \mathcal{O}_\mathcal{X})$.
\item Given $\mathcal{F}$ and $\mathcal{G}$ in
$\textit{Mod}(\mathcal{X}_\etale, \mathcal{O}_\mathcal{X})$
with the flat base change property then the tensor product
$\mathcal{F} \otimes_{\mathcal{O}_\mathcal{X}} \mathcal{G}$
has the flat base change property.
\item Given $\mathcal{F}$ and $\mathcal{G}$ in
$\textit{Mod}(\mathcal{X}_\etale, \mathcal{O}_\mathcal{X})$
with $\mathcal{F}$ of finite presentation and $\mathcal{G}$ having
the flat base change property then the sheaf
$\SheafHom_{\mathcal{O}_\mathcal{X}}(\mathcal{F}, \mathcal{G})$
has the flat base change property.
\end{enumerate}
\end{lemma}

\begin{proof}
Let $g : \mathcal{Y} \to \mathcal{X}$ be as in (1).
Let $y$ be an object of $\mathcal{Y}$ lying over a scheme $V$. By
Sheaves on Stacks, Lemma \ref{stacks-sheaves-lemma-comparison}
we have
$(g^*\mathcal{F})|_{V_\etale} = \mathcal{F}|_{V_\etale}$.
Moreover a comparison mapping for the sheaf $g^*\mathcal{F}$ on $\mathcal{Y}$
is a special case of a comparison map for the sheaf $\mathcal{F}$ on
$\mathcal{X}$, see
Sheaves on Stacks, Lemma \ref{stacks-sheaves-lemma-comparison}.
In this way (1) is clear.

\medskip\noindent
Proof of (2). We use the characterization of weak Serre subcategories of
Homology, Lemma \ref{homology-lemma-characterize-weak-serre-subcategory}.
Kernels and cokernels of
maps between sheaves having the flat base change property
also have the flat base change property. This is clear because
$f_{small}^*$ is exact for a flat morphism of schemes and since the
restriction functors $(-)|_{U_\etale}$ are exact (because we
are working in the \'etale topology). Finally, if
$0 \to \mathcal{F}_1 \to \mathcal{F}_2 \to \mathcal{F}_3 \to 0$
is a short exact sequence of
$\textit{Mod}(\mathcal{X}_\etale, \mathcal{O}_\mathcal{X})$
and the outer two sheaves have the flat base change property then
the middle one does as well, again because of the exactness of
$f_{small}^*$ and the restriction functors (and the 5 lemma).

\medskip\noindent
Proof of (3).
Let $f_i : \mathcal{X}_i \to \mathcal{X}$ be a jointly surjective family of
smooth morphisms of algebraic stacks and assume each $f_i^*\mathcal{F}$
has the flat base change property. By part (1), the definition of
an algebraic stack, and the fact that compositions of smooth morphisms
are smooth (see
Morphisms of Stacks, Lemma \ref{stacks-morphisms-lemma-composition-smooth})
we may assume that each $\mathcal{X}_i$ is representable by a scheme.
Let $\varphi : x \to x'$ be a morphism of $\mathcal{X}$ lying over
a flat morphism $a : U \to U'$ of schemes. By
Sheaves on Stacks, Lemma
\ref{stacks-sheaves-lemma-surjective-flat-locally-finite-presentation}
there exists a jointly surjective family of \'etale morphisms
$U'_i \to U'$ such that $U'_i \to U' \to \mathcal{X}$ factors through
$\mathcal{X}_i$. Thus we obtain commutative diagrams
$$
\xymatrix{
U_i = U \times_{U'} U_i' \ar[r]_-{a_i} \ar[d] &
U_i' \ar[r]_{x_i'} \ar[d] & \mathcal{X}_i \ar[d]^{f_i} \\
U \ar[r]^a & U' \ar[r]^{x'} & \mathcal{X}
}
$$
Note that each $a_i$ is a flat morphism of schemes as a base change of $a$.
Denote $\psi_i : x_i \to x'_i$ the morphism of $\mathcal{X}_i$ lying over
$a_i$ with target $x_i'$. By assumption the comparison maps
$c_{\psi_i} :
(a_i)_{small}^*\big(f_i^*\mathcal{F}|_{(U'_i)_\etale}\big)
\to f_i^*\mathcal{F}|_{(U_i)_\etale}$ is an isomorphism.
Because the vertical arrows $U_i' \to U'$ and $U_i \to U$ are \'etale,
the sheaves $f_i^*\mathcal{F}|_{(U_i')_\etale}$ and
$f_i^*\mathcal{F}|_{(U_i)_\etale}$ are the restrictions of
$\mathcal{F}|_{U'_\etale}$ and $\mathcal{F}|_{U_\etale}$
and the map $c_{\psi_i}$ is the restriction of $c_\varphi$ to
$(U_i)_\etale$, see
Sheaves on Stacks, Lemma \ref{stacks-sheaves-lemma-comparison}.
Since $\{U_i \to U\}$ is an \'etale covering, this implies
that the comparison map $c_\varphi$ is an isomorphism which is what
we wanted to prove.

\medskip\noindent
Proof of (4). Let $\mathcal{I} \to
\textit{Mod}(\mathcal{X}_\etale, \mathcal{O}_\mathcal{X})$,
$i \mapsto \mathcal{F}_i$ be a diagram and assume each $\mathcal{F}_i$
has the flat base change property. Let $\varphi : x \to x'$ be a morphism
of $\mathcal{X}$ lying over the flat morphism of schemes $f : U \to U'$.
Recall that $\colim_i \mathcal{F}_i$ is the sheafification of the presheaf
colimit. As we are using the \'etale topology, it is clear that
$$
(\colim_i \mathcal{F}_i)|_{U_\etale} =
\colim_i {\mathcal{F}_i}|_{U_\etale}
$$
and similarly for the restriction to $U'_\etale$. Hence
\begin{align*}
f_{small}^*((\colim_i \mathcal{F}_i)|_{U'_\etale})
& =
f_{small}^*(\colim_i {\mathcal{F}_i}|_{U'_\etale}) \\
& =
\colim_i f_{small}^*({\mathcal{F}_i}|_{U'_\etale}) \\
& \xrightarrow{\colim c_\varphi}
\colim_i \mathcal{F}_i|_{U_\etale} \\
& =
(\colim_i \mathcal{F}_i)|_{U_\etale}
\end{align*}
For the second equality we used that $f_{small}^*$ commutes with colimits
(as a left adjoint). The arrow is an isomorphism as each
$\mathcal{F}_i$ has the flat base change property. Thus the colimit
has the flat base change property and (4) is true.

\medskip\noindent
Part (5) holds because tensor products commute with pullbacks, see
Modules on Sites, Lemma \ref{sites-modules-lemma-tensor-product-pullback}.
Details omitted.

\medskip\noindent
Let $\mathcal{F}$ and $\mathcal{G}$ be as in (6). Since $\mathcal{F}$
is quasi-coherent it has the flat base change property by
Sheaves on Stacks, Lemma \ref{stacks-sheaves-lemma-quasi-coherent}.
Let $\varphi : x \to x'$ be a morphism of $\mathcal{X}$ lying
over the flat morphism of schemes $f : U \to U'$.
As we are using the \'etale topology, we have
$$
\SheafHom_{\mathcal{O}_\mathcal{X}}(\mathcal{F}, \mathcal{G})|_{U_\etale} =
\SheafHom_{\mathcal{O}_U}(\mathcal{F}|_{U_\etale}, \mathcal{G}|_{U_\etale})
$$
and similarly for the restriction to $U'_\etale$ (details omitted). Hence
\begin{align*}
f_{small}^*(
\SheafHom_{\mathcal{O}_\mathcal{X}}(\mathcal{F}, \mathcal{G})|_{U'_\etale})
& =
f_{small}^*(
\SheafHom_{\mathcal{O}_{U'}}(
\mathcal{F}|_{U'_\etale}, \mathcal{G}|_{U'_\etale})) \\
& =
\SheafHom_{\mathcal{O}_{U'}}(
f_{small}^*(\mathcal{F}|_{U'_\etale}),
f_{small}^*(\mathcal{G}|_{U'_\etale})) \\
& \xrightarrow{c_\varphi}
\SheafHom_{\mathcal{O}_U}(\mathcal{F}|_{U_\etale}, \mathcal{G}|_{U_\etale}) \\
& =
\SheafHom_{\mathcal{O}_\mathcal{X}}(\mathcal{F}, \mathcal{G})|_{U_\etale}
\end{align*}
Here the second equality is
Modules on Sites, Lemma \ref{sites-modules-lemma-pullback-internal-hom}
which uses that $f : U \to U'$ is flat and hence the morphism
of ringed sites $f_{small}$ is flat too.
The arrow is an isomorphism as both $\mathcal{F}$ and $\mathcal{G}$
have the flat base change property. Thus our $\SheafHom$
has the flat base change property too as desired.
\end{proof}

\begin{lemma}
\label{lemma-flat-comparison}
Let $f : \mathcal{X} \to \mathcal{Y}$ be a quasi-compact and
quasi-separated morphism of algebraic stacks. Let 
$\mathcal{F}$ be an object of
$\textit{Mod}(\mathcal{X}_\etale, \mathcal{O}_\mathcal{X})$
which is locally quasi-coherent and has the flat base change property.
Then each $R^if_*\mathcal{F}$ (computed in the \'etale topology)
has the flat base change property.
\end{lemma}

\begin{proof}
We will use
Lemma \ref{lemma-general-pushforward}
to prove this. For every algebraic stack $\mathcal{X}$ let
$\textit{LQCoh}^{fbc}(\mathcal{O}_\mathcal{X})$ denote the full subcategory of
$\textit{Mod}(\mathcal{X}_\etale, \mathcal{O}_\mathcal{X})$
consisting of locally quasi-coherent sheaves with the flat base
change property. Once we verify conditions (1) -- (4) of
Lemma \ref{lemma-general-pushforward}
the lemma will follow. Properties (1), (2), and (3) follow from
Sheaves on Stacks, Lemmas \ref{stacks-sheaves-lemma-pullback-lqc} and
\ref{stacks-sheaves-lemma-lqc-colimits}
and
Lemmas \ref{lemma-check-lqc-on-etale-covering} and
\ref{lemma-check-flat-comparison-on-etale-covering}.
Thus it suffices to show part (4).

\medskip\noindent
Suppose $f : \mathcal{X} \to \mathcal{Y}$ is a morphism of algebraic stacks
such that $\mathcal{X}$ and $\mathcal{Y}$ are representable by affine
schemes $X$ and $Y$. In this case, suppose that
$\psi : y \to y'$ is a morphism of $\mathcal{Y}$ lying over
a flat morphism $b : V \to V'$ of schemes. For clarity denote
$\mathcal{V} = (\Sch/V)_{fppf}$ and $\mathcal{V}' = (\Sch/V')_{fppf}$
the corresponding algebraic stacks. Consider the diagram
of algebraic stacks
$$
\xymatrix{
\mathcal{Z} \ar[d]_{f''} \ar[r]_a &
\mathcal{Z}' \ar[r]_{x'} \ar[d]_{f'} & \mathcal{X} \ar[d]^f \\
\mathcal{V} \ar[r]^b & \mathcal{V}' \ar[r]^{y'} & \mathcal{Y}
}
$$
with both squares cartesian. As $f$ is representable by schemes
(and quasi-compact and separated -- even affine) we see that $\mathcal{Z}$ and
$\mathcal{Z}'$ are representable by schemes $Z$ and $Z'$ and in
fact $Z = V \times_{V'} Z'$. Since $\mathcal{F}$ has the flat
base change property we see that
$$
a_{small}^*\big(\mathcal{F}|_{Z'_\etale}\big)
\longrightarrow
\mathcal{F}|_{Z_\etale}
$$
is an isomorphism. Moreover,
$$
R^if_*\mathcal{F}|_{V'_\etale} =
R^i(f')_{small, *}\big(\mathcal{F}|_{Z'_\etale}\big)
$$
and
$$
R^if_*\mathcal{F}|_{V_\etale} =
R^i(f'')_{small, *}\big(\mathcal{F}|_{Z_\etale}\big)
$$
by
Sheaves on Stacks, Lemma
\ref{stacks-sheaves-lemma-compare-representable-morphism-cohomology}.
Hence we see that the comparison map
$$
c_\psi :
b_{small}^*(R^if_*\mathcal{F}|_{V'_\etale})
\longrightarrow
R^if_*\mathcal{F}|_{V_\etale}
$$
is an isomorphism by
Cohomology of Spaces, Lemma
\ref{spaces-cohomology-lemma-flat-base-change-cohomology}.
Thus $R^if_*\mathcal{F}$ has the flat base change property.
Since $R^if_*\mathcal{F}$ is locally quasi-coherent by
Lemma \ref{lemma-pushforward-locally-quasi-coherent}
we win.
\end{proof}





\section{Locally quasi-coherent modules with the flat base change property}
\label{section-loc-qcoh-flat-base-change}

\noindent
Let $\mathcal{X}$ be an algebraic stack. We\footnote{Apologies
for the horrendous notation.} will denote
$$
\textit{LQCoh}^{fbc}(\mathcal{O}_\mathcal{X})
\subset
\textit{Mod}(\mathcal{X}_\etale, \mathcal{O}_\mathcal{X})
$$
the full subcategory whose objects are \'etale
$\mathcal{O}_\mathcal{X}$-modules $\mathcal{F}$
which are both locally quasi-coherent
(Section \ref{section-locally-quasi-coherent})
and have the flat base change property
(Section \ref{section-flat-comparison}).
We have
$$
\QCoh(\mathcal{O}_\mathcal{X}) \subset
\textit{LQCoh}^{fbc}(\mathcal{O}_\mathcal{X})
$$
by Sheaves on Stacks, Lemma \ref{stacks-sheaves-lemma-quasi-coherent}.

\begin{proposition}
\label{proposition-loc-qcoh-flat-base-change}
Summary of results on locally quasi-coherent modules having the flat
base change property.
\begin{enumerate}
\item Let $\mathcal{X}$ be an algebraic stack.
If $\mathcal{F}$ is in $\textit{LQCoh}^{fbc}(\mathcal{O}_\mathcal{X})$,
then $\mathcal{F}$ is a sheaf for the fppf topology, i.e., it is
an object of $\textit{Mod}(\mathcal{O}_\mathcal{X})$.
\item The category $\textit{LQCoh}^{fbc}(\mathcal{O}_\mathcal{X})$
is a weak Serre subcategory of both $\textit{Mod}(\mathcal{O}_\mathcal{X})$
and $\textit{Mod}(\mathcal{X}_\etale, \mathcal{O}_\mathcal{X})$.
\item Pullback $f^*$ along any morphism of algebraic stacks
$f : \mathcal{X} \to \mathcal{Y}$ induces a functor
$f^* : \textit{LQCoh}^{fbc}(\mathcal{O}_\mathcal{Y}) \to
\textit{LQCoh}^{fbc}(\mathcal{O}_\mathcal{X})$.
\item If $f : \mathcal{X} \to \mathcal{Y}$ is a
quasi-compact and quasi-separated morphism of algebraic stacks
and $\mathcal{F}$ is an object of
$\textit{LQCoh}^{fbc}(\mathcal{O}_\mathcal{X})$, then
\begin{enumerate}
\item the total direct image $Rf_*\mathcal{F}$ and the higher direct
images $R^if_*\mathcal{F}$ can be computed in either the \'etale or the
fppf topology with the same result, and
\item each $R^if_*\mathcal{F}$ is an object of
$\textit{LQCoh}^{fbc}(\mathcal{O}_\mathcal{Y})$.
\end{enumerate}
\item The category $\textit{LQCoh}^{fbc}(\mathcal{O}_\mathcal{X})$ has
colimits and they agree with colimits in
$\textit{Mod}(\mathcal{X}_\etale, \mathcal{O}_\mathcal{X})$
as well as in $\textit{Mod}(\mathcal{O}_\mathcal{X})$.
\item Given $\mathcal{F}$ and $\mathcal{G}$ in
$\textit{LQCoh}^{fbc}(\mathcal{O}_\mathcal{X})$ then the tensor product
$\mathcal{F} \otimes_{\mathcal{O}_\mathcal{X}} \mathcal{G}$
is in $\textit{LQCoh}^{fbc}(\mathcal{O}_\mathcal{X})$.
\item Given $\mathcal{F}$ of finite presentation and $\mathcal{G}$ in
$\textit{LQCoh}^{fbc}(\mathcal{O}_\mathcal{X})$ then
$\SheafHom_{\mathcal{O}_\mathcal{X}}(\mathcal{F}, \mathcal{G})$
is in $\textit{LQCoh}^{fbc}(\mathcal{O}_\mathcal{X})$.
\end{enumerate}
\end{proposition}

\begin{proof}
Part (1) is
Sheaves on Stacks, Lemma
\ref{stacks-sheaves-lemma-lqc-flat-base-change-fppf-sheaf}.

\medskip\noindent
Part (2) for the embedding
$\textit{LQCoh}^{fbc}(\mathcal{O}_\mathcal{X}) \subset
\textit{Mod}(\mathcal{X}_\etale, \mathcal{O}_\mathcal{X})$
we have seen in the proof of
Lemma \ref{lemma-flat-comparison}.
Let us prove (2) for the embedding
$\textit{LQCoh}^{fbc}(\mathcal{O}_\mathcal{X}) \subset
\textit{Mod}(\mathcal{O}_\mathcal{X})$.
Let $\varphi : \mathcal{F} \to \mathcal{G}$ be a morphism between
objects of $\textit{LQCoh}^{fbc}(\mathcal{O}_\mathcal{X})$. Since
$\Ker(\varphi)$
is the same whether computed in the \'etale or the fppf
topology, we see that $\Ker(\varphi)$ is in
$\textit{LQCoh}^{fbc}(\mathcal{O}_\mathcal{X})$
by the \'etale case. On the other hand,
the cokernel computed in the fppf topology is the fppf sheafification
of the cokernel computed in the \'etale topology. However, this
\'etale cokernel is in $\textit{LQCoh}^{fbc}(\mathcal{O}_\mathcal{X})$
hence an fppf sheaf
by (1) and we see that the cokernel is in
$\textit{LQCoh}^{fbc}(\mathcal{O}_\mathcal{X})$.
Finally, suppose that
$$
0 \to \mathcal{F}_1 \to \mathcal{F}_2 \to \mathcal{F}_3 \to 0
$$
is an exact sequence in $\textit{Mod}(\mathcal{O}_\mathcal{X})$
(i.e., using the fppf topology) with $\mathcal{F}_1$, $\mathcal{F}_2$
in $\textit{LQCoh}^{fbc}(\mathcal{O}_\mathcal{X})$. In order to show that
$\mathcal{F}_2$ is an object of $\textit{LQCoh}^{fbc}(\mathcal{O}_\mathcal{X})$
it suffices to show that
the sequence is also exact in the \'etale topology. To do this it
suffices to show that any element of $H^1_{fppf}(x, \mathcal{F}_1)$
becomes zero on the members of an \'etale covering of $x$ (for any
object $x$ of $\mathcal{X}$). This is true because
$H^1_{fppf}(x, \mathcal{F}_1) = H^1_\etale(x, \mathcal{F}_1)$ by
Sheaves on Stacks, Lemma \ref{stacks-sheaves-lemma-compare-fppf-etale}
and because of locality of cohomology, see
Cohomology on Sites, Lemma
\ref{sites-cohomology-lemma-kill-cohomology-class-on-covering}.
This proves (2).

\medskip\noindent
Part (3) follows from
Lemma \ref{lemma-check-flat-comparison-on-etale-covering}
and
Sheaves on Stacks, Lemma \ref{stacks-sheaves-lemma-pullback-lqc}.

\medskip\noindent
Part (4)(b) for $R^if_*\mathcal{F}$ computed in the \'etale cohomology
follows from Lemma \ref{lemma-flat-comparison}.
Whereupon part (4)(a) follows from
Sheaves on Stacks, Lemma \ref{stacks-sheaves-lemma-compare-fppf-etale}
combined with (1) above.

\medskip\noindent
Part (5) for the \'etale topology follows from
Sheaves on Stacks, Lemma \ref{stacks-sheaves-lemma-lqc-colimits} and
Lemma \ref{lemma-check-flat-comparison-on-etale-covering}.
The fppf version then follows as the colimit in the \'etale
topology is already an fppf sheaf by part (1).

\medskip\noindent
Parts (6) and (7) follow from the corresponding parts of
Lemma \ref{lemma-check-flat-comparison-on-etale-covering} and
Sheaves on Stacks, Lemma \ref{stacks-sheaves-lemma-lqc-colimits}.
\end{proof}

\begin{lemma}
\label{lemma-check-lqc-fbc-on-covering}
Let $\mathcal{X}$ be an algebraic stack.
\begin{enumerate}
\item Let $f_j : \mathcal{X}_j \to \mathcal{X}$ be a family of smooth
morphisms of algebraic stacks with
$|\mathcal{X}| =\bigcup |f_j|(|\mathcal{X}_j|)$.
Let $\mathcal{F}$ be a sheaf of $\mathcal{O}_\mathcal{X}$-modules
on $\mathcal{X}_\etale$. If each $f_j^{-1}\mathcal{F}$
is in $\textit{LQCoh}^{fpc}(\mathcal{O}_{\mathcal{X}_i})$, then
$\mathcal{F}$ is in
$\textit{LQCoh}^{fbc}(\mathcal{O}_\mathcal{X})$.
\item  Let $f_j : \mathcal{X}_j \to \mathcal{X}$ be a family of flat
and locally finitely presented morphisms of algebraic stacks with
$|\mathcal{X}| =\bigcup |f_j|(|\mathcal{X}_j|)$.
Let $\mathcal{F}$ be a sheaf of $\mathcal{O}_\mathcal{X}$-modules
on $\mathcal{X}_{fppf}$. If each $f_j^{-1}\mathcal{F}$
is in $\textit{LQCoh}^{fbc}(\mathcal{O}_{\mathcal{X}_i})$, then
$\mathcal{F}$ is in $\textit{LQCoh}^{fbc}(\mathcal{O}_\mathcal{X})$.
\end{enumerate}
\end{lemma}

\begin{proof}
Part (1) follows from a combination of
Lemmas \ref{lemma-check-lqc-on-etale-covering} and
\ref{lemma-check-flat-comparison-on-etale-covering}.
The proof of (2) is analogous to the proof of
Lemma \ref{lemma-check-lqc-on-flat-covering}.
Let $\mathcal{F}$ of a sheaf of $\mathcal{O}_\mathcal{X}$-modules
on $\mathcal{X}_{fppf}$.

\medskip\noindent
First, suppose there is a morphism $a : \mathcal{U} \to \mathcal{X}$
which is surjective, flat, locally of finite presentation, quasi-compact,
and quasi-separated such that $a^*\mathcal{F}$ is locally quasi-coherent
and has the flat base change property.
Then there is an exact sequence
$$
0 \to \mathcal{F} \to a_*a^*\mathcal{F} \to b_*b^*\mathcal{F}
$$
where $b$ is the morphism
$b : \mathcal{U} \times_\mathcal{X} \mathcal{U} \to \mathcal{X}$, see
Sheaves on Stacks, Proposition
\ref{stacks-sheaves-proposition-exactness-cech-complex} and
Lemma \ref{stacks-sheaves-lemma-surjective-flat-locally-finite-presentation}.
Moreover, the pullback $b^*\mathcal{F}$ is the pullback of $a^*\mathcal{F}$
via one of the projection morphisms, hence is locally quasi-coherent
and has the flat base change property, see
Proposition \ref{proposition-loc-qcoh-flat-base-change}.
The modules $a_*a^*\mathcal{F}$ and $b_*b^*\mathcal{F}$ are locally
quasi-coherent and have the flat base change property by
Proposition \ref{proposition-loc-qcoh-flat-base-change}.
We conclude that $\mathcal{F}$ is locally quasi-coherent and
has the flat base change property by
Proposition \ref{proposition-loc-qcoh-flat-base-change}.

\medskip\noindent
Choose a scheme $U$ and a surjective smooth morphism $x : U \to \mathcal{X}$.
By part (1) it suffices to show that $x^*\mathcal{F}$ is locally
quasi-coherent and has the flat base change property.
Again by part (1) it suffices to do this (Zariski) locally on $U$,
hence we may assume that $U$ is affine. By
Morphisms of Stacks, Lemma
\ref{stacks-morphisms-lemma-surjective-family-flat-locally-finite-presentation}
there exists an fppf covering $\{a_i : U_i \to U\}$ such that
each $x \circ a_i$ factors through some $f_j$. Hence the module
$a_i^*\mathcal{F}$ on $(\Sch/U_i)_{fppf}$
is locally quasi-coherent and has the flat base change property.
After refining the covering we may assume $\{U_i \to U\}_{i = 1, \ldots, n}$
is a standard fppf covering. Then $x^*\mathcal{F}$ is an fppf
module on $(\Sch/U)_{fppf}$ whose pullback by the morphism
$a : U_1 \amalg \ldots \amalg U_n \to U$ is locally quasi-coherent
and has the flat base change property.
Hence by the previous paragraph we see that $x^*\mathcal{F}$ is locally
quasi-coherent and has the flat base change property as desired.
\end{proof}

\begin{lemma}
\label{lemma-loc-qcoh-fbc-compute-higher-direct-image}
Let $f : \mathcal{X} \to \mathcal{Y}$ be a morphism of algebraic stacks
which is quasi-compact, quasi-separated, and
representable by algebraic spaces. Let $\mathcal{F}$ be in
$\textit{LQCoh}^{fbc}(\mathcal{O}_\mathcal{X})$.
Then for an object $y : V \to \mathcal{Y}$ of $\mathcal{Y}$ we have
$$
(R^if_*\mathcal{F})|_{V_\etale} = R^if'_{small, *}(\mathcal{F}|_{U_\etale})
$$
where $f' : U = V \times_\mathcal{Y} \mathcal{X} \to V$ is the base
change of $f$.
\end{lemma}

\begin{proof}
By Sheaves on Stacks, Lemma
\ref{stacks-sheaves-lemma-base-change-higher-direct-images}
we can reduce to the case where $\mathcal{X}$
is represented by $U$ and $\mathcal{Y}$ is
represented by $V$. Of course this also uses that the pullback of
$\mathcal{F}$ to $U$ is in $\textit{LQCoh}^{fbc}(\mathcal{O}_U)$
by Proposition \ref{proposition-loc-qcoh-flat-base-change}.
Then the result follows from
Sheaves on Stacks, Lemma
\ref{stacks-sheaves-lemma-compare-morphism-cohomology}
and the fact that $R^if_*$ may be computed in the
\'etale topology by
Proposition \ref{proposition-loc-qcoh-flat-base-change}.
\end{proof}

\begin{lemma}
\label{lemma-loc-qcoh-fbc-affine-direct-image}
Let $f : \mathcal{X} \to \mathcal{Y}$ be an affine morphism of algebraic
stacks. The functor $f_* : \textit{LQCoh}^{fbc}(\mathcal{O}_\mathcal{X}) \to
\textit{LQCoh}^{fbc}(\mathcal{O}_\mathcal{Y})$ is exact and commutes
with direct sums. The functors $R^if_*$ for $i > 0$ vanish on
$\textit{LQCoh}^{fbc}(\mathcal{O}_\mathcal{X})$.
\end{lemma}

\begin{proof}
The functors exist by Proposition \ref{proposition-loc-qcoh-flat-base-change}.
By Lemma \ref{lemma-loc-qcoh-fbc-compute-higher-direct-image}
this reduces to the case of an affine morphism of algebraic spaces
taking higher direct images in the setting of quasi-coherent
modules on algebraic spaces. By the discussion in
Cohomology of Spaces, Section
\ref{spaces-cohomology-section-higher-direct-image}
we reduce to the case of an affine morphism of schemes.
For affine morphisms of schemes we have the vanishing of
higher direct images on quasi-coherent modules by
Cohomology of Schemes, Lemma \ref{coherent-lemma-relative-affine-vanishing}.
The vanishing for $R^1f_*$ implies exactness of $f_*$.
Commuting with direct sums follows from
Morphisms, Lemma \ref{morphisms-lemma-affine-equivalence-modules}
for example.
\end{proof}










\section{Parasitic modules}
\label{section-parasitic}

\noindent
The following definition is compatible with
Descent, Definition \ref{descent-definition-parasitic}.

\begin{definition}
\label{definition-parasitic}
Let $\mathcal{X}$ be an algebraic stack.
A presheaf of $\mathcal{O}_\mathcal{X}$-modules $\mathcal{F}$ is
{\it parasitic} if we have $\mathcal{F}(x) = 0$ for any object $x$
of $\mathcal{X}$ which lies over a scheme $U$ such that the corresponding
morphism $x : U \to \mathcal{X}$ is flat.
\end{definition}

\noindent
Here is a lemma with some properties of this notion.

\begin{lemma}
\label{lemma-parasitic}
Let $\mathcal{X}$ be an algebraic stack. Let $\mathcal{F}$
be a presheaf of $\mathcal{O}_\mathcal{X}$-modules.
\begin{enumerate}
\item If $\mathcal{F}$ is parasitic and
$g : \mathcal{Y} \to \mathcal{X}$ is a flat morphism of algebraic stacks,
then $g^*\mathcal{F}$ is parasitic.
\item For $\tau \in \{Zariski, \etale, smooth, syntomic, fppf\}$
we have
\begin{enumerate}
\item the $\tau$ sheafification of a parasitic presheaf of modules is
parasitic, and
\item the full subcategory of
$\textit{Mod}(\mathcal{X}_\tau, \mathcal{O}_\mathcal{X})$
consisting of parasitic modules is a Serre subcategory.
\end{enumerate}
\item Suppose $\mathcal{F}$ is a sheaf for the \'etale topology.
Let $f_i : \mathcal{X}_i \to \mathcal{X}$ be a family of
smooth morphisms of algebraic stacks such that
$|\mathcal{X}| = \bigcup_i |f_i|(|\mathcal{X}_i|)$. If each
$f_i^*\mathcal{F}$ is parasitic then so is $\mathcal{F}$.
\item Suppose $\mathcal{F}$ is a sheaf for the fppf topology.
Let $f_i : \mathcal{X}_i \to \mathcal{X}$ be a family of
flat and locally finitely presented morphisms of algebraic stacks such that
$|\mathcal{X}| = \bigcup_i |f_i|(|\mathcal{X}_i|)$. If each
$f_i^*\mathcal{F}$ is parasitic then so is $\mathcal{F}$.
\end{enumerate}
\end{lemma}

\begin{proof}
To see part (1) let $y$ be an object of $\mathcal{Y}$ which lies
over a scheme $V$ such that the corresponding morphism $y : V \to \mathcal{Y}$
is flat. Then $g(y) : V \to \mathcal{Y} \to \mathcal{X}$ is flat
as a composition of flat morphisms (see
Morphisms of Stacks, Lemma \ref{stacks-morphisms-lemma-composition-flat})
hence $\mathcal{F}(g(y))$ is zero by assumption. Since
$g^*\mathcal{F} = g^{-1}\mathcal{F}(y) = \mathcal{F}(g(y))$ we conclude
$g^*\mathcal{F}$ is parasitic.

\medskip\noindent
To see part (2)(a) note that if $\{x_i \to x\}$ is a $\tau$-covering
of $\mathcal{X}$, then each of the morphisms $x_i \to x$ lies
over a flat morphism of schemes. Hence if $x$ lies over a scheme
$U$ such that $x : U \to \mathcal{X}$ is flat, so do all of the
objects $x_i$. Hence the presheaf $\mathcal{F}^+$ (see
Sites, Section \ref{sites-section-sheafification})
is parasitic if the presheaf $\mathcal{F}$ is
parasitic. This proves (2)(a) as the sheafification of $\mathcal{F}$
is $(\mathcal{F}^+)^+$.

\medskip\noindent
Let $\mathcal{F}$ be a parasitic $\tau$-module. It is immediate from the
definitions that any submodule of $\mathcal{F}$ is parasitic. On the other
hand, if $\mathcal{F}' \subset \mathcal{F}$ is a submodule, then it is
equally clear that the presheaf
$x \mapsto \mathcal{F}(x)/\mathcal{F}'(x)$
is parasitic. Hence the quotient $\mathcal{F}/\mathcal{F}'$ is a parasitic
module by (2)(a). Finally, we have to show that given a short exact sequence
$0 \to \mathcal{F}_1 \to \mathcal{F}_2 \to \mathcal{F}_3 \to 0$
with $\mathcal{F}_1$ and $\mathcal{F}_3$ parasitic, then $\mathcal{F}_2$
is parasitic. This follows immediately on evaluating on $x$ lying
over a scheme flat over $\mathcal{X}$. This proves (2)(b), see
Homology, Lemma \ref{homology-lemma-characterize-serre-subcategory}.

\medskip\noindent
Let $f_i : \mathcal{X}_i \to \mathcal{X}$ be a jointly surjective family of
smooth morphisms of algebraic stacks and assume each $f_i^*\mathcal{F}$
is parasitic. Let $x$ be an object of $\mathcal{X}$ which lies over a
scheme $U$ such that $x : U \to \mathcal{X}$ is flat. Consider a surjective
smooth covering $W_i \to U \times_{x, \mathcal{X}} \mathcal{X}_i$.
Denote $y_i : W_i \to \mathcal{X}_i$ the projection. It follows
that $\{f_i(y_i) \to x\}$ is a covering for the smooth topology
on $\mathcal{X}$. Since a composition of flat morphisms is flat we see that
$f_i^*\mathcal{F}(y_i) = 0$. On the other hand, as we saw in the proof
of (1), we have $f_i^*\mathcal{F}(y_i) = \mathcal{F}(f_i(y_i))$.
Hence we see that for some smooth covering $\{x_i \to x\}_{i \in I}$
in $\mathcal{X}$ we have $\mathcal{F}(x_i) = 0$. This implies
$\mathcal{F}(x) = 0$ because the smooth topology is the same
as the \'etale topology, see
More on Morphisms, Lemma \ref{more-morphisms-lemma-etale-dominates-smooth}.
Namely, $\{x_i \to x\}_{i \in I}$ lies over a smooth covering
$\{U_i \to U\}_{i \in I}$ of schemes. By the lemma just referenced
there exists an \'etale covering $\{V_j \to U\}_{j \in J}$ which
refines $\{U_i \to U\}_{i \in I}$. Denote $x'_j = x|_{V_j}$.
Then $\{x'_j \to x\}$ is an \'etale covering in $\mathcal{X}$
refining $\{x_i \to x\}_{i \in I}$. This means the map
$\mathcal{F}(x) \to \prod_{j \in J} \mathcal{F}(x'_j)$, which is
injective as $\mathcal{F}$ is a sheaf in the \'etale topology,
factors through $\mathcal{F}(x) \to \prod_{i \in I} \mathcal{F}(x_i)$
which is zero. Hence $\mathcal{F}(x) = 0$ as desired.

\medskip\noindent
Proof of (4): omitted. Hint: similar, but simpler, than the proof of (3).
\end{proof}

\noindent
Parasitic modules are preserved under absolutely any pushforward.

\begin{lemma}
\label{lemma-pushforward-parasitic}
Let $\tau \in \{\etale, fppf\}$.
Let $\mathcal{X}$ be an algebraic stack.
Let $\mathcal{F}$ be a parasitic object of
$\textit{Mod}(\mathcal{X}_\tau, \mathcal{O}_\mathcal{X})$.
\begin{enumerate}
\item $H^i_\tau(\mathcal{X}, \mathcal{F}) = 0$ for all $i$.
\item Let $f : \mathcal{X} \to \mathcal{Y}$ be a morphism of algebraic stacks.
Then $R^if_*\mathcal{F}$ (computed in $\tau$-topology) is a
parasitic object of $\textit{Mod}(\mathcal{Y}_\tau, \mathcal{O}_\mathcal{Y})$.
\end{enumerate}
\end{lemma}

\begin{proof}
We first reduce (2) to (1).
By Sheaves on Stacks, Lemma \ref{stacks-sheaves-lemma-pushforward-restriction}
we see that $R^if_*\mathcal{F}$ is the sheaf associated to the presheaf
$$
y \longmapsto
H^i_\tau\Big(V \times_{y, \mathcal{Y}} \mathcal{X},
\ \text{pr}^{-1}\mathcal{F}\Big)
$$
Here $y$ is a typical object of $\mathcal{Y}$ lying over the scheme $V$.
By Lemma \ref{lemma-parasitic} it suffices to show that
these cohomology groups are zero when $y : V \to \mathcal{Y}$ is flat.
Note that $\text{pr} : V \times_{y, \mathcal{Y}} \mathcal{X} \to \mathcal{X}$
is flat as a base change of $y$. Hence by
Lemma \ref{lemma-parasitic} we see that $\text{pr}^{-1}\mathcal{F}$
is parasitic. Thus it suffices to prove (1).

\medskip\noindent
To see (1) we can use the spectral sequence of
Sheaves on Stacks, Proposition
\ref{stacks-sheaves-proposition-smooth-covering-compute-cohomology}
to reduce this to the case where $\mathcal{X}$
is an algebraic stack representable by an algebraic space.
Note that in the spectral sequence each
$f_p^{-1}\mathcal{F} = f_p^*\mathcal{F}$ is a parasitic module by
Lemma \ref{lemma-parasitic} because the morphisms
$f_p : \mathcal{U}_p =
\mathcal{U} \times_\mathcal{X} \ldots
\times_\mathcal{X} \mathcal{U} \to \mathcal{X}$ are flat.
Reusing this spectral sequence one more time (as in the
proof of Lemma \ref{lemma-general-pushforward})
we reduce to the case where the
algebraic stack $\mathcal{X}$ is representable by a scheme $X$.
Then $H^i_\tau(\mathcal{X}, \mathcal{F}) = H^i((\Sch/X)_\tau, \mathcal{F})$.
In this case the vanishing follows easily from an argument
with {\v C}ech coverings, see
Descent, Lemma \ref{descent-lemma-cohomology-parasitic}.
\end{proof}

\noindent
The following lemma is one of the major reasons we care about
parasitic modules. To understand the statement, recall that
the functors
$\QCoh(\mathcal{O}_\mathcal{X}) \to
\textit{Mod}(\mathcal{X}_\etale, \mathcal{O}_\mathcal{X})$
and
$\QCoh(\mathcal{O}_\mathcal{X}) \to
\textit{Mod}(\mathcal{O}_\mathcal{X})$
aren't exact in general.

\begin{lemma}
\label{lemma-exact-sequence-quasi-coherent-parasitic-cohomology}
Let $\mathcal{X}$ be an algebraic stack. Let
$\alpha : \mathcal{F} \to \mathcal{G}$ and
$\beta : \mathcal{G} \to \mathcal{H}$
be maps in $\QCoh(\mathcal{O}_\mathcal{X})$ with
$\beta \circ \alpha = 0$. The following are equivalent:
\begin{enumerate}
\item in the abelian category $\QCoh(\mathcal{O}_\mathcal{X})$
the complex $\mathcal{F} \to \mathcal{G} \to \mathcal{H}$
is exact at $\mathcal{G}$,
\item $\Ker(\beta)/\Im(\alpha)$ computed in either
$\textit{Mod}(\mathcal{X}_\etale, \mathcal{O}_\mathcal{X})$ or
$\textit{Mod}(\mathcal{X}_{fppf}, \mathcal{O}_\mathcal{X})$
is parasitic.
\end{enumerate}
\end{lemma}

\begin{proof}
We have $\QCoh(\mathcal{O}_\mathcal{X}) \subset
\textit{LQCoh}^{fbc}(\mathcal{O}_\mathcal{X})$, see
Section \ref{section-loc-qcoh-flat-base-change}.
Hence $\Ker(\beta)/\Im(\alpha)$ computed in
$\textit{Mod}(\mathcal{X}_\etale, \mathcal{O}_\mathcal{X})$ or
$\textit{Mod}(\mathcal{X}_{fppf}, \mathcal{O}_\mathcal{X})$ agree, see
Proposition \ref{proposition-loc-qcoh-flat-base-change}.
From now on we will use the \'etale topology on $\mathcal{X}$.

\medskip\noindent
Let $\mathcal{E}$ be the cohomology of
$\mathcal{F} \to \mathcal{G} \to \mathcal{H}$
computed in the abelian category $\QCoh(\mathcal{O}_\mathcal{X})$.
Let $x : U \to \mathcal{X}$ be a flat morphism where $U$ is a scheme.
As we are using the \'etale topology, the restriction functor
$\textit{Mod}(\mathcal{X}_\etale, \mathcal{O}_\mathcal{X})
\to \textit{Mod}(U_\etale, \mathcal{O}_U)$ is exact.
On the other hand, by Lemma \ref{lemma-flat-pullback-quasi-coherent} and
Sheaves on Stacks, Lemma \ref{stacks-sheaves-lemma-compare-quasi-coherent}
the restriction functor
$$
\QCoh(\mathcal{O}_\mathcal{X}) \xrightarrow{x^*}
\QCoh((\Sch/U)_\etale, \mathcal{O}) \xrightarrow{{-}|_{U_\etale}}
\QCoh(U_\etale, \mathcal{O}_U)
$$
is exact too. We conclude that
$\mathcal{E}|_{U_\etale} = (\Ker(\beta)/\Im(\alpha))|_{U_\etale}$.

\medskip\noindent
If (1) holds, then $\mathcal{E} = 0$ hence $\Ker(\beta)/\Im(\alpha)$
restricts to zero on $U_\etale$ for all $U$ flat over $\mathcal{X}$ and this
is the definition of a parasitic module. If (2) holds, then
$\Ker(\beta)/\Im(\alpha)$ restricts to zero on $U_\etale$ for all $U$
flat over $\mathcal{X}$ hence $\mathcal{E}$ restricts to zero on
$U_\etale$ for all $U$ flat over $\mathcal{X}$. This certainly implies
that the quasi-coherent module $\mathcal{E}$ is zero, for example apply
Lemma \ref{lemma-quasi-coherent-check-exact}
to the map $0 \to \mathcal{E}$.
\end{proof}





\section{Quasi-coherent modules}
\label{section-quasi-coherent}

\noindent
We have seen that the category of quasi-coherent modules on an algebraic
stack is equivalent to the category of quasi-coherent modules on a
presentation, see
Sheaves on Stacks, Section
\ref{stacks-sheaves-section-quasi-coherent-algebraic-stacks}.
This fact is the basis for the following.

\begin{lemma}
\label{lemma-adjoint}
Let $\mathcal{X}$ be an algebraic stack. Let
$\textit{LQCoh}^{fbc}(\mathcal{O}_\mathcal{X})$
be the category of locally quasi-coherent modules with the
flat base change property, see
Section \ref{section-loc-qcoh-flat-base-change}.
The inclusion functor
$i : \QCoh(\mathcal{O}_\mathcal{X}) \to
\textit{LQCoh}^{fbc}(\mathcal{O}_\mathcal{X})$
has a right adjoint
$$
Q : \textit{LQCoh}^{fbc}(\mathcal{O}_\mathcal{X}) \to
\QCoh(\mathcal{O}_\mathcal{X})
$$
such that $Q \circ i$ is the identity functor.
\end{lemma}

\begin{proof}
Choose a scheme $U$ and a surjective smooth morphism $f : U \to \mathcal{X}$.
Set $R = U \times_\mathcal{X} U$ so that we obtain a smooth groupoid
$(U, R, s, t, c)$ in algebraic spaces with the property that
$\mathcal{X} = [U/R]$, see
Algebraic Stacks, Lemma \ref{algebraic-lemma-stack-presentation}.
We may and do replace $\mathcal{X}$ by $[U/R]$. By
Sheaves on Stacks, Proposition \ref{stacks-sheaves-proposition-quasi-coherent}
there is an equivalence
$$
q_1 :
\QCoh(U, R, s, t, c)
\longrightarrow
\QCoh(\mathcal{O}_\mathcal{X})
$$
Let us construct a functor
$$
q_2 :
\textit{LQCoh}^{fbc}(\mathcal{O}_\mathcal{X})
\longrightarrow
\QCoh(U, R, s, t, c)
$$
by the following rule: if $\mathcal{F}$ is an object of
$\textit{LQCoh}^{fbc}(\mathcal{O}_\mathcal{X})$ then we set
$$
q_2(\mathcal{F}) = (f^*\mathcal{F}|_{U_\etale}, \alpha)
$$
where $\alpha$ is the isomorphism
$$
t_{small}^*(f^*\mathcal{F}|_{U_\etale})
\to
t^*f^*\mathcal{F}|_{R_\etale} \to
s^*f^*\mathcal{F}|_{R_\etale} \to
s_{small}^*(f^*\mathcal{F}|_{U_\etale})
$$
where the outer two morphisms are the comparison maps. Note that
$q_2(\mathcal{F})$ is quasi-coherent precisely because $\mathcal{F}$ is
locally quasi-coherent and that we used (and needed)
the flat base change property in the construction of
the descent datum $\alpha$. We omit the
verification that the cocycle condition (see
Groupoids in Spaces, Definition
\ref{spaces-groupoids-definition-groupoid-module})
holds. Looking at the proof of
Sheaves on Stacks, Proposition \ref{stacks-sheaves-proposition-quasi-coherent}
we see that $q_2 \circ i$ is the quasi-inverse to $q_1$.
We define $Q = q_1 \circ q_2$.
Let $\mathcal{F}$ be an object of
$\textit{LQCoh}^{fbc}(\mathcal{O}_\mathcal{X})$ and
let $\mathcal{G}$ be an object of $\QCoh(\mathcal{O}_\mathcal{X})$.
We have
\begin{align*}
\Mor_{\textit{LQCoh}^{fbc}(\mathcal{O}_\mathcal{X})}
(i(\mathcal{G}), \mathcal{F})
& =
\Mor_{\QCoh(U, R, s, t, c)}(q_2(i(\mathcal{G})), q_2(\mathcal{F})) \\
& =
\Mor_{\QCoh(\mathcal{O}_\mathcal{X})}(\mathcal{G}, Q(\mathcal{F}))
\end{align*}
where the first equality is
Sheaves on Stacks, Lemma \ref{stacks-sheaves-lemma-map-from-quasi-coherent}
and the second equality holds because $q_1 \circ i$ and $q_2$ are quasi-inverse
equivalences of categories. The assertion $Q \circ i \cong \text{id}$
is a formal consequence of the fact that $i$ is fully faithful.
\end{proof}

\begin{lemma}
\label{lemma-adjoint-kernel-parasitic}
Let $\mathcal{X}$ be an algebraic stack.
Let $Q : \textit{LQCoh}^{fbc}(\mathcal{O}_\mathcal{X}) \to
\QCoh(\mathcal{O}_\mathcal{X})$
be the functor constructed in Lemma \ref{lemma-adjoint}.
\begin{enumerate}
\item The kernel of $Q$ is exactly the collection of parasitic objects
of $\textit{LQCoh}^{fbc}(\mathcal{O}_\mathcal{X})$.
\item For any object $\mathcal{F}$
of $\textit{LQCoh}^{fbc}(\mathcal{O}_\mathcal{X})$ both the kernel and the
cokernel of the
adjunction map $Q(\mathcal{F}) \to \mathcal{F}$ are parasitic.
\item The functor $Q$ is exact and commutes with all limits and colimits.
\end{enumerate}
\end{lemma}

\begin{proof}
Write $\mathcal{X} = [U/R]$ as in the proof of Lemma \ref{lemma-adjoint}.
Let $\mathcal{F}$ be an object of
$\textit{LQCoh}^{fbc}(\mathcal{O}_\mathcal{X})$.
It is clear from the proof of Lemma \ref{lemma-adjoint}
that $\mathcal{F}$ is in the kernel of $Q$ if and only if
$\mathcal{F}|_{U_\etale} = 0$.
In particular, if $\mathcal{F}$ is parasitic then $\mathcal{F}$ is in
the kernel. Next, let $x : V \to \mathcal{X}$ be a flat morphism, where
$V$ is a scheme. Set $W = V \times_\mathcal{X} U$ and consider the diagram
$$
\xymatrix{
W \ar[d]_p \ar[r]_q & V \ar[d] \\
U \ar[r] & \mathcal{X}
}
$$
Note that the projection $p : W \to U$ is flat and the projection
$q : W \to V$ is smooth and surjective. This implies that $q_{small}^*$
is a faithful functor on quasi-coherent modules. By assumption $\mathcal{F}$
has the flat base change property so that we obtain
$p_{small}^*\mathcal{F}|_{U_\etale} \cong
q_{small}^*\mathcal{F}|_{V_\etale}$. Thus if $\mathcal{F}$
is in the kernel of $Q$, then $\mathcal{F}|_{V_\etale} = 0$
which completes the proof of (1).

\medskip\noindent
Part (2) follows from the discussion above and the fact
that the map $Q(\mathcal{F}) \to \mathcal{F}$ becomes an isomorphism after
restricting to $U_\etale$.

\medskip\noindent
To see part (3) note that $Q$ is left exact as a right adjoint.
Let $0 \to \mathcal{F} \to \mathcal{G} \to \mathcal{H} \to 0$
be a short exact sequence in $\textit{LQCoh}^{fbc}(\mathcal{O}_\mathcal{X})$.
Consider the following commutative diagram
$$
\xymatrix{
0 \ar[r] &
Q(\mathcal{F}) \ar[r] \ar[d]_a &
Q(\mathcal{G}) \ar[r] \ar[d]_b &
Q(\mathcal{H}) \ar[r] \ar[d]_c & 0 \\
0 \ar[r] &
\mathcal{F} \ar[r] &
\mathcal{G} \ar[r] &
\mathcal{H} \ar[r] & 0
}
$$
Since the kernels and cokernels of $a$, $b$, and $c$ are parasitic by
part (2) and since the bottom row is a short exact sequence, we see that
the top row as a complex of $\mathcal{O}_\mathcal{X}$-modules has parasitic
cohomology sheaves (details omitted; this uses that the category of
parasitic modules is a Serre subcategory of the category of all modules).
By left exactness of $Q$ we see that only exactness at $Q(\mathcal{H})$
is at issue. However, the cokernel $\mathcal{Q}$ of
$Q(\mathcal{G}) \to Q(\mathcal{H}))$
may be computed either in $\textit{Mod}(\mathcal{O}_\mathcal{X})$
or in $\QCoh(\mathcal{O}_\mathcal{X})$ with the same result because
the inclusion functor $\QCoh(\mathcal{O}_\mathcal{X}) \to
\textit{LQCoh}^{fbc}(\mathcal{O}_\mathcal{X})$
is a left adjoint and hence right exact. Hence $\mathcal{Q} = Q(\mathcal{Q})$
is both quasi-coherent and parasitic, whence $0$ by part (1) as desired.

\medskip\noindent
As a right adjoint $Q$ commutes with all limits. Since $Q$ is exact, to
show that $Q$ commutes with all colimits it suffices to show that $Q$
commutes with direct sums, see
Categories, Lemma \ref{categories-lemma-colimits-coproducts-coequalizers}.
Let $\mathcal{F}_i$, $i \in I$ be a family of objects of
$\textit{LQCoh}^{fbc}(\mathcal{O}_\mathcal{X})$. To see that
$Q(\bigoplus \mathcal{F}_i)$ is equal to $\bigoplus Q(\mathcal{F}_i)$
we look at the construction of $Q$ in the proof of
Lemma \ref{lemma-adjoint}. This uses a presentation $\mathcal{X} = [U/R]$
where $U$ is a scheme. Then $Q(\mathcal{F})$
is computed by first taking the pair
$(\mathcal{F}|_{U_\etale}, \alpha)$ in $\QCoh(U, R, s, t, c)$
and then using the equivalence
$\QCoh(U, R, s, t, c) \cong \QCoh(\mathcal{O}_\mathcal{X})$.
Since the restriction functor $\textit{Mod}(\mathcal{O}_\mathcal{X}) \to
\textit{Mod}(\mathcal{O}_{U_\etale})$,
$\mathcal{F} \mapsto \mathcal{F}|_{U_\etale}$ commutes with
direct sums, the desired equality is clear.
\end{proof}

\begin{lemma}
\label{lemma-flat-morphism-and-Q}
Let $f : \mathcal{X} \to \mathcal{Y}$ be a flat morphism of algebraic stacks.
Then $Q_\mathcal{X} \circ f^* = f^* \circ Q_\mathcal{Y}$ where
$Q_\mathcal{X}$ and $Q_\mathcal{Y}$ are as in Lemma \ref{lemma-adjoint}.
\end{lemma}

\begin{proof}
Observe that $f^*$ preserves both $\QCoh$ and $\textit{LQCoh}^{fbc}$,
see Sheaves on Stacks, Lemma
\ref{stacks-sheaves-lemma-pullback-quasi-coherent} and
Proposition \ref{proposition-loc-qcoh-flat-base-change}.
If $\mathcal{F}$ is in $\textit{LQCoh}^{fbc}(\mathcal{O}_\mathcal{Y})$
then $Q_\mathcal{Y}(\mathcal{F}) \to \mathcal{F}$
has parasitic kernel and cokernel
by Lemma \ref{lemma-adjoint-kernel-parasitic}.
As $f$ is flat we get that $f^*Q_\mathcal{Y}(\mathcal{F}) \to f^*\mathcal{F}$
has parasitic kernel and cokernel by
Lemma \ref{lemma-parasitic}.
Thus the induced map
$f^*Q_\mathcal{Y}(\mathcal{F}) \to Q_\mathcal{X}(f^*\mathcal{F})$
has parasitic kernel and cokernel and hence is an isomorphism
for example by
Lemma \ref{lemma-exact-sequence-quasi-coherent-parasitic-cohomology}.
\end{proof}

\begin{lemma}
\label{lemma-flat-object-and-Q}
Let $\mathcal{X}$ be an algebraic stack. Let $x$ be an object of $\mathcal{X}$
lying over the scheme $U$ such that $x : U \to \mathcal{X}$ is flat.
Then for $\mathcal{F}$ in $\QCoh^{fbc}(\mathcal{O}_\mathcal{X})$
we have $Q(\mathcal{F})|_{U_\etale} = \mathcal{F}|_{U_\etale}$.
\end{lemma}

\begin{proof}
True because the kernel and cokernel of $Q(\mathcal{F}) \to \mathcal{F}$
are parasitic, see Lemma \ref{lemma-adjoint-kernel-parasitic}.
\end{proof}

\begin{remark}
\label{remark-QCoh-abelian}
Let $\mathcal{X}$ be an algebraic stack. The category
$\QCoh(\mathcal{O}_\mathcal{X})$ is abelian, the inclusion functor
$\QCoh(\mathcal{O}_\mathcal{X}) \to \textit{Mod}(\mathcal{O}_\mathcal{X})$
is right exact, but not exact in general, see Sheaves on Stacks, Lemma
\ref{stacks-sheaves-lemma-quasi-coherent-algebraic-stack}.
We can use the functor $Q$ from
Lemmas \ref{lemma-adjoint} and \ref{lemma-adjoint-kernel-parasitic}
to understand this. Namely, let $\varphi : \mathcal{F} \to \mathcal{G}$
be a map of quasi-coherent $\mathcal{O}_\mathcal{X}$-modules. Then
\begin{enumerate}
\item the cokernel $\Coker(\varphi)$ computed in
$\textit{Mod}(\mathcal{O}_\mathcal{X})$ is quasi-coherent and
is the cokernel of $\varphi$ in $\QCoh(\mathcal{O}_\mathcal{X})$,
\item the image $\Im(\varphi)$ computed in
$\textit{Mod}(\mathcal{O}_\mathcal{X})$ is quasi-coherent and
is the image of $\varphi$ in $\QCoh(\mathcal{O}_\mathcal{X})$, and
\item the kernel $\Ker(\varphi)$ computed in
$\textit{Mod}(\mathcal{O}_\mathcal{X})$
is in $\textit{LQCoh}^{fbc}(\mathcal{O}_\mathcal{X})$
by Proposition \ref{proposition-loc-qcoh-flat-base-change} and
$Q(\Ker(\varphi))$ is the kernel in $\QCoh(\mathcal{O}_\mathcal{X})$.
\end{enumerate}
This follows from the references given.
\end{remark}

\begin{remark}
\label{remark-QCoh-tensor}
Let $\mathcal{X}$ be an algebraic stack. Given two quasi-coherent
$\mathcal{O}_\mathcal{X}$-modules $\mathcal{F}$ and $\mathcal{G}$
the tensor product module
$\mathcal{F} \otimes_{\mathcal{O}_\mathcal{X}} \mathcal{G}$
is quasi-coherent, see Sheaves on Stacks, Lemma
\ref{stacks-sheaves-lemma-quasi-coherent-algebraic-stack} part (5).
Similarly, given two locally quasi-coherent modules with
the flat base change property, their tensor product has the
same property, see Proposition \ref{proposition-loc-qcoh-flat-base-change}.
Thus the inclusion functors
$$
\QCoh(\mathcal{O}_\mathcal{X}) \to
\textit{LQCoh}^{fbc}(\mathcal{O}_\mathcal{X}) \to
\textit{Mod}(\mathcal{O}_\mathcal{X})
$$
are functors of symmetric monoidal categories. What is more interesting
is that the functor
$$
Q :
\textit{LQCoh}^{fbc}(\mathcal{O}_\mathcal{X})
\longrightarrow
\QCoh(\mathcal{O}_\mathcal{X})
$$
is a functor of symmetric monoidal categories as well. Namely, given
$\mathcal{F}$ and $\mathcal{G}$ in
$\textit{LQCoh}^{fbc}(\mathcal{O}_\mathcal{X})$ we obtain
$$
\xymatrix{
Q(\mathcal{F})
\otimes_{\mathcal{O}_\mathcal{X}}
Q(\mathcal{G}) \ar[rr] \ar[rd] & &
\mathcal{F}
\otimes_{\mathcal{O}_\mathcal{X}}
\mathcal{G} \\
&
Q(\mathcal{F} \otimes_{\mathcal{O}_\mathcal{X}} \mathcal{G}) \ar[ru]
}
$$
where the south-west arrow comes from the universal property
of the north-west arrow (and the fact already mentioned that
the object in the upper left corner is quasi-coherent).
If we restrict this diagram to $U_\etale$ for $U \to \mathcal{X}$
flat, then all three arrows become isomorphisms (see
Lemmas \ref{lemma-adjoint} and \ref{lemma-adjoint-kernel-parasitic}
and Definition \ref{definition-parasitic}).
Hence $Q(\mathcal{F}) \otimes_{\mathcal{O}_\mathcal{X}}
Q(\mathcal{G}) \to
Q(\mathcal{F} \otimes_{\mathcal{O}_\mathcal{X}} \mathcal{G})$
is an isomorphism, see for example
Lemma \ref{lemma-quasi-coherent-check-exact}.
\end{remark}

\begin{remark}
\label{remark-bousfield-colocalization}
Let $\mathcal{X}$ be an algebraic stack. Let
$\textit{Parasitic}(\mathcal{O}_\mathcal{X}) \subset
\textit{Mod}(\mathcal{O}_\mathcal{X})$ denote the
full subcategory consiting of parasitic modules. The results of
Lemmas \ref{lemma-adjoint} and \ref{lemma-adjoint-kernel-parasitic}
imply that
$$
\QCoh(\mathcal{O}_\mathcal{X}) =
\textit{LQCoh}^{fbc}(\mathcal{O}_\mathcal{X}) /
\textit{Parasitic}(\mathcal{O}_\mathcal{X})
\cap \textit{LQCoh}^{fbc}(\mathcal{O}_\mathcal{X})
$$
in words: the category of quasi-coherent modules is the category
of locally quasi-coherent modules with the flat base change property
divided out by the Serre subcategory consisting of parasitic objects.
See Homology, Lemma \ref{homology-lemma-serre-subcategory-is-kernel}.
The existence of the inclusion functor
$i : \QCoh(\mathcal{O}_\mathcal{X}) \to
\textit{LQCoh}^{fbc}(\mathcal{O}_\mathcal{X})$
which is left adjoint to the quotient functor is a key feature of
the situation. In Derived Categories of Stacks, Section
\ref{stacks-perfect-section-derived} and especially
Lemma \ref{stacks-perfect-lemma-bousfield-colocalization}
we prove that a similar result holds on the level of derived categories.
\end{remark}

\begin{lemma}
\label{lemma-internal-hom-fp-into-qcoh}
Let $\mathcal{X}$ be an algebraic stack. Let $\mathcal{F}$ be an
$\mathcal{O}_\mathcal{X}$-module of finite presentation and let
$\mathcal{G}$ be a quasi-coherent $\mathcal{O}_\mathcal{X}$-module.
The internal homs
$\SheafHom_{\mathcal{O}_\mathcal{X}}(\mathcal{F}, \mathcal{G})$
computed in
$\textit{Mod}(\mathcal{X}_\etale, \mathcal{O}_\mathcal{X})$ or
$\textit{Mod}(\mathcal{O}_\mathcal{X})$ agree and the common value
is an object of $\textit{LQCoh}^{fbc}(\mathcal{O}_\mathcal{X})$.
The quasi-coherent module
$
hom(\mathcal{F}, \mathcal{G}) =
Q(\SheafHom_{\mathcal{O}_\mathcal{X}}(\mathcal{F}, \mathcal{G}))
$
has the following universal property
$$
\Hom_\mathcal{X}(\mathcal{H}, hom(\mathcal{F}, \mathcal{G})) =
\Hom_\mathcal{X}(\mathcal{H} \otimes_{\mathcal{O}_\mathcal{X}} \mathcal{F},
\mathcal{G})
$$
for $\mathcal{H}$ in $\QCoh(\mathcal{O}_\mathcal{X})$.
\end{lemma}

\begin{proof}
The construction of
$\SheafHom_{\mathcal{O}_\mathcal{X}}(\mathcal{F}, \mathcal{G})$ in
Modules on Sites, Section \ref{sites-modules-section-internal-hom}
depends only on $\mathcal{F}$ and $\mathcal{G}$ as presheaves
of modules; the output $\SheafHom$ is a sheaf for the fppf
topology because $\mathcal{F}$ and $\mathcal{G}$ are assumed
sheaves in the fppf topology, see
Modules on Sites, Lemma \ref{sites-modules-lemma-internal-hom}.
By Sheaves on Stacks, Lemma \ref{stacks-sheaves-lemma-lqc-colimits}
we see that $\SheafHom_{\mathcal{O}_\mathcal{X}}(\mathcal{F}, \mathcal{G})$
is locally quasi-coherent.
By Lemma \ref{lemma-check-flat-comparison-on-etale-covering}
we see that $\SheafHom_{\mathcal{O}_\mathcal{X}}(\mathcal{F}, \mathcal{G})$
has the flat base change property. Hence
$\SheafHom_{\mathcal{O}_\mathcal{X}}(\mathcal{F}, \mathcal{G})$
is an object of $\textit{LQCoh}^{fbc}(\mathcal{O}_\mathcal{X})$
and it makes sense to apply the functor $Q$ of Lemma \ref{lemma-adjoint}
to it. By the universal property of $Q$ we have
$$
\Hom_\mathcal{X}(\mathcal{H},
Q(\SheafHom_{\mathcal{O}_\mathcal{X}}(\mathcal{F}, \mathcal{G}))) =
\Hom_\mathcal{X}(\mathcal{H},
\SheafHom_{\mathcal{O}_\mathcal{X}}(\mathcal{F}, \mathcal{G}))
$$
for $\mathcal{H}$ quasi-coherent, hence the displayed formula 
of the lemma follows from Modules on Sites, Lemma
\ref{sites-modules-lemma-internal-hom-adjoint-tensor}.
\end{proof}

\begin{lemma}
\label{lemma-flat-morphism-and-hom}
Let $f : \mathcal{X} \to \mathcal{Y}$ be a flat morphism of algebraic stacks.
Let $\mathcal{F}$ be an
$\mathcal{O}_\mathcal{Y}$-module of finite presentation and let
$\mathcal{G}$ be a quasi-coherent $\mathcal{O}_\mathcal{Y}$-module.
Then $f^*hom(\mathcal{F}, \mathcal{G}) = hom(f^*\mathcal{F}, f^*\mathcal{G})$
with notation as in Lemma \ref{lemma-internal-hom-fp-into-qcoh}.
\end{lemma}

\begin{proof}
We have $f^*\SheafHom_{\mathcal{O}_\mathcal{Y}}(\mathcal{F}, \mathcal{G}) =
\SheafHom_{\mathcal{O}_\mathcal{X}}(f^*\mathcal{F}, f^*\mathcal{G})$
by Modules on Sites, Lemma \ref{sites-modules-lemma-pullback-internal-hom}.
(Observe that this step is not where the flatness of $f$ is used
as the morphism of ringed topoi associated to $f$ is always flat, see
Sheaves on Stacks, Remark \ref{stacks-sheaves-remark-flat}.)
Then apply Lemma \ref{lemma-flat-morphism-and-Q} (and here we do use
flatness of $f$).
\end{proof}








\section{Pushforward of quasi-coherent modules}
\label{section-pushforward-quasi-coherent}

\noindent
Let $f : \mathcal{X} \to \mathcal{Y}$ be a morphism of algebraic stacks.
Consider the pushforward
$$
f_* :
\textit{Mod}(\mathcal{O}_\mathcal{X})
\longrightarrow
\textit{Mod}(\mathcal{O}_\mathcal{Y})
$$
It turns out that this functor almost never preserves the subcategories
of quasi-coherent sheaves. For example, consider the morphism of schemes
$$
j : X = \mathbf{A}^2_k \setminus \{0\} \longrightarrow \mathbf{A}^2_k = Y.
$$
Associated to this we have the corresponding morphism of algebraic stacks
$$
f = j_{big} : \mathcal{X} = (\Sch/X)_{fppf} \to
(\Sch/Y)_{fppf} = \mathcal{Y}
$$
The pushforward $f_*\mathcal{O}_\mathcal{X}$ of the structure sheaf has
global sections $k[x, y]$. Hence if $f_*\mathcal{O}_\mathcal{X}$ is
quasi-coherent on $\mathcal{Y}$ then we would have
$f_*\mathcal{O}_\mathcal{X} = \mathcal{O}_\mathcal{Y}$. However,
consider $T = \Spec(k) \to \mathbf{A}^2_k = Y$ mapping to $0$.
Then $\Gamma(T, f_*\mathcal{O}_\mathcal{X}) = 0$ because
$X \times_Y T = \emptyset$ whereas $\Gamma(T, \mathcal{O}_\mathcal{Y}) = k$.
On the positive side, for any flat morphism $T \to Y$ we have the equality
$\Gamma(T, f_*\mathcal{O}_\mathcal{X}) = \Gamma(T, \mathcal{O}_\mathcal{Y})$
as follows from
Cohomology of Schemes, Lemma \ref{coherent-lemma-flat-base-change-cohomology}
using that $j$ is quasi-compact and quasi-separated.

\medskip\noindent
Let $f : \mathcal{X} \to \mathcal{Y}$ be a quasi-compact and
quasi-separated morphism of algebraic stacks. We work around the problem
mentioned above using the following three observations:
\begin{enumerate}
\item $f_*$ does preserve locally quasi-coherent
modules (Lemma \ref{lemma-pushforward-locally-quasi-coherent}),
\item $f_*$ transforms a quasi-coherent sheaf into a locally quasi-coherent
sheaf whose flat comparison maps are isomorphisms
(Lemma \ref{lemma-flat-comparison}), and
\item locally quasi-coherent $\mathcal{O}_\mathcal{Y}$-modules
with the flat base change property give rise to quasi-coherent
modules on a presentation of $\mathcal{Y}$ and hence quasi-coherent
modules on $\mathcal{Y}$, see
Sheaves on Stacks, Section
\ref{stacks-sheaves-section-quasi-coherent-algebraic-stacks}.
\end{enumerate}
Thus we obtain a functor
$$
f_{\QCoh, *} :
\QCoh(\mathcal{O}_\mathcal{X})
\longrightarrow
\QCoh(\mathcal{O}_\mathcal{Y})
$$
which is a right adjoint to
$f^* : \QCoh(\mathcal{O}_\mathcal{Y}) \to
\QCoh(\mathcal{O}_\mathcal{X})$
such that moreover
$$
\Gamma(y, f_*\mathcal{F}) = \Gamma(y, f_{\QCoh, *}\mathcal{F})
$$
for any $y \in \Ob(\mathcal{Y})$ such that the associated
$1$-morphism $y : V \to \mathcal{Y}$ is flat, see
Lemma \ref{lemma-direct-image-quasi-coherent-over-flat}.
Moreover, a similar construction will produce functors
$R^if_{\QCoh, *}$.
However, these results will not be sufficient to produce a
total direct image functor (of complexes with quasi-coherent
cohomology sheaves).

\begin{proposition}
\label{proposition-direct-image-quasi-coherent}
Let $f : \mathcal{X} \to \mathcal{Y}$ be a quasi-compact and quasi-separated
morphism of algebraic stacks. The functor
$f^* : \QCoh(\mathcal{O}_\mathcal{Y}) \to
\QCoh(\mathcal{O}_\mathcal{X})$
has a right adjoint
$$
f_{\QCoh, *} :
\QCoh(\mathcal{O}_\mathcal{X})
\longrightarrow
\QCoh(\mathcal{O}_\mathcal{Y})
$$
which can be defined as the composition
$$
\QCoh(\mathcal{O}_\mathcal{X}) \to
\textit{LQCoh}^{fbc}(\mathcal{O}_\mathcal{X})
\xrightarrow{f_*} \textit{LQCoh}^{fbc}(\mathcal{O}_\mathcal{Y})
\xrightarrow{Q} \QCoh(\mathcal{O}_\mathcal{Y})
$$
where the functors $f_*$ and $Q$ are as in
Proposition \ref{proposition-loc-qcoh-flat-base-change}
and
Lemma \ref{lemma-adjoint}.
Moreover, if we define $R^if_{\QCoh, *}$ as the composition
$$
\QCoh(\mathcal{O}_\mathcal{X}) \to
\textit{LQCoh}^{fbc}(\mathcal{O}_\mathcal{X})
\xrightarrow{R^if_*} \textit{LQCoh}^{fbc}(\mathcal{O}_\mathcal{Y})
\xrightarrow{Q} \QCoh(\mathcal{O}_\mathcal{Y})
$$
then the sequence of functors $\{R^if_{\QCoh, *}\}_{i \geq 0}$
forms a cohomological $\delta$-functor.
\end{proposition}

\begin{proof}
This is a combination of the results mentioned in the statement.
The adjointness can be shown as follows: Let $\mathcal{F}$
be a quasi-coherent $\mathcal{O}_\mathcal{X}$-module and let
$\mathcal{G}$ be a quasi-coherent $\mathcal{O}_\mathcal{Y}$-module.
Then we have
\begin{align*}
\Mor_{\QCoh(\mathcal{O}_\mathcal{X})}(f^*\mathcal{G}, \mathcal{F})
& =
\Mor_{\textit{LQCoh}^{fbc}(\mathcal{O}_\mathcal{Y})}
(\mathcal{G}, f_*\mathcal{F}) \\
& =
\Mor_{\QCoh(\mathcal{O}_\mathcal{Y})}(\mathcal{G}, Q(f_*\mathcal{F}))
\\
& =
\Mor_{\QCoh(\mathcal{O}_\mathcal{Y})}(\mathcal{G},
f_{\QCoh, *}\mathcal{F})
\end{align*}
the first equality by adjointness of $f_*$ and $f^*$ (for arbitrary sheaves
of modules). By
Proposition \ref{proposition-loc-qcoh-flat-base-change}
we see that $f_*\mathcal{F}$ is an object of
$\textit{LQCoh}^{fbc}(\mathcal{O}_\mathcal{Y})$
(and can be computed in either the fppf or \'etale topology) and we
obtain the second equality by Lemma \ref{lemma-adjoint}. The third
equality is the definition of $f_{\QCoh, *}$.

\medskip\noindent
To see that $\{R^if_{\QCoh, *}\}_{i \geq 0}$ is a cohomological
$\delta$-functor as defined in
Homology, Definition \ref{homology-definition-cohomological-delta-functor}
let
$$
0 \to \mathcal{F}_1 \to \mathcal{F}_2 \to \mathcal{F}_3 \to 0
$$
be a short exact sequence of $\QCoh(\mathcal{O}_\mathcal{X})$.
This sequence may not be an exact sequence in
$\textit{Mod}(\mathcal{O}_\mathcal{X})$ but we know that it is
up to parasitic modules, see
Lemma \ref{lemma-exact-sequence-quasi-coherent-parasitic-cohomology}.
Thus we may break up the sequence into short exact sequences
$$
\begin{matrix}
0 \to \mathcal{P}_1 \to \mathcal{F}_1 \to \mathcal{I}_2 \to 0 \\
0 \to \mathcal{I}_2 \to \mathcal{F}_2 \to \mathcal{Q}_2 \to 0 \\
0 \to \mathcal{P}_2 \to \mathcal{Q}_2 \to \mathcal{I}_3 \to 0 \\
0 \to \mathcal{I}_3 \to \mathcal{F}_3 \to \mathcal{P}_3 \to 0
\end{matrix}
$$
of $\textit{Mod}(\mathcal{O}_\mathcal{X})$ with $\mathcal{P}_i$ parasitic.
Note that each of the sheaves
$\mathcal{P}_j$, $\mathcal{I}_j$, $\mathcal{Q}_j$ is an object of
$\textit{LQCoh}^{fbc}(\mathcal{O}_\mathcal{X})$, see
Proposition \ref{proposition-loc-qcoh-flat-base-change}.
Applying $R^if_*$ we obtain long exact sequences 
$$
\begin{matrix}
0 \to f_*\mathcal{P}_1 \to f_*\mathcal{F}_1 \to f_*\mathcal{I}_2 \to
R^1f_*\mathcal{P}_1 \to \ldots \\
0 \to f_*\mathcal{I}_2 \to f_*\mathcal{F}_2 \to f_*\mathcal{Q}_2 \to
R^1f_*\mathcal{I}_2 \to \ldots \\
0 \to f_*\mathcal{P}_2 \to f_*\mathcal{Q}_2 \to f_*\mathcal{I}_3 \to
R^1f_*\mathcal{P}_2 \to \ldots \\
0 \to f_*\mathcal{I}_3 \to f_*\mathcal{F}_3 \to f_*\mathcal{P}_3 \to
R^1f_*\mathcal{I}_3 \to \ldots
\end{matrix}
$$
where are the terms are objects of
$\textit{LQCoh}^{fbc}(\mathcal{O}_\mathcal{Y})$ by
Proposition \ref{proposition-loc-qcoh-flat-base-change}.
By
Lemma \ref{lemma-pushforward-parasitic}
the sheaves $R^if_*\mathcal{P}_j$ are parasitic, hence vanish on applying
the functor $Q$, see
Lemma \ref{lemma-adjoint-kernel-parasitic}.
Since $Q$ is exact the maps
$$
Q(R^if_*\mathcal{F}_3)
\cong
Q(R^if_*\mathcal{I}_3)
\cong
Q(R^if_*\mathcal{Q}_2)
\rightarrow
Q(R^{i + 1}f_*\mathcal{I}_2)
\cong
Q(R^{i + 1}f_*\mathcal{F}_1)
$$
can serve as the connecting map which turns the family of functors
$\{R^if_{\QCoh, *}\}_{i \geq 0}$
into a cohomological $\delta$-functor.
\end{proof}

\begin{lemma}
\label{lemma-direct-image-quasi-coherent-over-flat}
Let $f : \mathcal{X} \to \mathcal{Y}$ be a quasi-compact and quasi-separated
morphism of algebraic stacks. Let $y : V \to \mathcal{Y}$ in $\Ob(\mathcal{Y})$
with $y$ a flat morphism. Let $\mathcal{F}$ be in
$\QCoh(\mathcal{O}_\mathcal{X})$.
Then $(f_*\mathcal{F})(y) = (f_{\QCoh, *}\mathcal{F})(y)$
and $(R^if_*\mathcal{F})(y) = (R^if_{\QCoh, *}\mathcal{F})(y)$
for all $i \in \mathbf{Z}$.
\end{lemma}

\begin{proof}
This follows from the construction of the functors $R^if_{\QCoh, *}$ in
Proposition \ref{proposition-direct-image-quasi-coherent},
the definition of parasitic modules in Definition \ref{definition-parasitic},
and Lemma \ref{lemma-adjoint-kernel-parasitic} part (2).
\end{proof}

\begin{remark}
\label{remark-direct-image-quasi-coherent-tensor}
Let $f : \mathcal{X} \to \mathcal{Y}$ be a quasi-compact and quasi-separated
morphism of algebraic stacks. Let $\mathcal{F}$ and $\mathcal{G}$ be in
$\QCoh(\mathcal{O}_\mathcal{X})$. Then there is a canonical commutative
diagram
$$
\xymatrix{
f_{\QCoh, *}\mathcal{F}
\otimes_{\mathcal{O}_\mathcal{Y}}
f_{\QCoh, *}\mathcal{G} \ar[r] \ar[d] &
f_*\mathcal{F}
\otimes_{\mathcal{O}_\mathcal{Y}}
f_*\mathcal{G} \ar[d]^c \\
f_{\QCoh, *}(\mathcal{F}
\otimes_{\mathcal{O}_\mathcal{X}}
\mathcal{G}) \ar[r] &
f_*(\mathcal{F}
\otimes_{\mathcal{O}_\mathcal{X}}
\mathcal{G})
}
$$
The vertical arrow $c$ on the right is the naive relative cup product
(in degree $0$),
see Cohomology on Sites, Section \ref{sites-cohomology-section-cup-product}.
The source and target of $c$ are in
$\textit{LQCoh}^{fbc}(\mathcal{O}_\mathcal{X})$, see
Proposition \ref{proposition-loc-qcoh-flat-base-change}.
Applying $Q$ to $c$ we obtain the left vertical arrow
as $Q$ commutes with tensor products, see
Remark \ref{remark-QCoh-tensor}.
This construction is functorial in $\mathcal{F}$ and $\mathcal{G}$.
\end{remark}

\begin{lemma}
\label{lemma-leray}
Let $f : \mathcal{X} \to \mathcal{Y}$
be a quasi-compact and quasi-separated morphism of algebraic stacks.
Let $\mathcal{F}$ be a quasi-coherent sheaf on $\mathcal{X}$. Then
there exists a spectral sequence with $E_2$-page
$$
E_2^{p, q} = H^p(\mathcal{Y}, R^qf_{\QCoh, *}\mathcal{F})
$$
converging to $H^{p + q}(\mathcal{X}, \mathcal{F})$.
\end{lemma}

\begin{proof}
By Cohomology on Sites, Lemma \ref{sites-cohomology-lemma-Leray}
the Leray spectral sequence with
$$
E_2^{p, q} = H^p(\mathcal{Y}, R^qf_*\mathcal{F})
$$
converges to $H^{p + q}(\mathcal{X}, \mathcal{F})$. The kernel and cokernel
of the adjunction map
$$
R^qf_{\QCoh, *}\mathcal{F} \longrightarrow R^qf_*\mathcal{F}
$$
are parasitic modules on $\mathcal{Y}$
(Lemma \ref{lemma-adjoint-kernel-parasitic})
hence have vanishing cohomology
(Lemma \ref{lemma-pushforward-parasitic}).
It follows formally that
$H^p(\mathcal{Y}, R^qf_{\QCoh, *}\mathcal{F}) =
H^p(\mathcal{Y}, R^qf_*\mathcal{F})$ and we win.
\end{proof}

\begin{lemma}
\label{lemma-relative-leray}
Let $f : \mathcal{X} \to \mathcal{Y}$ and $g : \mathcal{Y} \to \mathcal{Z}$
be quasi-compact and quasi-separated morphisms of algebraic stacks.
Let $\mathcal{F}$ be a quasi-coherent sheaf on $\mathcal{X}$. Then
there exists a spectral sequence with $E_2$-page
$$
E_2^{p, q} = R^pg_{\QCoh, *}(R^qf_{\QCoh, *}\mathcal{F})
$$
converging to $R^{p + q}(g \circ f)_{\QCoh, *}\mathcal{F}$.
\end{lemma}

\begin{proof}
By Cohomology on Sites, Lemma \ref{sites-cohomology-lemma-relative-Leray}
the Leray spectral sequence with
$$
E_2^{p, q} = R^pg_*(R^qf_*\mathcal{F})
$$
converges to $R^{p + q}(g \circ f)_*\mathcal{F}$. By the results of
Proposition \ref{proposition-loc-qcoh-flat-base-change}
all the terms of this spectral sequence are objects of
$\textit{LQCoh}^{fbc}(\mathcal{O}_\mathcal{Z})$. Applying the exact functor
$Q_\mathcal{Z} : \textit{LQCoh}^{fbc}(\mathcal{O}_\mathcal{Z}) \to
\QCoh(\mathcal{O}_\mathcal{Z})$ we obtain a spectral sequence in
$\QCoh(\mathcal{O}_\mathcal{Z})$ covering to
$R^{p + q}(g \circ f)_{\QCoh, *}\mathcal{F}$. Hence
the result follows if we can show that
$$
Q_\mathcal{Z}(R^pg_*(R^qf_*\mathcal{F})) =
Q_\mathcal{Z}(R^pg_*(Q_\mathcal{X}(R^qf_*\mathcal{F}))
$$
This follows from the fact that the kernel and cokernel of the map
$$
Q_\mathcal{X}(R^qf_*\mathcal{F}) \longrightarrow R^qf_*\mathcal{F}
$$
are parasitic (Lemma \ref{lemma-adjoint-kernel-parasitic}) and that
$R^pg_*$ transforms parasitic modules into parasitic modules
(Lemma \ref{lemma-pushforward-parasitic}).
\end{proof}

\noindent
To end this section we make explicit the spectral sequences
associated to a smooth covering by a scheme. Please compare with
Sheaves on Stacks, Sections \ref{stacks-sheaves-section-cohomology} and
\ref{stacks-sheaves-section-higher-direct-images}.

\begin{proposition}
\label{proposition-smooth-covering-compute-cohomology}
Let $f : \mathcal{U} \to \mathcal{X}$ be a morphism of algebraic stacks.
Assume $f$ is representable by algebraic spaces, surjective, flat, and
locally of finite presentation. Let $\mathcal{F}$ be a quasi-coherent
$\mathcal{O}_\mathcal{X}$-module. Then there is a spectral sequence
$$
E_2^{p, q} = H^q(\mathcal{U}_p, f_p^*\mathcal{F})
\Rightarrow
H^{p + q}(\mathcal{X}, \mathcal{F})
$$
where $f_p$ is the morphism
$\mathcal{U} \times_\mathcal{X} \ldots \times_\mathcal{X} \mathcal{U} \to
\mathcal{X}$ ($p + 1$ factors).
\end{proposition}

\begin{proof}
This is a special case of
Sheaves on Stacks, Proposition
\ref{stacks-sheaves-proposition-smooth-covering-compute-cohomology}.
\end{proof}

\begin{proposition}
\label{proposition-smooth-covering-compute-direct-image}
Let $f : \mathcal{U} \to \mathcal{X}$ and $g : \mathcal{X} \to \mathcal{Y}$
be composable morphisms of algebraic stacks.
Assume that
\begin{enumerate}
\item $f$ is representable by algebraic spaces, surjective,
flat, locally of finite presentation, quasi-compact, and quasi-separated, and
\item $g$ is quasi-compact and quasi-separated.
\end{enumerate}
If $\mathcal{F}$ is in $\QCoh(\mathcal{O}_\mathcal{X})$ then
there is a spectral sequence
$$
E_2^{p, q} = R^q(g \circ f_p)_{\QCoh, *}f_p^*\mathcal{F}
\Rightarrow
R^{p + q}g_{\QCoh, *}\mathcal{F}
$$
in $\QCoh(\mathcal{O}_\mathcal{Y})$.
\end{proposition}

\begin{proof}
Note that each of the morphisms
$f_p : \mathcal{U} \times_\mathcal{X} \ldots \times_\mathcal{X} \mathcal{U} \to
\mathcal{X}$ is quasi-compact and quasi-separated, hence $g \circ f_p$
is quasi-compact and quasi-separated, hence the assertion makes sense
(i.e., the functors $R^q(g \circ f_p)_{\QCoh, *}$ are defined).
There is a spectral sequence
$$
E_2^{p, q} = R^q(g \circ f_p)_*f_p^{-1}\mathcal{F}
\Rightarrow
R^{p + q}g_*\mathcal{F}
$$
by Sheaves on Stacks, Proposition
\ref{stacks-sheaves-proposition-smooth-covering-compute-direct-image}.
Applying the exact functor
$Q_\mathcal{Y} : \textit{LQCoh}^{fbc}(\mathcal{O}_\mathcal{Y}) \to
\QCoh(\mathcal{O}_\mathcal{Y})$ gives the desired spectral sequence in
$\QCoh(\mathcal{O}_\mathcal{Y})$.
\end{proof}







\section{Further remarks on quasi-coherent modules}
\label{section-further-remarks}

\noindent
In this section we collect some results that to help understand how
to use quasi-coherent modules on algebraic stacks.

\medskip\noindent
Let $f : \mathcal{U} \to \mathcal{X}$ be a morphism of algebraic stacks.
Assume $\mathcal{U}$ is represented by the algebraic space $U$.
Consider the functor
$$
a :
\textit{Mod}(\mathcal{X}_\etale, \mathcal{O}_\mathcal{X})
\longrightarrow
\textit{Mod}(U_\etale, \mathcal{O}_U),\quad
\mathcal{F}
\longmapsto
f^*\mathcal{F}|_{U_\etale}
$$
given by pullback (Sheaves on Stacks, Section
\ref{stacks-sheaves-section-modules}) followed by restriction
(Sheaves on Stacks, Section
\ref{stacks-sheaves-section-restriction-algebraic-spaces}).
Applying this functor to locally quasi-coherent modules we obtain a functor
$$
b : \textit{LQCoh}(\mathcal{O}_\mathcal{X})
\longrightarrow
\QCoh(U_\etale, \mathcal{O}_U)
$$
See Sheaves on Stacks, Lemmas \ref{stacks-sheaves-lemma-pullback-lqc} and
\ref{stacks-sheaves-lemma-compare-locally-quasi-coherent}.
We can further limit our functor to even smaller subcategories to obtain
$$
c :
\textit{LQCoh}^{fbc}(\mathcal{O}_\mathcal{X})
\longrightarrow
\QCoh(U_\etale, \mathcal{O}_U)
$$
and
$$
d :
\QCoh(\mathcal{O}_\mathcal{X})
\longrightarrow
\QCoh(U_\etale, \mathcal{O}_U)
$$
About these functors we can say the following:\footnote{We suggest
working out why these statements are true on a napkin instead of
following the references given.}
\begin{enumerate}
\item The functor $a$ is exact. Namely, pullback $f^* = f^{-1}$ is exact 
(Sheaves on Stacks, Section \ref{stacks-sheaves-section-modules})
and restriction to $U_\etale$ is exact, see
Sheaves on Stacks, Equation (\ref{stacks-sheaves-equation-restrict}).
\item The functor $b$ is exact. Namely, by
Sheaves on Stacks, Lemma \ref{stacks-sheaves-lemma-lqc-colimits}
the inclusion $\textit{LQCoh}(\mathcal{O}_\mathcal{X}) \to
\textit{Mod}(\mathcal{X}_\etale, \mathcal{O}_\mathcal{X})$ is exact.
\item The functor $c$ is exact. Namely, by
Proposition \ref{proposition-loc-qcoh-flat-base-change}
the inclusion functor $\textit{LQCoh}^{fbc}(\mathcal{O}_\mathcal{X}) \to
\textit{Mod}(\mathcal{X}_\etale, \mathcal{O}_\mathcal{X})$ is exact.
\item The functor $d$ is right exact but not exact in general.
Namely, by Sheaves on Stacks, Lemma
\ref{stacks-sheaves-lemma-qc-colimits}
the inclusion functor $\QCoh(\mathcal{O}_\mathcal{X}) \to
\textit{Mod}(\mathcal{X}_\etale, \mathcal{O}_\mathcal{X})$ is right
exact. We omit giving an example showing non-exactness.
\item If $f$ is flat, then $d$ is exact. This follows on combining
Lemma \ref{lemma-flat-pullback-quasi-coherent} and
Sheaves on Stacks, Lemma \ref{stacks-sheaves-lemma-compare-quasi-coherent}.
\item If $f$ is flat, then $c$ kills parasitic objects.
Namely, $f^*$ preserves parasitic object by Lemma \ref{lemma-parasitic}.
Then for any scheme $V$ \'etale over $U$ and hence flat over $\mathcal{X}$
we see that
$0 = f^*\mathcal{F}|_{V_\etale} = c(\mathcal{F})|_{V_\etale}$
by the compatibility of restriction with \'etale localization
Sheaves on Stacks, Remark \ref{stacks-sheaves-remark-compare}.
Hence clearly $c(\mathcal{F}) = 0$.
\item If $f$ is flat, then $c = d \circ Q$. Namely, the kernel and
cokernel of $Q(\mathcal{F}) \to \mathcal{F}$ are parasitic by
Lemma \ref{lemma-adjoint-kernel-parasitic}. Thus, since
$c$ is exact (3) and kills parasitic objects (6),
we see that $c$ applied to $Q(\mathcal{F}) \to \mathcal{F}$
is an isomorphism.
\item The functors $a, b, c, d$ commute with colimits
and arbitrary direct sums. This is true for $f^*$ and restriction
as left adjoints and hence it holds for $a$. Then it follows for
$b$, $c$, $d$ by the references given above.
\item The functors $a, b, c, d$ commute with tensor products.
\item If $f$ is flat and surjective, $\mathcal{F}$ is in
$\textit{LQCoh}^{fbc}(\mathcal{O}_\mathcal{X})$, and $c(\mathcal{F}) = 0$,
then $\mathcal{F}$ is parasitic. Namely, by (7) we get $d(Q(\mathcal{F})) = 0$.
We may assume $U$ is a scheme by the compatibility of restriction
with \'etale localization (see reference above). Then
Lemma \ref{lemma-quasi-coherent-check-exact} applied to $0 \to Q(\mathcal{F})$
and the morphism $f : U \to \mathcal{X}$ shows that $Q(\mathcal{F}) = 0$.
Thus $\mathcal{F}$ is parasitic by Lemma \ref{lemma-adjoint-kernel-parasitic}.
\item If $f$ is flat and surjective, then the functor $d$
reflects exactness. More precisely, let $\mathcal{F}^\bullet$ be a complex in
$\QCoh(\mathcal{O}_\mathcal{X})$. Then $\mathcal{F}^\bullet$ is exact
in $\QCoh(\mathcal{O}_\mathcal{X})$ if and only if $d(\mathcal{F}^\bullet)$
is exact. Namely, we have seen one implication in (5). For the other,
suppose that $H^i(d(\mathcal{F}^\bullet)) = 0$. Then
$\mathcal{G} = H^i(\mathcal{F}^\bullet)$ is an object of
$\QCoh(\mathcal{O}_\mathcal{X})$ with $d(\mathcal{G}) = 0$.
Hence $\mathcal{G}$ is both quasi-coherent and parasitic by (10),
whence $0$ for example by Remark \ref{remark-bousfield-colocalization}.
\item
\label{item-hom-restriction}
If $f$ is flat,
$\mathcal{F}, \mathcal{G} \in \Ob(\QCoh(\mathcal{O}_\mathcal{X}))$,
and $\mathcal{F}$ of finite presentation and let
then we have
$$
d(hom(\mathcal{F}, \mathcal{G})) =
\SheafHom_{\mathcal{O}_U}(d(\mathcal{F}), d(\mathcal{G}))
$$
with notation as in Lemma \ref{lemma-internal-hom-fp-into-qcoh}.
Perhaps the easiest way to see this is as follows
\begin{align*}
d(hom(\mathcal{F}, \mathcal{G}))
& =
d(Q(\SheafHom_{\mathcal{O}_\mathcal{X}}(\mathcal{F}, \mathcal{G}))) \\
& =
c(\SheafHom_{\mathcal{O}_\mathcal{X}}(\mathcal{F}, \mathcal{G})) \\
& =
f^*\SheafHom_{\mathcal{O}_\mathcal{X}}(\mathcal{F}, \mathcal{G})|_{U_\etale} \\
& =
\SheafHom_{\mathcal{O}_\mathcal{U}}(f^*\mathcal{F},
f^*\mathcal{G})|_{U_\etale} \\
& =
\SheafHom_{\mathcal{O}_U}(f^*\mathcal{F}|_{U_\etale},
f^*\mathcal{G}|_{U_\etale})
\end{align*}
The first equality by construction of $hom$. The second equality
by (7). The third equality by definition of $c$. The fourth equality
by Modules on Sites, Lemma \ref{sites-modules-lemma-pullback-internal-hom}.
The final equality by the same reference applied to the flat morphism
of ringed topoi $i_U (U_\etale, \mathcal{O}_U) \to
(\mathcal{U}_\etale, \mathcal{O}_\mathcal{U})$ of
Sheaves on Stacks, Lemma \ref{stacks-sheaves-lemma-compare}.
\item add more here.
\end{enumerate}






\section{Colimits and cohomology}
\label{section-colimits}

\noindent
The following lemma in particular applies to diagrams of quasi-coherent
sheaves.

\begin{lemma}
\label{lemma-colimits}
Let $\mathcal{X}$ be a quasi-compact and quasi-separated algebraic stack.
Then
$$
\colim_i H^p(\mathcal{X}, \mathcal{F}_i)
\longrightarrow
H^p(\mathcal{X}, \colim_i \mathcal{F}_i)
$$
is an isomorphism for every filtered diagram of abelian sheaves on
$\mathcal{X}$. The same is true for abelian sheaves on $\mathcal{X}_\etale$
taking cohomology in the \'etale topology.
\end{lemma}

\begin{proof}
Let $\tau = fppf$, resp.\ $\tau = \etale$. The lemma follows from
Cohomology on Sites, Lemma \ref{sites-cohomology-lemma-colim-global}
applied to the site $\mathcal{X}_\tau$. In order to check the assumptions we
use Cohomology on Sites, Remark \ref{sites-cohomology-remark-colim-global}.
Namely, let $\mathcal{B} \subset \Ob(\mathcal{X}_\tau)$ be the set
of objects lying over affine schemes. In other words, an element
of $\mathcal{B}$ is a morphism $x : U \to \mathcal{X}$ with $U$ affine.
We check each of the conditions (1) -- (4) of the remark in turn:
\begin{enumerate}
\item Since $\mathcal{X}$ is quasi-compact, there exists a
surjetive and smooth morphism $x : U \to \mathcal{X}$ with $U$ affine
(Properties of Stacks, Lemma
\ref{stacks-properties-lemma-quasi-compact-stack}).
Then $h_x^\# \to *$ is a surjective map of sheaves on $\mathcal{X}_\tau$.
\item Since coverings in $\mathcal{X}_\tau$ are fppf, resp.\ \'etale
coverings, we see that every covering of $U \in \mathcal{B}$ is refined
by a finite affine fppf covering, see
Topologies, Lemma \ref{topologies-lemma-fppf-affine},
resp.\ Lemma \ref{topologies-lemma-etale-affine}.
\item Let $x : U \to \mathcal{X}$ and $x' : U' \to \mathcal{X}$ be
in $\mathcal{B}$. The product $h_x^\# \times h_{x'}^\#$ in
$\Sh(\mathcal{X}_\tau)$ is equal to the sheaf on $\mathcal{X}_\tau$
determined by the algebraic space $W = U \times_{x, \mathcal{X}, x'} U'$ over
$\mathcal{X}$: for an object $y : V \to \mathcal{X}$ of
$\mathcal{X}_\tau$ we have
$(h_x^\# \times h_{x'}^\#)(y) = \{f : V \to W \mid y =
x \circ \text{pr}_1 \circ f = x' \circ \text{pr}_2 \circ f\}$.
The algebraic space $W$ is quasi-compact because $\mathcal{X}$
is quasi-separated, see Morphisms of Stacks, Lemma
\ref{stacks-morphisms-lemma-quasi-compact-quasi-separated-permanence}
for example. Hence we can choose an affine scheme $U''$ and
a surjective \'etale morphism $U'' \to W$. Denote $x'' : U'' \to \mathcal{X}$
the composition of $U'' \to W$ and $W \to \mathcal{X}$. Then
$h_{x''}^\# \to h_x^\# \times h_{x'}^\#$ is surjective as desired.
\item Let $x : U \to \mathcal{X}$ and $x' : U' \to \mathcal{X}$ be
in $\mathcal{B}$. Let $a, b : U \to U'$ be a morphism over $\mathcal{X}$,
i.e., $a, b  : x \to x'$ is a morphism in $\mathcal{X}_\tau$.
Then the equalizer of $h_a$ and $h_b$ is represented by the
equalizer of $a, b : U \to U'$ which is
affine scheme over $\mathcal{X}$ and hence in $\mathcal{B}$.
\end{enumerate}
This finished the proof.
\end{proof}

\begin{lemma}
\label{lemma-colimit-cohomology}
Let $f : \mathcal{X} \to \mathcal{Y}$ be a quasi-compact and quasi-separated
morphism of algebraic stacks. Let $\mathcal{F} = \colim \mathcal{F}_i$
be a filtered colimit of abelian sheaves on $\mathcal{X}$.
Then for any $p \geq 0$ we have
$$
R^pf_*\mathcal{F} = \colim R^pf_*\mathcal{F}_i.
$$
The same is true for abelian sheaves on $\mathcal{X}_\etale$
taking higher direct images in the \'etale topology.
\end{lemma}

\begin{proof}
We will prove this for the fppf topology; the proof for the \'etale
topology is the same. Recall that $R^if_*\mathcal{F}$ is the sheaf on
$\mathcal{Y}_{fppf}$ associated to the presheaf
$$
(y : V \to \mathcal{Y})
\longmapsto
H^i(V \times_{y, \mathcal{Y}} \mathcal{X}, \text{pr}^{-1}\mathcal{F})
$$
See
Sheaves on Stacks, Lemma \ref{stacks-sheaves-lemma-pushforward-restriction}.
Recall that the colimit is the sheaf associated to the presheaf colimit.
When $V$ is affine, the fibre product $V \times_\mathcal{Y} \mathcal{X}$
is quasi-compact and quasi-separated. Hence we can apply
Lemma \ref{lemma-colimits} to $H^p(V \times_\mathcal{Y} \mathcal{X}, -)$
where $V$ is affine. Since every $V$ has an fppf covering by
affine objects this proves the lemma. Some details omitted.
\end{proof}

\begin{lemma}
\label{lemma-quasi-coherent-pushforward-direct-sums}
Let $f : \mathcal{X} \to \mathcal{Y}$ be a quasi-compact and quasi-separated
morphism of algebraic stacks. The functor $f_{\QCoh, *}$
and the functors $R^if_{\QCoh, *}$ commute with direct sums
and filtered colimits.
\end{lemma}

\begin{proof}
The functors $f_*$ and $R^if_*$ commute with direct sums and filtered
colimits on all modules by Lemma \ref{lemma-colimit-cohomology}.
The lemma follows as $f_{\QCoh, *} = Q \circ f_*$ and
$R^if_{\QCoh, *} = Q \circ R^if_*$ and $Q$ commutes with
all colimits, see Lemma \ref{lemma-adjoint-kernel-parasitic}.
\end{proof}

\begin{lemma}
\label{lemma-quasi-coherent-pushforward-affine}
Let $f : \mathcal{X} \to \mathcal{Y}$ be an affine morphism of algebraic
stacks. The functors $R^if_{\QCoh, *}$, $i > 0$ vanish and the functor
$f_{\QCoh, *}$ is exact and commutes with direct sums and all colimits.
\end{lemma}

\begin{proof}
Since we have $R^if_{\QCoh, *} = Q \circ R^if_*$ we obtain the
vanishing from Lemma \ref{lemma-loc-qcoh-fbc-affine-direct-image}.
The vanishing implies that $f_{\QCoh, *}$ is exact as
$\{R^if_{\QCoh, *}\}_{i \geq 0}$ form a $\delta$-functor, see
Proposition \ref{proposition-direct-image-quasi-coherent}.
Then $f_{\QCoh, *}$ commutes with direct sums for example by
Lemma \ref{lemma-quasi-coherent-pushforward-direct-sums}.
An exact functor which commutes with direct sums commutes
with all colimits.
\end{proof}

\noindent
The following lemma tells us that finitely presented modules behave
as expected in quasi-compact and quasi-separated algebraic stacks.

\begin{lemma}
\label{lemma-finite-presentation-quasi-compact-colimit}
Let $\mathcal{X}$ be a quasi-compact and quasi-separated algebraic stack.
Let $I$ be a directed set and let $(\mathcal{F}_i, \varphi_{ii'})$ be a
system over $I$ of $\mathcal{O}_\mathcal{X}$-modules. Let $\mathcal{G}$ be an
$\mathcal{O}_\mathcal{X}$-module of finite presentation. Then we have
$$
\colim_i \Hom_\mathcal{X}(\mathcal{G}, \mathcal{F}_i)
=
\Hom_\mathcal{X}(\mathcal{G}, \colim_i \mathcal{F}_i).
$$
In particular, $\Hom_\mathcal{X}(\mathcal{G}, -)$ commutes with filtered
colimits in $\QCoh(\mathcal{O}_\mathcal{X})$.
\end{lemma}

\begin{proof}
The displayed equality is a special case of Modules on Sites, Lemma
\ref{sites-modules-lemma-finite-presentation-quasi-compact-colimit}.
In order to apply it, we need to check the hypotheses of
Sites, Lemma \ref{sites-lemma-directed-colimits-global-sections} part (4)
for the site $\mathcal{X}_{fppf}$. In order to do this, we will check
hypotheses (2)(a), (2)(b), (2)(c) of
Sites, Remark \ref{sites-remark-stronger-conditions}.
Namely, let $\mathcal{B} \subset \Ob(\mathcal{X}_{fppf})$ be the set
of objects lying over affine schemes. In other words, an element
of $\mathcal{B}$ is a morphism $x : U \to \mathcal{X}$ with $U$ affine.
We check each of the conditions (2)(a), (2)(b), and (2)(c)
of the remark in turn:
\begin{enumerate}
\item Since $\mathcal{X}$ is quasi-compact, there exists a
surjetive and smooth morphism $x : U \to \mathcal{X}$ with $U$ affine
(Properties of Stacks, Lemma
\ref{stacks-properties-lemma-quasi-compact-stack}).
Then $h_x^\# \to *$ is a surjective map of sheaves on $\mathcal{X}_{fppf}$.
\item Since coverings in $\mathcal{X}_{fppf}$ are fppf
coverings, we see that every covering of $U \in \mathcal{B}$ is refined
by a finite affine fppf covering, see
Topologies, Lemma \ref{topologies-lemma-fppf-affine}.
\item Let $x : U \to \mathcal{X}$ and $x' : U' \to \mathcal{X}$ be
in $\mathcal{B}$. The product $h_x^\# \times h_{x'}^\#$ in
$\Sh(\mathcal{X}_{fppf})$ is equal to the sheaf on $\mathcal{X}_{fppf}$
determined by the algebraic space $W = U \times_{x, \mathcal{X}, x'} U'$ over
$\mathcal{X}$: for an object $y : V \to \mathcal{X}$ of
$\mathcal{X}_{fppf}$ we have
$(h_x^\# \times h_{x'}^\#)(y) = \{f : V \to W \mid y =
x \circ \text{pr}_1 \circ f = x' \circ \text{pr}_2 \circ f\}$.
The algebraic space $W$ is quasi-compact because $\mathcal{X}$
is quasi-separated, see Morphisms of Stacks, Lemma
\ref{stacks-morphisms-lemma-quasi-compact-quasi-separated-permanence}
for example. Hence we can choose an affine scheme $U''$ and
a surjective \'etale morphism $U'' \to W$. Denote $x'' : U'' \to \mathcal{X}$
the composition of $U'' \to W$ and $W \to \mathcal{X}$. Then
$h_{x''}^\# \to h_x^\# \times h_{x'}^\#$ is surjective as desired.
\end{enumerate}
For the final statement, observe that the inclusion functor
$\QCoh(\mathcal{O}_X) \to \textit{Mod}(\mathcal{O}_X)$
commutes with colimits and that finitely presented modules
are quasi-coherent. See Sheaves on Stacks, Lemma
\ref{stacks-sheaves-lemma-quasi-coherent-algebraic-stack}.
\end{proof}



\section{The lisse-\'etale and the flat-fppf sites}
\label{section-lisse-etale}

\noindent
In the book \cite{LM-B} many of the results above are proved using the
lisse-\'etale site of an algebraic stack. We define this site here.
In Examples, Section \ref{examples-section-lisse-etale-not-functorial}
we show that the lisse-\'etale site isn't functorial.
We also define its analogue, the flat-fppf site, which is better suited
to the development of algebraic stacks as given in the Stacks project
(because we use the fppf topology as our base topology). Of course the
flat-fppf site isn't functorial either.

\begin{definition}
\label{definition-lisse-etale}
Let $\mathcal{X}$ be an algebraic stack.
\begin{enumerate}
\item The {\it lisse-\'etale site} of $\mathcal{X}$ is the full subcategory
$\mathcal{X}_{lisse,\etale}$\footnote{In the literature the
site is denoted $\text{Lis-\'et}(\mathcal{X})$ or
$\text{Lis-Et}(\mathcal{X})$ and the associated topos is denoted
$\mathcal{X}_{\text{lis-\'e}t}$ or $\mathcal{X}_{\text{lis-et}}$.
In the Stacks project our convention is to name the site and
denote the corresponding topos by $\Sh(\mathcal{C})$.} of $\mathcal{X}$
whose objects are those $x \in \Ob(\mathcal{X})$ lying over a scheme $U$
such that $x : U \to \mathcal{X}$ is smooth. A covering of
$\mathcal{X}_{lisse,\etale}$ is a family of morphisms
$\{x_i \to x\}_{i \in I}$ of $\mathcal{X}_{lisse,\etale}$
which forms a covering of $\mathcal{X}_\etale$.
\item The {\it flat-fppf site} of $\mathcal{X}$ is the full subcategory
$\mathcal{X}_{flat,fppf}$ of $\mathcal{X}$
whose objects are those $x \in \Ob(\mathcal{X})$ lying over a scheme $U$
such that $x : U \to \mathcal{X}$ is flat. A covering of
$\mathcal{X}_{flat,fppf}$ is a family of morphisms
$\{x_i \to x\}_{i \in I}$ of $\mathcal{X}_{flat,fppf}$
which forms a covering of $\mathcal{X}_{fppf}$.
\end{enumerate}
\end{definition}

\noindent
We denote $\mathcal{O}_{\mathcal{X}_{lisse,\etale}}$
the restriction of $\mathcal{O}_\mathcal{X}$ to the lisse-\'etale site
and similarly for $\mathcal{O}_{\mathcal{X}_{flat,fppf}}$.
The relationship between the lisse-\'etale site and the \'etale site is
as follows (we mainly stick to ``topological'' properties in this lemma).

\begin{lemma}
\label{lemma-lisse-etale}
Let $\mathcal{X}$ be an algebraic stack.
\begin{enumerate}
\item The inclusion functor
$\mathcal{X}_{lisse,\etale} \to \mathcal{X}_\etale$
is fully faithful, continuous and cocontinuous. It follows that
\begin{enumerate}
\item there is a morphism of topoi
$$
g :
\Sh(\mathcal{X}_{lisse,\etale})
\longrightarrow
\Sh(\mathcal{X}_\etale)
$$
with $g^{-1}$ given by restriction,
\item the functor $g^{-1}$ has a left adjoint $g_!^{Sh}$ on sheaves of sets,
\item the adjunction maps $g^{-1}g_* \to \text{id}$ and
$\text{id} \to g^{-1}g_!^{Sh}$ are isomorphisms,
\item the functor $g^{-1}$ has a left adjoint $g_!$ on abelian sheaves,
\item the adjunction map $\text{id} \to g^{-1}g_!$ is an isomorphism, and
\item we have $g^{-1}\mathcal{O}_\mathcal{X} =
\mathcal{O}_{\mathcal{X}_{lisse,\etale}}$ hence $g$ induces a flat
morphism of ringed topoi such that $g^{-1} = g^*$.
\end{enumerate}
\item The inclusion functor
$\mathcal{X}_{flat,fppf} \to \mathcal{X}_{fppf}$
is fully faithful, continuous and cocontinuous. It follows that
\begin{enumerate}
\item there is a morphism of topoi
$$
g :
\Sh(\mathcal{X}_{flat,fppf})
\longrightarrow
\Sh(\mathcal{X}_{fppf})
$$
with $g^{-1}$ given by restriction,
\item the functor $g^{-1}$ has a left adjoint $g_!^{Sh}$ on sheaves of sets,
\item the adjunction maps $g^{-1}g_* \to \text{id}$ and
$\text{id} \to g^{-1}g_!^{Sh}$ are isomorphisms,
\item the functor $g^{-1}$ has a left adjoint $g_!$ on abelian sheaves,
\item the adjunction map $\text{id} \to g^{-1}g_!$ is an isomorphism, and
\item we have $g^{-1}\mathcal{O}_\mathcal{X} =
\mathcal{O}_{\mathcal{X}_{flat,fppf}}$ hence $g$ induces a flat
morphism of ringed topoi such that $g^{-1} = g^*$.
\end{enumerate}
\end{enumerate}
\end{lemma}

\begin{proof}
In both cases it is immediate that the functor is fully faithful,
continuous, and cocontinuous (see
Sites, Definitions \ref{sites-definition-continuous} and
\ref{sites-definition-cocontinuous}).
Hence properties (a), (b), (c) follow from
Sites, Lemmas \ref{sites-lemma-when-shriek} and
\ref{sites-lemma-back-and-forth}.
Parts (d), (e) follow from
Modules on Sites, Lemmas \ref{sites-modules-lemma-g-shriek-adjoint} and
\ref{sites-modules-lemma-back-and-forth}.
Part (f) is immediate.
\end{proof}

\begin{lemma}
\label{lemma-lisse-etale-cohomology}
Let $\mathcal{X}$ be an algebraic stack. Notation as in
Lemma \ref{lemma-lisse-etale}.
\begin{enumerate}
\item For an abelian sheaf $\mathcal{F}$ on $\mathcal{X}_\etale$ we have
\begin{enumerate}
\item $H^p(\mathcal{X}_\etale, \mathcal{F}) =
H^p(\mathcal{X}_{lisse,\etale}, g^{-1}\mathcal{F})$, and
\item $H^p(x, \mathcal{F}) =
H^p(\mathcal{X}_{lisse,\etale}/x, g^{-1}\mathcal{F})$
for any object $x$ of $\mathcal{X}_{lisse,\etale}$.
\end{enumerate}
The same holds for sheaves of modules.
\item For an abelian sheaf $\mathcal{F}$ on $\mathcal{X}_{fppf}$ we have
\begin{enumerate}
\item $H^p(\mathcal{X}_{fppf}, \mathcal{F}) =
H^p(\mathcal{X}_{flat,fppf}, g^{-1}\mathcal{F})$, and
\item $H^p(x, \mathcal{F}) =
H^p(\mathcal{X}_{flat,fppf}/x, g^{-1}\mathcal{F})$
for any object $x$ of $\mathcal{X}_{flat,fppf}$.
\end{enumerate}
The same holds for sheaves of modules.
\end{enumerate}
\end{lemma}

\begin{proof}
Part (1)(a) follows from Sheaves on Stacks, Lemma
\ref{stacks-sheaves-lemma-cohomology-on-subcategory} applied to the inclusion
functor $\mathcal{X}_{lisse,\etale} \to \mathcal{X}_\etale$.
Part (1)(b) follows from part (1)(a). Namely, if $x$ lies over the
scheme $U$, then the site $\mathcal{X}_\etale/x$ is equivalent
to $(\Sch/U)_\etale$ and $\mathcal{X}_{lisse,\etale}$ is equivalent
to $U_{lisse,\etale}$. Part (2) is proved in the same manner.
\end{proof}

\begin{lemma}
\label{lemma-lisse-etale-modules}
Let $\mathcal{X}$ be an algebraic stack. Notation as in
Lemma \ref{lemma-lisse-etale}.
\begin{enumerate}
\item There exists a functor
$$
g_! :
\textit{Mod}(\mathcal{X}_{lisse,\etale},
\mathcal{O}_{\mathcal{X}_{lisse,\etale}})
\longrightarrow
\textit{Mod}(\mathcal{X}_\etale, \mathcal{O}_{\mathcal{X}})
$$
which is left adjoint to $g^*$. Moreover it agrees with the functor $g_!$
on abelian sheaves and $g^*g_! = \text{id}$.
\item There exists a functor
$$
g_! :
\textit{Mod}(\mathcal{X}_{flat,fppf},
\mathcal{O}_{\mathcal{X}_{flat,fppf}})
\longrightarrow
\textit{Mod}(\mathcal{X}_{fppf}, \mathcal{O}_{\mathcal{X}})
$$
which is left adjoint to $g^*$. Moreover it agrees with the functor $g_!$
on abelian sheaves and $g^*g_! = \text{id}$.
\end{enumerate}
\end{lemma}

\begin{proof}
In both cases, the existence of the functor $g_!$ follows from
Modules on Sites, Lemma \ref{sites-modules-lemma-lower-shriek-modules}.
To see that $g_!$ agrees with the functor on abelian sheaves we will
show the maps Modules on Sites, Equation
(\ref{sites-modules-equation-compare-on-localizations})
are isomorphisms.

\medskip\noindent
Lisse-\'etale case. Let $x \in \Ob(\mathcal{X}_{lisse,\etale})$
lying over a scheme $U$ with $x : U \to \mathcal{X}$ smooth.
Consider the induced fully faithful functor
$$
g' :
\mathcal{X}_{lisse,\etale}/x
\longrightarrow
\mathcal{X}_\etale/x
$$
The right hand side is identified with $(\Sch/U)_\etale$ and the
left hand side with the full subcategory of schemes $U'/U$ such that
the composition $U' \to U \to \mathcal{X}$ is smooth. Thus
\'Etale Cohomology, Lemma
\ref{etale-cohomology-lemma-compare-structure-sheaves}
applies.

\medskip\noindent
Flat-fppf case. Let $x \in \Ob(\mathcal{X}_{flat,fppf})$
lying over a scheme $U$ with $x : U \to \mathcal{X}$ flat.
Consider the induced fully faithful functor
$$
g' :
\mathcal{X}_{flat,fppf}/x
\longrightarrow
\mathcal{X}_{fppf}/x
$$
The right hand side is identified with $(\Sch/U)_{fppf}$ and the
left hand side with the full subcategory of schemes $U'/U$ such that
the composition $U' \to U \to \mathcal{X}$ is flat. Thus
\'Etale Cohomology, Lemma
\ref{etale-cohomology-lemma-compare-structure-sheaves}
applies.

\medskip\noindent
In both cases the equality $g^*g_! = \text{id}$ follows from
$g^* = g^{-1}$ and the
equality for abelian sheaves in Lemma \ref{lemma-lisse-etale}.
\end{proof}

\begin{lemma}
\label{lemma-lisse-etale-structure-sheaf}
Let $\mathcal{X}$ be an algebraic stack. Notation as in
Lemmas \ref{lemma-lisse-etale} and \ref{lemma-lisse-etale-modules}.
\begin{enumerate}
\item We have $g_!\mathcal{O}_{\mathcal{X}_{lisse,\etale}} =
\mathcal{O}_\mathcal{X}$.
\item We have $g_!\mathcal{O}_{\mathcal{X}_{flat, fppf}} =
\mathcal{O}_\mathcal{X}$.
\end{enumerate}
\end{lemma}

\begin{proof}
In this proof we write
$\mathcal{C} = \mathcal{X}_\etale$
(resp.\ $\mathcal{C} = \mathcal{X}_{fppf}$)
and we denote
$\mathcal{C}' = \mathcal{X}_{lisse,\etale}$
(resp.\ $\mathcal{C}' = \mathcal{X}_{flat, fppf}$).
Then $\mathcal{C}'$ is a full subcategory of $\mathcal{C}$.
In this proof we will think of objects $V$ of $\mathcal{C}$
as schemes over $\mathcal{X}$ and objects $U$ of $\mathcal{C}'$
as schemes smooth (resp.\ flat) over $\mathcal{X}$.
Finally, we write $\mathcal{O} = \mathcal{O}_\mathcal{X}$
and $\mathcal{O}' = \mathcal{O}_{\mathcal{X}_{lisse,\etale}}$
(resp.\ $\mathcal{O}' = \mathcal{O}_{\mathcal{X}_{flat,fppf}}$).
In the notation above we have $\mathcal{O}(V) = \Gamma(V, \mathcal{O}_V)$
and $\mathcal{O}'(U) = \Gamma(U, \mathcal{O}_U)$.
Consider the $\mathcal{O}$-module homomorphism
$g_!\mathcal{O}' \to \mathcal{O}$
adjoint to the identification $\mathcal{O}' = g^{-1}\mathcal{O}$.

\medskip\noindent
Recall that $g_!\mathcal{O}'$ is the sheaf associated to the presheaf
$g_{p!}\mathcal{O}'$ given by the rule
$$
V \longmapsto \colim_{V \to U} \mathcal{O}'(U)
$$
where the colimit is taken in the category of abelian groups
(Modules on Sites, Definition \ref{sites-modules-definition-g-shriek}).
Below we will use frequently that if
$$
V \to U \to U'
$$
are morphisms and if $f' \in \mathcal{O}'(U')$ restricts to
$f \in \mathcal{O}'(U)$, then $(V \to U, f)$ and $(V \to U', f')$
define the same element of the colimit. Also,
$g_!\mathcal{O}' \to \mathcal{O}$ maps the element
$(V \to U, f)$ simply to the pullback of $f$ to $V$.

\medskip\noindent
Let us prove that $g_!\mathcal{O}' \to \mathcal{O}$ is surjective.
Let $h \in \mathcal{O}(V)$ for some object $V$ of $\mathcal{C}$.
It suffices to show that $h$ is locally in the image. Choose an object
$U$ of $\mathcal{C}'$ corresponding to a surjective smooth morphism
$U \to \mathcal{X}$. Since $U \times_\mathcal{X} V \to V$ is surjective
smooth, after replacing $V$ by the members of an \'etale covering of $V$
we may assume there exists a morphism $V \to U$, see
Topologies on Spaces, Lemma
\ref{spaces-topologies-lemma-etale-dominates-smooth}. Using $h$ we obtain
a morphism $V \to U \times \mathbf{A}^1$ such that writing
$\mathbf{A}^1 = \Spec(\mathbf{Z}[t])$ the element
$t \in \mathcal{O}(U \times \mathbf{A}^1)$ pulls back to $h$.
Since $U \times \mathbf{A}^1$ is an object of $\mathcal{C}'$
we see that $(V \to U \times \mathbf{A}^1, t)$
is an element of the colimit above which maps to $h \in \mathcal{O}(V)$
as desired.

\medskip\noindent
Suppose that $s \in g_!\mathcal{O}'(V)$ is a section
mapping to zero in $\mathcal{O}(V)$. To finish the proof we have to show
that $s$ is zero. After replacing $V$ by the members
of a covering we may assume $s$ is an element of the colimit
$$
\colim_{V \to U} \mathcal{O}'(U)
$$
Say $s = \sum (\varphi_i, s_i)$ is a finite sum with
$\varphi_i : V \to U_i$, $U_i$ smooth (resp.\ flat) over $\mathcal{X}$, and
$s_i \in \Gamma(U_i, \mathcal{O}_{U_i})$. Choose a scheme $W$ surjective
\'etale over the algebraic space
$U = U_1 \times_\mathcal{X} \ldots \times_\mathcal{X} U_n$.
Note that $W$ is still smooth (resp.\ flat) over $\mathcal{X}$, i.e.,
defines an object of $\mathcal{C}'$. The fibre product
$$
V' = V \times_{(\varphi_1, \ldots, \varphi_n), U} W
$$
is surjective \'etale over $V$, hence it suffices to show that $s$ maps
to zero in $g_!\mathcal{O}'(V')$. Note that the restriction
$\sum (\varphi_i, s_i)|_{V'}$ corresponds to the sum of the pullbacks
of the functions $s_i$ to $W$. In other words, we have reduced to the case
of $(\varphi, s)$ where $\varphi : V \to U$ is a morphism with $U$ in
$\mathcal{C}'$ and $s \in \mathcal{O}'(U)$ restricts to zero in
$\mathcal{O}(V)$. By the commutative diagram
$$
\xymatrix{
V \ar[rr]_-{(\varphi, 0)} \ar[rrd]_\varphi & & U \times \mathbf{A}^1 \\
& & U \ar[u]_{(\text{id}, 0)}
}
$$
we see that
$((\varphi, 0) : V \to U \times \mathbf{A}^1, \text{pr}_2^*x)$
represents zero in the colimit above. Hence we may
replace $U$ by $U \times \mathbf{A}^1$, $\varphi$ by $(\varphi, 0)$
and $s$ by $\text{pr}_1^*s + \text{pr}_2^*x$. Thus we may assume that
the vanishing locus $Z : s = 0$ in $U$ of $s$ is smooth (resp.\ flat)
over $\mathcal{X}$. Then we see that $(V \to Z, 0)$ and $(\varphi, s)$
have the same value in the colimit, i.e., we see that the element $s$
is zero as desired.
\end{proof}

\noindent
The lisse-\'etale and the flat-fppf sites can be used to characterize
parasitic modules as follows.

\begin{lemma}
\label{lemma-parasitic-in-terms-flat-fppf}
Let $\mathcal{X}$ be an algebraic stack.
\begin{enumerate}
\item Let $\mathcal{F}$ be an $\mathcal{O}_\mathcal{X}$-module
with the flat base change property on $\mathcal{X}_\etale$.
The following are equivalent
\begin{enumerate}
\item $\mathcal{F}$ is parasitic, and
\item $g^*\mathcal{F} = 0$ where
$g : \Sh(\mathcal{X}_{lisse,\etale}) \to
\Sh(\mathcal{X}_\etale)$ is as in Lemma \ref{lemma-lisse-etale}.
\end{enumerate}
\item Let $\mathcal{F}$ be an $\mathcal{O}_\mathcal{X}$-module on
$\mathcal{X}_{fppf}$. The following are equivalent
\begin{enumerate}
\item $\mathcal{F}$ is parasitic, and
\item $g^*\mathcal{F} = 0$ where
$g :  \Sh(\mathcal{X}_{flat,fppf}) \to \Sh(\mathcal{X}_{fppf})$
is as in Lemma \ref{lemma-lisse-etale}.
\end{enumerate}
\end{enumerate}
\end{lemma}

\begin{proof}
Part (2) is immediate from the definitions (this is one of the advantages
of the flat-fppf site over the lisse-\'etale site). The implication
(1)(a) $\Rightarrow$ (1)(b) is immediate as well. To see (1)(b)
$\Rightarrow$ (1)(a) let $U$ be a scheme and let $x : U \to \mathcal{X}$
be a surjective smooth morphism. Then $x$ is an object of the
lisse-\'etale site of $\mathcal{X}$. Hence we see that (1)(b)
implies that $\mathcal{F}|_{U_\etale} = 0$. Let $V \to \mathcal{X}$
be an flat morphism where $V$ is a scheme. Set $W = U \times_\mathcal{X} V$
and consider the diagram
$$
\xymatrix{
W \ar[d]_p \ar[r]_q & V \ar[d] \\
U \ar[r] & \mathcal{X}
}
$$
Note that the projection $p : W \to U$ is flat and the projection
$q : W \to V$ is smooth and surjective. This implies that $q_{small}^*$
is a faithful functor on quasi-coherent modules. By assumption $\mathcal{F}$
has the flat base change property so that we obtain
$p_{small}^*\mathcal{F}|_{U_\etale} \cong
q_{small}^*\mathcal{F}|_{V_\etale}$. Thus if $\mathcal{F}$
is in the kernel of $g^*$, then $\mathcal{F}|_{V_\etale} = 0$
as desired.
\end{proof}






\section{Functoriality of the lisse-\'etale and flat-fppf sites}
\label{section-lisse-etale-functorial}

\noindent
The lisse-\'etale site is functorial for smooth morphisms of algebraic stacks
and the flat-fppf site is functorial for flat morphisms of algebraic stacks.
We warn the reader that the lisse-\'etale and flat-fppf topoi are not
functorial with respect to all morphisms of algebraic stacks, see
Examples, Section \ref{examples-section-lisse-etale-not-functorial}.

\begin{lemma}
\label{lemma-lisse-etale-functorial}
Let $f : \mathcal{X} \to \mathcal{Y}$ be a morphism of algebraic stacks.
\begin{enumerate}
\item If $f$ is smooth, then $f$ restricts to a continuous and cocontinuous
functor
$\mathcal{X}_{lisse,\etale} \to \mathcal{Y}_{lisse,\etale}$
which gives a morphism of ringed topoi fitting into the following
commutative diagram
$$
\xymatrix{
\Sh(\mathcal{X}_{lisse,\etale}) \ar[r]_{g'} \ar[d]_{f'} &
\Sh(\mathcal{X}_\etale) \ar[d]^f \\
\Sh(\mathcal{Y}_{lisse,\etale}) \ar[r]^g &
\Sh(\mathcal{Y}_\etale)
}
$$
We have $f'_*(g')^{-1} = g^{-1}f_*$ and $g'_!(f')^{-1} = f^{-1}g_!$.
\item If $f$ is flat, then $f$ restricts to a continuous and cocontinuous
functor
$\mathcal{X}_{flat,fppf} \to \mathcal{Y}_{flat,fppf}$
which gives a morphism of ringed topoi fitting into the following
commutative diagram
$$
\xymatrix{
\Sh(\mathcal{X}_{flat,fppf}) \ar[r]_{g'} \ar[d]_{f'} &
\Sh(\mathcal{X}_{fppf}) \ar[d]^f \\
\Sh(\mathcal{Y}_{flat,fppf}) \ar[r]^g &
\Sh(\mathcal{Y}_{fppf})
}
$$
We have $f'_*(g')^{-1} = g^{-1}f_*$ and $g'_!(f')^{-1} = f^{-1}g_!$.
\end{enumerate}
\end{lemma}

\begin{proof}
The initial statement comes from the fact that if $x \in \Ob(\mathcal{X})$
lies over a scheme $U$ such that $x : U \to \mathcal{X}$ is smooth
(resp.\ flat) and if $f$ is smooth (resp.\ flat) then
$f(x) : U \to \mathcal{Y}$ is smooth (resp.\ flat), see
Morphisms of Stacks, Lemmas \ref{stacks-morphisms-lemma-composition-smooth} and
\ref{stacks-morphisms-lemma-composition-flat}. The induced functor
$\mathcal{X}_{lisse,\etale} \to \mathcal{Y}_{lisse,\etale}$
(resp.\ $\mathcal{X}_{flat,fppf} \to \mathcal{Y}_{flat,fppf}$) is
continuous and cocontinuous by our definition of coverings in these
categories. Finally, the commutativity of the diagram is a consequence of the
fact that the horizontal morphisms are given by the inclusion functors (see
Lemma \ref{lemma-lisse-etale}) and
Sites, Lemma \ref{sites-lemma-composition-cocontinuous}.

\medskip\noindent
To show that $f'_*(g')^{-1} = g^{-1}f_*$ let $\mathcal{F}$ be a sheaf
on $\mathcal{X}_\etale$ (resp.\ $\mathcal{X}_{fppf}$).
There is a canonical pullback map
$$
g^{-1}f_*\mathcal{F} \longrightarrow f'_*(g')^{-1}\mathcal{F}
$$
see Sites, Section \ref{sites-section-pullback}.
We claim this map is an isomorphism.
To prove this pick an object $y$ of $\mathcal{Y}_{lisse,\etale}$
(resp.\ $\mathcal{Y}_{flat,fppf}$). Say $y$ lies over the scheme $V$
such that $y : V \to \mathcal{Y}$ is smooth (resp.\ flat). Since
$g^{-1}$ is the restriction we find that
$$
\left(g^{-1}f_*\mathcal{F}\right)(y) =
\Gamma(V \times_{y, \mathcal{Y}} \mathcal{X},\ \text{pr}^{-1}\mathcal{F})
$$
by Sheaves on Stacks, Equation (\ref{stacks-sheaves-equation-pushforward}).
Let
$(V \times_{y, \mathcal{Y}} \mathcal{X})' \subset
V \times_{y, \mathcal{Y}} \mathcal{X}$
be the full subcategory consisting of objects
$z : W \to V \times_{y, \mathcal{Y}} \mathcal{X}$ such that the induced
morphism $W \to \mathcal{X}$ is smooth (resp.\ flat). Denote
$$
\text{pr}' :
(V \times_{y, \mathcal{Y}} \mathcal{X})'
\longrightarrow
\mathcal{X}_{lisse,\etale}
\ (\text{resp. }\mathcal{X}_{flat,fppf})
$$
the restriction of the functor $\text{pr}$ used in the formula above.
Exactly the same argument that proves
Sheaves on Stacks, Equation (\ref{stacks-sheaves-equation-pushforward})
shows that for any sheaf $\mathcal{H}$ on $\mathcal{X}_{lisse,\etale}$
(resp.\ $\mathcal{X}_{flat,fppf}$) we have
\begin{equation}
\label{equation-pushforward-lisse-etale}
f'_*\mathcal{H}(y) =
\Gamma((V \times_{y, \mathcal{Y}} \mathcal{X})',
\ (\text{pr}')^{-1}\mathcal{H})
\end{equation}
Since $(g')^{-1}$ is restriction we see that
$$
\left(f'_*(g')^{-1}\mathcal{F}\right)(y) =
\Gamma((V \times_{y, \mathcal{Y}} \mathcal{X})',
\ \text{pr}^{-1}\mathcal{F}|_{(V \times_{y, \mathcal{Y}} \mathcal{X})'})
$$
By
Sheaves on Stacks, Lemma \ref{stacks-sheaves-lemma-cohomology-on-subcategory}
we see that
$$
\Gamma((V \times_{y, \mathcal{Y}} \mathcal{X})',
\ \text{pr}^{-1}\mathcal{F}|_{(V \times_{y, \mathcal{Y}} \mathcal{X})'})
=
\Gamma(V \times_{y, \mathcal{Y}} \mathcal{X},\ \text{pr}^{-1}\mathcal{F})
$$
are equal as desired; although we omit the verification of the assumptions
of the lemma we note that the fact that $V \to \mathcal{Y}$ is smooth
(resp.\ flat) is used to verify the second condition.

\medskip\noindent
Finally, the equality $g'_!(f')^{-1} = f^{-1}g_!$ follows formally from
the equality $f'_*(g')^{-1} = g^{-1}f_*$ by the adjointness of
$f^{-1}$ and $f_*$, the adjointness of $g_!$ and $g^{-1}$, and their
``primed'' versions.
\end{proof}

\begin{lemma}
\label{lemma-lisse-etale-functorial-pushforward}
With assumptions and notation as in Lemma \ref{lemma-lisse-etale-functorial}.
Let $\mathcal{H}$  be an abelian sheaf on $\mathcal{X}_{lisse,\etale}$
(resp.\ $\mathcal{X}_{flat,fppf}$). Then
\begin{equation}
\label{equation-higher-direct-image-lisse-etale}
R^pf'_*\mathcal{H} =
\text{sheaf associated to }y \longmapsto
H^p((V \times_{y, \mathcal{Y}} \mathcal{X})', (\text{pr}')^{-1}\mathcal{H})
\end{equation}
Here $y$ is an object of $\mathcal{Y}_{lisse,\etale}$
(resp.\ $\mathcal{Y}_{flat,fppf}$) lying over the scheme $V$
and the notation $(V \times_{y, \mathcal{Y}} \mathcal{X})'$
and $\text{pr}'$ are explained in the proof.
\end{lemma}

\begin{proof}
As in the proof of Lemma \ref{lemma-lisse-etale-functorial}
let $(V \times_{y, \mathcal{Y}} \mathcal{X})' \subset
V \times_{y, \mathcal{Y}} \mathcal{X}$ be the full subcategory
consisting of objects $(x, \varphi)$ where $x$ is an object of
$\mathcal{X}_{lisse,\etale}$ (resp.\ $\mathcal{X}_{flat,fppf}$)
and $\varphi : f(x) \to y$ is a morphism in $\mathcal{Y}$.
By Equation (\ref{equation-pushforward-lisse-etale}) 
we have
$$
f'_*\mathcal{H}(y) =
\Gamma((V \times_{y, \mathcal{Y}} \mathcal{X})',
\ (\text{pr}')^{-1}\mathcal{H})
$$
where $\text{pr}'$ is the projection. For an object $(x, \varphi)$
of $(V \times_{y, \mathcal{Y}} \mathcal{X})'$
we can think of $\varphi$ as a section of $(f')^{-1}h_y$ over $x$.
Thus $(V \times_\mathcal{Y} \mathcal{X})'$ is the localization
of the site $\mathcal{X}_{lisse,\etale}$
(resp. $\mathcal{X}_{flat,fppf}$) at the sheaf of sets $(f')^{-1}h_y$, see
Sites, Lemma \ref{sites-lemma-localize-topos-site}. The morphism
$$
\text{pr}' : (V \times_{y, \mathcal{Y}} \mathcal{X})'
\to \mathcal{X}_{lisse,\etale}
\ (\text{resp. }
\text{pr}' : (V \times_{y, \mathcal{Y}} \mathcal{X})'
\to \mathcal{X}_{flat,fppf})
$$
is the localization morphism.
In particular, the pullback $(\text{pr}')^{-1}$ preserves
injective abelian sheaves, see
Cohomology on Sites, Lemma
\ref{sites-cohomology-lemma-cohomology-on-sheaf-sets}.

\medskip\noindent
Choose an injective resolution $\mathcal{H} \to \mathcal{I}^\bullet$
on $\mathcal{X}_{lisse,\etale}$ (resp.\ $\mathcal{X}_{flat,fppf}$).
By the formula for pushforward we see that $R^if'_*\mathcal{H}$ is the
sheaf associated to the presheaf which associates to $y$ the cohomology
of the complex
$$
\begin{matrix}
\Gamma\Big((V \times_{y, \mathcal{Y}} \mathcal{X})',
(\text{pr}')^{-1}\mathcal{I}^{i - 1}\Big) \\
\downarrow \\
\Gamma\Big((V \times_{y, \mathcal{Y}} \mathcal{X})',
(\text{pr}')^{-1}\mathcal{I}^i\Big) \\
\downarrow \\
\Gamma\Big((V \times_{y, \mathcal{Y}} \mathcal{X})',
(\text{pr}')^{-1}\mathcal{I}^{i + 1}\Big)
\end{matrix}
$$
Since $(\text{pr}')^{-1}$ is exact and preserves injectives the
complex $(\text{pr}')^{-1}\mathcal{I}^\bullet$ is an
injective resolution of $(\text{pr}')^{-1}\mathcal{H}$
and the proof is complete.
\end{proof}

\begin{lemma}
\label{lemma-lisse-etale-functorial-cohomology}
With assumptions and notation as in Lemma \ref{lemma-lisse-etale-functorial}
the canonical (base change) map
$$
g^{-1}Rf_*\mathcal{F} \longrightarrow Rf'_*(g')^{-1}\mathcal{F}
$$
is an isomorphism for any abelian sheaf $\mathcal{F}$
on $\mathcal{X}_\etale$ (resp.\ $\mathcal{X}_{fppf}$).
\end{lemma}

\begin{proof}
Comparing the formula for $g^{-1}R^pf_*\mathcal{F}$
and $R^pf'_*(g')^{-1}\mathcal{F}$ given in
Sheaves on Stacks, Lemma \ref{stacks-sheaves-lemma-pushforward-restriction}
and Lemma \ref{lemma-lisse-etale-functorial-pushforward}
we see that it suffices to show
$$
H^p((V \times_{y, \mathcal{Y}} \mathcal{X})',
\ \text{pr}^{-1}\mathcal{F}|_{(V \times_{y, \mathcal{Y}} \mathcal{X})'})
=
H^p_\tau(V \times_{y, \mathcal{Y}} \mathcal{X},\ \text{pr}^{-1}\mathcal{F})
$$
where $\tau = \etale$ (resp.\ $\tau = fppf$). Here $y$ is an object
of $\mathcal{Y}$ lying over a scheme $V$ such that the morphism
$y : V \to \mathcal{Y}$ is smooth (resp.\ flat).
This equality follows from
Sheaves on Stacks, Lemma \ref{stacks-sheaves-lemma-cohomology-on-subcategory}.
Although we omit the verification of the assumptions
of the lemma, we note that the fact that $V \to \mathcal{Y}$ is smooth
(resp.\ flat) is used to verify the second condition.
\end{proof}






\section{Quasi-coherent modules and the lisse-\'etale and flat-fppf sites}
\label{section-quasi-coherent-modules-II}

\noindent
In this section we explain how to think of quasi-coherent modules
on an algebraic stack in terms of its lisse-\'etale or flat-fppf site.

\begin{lemma}
\label{lemma-check-qc-on-etale-covering}
Let $\mathcal{X}$ be an algebraic stack.
\begin{enumerate}
\item Let $f_j : \mathcal{X}_j \to \mathcal{X}$ be a family of smooth
morphisms of algebraic stacks with
$|\mathcal{X}| =\bigcup |f_j|(|\mathcal{X}_j|)$.
Let $\mathcal{F}$ be a sheaf of $\mathcal{O}_\mathcal{X}$-modules
on $\mathcal{X}_\etale$. If each $f_j^{-1}\mathcal{F}$
is quasi-coherent, then so is $\mathcal{F}$.
\item Let $f_j : \mathcal{X}_j \to \mathcal{X}$ be a family of flat and
locally finitely presented morphisms of algebraic stacks with
$|\mathcal{X}| =\bigcup |f_j|(|\mathcal{X}_j|)$.
Let $\mathcal{F}$ be a sheaf of $\mathcal{O}_\mathcal{X}$-modules
on $\mathcal{X}_{fppf}$. If each $f_j^{-1}\mathcal{F}$
is quasi-coherent, then so is $\mathcal{F}$.
\end{enumerate}
\end{lemma}

\begin{proof}
Proof of (1). We may replace each of the algebraic stacks $\mathcal{X}_j$
by a scheme $U_j$ (using that any algebraic stack has a smooth covering by
a scheme and that compositions of smooth morphisms are smooth, see
Morphisms of Stacks, Lemma \ref{stacks-morphisms-lemma-composition-smooth}).
The pullback of $\mathcal{F}$ to $(\Sch/U_j)_\etale$ is still
quasi-coherent, see
Modules on Sites, Lemma \ref{sites-modules-lemma-local-pullback}.
Then $f = \coprod f_j : U = \coprod U_j \to \mathcal{X}$ is a smooth surjective
morphism. Let $x : V \to \mathcal{X}$ be an object of $\mathcal{X}$. By
Sheaves on Stacks, Lemma
\ref{stacks-sheaves-lemma-surjective-flat-locally-finite-presentation}
there exists an \'etale covering $\{x_i \to x\}_{i \in I}$
such that each $x_i$ lifts to an object $u_i$ of $(\Sch/U)_\etale$.
This just means that $x_i$ lives over a scheme $V_i$, that
$\{V_i \to V\}$ is an \'etale covering, and that $x_i$ comes from
a morphism $u_i : V_i \to U$. Then
$x_i^*\mathcal{F} = u_i^*f^*\mathcal{F}$ is quasi-coherent.
This implies that $x^*\mathcal{F}$ on $(\Sch/V)_\etale$
is quasi-coherent, for example by
Modules on Sites, Lemma \ref{sites-modules-lemma-local-final-object}.
By Sheaves on Stacks, Lemma
\ref{stacks-sheaves-lemma-characterize-quasi-coherent-bis}
we see that $x^*\mathcal{F}$ is an fppf sheaf and since $x$
was arbitrary we see that $\mathcal{F}$ is a sheaf in the
fppf topology. Applying Sheaves on Stacks, Lemma
\ref{stacks-sheaves-lemma-characterize-quasi-coherent}
we see that $\mathcal{F}$ is quasi-coherent.

\medskip\noindent
Proof of (2). This is proved using exactly the same argument, which we fully
write out here. We may replace each of the algebraic stacks $\mathcal{X}_j$
by a scheme $U_j$ (using that any algebraic stack has a smooth covering by
a scheme and that flat and locally finite presented morphisms are preserved
by composition, see Morphisms of Stacks, Lemmas
\ref{stacks-morphisms-lemma-composition-flat} and
\ref{stacks-morphisms-lemma-composition-finite-presentation}).
The pullback of $\mathcal{F}$ to $(\Sch/U_j)_\etale$ is still
locally quasi-coherent, see
Sheaves on Stacks, Lemma \ref{stacks-sheaves-lemma-pullback-quasi-coherent}.
Then $f = \coprod f_j : U = \coprod U_j \to \mathcal{X}$ is a surjective,
flat, and locally finitely presented morphism. Let
$x : V \to \mathcal{X}$ be an object of $\mathcal{X}$. By
Sheaves on Stacks, Lemma
\ref{stacks-sheaves-lemma-surjective-flat-locally-finite-presentation}
there exists an fppf covering $\{x_i \to x\}_{i \in I}$
such that each $x_i$ lifts to an object $u_i$ of $(\Sch/U)_\etale$.
This just means that $x_i$ lives over a scheme $V_i$, that
$\{V_i \to V\}$ is an fppf covering, and that $x_i$ comes from
a morphism $u_i : V_i \to U$. Then
$x_i^*\mathcal{F} = u_i^*f^*\mathcal{F}$ is quasi-coherent.
This implies that $x^*\mathcal{F}$ on $(\Sch/V)_\etale$
is quasi-coherent, for example by
Modules on Sites, Lemma \ref{sites-modules-lemma-local-final-object}.
By Sheaves on Stacks, Lemma
\ref{stacks-sheaves-lemma-characterize-quasi-coherent}
we see that $\mathcal{F}$ is quasi-coherent.
\end{proof}

\noindent
We recall that we have defined the notion of a quasi-coherent module on
any ringed topos in
Modules on Sites, Section \ref{sites-modules-section-local}.

\begin{lemma}
\label{lemma-shriek-quasi-coherent}
Let $\mathcal{X}$ be an algebraic stack. Notation as in
Lemma \ref{lemma-lisse-etale}.
\begin{enumerate}
\item Let $\mathcal{H}$ be a quasi-coherent
$\mathcal{O}_{\mathcal{X}_{lisse,\etale}}$-module 
on the lisse-\'etale site of $\mathcal{X}$. Then $g_!\mathcal{H}$ is a
quasi-coherent module on $\mathcal{X}$.
\item Let $\mathcal{H}$ be a quasi-coherent
$\mathcal{O}_{\mathcal{X}_{flat,fppf}}$-module 
on the flat-fppf site of $\mathcal{X}$. Then $g_!\mathcal{H}$ is a
quasi-coherent module on $\mathcal{X}$.
\end{enumerate}
\end{lemma}

\begin{proof}
Pick a scheme $U$ and a surjective smooth morphism $x : U \to \mathcal{X}$.
By
Modules on Sites, Definition \ref{sites-modules-definition-site-local}
there exists an \'etale (resp.\ fppf) covering
$\{U_i \to U\}_{i \in I}$ such that each pullback $f_i^{-1}\mathcal{H}$
has a global presentation (see
Modules on Sites, Definition \ref{sites-modules-definition-global}).
Here $f_i : U_i \to \mathcal{X}$ is the composition
$U_i \to U \to \mathcal{X}$ which is a morphism of algebraic stacks.
(Recall that the pullback ``is'' the restriction to $\mathcal{X}/f_i$, see
Sheaves on Stacks, Definition \ref{stacks-sheaves-definition-pullback}
and the discussion following.) Since each $f_i$ is smooth (resp.\ flat) by
Lemma \ref{lemma-lisse-etale-functorial}
we see that $f_i^{-1}g_!\mathcal{H} = g_{i, !}(f'_i)^{-1}\mathcal{H}$.
Using Lemma \ref{lemma-check-qc-on-etale-covering}
we reduce the statement of the lemma to the case where $\mathcal{H}$
has a global presentation. Say we have
$$
\bigoplus\nolimits_{j \in J} \mathcal{O} \longrightarrow
\bigoplus\nolimits_{i \in I} \mathcal{O} \longrightarrow
\mathcal{H} \longrightarrow 0
$$
of $\mathcal{O}$-modules where
$\mathcal{O} = \mathcal{O}_{\mathcal{X}_{lisse,\etale}}$
(resp.\ $\mathcal{O} = \mathcal{O}_{\mathcal{X}_{flat,fppf}}$).
Since $g_!$ commutes with arbitrary colimits (as a left adjoint functor, see
Lemma \ref{lemma-lisse-etale-modules} and
Categories, Lemma \ref{categories-lemma-adjoint-exact})
we conclude that there exists an exact sequence
$$
\bigoplus\nolimits_{j \in J} g_!\mathcal{O} \longrightarrow
\bigoplus\nolimits_{i \in I} g_!\mathcal{O} \longrightarrow
g_!\mathcal{H} \longrightarrow 0
$$
Lemma \ref{lemma-lisse-etale-structure-sheaf}
shows that $g_!\mathcal{O} = \mathcal{O}_\mathcal{X}$.
In case (2) we are done. In case (1) we apply
Sheaves on Stacks, Lemma
\ref{stacks-sheaves-lemma-characterize-quasi-coherent-bis}
to conclude.
\end{proof}

\begin{lemma}
\label{lemma-quasi-coherent}
Let $\mathcal{X}$ be an algebraic stack.
\begin{enumerate}
\item With $g$ as in Lemma \ref{lemma-lisse-etale}
for the lisse-\'etale site we have
\begin{enumerate}
\item the functors $g^{-1}$ and $g_!$ define mutually inverse functors
$$
\xymatrix{
\QCoh(\mathcal{O}_\mathcal{X}) \ar@<1ex>[r]^-{g^{-1}} &
\QCoh(\mathcal{X}_{lisse,\etale},
\mathcal{O}_{\mathcal{X}_{lisse,\etale}}) \ar@<1ex>[l]^-{g_!}
}
$$
\item if $\mathcal{F}$ is in $\textit{LQCoh}^{fbc}(\mathcal{O}_\mathcal{X})$
then $g^{-1}\mathcal{F}$ is in
$\QCoh(\mathcal{O}_{\mathcal{X}_{lisse,\etale}})$ and
\item $Q(\mathcal{F}) = g_!g^{-1}\mathcal{F}$ where $Q$ is as in
Lemma \ref{lemma-adjoint}.
\end{enumerate}
\item With $g$ as in Lemma \ref{lemma-lisse-etale}
for the flat-fppf site we have
\begin{enumerate}
\item the functors $g^{-1}$ and $g_!$ define mutually inverse functors
$$
\xymatrix{
\QCoh(\mathcal{O}_\mathcal{X}) \ar@<1ex>[r]^-{g^{-1}} &
\QCoh(\mathcal{X}_{flat,fppf},
\mathcal{O}_{\mathcal{X}_{flat,fppf}}) \ar@<1ex>[l]^-{g_!}
}
$$
\item if $\mathcal{F}$ is in $\textit{LQCoh}^{fbc}(\mathcal{O}_\mathcal{X})$
then $g^{-1}\mathcal{F}$ is in
$\QCoh(\mathcal{O}_{\mathcal{X}_{flat,fppf}})$
and
\item $Q(\mathcal{F}) = g_!g^{-1}\mathcal{F}$ where $Q$ is as in
Lemma \ref{lemma-adjoint}.
\end{enumerate}
\end{enumerate}
\end{lemma}

\begin{proof}
Pullback by any morphism of ringed topoi preserves categories of quasi-coherent
modules, see
Modules on Sites, Lemma \ref{sites-modules-lemma-local-pullback}.
Hence $g^{-1}$ preserves the categories of quasi-coherent modules;
here we use that
$\QCoh(\mathcal{O}_\mathcal{X}) =
\QCoh(\mathcal{X}_\etale, \mathcal{O}_\mathcal{X})$
by Sheaves on Stacks, Lemma
\ref{stacks-sheaves-lemma-characterize-quasi-coherent-bis}.
The same is true for $g_!$ by
Lemma \ref{lemma-shriek-quasi-coherent}.
We know that $\mathcal{H} \to g^{-1}g_!\mathcal{H}$ is an isomorphism by
Lemma \ref{lemma-lisse-etale}.
Conversely, if $\mathcal{F}$ is in $\QCoh(\mathcal{O}_\mathcal{X})$
then the map $g_!g^{-1}\mathcal{F} \to \mathcal{F}$ is a map of quasi-coherent
modules on $\mathcal{X}$ whose restriction to any scheme smooth over
$\mathcal{X}$ is an isomorphism. Then the discussion in
Sheaves on Stacks, Sections
\ref{stacks-sheaves-section-quasi-coherent-presentation} and
\ref{stacks-sheaves-section-quasi-coherent-algebraic-stacks}
(comparing with quasi-coherent modules on presentations)
shows it is an isomorphism. This proves (1)(a) and (2)(a).

\medskip\noindent
Let $\mathcal{F}$ be an object of
$\textit{LQCoh}^{fbc}(\mathcal{O}_\mathcal{X})$. By
Lemma \ref{lemma-adjoint-kernel-parasitic}
the kernel and cokernel of the map
$Q(\mathcal{F}) \to \mathcal{F}$ are parasitic. Hence by
Lemma \ref{lemma-parasitic-in-terms-flat-fppf}
and since $g^* = g^{-1}$ is exact, we conclude
$g^*Q(\mathcal{F}) \to g^*\mathcal{F}$ is an isomorphism. Thus
$g^*\mathcal{F}$ is quasi-coherent. This proves (1)(b) and (2)(b).
Finally, (1)(c) and (2)(c) follow because
$g_!g^*Q(\mathcal{F}) \to Q(\mathcal{F})$ is an isomorphism by
our arguments above.
\end{proof}

\begin{lemma}
\label{lemma-quasi-coherent-weak-serre}
Let $\mathcal{X}$ be an algebraic stack.
\begin{enumerate}
\item $\QCoh(\mathcal{O}_{\mathcal{X}_{lisse,\etale}})$
is a weak Serre subcategory of
$\textit{Mod}(\mathcal{O}_{\mathcal{X}_{lisse,\etale}})$.
\item $\QCoh(\mathcal{O}_{\mathcal{X}_{flat,fppf}})$
is a weak Serre subcategory of
$\textit{Mod}(\mathcal{O}_{\mathcal{X}_{flat,fppf}})$.
\end{enumerate}
\end{lemma}

\begin{proof}
We will verify conditions (1), (2), (3), (4) of
Homology, Lemma \ref{homology-lemma-characterize-weak-serre-subcategory}.

\medskip\noindent
Since $0$ is a quasi-coherent module on any ringed site we see that (1)
holds.

\medskip\noindent
By definition $\QCoh(\mathcal{O})$
is a strictly full subcategory $\textit{Mod}(\mathcal{O})$, so (2) holds.

\medskip\noindent
Let $\varphi : \mathcal{G} \to \mathcal{F}$ be a morphism of quasi-coherent
modules on $\mathcal{X}_{lisse,\etale}$ or $\mathcal{X}_{flat,fppf}$.
We have $g^*g_!\mathcal{F} = \mathcal{F}$ and similarly for
$\mathcal{G}$ and $\varphi$, see Lemma \ref{lemma-lisse-etale-modules}.
By Lemma \ref{lemma-shriek-quasi-coherent}
we see that $g_!\mathcal{F}$ and $g_!\mathcal{G}$ are quasi-coherent
$\mathcal{O}_\mathcal{X}$-modules. By Sheaves on Stacks, Lemma
\ref{stacks-sheaves-lemma-quasi-coherent-algebraic-stack}
we have that $\Coker(g_!\varphi)$ is a quasi-coherent
module on $\mathcal{X}$ (and the cokernel in the category
of quasi-coherent modules on $\mathcal{X}$).
Since $g^*$ is exact (see Lemma \ref{lemma-lisse-etale})
$g^*\Coker(g_!\varphi) = \Coker(g^*g_!\varphi) = \Coker(\varphi)$
is quasi-coherent too (see Lemma \ref{lemma-quasi-coherent}).
By Proposition \ref{proposition-loc-qcoh-flat-base-change}
the kernel $\Ker(g_!\varphi)$ is in
$\textit{LQCoh}^{fbc}(\mathcal{O}_\mathcal{X})$.
Since $g^*$ is exact, we have
$g^*\Ker(g_!\varphi) = \Ker(g^*g_!\varphi) = \Ker(\varphi)$.
Since $g^*$ maps objects of
$\textit{LQCoh}^{fbc}(\mathcal{O}_\mathcal{X})$ to quasi-coherent
modules by Lemma \ref{lemma-quasi-coherent} we conclude that
$\Ker(\varphi)$ is quasi-coherent as well. This proves (3).

\medskip\noindent
Finally, suppose that
$$
0 \to \mathcal{F} \to \mathcal{E} \to \mathcal{G} \to 0
$$
is an extension of $\mathcal{O}_{\mathcal{X}_{lisse,\etale}}$-modules
(resp.\ $\mathcal{O}_{\mathcal{X}_{flat,fppf}}$-modules) with $\mathcal{F}$
and $\mathcal{G}$ quasi-coherent. To prove (4) and finish the proof
we have to show that
$\mathcal{E}$ is quasi-coherent on $\mathcal{X}_{lisse,\etale}$
(resp.\ $\mathcal{X}_{flat,fppf}$). Let $U$
be an object of $\mathcal{X}_{lisse,\etale}$
(resp.\ $\mathcal{X}_{flat,fppf}$; we think of $U$ as a scheme smooth
(resp.\ flat) over $\mathcal{X}$. We have to show that the restriction
of $\mathcal{E}$ to $U_{lisse,\etale}$ (resp.\ $=U_{flat,fppf}$)
is quasi-coherent. Thus we may assume
that $\mathcal{X} = U$ is a scheme. Because $\mathcal{G}$
is quasi-coherent on $U_{lisse,\etale}$ (resp.\ $U_{flat,fppf}$),
we may assume, after replacing $U$ by the members
of an \'etale (resp.\ fppf) covering, that $\mathcal{G}$ has a
presentation
$$
\bigoplus\nolimits_{j \in J} \mathcal{O} \longrightarrow
\bigoplus\nolimits_{i \in I} \mathcal{O} \longrightarrow
\mathcal{G} \longrightarrow 0
$$
on $U_{lisse,\etale}$ (resp.\ $U_{flat,fppf}$) where $\mathcal{O}$
is the structure sheaf on the site. We may also assume $U$ is affine.
Since $\mathcal{F}$ is quasi-coherent, we have
$$
H^1(U_{lisse,\etale}, \mathcal{F}) = 0,
\quad\text{resp.}\quad
H^1(U_{flat,fppf}, \mathcal{F}) = 0
$$
Namely, $\mathcal{F}$ is the pullback of a quasi-coherent module
$\mathcal{F}'$ on the big site of $U$
(by Lemma \ref{lemma-quasi-coherent}), cohomology
of $\mathcal{F}$ and $\mathcal{F}'$ agree (by
Lemma \ref{lemma-lisse-etale-cohomology}), and we know that
the cohomology of $\mathcal{F}'$ on the big site of the affine scheme $U$
is zero (to get this in the current situation you have to
combine Descent, Propositions
\ref{descent-proposition-equivalence-quasi-coherent} and
\ref{descent-proposition-same-cohomology-quasi-coherent} with
Cohomology of Schemes, Lemma
\ref{coherent-lemma-quasi-coherent-affine-cohomology-zero}).
Thus we can lift the map
$\bigoplus_{i \in I} \mathcal{O} \to \mathcal{G}$
to $\mathcal{E}$. A diagram chase shows that we obtain
an exact sequence
$$
\bigoplus\nolimits_{j \in J} \mathcal{O} \to
\mathcal{F} \oplus \bigoplus\nolimits_{i \in I} \mathcal{O}
\to
\mathcal{E} \to 0
$$
By (3) proved above, we conclude that $\mathcal{E}$ is quasi-coherent
as desired.
\end{proof}









\section{Coherent sheaves on locally Noetherian stacks}
\label{section-coherent-sheaves}

\noindent
This section is the analogue of
Cohomology of Spaces, Section
\ref{spaces-cohomology-section-coherent}.
We have defined the notion of a coherent module on any ringed topos in
Modules on Sites, Section \ref{sites-modules-section-local}.
However, for any algebraic stack $\mathcal{X}$ the category of
coherent $\mathcal{O}_\mathcal{X}$-modules is zero, essentially because
the site $\mathcal{X}$ contains too many non-Noetherian objects (even
if $\mathcal{X}$ is itself locally Noetherian). Instead, we will define
coherent modules using the following lemma.

\begin{lemma}
\label{lemma-coherent-Noetherian}
Let $\mathcal{X}$ be a locally Noetherian algebraic stack.
Let $\mathcal{F}$ be an $\mathcal{O}_\mathcal{X}$-module.
The following are equivalent
\begin{enumerate}
\item $\mathcal{F}$ is a quasi-coherent, finite type
$\mathcal{O}_\mathcal{X}$-module,
\item $\mathcal{F}$ is an $\mathcal{O}_\mathcal{X}$-module
of finite presentation,
\item $\mathcal{F}$ is quasi-coherent and for any morphism
$f : U \to \mathcal{X}$ where $U$ is a locally Noetherian algebraic space,
the pullback $f^*\mathcal{F}|_{U_\etale}$ is coherent, and
\item $\mathcal{F}$ is quasi-coherent and there exists an algebraic space
$U$ and a morphism $f : U \to \mathcal{X}$ which is locally of finite type,
flat, and surjective, such that the pullback $f^*\mathcal{F}|_{U_\etale}$
is coherent.
\end{enumerate}
\end{lemma}

\begin{proof}
Let $f : U \to \mathcal{X}$ be as in (4).
Then $U$ is locally Noetherian (Morphisms of Stacks, Lemma
\ref{stacks-morphisms-lemma-locally-finite-type-locally-noetherian})
and we see that the statement of the lemma makes sense. Additionally,
$f$ is locally of finite presentation by Morphisms of Stacks, Lemma
\ref{stacks-morphisms-lemma-noetherian-finite-type-finite-presentation}.
Let $x$ be an object of $\mathcal{X}$ lying over the scheme $V$.
In order to prove (2) we have to show that, after replacing $V$
by the members of an fppf covering of $V$, the restriction $x^*\mathcal{F}$
has a global finite presentation on $\mathcal{X}/x \cong (\Sch/V)_{fppf}$.
The projection $W = U \times_\mathcal{X} V \to V$ is
locally of finite presentation, flat, and surjective.
Hence we may replace $V$ by the members of an \'etale covering of $W$
by schemes and assume we have a morphism $h : V \to U$ with $f \circ h = x$.
Since $\mathcal{F}$ is quasi-coherent, we see that
the restriction $x^*\mathcal{F}$
is the pullback of $h_{small}^*(f^*\mathcal{F})|_{U_\etale}$
by $\pi_V$, see Sheaves on Stacks, Lemma
\ref{stacks-sheaves-lemma-compare-quasi-coherent}.
Since $f^*\mathcal{F}|_{U_\etale}$ locally in the
\'etale topology has a finite presentation by assumption, we conclude
(4) $\Rightarrow$ (2).

\medskip\noindent
Part (2) implies (1) for any ringed topos (immediate from the definition).
The properties ``finite type'' and ``quasi-coherent''
are preserved under pullback by any morphism of ringed topoi, see
Modules on Sites, Lemma \ref{sites-modules-lemma-local-pullback}.
Hence (1) implies (3), see Cohomology of Spaces, Lemma
\ref{spaces-cohomology-lemma-coherent-Noetherian}.
Finally, (3) trivially implies (4).
\end{proof}

\begin{definition}
\label{definition-coherent}
Let $\mathcal{X}$ be a locally Noetherian algebraic stack.
An $\mathcal{O}_\mathcal{X}$-module $\mathcal{F}$ is called {\it coherent}
if $\mathcal{F}$ satisfies one (and hence all) of the equivalent
conditions of Lemma \ref{lemma-coherent-Noetherian}.
The category of coherent $\mathcal{O}_\mathcal{X}$-modules is
denote $\textit{Coh}(\mathcal{O}_\mathcal{X})$.
\end{definition}

\begin{lemma}
\label{lemma-elementary-coherent}
Let $\mathcal{X}$ be a locally Noetherian algebraic stack.
The module $\mathcal{O}_\mathcal{X}$ is coherent, any invertible
$\mathcal{O}_\mathcal{X}$-module is coherent, and more generally any
finite locally free $\mathcal{O}_\mathcal{X}$-module is coherent.
\end{lemma}

\begin{proof}
Follows from the definition and
Cohomology of Spaces, Lemma
\ref{spaces-cohomology-lemma-coherent-Noetherian}.
\end{proof}

\begin{lemma}
\label{lemma-pullback-coherent}
Let $f : \mathcal{X} \to \mathcal{Y}$ be a morphism of locally
Noetherian algebraic stacks. Then $f^*$ sends coherent modules
on $\mathcal{Y}$ to coherent modules on $\mathcal{X}$.
\end{lemma}

\begin{proof}
Immediate from the definition and the fact that pullback for any morphism
of ringed topoi preserves finitely presented modules, see
Modules on Sites, Lemma \ref{sites-modules-lemma-local-pullback}.
\end{proof}

\begin{lemma}
\label{lemma-coherent-abelian-Noetherian}
Let $\mathcal{X}$ be a locally Noetherian algebraic stack.
The category of coherent $\mathcal{O}_\mathcal{X}$-modules is abelian.
If $\varphi : \mathcal{F} \to \mathcal{G}$ is a map
of coherent $\mathcal{O}_\mathcal{X}$-modules, then
\begin{enumerate}
\item the cokernel $\Coker(\varphi)$ computed in
$\textit{Mod}(\mathcal{O}_\mathcal{X})$ is a coherent
$\mathcal{O}_\mathcal{X}$-module,
\item the image $\Im(\varphi)$ computed in
$\textit{Mod}(\mathcal{O}_\mathcal{X})$ is a coherent
$\mathcal{O}_\mathcal{X}$-module, and
\item the kernel $\Ker(\varphi)$ computed in
$\textit{Mod}(\mathcal{O}_\mathcal{X})$
may not be coherent, but it is
in $\textit{LQCoh}^{fbc}(\mathcal{O}_\mathcal{X})$ and $Q(\Ker(\varphi))$
is coherent and is the kernel of $\varphi$ in
$\textit{Coh}(\mathcal{O}_\mathcal{X})$.
\end{enumerate}
The inclusion functor $\textit{Coh}(\mathcal{O}_\mathcal{X}) \to
\QCoh(\mathcal{O}_\mathcal{X})$ is exact.
\end{lemma}

\begin{proof}
The rules given for taking kernels, images, and cokernels in
$\textit{Coh}(\mathcal{O}_\mathcal{X})$ agree with the prescription
for quasi-coherent modules in Remark \ref{remark-QCoh-abelian}.
Hence the lemma will follow if we can show that the quasi-coherent modules
$\Coker(\varphi)$, $\Im(\varphi)$, and $Q(\Ker(\varphi))$ are coherent.
By Lemma \ref{lemma-coherent-Noetherian} it suffices to prove this after
restricting to $U_\etale$ for some surjective smooth morphism
$f : U \to \mathcal{X}$. The functor
$\mathcal{F} \mapsto f^*\mathcal{F}|_{U_\etale}$
is exact. Hence $f^*\Coker(\varphi)$ and $f^*\Im(\varphi)$ are
the cokernel and image of a map between coherent $\mathcal{O}_U$-modules
hence coherent as desired. The functor
$\mathcal{F} \mapsto f^*\mathcal{F}|_{U_\etale}$
kills parasitic modules by Lemma \ref{lemma-parasitic}.
Hence $f^*Q(\Ker(\varphi))|_{U_\etale} = f^*\Ker(\varphi)|_{U_\etale}$
by part (2) of Lemma \ref{lemma-adjoint-kernel-parasitic}.
Thus we conclude that $Q(\Ker(\varphi))$ is coherent in the same way.
\end{proof}

\begin{lemma}
\label{lemma-extension-coherent}
Let $\mathcal{X}$ be a locally Noetherian algebraic stack.
Given a short exact sequence
$0 \to \mathcal{F}_1 \to \mathcal{F}_2 \to \mathcal{F}_3 \to 0$
in $\textit{Mod}(\mathcal{O}_\mathcal{X})$
with $\mathcal{F}_1$ and $\mathcal{F}_3$ coherent, then
$\mathcal{F}_2$ is coherent.
\end{lemma}

\begin{proof}
By Sheaves on Stacks, Lemma
\ref{stacks-sheaves-lemma-quasi-coherent-algebraic-stack} part (7) we see that
$\mathcal{F}_2$ is quasi-coherent. Then we can check that
$\mathcal{F}_2$ is coherent by restricting to $U_\etale$
for some $U \to \mathcal{X}$ surjective and smooth.
This follows from Cohomology of Spaces, Lemma
\ref{spaces-cohomology-lemma-coherent-abelian-Noetherian}.
Some details omitted.
\end{proof}

\noindent
Coherent modules form a Serre subcategory of the
category of quasi-coherent $\mathcal{O}_\mathcal{X}$-modules.
This does not hold for modules on a general ringed topos.

\begin{lemma}
\label{lemma-coherent-Noetherian-quasi-coherent-sub-quotient}
Let $\mathcal{X}$ be a locally Noetherian algebraic stack.
Then $\textit{Coh}(\mathcal{O}_\mathcal{X})$ is a Serre subcategory of
$\QCoh(\mathcal{O}_\mathcal{X})$. Let $\varphi : \mathcal{F} \to \mathcal{G}$
be a map of quasi-coherent $\mathcal{O}_\mathcal{X}$-modules. We have
\begin{enumerate}
\item if $\mathcal{F}$ is coherent and $\varphi$ surjective,
then $\mathcal{G}$ is coherent,
\item if $\mathcal{F}$ is coherent, then $\Im(\varphi)$ is coherent, and
\item if $\mathcal{G}$ coherent and $\Ker(\varphi)$ parasitic, then
$\mathcal{F}$ is coherent.
\end{enumerate}
\end{lemma}

\begin{proof}
Choose a scheme $U$ and a surjective smooth morphism $f : U \to \mathcal{X}$.
Then the functor
$f^* : \QCoh(\mathcal{O}_\mathcal{X}) \to \QCoh(\mathcal{O}_U)$
is exact (Lemma \ref{lemma-flat-pullback-quasi-coherent})
and moreover by definition $\textit{Coh}(\mathcal{O}_\mathcal{X})$
is the full subcategory of $\QCoh(\mathcal{O}_\mathcal{X})$ consisting
of objects $\mathcal{F}$ such that $f^*\mathcal{F}$ is in
$\textit{Coh}(\mathcal{O}_U)$. The statement that
$\textit{Coh}(\mathcal{O}_\mathcal{X})$ is a Serre subcategory of
$\QCoh(\mathcal{O}_\mathcal{X})$ follows immediately from this
and the corresponding fact for $U$, see
Cohomology of Spaces, Lemmas
\ref{spaces-cohomology-lemma-coherent-abelian-Noetherian} and
\ref{spaces-cohomology-lemma-coherent-Noetherian-quasi-coherent-sub-quotient}.
We omit the proof of (1), (2), and (3). Hint: compare
with the proof of Lemma \ref{lemma-coherent-abelian-Noetherian}.
\end{proof}

\noindent
Let $\mathcal{X}$ be a locally Noetherian algebraic stack.
Let $U$ be an algebraic space and let $f : U \to \mathcal{X}$
be surjective, locally of finite presentation, and flat.
Observe that $U$ is locally Noetherian
(Morphisms of Stacks, Lemma
\ref{stacks-morphisms-lemma-locally-finite-type-locally-noetherian}).
Let $(U, R, s, t, c)$ be the groupoid in algebraic spaces
and $f_{can} : [U/R] \to \mathcal{X}$ the isomorphism
constructed in
Algebraic Stacks, Lemma \ref{algebraic-lemma-map-space-into-stack} and
Remark \ref{algebraic-remark-flat-fp-presentation}.
As in Sheaves on Stacks, Section
\ref{stacks-sheaves-section-quasi-coherent-algebraic-stacks}
we obtain equivalences
$$
\QCoh(\mathcal{O}_\mathcal{X})
\cong
\QCoh(\mathcal{O}_{[U/R]})
\cong
\QCoh(U, R, s, t, c)
$$
where the second equivalence is
Sheaves on Stacks, Proposition \ref{stacks-sheaves-proposition-quasi-coherent}.
Recall that in Groupoids in Spaces, Section
\ref{spaces-groupoids-section-colimits}
we have defined the full subcategory
$$
\textit{Coh}(U, R, s, t, c) \subset \QCoh(U, R, s, t, c)
$$
of {\it coherent modules} as those $(\mathcal{G}, \alpha)$ such that
$\mathcal{G}$ is a coherent $\mathcal{O}_U$-module.

\begin{lemma}
\label{lemma-coherent-presentation}
In the situation discussed above, the equivalence
$\QCoh(\mathcal{O}_\mathcal{X}) \cong \QCoh(U, R, s, t, c)$
sends coherent sheaves to coherent sheaves and vice versa, i.e.,
induces an equivalence
$\textit{Coh}(\mathcal{O}_\mathcal{X}) \cong \textit{Coh}(U, R, s, t, c)$.
\end{lemma}

\begin{proof}
This is immediate from the definition of coherent
$\mathcal{O}_\mathcal{X}$-modules.
For bookkeeping purposes: the material above uses
Morphisms of Stacks, Lemma
\ref{stacks-morphisms-lemma-locally-finite-type-locally-noetherian},
Algebraic Stacks, Lemma \ref{algebraic-lemma-map-space-into-stack} and
Remark \ref{algebraic-remark-flat-fp-presentation},
Sheaves on Stacks, Section
\ref{stacks-sheaves-section-quasi-coherent-algebraic-stacks},
Sheaves on Stacks, Proposition \ref{stacks-sheaves-proposition-quasi-coherent},
and Groupoids in Spaces, Section
\ref{spaces-groupoids-section-colimits}.
\end{proof}

\begin{lemma}
\label{lemma-coherent-hom}
Let $\mathcal{X}$ be a locally Noetherian algebraic stack. Let $\mathcal{F}$
and $\mathcal{G}$ be coherent be $\mathcal{O}_\mathcal{X}$-modules. Then
the internal hom $hom(\mathcal{F}, \mathcal{G})$
constructed in Lemma \ref{lemma-internal-hom-fp-into-qcoh}
is a coherent $\mathcal{O}_\mathcal{X}$-module.
\end{lemma}

\begin{proof}
Let $U \to \mathcal{X}$ be a smooth surjective morphism from a scheme.
By item (\ref{item-hom-restriction}) in Section \ref{section-further-remarks}
we see that the restriction of
$hom(\mathcal{F}, \mathcal{G})$ to $U$ is the Hom sheaf of the restrictions.
Hence this lemma follows from the case of algebraic spaces, see
Cohomology of Spaces, Lemma \ref{spaces-cohomology-lemma-tensor-hom-coherent}.
\end{proof}






\section{Coherent sheaves on Noetherian stacks}
\label{section-coherent-on-noetherian}

\noindent
This section is the analogue of
Cohomology of Spaces, Section
\ref{spaces-cohomology-section-coherent-quasi-compact}.

\begin{lemma}
\label{lemma-directed-colimit-coherent}
Let $\mathcal{X}$ be a Noetherian algebraic stack. Every quasi-coherent
$\mathcal{O}_\mathcal{X}$-module is the filtered colimit of its coherent
submodules.
\end{lemma}

\begin{proof}
Let $\mathcal{F}$ be a quasi-coherent $\mathcal{O}_\mathcal{X}$-module.
If $\mathcal{G}, \mathcal{H} \subset \mathcal{F}$ are coherent
$\mathcal{O}_\mathcal{X}$-submodules then the image of
$\mathcal{G} \oplus \mathcal{H} \to \mathcal{F}$ is another
coherent $\mathcal{O}_\mathcal{X}$-submodule which contains both of them,
see Lemma \ref{lemma-coherent-Noetherian-quasi-coherent-sub-quotient}.
In this way we see that the system is directed.
Hence it now suffices to show that $\mathcal{F}$ can be written as
a filtered colimit of coherent modules, as then we can take the
images of these modules in $\mathcal{F}$ to conclude there are
enough of them.

\medskip\noindent
Let $U$ be an affine scheme and $U \to \mathcal{X}$
a surjective smooth morphism (Properties of Stacks, Lemma
\ref{stacks-properties-lemma-quasi-compact-stack}).
Set $R = U \times_\mathcal{X} U$ so that $\mathcal{X} = [U/R]$
as in Algebraic Stacks, Lemma \ref{algebraic-lemma-stack-presentation}.
By Lemma \ref{lemma-coherent-presentation} we have
$\QCoh(\mathcal{O}_X) = \QCoh(U, R, s, t, c)$ and
$\textit{Coh}(\mathcal{O}_X) = \textit{Coh}(U, R, s, t, c)$.
In this way we reduce to the problem of proving the corresponding thing for
$\QCoh(U, R, s, t, c)$. This is
Groupoids in Spaces, Lemma \ref{spaces-groupoids-lemma-colimit-coherent};
we check its assumptions in the next paragraph.

\medskip\noindent
We urge the reader to skip the rest of the proof.
The affine scheme $U$ is Noetherian; this follows from our definition
of $\mathcal{X}$ being locally Noetherian, see
Properties of Stacks, Definition
\ref{stacks-properties-definition-type-property} and
Remark \ref{stacks-properties-remark-list-properties-local-smooth-topology}.
The projection morphisms $s, t : R \to U$ are smooth (see reference
given above)
and quasi-separated and quasi-compact (Morphisms of Stacks, Lemma
\ref{stacks-morphisms-lemma-quasi-compact-quasi-separated-permanence}).
In particular, $R$ is a quasi-compact and quasi-separated algebraic space
smooth over $U$ and hence Noetherian
(Morphisms of Spaces, Lemma
\ref{spaces-morphisms-lemma-finite-presentation-noetherian}).
\end{proof}







\begin{multicols}{2}[\section{Other chapters}]
\noindent
Preliminaries
\begin{enumerate}
\item \hyperref[introduction-section-phantom]{Introduction}
\item \hyperref[conventions-section-phantom]{Conventions}
\item \hyperref[sets-section-phantom]{Set Theory}
\item \hyperref[categories-section-phantom]{Categories}
\item \hyperref[topology-section-phantom]{Topology}
\item \hyperref[sheaves-section-phantom]{Sheaves on Spaces}
\item \hyperref[sites-section-phantom]{Sites and Sheaves}
\item \hyperref[stacks-section-phantom]{Stacks}
\item \hyperref[fields-section-phantom]{Fields}
\item \hyperref[algebra-section-phantom]{Commutative Algebra}
\item \hyperref[brauer-section-phantom]{Brauer Groups}
\item \hyperref[homology-section-phantom]{Homological Algebra}
\item \hyperref[derived-section-phantom]{Derived Categories}
\item \hyperref[simplicial-section-phantom]{Simplicial Methods}
\item \hyperref[more-algebra-section-phantom]{More on Algebra}
\item \hyperref[smoothing-section-phantom]{Smoothing Ring Maps}
\item \hyperref[modules-section-phantom]{Sheaves of Modules}
\item \hyperref[sites-modules-section-phantom]{Modules on Sites}
\item \hyperref[injectives-section-phantom]{Injectives}
\item \hyperref[cohomology-section-phantom]{Cohomology of Sheaves}
\item \hyperref[sites-cohomology-section-phantom]{Cohomology on Sites}
\item \hyperref[dga-section-phantom]{Differential Graded Algebra}
\item \hyperref[dpa-section-phantom]{Divided Power Algebra}
\item \hyperref[sdga-section-phantom]{Differential Graded Sheaves}
\item \hyperref[hypercovering-section-phantom]{Hypercoverings}
\end{enumerate}
Schemes
\begin{enumerate}
\setcounter{enumi}{25}
\item \hyperref[schemes-section-phantom]{Schemes}
\item \hyperref[constructions-section-phantom]{Constructions of Schemes}
\item \hyperref[properties-section-phantom]{Properties of Schemes}
\item \hyperref[morphisms-section-phantom]{Morphisms of Schemes}
\item \hyperref[coherent-section-phantom]{Cohomology of Schemes}
\item \hyperref[divisors-section-phantom]{Divisors}
\item \hyperref[limits-section-phantom]{Limits of Schemes}
\item \hyperref[varieties-section-phantom]{Varieties}
\item \hyperref[topologies-section-phantom]{Topologies on Schemes}
\item \hyperref[descent-section-phantom]{Descent}
\item \hyperref[perfect-section-phantom]{Derived Categories of Schemes}
\item \hyperref[more-morphisms-section-phantom]{More on Morphisms}
\item \hyperref[flat-section-phantom]{More on Flatness}
\item \hyperref[groupoids-section-phantom]{Groupoid Schemes}
\item \hyperref[more-groupoids-section-phantom]{More on Groupoid Schemes}
\item \hyperref[etale-section-phantom]{\'Etale Morphisms of Schemes}
\end{enumerate}
Topics in Scheme Theory
\begin{enumerate}
\setcounter{enumi}{41}
\item \hyperref[chow-section-phantom]{Chow Homology}
\item \hyperref[intersection-section-phantom]{Intersection Theory}
\item \hyperref[pic-section-phantom]{Picard Schemes of Curves}
\item \hyperref[weil-section-phantom]{Weil Cohomology Theories}
\item \hyperref[adequate-section-phantom]{Adequate Modules}
\item \hyperref[dualizing-section-phantom]{Dualizing Complexes}
\item \hyperref[duality-section-phantom]{Duality for Schemes}
\item \hyperref[discriminant-section-phantom]{Discriminants and Differents}
\item \hyperref[derham-section-phantom]{de Rham Cohomology}
\item \hyperref[local-cohomology-section-phantom]{Local Cohomology}
\item \hyperref[algebraization-section-phantom]{Algebraic and Formal Geometry}
\item \hyperref[curves-section-phantom]{Algebraic Curves}
\item \hyperref[resolve-section-phantom]{Resolution of Surfaces}
\item \hyperref[models-section-phantom]{Semistable Reduction}
\item \hyperref[functors-section-phantom]{Functors and Morphisms}
\item \hyperref[equiv-section-phantom]{Derived Categories of Varieties}
\item \hyperref[pione-section-phantom]{Fundamental Groups of Schemes}
\item \hyperref[etale-cohomology-section-phantom]{\'Etale Cohomology}
\item \hyperref[crystalline-section-phantom]{Crystalline Cohomology}
\item \hyperref[proetale-section-phantom]{Pro-\'etale Cohomology}
\item \hyperref[relative-cycles-section-phantom]{Relative Cycles}
\item \hyperref[more-etale-section-phantom]{More \'Etale Cohomology}
\item \hyperref[trace-section-phantom]{The Trace Formula}
\end{enumerate}
Algebraic Spaces
\begin{enumerate}
\setcounter{enumi}{64}
\item \hyperref[spaces-section-phantom]{Algebraic Spaces}
\item \hyperref[spaces-properties-section-phantom]{Properties of Algebraic Spaces}
\item \hyperref[spaces-morphisms-section-phantom]{Morphisms of Algebraic Spaces}
\item \hyperref[decent-spaces-section-phantom]{Decent Algebraic Spaces}
\item \hyperref[spaces-cohomology-section-phantom]{Cohomology of Algebraic Spaces}
\item \hyperref[spaces-limits-section-phantom]{Limits of Algebraic Spaces}
\item \hyperref[spaces-divisors-section-phantom]{Divisors on Algebraic Spaces}
\item \hyperref[spaces-over-fields-section-phantom]{Algebraic Spaces over Fields}
\item \hyperref[spaces-topologies-section-phantom]{Topologies on Algebraic Spaces}
\item \hyperref[spaces-descent-section-phantom]{Descent and Algebraic Spaces}
\item \hyperref[spaces-perfect-section-phantom]{Derived Categories of Spaces}
\item \hyperref[spaces-more-morphisms-section-phantom]{More on Morphisms of Spaces}
\item \hyperref[spaces-flat-section-phantom]{Flatness on Algebraic Spaces}
\item \hyperref[spaces-groupoids-section-phantom]{Groupoids in Algebraic Spaces}
\item \hyperref[spaces-more-groupoids-section-phantom]{More on Groupoids in Spaces}
\item \hyperref[bootstrap-section-phantom]{Bootstrap}
\item \hyperref[spaces-pushouts-section-phantom]{Pushouts of Algebraic Spaces}
\end{enumerate}
Topics in Geometry
\begin{enumerate}
\setcounter{enumi}{81}
\item \hyperref[spaces-chow-section-phantom]{Chow Groups of Spaces}
\item \hyperref[groupoids-quotients-section-phantom]{Quotients of Groupoids}
\item \hyperref[spaces-more-cohomology-section-phantom]{More on Cohomology of Spaces}
\item \hyperref[spaces-simplicial-section-phantom]{Simplicial Spaces}
\item \hyperref[spaces-duality-section-phantom]{Duality for Spaces}
\item \hyperref[formal-spaces-section-phantom]{Formal Algebraic Spaces}
\item \hyperref[restricted-section-phantom]{Algebraization of Formal Spaces}
\item \hyperref[spaces-resolve-section-phantom]{Resolution of Surfaces Revisited}
\end{enumerate}
Deformation Theory
\begin{enumerate}
\setcounter{enumi}{89}
\item \hyperref[formal-defos-section-phantom]{Formal Deformation Theory}
\item \hyperref[defos-section-phantom]{Deformation Theory}
\item \hyperref[cotangent-section-phantom]{The Cotangent Complex}
\item \hyperref[examples-defos-section-phantom]{Deformation Problems}
\end{enumerate}
Algebraic Stacks
\begin{enumerate}
\setcounter{enumi}{93}
\item \hyperref[algebraic-section-phantom]{Algebraic Stacks}
\item \hyperref[examples-stacks-section-phantom]{Examples of Stacks}
\item \hyperref[stacks-sheaves-section-phantom]{Sheaves on Algebraic Stacks}
\item \hyperref[criteria-section-phantom]{Criteria for Representability}
\item \hyperref[artin-section-phantom]{Artin's Axioms}
\item \hyperref[quot-section-phantom]{Quot and Hilbert Spaces}
\item \hyperref[stacks-properties-section-phantom]{Properties of Algebraic Stacks}
\item \hyperref[stacks-morphisms-section-phantom]{Morphisms of Algebraic Stacks}
\item \hyperref[stacks-limits-section-phantom]{Limits of Algebraic Stacks}
\item \hyperref[stacks-cohomology-section-phantom]{Cohomology of Algebraic Stacks}
\item \hyperref[stacks-perfect-section-phantom]{Derived Categories of Stacks}
\item \hyperref[stacks-introduction-section-phantom]{Introducing Algebraic Stacks}
\item \hyperref[stacks-more-morphisms-section-phantom]{More on Morphisms of Stacks}
\item \hyperref[stacks-geometry-section-phantom]{The Geometry of Stacks}
\end{enumerate}
Topics in Moduli Theory
\begin{enumerate}
\setcounter{enumi}{107}
\item \hyperref[moduli-section-phantom]{Moduli Stacks}
\item \hyperref[moduli-curves-section-phantom]{Moduli of Curves}
\end{enumerate}
Miscellany
\begin{enumerate}
\setcounter{enumi}{109}
\item \hyperref[examples-section-phantom]{Examples}
\item \hyperref[exercises-section-phantom]{Exercises}
\item \hyperref[guide-section-phantom]{Guide to Literature}
\item \hyperref[desirables-section-phantom]{Desirables}
\item \hyperref[coding-section-phantom]{Coding Style}
\item \hyperref[obsolete-section-phantom]{Obsolete}
\item \hyperref[fdl-section-phantom]{GNU Free Documentation License}
\item \hyperref[index-section-phantom]{Auto Generated Index}
\end{enumerate}
\end{multicols}


\bibliography{my}
\bibliographystyle{amsalpha}

\end{document}
