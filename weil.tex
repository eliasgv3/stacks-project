\IfFileExists{stacks-project.cls}{%
\documentclass{stacks-project}
}{%
\documentclass{amsart}
}

% For dealing with references we use the comment environment
\usepackage{verbatim}
\newenvironment{reference}{\comment}{\endcomment}
%\newenvironment{reference}{}{}
\newenvironment{slogan}{\comment}{\endcomment}
\newenvironment{history}{\comment}{\endcomment}

% For commutative diagrams we use Xy-pic
\usepackage[all]{xy}

% We use 2cell for 2-commutative diagrams.
\xyoption{2cell}
\UseAllTwocells

% We use multicol for the list of chapters between chapters
\usepackage{multicol}

% This is generall recommended for better output
\usepackage{lmodern}
\usepackage[T1]{fontenc}

% For cross-file-references
\usepackage{xr-hyper}

% Package for hypertext links:
\usepackage{hyperref}

% For any local file, say "hello.tex" you want to link to please
% use \externaldocument[hello-]{hello}
\externaldocument[introduction-]{introduction}
\externaldocument[conventions-]{conventions}
\externaldocument[sets-]{sets}
\externaldocument[categories-]{categories}
\externaldocument[topology-]{topology}
\externaldocument[sheaves-]{sheaves}
\externaldocument[sites-]{sites}
\externaldocument[stacks-]{stacks}
\externaldocument[fields-]{fields}
\externaldocument[algebra-]{algebra}
\externaldocument[brauer-]{brauer}
\externaldocument[homology-]{homology}
\externaldocument[derived-]{derived}
\externaldocument[simplicial-]{simplicial}
\externaldocument[more-algebra-]{more-algebra}
\externaldocument[smoothing-]{smoothing}
\externaldocument[modules-]{modules}
\externaldocument[sites-modules-]{sites-modules}
\externaldocument[injectives-]{injectives}
\externaldocument[cohomology-]{cohomology}
\externaldocument[sites-cohomology-]{sites-cohomology}
\externaldocument[dga-]{dga}
\externaldocument[dpa-]{dpa}
\externaldocument[sdga-]{sdga}
\externaldocument[hypercovering-]{hypercovering}
\externaldocument[schemes-]{schemes}
\externaldocument[constructions-]{constructions}
\externaldocument[properties-]{properties}
\externaldocument[morphisms-]{morphisms}
\externaldocument[coherent-]{coherent}
\externaldocument[divisors-]{divisors}
\externaldocument[limits-]{limits}
\externaldocument[varieties-]{varieties}
\externaldocument[topologies-]{topologies}
\externaldocument[descent-]{descent}
\externaldocument[perfect-]{perfect}
\externaldocument[more-morphisms-]{more-morphisms}
\externaldocument[flat-]{flat}
\externaldocument[groupoids-]{groupoids}
\externaldocument[more-groupoids-]{more-groupoids}
\externaldocument[etale-]{etale}
\externaldocument[chow-]{chow}
\externaldocument[intersection-]{intersection}
\externaldocument[pic-]{pic}
\externaldocument[weil-]{weil}
\externaldocument[adequate-]{adequate}
\externaldocument[dualizing-]{dualizing}
\externaldocument[duality-]{duality}
\externaldocument[discriminant-]{discriminant}
\externaldocument[derham-]{derham}
\externaldocument[local-cohomology-]{local-cohomology}
\externaldocument[algebraization-]{algebraization}
\externaldocument[curves-]{curves}
\externaldocument[resolve-]{resolve}
\externaldocument[models-]{models}
\externaldocument[functors-]{functors}
\externaldocument[equiv-]{equiv}
\externaldocument[pione-]{pione}
\externaldocument[etale-cohomology-]{etale-cohomology}
\externaldocument[proetale-]{proetale}
\externaldocument[relative-cycles-]{relative-cycles}
\externaldocument[more-etale-]{more-etale}
\externaldocument[trace-]{trace}
\externaldocument[crystalline-]{crystalline}
\externaldocument[spaces-]{spaces}
\externaldocument[spaces-properties-]{spaces-properties}
\externaldocument[spaces-morphisms-]{spaces-morphisms}
\externaldocument[decent-spaces-]{decent-spaces}
\externaldocument[spaces-cohomology-]{spaces-cohomology}
\externaldocument[spaces-limits-]{spaces-limits}
\externaldocument[spaces-divisors-]{spaces-divisors}
\externaldocument[spaces-over-fields-]{spaces-over-fields}
\externaldocument[spaces-topologies-]{spaces-topologies}
\externaldocument[spaces-descent-]{spaces-descent}
\externaldocument[spaces-perfect-]{spaces-perfect}
\externaldocument[spaces-more-morphisms-]{spaces-more-morphisms}
\externaldocument[spaces-flat-]{spaces-flat}
\externaldocument[spaces-groupoids-]{spaces-groupoids}
\externaldocument[spaces-more-groupoids-]{spaces-more-groupoids}
\externaldocument[bootstrap-]{bootstrap}
\externaldocument[spaces-pushouts-]{spaces-pushouts}
\externaldocument[spaces-chow-]{spaces-chow}
\externaldocument[groupoids-quotients-]{groupoids-quotients}
\externaldocument[spaces-more-cohomology-]{spaces-more-cohomology}
\externaldocument[spaces-simplicial-]{spaces-simplicial}
\externaldocument[spaces-duality-]{spaces-duality}
\externaldocument[formal-spaces-]{formal-spaces}
\externaldocument[restricted-]{restricted}
\externaldocument[spaces-resolve-]{spaces-resolve}
\externaldocument[formal-defos-]{formal-defos}
\externaldocument[defos-]{defos}
\externaldocument[cotangent-]{cotangent}
\externaldocument[examples-defos-]{examples-defos}
\externaldocument[algebraic-]{algebraic}
\externaldocument[examples-stacks-]{examples-stacks}
\externaldocument[stacks-sheaves-]{stacks-sheaves}
\externaldocument[criteria-]{criteria}
\externaldocument[artin-]{artin}
\externaldocument[quot-]{quot}
\externaldocument[stacks-properties-]{stacks-properties}
\externaldocument[stacks-morphisms-]{stacks-morphisms}
\externaldocument[stacks-limits-]{stacks-limits}
\externaldocument[stacks-cohomology-]{stacks-cohomology}
\externaldocument[stacks-perfect-]{stacks-perfect}
\externaldocument[stacks-introduction-]{stacks-introduction}
\externaldocument[stacks-more-morphisms-]{stacks-more-morphisms}
\externaldocument[stacks-geometry-]{stacks-geometry}
\externaldocument[moduli-]{moduli}
\externaldocument[moduli-curves-]{moduli-curves}
\externaldocument[examples-]{examples}
\externaldocument[exercises-]{exercises}
\externaldocument[guide-]{guide}
\externaldocument[desirables-]{desirables}
\externaldocument[coding-]{coding}
\externaldocument[obsolete-]{obsolete}
\externaldocument[fdl-]{fdl}
\externaldocument[index-]{index}

% Theorem environments.
%
\theoremstyle{plain}
\newtheorem{theorem}[subsection]{Theorem}
\newtheorem{proposition}[subsection]{Proposition}
\newtheorem{lemma}[subsection]{Lemma}

\theoremstyle{definition}
\newtheorem{definition}[subsection]{Definition}
\newtheorem{example}[subsection]{Example}
\newtheorem{exercise}[subsection]{Exercise}
\newtheorem{situation}[subsection]{Situation}

\theoremstyle{remark}
\newtheorem{remark}[subsection]{Remark}
\newtheorem{remarks}[subsection]{Remarks}

\numberwithin{equation}{subsection}

% Macros
%
\def\lim{\mathop{\mathrm{lim}}\nolimits}
\def\colim{\mathop{\mathrm{colim}}\nolimits}
\def\Spec{\mathop{\mathrm{Spec}}}
\def\Hom{\mathop{\mathrm{Hom}}\nolimits}
\def\Ext{\mathop{\mathrm{Ext}}\nolimits}
\def\SheafHom{\mathop{\mathcal{H}\!\mathit{om}}\nolimits}
\def\SheafExt{\mathop{\mathcal{E}\!\mathit{xt}}\nolimits}
\def\Sch{\mathit{Sch}}
\def\Mor{\mathop{\mathrm{Mor}}\nolimits}
\def\Ob{\mathop{\mathrm{Ob}}\nolimits}
\def\Sh{\mathop{\mathit{Sh}}\nolimits}
\def\NL{\mathop{N\!L}\nolimits}
\def\CH{\mathop{\mathrm{CH}}\nolimits}
\def\proetale{{pro\text{-}\acute{e}tale}}
\def\etale{{\acute{e}tale}}
\def\QCoh{\mathit{QCoh}}
\def\Ker{\mathop{\mathrm{Ker}}}
\def\Im{\mathop{\mathrm{Im}}}
\def\Coker{\mathop{\mathrm{Coker}}}
\def\Coim{\mathop{\mathrm{Coim}}}

% Boxtimes
%
\DeclareMathSymbol{\boxtimes}{\mathbin}{AMSa}{"02}

%
% Macros for moduli stacks/spaces
%
\def\QCohstack{\mathcal{QC}\!\mathit{oh}}
\def\Cohstack{\mathcal{C}\!\mathit{oh}}
\def\Spacesstack{\mathcal{S}\!\mathit{paces}}
\def\Quotfunctor{\mathrm{Quot}}
\def\Hilbfunctor{\mathrm{Hilb}}
\def\Curvesstack{\mathcal{C}\!\mathit{urves}}
\def\Polarizedstack{\mathcal{P}\!\mathit{olarized}}
\def\Complexesstack{\mathcal{C}\!\mathit{omplexes}}
% \Pic is the operator that assigns to X its picard group, usage \Pic(X)
% \Picardstack_{X/B} denotes the Picard stack of X over B
% \Picardfunctor_{X/B} denotes the Picard functor of X over B
\def\Pic{\mathop{\mathrm{Pic}}\nolimits}
\def\Picardstack{\mathcal{P}\!\mathit{ic}}
\def\Picardfunctor{\mathrm{Pic}}
\def\Deformationcategory{\mathcal{D}\!\mathit{ef}}


% OK, start here.
%
\begin{document}

\title{Weil Cohomology Theories}


\maketitle

\phantomsection
\label{section-phantom}

\tableofcontents



\section{Introduction}
\label{section-introduction}

\noindent
In this chapter we discuss Weil cohomology theories for smooth
projective schemes over a base field. Briefly, for us such a cohomology
theory $H^*$ is one which has K\"unneth, Poincar\'e duality,
and cycle classes (with suitable compatibilities). We warn the reader that
there is no universal agreement in the literature as to what
constitutes a ``Weil cohomology theory''.

\medskip\noindent
Before reading this chapter the reader should take a look at
Categories, Section \ref{categories-section-monoidal} and
Homology, Section \ref{homology-section-monoidal} where
we define (symmetric) monoidal categories and we develop just enough
basic language concerning these categories for the needs of this chapter.
Equipped with this language we construct in
Section \ref{section-correspondences} the symmetric monoidal
graded category whose objects are smooth projective schemes and
whose morphisms are correspondences. In Section \ref{section-chow-motives}
we add images of projectors and invert the Lefschetz motive in
order to obtain the symmetric monoidal Karoubian category $M_k$
of Chow motives. This category comes equipped with a contravariant functor
$$
h : \{\text{smooth projective schemes over }k\} \longrightarrow M_k
$$
As we will see below, a key property of a Weil cohomology theory is
that it factors over $h$.

\medskip\noindent
First, in the case of an algebraically closed base field, we define
what we call a ``classical Weil cohomology theory'',
see Section \ref{section-axioms-classical}. This notion is the
same as the notion introduced in \cite[Section 1.2]{Kleiman-cycles} and
agrees with the notion introduced in \cite[page 65]{Kleiman-motives}.
However, our notion does not a priori agree with the notion introduced in
\cite[page 10]{Kleiman-standard} because there the author adds two Lefschetz
type axioms and it isn't known whether any classical Weil cohomology
theory as defined in this chapter satisfies those axioms.
At the end of Section \ref{section-axioms-classical} we show that
a classical Weil cohomology theory is of the form $H^* = G \circ h$
where $G$ is a symmetric monoidal functor from $M_k$ to the category
of graded vector spaces over the coefficient field of $H^*$.

\medskip\noindent
In Section \ref{section-cycles-nonclosed} we prove a couple of lemmas
on cycle groups over non-closed fields which will be used in discussing
Weil cohomology theories on smooth projective schemes over arbitrary fields.

\medskip\noindent
Our motivation for our axioms of a Weil cohomology theory $H^*$
over a general base field $k$ are the following
\begin{enumerate}
\item $H^* = G \circ h$ for a symmetric monoidal functor $G$ from $M_k$
to the category of graded vector spaces over the coefficient field $F$ of $H^*$,
\item $G$ should send the Tate motive (inverse of the Lefschetz motive)
to a $1$-dimensional vector space $F(1)$ sitting in degree $-2$,
\item when $k$ is algebraically closed we should recover the notion
discussion in Section \ref{section-axioms-classical}
up to choosing a basis element of $F(1)$.
\end{enumerate}
First, in Section \ref{section-axioms} we analyze the first two conditions.
After developing a few more results in Section \ref{section-further}
in Section \ref{section-old} we add the necessary axioms to obtain
property (3).

\medskip\noindent
In the final Section \ref{section-c1} we detail an alternative approach
to Weil cohomology theories, using a first Chern class map
instead of cycle classes. It is this approach that will be most
suited for proving that certain cohomology theories are Weil cohomology
theories in later chapters, see
de Rham Cohomology, Section \ref{derham-section-weil}.





\section{Conventions and notation}
\label{section-conventions}

\noindent
Let $F$ be a field. In this chapter we view the category of $F$-graded vector
spaces as an $F$-linear symmetric monoidal category with associativity
constraint as usual and with commutativity constraint involving signs.
See Homology, Example \ref{homology-example-graded-vector-spaces}.

\medskip\noindent
Let $R$ be a ring. In this chapter a
{\it graded commutative $R$-algebra} $A$ is a
commutative differential graded $R$-algebra
(Differential Graded Algebra, Definitions \ref{dga-definition-dga} and
\ref{dga-definition-cdga}) whose differential is zero. Thus $A$
is an $R$-module endowed with a grading
$A = \bigoplus_{n \in \mathbf{Z}} A^n$ by
$R$-submodules. The $R$-bilinear multiplication
$$
A^n \times A^m \longrightarrow A^{n + m},\quad
\alpha \times \beta \longmapsto \alpha \cup \beta
$$
will be called the {\it cup product} in this chapter.
The commutativity constraint is
$\alpha \cup \beta = (-1)^{nm} \beta \cup \alpha$ if
$\alpha \in A^n$ and $\beta \in A^m$. Finally, there is
a multiplicative unit $1 \in A^0$, or equivalently, there is an
additive and multiplicative map $R \to A^0$ which is compatible the
$R$-module structure on $A$.

\medskip\noindent
Let $k$ be a field. Let $X$ be a scheme of finite type over $k$.
The Chow groups $\CH_k(X)$ of $X$ of cycles of dimension $k$
modulo rational equivalence have been defined in
Chow Homology, Definition \ref{chow-definition-rational-equivalence}.
If $X$ is normal or Cohen-Macaulay, then we can also consider
the Chow groups $\CH^p(X)$ of cycles of codimension $p$
(Chow Homology, Section \ref{chow-section-cycles-codimension})
and then $[X] \in \CH^0(X)$ denotes the ``fundamental class'' of $X$, see
Chow Homology, Remark \ref{chow-remark-fundamental-class}.
If $X$ is smooth and $\alpha$ and $\beta$ are cycles on $X$,
then $\alpha \cdot \beta$ denotes the intersection product of
$\alpha$ and $\beta$, see
Chow Homology, Section \ref{chow-section-intersection-product}.













\section{Correspondences}
\label{section-correspondences}

\noindent
Let $k$ be a field. For schemes $X$ and $Y$ over $k$ we denote
$X \times Y$ the product of $X$ and $Y$ in the category of schemes
over $k$. In this section we construct the graded category over
$\mathbf{Q}$ whose objects are smooth projective schemes over $k$ and whose
morphisms are correspondences.

\medskip\noindent
Let $X$ and $Y$ be smooth projective schemes over $k$.
Let $X = \coprod X_d$ be the decomposition of $X$ into
the open and closed subschemes which are equidimensional
with $\dim(X_d) = d$. We define the $\mathbf{Q}$-vector space
{\it of correspondences of degree $r$ from $X$ to $Y$}
by the formula:
$$
\text{Corr}^r(X, Y) =
\bigoplus\nolimits_d \CH^{d + r}(X_d \times Y) \otimes \mathbf{Q}
\subset
\CH^*(X \times Y) \otimes \mathbf{Q}
$$
Given $c \in \text{Corr}^r(X, Y)$ and $\beta \in \CH_k(Y) \otimes \mathbf{Q}$
we can define the {\it pullback} of $\beta$ by $c$ using the formula
$$
c^*(\beta) = \text{pr}_{1, *}(c \cdot \text{pr}_2^*\beta)
\quad\text{in}\quad
\CH_{k - r}(X) \otimes \mathbf{Q}
$$
This makes sense because $\text{pr}_2$ is flat of relative dimension
$d$ on $X_d \times Y$, hence $\text{pr}_2^*\beta$ is a cycle of
dimension $d + k$ on $X_d \times Y$, hence $c \cdot \text{pr}_2^*\alpha$
is a cycle of dimension $k - r$ on $X_d \times Y$ whose pushforward
by the proper morphism $\text{pr}_1$ is a cycle of the same dimension.
Similarly, switching to grading by codimension,
given $\alpha \in \CH^i(X) \otimes \mathbf{Q}$ we can define the
{\it pushforward} of $\alpha$ by $c$ using the formula
$$
c_*(\alpha) = \text{pr}_{2, *}(c \cdot \text{pr}_1^*\alpha)
\quad\text{in}\quad
\CH^{i + r}(Y) \otimes \mathbf{Q}
$$
This makes sense because $\text{pr}_1^*\alpha$ is a cycle of codimension
$i$ on $X \times Y$, hence $c \cdot \text{pr}_1^*\alpha$ is a cycle
of codimension $i + d + r$ on $X_d \times Y$, which pushes forward
to a cycle of codimension $i + r$ on $Y$.

\medskip\noindent
Given a three smooth projective schemes $X, Y, Z$ over $k$ we define a
{\it composition of correspondences}
$$
\text{Corr}^s(Y, Z)
\times
\text{Corr}^r(X, Y)
\longrightarrow
\text{Corr}^{r + s}(X, Z)
$$
by the rule
$$
(c', c)
\longmapsto
c' \circ c =
\text{pr}_{13, *}(\text{pr}_{12}^*c \cdot \text{pr}_{23}^*c')
$$
where $\text{pr}_{12} : X \times Y \times Z \to X \times Y$
is the projection and similarly for $\text{pr}_{13}$ and $\text{pr}_{23}$.

\begin{lemma}
\label{lemma-composition-correspondences}
We have the following for correspondences:
\begin{enumerate}
\item composition of correspondences is $\mathbf{Q}$-bilinear
and associative,
\item there is a canonical isomorphism
$$
\CH_{-r}(X) \otimes \mathbf{Q} = \text{Corr}^r(X, \Spec(k))
$$
such that pullback by correspondences corresponds to composition,
\item there is a canonical isomorphism
$$
\CH^r(X) \otimes \mathbf{Q} = \text{Corr}^r(\Spec(k), X)
$$
such that pushforward by correspondences corresponds to composition,
\item composition of correspondences is compatible with pushforward and
pullback of cycles.
\end{enumerate}
\end{lemma}

\begin{proof}
Bilinearity follows immediately from the linearity of pushforward
and pullback and the bilinearity of the intersection product.
To prove associativity, say we have
$X, Y, Z, W$ and $c \in \text{Corr}(X, Y)$, $c' \in \text{Corr}(Y, Z)$, and
$c'' \in \text{Corr}(Z, W)$. Then we have
\begin{align*}
c'' \circ (c' \circ c)
& =
\text{pr}^{134}_{14, *}(
\text{pr}^{134, *}_{13}
\text{pr}^{123}_{13, *}(\text{pr}^{123, *}_{12}c \cdot
\text{pr}^{123, *}_{23}c')
\cdot \text{pr}^{134, *}_{34}c'') \\
& =
\text{pr}^{134}_{14, *}(
\text{pr}^{1234}_{134, *}
\text{pr}^{1234, *}_{123}(\text{pr}^{123, *}_{12}c \cdot
\text{pr}^{123, *}_{23}c')
\cdot \text{pr}^{134, *}_{34}c'') \\
& =
\text{pr}^{134}_{14, *}(
\text{pr}^{1234}_{134, *}
(\text{pr}^{1234, *}_{12}c \cdot
\text{pr}^{1234, *}_{23}c')
\cdot \text{pr}^{134, *}_{34}c'') \\
& =
\text{pr}^{134}_{14, *}
\text{pr}^{1234}_{134, *}
((\text{pr}^{1234, *}_{12}c \cdot
\text{pr}^{1234, *}_{23}c')
\cdot \text{pr}^{1234, *}_{34}c'') \\
& =
\text{pr}^{1234}_{14, *}(
(\text{pr}^{1234, *}_{12}c \cdot
\text{pr}^{1234, *}_{23}c') \cdot
\text{pr}^{1234, *}_{34}c'')
\end{align*}
Here we use the notation
$$
p^{1234}_{134} : X \times Y \times Z \times W \to X \times Z \times W
\quad\text{and}\quad
p^{134}_{14} : X \times Z \times W \to X \times W
$$
the projections and similarly for other indices.
The first equality is the definition of the composition.
The second equality holds because
$\text{pr}^{134, *}_{13} \text{pr}^{123}_{13, *} =
\text{pr}^{1234}_{134, *} \text{pr}^{1234, *}_{123}$
by Chow Homology, Lemma \ref{chow-lemma-flat-pullback-proper-pushforward}.
The third equality holds because intersection product commutes
with the gysin map for $p^{1234}_{123}$ (which is given by flat pullback), see
Chow Homology, Lemma \ref{chow-lemma-lci-gysin-product}.
The fourth equality follows from the projection formula for
$p^{1234}_{134}$, see Chow Homology, Lemma \ref{chow-lemma-projection-formula}.
The fourth equality is that proper pushforward is compatible
with composition, see Chow Homology, Lemma \ref{chow-lemma-compose-pushforward}.
Since intersection product is associative by
Chow Homology, Lemma \ref{chow-lemma-associative}
this concludes the proof of associativity of composition of correspondences.

\medskip\noindent
We omit the proofs of (2) and (3) as these are essentially proved by
carefully bookkeeping where various cycles live and in what (co)dimension.

\medskip\noindent
The statement on pushforward and pullback of cycles
means that $(c' \circ c)^*(\alpha) = c^*((c')^*(\alpha))$ and
$(c' \circ c)_*(\alpha) = (c')_*(c_*(\alpha))$.
This follows on combining (1), (2), and (3).
\end{proof}

\begin{example}
\label{example-graph-correspondence}
Let $f : Y \to X$ be a morphism of smooth projective schemes over $k$.
Denote $\Gamma_f \subset X \times Y$ the graph of $f$. More precisely,
$\Gamma_f$ is the image of the closed immersion
$$
(f, \text{id}_Y) : Y \longrightarrow X \times Y
$$
Let $X = \coprod X_d$ be the decomposition of $X$ into its
open and closed parts $X_d$ which are equidimensional of dimension $d$.
Then $\Gamma_f \cap (X_d \times Y)$ has pure codimension $d$. Hence
$[\Gamma_f] \in \CH^*(X \times Y) \otimes \mathbf{Q}$
is contained in $\text{Corr}^0(X \times Y)$, i.e., $[\Gamma_f]$
is a correspondence of degree $0$ from $X$ to $Y$.
\end{example}

\begin{lemma}
\label{lemma-category-correspondences}
Smooth projective schemes over $k$ with correspondences and composition
of correspondences as defined above form a graded category over
$\mathbf{Q}$
(Differential Graded Algebra, Definition \ref{dga-definition-graded-category}).
\end{lemma}

\begin{proof}
Everything is clear from the construction and
Lemma \ref{lemma-composition-correspondences}
except for the existence of identity morphisms.
Given a smooth projective scheme $X$ consider
the class $[\Delta]$ of the diagonal $\Delta \subset X \times X$
in $\text{Corr}^0(X, X)$. We note that $\Delta$ is
equal to the graph of the identity $\text{id}_X : X \to X$
which is a fact we will use below.

\medskip\noindent
To prove that $[\Delta]$ can serve as an identity we have to show that
$[\Delta] \circ c = c$ and $c' \circ [\Delta] = c'$ for any correspondences
$c \in \text{Corr}^r(Y, X)$ and $c' \in \text{Corr}^s(X, Y)$.
For the second case we have to show that
$$
c' = \text{pr}_{13, *}(\text{pr}_{12}^*[\Delta] \cdot \text{pr}_{23}^*c')
$$
where $\text{pr}_{12} : X \times X \times Y \to X \times X$ is the
projection and simlarly for $\text{pr}_{13}$ and $\text{pr}_{23}$.
We may write $c' = \sum a_i [Z_i]$ for some integral closed subschemes
$Z_i \subset X \times Y$ and rational numers $a_i$. Thus it clearly
suffices to show that
$$
[Z] = \text{pr}_{13, *}(\text{pr}_{12}^*[\Delta] \cdot \text{pr}_{23}^*[Z])
$$
in the chow group of $X \times Y$ for any integral closed subscheme $Z$
of $X \times Y$. After replacing $X$ and $Y$ by the
irreducible component containing the image of $Z$ under the two projections
we may assume $X$ and $Y$ are integral as well. Then we have to show
$$
[Z] = \text{pr}_{13, *}([\Delta \times Y] \cdot [X \times Z])
$$
Denote $Z' \subset X \times X \times Y$ the image of $Z$ by the morphism
$(\Delta, 1) : X \times Y \to X \times X \times Y$. Then $Z'$
is a closed subscheme of $X \times X \times Y$ isomorphic to $Z$ and
$Z' = \Delta \times Y \cap X \times Z$ scheme theoretically.
By Chow Homology, Lemma \ref{chow-lemma-intersect-properly}\footnote{The
reader verifies that $\dim(Z') = \dim(\Delta \times Y) + \dim(X \times Z) -
\dim(X \times X \times Y)$ and that $Z'$ has a unique generic point
mapping to the generic point of $Z$ (where the local ring is CM)
and to some point of $X$ (where the local ring is CM). Thus all the
hypothese of the lemma are indeed verified.}
we conclude that
$$
[Z'] = [\Delta \times Y] \cdot [X \times Z]
$$
Since $Z'$ maps isomorphically to $Z$ by $\text{pr}_{13}$ also
we conclude. The verification that
$[\Delta] \circ c = c$ is similar and we omit it.
\end{proof}

\begin{lemma}
\label{lemma-contravariant-functor}
There is a contravariant functor from the category of smooth
projective schemes over $k$ to the category of correspondences
which is the identity on objects and sends $f : Y \to X$ to
the element $[\Gamma_f] \in \text{Corr}^0(X, Y)$.
\end{lemma}

\begin{proof}
In the proof of Lemma \ref{lemma-category-correspondences}
we have seen that this construction sends identities to
identities. To finish the proof we have to show if $g : Z \to Y$
is another morphism of smooth projective schemes over $k$, then we have
$[\Gamma_g] \circ [\Gamma_f] = [\Gamma_{f \circ g}]$ in
$\text{Corr}^0(X, Z)$. Arguing as in the proof of
Lemma \ref{lemma-category-correspondences} we see that it
suffices to show
$$
[\Gamma_{f \circ g}] =
\text{pr}_{13, *}([\Gamma_f \times Z] \cdot [X \times \Gamma_g])
$$
in $\CH^*(X \times Z)$ when $X$, $Y$, $Z$ are integral.
Denote $Z' \subset X \times Y \times Z$ the image of the closed immersion
$(f \circ g, g, 1) : Z \to X \times Y \times Z$.
Then $Z' = \Gamma_f \times Z \cap X \times \Gamma_g$
scheme theoretically and we conclude using
Chow Homology, Lemma \ref{chow-lemma-intersect-properly}
that
$$
[Z'] = [\Gamma_f \times Z] \cdot [X \times \Gamma_g]
$$
Since it is clear that $\text{pr}_{13, *}([Z']) = [\Gamma_{f \circ g}]$
the proof is complete.
\end{proof}

\begin{remark}
\label{remark-transpose}
Let $X$ and $Y$ be smooth projective schemes over $k$.
Assume $X$ is equidimensional of dimension $d$ and
$Y$ is equidimensional of dimension $e$. Then the isomorphism
$X \times Y \to Y \times X$ switching the factors determines
an isomorphism
$$
\text{Corr}^r(X, Y) \longrightarrow \text{Corr}^{d - e + r}(Y, X),\quad
c \longmapsto c^t
$$
called the {\it transpose}. It acts on cycles as well as cycle classes.
An example which is sometimes useful, is the transpose
$[\Gamma_f]^t = [\Gamma_f^t]$ of the graph of a morphism $f : Y \to X$.
\end{remark}

\begin{lemma}
\label{lemma-functor-and-cycles}
Let $f : Y \to X$ be a morphism of smooth projective schemes over $k$.
Let $[\Gamma_f] \in \text{Corr}^0(X, Y)$ be as in
Example \ref{example-graph-correspondence}. Then
\begin{enumerate}
\item pushforward of cycles by the correspondence $[\Gamma_f]$
agrees with the gysin map $f^! : \CH^*(X) \to \CH^*(Y)$,
\item pullback of cycles by the correspondence $[\Gamma_f]$
agrees with the pushforward map $f_* : \CH_*(Y) \to \CH_*(X)$,
\item if $X$ and $Y$ are equidimensional of dimensions $d$ and $e$,
then
\begin{enumerate}
\item pushforward of cycles by the correspondence
$[\Gamma_f^t]$ of Remark \ref{remark-transpose}
corresponds to pushforward of cycles by $f$, and
\item pullback of cycles by the correspondence
$[\Gamma_f^t]$ of Remark \ref{remark-transpose}
corresponds to the gysin map $f^!$.
\end{enumerate}
\end{enumerate}
\end{lemma}

\begin{proof}
Proof of (1). Recall that
$[\Gamma_f]_*(\alpha) =
\text{pr}_{2, *}([\Gamma_f] \cdot \text{pr}_1^*\alpha)$.
We have
$$
[\Gamma_f] \cdot \text{pr}_1^*\alpha =
(f, 1)_*((f, 1)^! \text{pr}_1^*\alpha) =
(f, 1)_*((f, 1)^! \text{pr}_1^!\alpha) =
(f, 1)_*(f^!\alpha)
$$
The first equality by Chow Homology, Lemma
\ref{chow-lemma-intersect-regularly-embedded}.
The second by 
Chow Homology, Lemma \ref{chow-lemma-lci-gysin-flat}.
The third because $\text{pr}_1 \circ (f, 1) = f$ and
Chow Homology, Lemma \ref{chow-lemma-lci-gysin-composition}.
Then we coclude because
$\text{pr}_{2, *} \circ (f, 1)_* = 1_*$ by
Chow Homology, Lemma \ref{chow-lemma-compose-pushforward}.

\medskip\noindent
Proof of (2). Recall that $[\Gamma_f]_*(\beta) =
\text{pr}_{1, *}([\Gamma_f] \cdot \text{pr}_2^*\beta)$.
Arguing exactly as above we have
$$
[\Gamma_f] \cdot \text{pr}_2^*\beta = (f, 1)_*\beta
$$
Thus the result follows as before.

\medskip\noindent
Proof of (3). Proved in exactly the same manner as above.
\end{proof}

\begin{example}
\label{example-decompose-P1}
Let $X = \mathbf{P}^1_k$. Then we have
$$
\text{Corr}^0(X, X) = \CH^1(X \times X) \otimes \mathbf{Q} =
\CH_1(X \times X) \otimes \mathbf{Q}
$$
Choose a $k$-rational point $x \in X$ and
consider the cycles $c_0 = [x \times X]$ and $c_2 = [X \times x]$.
A computation shows that $1 = [\Delta] = c_0 + c_2$ in $\text{Corr}^0(X, X)$
and that we have the following rules for composition
$c_0 \circ c_0 = c_0$,
$c_0 \circ c_2 = 0$,
$c_2 \circ c_0 = 0$, and
$c_2 \circ c_2 = c_2$.
In other words, $c_0$ and $c_2$ are orthogonal idempotents in
the algebra $\text{Corr}^0(X, X)$ and in fact we get
$$
\text{Corr}^0(X, X) = \mathbf{Q} \times \mathbf{Q}
$$
as a $\mathbf{Q}$-algebra.
\end{example}

\noindent
The category of correspondences is a symmetric monoidal category.
Given smooth projective schemes $X$ and $Y$ over $k$, we define
$X \otimes Y = X \times Y$. Given four smooth projective schemes
$X, X', Y, Y'$ over $k$ we define a tensor product
$$
\otimes :
\text{Corr}^r(X, Y) \times \text{Corr}^{r'}(X', Y')
\longrightarrow
\text{Corr}^{r + r'}(X \times X', Y \times Y')
$$
by the rule
$$
(c, c') \longmapsto
c \otimes c' = \text{pr}_{13}^*c \cdot \text{pr}_{24}^*c'
$$
where $\text{pr}_{13} : X \times X' \times Y \times Y' \to X \times Y$
and $\text{pr}_{24} : X \times X' \times Y \times Y' \to X' \times Y'$
are the projections. As associativity constraint
$$
X \otimes (Y \otimes Z) = (X \otimes Y) \otimes Z
$$
we use the usual associativity constraint on products of schemes.
The commutativity constraint will be given by the isomorphism
$X \times Y \to Y \times X$ switching the factors.

\begin{lemma}
\label{lemma-tensor-product}
The tensor product of correspondences defined above turns the category of
correspondences into a symmetric monoidal category with unit $\Spec(k)$.
\end{lemma}

\begin{proof}
Omitted.
\end{proof}

\begin{lemma}
\label{lemma-prep-dual}
Let $f : Y \to X$ be a morphism of smooth projective schemes over $k$.
Assume $X$ and $Y$ equidimensional of dimensions $d$ and $e$.
Denote $a = [\Gamma_f] \in \text{Corr}^0(X, Y)$ and
$a^t = [\Gamma_f^t] \in \text{Corr}^{d - e}(Y, X)$.
Set
$\eta_X = [\Gamma_{X \to X \times X}] \in \text{Corr}^0(X \times X, X)$,
$\eta_Y = [\Gamma_{Y \to Y \times Y}] \in \text{Corr}^0(Y \times Y, Y)$,
$[X] \in \text{Corr}^{-d}(X, \Spec(k))$, and
$[Y] \in \text{Corr}^{-e}(Y, \Spec(k))$. The diagram
$$
\xymatrix{
X \otimes Y \ar[r]_{a \otimes \text{id}} \ar[d]_{\text{id} \otimes a^t} &
Y \otimes Y \ar[r]_{\eta_Y} &
Y \ar[d]^{[Y]} \\
X \otimes X \ar[r]^{\eta_X} &
X \ar[r]^{[X]} &
\Spec(k)
}
$$
is commutative in the category of correspondences.
\end{lemma}

\begin{proof}
Recall that $\text{Corr}^r(W, \Spec(k)) = \CH_{-r}(W)$ for any
smooth projective scheme $W$ over $k$
and given $c \in \text{Corr}^s(W', W)$ the composition
with $c$ agrees with pullback by $c$ as a map
$\CH_{-r}(W) \to \CH_{-r - s}(W')$
(Lemma \ref{lemma-composition-correspondences}).
Finally, we have Lemma \ref{lemma-functor-and-cycles}
which tells us how to convert this into usual
pushforward and pullback of cycles.
We have
$$
(a \otimes \text{id})^* \eta_Y^* [Y] =
(a \otimes \text{id})^* [\Delta_Y] =
(f \times \text{id})_*\Delta_Y = [\Gamma_f]
$$
and the other way around we get
$$
(\text{id} \otimes a^t)^* \eta_X^* [X] =
(\text{id} \otimes a^t)^* [\Delta_X] =
(\text{id} \times f)^![\Delta_X] = [\Gamma_f]
$$
The last equality follows from
Chow Homology, Lemma \ref{chow-lemma-lci-gysin-easy}.
In other words, going either way around the diagram we
obtain the element of $\text{Corr}^d(X \times Y, \Spec(k))$
corresponding to the cycle $\Gamma_f \subset X \times Y$.
\end{proof}







\section{Chow motives}
\label{section-chow-motives}

\noindent
We fix a base field $k$. In this section we construct an additive
Karoubian $\mathbf{Q}$-linear category $M_k$ endowed
with a symmetric monoidal structure and a contravariant functor
$$
h : \{\text{smooth projective schemes over }k\} \longrightarrow M_k
$$
which maps products to tensor products and disjoint unions to direct sums.
Our construction will be characterized by the fact that $h$ factors through
the symmetric monoidal category whose objects are smooth projective varieties
and whose morphisms are correspondences of degree $0$ such that
the image of the projector $c_2$ on $h(\mathbf{P}^1_k)$ from
Example \ref{example-decompose-P1} is invertible in $M_k$, see
Lemma \ref{lemma-characterize-motives}.
At the end of the section we will show that every motive, i.e.,
every object of $M_k$ to has a (left) dual, see Lemma \ref{lemma-dual-general}.

\medskip\noindent
A {\it motive} or a {\it Chow motive} over $k$ will be a triple
$(X, p, m)$ where
\begin{enumerate}
\item $X$ is a smooth projective scheme over $k$,
\item $p \in \text{Corr}^0(X, X)$ satisfies $p \circ p = p$,
\item $m \in \mathbf{Z}$.
\end{enumerate}
Given a second motive $(Y, q, n)$ we define a
{\it morphism of motives} or a {\it morphism of Chow motives}
to be an element of
$$
\Hom((X, p, m), (Y, q, n)) =
q \circ \text{Corr}^{n - m}(X, Y) \circ p \subset \text{Corr}^{n - m}(X, Y)
$$
Composition of morphisms of motives is defined using the composition of
correspondences defined above.

\begin{lemma}
\label{lemma-motives}
The category $M_k$ whose objects are motives over $k$ and morphisms
are morphisms of motives over $k$ is a $\mathbf{Q}$-linear category.
There is a contravariant functor
$$
h : \{\text{smooth projective schemes over }k\} \longrightarrow M_k
$$
defined by $h(X) = (X, 1, 0)$ and $h(f) = [\Gamma_f]$.
\end{lemma}

\begin{proof}
Follows immediately from Lemma \ref{lemma-contravariant-functor}.
\end{proof}

\begin{lemma}
\label{lemma-Karoubian}
The category $M_k$ is Karoubian.
\end{lemma}

\begin{proof}
Let $M = (X, p, m)$ be a motive and let $a \in \Mor(M, M)$
be a projector. Then $a = a \circ a$ both in $\Mor(M, M)$
as well as in $\text{Corr}^0(X, X)$. Set $N = (X, a, m)$.
Since we have $a = p \circ a \circ a$ in $\text{Corr}^0(X, X)$
we see that $a : N \to M$ is a morphism of $M_k$.
Next, suppose that $b : (Y, q, n) \to M$ is a morphism
such that $(1 - a) \circ b = 0$. Then $b = a \circ b$ as well as
$b = b \circ q$. Hence $b$ is a morphism $b : (Y, q, n) \to N$.
Thus we see that the projector $1 - a$ has a kernel, namely $N$
and we find that $M_k$ is Karoubian, see
Homology, Definition \ref{homology-definition-karoubian}.
\end{proof}

\noindent
We define a functor
$$
\otimes : M_k \times M_k \longrightarrow M_k
$$
On objects we use the formula
$$
(X, p, m) \otimes (Y, q, n) = (X \times Y, p \otimes q, m + n)
$$
On morphisms, we use
$$
\xymatrix{
\Mor((X, p, m), (Y, q, n)) \times
\Mor((X', p', m'), (Y', q', n')) \ar[d] \\
\Mor(
(X \times X', p \otimes p', m + m'),
(Y \times Y', q \otimes q', n + n'))
}
$$
given by the rule $(a, a') \longmapsto a \otimes a'$ where
$\otimes$ on correspondences is as in Section \ref{section-correspondences}.
This makes sense: by definition of morphisms of motives
we can write $a = q \circ c \circ p$ and $a' = q' \circ c' \circ p'$
with $c \in \text{Corr}^{n - m}(X, Y)$ and
$c' \in \text{Corr}^{n' - m'}(X', Y')$
and then we obtain
$$
a \otimes a' =
(q \circ c \circ p) \otimes (q' \circ c' \circ p') =
(q \otimes q') \circ (c \otimes c') \circ (p \otimes p')
$$
which is indeed a morphism of motives from
$(X \times X', p \otimes p', m + m')$ to
$(Y \times Y', q \otimes q', n + n')$.

\begin{lemma}
\label{lemma-motives-monoidal}
The category $M_k$ with tensor product defined as above
is symmetric monoidal with the obvious associativity and commutativity
constraints and with unit $\mathbf{1} = (\Spec(k), 1, 0)$.
\end{lemma}

\begin{proof}
Follows readily from Lemma \ref{lemma-tensor-product}. Details omitted.
\end{proof}

\noindent
The motives $\mathbf{1}(n) = (\Spec(k), 1, n)$ are useful. Observe that
$$
\mathbf{1} = \mathbf{1}(0)
\quad\text{and}\quad
\mathbf{1}(n + m) = \mathbf{1}(n) \otimes \mathbf{1}(m)
$$
Thus tensoring with $\mathbf{1}(1)$ is an autoequivalence of the
category of motives. Given a motive $M$ we sometimes write
$M(n) = M \otimes \mathbf{1}(n)$. Observe that if $M = (X, p, m)$,
then $M(n) = (X, p, m + n)$.

\begin{lemma}
\label{lemma-inverse-h2}
With notation as in Example \ref{example-decompose-P1}
\begin{enumerate}
\item
the motive $(X, c_0, 0)$ is isomorphic to the motive
$\mathbf{1} = (\Spec(k), 1, 0)$.
\item
the motive $(X, c_2, 0)$ is isomorphic to the motive
$\mathbf{1}(-1) = (\Spec(k), 1, -1)$.
\end{enumerate}
\end{lemma}

\begin{proof}
We will use Lemma \ref{lemma-contravariant-functor} without further mention.
The structure morphism $X \to \Spec(k)$ gives a correspondence
$a \in \text{Corr}^0(\Spec(k), X)$. On the other hand, the rational
point $x$ is a morphism $\Spec(k) \to X$ which gives a correspondence
$b \in \text{Corr}^0(X, \Spec(k))$. We have $b \circ a = 1$ as a
correspondence on $\Spec(k)$. The composition $a \circ b$ corresponds
to the graph of the composition $X \to x \to X$ which is
$c_0 = [x \times X]$. Thus $a = a \circ b \circ a = c_0 \circ a$
and $b = a \circ b \circ a = b \circ c_0$.
Hence, unwinding the definitions, we see that
$a$ and $b$ are mutually inverse morphisms
$a : (\Spec(k), 1, 0) \to (X, c_0, 0)$ and
$b : (X, c_0, 0) \to (\Spec(k), 1, 0)$.

\medskip\noindent
We will proceed exactly as above to prove the second statement.
Denote
$$
a' \in \text{Corr}^1(\Spec(k), X) = \CH^1(X)
$$
the class of the point $x$. Denote
$$
b' \in \text{Corr}^{-1}(X, \Spec(k)) = \CH_1(X)
$$
the class of $[X]$. We have $b' \circ a' = 1$ as a correspondence
on $\Spec(k)$ because $[x] \cdot [X] = [x]$ on
$X = \Spec(k) \times X \times \Spec(k)$. Computing the
intersection product $\text{pr}_{12}^*b' \cdot \text{pr}_{23}^*a'$
on $X \times \Spec(k) \times X$ gives the cycle
$X \times \Spec(k) \times x$. Hence
the composition $a' \circ b'$ is equal to $c_2$ as a
correspondence on $X$. Thus $a' = a' \circ b \circ a' = c_2 \circ a'$
and $b' = b' \circ a' \circ b' = b' \circ c_2$. Recall that
$$
\Mor((\Spec(k), 1, -1), (X, c_2, 0)) =
c_2 \circ \text{Corr}^1(\Spec(k), X)
\subset
\text{Corr}^1(\Spec(k), X)
$$
and
$$
\Mor((X, c_2, 0), (\Spec(k), 1, -1)) =
\text{Corr}^{-1}(X, \Spec(k)) \circ c_2
\subset
\text{Corr}^{-1}(X, \Spec(k))
$$
Hence, we see that $a'$ and $b'$ are mutually inverse morphisms
$a' : (\Spec(k), 1, -1) \to (X, c_0, 0)$ and
$b' : (X, c_0, 0) \to (\Spec(k), 1, -1)$.
\end{proof}

\begin{remark}[Lefschetz and Tate motive]
\label{remark-lefschetz-tate}
Let $X = \mathbf{P}^1_k$ and $c_2$ be as in Example \ref{example-decompose-P1}.
In the literature the motive $(X, c_2, 0)$ is sometimes called the
{\it Lefschetz motive} and depending on the reference the notation
$L$, $\mathbf{L}$, $\mathbf{Q}(-1)$, or $h^2(\mathbf{P}^1_k)$
may be used to denote it. By Lemma \ref{lemma-inverse-h2} the Lefschetz motive
is isomorphic to $\mathbf{1}(-1)$. Hence the Lefschetz motive is
invertible (Categories, Definition \ref{categories-definition-invertible})
with inverse
$\mathbf{1}(1)$. The motive $\mathbf{1}(1)$ is sometimes called the
{\it Tate motive} and depending on the reference the notation
$L^{-1}$, $\mathbf{L}^{-1}$, $\mathbf{T}$, or $\mathbf{Q}(1)$ may
be used to denote it.
\end{remark}

\begin{lemma}
\label{lemma-additive}
The category $M_k$ is additive.
\end{lemma}

\begin{proof}
Let $(Y, p, m)$ and $(Z, q, n)$ be motives. If $n = m$, then a
direct sum is given by $(Y \amalg Z, p + q, m)$, with obvious notation.
Details omitted.

\medskip\noindent
Suppose that $n < m$. Let $X$, $c_2$ be as in
Example \ref{example-decompose-P1}. Then we consider
\begin{align*}
(Z, q, n)
& =
(Z, q, m) \otimes (\Spec(k), 1, -1) \otimes \ldots \otimes
(\Spec(k), 1, -1) \\
& \cong
(Z, q, m) \otimes (X, c_2, 0) \otimes \ldots \otimes (X, c_2, 0) \\
& \cong
(Z \times X^{m - n}, q \otimes c_2 \otimes \ldots \otimes c_2, m)
\end{align*}
where we have used Lemma \ref{lemma-inverse-h2}.
This reduces us to the case discussed in the first paragraph.
\end{proof}

\begin{lemma}
\label{lemma-decompose-P1}
In $M_k$ we have $h(\mathbf{P}^1_k) \cong \mathbf{1} \oplus \mathbf{1}(-1)$.
\end{lemma}

\begin{proof}
This follows from Example \ref{example-decompose-P1} and
Lemma \ref{lemma-inverse-h2}.
\end{proof}

\begin{lemma}
\label{lemma-characterize-motives}
Let $X$, $c_2$ be as in Example \ref{example-decompose-P1}.
Let $\mathcal{C}$ be a $\mathbf{Q}$-linear Karoubian symmetric
monoidal category. Any $\mathbf{Q}$-linear functor
$$
F :
\left\{
\begin{matrix}
\text{smooth projective schemes over }k\\
\text{morphisms are correspondences of degree }0
\end{matrix}
\right\}
\longrightarrow
\mathcal{C}
$$
of symmetric monoidal categories such that the image of $F(c_2)$ on
$F(X)$ is an invertible object, factors uniquely through a functor
$F : M_k \to \mathcal{C}$ of symmetric monoidal categories.
\end{lemma}

\begin{proof}
Denote $U$ in $\mathcal{C}$ the invertible object which is assumed to exist
in the statement of the lemma. We extend $F$ to motives by setting
$$
F(X, p, m) = \left(\text{the image of
the projector }F(p)\text{ in }F(X)\right) \otimes U^{\otimes -m}
$$
which makes sense because $U$ is invertible and because $\mathcal{C}$
is Karoubian. An important feature of this choice is that
$F(X, c_2, 0) = U$. Observe that
\begin{align*}
F((X, p, m) \otimes (Y, q, n))
& =
F(X \times Y, p \otimes q, m + n) \\
& =
\left(\text{the image of }F(p \otimes q)\text{ in }F(X \times Y)\right)
\otimes U^{\otimes -m - n} \\
& =
F(X, p, m) \otimes F(Y, q, n)
\end{align*}
Thus we see that our rule is compatible with tensor products on
the level of objects (details omitted).

\medskip\noindent
Next, we extend $F$ to morphisms of motives. Suppose that
$$
a \in
\Hom((Y, p, m), (Z, q, n)) =
q \circ \text{Corr}^{n - m}(Y, Z) \circ p \subset \text{Corr}^{n - m}(Y, Z)
$$
is a morphism. If $n = m$, then $a$ is a correspondence of degree $0$
and we can use $F(a) : F(Y) \to F(Z)$ to get the desired map
$F(Y, p, m) \to F(Z, q, n)$. If $n < m$ we get canonical identifications
\begin{align*}
s : F((Z, q, n))
& \to
F(Z, q, m) \otimes U^{m - n} \\
& \to
F(Z, q, m) \otimes F(X, c_2, 0) \otimes \ldots \otimes F(X, c_2, 0) \\
& \to
F((Z, q, m) \otimes (X, c_2, 0) \otimes \ldots \otimes (X, c_2, 0)) \\
& \to
F((Z \times X^{m - n}, q \otimes c_2 \otimes \ldots \otimes c_2, m))
\end{align*}
Namely, for the first isomorphism we use the definition of $F$ on motives
above. For the second, we use the choice of $U$. For the third we use
the compatibility of $F$ on tensor products of motives. The fourth
is the definition of tensor products on motives. On the other hand, since
we similarly have an isomorphism
$$
\sigma : (Z, q, n) \to
(Z \times X^{m - n}, q \otimes c_2 \otimes \ldots \otimes c_2, m)
$$
(see proof of Lemma \ref{lemma-additive}). Composing $a$ with this isomorphism
gives
$$
\sigma \circ a \in
\Hom((Y, p, m),
(Z \times X^{m - n}, q \otimes c_2 \otimes \ldots \otimes c_2, m))
$$
Putting everything together we obtain
$$
s^{-1} \circ F(\sigma \circ a) :
F(Y, p, m) \to
F(Z, q, n)
$$
If $n > m$ we similarly define isomorphisms
$$
t : F((Y, p, m)) \to
F((Y \times X^{n - m}, p \otimes c_2 \otimes \ldots \otimes c_2, n))
$$
and
$$
\tau : (Y, p, m)) \to
(Y \times X^{n - m}, p \otimes c_2 \otimes \ldots \otimes c_2, n)
$$
and we set $F(a) = F(a \circ \tau^{-1}) \circ t$.
We omit the verification that this construction defines a functor
of symmetric monoidal categories.
\end{proof}

\begin{lemma}
\label{lemma-dual}
Let $X$ be a smooth projective scheme over $k$ which is equidimensional
of dimension $d$. Then $h(X)(d)$ is a left dual to $h(X)$ in $M_k$.
\end{lemma}

\begin{proof}
We will use Lemma \ref{lemma-composition-correspondences}
without further mention. We compute
$$
\Hom(\mathbf{1}, h(X) \otimes h(X)(d)) =
\text{Corr}^d(\Spec(k), X \times X) = \CH^d(X \times X)
$$
Here we have $\eta = [\Delta]$. On the other hand, we have
$$
\Hom(h(X)(d) \otimes h(X), \mathbf{1}) =
\text{Corr}^{-d}(X \times X, \Spec(k)) = \CH_d(X \times X)
$$
and here we have the class $\epsilon = [\Delta]$
of the diagonal as well. The composition of the correspondence
$[\Delta] \otimes 1$ with $1 \otimes [\Delta]$ either way
is the correspondence $[\Delta] = 1$ in $\text{Corr}^0(X, X)$ which proves
the required diagrams of
Categories, Definition \ref{categories-definition-dual} commute.
Namely, observe that
$$
[\Delta] \otimes 1 \in \text{Corr}^d(X, X \times X \times X) =
\CH^{2d}(X \times X \times X \times X)
$$
is given by the class of the cycle
$\text{pr}^{1234, -1}_{23}(\Delta) \cap \text{pr}^{1234, -1}_{14}(\Delta)$ with
obvious notation. Similarly, the class
$$
1 \otimes [\Delta] \in \text{Corr}^{-d}(X \times X \times X, X) =
\CH^{2d}(X \times X \times X \times X)
$$
is given by the class of the cycle
$\text{pr}^{1234, -1}_{23}(\Delta) \cap \text{pr}^{1234, -1}_{14}(\Delta)$.
The composition $(1 \otimes [\Delta]) \circ ([\Delta] \otimes 1)$
is by definition the pushforward $\text{pr}^{12345}_{15, *}$
of the intersection product
$$
[\text{pr}^{12345, -1}_{23}(\Delta) \cap \text{pr}^{12345, -1}_{14}(\Delta)]
\cdot
[\text{pr}^{12345, -1}_{34}(\Delta) \cap \text{pr}^{12345, -1}_{15}(\Delta)]
=
[\text{small diagonal in } X^5]
$$
which is equal to $\Delta$ as desired. We omit the proof of the formula
for the composition in the other order.
\end{proof}

\begin{lemma}
\label{lemma-dual-general}
Every object of $M_k$ has a left dual.
\end{lemma}

\begin{proof}
Let $M = (X, p, m)$ be an object of $M_k$. Then $M$ is a summand of
$(X, 0, m) = h(X)(m)$.
By Homology, Lemma \ref{homology-lemma-Karoubian-dual}
it suffices to show that
$h(X)(m) = h(X) \otimes \mathbf{1}(m)$ has a dual.
By construction $\mathbf{1}(-m)$ is a left dual of $\mathbf{1}(m)$.
Hence it suffices to show that $h(X)$ has a left dual, see
Categories, Lemma \ref{categories-lemma-tensor-dual}.
Let $X = \coprod X_i$ be the decomposition of $X$ into
irreducible components. Then $h(X) = \bigoplus h(X_i)$
and it suffices to show that $h(X_i)$ has a left dual, see
Homology, Lemma \ref{homology-lemma-additive-dual}.
This follows from Lemma \ref{lemma-dual}.
\end{proof}






\section{Chow groups of motives}
\label{section-chow-groups-motives}

\noindent
We define the Chow groups of a motive as follows.

\begin{definition}
\label{definition-chow-group-motives}
Let $k$ be a base field. Let $M = (X, p, m)$ be a Chow motive over $k$.
For $i \in \mathbf{Z}$ we define the {\it $i$th Chow group of $M$}
by the formula
$$
\CH^i(M) = p\left(\CH^{i + m}(X) \otimes \mathbf{Q}\right)
$$
\end{definition}

\noindent
We have $\CH^i(h(X)) = \CH^i(X) \otimes \mathbf{Q}$
if $X$ is a smooth projective scheme over $k$.

\medskip\noindent
Observe that $\CH^i(-)$ is a functor from $M_k$ to $\mathbf{Q}$-vector spaces.
Indeed, if $c : M \to N$ is a morphism of motives
$M = (X, p, m)$ and $N = (Y, q, n)$, then $c$ is a correspondence of
degree $n - m$ from $X$ to $Y$ and hence pushforward along $c$
(Section \ref{section-correspondences}) is a family of maps
$$
c_* :
\CH^{i + m}(X) \otimes \mathbf{Q}
\longrightarrow
\CH^{i + n}(Y) \otimes \mathbf{Q}
$$
Since $c = q \circ c \circ p$ by definition of morphisms of motives,
we see that indeed we obtain
$$
c_* : \CH^i(M) \to \CH^i(N)
$$
for all $i \in \mathbf{Z}$. This is compatible with compositions of
morphisms of motives by Lemma \ref{lemma-composition-correspondences}.
This functoriality of Chow groups can also be deduced from the following
lemma.

\begin{lemma}
\label{lemma-chow-groups-representable}
Let $k$ be a base field. The functor $\CH^i(-)$ on the category
of motives $M_k$ is representable by $\mathbf{1}(-i)$, i.e., we
have
$$
\CH^i(M) = \Hom_{M_k}(\mathbf{1}(-i), M)
$$
functorially in $M$ in $M_k$.
\end{lemma}

\begin{proof}
Immediate from the definitions and
Lemma \ref{lemma-composition-correspondences}.
\end{proof}

\noindent
The reader can imagine that we can use
Lemma \ref{lemma-chow-groups-representable}, the Yoneda lemma, and
the duality in Lemma \ref{lemma-dual} to obtain the following.

\begin{lemma}[Manin]
\label{lemma-manin}
Let $k$ be a base field. Let $c : M \to N$ be a morphism of motives.
If for every smooth projective scheme $X$ over $k$ the map
$c \otimes 1 : M \otimes h(X) \to N \otimes h(X)$ induces an isomorphism on
Chow groups, then $c$ is an isomorphism.
\end{lemma}

\begin{proof}
Any object $L$ of $M_k$ is a summand of $h(X)(m)$ for some smooth projective
scheme $X$ over $k$ and some $m \in \mathbf{Z}$. Observe that the Chow groups
of $M \otimes h(X)(m)$ are the same as the Chow groups of of $M \otimes h(X)$
up to a shift in degrees. Hence our assumption implies
that $c \otimes 1 : M \otimes L \to N \otimes L$ induces an isomorphism on
Chow groups for every object $L$ of $M_k$. By
Lemma \ref{lemma-chow-groups-representable}
we see that
$$
\Hom_{M_k}(\mathbf{1}, M \otimes L) \to
\Hom_{M_k}(\mathbf{1}, N \otimes L)
$$
is an isomorphism for every $L$. Since every object of $M_k$ has a left dual
(Lemma \ref{lemma-dual-general}) we conclude that
$$
\Hom_{M_k}(K, M) \to \Hom_{M_k}(K, N)
$$
is an isomorphism for every object $K$ of $M_k$, see
Categories, Lemma \ref{categories-lemma-left-dual}.
We conclude by the Yoneda lemma
(Categories, Lemma \ref{categories-lemma-yoneda}).
\end{proof}




\section{Projective space bundle formula}
\label{section-projective-space-bundle}

\noindent
Let $k$ be a base field. Let $X$ be a smooth projective scheme over $k$.
Let $\mathcal{E}$ be a locally free $\mathcal{O}_X$-module of rank $r$.
Our convention is that the {\it projective bundle associated to
$\mathcal{E}$} is the morphism
$$
\xymatrix{
P = \mathbf{P}(\mathcal{E}) =
\underline{\text{Proj}}_X(\text{Sym}^*(\mathcal{E}))
\ar[r]^-p
& X
}
$$
over $X$ with $\mathcal{O}_P(1)$ normalized so that
$p_*(\mathcal{O}_P(1)) = \mathcal{E}$. Recall that
$$
[\Gamma_p] \in \text{Corr}^0(X, P) \subset \CH^*(X \times P) \otimes \mathbf{Q}
$$
See Example \ref{example-graph-correspondence}.
For $i = 0, \ldots, r - 1$ consider the correspondences
$$
c_i = c_1(\text{pr}_2^*\mathcal{O}_P(1))^i \cap [\Gamma_p]
\in \text{Corr}^i(X, P)
$$
We may and do think of $c_i$ as a morphism $h(X)(-i) \to h(P)$.

\begin{lemma}[Projective space bundle formula]
\label{lemma-projective-space-bundle-formula}
In the situation above, the map
$$
\sum\nolimits_{i = 0, \ldots, r - 1} c_i :
\bigoplus\nolimits_{i = 0, \ldots, r - 1} h(X)(-i)
\longrightarrow
h(P)
$$
is an isomorphism in the category of motives.
\end{lemma}

\begin{proof}
By Lemma \ref{lemma-manin} it suffices to show that
our map defines an isomorphism on Chow groups of motives
after taking the product with any smooth projective scheme $Z$.
Observe that $P \times Z \to X \times Z$ is the projective
bundle associated to the pullback of $\mathcal{E}$ to $X \times Z$.
Hence the statement on Chow groups is true by the projective space bundle
formula given in
Chow Homology, Lemma \ref{chow-lemma-chow-ring-projective-bundle}.
Namely, pushforward of cycles along $[\Gamma_p]$ is given by pullback
of cycles by $p$ according to Lemma \ref{lemma-functor-and-cycles} and
Chow Homology, Lemma \ref{chow-lemma-lci-gysin-flat}. Hence pushforward
along $c_i$ sends $\alpha$ to $c_1(\mathcal{O}_P(1))^i \cap p^*\alpha$.
Some details omitted.
\end{proof}

\noindent
In the situation above, for $j = 0, \ldots, r - 1$ consider
the correspondences
$$
c'_j = c_1(\text{pr}_1^*\mathcal{O}_P(1))^{r - 1 - j} \cap [\Gamma_p^t] \in
\text{Corr}^{-j}(P, X)
$$
For $i, j \in \{0, \ldots, r - 1\}$ we have
$$
c'_j \circ c_i =
\text{pr}_{13, *}\left(
c_1(\text{pr}_2^*\mathcal{O}_P(1))^{i + r - 1 - j} \cap
(\text{pr}_{12}^*[\Gamma_p] \cdot \text{pr}_{23}^*[\Gamma_p^t])
\right)
$$
The cycles $\text{pr}_{12}^{-1}\Gamma_p$ and 
$\text{pr}_{23}^{-1}\Gamma_p^t$ intersect transversally and
with intersection equal to the image of
$(p, 1, p) : P \to X \times P \times X$.
Observe that the fibres of
$(p, p) = \text{pr}_{13} \circ (p, 1, p) : P \to X \times X$
have dimension $r - 1$. We immediately conclude
$c'_j \circ c_i = 0$ for $i + r - 1 - j < r - 1$, in other words when $i < j$.
On the other hand, by the projective space bundle formula
(Chow Homology, Lemma \ref{chow-lemma-chow-ring-projective-bundle})
the cycle $c_1(\mathcal{O}_P(1))^{r - 1} \cap [P]$ maps
to $[X]$ in $X$. Hence for $i = j$ the pushforward above
gives the class of the diagonal and hence
we see that
$$
c'_i \circ c_i = 1 \in \text{Corr}^0(X, X)
$$
for all $i \in \{0, \ldots, r - 1\}$. Thus we see that the matrix
of the composition
$$
\bigoplus h(X)(-i)
\xrightarrow{\bigoplus c_i}
h(P)
\xrightarrow{\bigoplus c'_j}
\bigoplus h(X)(-j)
$$
is invertible (upper triangular with $1$s on the diagonal).
We conclude from the projective space bundle formula
(Lemma \ref{lemma-projective-space-bundle-formula})
that also the composition the other way around is
invertible, but it seems a bit harder to prove this directly.

\begin{lemma}
\label{lemma-diagonal-projective-bundle}
Let $p : P \to X$ be as in Lemma \ref{lemma-projective-space-bundle-formula}.
The class $[\Delta_P]$ of the diagonal of $P$ in $\CH^*(P \times P)$
can be written
as
$$
[\Delta_P] =
\left(\sum\nolimits_{i = 0, \ldots, r - 1}
{r - 1 \choose i} c_{r - 1 - i}(\text{pr}_1^*\mathcal{S}^\vee) \cap
c_1(\text{pr}_2^*\mathcal{O}_P(1))^i\right)
\cap
(p \times p)^*[\Delta_X]
$$
where $\mathcal{S}$ is the kernel of the canonical surjection
$p^*\mathcal{E} \to \mathcal{O}_P(1)$.
\end{lemma}

\begin{proof}
Observe that $(p \times p)^*[\Delta_X] = [P \times_X P]$.
Since $\Delta_P \subset P \times_X P \subset P \times P$
and since capping with Chern classes commutes with proper pushforward
(Chow Homology, Lemma \ref{chow-lemma-pushforward-cap-cj})
it suffices to show that the class of
$\Delta_P \subset P \times_X P$ in $\CH^*(P \times_X P)$
is equal to
$$
\left(\sum\nolimits_{i = 0, \ldots, r - 1}
{r - 1 \choose i} c_{r - 1 - i}(q_1^*\mathcal{S}^\vee) \cap
c_1(q_2^*\mathcal{O}_P(1))^i\right)
\cap
[P \times_X P]
$$
where $q_i : P \times_X P \to P$, $i = 1, 2$ are the projections.
Set $q = p \circ q_1 = p \circ q_2 : P \times_X P \to X$.
Consider the maps
$$
q_1^*\mathcal{S} \otimes q_2^*\mathcal{O}_P(-1) \to
q^*\mathcal{E} \otimes q^*\mathcal{E}^\vee \to
\mathcal{O}_{P \times_X P}
$$
where the final arrow is the pullback by $q$ of the evaluation map
$\mathcal{E} \otimes_{\mathcal{O}_X} \mathcal{E}^\vee \to \mathcal{O}_X$.
The source of the composition is a module locally free of rank $r - 1$
and a local calculation shows that this map vanishes exactly along
$\Delta_P$. By Chow Homology, Lemma \ref{chow-lemma-top-chern-class}
the class $[\Delta_P]$ is the top Chern class of the dual
$$
q_1^*\mathcal{S}^\vee \otimes q_2^*\mathcal{O}_P(1)
$$
The desired result follows from Chow Homology, Lemma
\ref{chow-lemma-chern-classes-E-tensor-L}.
\end{proof}








\section{Classical Weil cohomology theories}
\label{section-axioms-classical}

\noindent
In this section we define what we will call a classical Weil cohomology
theory. This is exactly what is called a Weil cohomology theory in
\cite[Section 1.2]{Kleiman-cycles}.

\medskip\noindent
We fix an algebraically closed field $k$ (the base field).
In this section {\it variety} will mean a variety over $k$, see
Varieties, Section \ref{varieties-section-varieties}.
We fix a field $F$ of characteristic $0$ (the coefficient field).
A Weil cohomology theory is given by data (D1), (D2), and (D3)
subject to axioms (A), (B), and (C).

\medskip\noindent
The data is given by:
\begin{enumerate}
\item[(D1)] A contravariant functor $H^*$ from the category
of smooth projective varieties to the category of
graded commutative $F$-algebras.
\item[(D2)] For every smooth projective variety $X$
a group homomorphism $\gamma : \CH^i(X) \to H^{2i}(X)$.
\item[(D3)] For every smooth projective variety $X$ of dimension $d$
a map $\int_X : H^{2d}(X) \to F$.
\end{enumerate}
We make some remarks to explain what this means and to introduce
some terminology associated with this.

\medskip\noindent
Remarks on (D1). Given a smooth projective variety $X$
we say that $H^*(X)$ is the {\it cohomology} of $X$. Given a morphism
$f : X \to Y$ of smooth projective varieties we denote
$f^* : H^*(Y) \to H^*(X)$ the map $H^*(f)$ and we call it the
{\it pullback map}.

\medskip\noindent
Remarks on (D2). The map $\gamma$ is called the {\it cycle class map}.
We say that $\gamma(\alpha)$ is the {\it cohomology class} of $\alpha$.
If $Z \subset Y \subset X$ are closed subschemes with $Y$ and $X$
smooth projective varieties and $Z$ integral, then $[Z]$ could
mean the class of the cycle $[Z]$ in $\CH^*(Y)$ or in $\CH^*(X)$.
In this case the notation $\gamma([Z])$ is ambiguous and the intended meaning
has to be deduced from context.

\medskip\noindent
Remarks on (D3). The map $\int_X$ is sometimes called the
{\it trace map} and is sometimes denoted $\text{Tr}_X$.

\medskip\noindent
The first axiom is often called {\it Poincar\'e duality}
\begin{enumerate}
\item[(A)] Let $X$ be a smooth projective variety of dimension $d$. Then
\begin{enumerate}
\item $\dim_F H^i(X) < \infty$ for all $i$,
\item $H^i(X) \times H^{2d - i}(X) \rightarrow H^{2d}(X) \rightarrow F$
is a perfect pairing for all $i$ where the final
map is the trace map $\int_X$,
\item $H^i(X) = 0$ unless $i \in [0, 2d]$, and
\item $\int_X : H^{2d}(X) \to F$ is an isomorphism.
\end{enumerate}
\end{enumerate}
Let $f : X \to Y$ be a morphism of smooth projective varieties
with $\dim(X) = d$ and $\dim(Y) = e$. Using Poincar\'e duality
we can define a {\it pushforward}
$$
f_* : H^{2d - i}(X) \longrightarrow H^{2e - i}(Y)
$$
as the contragredient of the linear map $f^* : H^i(Y) \to H^i(X)$. In a
formula, for $a \in H^{2d - i}(X)$, the element $f_*a \in H^{2e - i}(Y)$
is characterized by
$$
\int_X f^*b \cup a = \int_Y b \cup f_*a
$$
for all $b \in H^i(Y)$.

\begin{lemma}
\label{lemma-pushforward-classical}
Assume given (D1) and (D3) satisfying (A). For $f : X \to Y$
a morphism of smooth projective varieties we have
$f_*(f^*b \cup a) = b \cup f_*a$. If $g : Y \to Z$ is a second morphism
of smooth projective varieties, then $g_* \circ f_* = (g \circ f)_*$.
\end{lemma}

\begin{proof}
The first equality holds because
$$
\int_Y c \cup b \cup f_*a =
\int_X f^*c \cup f^*b \cup a =
\int_Y c \cup f_*(f^*b \cup a).
$$
The second equality holds because
$$
\int_Z c \cup (g \circ f)_*a = \int_X (g \circ f)^*c \cup a =
\int_X f^* g^* c \cup a = \int_Y g^*c \cup f_*a = \int_Z c \cup g_*f_*a
$$
This ends the proof.
\end{proof}

\noindent
The second axiom says that $H^*$ respects the monoidal structure
given by products via the {\it K\"unneth formula}
\begin{enumerate}
\item[(B)] Let $X$ and $Y$ be smooth projective varieties. The map
$$
H^*(X) \otimes_F H^*(Y) \to H^*(X \times Y),\quad
a \otimes b \mapsto \text{pr}_1^*a \cup \text{pr}_2^*b
$$
is an isomorphism.
\end{enumerate}
The third axiom concerns the cycle class maps
\begin{enumerate}
\item[(C)] The cycle class maps satisfy the following rules
\begin{enumerate}
\item for a morphism $f : X \to Y$ of smooth projective varieties
we have $\gamma(f^!\beta) = f^*\gamma(\beta)$ for $\beta \in \CH^*(Y)$,
\item for a morphism $f : X \to Y$ of smooth projective varieties we have
$\gamma(f_*\alpha) = f_*\gamma(\alpha)$ for $\alpha \in \CH^*(X)$,
\item for any smooth projective variety $X$ we have
$\gamma(\alpha \cdot \beta) = \gamma(\alpha) \cup \gamma(\beta)$
for $\alpha, \beta \in \CH^*(X)$, and
\item $\int_{\Spec(k)} \gamma([\Spec(k)]) = 1$.
\end{enumerate}
\end{enumerate}

\begin{remark}
\label{remark-replace-cup-product-classical}
Let $X$ be a smooth projective variety. We obtain maps
$$
H^*(X) \otimes_F H^*(X) \longrightarrow H^*(X \times X)
\xrightarrow{\Delta^*} H^*(X)
$$
where the first arrow is as in axiom (B) and $\Delta^*$
is pullback along the diagonal morphism $\Delta : X \to X \times X$.
The composition is the cup product as pullback is an algebra homomorphism and
$\text{pr}_i \circ \Delta = \text{id}$.
On the other hand, given cycles $\alpha, \beta$ on $X$ 
the intersection product is defined by the formula
$$
\alpha \cdot \beta =
\Delta^!(\alpha \times \beta)
$$
In other words, $\alpha \cdot \beta$ is the pullback of the
exterior product $\alpha \times \beta$ on $X \times X$ by
the diagonal. Note also that
$\alpha \times \beta = \text{pr}_1^*\alpha \cdot \text{pr}_2^*\beta$
in $\CH^*(X \times X)$ (we omit the proof). Hence, given axiom (C)(a),
axiom (C)(c) is equivalent to the statement that $\gamma$ is
compatible with exterior product in the sense that
$\gamma(\alpha \times \beta)$ is equal to
$\text{pr}_1^*\gamma(\alpha) \cup \text{pr}_2^*\gamma(\beta)$.
This is how axiom (C)(c) is formulated in \cite{Kleiman-cycles}.
\end{remark}

\begin{definition}
\label{definition-weil-cohomology-theory-classical}
Let $k$ be an algebraically closed field.
Let $F$ be a field of characteristic $0$.
A {\it classical Weil cohomology theory} over $k$ with coefficients in $F$
is given by data (D1), (D2), and (D3) satisfying
Poincar\'e duality, the K\"unneth formula, and compatibility
with cycle classes, more precisely, satisfying (A), (B), and (C).
\end{definition}

\noindent
We do a tiny bit of work.

\begin{lemma}
\label{lemma-degrees-cycles-classical}
Let $H^*$ be a classical Weil cohomology theory
(Definition \ref{definition-weil-cohomology-theory-classical}).
Let $X$ be a smooth projective variety of dimension $d$. The diagram
$$
\xymatrix{
\CH^d(X) \ar[r]_-\gamma \ar@{=}[d] &
H^{2d}(X) \ar[d]^{\int_X} \\
\CH_0(X) \ar[r]^\deg & F
}
$$
commutes where $\deg : \CH_0(X) \to \mathbf{Z}$ is the degree of
zero cycles discussed in Chow Homology, Section
\ref{chow-section-degree-zero-cycles}.
\end{lemma}

\begin{proof}
The result holds for $\Spec(k)$ by axiom (C)(d). Let $x : \Spec(k) \to X$
be a closed point of $X$. Then we have $\gamma([x]) = x_*\gamma([\Spec(k)])$
in $H^{2d}(X)$ by axiom (C)(b). Hence $\int_X \gamma([x]) = 1$ by the
definition of $x_*$.
\end{proof}

\begin{lemma}
\label{lemma-trace-product-classical}
Let $H^*$ be a classical Weil cohomology theory
(Definition \ref{definition-weil-cohomology-theory-classical}).
Let $X$ and $Y$ be smooth projective varieties.
Then $\int_{X \times Y} = \int_X \otimes \int_Y$.
\end{lemma}

\begin{proof}
Say $\dim(X) = d$ and $\dim(Y) = e$. By axiom (B) we have
$H^{2d + 2e}(X \times Y) = H^{2d}(X) \otimes H^{2e}(Y)$
and by axiom (A)(d) this is $1$-dimensional.
By Lemma \ref{lemma-degrees-cycles-classical}
this $1$-dimensional vector space generated by the
class $\gamma([x \times y])$ of a closed point $(x, y)$ and
$\int_{X \times Y} \gamma([x \times y]) = 1$.
Since $\gamma([x \times y]) = \gamma([x]) \otimes \gamma([y])$
by axioms (C)(a) and (C)(c) and since $\int_X \gamma([x]) = 1$ and
$\int_Y \gamma([y]) = 1$ we conclude.
\end{proof}

\begin{lemma}
\label{lemma-pr2star-classical}
Let $H^*$ be a classical Weil cohomology theory
(Definition \ref{definition-weil-cohomology-theory-classical}).
Let $X$ and $Y$ be smooth projective varieties.
Then $\text{pr}_{2, *} : H^*(X \times Y) \to H^*(Y)$
sends $a \otimes b$ to $(\int_X a) b$.
\end{lemma}

\begin{proof}
This is equivalent to the result of Lemma \ref{lemma-trace-product-classical}.
\end{proof}

\begin{lemma}
\label{lemma-class-diagonal-classical}
Let $H^*$ be a classical Weil cohomology theory
(Definition \ref{definition-weil-cohomology-theory-classical}).
Let $X$ be a smooth projective variety of dimension $d$.
Choose a basis $e_{i, j}, j = 1, \ldots, \beta_i$ of $H^i(X)$ over $F$.
Using K\"unneth write
$$
\gamma([\Delta]) =
\sum\nolimits_{i = 0, \ldots, 2d}
\sum\nolimits_j e_{i, j} \otimes e'_{2d - i , j}
\quad\text{in}\quad
\bigoplus\nolimits_i H^i(X) \otimes_F H^{2d - i}(X)
$$
with $e'_{2d - i, j} \in H^{2d - i}(X)$.
Then $\int_X e_{i, j} \cup e'_{2d - i, j'} = (-1)^i\delta_{jj'}$.
\end{lemma}

\begin{proof}
Recall that $\Delta^* : H^*(X \times X) \to H^*(X)$ is equal to the
cup product map $H^*(X) \otimes_F H^*(X) \to H^*(X)$, see
Remark \ref{remark-replace-cup-product-classical}. On the other hand we have
$\gamma([\Delta]) = \Delta_*\gamma([X]) = \Delta_*1$ by
axiom (C)(b) and the fact that $\gamma([X]) = 1$. Namely,
$[X] \cdot [X] = [X]$ hence by axiom (C)(c) the cohomology class
$\gamma([X])$ is $0$ or $1$ in the $1$-dimensional $F$-algebra $H^0(X)$;
here we have also used axioms (A)(d) and (A)(b).
But $\gamma([X])$ cannot be zero as $[X] \cdot [x] = [x]$
for a closed point $x$ of $X$ and we have the nonvanishing
of $\gamma([x])$ by Lemma \ref{lemma-degrees-cycles-classical}.
Hence
$$
\int_{X \times X} \gamma([\Delta]) \cup a \otimes b =
\int_{X \times X} \Delta_*1 \cup a \otimes b =
\int_X a \cup b
$$
by the definition of $\Delta_*$. On the other hand, we have
$$
\int_{X \times X} (\sum e_{i, j} \otimes e'_{2d -i , j}) \cup a \otimes b =
\sum (\int_X a \cup e_{i, j})(\int_X e'_{2d - i, j} \cup b)
$$
by Lemma \ref{lemma-trace-product-classical}; note that we made
two switches of order so that the sign is $1$.
Thus if we choose $a$ such that $\int_X a \cup e_{i, j} = 1$
and all other pairings equal to zero, then we conclude that
$\int_X e'_{2d - i, j} \cup b = \int_X a \cup b$ for all $b$, i.e.,
$e'_{2d - i, j} = a$. This proves the lemma.
\end{proof}

\begin{lemma}
\label{lemma-square-diagonal-classical}
Let $H^*$ be a classical Weil cohomology theory
(Definition \ref{definition-weil-cohomology-theory-classical}).
Let $X$ be a smooth projective variety. We have
$$
\sum\nolimits_{i = 0, \ldots, 2\dim(X)} (-1)^i\dim_F H^i(X) =
\deg([\Delta] \cdot [\Delta]) = \deg(c_d(\mathcal{T}_X) \cap [X])
$$
\end{lemma}

\begin{proof}
Equality on the right. We have
$[\Delta] \cdot [\Delta] = \Delta_*(\Delta^![\Delta])$
(Chow Homology, Lemma \ref{chow-lemma-intersect-regularly-embedded}).
Since $\Delta_*$ preserves degrees of $0$-cycles it suffices to compute
the degree of $\Delta^![\Delta]$. The class $\Delta^![\Delta]$ is given
by capping $[\Delta]$ with
the top Chern class of the normal sheaf of $\Delta \subset X \times X$
(Chow Homology, Lemma \ref{chow-lemma-gysin-fundamental}).
Since the conormal sheaf of $\Delta$ is $\Omega_{X/k}$
(Morphisms, Lemma \ref{morphisms-lemma-differentials-diagonal})
we see that the normal sheaf is equal to the tangent sheaf
$\mathcal{T}_X = \SheafHom_{\mathcal{O}_X}(\Omega_{X/k}, \mathcal{O}_X)$
as desired.

\medskip\noindent
Equality on the left. By Lemma \ref{lemma-degrees-cycles-classical} we have
\begin{align*}
\deg([\Delta] \cdot [\Delta])
& =
\int_{X \times X} \gamma([\Delta]) \cup \gamma([\Delta]) \\
& =
\int_{X \times X} \Delta_*1 \cup \gamma([\Delta]) \\
& =
\int_{X \times X} \Delta_*(\Delta^*\gamma([\Delta])) \\
& =
\int_X \Delta^*\gamma([\Delta])
\end{align*}
Write $\gamma([\Delta]) = \sum  e_{i, j} \otimes e'_{2d - i , j}$
as in Lemma \ref{lemma-class-diagonal-classical}.
Recalling that $\Delta^*$ is given by cup product we obtain
$$
\int_X \sum\nolimits_{i, j} e_{i, j} \cup e'_{2d - i, j} =
\sum\nolimits_{i, j} \int_X e_{i, j} \cup e'_{2d - i, j} =
\sum\nolimits_{i, j} (-1)^i = \sum (-1)^i\beta_i
$$
as desired.
\end{proof}




\noindent
We will now tie classical Weil cohomology theories in with motives as follows.

\begin{lemma}
\label{lemma-from-functor-to-weil-classical}
Let $k$ be an algebraically closed field. Let $F$ be a field of
characteristic $0$. Consider a $\mathbf{Q}$-linear functor
$$
G : M_k \longrightarrow \text{graded }F\text{-vector spaces}
$$
of symmetric monoidal categories such that $G(\mathbf{1}(1))$
is nonzero only in degree $-2$. Then we obtain data (D1), (D2), (D3)
satisfying all of (A), (B), (C) except for possibly (A)(c) and (A)(d).
\end{lemma}

\begin{proof}
We obtain a contravariant functor from the category of smooth
projective varieties to the category of graded $F$-vector spaces
by setting $H^*(X) = G(h(X))$. By assumption we have a canonical
isomorphism
$$
H^*(X \times Y) = G(h(X \times Y)) = G(h(X) \otimes h(Y)) =
G(h(X)) \otimes G(h(Y)) = H^*(X) \otimes H^*(Y)
$$
compatible with pullbacks. Using pullback along the diagonal
$\Delta : X \to X \times X$ we obtain a canonical map
$$
H^*(X) \otimes H^*(X) = H^*(X \times X) \to H^*(X)
$$
of graded vector spaces compatible with pullbacks.
This defines a functorial graded $F$-algebra structure on
$H^*(X)$. Since $\Delta$ commutes with the commutativity
constraint $h(X) \otimes h(X) \to h(X) \otimes h(X)$ (switching the factors)
and since $G$ is a functor of symmetric monoidal categories (so compatible with
commutativity constraints), and by our convention in
Homology, Example \ref{homology-example-graded-vector-spaces}
we conclude that $H^*(X)$ is a graded
commutative algebra! Hence we get our datum (D1).

\medskip\noindent
Since $\mathbf{1}(1)$ is invertible in the category of motives
we see that $G(\mathbf{1}(1))$ is invertible in the category of
graded $F$-vector spaces. Thus $\sum_i \dim_F G^i(\mathbf{1}(1)) = 1$.
By assumption we only get something nonzero in degree $-2$ and we may
choose an isomorphism $F[2] \to G(\mathbf{1}(1))$ of graded $F$-vector spaces.
Here and below $F[n]$ means the graded $F$-vector space which has
$F$ in degree $-n$ and zero elsewhere. Using compatibility with
tensor products, we find for all $n \in \mathbf{Z}$ an isomorphism
$F[2n] \to G(\mathbf{1}(n))$ compatible with tensor products.

\medskip\noindent
Let $X$ be a smooth projective variety. By
Lemma \ref{lemma-composition-correspondences} we have
$$
\CH^r(X) \otimes \mathbf{Q} = \text{Corr}^r(\Spec(k), X) =
\Hom(\mathbf{1}(-r), h(X))
$$
Applying the functor $G$ we obtain
$$
\gamma :
\CH^r(X) \otimes \mathbf{Q} \longrightarrow
\Hom(G(\mathbf{1}(-r)), H^*(X)) = H^{2r}(X)
$$
This is the datum (D2).

\medskip\noindent
Let $X$ be a smooth projective variety of dimension $d$. By
Lemma \ref{lemma-composition-correspondences} we have
$$
\Mor(h(X)(d), \mathbf{1}) = \Mor((X, 1, d), (\Spec(k), 1, 0)) =
\text{Corr}^{-d}(X, \Spec(k)) = \CH_d(X)
$$
Thus the class of the cycle $[X]$ in $\CH_d(X)$ defines a morphism
$h(X)(d) \to \mathbf{1}$. Applying $G$ we obtain
$$
H^*(X) \otimes F[-2d] = G(h(X)(d)) \longrightarrow G(\mathbf{1}) = F
$$
This map is zero except in degree $0$ where we obtain
$\int_X : H^{2d}(X) \to F$. This is the datum (D3).

\medskip\noindent
Let $X$ be a smooth projective variety of dimension $d$.
By Lemma \ref{lemma-dual}
we know that $h(X)(d)$ is a left dual to $h(X)$. Hence
$G(h(X)(d)) = H^*(X) \otimes F[-2d]$ is a left dual to
$H^*(X)$ in the category of graded $F$-vector spaces.
By Homology, Lemma \ref{homology-lemma-left-dual-graded-vector-spaces}
we find that $\sum_i \dim_F H^i(X) < \infty$ and that
$\epsilon : h(X)(d) \otimes h(X) \to \mathbf{1}$ produces
nondegenerate pairings $H^{2d - i}(X) \otimes_F H^i(X) \to F$.
In the proof of Lemma \ref{lemma-dual} we have seen that
$\epsilon$ is given by $[\Delta]$ via the identifications
$$
\Hom(h(X)(d) \otimes h(X), \mathbf{1}) =
\text{Corr}^{-d}(X \times X, \Spec(k)) =
\CH_d(X \times X)
$$
Thus $\epsilon$ is the composition of $[X] : h(X)(d) \to \mathbf{1}$
and $h(\Delta)(d) : h(X)(d) \otimes h(X) \to h(X)(d)$. It follows
that the pairings above are given by cup product followed by
$\int_X$. This proves axiom (A) parts (a) and (b).

\medskip\noindent
Axiom (B) follows from the assumption that $G$ is compatible
with tensor structures and our construction of the cup product above.

\medskip\noindent
Axiom (C). Our construction of $\gamma$ takes a cycle $\alpha$ on $X$,
interprets it as a correspondence $a$ from $\Spec(k)$ to $X$ of some degree,
and then applies $G$. If $f : Y \to X$ is a morphism of smooth projective
varieties, then $f^!\alpha$ is the pushforward (!) of $\alpha$
by the correspondence $[\Gamma_f]$ from $X$ to $Y$, see
Lemma \ref{lemma-functor-and-cycles}. Hence
$f^!\alpha$ viewed as a correspondence from $\Spec(k)$ to $Y$
is equal to $a \circ [\Gamma_f]$, see
Lemma \ref{lemma-composition-correspondences}.
Since $G$ is a functor, we conclude
$\gamma$ is compatible with pullbacks, i.e., axiom (C)(a) holds.

\medskip\noindent
Let $f : Y \to X$ be a morphism of smooth projective varieties and
let $\beta \in \CH^r(Y)$ be a cycle on $Y$. We have to show that
$$
\int_Y \gamma(\beta) \cup f^*c = \int_X \gamma(f_*\beta) \cup c
$$
for all $c \in H^*(X)$. Let $a, a^t, \eta_X, \eta_Y, [X], [Y]$
be as in Lemma \ref{lemma-prep-dual}.
Let $b$ be $\beta$ viewed as a correspondence from $\Spec(k)$ to $Y$
of degree $r$. Then $f_*\beta$ viewed as a correspondence from
$\Spec(k)$ to $X$ is equal to $a^t \circ b$, see
Lemmas \ref{lemma-functor-and-cycles} and
\ref{lemma-composition-correspondences}.
The displayed equality above holds if we can show that
$$
h(X) = \mathbf{1} \otimes h(X)
\xrightarrow{b \otimes 1}
h(Y)(r) \otimes h(X)
\xrightarrow{1 \otimes a}
h(Y)(r) \otimes h(Y)
\xrightarrow{\eta_Y}
h(Y)(r)
\xrightarrow{[Y]}
\mathbf{1}(r - e)
$$
is equal to
$$
h(X) = \mathbf{1} \otimes h(X)
\xrightarrow{a^t \circ b \otimes 1}
h(X)(r + d - e) \otimes h(X)
\xrightarrow{\eta_X}
h(X)(r + d - e)
\xrightarrow{[X]}
\mathbf{1}(r - e)
$$
This follows immediately from Lemma \ref{lemma-prep-dual}.
Thus we have axiom (C)(b).

\medskip\noindent
To prove axiom (C)(c) we use the discussion in
Remark \ref{remark-replace-cup-product-classical}.
Hence it suffices to prove that $\gamma$ is compatible with
exterior products. Let $X$, $Y$ be smooth projective varieties and
let $\alpha$, $\beta$ be cycles on them. Denote
$a$, $b$ the corresponding correspondences from $\Spec(k)$ to
$X$, $Y$. Then $\alpha \times \beta$ corresponds to the
correspondence $a \otimes b$ from $\Spec(k)$ to $X \otimes Y = X \times Y$.
Hence the requirement follows from the fact that $G$ is
compatible with the tensor structures on both sides.

\medskip\noindent
Axiom (C)(d) follows because the cycle $[\Spec(k)]$
corresponds to the identity morphism on $h(\Spec(k))$.
This finishes the proof of the lemma.
\end{proof}

\begin{lemma}
\label{lemma-from-weil-to-functor-classical}
Let $k$ be an algebraically closed field. Let $F$ be a field of
characteristic $0$. Let $H^*$ be a classical Weil cohomology theory.
Then we can construct a $\mathbf{Q}$-linear functor
$$
G : M_k \longrightarrow \text{graded }F\text{-vector spaces}
$$
of symmetric monoidal categories such that $H^*(X) = G(h(X))$.
\end{lemma}

\begin{proof}
By Lemma \ref{lemma-characterize-motives} it suffices to construct a functor
$G$ on the category of smooth projective schemes over $k$
with morphisms given by correspondences of degree $0$ such that
the image of $G(c_2)$ on $G(\mathbf{P}^1)$ is an invertible graded
$F$-vector space.
Since every smooth projective scheme is canonically a disjoint
union of smooth projective varieties, it suffices to construct
$G$ on the category whose objects are smooth projective varieties
and whose morphisms are correspondences of degree $0$. (Some details
omitted.)

\medskip\noindent
Given a smooth projective variety $X$ we set $G(X) = H^*(X)$.

\medskip\noindent
Given a correspondence $c \in \text{Corr}^0(X, Y)$ between smooth
projective varieties we consider the map
$G(c) : G(X) = H^*(X) \to G(Y) = H^*(Y)$ given by the rule
$$
a \longmapsto
G(c)(a) = \text{pr}_{2, *}(\gamma(c) \cup \text{pr}_1^*a)
$$
It is clear that $G(c)$ is additive in $c$ and hence $\mathbf{Q}$-linear.
Compatibility of $\gamma$ with pullbacks, pushforwards, and
intersection products given by axioms (C)(a), (C)(b), and (C)(c)
shows that we have
$G(c' \circ c) = G(c') \circ G(c)$ if $c' \in \text{Corr}^0(Y, Z)$.
Namely, for $a \in H^*(X)$ we have
\begin{align*}
(G(c') \circ G(c))(a)
& =
\text{pr}^{23}_{3, *}(\gamma(c') \cup
\text{pr}^{23, *}_2(\text{pr}^{12}_{2, *}(\gamma(c) \cup
\text{pr}^{12, *}_1a))) \\
& =
\text{pr}^{23}_{3, *}(\gamma(c') \cup
\text{pr}^{123}_{23, *}(\text{pr}^{123, *}_{12}(\gamma(c) \cup
\text{pr}^{12, *}_1 a))) \\
& =
\text{pr}^{23}_{3, *}
\text{pr}^{123}_{23, *}(
\text{pr}^{123, *}_{23}\gamma(c') \cup
\text{pr}^{123, *}_{12}\gamma(c) \cup
\text{pr}^{123, *}_1 a) \\
& =
\text{pr}^{23}_{3, *}
\text{pr}^{123}_{23, *}(
\gamma(\text{pr}^{123, *}_{23}c') \cup
\gamma(\text{pr}^{123, *}_{12}c) \cup
\text{pr}^{123, *}_1 a) \\
& =
\text{pr}^{13}_{3, *}
\text{pr}^{123}_{13, *}(
\gamma(\text{pr}^{123, *}_{23}c' \cdot \text{pr}^{123, *}_{12}c) \cup
\text{pr}^{123, *}_1 a) \\
& =
\text{pr}^{13}_{3, *}(
\gamma(\text{pr}^{123}_{13, *}(
\text{pr}^{123, *}_{23}c' \cdot \text{pr}^{123, *}_{12}c)) \cup
\text{pr}^{13, *}_1 a) \\
& =
G(c' \circ c)(a)
\end{align*}
with obvious notation. The first equality follows from the definitions.
The second equality holds because
$\text{pr}^{23, *}_2 \circ \text{pr}^{12}_{2, *} =
\text{pr}^{123}_{23, *} \circ \text{pr}^{123, *}_{12}$
as follows immediately from the description of pushforward
along projections given in Lemma \ref{lemma-pr2star-classical}.
The third equality holds by Lemma \ref{lemma-pushforward-classical}
and the fact that $H^*$ is a functor.
The fourth equalith holds by axiom (C)(a) and the fact that
the gysin map agrees with flat pullback for flat morphisms
(Chow Homology, Lemma \ref{chow-lemma-lci-gysin-flat}).
The fifth equality uses axiom (C)(c) as well as
Lemma \ref{lemma-pushforward-classical} to see that
$\text{pr}^{23}_{3, *} \circ \text{pr}^{123}_{23, *} =
\text{pr}^{13}_{3, *} \circ \text{pr}^{123}_{13, *}$.
The sixth equality uses the projection formula from
Lemma \ref{lemma-pushforward-classical} as well as
axiom (C)(b) to see that $
\text{pr}^{123}_{13, *}
\gamma(\text{pr}^{123, *}_{23}c' \cdot \text{pr}^{123, *}_{12}c) =
\gamma(\text{pr}^{123}_{13, *}(
\text{pr}^{123, *}_{23}c' \cdot \text{pr}^{123, *}_{12}c))$.
Finally, the last equality is the definition.

\medskip\noindent
To finish the proof that $G$ is a functor,
we have to show identities are preserved. In other words, if
$1 = [\Delta] \in \text{Corr}^0(X, X)$ is the identity
in the category of correspondences (see
Lemma \ref{lemma-category-correspondences} and its proof),
then we have to show that $G([\Delta]) = \text{id}$.
This follows from the determination of
$\gamma([\Delta])$ in Lemma \ref{lemma-class-diagonal-classical}
and Lemma \ref{lemma-pr2star-classical}.
This finishes the construction of $G$ as a functor on
smooth projective varieties and correspondences of degree $0$.

\medskip\noindent
It follows from axioms (A)(c) and (A)(d) that
$G(\Spec(k)) = H^*(\Spec(k))$ is canonically isomorphic to $F$
as an $F$-algebra.
The K\"unneth axiom (B) shows our functor is compatible with tensor products.
Thus our functor is a functor of symmetric monoidal categories.

\medskip\noindent
We still have to check that the image of $G(c_2)$ on $G(\mathbf{P}^1)$
is an invertible graded $F$-vector space (in particular we don't know yet
that $G$ extends to $M_k$).
By axiom (A)(d) the map $\int_{\mathbf{P}^1} : H^2(\mathbf{P}^1) \to F$
is an isomorphism. By axiom (A)(b) we see that $\dim_F H^0(\mathbf{P}^1) = 1$.
By Lemma \ref{lemma-square-diagonal-classical} and axiom (A)(c)
we obtain $2 - \dim_F H^1(\mathbf{P}^1) = c_1(T_{\mathbf{P}^1}) = 2$.
Hence $H^1(\mathbf{P}^1) = 0$. Thus
$$
G(\mathbf{P}^1) = H^0(\mathbf{P}^1) \oplus H^2(\mathbf{P}^1)
$$
Recall that $1 = c_0 + c_2$ is a decomposition of the identity
into a sum of orthogonal idempotents in
$\text{Corr}^0(\mathbf{P}^1, \mathbf{P}^1)$, see
Example \ref{example-decompose-P1}. We have $c_0 = a \circ b$ where
$a \in \text{Corr}^0(\Spec(k), \mathbf{P}^1)$ and
$b \in \text{Corr}^0(\mathbf{P}^1, \Spec(k))$ and where
$b \circ a = 1$ in $\text{Corr}^0(\Spec(k), \Spec(k))$, see proof of
Lemma \ref{lemma-inverse-h2}. Since $F = G(\Spec(k))$, it follows from
functoriality that $G(c_0)$ is the projector onto the summand
$H^0(\mathbf{P}^1) \subset G(\mathbf{P}^1)$. Hence
$G(c_2)$ must necessarily be the projection onto $H^2(\mathbf{P}^1)$
and the proof is complete.
\end{proof}

\begin{proposition}
\label{proposition-weil-cohomology-theory-classical}
Let $k$ be an algebraically closed field. Let $F$ be a field of
characteristic $0$. A classical Weil cohomology theory is the same thing
as a $\mathbf{Q}$-linear functor
$$
G : M_k \longrightarrow \text{graded }F\text{-vector spaces}
$$
of symmetric monoidal categories together with an isomorphism
$F[2] \to G(\mathbf{1}(1))$ of graded $F$-vector spaces such that
in addition
\begin{enumerate}
\item $G(h(X))$ lives in nonnegative degrees, and
\item $\dim_F G^0(h(X)) = 1$
\end{enumerate}
for any smooth projective variety $X$.
\end{proposition}

\begin{proof}
Given $G$ and $F[2] \to G(\mathbf{1}(1))$ by setting $H^*(X) = G(h(X))$
we obtain data (D1), (D2), and (D3) satisfying all of (A), (B), and (C)
except for possibly (A)(c) and (A)(d), see
Lemma \ref{lemma-from-functor-to-weil-classical} and its proof.
Observe that assumptions (1) and (2) imply axioms (A)(c) and (A)(d)
in the presence of the known axioms (A)(a) and (A)(b).

\medskip\noindent
Conversely, given $H^*$ we get a functor $G$ by the construction of
Lemma \ref{lemma-from-weil-to-functor-classical}.
Let $X = \mathbf{P}^1, c_0, c_2$ be as in Example \ref{example-decompose-P1}.
We have constructed an isomorphism $1(-1) \to (X, c_2, 0)$ of motives in
Lemma \ref{lemma-inverse-h2}. In the proof of
Lemma \ref{lemma-from-weil-to-functor-classical} we have seen that
$G(1(-1)) = G(X, c_2, 0) = H^2(\mathbf{P}^1)[-2]$.
Hence the isomorphism $\int_{\mathbf{P}^1} : H^2(\mathbf{P}^1) \to F$
of axiom (A)(d) gives an isomorphism $G(1(-1)) \to F[-2]$ which
determines an isomorphism $F[2] \to G(\mathbf{1}(1))$.
Finally, since $G(h(X)) = H^*(X)$ assumptions (1) and (2)
follow from axiom (A).
\end{proof}











\section{Cycles over non-closed fields}
\label{section-cycles-nonclosed}

\noindent
Some lemmas which will help us in our study of motives
over base fields which are not algebraically closed.

\begin{lemma}
\label{lemma-generated-by-separable}
Let $k$ be a field. Let $X$ be a smooth projective scheme over $k$.
Then $\CH_0(X)$ is generated by classes of closed points whose residue
fields are separable over $k$.
\end{lemma}

\begin{proof}
The lemma is immediate if $k$ has characteristic $0$ or is perfect.
Thus we may assume $k$ is an infinite field of characteristic $p > 0$.

\medskip\noindent
We may assume $X$ is irreducible of dimension $d$.
Then $k' = H^0(X, \mathcal{O}_X)$ is a finite separable field
extension of $k$ and that $X$ is geometrically integral over $k'$.
See Varieties, Lemmas \ref{varieties-lemma-smooth-geometrically-normal},
\ref{varieties-lemma-proper-geometrically-reduced-global-sections}, and
\ref{varieties-lemma-baby-stein}. We may and do replace $k$ by $k'$
and assume that $X$ is geometrically integral.

\medskip\noindent
Let $x \in X$ be a closed point. To prove the lemma we are going to show that
$[x] \in \CH_0(X)$ is rationally equivalent to an integer linear
combination of classes of closed points whose residue fields
are separable over $k$. Choose an ample invertible
$\mathcal{O}_X$-module $\mathcal{L}$. Set
$$
V = \{s \in H^0(X, \mathcal{L}) \mid s(x) = 0 \}
$$
After replacing $\mathcal{L}$ by a power we may assume
(a) $\mathcal{L}$ is very ample, (b) $V$ generates
$\mathcal{L}$ over $X \setminus x$, (c) the morphism
$X \setminus x \to \mathbf{P}(V)$ is an immersion, (d)
the map $V \to \mathfrak m_x\mathcal{L}_x/\mathfrak m_x^2\mathcal{L}_x$
is surjective, see
Morphisms, Lemma \ref{morphisms-lemma-finite-type-ample-very-ample},
Varieties, Lemma \ref{varieties-lemma-generate-over-complement}, and
Properties, Proposition \ref{properties-proposition-characterize-ample}.
Consider the set
$$
V^d \supset U =
\{
(s_1, \ldots, s_d) \in V^d \mid s_1, \ldots, s_d
\text{ generate }
\mathfrak m_x\mathcal{L}_x/\mathfrak m_x^2\mathcal{L}_x
\text{ over }\kappa(x)
\}
$$
Since $\mathcal{O}_{X, x}$ is a regular local ring of dimension $d$
we have $\dim_{\kappa(x)}(\mathfrak m_x/\mathfrak m_x^2) = d$
and hence we see that $U$ is a nonempty (Zariski) open of $V^d$.
For $(s_1, \ldots, s_d) \in U$ set $H_i = Z(s_i)$. Since
$s_1, \ldots, s_d$ generate $\mathfrak m_x\mathcal{L}_x$
we see that
$$
H_1 \cap \ldots \cap H_d = x \amalg Z
$$
scheme theoretically for some closed subscheme $Z \subset X$.
By Bertini (in the form of Varieties, Lemma \ref{varieties-lemma-bertini})
for a general element $s_1 \in V$ the scheme $H_1 \cap (X \setminus x)$
is smooth over $k$ of dimension $d - 1$.
Having chosen $s_1$, for a general element
$s_2 \in V$ the scheme $H_1 \cap H_2 \cap (X \setminus x)$
is smooth over $k$ of dimension $d - 2$. And so on.
We conclude that for sufficiently general
$(s_1, \ldots, s_d) \in U$ the scheme $Z$ is \'etale over $\Spec(k)$.
In particular $H_1 \cap \ldots \cap H_d$ has dimension $0$
and hence
$$
[H_1] \cdot \ldots \cdot [H_d] = [x] + [Z]
$$
in $\CH_0(X)$ by repeated application of
Chow Homology, Lemma \ref{chow-lemma-intersect-properly} (details omitted).
This finishes the proof as it shows that $[x] \sim_{rat} - [Z] + [Z']$
where $Z' = H'_1 \cap \ldots \cap H'_d$ is a general complete
intersection of vanishing loci of sufficiently general sections
of $\mathcal{L}$ which will be \'etale over $k$ by the same argument
as before.
\end{proof}

\begin{lemma}
\label{lemma-chow-limit}
Let $K/k$ be an algebraic field extension. Let $X$ be a finite type
scheme over $k$. Then $\CH_i(X_K) = \colim \CH_i(X_{k'})$ where the
colimit is over the subextensions $K/k'/k$ with $k'/k$ finite.
\end{lemma}

\begin{proof}
This is a special case of
Chow Homology, Lemma \ref{chow-lemma-chow-limit}.
\end{proof}

\begin{lemma}
\label{lemma-divide-difference-points}
Let $k$ be a field. Let $X$ be a geometrically irreducible
smooth projective scheme over $k$. Let $x, x' \in X$ be $k$-rational points.
Let $n$ be an integer invertible in $k$.
Then there exists a finite separable extension $k'/k$ such that
the pullback of $[x] - [x']$ to $X_{k'}$
is divisible by $n$ in $\CH_0(X_{k'})$.
\end{lemma}

\begin{proof}
Let $k'$ be a separable algebraic closure of $k$. Suppose that we can show
the the pullback of $[x] - [x']$ to $X_{k'}$ is divisible by $n$ in
$\CH_0(X_{k'})$. Then we conclude by Lemma \ref{lemma-chow-limit}.
Thus we may and do assume $k$ is separably algebraically closed.

\medskip\noindent
Suppose $\dim(X) > 1$. Let $\mathcal{L}$ be an ample invertible sheaf on $X$.
Set
$$
V = \{s \in H^0(X, \mathcal{L}) \mid s(x) = 0\text{ and }s(x') = 0 \}
$$
After replacing $\mathcal{L}$ by a power we see that for
a general $v \in V$ the corresponding divisor $H_v \subset X$ is smooth
away from $x$ and $x'$, see
Varieties, Lemmas \ref{varieties-lemma-generate-over-complement} and
\ref{varieties-lemma-bertini}. To find $v$ we use that $k$ is infinite (being
separably algebraically closed).
If we choose $s$ general, then the image of $s$ in
$\mathfrak m_x\mathcal{L}_x/\mathfrak m_x^2\mathcal{L}_x$
will be nonzero, which implies that $H_v$ is smooth at $x$
(details omitted). Similarly for $x'$. Thus $H_v$ is smooth.
By Varieties, Lemma \ref{varieties-lemma-connectedness-ample-divisor}
(applied to the base change of everything
to the algebraic closure of $k$)
we see that $H_v$ is geometrically connected.
It suffices to prove the result for
$[x] - [x']$ seen as an element of $\CH_0(H_v)$.
In this way we reduce to the case of a curve.

\medskip\noindent
Assume $X$ is a curve. Then we see that $\mathcal{O}_X(x - x')$
defines a $k$-rational point $g$ of $J = \underline{\Pic}^0_{X/k}$, see
Picard Schemes of Curves, Lemma \ref{pic-lemma-picard-pieces}.
Recall that $J$ is a proper smooth variety over $k$
which is also a group scheme over $k$ (same reference).
Hence $J$ is geometrically integral
(see Varieties, Lemma \ref{varieties-lemma-geometrically-connected-criterion}
and \ref{varieties-lemma-smooth-geometrically-normal}).
In other words, $J$ is an abelian variety, see
Groupoids, Definition \ref{groupoids-definition-abelian-variety}.
Thus $[n] : J \to J$ is finite \'etale by
Groupoids, Proposition \ref{groupoids-proposition-review-abelian-varieties}
(this is where we use $n$ is invertible in $k$).
Since $k$ is separably closed we conclude that $g = [n](g')$
for some $g' \in J(k)$. If $\mathcal{L}$ is the degree $0$
invertible module on $X$ corresponding to $g'$, then we conclude
that $\mathcal{O}_X(x - x') \cong \mathcal{L}^{\otimes n}$ as desired.
\end{proof}

\begin{lemma}
\label{lemma-kernel-to-closure}
Let $K/k$ be an algebraic extension of fields.
Let $X$ be a finite type scheme over $k$.
The kernel of the map $\CH_i(X) \to \CH_i(X_K)$
constructed in Lemma \ref{lemma-chow-limit}
is torsion.
\end{lemma}

\begin{proof}
It clearly suffices to show that the kernel
of flat pullback $\CH_i(X) \to \CH_i(X_{k'})$
by $\pi : X_{k'} \to X$ is torsion
for any finite extension $k'/k$. This is clear because
$\pi_* \pi^* \alpha = [k' : k] \alpha$ by
Chow Homology, Lemma \ref{chow-lemma-finite-flat}.
\end{proof}

\begin{lemma}[Voevodsky]
\label{lemma-smash-nilpotence}
\begin{reference}
\cite{nilpotence}
\end{reference}
Let $k$ be a field. Let $X$ be a geometrically irreducible
smooth projective scheme over $k$. Let $x, x' \in X$ be $k$-rational points.
For $n$ large enough the class of the zero cycle
$$
([x] - [x']) \times \ldots \times ([x] - [x']) \in
\CH_0(X^n)
$$
is torsion.
\end{lemma}

\begin{proof}
If we can show this after base change to the algebraic closure of $k$,
then the result follows over $k$ because the kernel of pullback
is torsion by Lemma \ref{lemma-kernel-to-closure}.
Hence we may and do assume $k$ is algebraically closed.

\medskip\noindent
Using Bertini we can choose a smooth curve $C \subset X$ passing through
$x$ and $x'$. See proof of Lemma \ref{lemma-divide-difference-points}.
Hence we may assume $X$ is a curve.

\medskip\noindent
Assume $X$ is a curve and $k$ is algebraically closed.
Write $S^n(X) = \underline{\Hilbfunctor}^n_{X/k}$ with notation as in
Picard Schemes of Curves, Sections \ref{pic-section-hilbert-scheme-points}
and \ref{pic-section-divisors}. There is a canonical morphism
$$
\pi : X^n \longrightarrow S^n(X)
$$
which sends the $k$-rational point $(x_1, \ldots, x_n)$ to the $k$-rational
point corresponding to the divisor $[x_1] + \ldots + [x_n]$ on $X$.
There is a faithful action of the symmetric group $S_n$ on $X^n$.
The morphism $\pi$ is $S_n$-invariant and the fibres of $\pi$ are
$S_n$-orbits (set theoretically). Finally, $\pi$ is finite flat of
degree $n!$, see Picard Schemes of Curves, Lemma
\ref{pic-lemma-universal-object}.

\medskip\noindent
Let $\alpha_n$ be the zero cycle on $X^n$ given by the formula in the
statement of the lemma. Let $\mathcal{L} = \mathcal{O}_X(x - x')$. Then
$c_1(\mathcal{L}) \cap [X] = [x] - [x']$. Thus
$$
\alpha_n = c_1(\mathcal{L}_1) \cap \ldots \cap c_1(\mathcal{L}_n) \cap [X^n]
$$
where $\mathcal{L}_i = \text{pr}_i^*\mathcal{L}$ and $\text{pr}_i : X^n \to X$
is the $i$th projection. By either
Divisors, Lemma \ref{divisors-lemma-finite-locally-free-has-norm} or
Divisors, Lemma \ref{divisors-lemma-norm-in-normal-case}
there is a norm for $\pi$. Set
$\mathcal{N} = \text{Norm}_\pi(\mathcal{L}_1)$, 
see Divisors, Lemma \ref{divisors-lemma-norm-invertible}. We have
$$
\pi^*\mathcal{N} =
(\mathcal{L}_1 \otimes \ldots \otimes \mathcal{L}_n)^{\otimes (n - 1)!}
$$
in $\Pic(X^n)$ by a calculation. Deails omitted; hint: this follows from
the fact that
$\text{Norm}_\pi : \pi_*\mathcal{O}_{X^n} \to \mathcal{O}_{S^n(X)}$
composed with the natural map $\pi_*\mathcal{O}_{S^n(X)} \to \mathcal{O}_{X^n}$
is equal to the product over all $\sigma \in S_n$ of the action of
$\sigma$ on $\pi_*\mathcal{O}_{X^n}$. Consider
$$
\beta_n = c_1(\mathcal{N})^n \cap [S^n(X)]
$$
in $\CH_0(S^n(X))$. Observe that
$c_1(\mathcal{L}_i) \cap c_1(\mathcal{L}_i) = 0$
because $\mathcal{L}_i$ is pulled back from a curve, see
Chow Homology, Lemma \ref{chow-lemma-vanish-above-dimension}. Thus we see that
\begin{align*}
\pi^*\beta_n
& =
((n - 1)!)^n
(\sum\nolimits_{i = 1, \ldots, n} c_1(\mathcal{L}_i))^n \cap [X^n] \\
& =
((n - 1)!)^n n^n 
c_1(\mathcal{L}_1) \cap \ldots \cap c_1(\mathcal{L}_n) \cap [X^n] \\
& =
(n!)^n \alpha_n
\end{align*}
Thus it suffices to show that $\beta_n$ is torsion.

\medskip\noindent
There is a canonical morphism
$$
f : S^n(X) \longrightarrow \underline{\Picardfunctor}^n_{X/k}
$$
See Picard Schemes of Curves, Lemma \ref{pic-lemma-picard-pieces}.
For $n \geq 2g - 1$ this morphism is a projective space bundle
(details omitted; compare with the
proof of Picard Schemes of Curves, Lemma \ref{pic-lemma-picard-pieces}).
The invertible sheaf $\mathcal{N}$ is trivial on the
fibres of $f$, see below. Thus by the projective space bundle formula
(Chow Homology, Lemma \ref{chow-lemma-chow-ring-projective-bundle})
we see that $\mathcal{N} = f^*\mathcal{M}$ for some invertible
module $\mathcal{M}$ on $\underline{\Picardfunctor}^n_{X/k}$.
Of course, then we see that
$$
c_1(\mathcal{N})^n = f^*(c_1(\mathcal{M})^n)
$$
is zero because $n > g = \dim(\underline{\Picardfunctor}^n_{X/k})$
and we can use Chow Homology, Lemma \ref{chow-lemma-vanish-above-dimension}
as before.

\medskip\noindent
We still have to show that $\mathcal{N}$ is trivial on a fibre $F$
of $f$. Since the fibres of $f$ are projective spaces and since
$\Pic(\mathbf{P}^m_k) = \mathbf{Z}$
(Divisors, Lemma \ref{divisors-lemma-Pic-projective-space-UFD}),
this can be shown by computing the degree of $\mathcal{N}$
on a line contained in the fibre. Instead we will prove it by
proving that $\mathcal{N}$ is algebraically
equivalent to zero. First we claim there is a connected finite type scheme $T$
over $k$, an invertible module $\mathcal{L}'$ on $T \times X$ and
$k$-rational points $p, q \in T$ such that
$\mathcal{M}_p \cong \mathcal{O}_X$ and $\mathcal{M}_q = \mathcal{L}$.
Namely, since $\mathcal{L} = \mathcal{O}_X(x - x')$ we can take
$T = X$, $p = x'$, $q = x$, and
$\mathcal{L}' = \mathcal{O}_{X \times X}(\Delta)
\otimes \text{pr}_2^*\mathcal{O}_X(-x')$.
Then we let $\mathcal{L}'_i$ on
$T \times X^n$ for $i = 1, \ldots, n$
be the pullback of $\mathcal{L}'$ by
$\text{id}_T \times \text{pr}_i : T \times X^n \to T \times X$.
Finally, we let
$\mathcal{N}' = \text{Norm}_{\text{id}_T \times \pi}(\mathcal{L}'_1)$
on $T \times S^n(X)$.
By construction we have $\mathcal{N}'_p = \mathcal{O}_{S^n(X)}$
and $\mathcal{N}'_q = \mathcal{N}$.
We conclude that
$$
\mathcal{N}'|_{T \times F}
$$
is an invertible module on $T \times F \cong T \times \mathbf{P}^m_k$
whose fibre over $p$ is the trivial invertible module and whose fibre
over $q$ is $\mathcal{N}|_F$. Since the euler characteristic
of the trivial bundle is $1$ and since this euler characteristic
is locally constant in families (Derived Categories of Schemes,
Lemma \ref{perfect-lemma-chi-locally-constant-geometric})
we conclude $\chi(F, \mathcal{N}^{\otimes s}|_F) = 1$
for all $s \in \mathbf{Z}$. This can happen only if
$\mathcal{N}|_F \cong \mathcal{O}_F$ (see
Cohomology of Schemes, Lemma
\ref{coherent-lemma-cohomology-projective-space-over-ring})
and the proof is complete. Some details omitted.
\end{proof}















\section{Weil cohomology theories, I}
\label{section-axioms}

\noindent
This section is the analogue of Section \ref{section-axioms-classical}
over arbitrary fields. In other words, we work out what data and
axioms correspond to functors $G$ of symmetric monoidal categories from
the category of motives to the category of graded vector spaces such that
$G(\mathbf{1}(1))$ sits in degree $-2$. In Section \ref{section-old}
we will define a Weil cohomology theory by adding a single suplementary
condition.

\medskip\noindent
We fix a field $k$ (the base field).
We fix a field $F$ of characteristic $0$ (the coefficient field).
The data is given by:
\begin{enumerate}
\item[(D0)] A $1$-dimensional $F$-vector space $F(1)$.
\item[(D1)] A contravariant functor $H^*$ from the category
of smooth projective schemes over $k$ to the category of
graded commutative $F$-algebras.
\item[(D2)] For every smooth projective scheme $X$ over $k$
a group homomorphism $\gamma : \CH^i(X) \to H^{2i}(X)(i)$.
\item[(D3)] For every nonempty smooth projective scheme $X$ over $k$
which is equidimensional of dimension $d$ a map
$\int_X : H^{2d}(X)(d) \to F$.
\end{enumerate}
We make some remarks to explain what this means and to introduce
some terminology associated with this.

\medskip\noindent
Remarks on (D0).
The vector space $F(1)$ gives rise to {\it Tate twists} on the category of
$F$-vector spaces. Namely, for $n \in \mathbf{Z}$ we set
$F(n) = F(1)^{\otimes n}$ if $n \geq 0$, we set $F(-1) = \Hom_F(F(1), F)$,
and we set $F(n) = F(-1)^{\otimes - n}$ if $n < 0$. Please compare
with More on Algebra, Section \ref{more-algebra-section-picard}.
For an $F$-vector space $V$ we define the {\it $n$th Tate twist}
$$
V(n) = V \otimes_F F(n)
$$
We will use obvious notation, e.g., given $F$-vector spaces $U$, $V$
and $W$ and a linear map $U \otimes_F V \to W$ we obtain a linear
map $U(n) \otimes_F V(m) \to W(n + m)$ for $n, m \in \mathbf{Z}$.

\medskip\noindent
Remarks on (D1).
Given a smooth projective scheme $X$ over $k$ we say that $H^*(X)$
is the {\it cohomology} of $X$. Given a morphism $f : X \to Y$
of smooth projective schemes over $k$ we denote $f^* : H^*(Y) \to H^*(X)$
the map $H^*(f)$ and we call it the {\it pullback map}.

\medskip\noindent
Remarks on (D2). The map $\gamma$ is called the {\it cycle class map}.
We say that $\gamma(\alpha)$ is the {\it cohomology class} of $\alpha$.
If $Z \subset Y \subset X$ are closed subschemes with $Y$ and $X$
smooth projective over $k$ and $Z$ integral, then $[Z]$ could
mean the class of the cycle $[Z]$ in $\CH^*(Y)$ or in $\CH^*(X)$.
In this case the notation $\gamma([Z])$ is ambiguous and the intended meaning
has to be deduced from context.

\medskip\noindent
Remarks on (D3). The map $\int_X$ is sometimes called the
{\it trace map} and is sometimes denoted $\text{Tr}_X$.

\medskip\noindent
The first axiom is often called {\it Poincar\'e duality}
\begin{enumerate}
\item[(A)] Let $X$ be a nonempty smooth projective scheme over $k$
which is equidimensional of dimension $d$. Then
\begin{enumerate}
\item $\dim_F H^i(X) < \infty$ for all $i$,
\item $H^i(X) \times H^{2d - i}(X)(d) \rightarrow
H^{2d}(X)(d) \rightarrow F$
is a perfect pairing for all $i$ where the final
map is the trace map $\int_X$.
\end{enumerate}
\end{enumerate}
Let $f : X \to Y$ be a morphism of nonempty smooth projective schemes with $X$
equidimensional of dimension $d$ and $Y$ is equidimensional of dimension $e$.
Using Poincar\'e duality we can define a {\it pushforward}
$$
f_* : H^{2d - i}(X)(d) \longrightarrow H^{2e - i}(Y)(e)
$$
as the contragredient of the linear map $f^* : H^i(Y) \to H^i(X)$. In a
formula, for $a \in H^{2d - i}(X)(d)$, the element $f_*a \in H^{2e - i}(Y)(e)$
is characterized by
$$
\int_X f^*b \cup a = \int_Y b \cup f_*a
$$
for all $b \in H^i(Y)$.

\begin{lemma}
\label{lemma-pushforward}
Assume given (D0), (D1), and (D3) satisfying (A). For $f : X \to Y$
a morphism of nonempty equidimensional smooth projective schemes over $k$
we have $f_*(f^*b \cup a) = b \cup f_*a$. If $g : Y \to Z$ is a second morphism
with $Z$ nonempty smooth projective and equidimensional, then
$g_* \circ f_* = (g \circ f)_*$.
\end{lemma}

\begin{proof}
The first equality holds because
$$
\int_Y c \cup b \cup f_*a =
\int_X f^*c \cup f^*b \cup a =
\int_Y c \cup f_*(f^*b \cup a).
$$
The second equality holds because
$$
\int_Z c \cup (g \circ f)_*a = \int_X (g \circ f)^*c \cup a =
\int_X f^* g^* c \cup a = \int_Y g^*c \cup f_*a = \int_Z c \cup g_*f_*a
$$
This ends the proof.
\end{proof}

\noindent
The second axiom says that $H^*$ respects the monoidal structure
given by products via the {\it K\"unneth formula}
\begin{enumerate}
\item[(B)] Let $X$ and $Y$ be smooth projective schemes over $k$.
\begin{enumerate}
\item $H^*(X) \otimes_F H^*(Y) \to H^*(X \times Y)$,
$\alpha \otimes \beta \mapsto \text{pr}_1^*\alpha \cup \text{pr}_2^*\beta$
is an isomorphism,
\item if $X$ and $Y$ are nonempty and equidimensional, then
$\int_{X \times Y} = \int_X \otimes \int_Y$ via (a).
\end{enumerate}
\end{enumerate}
Using axiom (B)(b) we can compute pushforwards along projections.

\begin{lemma}
\label{lemma-pr2star}
Assume given (D0), (D1), and (D3) satisfying (A) and (B).
Let $X$ and $Y$ be nonempty smooth projective schemes over $k$
equidimensional of dimensions $d$ and $e$. Then
$\text{pr}_{2, *} : H^*(X \times Y)(d + e) \to H^*(Y)(e)$ sends
$a \otimes b$ to $(\int_X a) b$.
\end{lemma}

\begin{proof}
This follows from axioms (B)(a) and (B)(b).
\end{proof}

\noindent
The third axiom concerns the cycle class maps
\begin{enumerate}
\item[(C)] The cycle class maps satisfy the following rules
\begin{enumerate}
\item for a morphism $f : X \to Y$ of smooth projective schemes over
$k$ we have $\gamma(f^!\beta) = f^*\gamma(\beta)$ for $\beta \in \CH^*(Y)$,
\item for a morphism $f : X \to Y$ of nonempty
equidimensional smooth projective schemes over $k$ we have
$\gamma(f_*\alpha) = f_*\gamma(\alpha)$ for $\alpha \in \CH^*(X)$,
\item for any smooth projective scheme $X$ over $k$ we have
$\gamma(\alpha \cdot \beta) = \gamma(\alpha) \cup \gamma(\beta)$
for $\alpha, \beta \in \CH^*(X)$, and
\item $\int_{\Spec(k)} \gamma([\Spec(k)]) = 1$.
\end{enumerate}
\end{enumerate}
Let us elucidate axiom (C)(b). Namely, say $f : X \to Y$ is
as in (C)(b) with $\dim(X) = d$ and $\dim(Y) = e$. Then we
see that pushforward on Chow groups gives
$$
f_* : \CH^{d - i}(X) = \CH_i(X) \to \CH_i(Y) = \CH^{e - i}(Y)
$$
Say $\alpha \in \CH^{d - i}(X)$. On the one hand, we have
$f_*\alpha \in \CH^{e - i}(Y)$ and hence
$\gamma(f_*\alpha) \in H^{2e - 2i}(Y)(e - i)$.
On the other hand, we have
$\gamma(\alpha) \in H^{2d - 2i}(X)(d - i)$ and hence
$f_*\gamma(\alpha) \in H^{2e - 2i}(Y)(e - i)$ as well.
Thus the condition $\gamma(f_*\alpha) = f_*\gamma(\alpha)$ makes sense.

\begin{remark}
\label{remark-replace-cup-product}
Assume given (D0), (D1), (D2), and (D3) satisfying (A), (B), and (C)(a).
Let $X$ be a smooth projective scheme over $k$. We obtain maps
$$
H^*(X) \otimes_F H^*(X) \longrightarrow H^*(X \times X)
\xrightarrow{\Delta^*} H^*(X)
$$
where the first arrow is as in axiom (B) and $\Delta^*$
is pullback along the diagonal morphism $\Delta : X \to X \times X$.
The composition is the cup product as pullback is an algebra homomorphism and
$\text{pr}_i \circ \Delta = \text{id}$.
On the other hand, given cycles $\alpha, \beta$ on $X$ 
the intersection product is defined by the formula
$$
\alpha \cdot \beta =
\Delta^!(\alpha \times \beta)
$$
In other words, $\alpha \cdot \beta$ is the pullback of the
exterior product $\alpha \times \beta$ on $X \times X$ by
the diagonal. Note also that
$\alpha \times \beta = \text{pr}_1^*\alpha \cdot \text{pr}_2^*\beta$
in $\CH^*(X \times X)$ (we omit the proof). Hence, given axiom (C)(a),
axiom (C)(c) is equivalent to the statement that $\gamma$ is
compatible with exterior product in the sense that
$\gamma(\alpha \times \beta)$ is equal to
$\text{pr}_1^*\gamma(\alpha) \cup \text{pr}_2^*\gamma(\beta)$.
\end{remark}

\begin{lemma}
\label{lemma-base}
Assume given (D0), (D1), (D2), and (D3) satisfying (A), (B), and (C).
Then $H^i(\Spec(k)) = 0$ for $i \not = 0$ and there is a
unique $F$-algebra isomorphism $F = H^0(\Spec(k))$.
We have $\gamma([\Spec(k)]) = 1$ and $\int_{\Spec(k)} 1 = 1$.
\end{lemma}

\begin{proof}
By axiom (C)(d) we see that $H^0(\Spec(k))$ is nonzero and even
$\gamma([\Spec(k)])$ is nonzero.
Since $\Spec(k) \times \Spec(k) = \Spec(k)$ we get
$$
H^*(\Spec(k)) \otimes_F H^*(\Spec(k)) = H^*(\Spec(k))
$$
by axiom (B)(a) which implies (look at dimensions) that only
$H^0$ is nonzero and moreover has dimension $1$. Thus
$F = H^0(\Spec(k))$ via the unique $F$-algebra isomorphism
given by mapping $1 \in F$ to $1 \in H^0(\Spec(k))$.
Since $[\Spec(k)] \cdot [\Spec(k)] = [\Spec(k)]$ in the
Chow ring of $\Spec(k)$ we conclude that
$\gamma([\Spec(k)) \cup \gamma([\Spec(k)]) = \gamma([\Spec(k)])$
by axiom (C)(c). Since we already know that $\gamma([\Spec(k)])$ is nonzero
we conclude that it has to be equal to $1$.
Finally, we have $\int_{\Spec(k)} 1 = 1$ since
$\int_{\Spec(k)} \gamma([\Spec(k)]) = 1$
by axiom (C)(d).
\end{proof}

\begin{lemma}
\label{lemma-unit}
Assume given (D0), (D1), (D2), and (D3) satisfying (A), (B), and (C).
Let $X$ be a smooth projective scheme over $k$.
If $X = \emptyset$, then $H^*(X) = 0$.
If $X$ is nonempty, then $\gamma([X]) = 1$ and $1 \not = 0$ in $H^0(X)$.
\end{lemma}

\begin{proof}
First assume $X$ is nonempty.
Observe that $[X]$ is the pullback of $[\Spec(k)]$ by the structure morphism
$p : X \to \Spec(k)$. Hence we get $\gamma([X]) = 1$ by axiom (C)(a)
and Lemma \ref{lemma-base}. Let $X' \subset X$ be an irreducible component.
By functoriality it suffices to show $1 \not = 0$ in $H^0(X')$.
Thus we may and do assume $X$ is irreducible, and in particular
nonempty and equidimensional, say of dimension $d$.
To see that $1 \not = 0$ it suffices to show that $H^*(X)$ is nonzero.

\medskip\noindent
Let $x \in X$ be a closed point whose residue field $k'$
is separable over $k$, see
Varieties, Lemma \ref{varieties-lemma-smooth-separable-closed-points-dense}.
Let $i : \Spec(k') \to X$ be the inclusion morphism.
Denote  $p : X \to \Spec(k)$ is the structure morphism.
Observe that
$p_*i_*[\Spec(k')] = [k' : k][\Spec(k)]$ in $\CH_0(\Spec(k))$.
Using axiom (C)(b) twice and Lemma \ref{lemma-base}
we conclude that
$$
p_*i_*\gamma([\Spec(k')]) = \gamma([k' : k][\Spec(k)]) = [k' : k]
\in F = H^0(\Spec(k))
$$
is nonzero. Thus $i_*\gamma([\Spec(k)]) \in H^{2d}(X)(d)$ is nonzero
(because it maps to something nonzero via $p_*$). This concludes the proof
in case $X$ is nonempty.

\medskip\noindent
Finally, we consider the case of the empty scheme. Axiom (B)(a) gives
$H^*(\emptyset) \otimes H^*(\emptyset) = H^*(\emptyset)$ and
we get that $H^*(\emptyset)$ is either zero or $1$-dimensional
in degree $0$. Then axiom (B)(a) again shows that
$H^*(\emptyset) \otimes H^*(X) = H^*(\emptyset)$ for
all smooth projective schemes $X$ over $k$. Using axiom (A)(b)
and the nonvanishing of $H^0(X)$ we've seen above
we find that $H^*(X)$ is nonzero in at least two degrees
if $\dim(X) > 0$. This then forces $H^*(\emptyset)$ to be zero.
\end{proof}

\begin{lemma}
\label{lemma-push-unit}
Assume given (D0), (D1), (D2), and (D3) satisfying (A), (B), and (C).
Let $i : X \to Y$ be a closed immersion of nonempty smooth projective
equidimensional schemes over $k$. Then
$\gamma([X]) = i_*1$ in $H^{2c}(Y)(c)$ where $c = \dim(Y) - \dim(X)$.
\end{lemma}

\begin{proof}
This is true because $1 = \gamma([X])$ in $H^0(X)$ by Lemma \ref{lemma-unit}
and then we can apply axiom (C)(b).
\end{proof}

\begin{lemma}
\label{lemma-class-diagonal}
Assume given (D0), (D1), (D2), and (D3) satisfying (A), (B), and (C).
Let $X$ be a nonempty smooth projective scheme over $k$ equidimensional
of dimension $d$. Choose a basis $e_{i, j}, j = 1, \ldots, \beta_i$ of
$H^i(X)$ over $F$. Using K\"unneth write
$$
\gamma([\Delta]) =
\sum\nolimits_i
\sum\nolimits_j e_{i, j} \otimes e'_{2d - i , j}
\quad\text{in}\quad
\bigoplus\nolimits_i H^i(X) \otimes_F H^{2d - i}(X)(d)
$$
with $e'_{2d - i, j} \in H^{2d - i}(X)(d)$.
Then $\int_X e_{i, j} \cup e'_{2d - i, j'} = (-1)^i\delta_{jj'}$.
\end{lemma}

\begin{proof}
Recall that $\Delta^* : H^*(X \times X) \to H^*(X)$ is equal to the
cup product map $H^*(X) \otimes_F H^*(X) \to H^*(X)$, see
Remark \ref{remark-replace-cup-product}. On the other
hand, recall that $\gamma([\Delta]) = \Delta_*1$ (Lemma \ref{lemma-push-unit})
and hence
$$
\int_{X \times X} \gamma([\Delta]) \cup a \otimes b =
\int_{X \times X} \Delta_*1 \cup a \otimes b =
\int_X a \cup b
$$
by Lemma \ref{lemma-pushforward}.
On the other hand, we have
$$
\int_{X \times X} (\sum e_{i, j} \otimes e'_{2d -i , j}) \cup a \otimes b =
\sum (\int_X a \cup e_{i, j})(\int_X e'_{2d - i, j} \cup b)
$$
by axiom (B)(b); note that we made two switches of order so that the sign
for each term is $1$. Thus if we choose $a$ such that
$\int_X a \cup e_{i, j} = 1$ and all other pairings equal to zero, then
we conclude that $\int_X e'_{2d - i, j} \cup b = \int_X a \cup b$
for all $b$, i.e., $e'_{2d - i, j} = a$. This proves the lemma.
\end{proof}

\begin{lemma}
\label{lemma-cohomology-P1}
Assume given (D0), (D1), (D2), and (D3) satisfying (A), (B), and (C).
Then $H^*(\mathbf{P}^1_k)$ is $1$-dimensional in dimensions $0$ and $2$
and zero in other degrees.
\end{lemma}

\begin{proof}
Let $x \in \mathbf{P}^1_k$ be a $k$-rational point. Observe that
$\Delta = \text{pr}_1^*x + \text{pr}_2^*x$ as divisors on
$\mathbf{P}^1_k \times \mathbf{P}^1_k$. Using axiom (C)(a)
and additivity of $\gamma$ we see that
$$
\gamma([\Delta]) =
\text{pr}_1^*\gamma([x]) +
\text{pr}_2^*\gamma([x]) =
\gamma([x]) \otimes 1 + 1 \otimes \gamma([x])
$$
in $H^*(\mathbf{P}^1_k \times \mathbf{P}^1_k) =
H^*(\mathbf{P}^1_k) \otimes_F H^*(\mathbf{P}^1_k)$.
However, by Lemma \ref{lemma-class-diagonal}
we know that $\gamma([\Delta])$ cannot be written
as a sum of fewer than $\sum \beta_i$ pure tensors
where $\beta_i = \dim_F H^i(\mathbf{P}^1_k)$.
Thus we see that $\sum \beta_i \leq 2$.
By Lemma \ref{lemma-unit} we have $H^0(\mathbf{P}^1_k) \not = 0$.
By Poincar\'e duality, more precisely axiom (A)(b),
we have $\beta_0 = \beta_2$. Therefore the lemma holds.
\end{proof}

\begin{lemma}
\label{lemma-weil-additive}
Assume given (D0), (D1), (D2), and (D3) satisfying (A), (B), and (C).
If $X$ and $Y$ are smooth projective schemes over $k$, then
$H^*(X \amalg Y) \to H^*(X) \times H^*(Y)$,
$a \mapsto (i^*a, j^*a)$ is an isomorphism where $i$, $j$
are the coprojections.
\end{lemma}

\begin{proof}
If $X$ or $Y$ is empty, then this is true because
$H^*(\emptyset) = 0$ by Lemma \ref{lemma-unit}.
Thus we may assume both $X$ and $Y$ are nonempty.

\medskip\noindent
We first show that the map is injective. First, observe that
we can find morphisms $X' \to X$ and $Y' \to Y$
of smooth projective schemes so that $X'$ and $Y'$ are
equidimensional of the same dimension and such that
$X' \to X$ and $Y' \to Y$ each have a section. Namely,
decompose $X = \coprod X_d$ and $Y = \coprod Y_e$
into open and closed subschemes equidimensional of
dimension $d$ and $e$. Then take
$X' = \coprod X_d \times \mathbf{P}^{n - d}$
and $Y' = \coprod Y_e \times \mathbf{P}^{n - e}$ for some
$n$ sufficiently large. Thus pullback by
$X' \amalg Y' \to X \amalg Y$ is injective
(because there is a section) and
it suffices to show the injectivity for $X', Y'$
as we do in the next parapgrah.

\medskip\noindent
Let us show the map is injective when $X$ and $Y$ are equidimensional
of the same dimension $d$.
Observe that $[X \amalg Y] = [X] + [Y]$ in $\CH^0(X \amalg Y)$
and that $[X]$ and $[Y]$ are orthogonal idempotents in $\CH^0(X \amalg Y)$.
Thus
$$
1 = \gamma([X \amalg Y] = \gamma([X]) + \gamma([Y]) = i_*1 + j_*1
$$
is a decomposition into orthogonal idempotents. Here we have used
Lemmas \ref{lemma-unit} and \ref{lemma-push-unit} and axiom (C)(c).
Then we see that
$$
a = a \cup 1 = a \cup i_*1 + a \cup j_*1 =
i_*(i^*a) + j_*(j^*a)
$$
by the projection formula (Lemma \ref{lemma-pushforward}) and hence the map
is injective.

\medskip\noindent
We show the map is surjective. Write $e = \gamma([X])$ and $f = \gamma([Y])$
viewed as elements in $H^0(X \amalg Y)$. We have
$i^*e = 1$, $i^*f = 0$, $j^*e = 0$, and $j^*f = 1$ by axiom (C)(a).
Hence if $i^* : H^*(X \amalg Y) \to  H^*(X)$
and $j^* : H^*(X \amalg Y) \to H^*(Y)$ are surjective, then
so is $(i^*, j^*)$. Namely, for $a, a' \in H^*(X \amalg Y)$
we have
$$
(i^*a, j^*a') = (i^*(a \cup e + a' \cup f), j^*(a \cup e + a' \cup f))
$$
By symmetry it suffices to show $i^* : H^*(X \amalg Y) \to  H^*(X)$
is surjective. If there is a morphism $Y \to X$, then there is a morphism
$g : X \amalg Y \to X$ with $g \circ i = \text{id}_X$ and we conclude.
To finish the proof, observe that in order to prove
$i^*$ is surjective, it suffices to do so after tensoring
by a nonzero graded $F$-vector space. Hence by axiom (B)(b)
and nonvanishing of cohomology (Lemma \ref{lemma-unit})
it suffices to prove $i^*$ is surjective after replacing
$X$ and $Y$ by $X \times \Spec(k')$ and $Y \times \Spec(k')$
for some finite separable extension $k'/k$.
If we choose $k'$ such that there exists a closed point
$x \in X$ with $\kappa(x) = k'$ (and this is possible by
Varieties, Lemma \ref{varieties-lemma-smooth-separable-closed-points-dense})
then there is a morphism $Y \times \Spec(k') \to X \times \Spec(k')$
and we find that the proof is complete.
\end{proof}

\begin{lemma}
\label{lemma-from-functor-to-weil}
Let $k$ be a field. Let $F$ be a field of characteristic $0$.
Assume given a $\mathbf{Q}$-linear functor
$$
G : M_k \longrightarrow \text{graded }F\text{-vector spaces}
$$
of symmetric monoidal categories such that $G(\mathbf{1}(1))$
is nonzero only in degree $-2$. Then we obtain data (D0), (D1), (D2), and (D3)
satisfying all of (A), (B), and (C) above.
\end{lemma}

\begin{proof}
This proof is the same as the proof of
Lemma \ref{lemma-from-functor-to-weil-classical};
we urge the reader to read the proof of that lemma instead.

\medskip\noindent
We obtain a contravariant functor from the category of smooth
projective schemes over $k$ to the category of graded $F$-vector spaces
by setting $H^*(X) = G(h(X))$. By assumption we have a canonical
isomorphism
$$
H^*(X \times Y) = G(h(X \times Y)) = G(h(X) \otimes h(Y)) =
G(h(X)) \otimes G(h(Y)) = H^*(X) \otimes H^*(Y)
$$
compatible with pullbacks. Using pullback along the diagonal
$\Delta : X \to X \times X$ we obtain a canonical map
$$
H^*(X) \otimes H^*(X) = H^*(X \times X) \to H^*(X)
$$
of graded vector spaces compatible with pullbacks.
This defines a functorial graded $F$-algebra structure on
$H^*(X)$. Since $\Delta$ commutes with the commutativity
constraint $h(X) \otimes h(X) \to h(X) \otimes h(X)$ (switching the factors)
and since $G$ is a functor of symmetric monoidal categories (so compatible with
commutativity constraints), and by our convention in
Homology, Example \ref{homology-example-graded-vector-spaces}
we conclude that $H^*(X)$ is a graded
commutative algebra! Hence we get our datum (D1).

\medskip\noindent
Since $\mathbf{1}(1)$ is invertible in the category of motives
we see that $G(\mathbf{1}(1))$ is invertible in the category of
graded $F$-vector spaces. Thus $\sum_i \dim_F G^i(\mathbf{1}(1)) = 1$.
By assumption we only get something nonzero in degree $-2$.
Our datum (D0) is the vector space $F(1) = G^{-2}(\mathbf{1}(1))$.
Since $G$ is a symmetric monoidal functor we see that
$F(n) = G^{-2n}(\mathbf{1}(n))$ for all $n \in \mathbf{Z}$.
It follows that
$$
H^{2r}(X)(r) = G^{2r}(h(X)) \otimes G^{-2r}(\mathbf{1}(r)) =
G^0(h(X)(r))
$$
a formula we will frequently use below.

\medskip\noindent
Let $X$ be a smooth projective scheme over $k$. By
Lemma \ref{lemma-composition-correspondences} we have
$$
\CH^r(X) \otimes \mathbf{Q} = \text{Corr}^r(\Spec(k), X) =
\Hom(\mathbf{1}(-r), h(X)) = \Hom(\mathbf{1}, h(X)(r))
$$
Applying the functor $G$ this maps into
$\Hom(G(\mathbf{1}), G(h(X)(r)))$.
By taking the image of $1$ in $G^0(\mathbf{1}) = F$
into $G^0(h(X)(r)) = H^{2r}(X)(r)$ we obtain
$$
\gamma :
\CH^r(X) \otimes \mathbf{Q} \longrightarrow H^{2r}(X)(r)
$$
This is the datum (D2).

\medskip\noindent
Let $X$ be a nonempty smooth projective scheme over $k$
which is equidimensional of dimension $d$. By
Lemma \ref{lemma-composition-correspondences} we have
$$
\Mor(h(X)(d), \mathbf{1}) = \Mor((X, 1, d), (\Spec(k), 1, 0)) =
\text{Corr}^{-d}(X, \Spec(k)) = \CH_d(X)
$$
Thus the class of the cycle $[X]$ in $\CH_d(X)$ defines a morphism
$h(X)(d) \to \mathbf{1}$. Applying $G$ and taking degree $0$
parts we obtain
$$
H^{2d}(X)(d) = G^0(h(X)(d)) \longrightarrow G^0(\mathbf{1}) = F
$$
This map $\int_X : H^{2d}(X)(d) \to F$ is the datum (D3).

\medskip\noindent
Let $X$ be a smooth projective scheme over $k$ which is
nonempty and equidimensional of dimension $d$. By Lemma \ref{lemma-dual}
we know that $h(X)(d)$ is a left dual to $h(X)$. Hence
$G(h(X)(d)) = H^*(X) \otimes_F F(d)[2d]$
is a left dual to $H^*(X)$ in the category of graded $F$-vector spaces.
Here $[n]$ is the shift functor on graded vector spaces.
By Homology, Lemma \ref{homology-lemma-left-dual-graded-vector-spaces}
we find that $\sum_i \dim_F H^i(X) < \infty$ and that
$\epsilon : h(X)(d) \otimes h(X) \to \mathbf{1}$ produces
nondegenerate pairings $H^{2d - i}(X)(d) \otimes_F H^i(X) \to F$.
In the proof of Lemma \ref{lemma-dual} we have seen that
$\epsilon$ is given by $[\Delta]$ via the identifications
$$
\Hom(h(X)(d) \otimes h(X), \mathbf{1}) =
\text{Corr}^{-d}(X \times X, \Spec(k)) =
\CH_d(X \times X)
$$
Thus $\epsilon$ is the composition of $[X] : h(X)(d) \to \mathbf{1}$
and $h(\Delta)(d) : h(X)(d) \otimes h(X) \to h(X)(d)$. It follows
that the pairings above are given by cup product followed by
$\int_X$. This proves axiom (A).

\medskip\noindent
Axiom (B) follows from the assumption that $G$ is compatible
with tensor structures and our construction of the cup product above.

\medskip\noindent
Axiom (C). Our construction of $\gamma$ takes a cycle $\alpha$ on $X$,
interprets it a correspondence $a$ from $\Spec(k)$ to $X$ of some degree,
and then applies $G$. If $f : Y \to X$ is a morphism of nonempty
equidimensional smooth projective schemes over $k$, then
$f^!\alpha$ is the pushforward (!) of $\alpha$
by the correspondence $[\Gamma_f]$ from $X$ to $Y$, see
Lemma \ref{lemma-functor-and-cycles}. Hence
$f^!\alpha$ viewed as a correspondence from $\Spec(k)$ to $Y$
is equal to $a \circ [\Gamma_f]$, see
Lemma \ref{lemma-composition-correspondences}.
Since $G$ is a functor, we conclude
$\gamma$ is compatible with pullbacks, i.e., axiom (C)(a) holds.

\medskip\noindent
Let $f : Y \to X$ be a morphism of nonempty equidimensional
smooth projective schemes over $k$ and
let $\beta \in \CH^r(Y)$ be a cycle on $Y$. We have to show that
$$
\int_Y \gamma(\beta) \cup f^*c = \int_X \gamma(f_*\beta) \cup c
$$
for all $c \in H^*(X)$. Let $a, a^t, \eta_X, \eta_Y, [X], [Y]$
be as in Lemma \ref{lemma-prep-dual}.
Let $b$ be $\beta$ viewed as a correspondence from $\Spec(k)$ to $Y$
of degree $r$. Then $f_*\beta$ viewed as a correspondence from
$\Spec(k)$ to $X$ is equal to $a^t \circ b$, see
Lemmas \ref{lemma-functor-and-cycles} and
\ref{lemma-composition-correspondences}.
The displayed equality above holds if we can show that
$$
h(X) = \mathbf{1} \otimes h(X)
\xrightarrow{b \otimes 1}
h(Y)(r) \otimes h(X)
\xrightarrow{1 \otimes a}
h(Y)(r) \otimes h(Y)
\xrightarrow{\eta_Y}
h(Y)(r)
\xrightarrow{[Y]}
\mathbf{1}(r - e)
$$
is equal to
$$
h(X) = \mathbf{1} \otimes h(X)
\xrightarrow{a^t \circ b \otimes 1}
h(X)(r + d - e) \otimes h(X)
\xrightarrow{\eta_X}
h(X)(r + d - e)
\xrightarrow{[X]}
\mathbf{1}(r - e)
$$
This follows immediately from Lemma \ref{lemma-prep-dual}.
Thus we have axiom (C)(b).

\medskip\noindent
To prove axiom (C)(c) we use the discussion in
Remark \ref{remark-replace-cup-product-classical}.
Hence it suffices to prove that $\gamma$ is compatible with
exterior products. Let $X$, $Y$ be nonempty smooth projective
schemes over $k$ and let $\alpha$, $\beta$ be cycles on them. Denote
$a$, $b$ the corresponding correspondences from $\Spec(k)$ to
$X$, $Y$. Then $\alpha \times \beta$ corresponds to the
correspondence $a \otimes b$ from $\Spec(k)$ to $X \otimes Y = X \times Y$.
Hence the requirement follows from the fact that $G$ is
compatible with the tensor structures on both sides.

\medskip\noindent
Axiom (C)(d) follows because the cycle $[\Spec(k)]$
corresponds to the identity morphism on $h(\Spec(k))$.
This finishes the proof of the lemma.
\end{proof}

\begin{lemma}
\label{lemma-from-weil-to-functor}
Let $k$ be a field. Let $F$ be a field of characteristic $0$. Given
(D0), (D1), (D2), and (D3) satisfying (A), (B), and (C)
we can construct a $\mathbf{Q}$-linear functor
$$
G : M_k \longrightarrow \text{graded }F\text{-vector spaces}
$$
of symmetric monoidal categories such that $H^*(X) = G(h(X))$.
\end{lemma}

\begin{proof}
The proof of this lemma is the same as the proof of
Lemma \ref{lemma-from-weil-to-functor-classical};
we urge the reader to read the proof of that lemma instead.

\medskip\noindent
By Lemma \ref{lemma-characterize-motives} it suffices to construct a functor
$G$ on the category of smooth projective schemes over $k$
with morphisms given by correspondences of degree $0$ such that
the image of $G(c_2)$ on $G(\mathbf{P}^1_k)$ is an invertible graded
$F$-vector space.

\medskip\noindent
Let $X$ be a smooth projective scheme over $k$. There is a canonical
decomposition
$$
X = \coprod\nolimits_{0 \leq d \le \dim(X)} X_d
$$
into open and closed subschemes such that $X_d$ is equidimensional
of dimension $d$. By Lemma \ref{lemma-weil-additive} we have correspondingly
$$
H^*(X) \longrightarrow \prod\nolimits_{0 \leq d \le \dim(X)} H^*(X_d)
$$
If $Y$ is a second smooth projective scheme over $k$
and we similarly decompose $Y = \coprod Y_e$, then
$$
\text{Corr}^0(X, Y) = \bigoplus \text{Corr}^0(X_d, Y_e)
$$
As well we have $X \otimes Y = \coprod X_d \otimes Y_e$ in the
category of correspondences. From these observations it follows
that it suffices to construct $G$ on the category whose objects
are equidimensional smooth projective schemes over $k$
and whose morphisms are correspondences of degree $0$. (Some details
omitted.)

\medskip\noindent
Given an equdimensional smooth projective scheme
$X$ over $k$ we set $G(X) = H^*(X)$. Observe that $G(X) = 0$
if $X = \emptyset$ (Lemma \ref{lemma-unit}). Thus maps
from and to $G(\emptyset)$ are zero and we may and do
assume our schemes are nonempty in the discussions below.

\medskip\noindent
Given a correspondence $c \in \text{Corr}^0(X, Y)$ between
nonempty equidmensional smooth projective schemes over $k$
we consider the map $G(c) : G(X) = H^*(X) \to G(Y) = H^*(Y)$
given by the rule
$$
a \longmapsto
G(c)(a) = \text{pr}_{2, *}(\gamma(c) \cup \text{pr}_1^*a)
$$
It is clear that $G(c)$ is additive in $c$ and hence $\mathbf{Q}$-linear.
Compatibility of $\gamma$ with pullbacks, pushforwards, and
intersection products given by axioms (C)(a), (C)(b), and (C)(c)
shows that we have
$G(c' \circ c) = G(c') \circ G(c)$ if $c' \in \text{Corr}^0(Y, Z)$.
Namely, for $a \in H^*(X)$ we have
\begin{align*}
(G(c') \circ G(c))(a)
& =
\text{pr}^{23}_{3, *}(\gamma(c') \cup
\text{pr}^{23, *}_2(\text{pr}^{12}_{2, *}(\gamma(c) \cup
\text{pr}^{12, *}_1a))) \\
& =
\text{pr}^{23}_{3, *}(\gamma(c') \cup
\text{pr}^{123}_{23, *}(\text{pr}^{123, *}_{12}(\gamma(c) \cup
\text{pr}^{12, *}_1 a))) \\
& =
\text{pr}^{23}_{3, *}
\text{pr}^{123}_{23, *}(
\text{pr}^{123, *}_{23}\gamma(c') \cup
\text{pr}^{123, *}_{12}\gamma(c) \cup
\text{pr}^{123, *}_1 a) \\
& =
\text{pr}^{23}_{3, *}
\text{pr}^{123}_{23, *}(
\gamma(\text{pr}^{123, *}_{23}c') \cup
\gamma(\text{pr}^{123, *}_{12}c) \cup
\text{pr}^{123, *}_1 a) \\
& =
\text{pr}^{13}_{3, *}
\text{pr}^{123}_{13, *}(
\gamma(\text{pr}^{123, *}_{23}c' \cdot \text{pr}^{123, *}_{12}c) \cup
\text{pr}^{123, *}_1 a) \\
& =
\text{pr}^{13}_{3, *}(
\gamma(\text{pr}^{123}_{13, *}(\text{pr}^{123, *}_{23}c' \cdot
\text{pr}^{123, *}_{12}c)) \cup
\text{pr}^{13, *}_1 a) \\
& =
G(c' \circ c)(a)
\end{align*}
with obvious notation. The first equality follows from the definitions.
The second equality holds because
$\text{pr}^{23, *}_2 \circ \text{pr}^{12}_{2, *} =
\text{pr}^{123}_{23, *} \circ \text{pr}^{123, *}_{12}$
as follows immediately from the description of pushforward
along projections given in Lemma \ref{lemma-pr2star}.
The third equality holds by Lemma \ref{lemma-pushforward}
and the fact that $H^*$ is a functor.
The fourth equalith holds by axiom (C)(a) and the fact that
the gysin map agrees with flat pullback for flat morphisms
(Chow Homology, Lemma \ref{chow-lemma-lci-gysin-flat}).
The fifth equality uses axiom (C)(c) as well as
Lemma \ref{lemma-pushforward} to see that
$\text{pr}^{23}_{3, *} \circ \text{pr}^{123}_{23, *} =
\text{pr}^{13}_{3, *} \circ \text{pr}^{123}_{13, *}$.
The sixth equality uses the projection formula from
Lemma \ref{lemma-pushforward} as well as
axiom (C)(b) to see that $
\text{pr}^{123}_{13, *}
\gamma(\text{pr}^{123, *}_{23}c' \cdot \text{pr}^{123, *}_{12}c) =
\gamma(\text{pr}^{123}_{13, *}(
\text{pr}^{123, *}_{23}c' \cdot \text{pr}^{123, *}_{12}c))$.
Finally, the last equality is the definition.

\medskip\noindent
To finish the proof that $G$ is a functor,
we have to show identities are preserved. In other words, if
$1 = [\Delta] \in \text{Corr}^0(X, X)$ is the identity in the category
of correspondences (Lemma \ref{lemma-category-correspondences}),
then we have to show that $G([\Delta]) = \text{id}$.
This follows from the determination
of $\gamma([\Delta])$ in Lemma \ref{lemma-class-diagonal}
and Lemma \ref{lemma-pr2star}.
This finishes the construction of $G$ as a functor on
smooth projective schemes over $k$ and correspondences of degree $0$.

\medskip\noindent
By Lemma \ref{lemma-base} we have that
$G(\Spec(k)) = H^*(\Spec(k))$ is canonically isomorphic to $F$
as an $F$-algebra. The K\"unneth axiom (B)(a)
shows our functor is compatible with tensor products.
Thus our functor is a functor of symmetric monoidal categories.

\medskip\noindent
We still have to check that the image of $G(c_2)$ on
$G(\mathbf{P}^1_k) = H^*(\mathbf{P}^1_k)$
is an invertible graded $F$-vector space (in particular we don't know yet
that $G$ extends to $M_k$). By Lemma \ref{lemma-cohomology-P1}
we only have nonzero cohomology in degrees $0$ and $2$
both of dimension $1$. We have $1 = c_0 + c_2$ is a decomposition
of the identity into a sum of orthogonal idempotents in
$\text{Corr}^0(\mathbf{P}^1_k, \mathbf{P}^1_k)$, see
Example \ref{example-decompose-P1}. Further we have $c_0 = a \circ b$ where
$a \in \text{Corr}^0(\Spec(k), \mathbf{P}^1_k)$ and
$b \in \text{Corr}^0(\mathbf{P}^1_k, \Spec(k))$ and where
$b \circ a = 1$ in $\text{Corr}^0(\Spec(k), \Spec(k))$, see proof of
Lemma \ref{lemma-inverse-h2}. Thus $G(c_0)$ is the projector
onto the degree $0$ part. It follows that $G(c_2)$ must
be the projector onto the degree $2$ part and the proof is complete.
\end{proof}

\begin{proposition}
\label{proposition-weil-cohomology-theory}
Let $k$ be a field. Let $F$ be a field of characteristic $0$. There is a
$1$-to-$1$ correspondence between the following
\begin{enumerate}
\item data (D0), (D1), (D2), and (D3) satisfying (A), (B), and(C), and
\item $\mathbf{Q}$-linear symmetric monoidal functors
$$
G : M_k \longrightarrow \text{graded }F\text{-vector spaces}
$$
such that $G(\mathbf{1}(1))$ is nonzero only in degree $-2$.
\end{enumerate}
\end{proposition}

\begin{proof}
Given $G$ as in (2) by setting $H^*(X) = G(h(X))$ we obtain data
(D0), (D1), (D2), and (D3) satisfying (A), (B), and (C),
see Lemma \ref{lemma-from-functor-to-weil} and its proof.

\medskip\noindent
Conversely, given data (D0), (D1), (D2), and (D3)
satisfying (A), (B), and (C) we get a functor $G$ as in (2)
by the construction of the proof of Lemma \ref{lemma-from-weil-to-functor}.

\medskip\noindent
We omit the detailed proof that these constructions are inverse
to each other.
\end{proof}











\section{Further properties}
\label{section-further}

\noindent
In this section we prove a few more results one obtains if
given data (D0), (D1), (D2), and (D3) satisfying (A), (B), and (C) as in
Section \ref{section-axioms}.

\begin{lemma}
\label{lemma-trace-disjoint-union}
Assume given (D0), (D1), (D2), and (D3) satisfying (A), (B), and (C).
Let $X, Y$ be nonempty smooth projective schemes both equidimensional
of dimension $d$ over $k$. Then $\int_{X \amalg Y} = \int_X + \int_Y$.
\end{lemma}

\begin{proof}
Denote $i : X \to X \amalg Y$ and $j : Y \to X \amalg Y$ be the coprojections.
By Lemma \ref{lemma-weil-additive} the map
$(i^*, j^*) : H^*(X \amalg Y) \to H^*(X) \times H^*(Y)$ is an isomorphism.
The statement of the lemma means that under the isomorphism
$(i^*, j^*) : H^{2d}(X \amalg Y)(d) \to H^{2d}(X)(d) \oplus H^{2d}(Y)(d)$
the map $\int_X + \int_Y$ is tranformed into $\int_{X \amalg Y}$.
This is true because
$$
\int_{X \amalg Y} a =
\int_{X \amalg Y} i_*(i^*a) + j_*(j^*a) =
\int_X i^*a + \int_Y j^*a
$$
where the equality $a = i_*(i^*a) + j_*(j^*a)$ was shown in
the proof of Lemma \ref{lemma-weil-additive}.
\end{proof}

\begin{lemma}
\label{lemma-dim-0}
Assume given (D0), (D1), (D2), and (D3) satisfying (A), (B), and (C).
Let $X$ be a smooth projective scheme of dimension zero over $k$.
Then
\begin{enumerate}
\item $H^i(X) = 0$ for $i \not = 0$,
\item $H^0(X)$ is a finite separable algebra over $F$,
\item $\dim_F H^0(X) = \deg(X \to \Spec(F))$,
\item $\int_X : H^0(X) \to F$ is the trace map,
\item $\gamma([X]) = 1$, and
\item $\int_X \gamma([X]) = \deg(X \to \Spec(k))$.
\end{enumerate}
\end{lemma}

\begin{proof}
We can write $X = \Spec(k')$ where $k'$ is a finite separable
algebra over $k$. Observe that $\deg(X \to \Spec(k)) = [k' : k]$.
Choose a finite Galois extension $k''/k$ containing each of the
factors of $k'$. (Recall that a finite separable $k$-algebra is
a product of finite separable field extension of $k$.)
Set $\Sigma = \Hom_k(k', k'')$. Then we get
$$
k' \otimes_k k'' = \prod\nolimits_{\sigma \in \Sigma} k''
$$
Setting $Y = \Spec(k'')$ axioms (B)(a) and Lemma \ref{lemma-weil-additive} give
$$
H^*(X) \otimes_F H^*(Y) =
\prod\nolimits_{\sigma \in \Sigma} H^*(Y)
$$
as graded commutative $F$-algebras. By Lemma \ref{lemma-unit} the
$F$-algebra $H^*(Y)$ is nonzero. Comparing dimensions on either side
of the displayed equation we conclude that $H^*(X)$ sits only in degree $0$
and $\dim_F H^0(X) = [k' : k]$. Applying this to $Y$ we get
$H^*(Y) = H^0(Y)$. Since
$$
H^0(X) \otimes_F H^0(Y) = H^0(Y) \times \ldots \times H^0(Y)
$$
as $F$-algebras, it follows that $H^0(X)$ is a separable $F$-algebra
because we may check this after the faithfully flat base change
$F \to H^0(Y)$.

\medskip\noindent
The displayed isomorphism above is given by the map
$$
H^0(X) \otimes_F H^0(Y) \longrightarrow
\prod\nolimits_{\sigma \in \Sigma} H^0(Y),\quad
a \otimes b \longmapsto \prod\nolimits_\sigma \Spec(\sigma)^*a \cup b
$$
Via this isomorphism we have $\int_{X \times Y} = \sum_\sigma \int_Y$ by
Lemma \ref{lemma-trace-disjoint-union}. Thus
$$
\int_X a = \text{pr}_{1, *}(a \otimes 1) = \sum \Spec(\sigma)^*a
$$
in $H^0(Y)$; the first equality by Lemma \ref{lemma-pr2star}
and the second by the observation we just made. Choose an
algebraic closure $\overline{F}$ and
a $F$-algebra map $\tau : H^0(Y) \to \overline{F}$.
The isomorphism above base changes to the isomorphism
$$
H^0(X) \otimes_F \overline{F} \longrightarrow
\prod\nolimits_{\sigma \in \Sigma} \overline{F},\quad
a \otimes b \longmapsto \prod\nolimits_\sigma \tau(\Spec(\sigma)^*a) b
$$
It follows that $a \mapsto \tau(\Spec(\sigma)^*a)$ is a full set
of embeddings of $H^0(X)$ into $\overline{F}$. Applying $\tau$
to the formula for $\int_X a$ obtained above we conclude
that $\int_X$ is the trace map.
By Lemma \ref{lemma-unit} we have $\gamma([X]) = 1$.
Finally, we have $\int_X \gamma([X]) = \deg(X \to \Spec(k))$
because $\gamma([X]) = 1$ and the trace of $1$ is equal to $[k' : k]$
\end{proof}

\begin{lemma}
\label{lemma-degrees-cycles}
Assume given (D0), (D1), (D2), and (D3) satisfying (A), (B), and (C).
Let $X$ be a nonempty smooth projective scheme
equidimensional of dimension $d$ over $k$. The diagram
$$
\xymatrix{
\CH^d(X) \ar[r]_-\gamma \ar@{=}[d] &
H^{2d}(X)(d) \ar[d]^{\int_X} \\
\CH_0(X) \ar[r]^\deg & F
}
$$
commutes where $\deg : \CH_0(X) \to \mathbf{Z}$ is the degree of
zero cycles discussed in Chow Homology, Section
\ref{chow-section-degree-zero-cycles}.
\end{lemma}

\begin{proof}
Let $x$ be a closed point of $X$ whose residue field is separable
over $k$. View $x$ as a scheme and denote $i : x \to X$ the inclusion morphism.
To avoid confusion denote $\gamma' : \CH_0(x) \to H^0(x)$ the cycle class map
for $x$. Then we have
$$
\int_X \gamma([x]) = \int_X \gamma(i_*[x]) =
\int_X i_*\gamma'([x]) = \int_x \gamma'([x]) = \deg(x \to \Spec(k))
$$
The second equality is axiom (C)(b) and the third equality is
the definition of $i_*$ on cohomology. The final equality is
Lemma \ref{lemma-dim-0}. This proves the lemma
because $\CH_0(X)$ is generated by the classes of points $x$ as above
by Lemma \ref{lemma-generated-by-separable}.
\end{proof}

\begin{lemma}
\label{lemma-square-diagonal}
Assume given (D0), (D1), (D2), and (D3) satisfying (A), (B), and (C).
Let $X$ be a nonempty smooth projective scheme over $k$ which is
equidimensional of dimension $d$. We have
$$
\sum\nolimits_i (-1)^i\dim_F H^i(X) =
\deg(\Delta \cdot \Delta) = \deg(c_d(\mathcal{T}_{X/k}))
$$
\end{lemma}

\begin{proof}
Equality on the right. We have
$[\Delta] \cdot [\Delta] = \Delta_*(\Delta^![\Delta])$
(Chow Homology, Lemma \ref{chow-lemma-intersect-regularly-embedded}).
Since $\Delta_*$ preserves degrees of $0$-cycles it suffices to compute
the degree of $\Delta^![\Delta]$. The class $\Delta^![\Delta]$ is given
by capping $[\Delta]$ with
the top Chern class of the normal sheaf of $\Delta \subset X \times X$
(Chow Homology, Lemma \ref{chow-lemma-gysin-fundamental}).
Since the conormal sheaf of $\Delta$ is $\Omega_{X/k}$
(Morphisms, Lemma \ref{morphisms-lemma-differentials-diagonal})
we see that the normal sheaf is equal to the tangent sheaf
$\mathcal{T}_{X/k} = \SheafHom_{\mathcal{O}_X}(\Omega_{X/k}, \mathcal{O}_X)$
as desired.

\medskip\noindent
Equality on the left. By Lemma \ref{lemma-degrees-cycles} we have
\begin{align*}
\deg([\Delta] \cdot [\Delta])
& =
\int_{X \times X} \gamma([\Delta]) \cup \gamma([\Delta]) \\
& =
\int_{X \times X} \Delta_*1 \cup \gamma([\Delta]) \\
& =
\int_{X \times X} \Delta_*(\Delta^*\gamma([\Delta])) \\
& =
\int_X \Delta^*\gamma([\Delta])
\end{align*}
We have used Lemmas \ref{lemma-push-unit} and
\ref{lemma-pushforward}.
Write $\gamma([\Delta]) = \sum  e_{i, j} \otimes e'_{2d - i , j}$
as in Lemma \ref{lemma-class-diagonal}.
Recalling that $\Delta^*$ is given by cup product
(Remark \ref{remark-replace-cup-product}) we obtain
$$
\int_X \sum\nolimits_{i, j} e_{i, j} \cup e'_{2d - i, j} =
\sum\nolimits_{i, j} \int_X e_{i, j} \cup e'_{2d - i, j} =
\sum\nolimits_{i, j} (-1)^i = \sum (-1)^i\beta_i
$$
as desired.
\end{proof}

\begin{lemma}
\label{lemma-algebra-relations}
Let $F$ be a field of characteristic $0$.
Let $F'$ and $F_i$, $i = 1, \ldots, r$
be finite separable $F$-algebras. Let $A$ be a finite $F$-algebra.
Let $\sigma, \sigma' : A \to F'$ and $\sigma_i : A \to F_i$
be $F$-algebra maps. Assume $\sigma$ and $\sigma'$ surjective.
If there is a relation
$$
\text{Tr}_{F'/F} \circ \sigma - \text{Tr}_{F'/F} \circ \sigma' =
n(\sum m_i \text{Tr}_{F_i/F} \circ \sigma_i)
$$
where $n > 1$ and $m_i$ are integers, then $\sigma = \sigma'$.
\end{lemma}

\begin{proof}
We may write $A = \prod A_j$ as a finite product of
local Artinian $F$-algebras $(A_j, \mathfrak m_j, \kappa_j)$, see
Algebra, Lemma \ref{algebra-lemma-finite-dimensional-algebra} and
Proposition \ref{algebra-proposition-dimension-zero-ring}.
Denote $A' = \prod \kappa_j$ where the product is over those $j$
such that $\kappa_j/k$ is separable. Then each of the maps
$\sigma, \sigma', \sigma_i$ factors over the map
$A \to A'$. After replacing $A$ by $A'$ we may assume
$A$ is a finite separable $F$-algebra.

\medskip\noindent
Choose an algebraic closure $\overline{F}$. Set
$\overline{A} = A \otimes_F \overline{F}$,
$\overline{F}' = F' \otimes_F \overline{F}$, and
$\overline{F}_i = F_i \otimes_F \overline{F}$.
We can base change $\sigma$, $\sigma'$, $\sigma_i$
to get $\overline{F}$ algebra maps $\overline{A} \to \overline{F}'$
and $\overline{A} \to \overline{F}_i$. Moreover
$\text{Tr}_{\overline{F}'/\overline{F}}$ is the base
change of $\text{Tr}_{F'/F}$ and similarly for
$\text{Tr}_{F_i/F}$. Thus we may replace
$F$ by $\overline{F}$ and we reduce to the case discussed in
the next paragraph.

\medskip\noindent
Assume $F$ is algebraically closed and $A$ a finite separable $F$-algebra.
Then each of $A$, $F'$, $F_i$ is a product of copies of $F$.
Let us say an element $e$ of a product
$F \times \ldots \times F$ of copies of $F$ is a minimal idempotent
if it generates one of the factors, i.e., if
$e = (0, \ldots, 0, 1, 0, \ldots, 0)$. Let $e \in A$ be a minimal idempotent.
Since $\sigma$ and $\sigma'$ 
are surjective, we see that $\sigma(e)$ and $\sigma'(e)$ are minimal
idempotents or zero. If $\sigma \not = \sigma'$, then we can choose
a minimal idempotent $e \in A$ such that $\sigma(e) = 0$ and
$\sigma'(e) \not = 0$ or vice versa. Then
$\text{Tr}_{F'/F}(\sigma(e)) = 0$ and
$\text{Tr}_{F'/F}(\sigma'(e)) = 1$ or vice versa.
On the other hand, $\sigma_i(e)$ is an idempotent
and hence $\text{Tr}_{F_i/F}(\sigma_i(e)) = r_i$ is an integer.
We conclude that
$$
-1 = \sum n m_i r_i = n (\sum m_i r_i)
\quad\text{or}\quad
1 = \sum n m_i r_i = n (\sum m_i r_i)
$$
which is impossible.
\end{proof}

\begin{lemma}
\label{lemma-relations-classes-points}
Assume given (D0), (D1), (D2), and (D3) satisfying (A), (B), and (C).
Let $k'/k$ be a finite separable extension.
Let $X$ be a smooth projective scheme over $k'$.
Let $x, x' \in X$ be $k'$-rational points.
If $\gamma(x) \not = \gamma(x')$, then
$[x] - [x']$ is not divisible by any integer $n > 1$ in $\CH_0(X)$.
\end{lemma}

\begin{proof}
If $x$ and $x'$ lie on distinct irreducible components of $X$, then
the result is obvious. Thus we may $X$ irreducible of dimension $d$.
Say $[x] - [x']$ is divisible by $n > 1$ in $\CH_0(X)$.
We may write $[x] - [x'] = n(\sum m_i [x_i])$ in $\CH_0(X)$
for some $x_i \in X$ closed points
whose residue fields are separable over $k$ by
Lemma \ref{lemma-generated-by-separable}.
Then
$$
\gamma([x]) - \gamma([x']) = n (\sum m_i \gamma([x_i]))
$$
in $H^{2d}(X)(d)$. Denote $i^*, (i')^*, i_i^*$ the pullback maps
$H^0(X) \to H^0(x)$, $H^0(X) \to H^0(x')$, $H^0(X) \to H^0(x_i)$.
Recall that $H^0(x)$ is a finite separable $F$-algebra
and that $\int_x : H^0(x) \to F$ is the trace map
(Lemma \ref{lemma-dim-0}) which we will denote $\text{Tr}_x$.
Similarly for $x'$ and $x_i$. Then by Poincar\'e duality in the form of
axiom (A)(b) the equation above is dual to
$$
\text{Tr}_x \circ i^* - \text{Tr}_{x'} \circ (i')^* =
n(\sum m_i \text{Tr}_{x_i} \circ i_i^*)
$$
which takes place in $\Hom_F(H^0(X), F)$. Finally, observe that
$i^*$ and $(i')^*$ are surjective as $x$ and $x'$ are $k'$-rational
points and hence the compositions $H^0(\Spec(k')) \to H^0(X) \to H^0(x)$
and $H^0(\Spec(k')) \to H^0(X) \to H^0(x')$ are isomorphisms.
By Lemma \ref{lemma-algebra-relations} we conclude that $i^* = (i')^*$
which contradicts the assumption that $\gamma([x]) \not = \gamma([x'])$.
\end{proof}

\begin{lemma}
\label{lemma-classes-points}
Assume given (D0), (D1), (D2), and (D3) satisfying (A), (B), and (C).
Let $k'/k$ be a finite separable extension. Let $X$ be a geometrically
irreducible smooth projective scheme over $k'$ of dimension $d$.
Then $\gamma : \CH_0(X) \to H^{2d}(X)(d)$ factors through
$\deg : \CH_0(X) \to \mathbf{Z}$.
\end{lemma}

\begin{proof}
By Lemma \ref{lemma-generated-by-separable} it suffices to show: given
closed points $x, x' \in X$ whose residue fields are separable over $k$
we have $\deg(x') \gamma([x]) = \deg(x) \gamma([x'])$.

\medskip\noindent
We first reduce to the case of $k'$-rational points. Let $k''/k'$ be a
Galois extension such that $\kappa(x)$ and $\kappa(x')$ embed into $k''$
over $k$. Set $Y = X \times_{\Spec(k')} \Spec(k'')$ and denote $p : Y \to X$
the projection. By our choice of $k''/k'$ there exists a
$k''$-rational point $y$, resp.\ $y'$ on $Y$ mapping to $x$, resp.\ $x'$.
Then $p_*[y] = [k'' : \kappa(x)][x]$ and
$p_*[y'] = [k'' : \kappa(x')][x']$ in $\CH_0(X)$.
By compatibility with pushforwards given in axiom (C)(b)
it suffices to prove $\gamma([y]) = \gamma([y'])$ in $\CH^{2d}(Y)(d)$.
This reduces us to the discussion in the next paragraph.

\medskip\noindent
Assume $x$ and $x'$ are $k'$-rational points. By
Lemma \ref{lemma-divide-difference-points} there
exists a finite separable extension $k''/k'$ of fields
such that the pullback $[y] - [y']$
of the difference $[x] - [x']$ becomes divisible
by an integer $n > 1$ on $Y = X \times_{\Spec(k')} \Spec(k'')$.
(Note that $y, y' \in Y$ are $k''$-rational points.)
By Lemma \ref{lemma-relations-classes-points} we have
$\gamma([y]) = \gamma([y'])$ in $H^{2d}(Y)(d)$.
By compatibility with pushforward in axiom (C)(b)
we conclude the same for $x$ and $x'$.
\end{proof}

\begin{lemma}
\label{lemma-injective}
Assume given (D0), (D1), (D2), and (D3) satisfying (A), (B), and (C). Let
$f : X \to Y$ be a dominant morphism of irreducible smooth projective schemes
over $k$. Then $H^*(Y) \to H^*(X)$ is injective.
\end{lemma}

\begin{proof}
There exists an integral closed subscheme $Z \subset X$ of the same
dimension as $Y$ mapping onto $Y$. Thus $f_*[Z] = m[Y]$ for some $m > 0$.
Then $f_* \gamma([Z]) = m \gamma([Y]) = m$ in $H^*(Y)$ because of
Lemma \ref{lemma-unit}. Hence by the projection formula
(Lemma \ref{lemma-pushforward})
we have $f_*(f^*a \cup \gamma([Z])) = m a$ and we conclude.
\end{proof}

\begin{lemma}
\label{lemma-otimes}
Assume given (D0), (D1), (D2), and (D3) satisfying (A), (B), and (C). Let
$k''/k'/k$ be finite separable algebras and let $X$ be a
smooth projective scheme over $k'$. Then
$$
H^*(X) \otimes_{H^0(\Spec(k'))} H^0(\Spec(k'')) =
H^*(X \times_{\Spec(k')} \Spec(k''))
$$
\end{lemma}

\begin{proof}
We will use the results of Lemma \ref{lemma-dim-0} without further mention.
Write
$$
k' \otimes_k k'' = k'' \times l
$$
for some finite separable $k'$-algebra $l$. Write
$F' = H^0(\Spec(k'))$, $F'' = H^0(\Spec(k''))$, and $G = H^0(\Spec(l))$.
Since $\Spec(k') \times \Spec(k'') = \Spec(k'') \amalg \Spec(l)$ we
deduce from axiom (B)(a) and Lemma \ref{lemma-weil-additive}
that we have
$$
F' \otimes_F F'' = F'' \times G
$$
The map from left to right identifies $F''$ with $F' \otimes_{F'} F''$.
By the same token we have
$$
H^*(X) \otimes_F F'' = H^*(X \times_{\Spec(k')} \Spec(k''))
\times H^*(X \times_{\Spec(k')} \Spec(l))
$$
as modules over $F' \otimes_F F'' = F'' \times G$. This proves the lemma.
\end{proof}






















\section{Weil cohomology theories, II}
\label{section-old}

\noindent
For us a Weil cohomology theory will be the analogue of a
classical Weil cohomology theory (Section \ref{section-axioms-classical})
when the ground field $k$ is not algebraically closed.
In Section \ref{section-axioms} we listed axioms which guarantee
our cohomology theory comes from a symmetric monoidal functor
on the category of motives over $k$. Missing from our axioms so
far are the condition $H^i(X) = 0$ for $i < 0$ and
a condition on $H^{2d}(X)(d)$ for $X$ equidimensional of dimension $d$
corresponding to the classical axioms (A)(c) and (A)(d).
Let us first convince the reader that it is necessary to impose
such conditions.

\begin{example}
\label{example-weird-weil}
Let $k = \mathbf{C}$ and $F = \mathbf{C}$ both be equal to the field
of complex numbers. For $X$ smooth projective over $k$ denote
$H^{p, q}(X) = H^q(X, \Omega^p_{X/k})$. Let $(H')^*$ be the functor
which sends $X$ to $(H')^*(X) = \bigoplus H^{p, q}(X)$ with the
usual cup product.
This is a classical Weil cohomology theory (insert future reference here).
By Proposition \ref{proposition-weil-cohomology-theory-classical}
we obtain a $\mathbf{Q}$-linear symmetric monoidal functor $G'$ from $M_k$
to the category of graded $F$-vector spaces. Of course, in this case
for every $M$ in $M_k$ the value $G'(M)$ is naturally bigraded, i.e.,
we have
$$
(G')(M) = \bigoplus (G')^{p, q}(M),\quad
(G')^n = \bigoplus\nolimits_{n = p + q} (G')^{p, q}(M)
$$
with $(G')^{p, q}$ sitting in total degree $p + q$ as indicated.
Now we are going to construct a $\mathbf{Q}$-linear symmetric monoidal functor
$G$ to the category of graded $F$-vector spaces by setting
$$
G^n(M) = \bigoplus\nolimits_{n = 3p - q} (G')^{p, q}(M)
$$
We omit the verification that this defines a symmetric monoidal
functor (a technical point is that because we chose odd numbers
$3$ and $-1$ above the functor $G$ is compatible with the
commutativity constraints).
Observe that $G(\mathbf{1}(1))$ is still sitting in degree $-2$!
Hence by Lemma \ref{lemma-from-functor-to-weil-classical}
we obtain a functor $H^*$, cycle classes $\gamma$, and trace maps
satisfying all classical axioms (A), (B), (C), except for possibly
the classical axioms (A)(a) and (A)(d).
However, if $E$ is an elliptic curve over $k$, then we find
$\dim H^{-1}(E) = 1$, i.e., axiom (A)(a) is indeed violated.
\end{example}

\begin{lemma}
\label{lemma-H-0-separable}
Assume given (D0), (D1), (D2), and (D3) satisfying (A), (B), and (C).
Let $X$ be a smooth projective scheme over $k$.
Set $k' = \Gamma(X, \mathcal{O}_X)$. The following are equivalent
\begin{enumerate}
\item there exist finitely many closed points $x_1, \ldots, x_r \in X$
whose residue fields are separable over $k$ such that
$H^0(X) \to H^0(x_1) \oplus \ldots \oplus H^0(x_r)$ is injective,
\item the map $H^0(\Spec(k')) \to H^0(X)$ is an isomorphism.
\end{enumerate}
If this is true, then $H^0(X)$ is a finite separable algebra over $F$.
If $X$ is equidimensional of dimension $d$, then (1) and (2)
are also equivalent to
\begin{enumerate}
\item[(3)] the classes of closed points generate $H^{2d}(X)(d)$
as a module over $H^0(X)$.
\end{enumerate}
\end{lemma}

\begin{proof}
We observe that the statement makes sense because $k'$ is a finite separable
algebra over $k$ (Varieties, Lemma
\ref{varieties-lemma-proper-geometrically-reduced-global-sections})
and hence $\Spec(k')$ is smooth and projective over $k$.
The compatibility of $H^*$ with direct sums
(Lemmas \ref{lemma-weil-additive} and \ref{lemma-trace-disjoint-union})
shows that it suffices to prove the lemma when $X$ is connected.
Hence we may assume $X$ is irreducible and we have to show the
equivalence of (1), (2), and (3). Set $d = \dim(X)$.
This implies that $k'$ is a field finite separable
over $k$ and that $X$ is geometrically irreducible over $k'$, see
Varieties, Lemmas
\ref{varieties-lemma-proper-geometrically-reduced-global-sections} and
\ref{varieties-lemma-baby-stein}.

\medskip\noindent
By Lemma \ref{lemma-generated-by-separable} we see that the closed
points in (3) may be assumed to have separable residue fields over $k$.
By axioms (A)(a) and (A)(b) we see that conditions (1) and (3) are equivalent.

\medskip\noindent
If (2) holds, then pick any closed point $x \in X$ whose residue field
is finite separable over $k'$. Then
$H^0(\Spec(k')) = H^0(X) \to H^0(x)$ is injective for example by
Lemma \ref{lemma-injective}.

\medskip\noindent
Assume the equivalent conditions (1) and (3) hold. Choose
$x_1, \ldots, x_r \in X$ as in (1). Choose a finite separable
extension $k''/k'$. By Lemma \ref{lemma-otimes} we have
$$
H^0(X) \otimes_{H^0(\Spec(k'))} H^0(\Spec(k'')) =
H^0(X \times_{\Spec(k')} \Spec(k''))
$$
Thus in order to show that
$H^0(\Spec(k')) \to H^0(X)$ is an isomorphism
we may replace $k'$ by $k''$. Thus we may assume $x_1, \ldots, x_r$
are $k'$-rational points (this replaces each $x_i$ with multiple
points, so $r$ is increased in this step). By Lemma \ref{lemma-classes-points}
$\gamma(x_1) = \gamma(x_2) = \ldots = \gamma(x_r)$.
By axiom (A)(b) all the maps $H^0(X) \to H^0(x_i)$
are the same. This means (2) holds.

\medskip\noindent
Finally, Lemma \ref{lemma-dim-0} implies
$H^0(X)$ is a separable $F$-algebra if (1) holds.
\end{proof}

\begin{lemma}
\label{lemma-negative-cohomology}
Assume given (D0), (D1), (D2), and (D3) satisfying (A), (B), and (C).
If there exists a smooth projective scheme $Y$ over $k$ such that
$H^i(Y)$ is nonzero for some $i < 0$, then there exists an equidimensional
smooth projective scheme $X$ over $k$ such that the equivalent conditions
of Lemma \ref{lemma-H-0-separable} fail for $X$.
\end{lemma}

\begin{proof}
By Lemma \ref{lemma-weil-additive} we may assume $Y$ is irreducible
and a fortiori equidimensional. If $i$ is odd, then after replacing
$Y$ by $Y \times Y$ we find an example where $Y$ is equidimensional
and $i = -2l$ for some $l > 0$. Set $X = Y \times (\mathbf{P}^1_k)^l$.
Using axiom (B)(a) we obtain
$$
H^0(X) \supset H^0(Y) \oplus
H^i(Y) \otimes_F H^2(\mathbf{P}^1_k)^{\otimes_F l}
$$
with both summands nonzero. Thus it is clear that $H^0(X)$ cannot be
isomorphic to $H^0$ of the spectrum of
$\Gamma(X, \mathcal{O}_X) = \Gamma(Y, \mathcal{O}_Y)$
as this falls into the first summand.
\end{proof}

\noindent
Thus it makes sense to finally make the following definition.

\begin{definition}
\label{definition-weil-cohomology-theory}
Let $k$ be a field. Let $F$ be a field of characteristic $0$.
A {\it Weil cohomology theory} over $k$ with coefficients in $F$
is given by data (D0), (D1), (D2), and (D3) satisfying
Poincar\'e duality, the K\"unneth formula, and compatibility
with cycle classes, more precisely, satisfying axioms (A), (B), and (C)
of Section \ref{section-axioms}
and in addition such that the equivalent conditions (1) and (2) of
Lemma \ref{lemma-H-0-separable} hold for every smooth projective $X$ over $k$.
\end{definition}

\noindent
By Lemma \ref{lemma-negative-cohomology} this means also that there are no
nonzero negative cohomology groups. In particular, if $k$ is algebraically
closed, then a Weil cohomology theory as above together with an isomorphism
$F \to F(1)$ is the same thing as a classical Weil cohomology theory.

\begin{remark}
\label{remark-betti-numbers-in-some-sense}
Let $H^*$ be a Weil cohomology theory
(Definition \ref{definition-weil-cohomology-theory}).
Let $X$ be a geometrically irreducible smooth projective scheme
of dimension $d$ over $k'$ with $k'/k$ a finite separable extension of fields.
Suppose that
$$
H^0(\Spec(k')) = F_1 \times \ldots \times F_r
$$
for some fields $F_i$. Then we accordingly can write
$$
H^*(X) = \prod\nolimits_{i = 1, \ldots, r}
H^*(X) \otimes_{H^0(\Spec(k'))} F_i
$$
Now, our final assumption in Definition \ref{definition-weil-cohomology-theory}
tells us that $H^0(X)$ is free of rank $1$ over $\prod F_i$.
In other words, each of the factors
$H^0(X) \otimes_{H^0(\Spec(k'))} F_i$ has dimension $1$ over $F_i$.
Poincar\'e duality then tells us that the same is true for
cohomology in degree $2d$.
What isn't clear however is that the same holds in other degrees.
Namely, we don't know that given $0 < n < \dim(X)$ the integers
$$
\dim_{F_i} H^n(X) \otimes_{H^0(\Spec(k'))} F_i
$$
are independent of $i$! This question is closely related to the following
open question: given an algebraically closed base field $\overline{k}$,
a field of characteristic zero $F$, a classical Weil cohomology theory
$H^*$ over $\overline{k}$ with coefficient field $F$, and a smooth projective
variety $X$ over $\overline{k}$ is it true that the betti numbers of $X$
$$
\beta_i = \dim_F H^i(X)
$$
are independent of $F$ and the Weil cohomology theory $H^*$?
\end{remark}

\begin{proposition}
\label{proposition-weil-cohomology-theory-again}
Let $k$ be a field. Let $F$ be a field of characteristic $0$.
A Weil cohomology theory is the same thing as a $\mathbf{Q}$-linear
symmetric monoidal functor
$$
G : M_k \longrightarrow \text{graded }F\text{-vector spaces}
$$
such that
\begin{enumerate}
\item $G(\mathbf{1}(1))$ is nonzero only in degree $-2$, and
\item for every smooth projective scheme $X$ over $k$ with
$k' = \Gamma(X, \mathcal{O}_X)$ the homomorphism
$G(h(\Spec(k'))) \to G(h(X))$ of graded $F$-vector spaces
is an isomorphism in degree $0$.
\end{enumerate}
\end{proposition}

\begin{proof}
Immediate consequence of Proposition \ref{proposition-weil-cohomology-theory}
and Definition \ref{definition-weil-cohomology-theory}. Of course we could
replace (2) by the condition that $G(h(X)) \to \bigoplus G(h(x_i))$
is injective in degree $0$ for some choice of closed points
$x_1, \ldots, x_r \in X$ whose residue fields are separable over $k$.
\end{proof}







\section{Chern classes}
\label{section-chern}

\noindent
In this section we discuss how given a first Chern class and a projective
space bundle formula we can get all Chern classes.
A reference for this section is \cite{Grothendieck-chern} although our
axioms are slightly different.

\medskip\noindent
Let $\mathcal{C}$ be a category of schemes with the following properties
\begin{enumerate}
\item Every $X \in \Ob(\mathcal{C})$ is quasi-compact and quasi-separated.
\item If $X \in \Ob(\mathcal{C})$ and $U \subset X$ is open and closed,
then $U \to X$ is a morphism of $\mathcal{C}$. If $X' \to X$ is a morphism
of $\mathcal{C}$ factoring through $U$, then $X' \to U$ is a morphism
of $\mathcal{C}$.
\item If $X \in \Ob(\mathcal{C})$ and if $\mathcal{E}$ is a finite
locally free $\mathcal{O}_X$-module, then
\begin{enumerate}
\item $p : \mathbf{P}(\mathcal{E}) \to X$ is a morphism of $\mathcal{C}$,
\item for a morphism $f : X' \to X$ in $\mathcal{C}$ the
induced morphism $\mathbf{P}(f^*\mathcal{E}) \to \mathbf{P}(\mathcal{E})$
is a morphism of $\mathcal{C}$,
\item if $\mathcal{E} \to \mathcal{F}$ is a surjection onto another finite
locally free $\mathcal{O}_X$-module then the closed immersion
$\mathbf{P}(\mathcal{F}) \to \mathbf{P}(\mathcal{E})$
is a morphism of $\mathcal{C}$.
\end{enumerate}
\end{enumerate}
Next, assume given a contravariant functor $A$ from the
category $\mathcal{C}$ to the category of graded algebras.
Here a graded algebra $A$ is a unital, associative,
not necessarily commutative $\mathbf{Z}$-algebra $A$ endowed with a grading
$A = \bigoplus_{i \geq 0} A^i$. Given a morphism
$f : X' \to X$ of $\mathcal{C}$ we denote $f^* : A(X) \to A(X')$ the
induced algebra map.
We will denote the product of $a, b \in A(X)$ by $a \cup b$.
Finally, we assume
given for every object $X$ of $\mathcal{C}$ an additive  map
$$
c_1^A : \Pic(X) \longrightarrow A^1(X)
$$
We assume the following axioms are satisfied
\begin{enumerate}
\item Given $X \in \Ob(\mathcal{C})$ and $\mathcal{L} \in \Pic(X)$
the element $c_1^A(\mathcal{L})$ is in the center of the algebra $A(X)$.
\item If $X \in \Ob(\mathcal{C})$ and $X = U \amalg V$ with $U$ and $V$
open and closed, then $A(X) = A(U) \times A(V)$ via the induced maps
$A(X) \to A(U)$ and $A(X) \to A(V)$.
\item If $f : X' \to X$ is a morphism of $\mathcal{C}$ and $\mathcal{L}$
is an invertible $\mathcal{O}_X$-module, then $f^*c_1^A(\mathcal{L}) =
c_1^A(f^*\mathcal{L})$.
\item Given $X \in \Ob(\mathcal{C})$ and locally free $\mathcal{O}_X$-module
$\mathcal{E}$ of constant rank $r$ consider the morphism
$p : P = \mathbf{P}(\mathcal{E}) \to X$ of $\mathcal{C}$.
Then the map
$$
\bigoplus\nolimits_{i = 0, \ldots, r - 1} A(X)
\longrightarrow A(P),\quad
(a_0, \ldots, a_{r - 1}) \longmapsto
\sum c_1^A(\mathcal{O}_P(1))^i \cup p^*(a_i)
$$
is bijective.
\item Let $X \in \Ob(\mathcal{C})$ and let $\mathcal{E} \to \mathcal{F}$
be a surjection of finite locally free $\mathcal{O}_X$-modules
of ranks $r + 1$ and $r$. Denote
$i : P' = \mathbf{P}(\mathcal{F}) \to \mathbf{P}(\mathcal{E}) = P$ the
corresponding incusion morphism. This is a morphism of $\mathcal{C}$
which exhibits $P'$ as an effective Cartier divisor on $P$. Then for
$a \in A(P)$ with $i^*a = 0$ we have
$a \cup c_1^A(\mathcal{O}_P(P')) = 0$.
\end{enumerate}
To formulate our result recall that $\textit{Vect}(X)$ denotes the
(exact) category of finite locally free $\mathcal{O}_X$-modules.
In Derived Categories of Schemes, Section \ref{perfect-section-K-groups}
we have defined the zeroth $K$-group
$K_0(\textit{Vect}(X))$ of this category.
Moreover, we have seen that $K_0(\textit{Vect}(X))$ is a ring, see
Derived Categories of Schemes, Remark \ref{perfect-remark-K-ring}.

\begin{proposition}
\label{proposition-chern-class}
In the situation above there is a unique rule which assigns to
every $X \in \Ob(\mathcal{C})$ a ``total Chern class''
$$
c^A : K_0(\textit{Vect}(X)) \longrightarrow  \prod\nolimits_{i \geq 0} A^i(X)
$$
with the following properties
\begin{enumerate}
\item For $X \in \Ob(\mathcal{C})$ we have
$c^A(\alpha + \beta) = c^A(\alpha) c^A(\beta)$
and $c^A(0) = 1$.
\item If $f : X' \to X$ is a morphism of $\mathcal{C}$, then
$f^* \circ c^A =  c^A \circ f^*$.
\item Given $X \in \Ob(\mathcal{C})$ and $\mathcal{L} \in \Pic(X)$
we have $c^A([\mathcal{L}]) = 1 + c_1^A(\mathcal{L})$.
\end{enumerate}
\end{proposition}

\begin{proof}
Let $X \in \Ob(\mathcal{C})$ and let $\mathcal{E}$ be a finite
locally free $\mathcal{O}_X$-module. We first show how to define
an element $c^A(\mathcal{E}) \in A(X)$.

\medskip\noindent
As a first step, let $X = \bigcup X_r$ be the decomposition into
open and closed subschemes such that $\mathcal{E}|_{X_r}$ has
constant rank $r$. Since $X$ is quasi-compact, this decomposition
is finite. Hence $A(X) = \prod A(X_r)$. Thus it suffices to define
$c^A(\mathcal{E})$ when $\mathcal{E}$ has constant rank $r$. In this
case let $p : P \to X$ be the projective bundle of $\mathcal{E}$.
We can uniquely define elements $c_i^A(\mathcal{E}) \in A^i(X)$
for $i \geq 0$ such that $c_0^A(\mathcal{E}) = 1$ and the equation
\begin{equation}
\label{equation-chern-classes}
\sum\nolimits_{i = 0}^r
(-1)^i c_1(\mathcal{O}_P(1))^i \cup p^*c^A_{r - i}(\mathcal{E})
= 0
\end{equation}
is true. As usual we set
$c^A(\mathcal{E}) = c_0^A(\mathcal{E}) + c_1^A(\mathcal{E}) + \ldots
+ c_r^A(\mathcal{E})$ in $A(X)$.

\medskip\noindent
If $\mathcal{E}$ is invertible, then
$c^A(\mathcal{E}) = 1 + c_1^A(\mathcal{L})$.
This follows immediately from the construction above.

\medskip\noindent
The elements $c_i^A(\mathcal{E})$ are in the center of $A(X)$.
Namely, to prove this we may assume $\mathcal{E}$ has constant rank $r$.
Let $p : P \to X$ be the corresponding projective bundle.
if $a \in A(X)$ then $p^*a \cup (-1)^r c_1(\mathcal{O}_P(1))^r =
(-1)^r c_1(\mathcal{O}_P(1))^r \cup p^*a$ and hence we must have the same
for all the other terms in the expression defining $c_i^A(\mathcal{E})$
as well and we conclude.

\medskip\noindent
If $f : X' \to X$ is a morphism of $\mathcal{C}$, then
$f^*c_i^A(\mathcal{E}) = c_i^A(f^*\mathcal{E})$.
Namely, to prove this we may assume $\mathcal{E}$ has constant rank $r$.
Let $p : P \to X$ and $p' : P' \to X'$ be the projective
bundles corresponding to $\mathcal{E}$ and $f^*\mathcal{E}$.
The induced morphism $g : P' \to P$ is a morphism of $\mathcal{C}$.
The pullback by $g$ of the equality defining $c_i^A(\mathcal{E})$
is the corresponding equation for $f^*\mathcal{E}$ and we conclude.

\medskip\noindent
Let $X \in \Ob(\mathcal{C})$. Consider a short exact sequence
$$
0 \to \mathcal{L} \to \mathcal{E} \to \mathcal{F} \to 0
$$
of finite locally free $\mathcal{O}_X$-modules with $\mathcal{L}$ invertible.
Then
$$
c^A(\mathcal{E}) = c^A(\mathcal{L}) c^A(\mathcal{F})
$$
Namely, by the construction of $c^A_i$ we may assume $\mathcal{E}$ has
constant rank $r + 1$ and $\mathcal{F}$ has constant rank $r$.
The inclusion
$$
i : P' = \mathbf{P}(\mathcal{F}) \longrightarrow \mathbf{P}(\mathcal{E}) = P
$$
is a morphism of $\mathcal{C}$ and it is the zero scheme of a regular
section of the invertible module
$\mathcal{L}^{\otimes -1} \otimes \mathcal{O}_P(1)$.
The element
$$
\sum\nolimits_{i = 0}^r (-1)^i c_1^A(\mathcal{O}_P(1))^i \cup
p^*c^A_i(\mathcal{F})
$$
pulls back to zero on $P'$ by definition. Hence we see that
$$
\left(c_1^A(\mathcal{O}_P(1)) - c_1^A(\mathcal{L})\right) \cup
\left(\sum\nolimits_{i = 0}^r (-1)^i c_1^A(\mathcal{O}_P(1))^i \cup
p^*c^A_i(\mathcal{F})\right) = 0
$$
in $A^*(P)$ by assumption (5) on our cohomology $A$.
By definition of $c_1^A(\mathcal{E})$
this gives the desired equality.

\medskip\noindent
Let $X \in \Ob(\mathcal{C})$. Consider a short exact sequence
$$
0 \to \mathcal{E} \to \mathcal{F} \to \mathcal{G} \to 0
$$
of finite locally free $\mathcal{O}_X$-modules. Then
$$
c^A(\mathcal{F}) = c^A(\mathcal{E}) c^A(\mathcal{G})
$$
Namely, by the construction of $c^A_i$ we may assume
$\mathcal{E}$, $\mathcal{F}$, and $\mathcal{G}$ have
constant ranks $r$, $s$, and $t$. We prove it by induction on $r$.
The case $r = 1$ was done above. If $r > 1$, then it suffices to check
this after pulling back by the morphism $\mathbf{P}(\mathcal{E}^\vee) \to X$.
Thus we may assume we have an invertible submodule
$\mathcal{L} \subset \mathcal{E}$ such that both
$\mathcal{E}' = \mathcal{E}/\mathcal{L}$ and
$\mathcal{F}' = \mathcal{E}/\mathcal{L}$ are finite locally free
(of ranks $s - 1$ and $t - 1$). Then we have
$$
c^A(\mathcal{E}) = c^A(\mathcal{L}) c^A(\mathcal{E}')
\quad\text{and}\quad
c^A(\mathcal{F}) = c^A(\mathcal{L}) c^A(\mathcal{F}')
$$
Since we have the short exact sequence
$$
0 \to \mathcal{E}' \to \mathcal{F}' \to \mathcal{G} \to 0
$$
we see by induction hypothesis that
$$
c^A(\mathcal{F}') = c^A(\mathcal{E}') c^A(\mathcal{G})
$$
Thus the result follows from a formal calculation.

\medskip\noindent
At this point for $X \in \Ob(\mathcal{C})$
we can define $c^A : K_0(\textit{Vect}(X)) \to A(X)$.
Namely, we send a generator $[\mathcal{E}]$ to $c^A(\mathcal{E})$
and we extend multiplicatively. Thus for example
$c^A(-[\mathcal{E}]) = c^A(\mathcal{E})^{-1}$ is the formal
inverse of $a^A([\mathcal{E}])$.
The multiplicativity in short exact sequences shown above
guarantees that this works.

\medskip\noindent
Uniqueness. Suppose $X \in \Ob(\mathcal{C})$ and $\mathcal{E}$
is a finite locally free $\mathcal{O}_X$-module. We want to show
that conditions (1), (2), and (3) of the lemma uniquely determine
$c^A([\mathcal{E}])$. To prove this we may assume $\mathcal{E}$
has constant rank $r$; this already uses (2). Then we may use induction on $r$.
If $r = 1$, then uniqueness follows from (3).
If $r > 1$ we pullback using (2) to the projective bundle $p : P \to X$
and we see that we may assume we have a short exact sequence
$0 \to \mathcal{E}' \to \mathcal{E} \to \mathcal{E}'' \to 0$
with $\mathcal{E}'$ and $\mathcal{E}''$ having lower rank.
By induction hypothesis $c^A(\mathcal{E}')$ and $c^A(\mathcal{E}'')$
are uniquely determined. Thus uniqueness for $\mathcal{E}$ by
the axiom (1).
\end{proof}

\begin{lemma}
\label{lemma-splitting-principle}
In the situation above. Let $X \in \Ob(\mathcal{C})$. Let $\mathcal{E}_i$
be a finite collection of locally free $\mathcal{O}_X$-modules of rank $r_i$.
There exists a morphism $p : P \to X$ in $\mathcal{C}$ such that
\begin{enumerate}
\item $p^* : A(X) \to A(P)$ is injective,
\item each $p^*\mathcal{E}_i$ has a filtration whose successive quotients
$\mathcal{L}_{i, 1}, \ldots, \mathcal{L}_{i, r_i}$
are invertible $\mathcal{O}_P$-modules.
\end{enumerate}
\end{lemma}

\begin{proof}
We may assume $r_i \geq 1$ for all $i$. We will prove the lemma by induction
on $\sum (r_i - 1)$. If this integer is $0$, then $\mathcal{E}_i$
is invertible for all $i$ and we conclude by taking $\pi = \text{id}_X$.
If not, then we can pick an $i$ such that $r_i > 1$ and consider the
projective bundle $p : P \to X$ associated to $\mathcal{E}_i$.
We have a short exact sequence
$$
0 \to \mathcal{F} \to p^*\mathcal{E}_i \to \mathcal{O}_P(1) \to 0
$$
of finite locally free $\mathcal{O}_P$-modules of ranks $r_i - 1$,
$r_i$, and $1$. Observe that $p^* : A(X) \to A(P)$ is injective
by assumption. By the induction hypothesis applied to the finite locally free
$\mathcal{O}_P$-modules $\mathcal{F}$ and $p^*\mathcal{E}_{i'}$
for $i' \not = i$, we find a morphism $p' : P' \to P$ with
properties stated as in the lemma. Then the composition
$p \circ p' : P' \to X$ does the job.
\end{proof}

\begin{lemma}
\label{lemma-chern-classes-E-tensor-L}
Let $X \in \Ob(\mathcal{C})$. Let $\mathcal{E}$ be a finite locally free
$\mathcal{O}_X$-module. Let $\mathcal{L}$ be an invertible
$\mathcal{O}_X$-module. Then
$$
c^A_i({\mathcal E} \otimes {\mathcal L})
=
\sum\nolimits_{j = 0}^i
\binom{r - i + j}{j} c^A_{i - j}({\mathcal E}) \cup c^A_1({\mathcal L})^j
$$
\end{lemma}

\begin{proof}
By the construction of $c^A_i$ we may assume $\mathcal{E}$ has
constant rank $r$. Let $p : P \to X$ and $p' : P' \to X$ be the
projective bundle associated to $\mathcal{E}$ and
$\mathcal{E} \otimes \mathcal{L}$.
Then there is an isomorphism $g : P \to P'$ such that
$g^*\mathcal{O}_{P'}(1) = \mathcal{O}_P(1) \otimes p^*\mathcal{L}$.
See Constructions, Lemma \ref{constructions-lemma-twisting-and-proj}.
Thus
$$
g^*c_1^A(\mathcal{O}_{P'}(1)) =
c_1^A(\mathcal{O}_P(1)) + p^*c_1^A(\mathcal{L})
$$
The desired equality follows formally from this and the definition
of Chern classes using equation (\ref{equation-chern-classes}).
\end{proof}

\begin{proposition}
\label{proposition-chern-character}
In the situation above assume $A(X)$ is a $\mathbf{Q}$-algebra for all
$X \in \Ob(\mathcal{C})$. Then there is a unique rule which assigns to
every $X \in \Ob(\mathcal{C})$ a ``chern character''
$$
ch^A : K_0(\textit{Vect}(X)) \longrightarrow
\prod\nolimits_{i \geq 0} A^i(X)
$$
with the following properties
\begin{enumerate}
\item $ch^A$ is a ring map for all $X \in \Ob(\mathcal{C})$.
\item If $f : X' \to X$ is a morphism of $\mathcal{C}$, then
$f^* \circ ch^A =  ch^A \circ f^*$.
\item Given $X \in \Ob(\mathcal{C})$ and $\mathcal{L} \in \Pic(X)$
we have $ch^A([\mathcal{L}]) = \exp(c_1^A(\mathcal{L}))$.
\end{enumerate}
\end{proposition}

\begin{proof}
Let $X \in \Ob(\mathcal{C})$ and let $\mathcal{E}$ be a finite
locally free $\mathcal{O}_X$-module. We first show how to define
the rank $r^A(\mathcal{E}) \in A^0(X)$. Namely, let $X = \bigcup X_r$
be the decomposition into open and closed subschemes such that
$\mathcal{E}|_{X_r}$ has constant rank $r$. Since $X$ is quasi-compact, this
decomposition is finite, say $X = X_0 \amalg X_1 \amalg \ldots \amalg X_n$.
Then $A(X) = A(X_0) \times A(X_1) \times \ldots \times A(X_n)$. Thus we
can define $r^A(\mathcal{E}) = (0, 1, \ldots, n) \in A^0(X)$.

\medskip\noindent
Let $P_p(c_1, \ldots, c_p)$ be the polynomials constructed in
Chow Homology, Example \ref{chow-example-power-sum}.
Then we can define
$$
ch^A(\mathcal{E}) = r^A(\mathcal{E}) +
\sum\nolimits_{i \geq 1} (1/i!)
P_i(c^A_1(\mathcal{E}), \ldots, c^A_i(\mathcal{E}))
\in \prod\nolimits_{i \geq 0} A^i(X)
$$
where $ci^A$ are the Chern classes of
Proposition \ref{proposition-chern-class}.
It follows immediately that we have property (2) and (3) of the lemma.

\medskip\noindent
We still have to show the following three statements
\begin{enumerate}
\item If $0 \to \mathcal{E}_1 \to \mathcal{E} \to \mathcal{E}_2 \to 0$
is a short exact sequence of finite locally free $\mathcal{O}_X$-modules
on $X \in \Ob(\mathcal{C})$, then
$ch^A(\mathcal{E}) = ch^A(\mathcal{E}_1) + ch^A(\mathcal{E}_2)$.
\item If $\mathcal{E}_1$ and $\mathcal{E}_2 \to 0$ are finite locally free
$\mathcal{O}_X$-modules on $X \in \Ob(\mathcal{C})$, then
$ch^A(\mathcal{E}_1 \otimes \mathcal{E}_2) =
ch^A(\mathcal{E}_1) ch^A(\mathcal{E}_2)$.
\end{enumerate}
Namely, the first will prove that $ch^A$ factors through
$K_0(\textit{Vect}(X))$ and the first and the second will combined
show that $ch^A$ is a ring map.

\medskip\noindent
To prove these statements we can reduce to the case where $\mathcal{E}_1$
and $\mathcal{E}_2$ have constant ranks $r_1$ and $r_2$. In this case the
equalities in $A^0(X)$ are immediate. To prove the equalities in higher
degrees, by Lemma \ref{lemma-splitting-principle} we may
assume that $\mathcal{E}_1$ and $\mathcal{E}_2$ have filtrations
whose graded pieces are invertible modules
$\mathcal{L}_{1, j}$, $j = 1, \ldots, r_1$ and
$\mathcal{L}_{2, j}$, $j = 1, \ldots, r_2$.
Using the multiplicativity of Chern classes we get
$$
c_i^A(\mathcal{E}_1) =
s_i(c_1^A(\mathcal{L}_{1, 1}), \ldots, c_1^A(\mathcal{L}_{1, r_1}))
$$
where $s_i$ is the $i$th elementary symmetric function as in
Chow Homology, Example \ref{chow-example-power-sum}.
Similarly for $c_i^A(\mathcal{E}_2)$. In case (1) we get
$$
c_i^A(\mathcal{E}) =
s_i(c_1^A(\mathcal{L}_{1, 1}), \ldots, c_1^A(\mathcal{L}_{1, r_1}),
c_1^A(\mathcal{L}_{2, 1}), \ldots, c_1^A(\mathcal{L}_{2, r_2}))
$$
and for case (2) we get
$$
c_i^A(\mathcal{E}_1 \otimes \mathcal{E}_2) =
s_i(c_1^A(\mathcal{L}_{1, 1}) + c_1^A(\mathcal{L}_{2, 1}),
\ldots, c_1^A(\mathcal{L}_{1, r_1}) + c_1^A(\mathcal{L}_{2, r_2}))
$$
By the definition of the polynomials $P_i$ we see that this means
$$
P_i(c^A_1(\mathcal{E}_1), \ldots, c^A_i(\mathcal{E}_1)) =
\sum\nolimits_{j = 1, \ldots, r_1} c_1^A(\mathcal{L}_{1, j})^i
$$
and similarly for $\mathcal{E}_2$. In case (1) we have also
$$
P_i(c^A_1(\mathcal{E}), \ldots, c^A_i(\mathcal{E})) =
\sum\nolimits_{j = 1, \ldots, r_1} c_1^A(\mathcal{L}_{1, j})^i +
\sum\nolimits_{j = 1, \ldots, r_2} c_1^A(\mathcal{L}_{2, j})^i
$$
In case (2) we get accordingly
$$
P_i(c^A_1(\mathcal{E}_1 \otimes \mathcal{E}_2), \ldots,
c^A_i(\mathcal{E}_1 \otimes \mathcal{E}_2)) =
\sum\nolimits_{j = 1, \ldots, r_1}
\sum\nolimits_{j' = 1, \ldots, r_2}
(c_1^A(\mathcal{L}_{1, j}) + c_1^A(\mathcal{L}_{2, j'}))^i
$$
Thus the desired equalities are now consequences of elementary
identities between symmetric polynomials.

\medskip\noindent
We omit the proof of uniqueness.
\end{proof}

\begin{lemma}
\label{lemma-adams-and-chern}
In the situation above let $X \in \Ob(\mathcal{C})$.
If $\psi^2$ is as in
Chow Homology, Lemma \ref{chow-lemma-second-adams-operator}
and $c^A$ and $ch^A$ are as in
Propositions \ref{proposition-chern-class} and
\ref{proposition-chern-character}
then we have $c^A_i(\psi^2(\alpha)) = 2^i c^A_i(\alpha)$ and
$ch^A_i(\psi^2(\alpha)) = 2^i ch^A_i(\alpha)$
for all $\alpha \in K_0(\textit{Vect}(X))$.
\end{lemma}

\begin{proof}
Observe that the map $\prod_{i \geq 0} A^i(X) \to \prod_{i \geq 0} A^i(X)$
multiplying by $2^i$ on $A^i(X)$ is a ring map. Hence, since $\psi^2$
is also a ring map, it suffices to prove the formulas for additive generators
of $K_0(\textit{Vect}(X))$. Thus we may assume $\alpha = [\mathcal{E}]$
for some finite locally free $\mathcal{O}_X$-module $\mathcal{E}$.
By construction of the Chern classes of $\mathcal{E}$ we immediately
reduce to the case where $\mathcal{E}$ has constant rank $r$.
In this case, we can choose a projective smooth morphism $p : P \to X$
such that restriction $A^*(X) \to A^*(P)$ is injective
and such that $p^*\mathcal{E}$ has a finite filtration whose
graded parts are invertible $\mathcal{O}_P$-modules $\mathcal{L}_j$, see
Lemma \ref{lemma-splitting-principle}. Then
$[p^*\mathcal{E}] = \sum [\mathcal{L}_j]$ and hence
$\psi^2([p^*\mathcal{E}]) = \sum [\mathcal{L}_j^{\otimes 2}]$
by definition of $\psi^2$. Setting $x_j  = c^A_1(\mathcal{L}_j)$
we have
$$
c^A(\alpha) = \prod (1 + x_j)
\quad\text{and}\quad
c^A(\psi^2(\alpha)) = \prod (1 + 2 x_j)
$$
in $\prod A^i(P)$ and we have
$$
ch^A(\alpha) = \sum \exp(x_j)
\quad\text{and}\quad
ch^A(\psi^2(\alpha)) = \sum \exp(2 x_j)
$$
in $\prod A^i(P)$. From these formulas the desired result follows.
\end{proof}







\section{Exterior powers and K-groups}
\label{section-lambda-operations}

\noindent
We do the minimal amount of work to define the lambda operators.
Let $X$ be a scheme. Recall that $\textit{Vect}(X)$ denotes the
category of finite locally free $\mathcal{O}_X$-modules.
Moreover, recall that we have constructed a zeroth $K$-group
$K_0(\textit{Vect}(X))$ associated to this category in
Derived Categories of Schemes, Section \ref{perfect-section-K-groups}.
Finally, $K_0(\textit{Vect}(X))$ is a ring, see
Derived Categories of Schemes, Remark \ref{perfect-remark-K-ring}.

\begin{lemma}
\label{lemma-lambda-operations}
Let $X$ be a scheme. There are maps
$$
\lambda^r : K_0(\textit{Vect}(X)) \longrightarrow K_0(\textit{Vect}(X))
$$
which sends $[\mathcal{E}]$ to $[\wedge^r(\mathcal{E})]$
when $\mathcal{E}$ is a finite locally free $\mathcal{O}_X$-module
and which are compatible with pullbacks.
\end{lemma}

\begin{proof}
Consider the ring $R = K_0(\textit{Vect}(X))[[t]]$ where $t$ is a
variable. For a finite locally free $\mathcal{O}_X$-module
$\mathcal{E}$ we set
$$
c(\mathcal{E}) = \sum\nolimits_{i = 0}^\infty [\wedge^i(\mathcal{E})] t^i
$$
in $R$. We claim that given a short exact sequence
$$
0 \to \mathcal{E}' \to \mathcal{E} \to \mathcal{E}'' \to 0
$$
of finite locally free $\mathcal{O}_X$-modules
we have $c(\mathcal{E}) = c(\mathcal{E}') c(\mathcal{E}'')$.
The claim implies that $c$ extends to a map
$$
c :  K_0(\textit{Vect}(X)) \longrightarrow R
$$
which converts addition in $K_0(\textit{Vect}(X))$ to multiplication in $R$.
Writing $c(\alpha) = \sum \lambda^i(\alpha) t^i$ we obtain the desired
operators $\lambda^i$.

\medskip\noindent
To see the claim, we consider the short exact sequence as a
filtration on $\mathcal{E}$ with $2$ steps. We obtain an induced
filtration on $\wedge^r(\mathcal{E})$ with $r + 1$ steps and
subquotients
$$
\wedge^r(\mathcal{E}'),
\wedge^{r - 1}(\mathcal{E}') \otimes \mathcal{E}'',
\wedge^{r - 2}(\mathcal{E}') \otimes \wedge^2(\mathcal{E}''), \ldots
\wedge^r(\mathcal{E}'')
$$
Thus we see that $[\wedge^r(\mathcal{E})]$ is equal to
$$
\sum\nolimits_{i = 0}^r
[\wedge^{r - i}(\mathcal{E}')] [\wedge^i(\mathcal{E}'')]
$$
and the result follows easily from this and elementary algebra.
\end{proof}










\section{Weil cohomology theories, III}
\label{section-c1}

\noindent
Let $k$ be a field. Let $F$ be a field of characteristic zero.
Suppose we are given the following data
\begin{enumerate}
\item[(D0)] A $1$-dimensional $F$-vector space $F(1)$.
\item[(D1)] A contravariant functor $H^*(-)$ from the category of smooth
projective schemes over $k$ to the category of graded commutative
$F$-algebras.
\item[(D2')] For every smooth projective scheme $X$ over $k$ a homomorphism
$c_1^H : \Pic(X) \to H^2(X)(1)$ of abelian groups.
\end{enumerate}
We will use the terminology, notation, and conventions regarding
(D0) and (D1) as discussed in Section \ref{section-axioms}.
Given a smooth projective scheme $X$ over $k$ and an invertible
$\mathcal{O}_X$-module $\mathcal{L}$ the cohomology class
$c_1^H(\mathcal{L}) \in H^2(X)(1)$ of (D2')
is sometimes called the {\it first Chern class of $\mathcal{L}$
in cohomology}.

\medskip\noindent
Here is the list of axioms.
\begin{enumerate}
\item[(A1)] $H^*$ is compatible with finite coproducts
\item[(A2)] $c_1^H$ is compatible with pullbacks
\item[(A3)] Let $X$ be a smooth projective scheme over $k$.
Let $\mathcal{E}$ be a locally free $\mathcal{O}_X$-module of rank $r \geq 1$.
Consider the morphism $p : P = \mathbf{P}(\mathcal{E}) \to X$.
Then the map
$$
\bigoplus\nolimits_{i = 0, \ldots, r - 1} H^*(X)(-i)
\longrightarrow H^*(P),\quad
(a_0, \ldots, a_{r - 1}) \longmapsto
\sum c_1^H(\mathcal{O}_P(1))^i \cup p^*(a_i)
$$
is an isomorphism of $F$-vector spaces.
\item[(A4)] Let $i : Y \to X$ be the inclusion of an effective
Cartier divisor over $k$ with both $X$ and $Y$ smooth and projective
over $k$. For $a \in H^*(X)$ with
$i^*a = 0$ we have $a \cup c_1^H(\mathcal{O}_X(Y)) = 0$.
\item[(A5)] $H^*$ is compatible with finite products
\item[(A6)] Let $X$ be a nonempty smooth, projective scheme over $k$
equidimensional of dimension $d$. Then there exists an
$F$-linear map $\lambda : H^{2d}(X)(d) \to F$ such that
$(\text{id} \otimes \lambda) \gamma([\Delta]) =  1$ in $H^*(X)$.
\item[(A7)] If $b : X' \to X$ is the blowing up of a smooth
center in a smooth projective scheme $X$ over $k$\footnote{Then $X'$ is
smooth and projective over $k$ as well, see
More on Morphisms, Lemma \ref{more-morphisms-lemma-blowup}.}, then
$b^* : H^*(X) \to H^*(X')$ is injective.
\item[(A8)] If $X$ is a smooth projective scheme over $k$ and
$k' = \Gamma(X, \mathcal{O}_X)$, then the map $H^0(\Spec(k')) \to H^0(X)$
is an isomorphism.
\item[(A9)] Let $X$ be a nonempty smooth projective scheme over $k$
equidimensional of dimension $d$. Let $i : Y \to X$ be a nonempty
effective Cartier divisor smooth over $k$. For $a \in H^{2d - 2}(X)(d - 1)$
we have $\lambda_Y(i^*(a)) = \lambda_X(a \cup c_1^H(\mathcal{O}_X(Y))$ where
$\lambda_Y$ and $\lambda_X$ are as in axiom (A6) for $X$ and $Y$.
\end{enumerate}
Let us explain more precisely what we mean by each of these axioms.
Axioms (A3), (A4), and (A7) are clear as stated.

\medskip\noindent
Ad (A1). This means that $H^*(\emptyset) = 0$ and that
$(i^*, j^*) : H^*(X \amalg Y) \to H^*(X) \times H^*(Y)$
is an isomorphism where $i$ and $j$ are the coprojections.

\medskip\noindent
Ad (A2). This means that given a morphism $f : X \to Y$ of smooth projective
schemes over $k$ and an invertible $\mathcal{O}_Y$-module $\mathcal{N}$
we have $f^*c_1^H(\mathcal{L}) = c_1^H(f^*\mathcal{L})$.

\medskip\noindent
Ad (A5). This means that $H^*(\Spec(k)) = F$ and that for $X$ and $Y$ smooth
projective over $k$ the map $H^*(X) \otimes_F H^*(Y) \to H^*(X \times Y)$,
$a \otimes b \mapsto p^*(a) \cup q^*(b)$ is an isomorphism
where $p$ and $q$ are the projections.

\medskip\noindent
Ad (A6). Let $X$ be a nonempty smooth projective scheme over $k$
which is equidimensional of dimension $d$. By Lemma \ref{lemma-cycle-classes}
if we have axioms (A1) -- (A4) we can consider the class of the diagonal
$$
\gamma([\Delta]) \in
H^{2d}(X \times X)(d) = \bigoplus\nolimits_i H^i(X) \otimes_F H^{2d - i}(X)(d)
$$
where the tensor decomposition comes from axiom (A5).
Given an $F$-linear map $\lambda : H^{2d}(X)(d) \to F$ we may also view
$\lambda$ as an $F$-linear map $\lambda : H^*(X)(d) \to F$ by precomposing
with the projection onto $H^{2d}(X)(d)$. Having said this axiom (A6)
makes sense.

\medskip\noindent
Ad (A8). Let $X$ be a smooth projective scheme over $k$.
Then $k' = \Gamma(X, \mathcal{O}_X)$ is a finite separable
$k$-algebra (Varieties, Lemma
\ref{varieties-lemma-proper-geometrically-reduced-global-sections})
and hence $\Spec(k')$ is smooth and projective over $k$.
Thus we may apply $H^*$ to $\Spec(k')$ and axiom (A8) makes sense.

\medskip\noindent
Ad (A9). We will see in Remark \ref{remark-trace} that if we have
axioms (A1) -- (A7) then the map $\lambda$ of axiom (A6) is unique.

\begin{lemma}
\label{lemma-chern-classes}
Assume given (D0), (D1), and (D2') satisfying axioms (A1), (A2), (A3), and (A4).
There is a unique rule which assigns to every smooth projective $X$ over $k$
a ring homomorphism
$$
ch^H :
K_0(\textit{Vect}(X))
\longrightarrow
\prod\nolimits_{i \geq 0} H^{2i}(X)(i)
$$
compatible with pullbacks such that
$ch^H(\mathcal{L}) = \exp(c_1^H(\mathcal{L}))$
for any invertible $\mathcal{O}_X$-module $\mathcal{L}$.
\end{lemma}

\begin{proof}
Immediate from Proposition \ref{proposition-chern-character}
applied to the category of smooth projective schemes over $k$,
the functor $A : X \mapsto \bigoplus_{i \geq 0} H^{2i}(X)(i)$,
and the map $c_1^H$.
\end{proof}

\begin{lemma}
\label{lemma-cycle-classes}
Assume given (D0), (D1), and (D2') satisfying axioms (A1), (A2), (A3), and (A4).
There is a unique rule which assigns to every smooth projective $X$ over $k$
a graded ring homomorphism
$$
\gamma : \CH^*(X) \longrightarrow \bigoplus\nolimits_{i \geq 0} H^{2i}(X)(i)
$$
compatible with pullbacks such that $ch^H(\alpha) = \gamma(ch(\alpha))$
for $\alpha$ in $K_0(\textit{Vect}(X))$.
\end{lemma}

\begin{proof}
Recall that we have an isomorphism
$$
K_0(\textit{Vect}(X)) \otimes \mathbf{Q}
\longrightarrow \CH^*(X) \otimes \mathbf{Q},\quad
\alpha \longmapsto ch(\alpha) \cap [X]
$$
see Chow Homology, Lemma \ref{chow-lemma-K-tensor-Q}. It is an isomorphism
of rings by Chow Homology, Remark \ref{chow-remark-chern-character-K}.
We define $\gamma$ by the formula $\gamma(\alpha) = ch^H(\alpha')$
where $ch^H$ is as in Lemma \ref{lemma-chern-classes} and
$\alpha' \in K_0(\textit{Vect}(X))$ is such that
$ch(\alpha') \cap [X] = \alpha$ in $\CH^*(X) \otimes \mathbf{Q}$.

\medskip\noindent
The construction $\alpha \mapsto \gamma(\alpha)$ is compatible
with pullbacks because both $ch^H$ and taking Chern classes
is compatible with pullbacks, see
Lemma \ref{lemma-chern-classes} and
Chow Homology, Remark \ref{chow-remark-gysin-chern-classes}.

\medskip\noindent
We still have to see that $\gamma$ is graded.
Let $\psi^2 : K_0(\textit{Vect}(X)) \to K_0(\textit{Vect}(X))$
be the second Adams operator, see Chow Homology,
Lemma \ref{chow-lemma-second-adams-operator}.
If $\alpha \in \CH^i(X)$ and
$\alpha' \in K_0(\textit{Vect}(X)) \otimes \mathbf{Q}$
is the unique element with $ch(\alpha') \cap [X] = \alpha$,
then we have seen in
Chow Homology, Section \ref{chow-section-intersection-regular}
that $\psi^2(\alpha') = 2^i \alpha'$.
Hence we conclude that $ch^H(\alpha') \in H^{2i}(X)(i)$
by Lemma \ref{lemma-adams-and-chern} as desired.
\end{proof}

\begin{lemma}
\label{lemma-divide-pullback-good-blowing-up}
Let $b : X' \to X$ be the blowing up of a smooth projective
scheme over $k$ in a smooth closed subscheme $Z \subset X$.
Picture
$$
\xymatrix{
E \ar[r]_j \ar[d]_\pi & X' \ar[d]^b \\
Z \ar[r]^i & X
}
$$
Assume there exists an element of $K_0(X)$ whose restriction to
$Z$ is equal to the class of $\mathcal{C}_{Z/X}$ in $K_0(Z)$.
Assume every irreducible component of $Z$ has codimension $r$ in $X$.
Then there exists a cycle $\theta \in \CH^{r - 1}(X')$
such that $b^![Z] = [E] \cdot \theta$ in $\CH^r(X')$ and
$\pi_*j^!(\theta) = [Z]$ in $\CH^r(Z)$.
\end{lemma}

\begin{proof}
The scheme $X$ is smooth and projective over $k$ and hence we have
$K_0(X) = K_0(\textit{Vect}(X))$. See
Derived Categories of Schemes, Lemmas
\ref{perfect-lemma-resolution-property-ample} and
\ref{perfect-lemma-K-is-old-K}.
Let $\alpha \in K_0(\text{Vect}(X))$ be an element
whose restriction to $Z$ is $\mathcal{C}_{Z/X}$.
By Chow Homology, Lemma \ref{chow-lemma-minus-adams-operator}
there exists an element $\alpha^\vee$ which restricts to
$\mathcal{C}_{Z/X}^\vee$. By the blow up formula
(Chow Homology, Lemma \ref{chow-lemma-blow-up-formula})
we have
$$
b^![Z] = b^!i_*[Z] = j_* res(b^!)([Z]) =
j_*(c_{r - 1}(\mathcal{F}^\vee) \cap \pi^*[Z]) =
j_*(c_{r - 1}(\mathcal{F}^\vee) \cap [E])
$$
where $\mathcal{F}$ is the kernel of the surjection
$\pi^*\mathcal{C}_{Z/X} \to \mathcal{C}_{E/X'}$.
Observe that $b^*\alpha^\vee - [\mathcal{O}_{X'}(E)]$
is an element of $K_0(\text{Vect}(X'))$ which
restricts to $[\pi^*\mathcal{C}_{Z/X}^\vee] - [\mathcal{C}_{E/X'}^\vee] =
[\mathcal{F}^\vee]$ on $E$. Since capping with Chern classes
commutes with $j_*$ we conclude that the above is equal to
$$
c_{r - 1}(b^*\alpha^\vee - [\mathcal{O}_{X'}(E)]) \cap [E]
$$
in the chow group of $X'$. Hence we see that setting
$$
\theta = c_{r - 1}(b^*\alpha^\vee - [\mathcal{O}_{X'}(E)]) \cap [X']
$$
we get the first relation $\theta \cdot [E] = b^![Z]$
for example by Chow Homology, Lemma \ref{chow-lemma-identify-chow-for-smooth}.
For the second relation observe that
$$
j^!\theta = j^!(c_{r - 1}(b^*\alpha^\vee - [\mathcal{O}_{X'}(E)]) \cap [X'])
= c_{r - 1}(\mathcal{F}^\vee) \cap j^![X'] =
c_{r - 1}(\mathcal{F}^\vee) \cap [E]
$$
in the chow groups of $E$. To prove that $\pi_*$ of this is equal to $[Z]$ it
suffices to prove that the degree of the codimension $r - 1$ cycle
$(-1)^{r - 1}c_{r - 1}(\mathcal{F}) \cap [E]$ on the fibres of $\pi$ is $1$.
This is a computation we omit.
\end{proof}

\begin{lemma}
\label{lemma-A5-A6-imply}
Assume given data (D0), (D1), and (D2') satisfying axioms (A1) -- (A4)
and (A7). Let $X$ be a smooth projective scheme over $k$. Let $Z \subset X$
be a smooth closed subscheme such that every irreducible component of $Z$
has codimension $r$ in $X$. Assume the class of
$\mathcal{C}_{Z/X}$ in $K_0(Z)$ is the restriction of an element of $K_0(X)$.
If $a \in H^*(X)$ and $a|_Z = 0$ in $H^*(Z)$, then
$\gamma([Z]) \cup a = 0$.
\end{lemma}

\begin{proof}
Let $b : X' \to X$ be the blowing up. By (A7) it suffices to show
that
$$
b^*(\gamma([Z]) \cup a) = b^*\gamma([Z]) \cup b^*a = 0
$$
By Lemma \ref{lemma-divide-pullback-good-blowing-up} we have
$$
b^*\gamma([Z]) = \gamma(b^*[Z]) =
\gamma([E] \cdot \theta) =
\gamma([E]) \cup \gamma(\theta)
$$
Hence because $b^*a$ restricts to zero on $E$ and since
$\gamma([E]) = c^H_1(\mathcal{O}_{X'}(E))$ we get what we want from (A4).
\end{proof}

\begin{lemma}
\label{lemma-poincare-duality}
Assume given data (D0), (D1), and (D2') satisfying axioms (A1) -- (A7).
Then axiom (A) of Section \ref{section-axioms} holds with
$\int_X = \lambda$ as in axiom (A6).
\end{lemma}

\begin{proof}
Let $X$ be a nonempty smooth projective scheme over $k$ which is
equidimensional of dimension $d$. We will show that the graded $F$-vector space
$H^*(X)(d)[2d]$ is a left dual to $H^*(X)$. This will prove what we want by
Homology, Lemma \ref{homology-lemma-left-dual-graded-vector-spaces}. We are
going to use axiom (A5) which in particular says that
$$
H^*(X \times X)(d) =
\bigoplus H^i(X) \otimes H^j(X)(d) =
\bigoplus H^i(X)(d) \otimes H^j(X)
$$
Define a map
$$
\eta : F \longrightarrow H^*(X \times X)(d)
$$
by multiplying by $\gamma([\Delta]) \in H^{2d}(X \times X)(d)$.
On the other hand, define a map
$$
\epsilon :
H^*(X \times X)(d) \longrightarrow H^*(X)(d) \xrightarrow{\lambda} F
$$
by first using pullback $\Delta^*$ by the diagonal morphism
$\Delta : X \to X \times X$ and then using the $F$-linear map
$\lambda : H^{2d}(X)(d) \to F$ of axiom (A6) precomposed
by the projection $H^*(X)(d) \to H^{2d}(X)(d)$.
In order to show that $H^*(X)(d)$ is a left dual to $H^*(X)$
we have to show that the composition of the maps
$$
\eta \otimes 1 :
H^*(X) \longrightarrow H^*(X \times X \times X)(d)
$$
and
$$
1 \otimes \epsilon : H^*(X \times X \times X)(d) \longrightarrow H^*(X)
$$
is the identity. If $a \in H^*(X)$ then we see
that the composition maps $a$ to
$$
(1 \otimes \lambda)(\Delta_{23}^*(q_{12}^*\gamma([\Delta]) \cup q_3^*a)) =
(1 \otimes \lambda)(\gamma([\Delta]) \cup p_2^*a)
$$
where $q_i : X \times X \times X \to X$ and
$q_{ij} : X \times X \times X \to X \times X$ are the projections,
$\Delta_{23} : X \times X \to X \times X \times X$ is the diagonal, and
$p_i : X \times X \to X$ are the projections.
The equality holds because $\Delta_{23}^*(q_{12}^*\gamma([\Delta]) =
\Delta_{23}^*\gamma([\Delta \times X]) = \gamma([\Delta])$
and because $\Delta_{23}^* q_3^*a = p_2^*a$.
Since $\gamma([\Delta]) \cup p_1^*a = \gamma([\Delta]) \cup p_2^*a$
(see below) the above simplifies to
$$
(1 \otimes \lambda)(\gamma([\Delta]) \cup p_1^*a) = a
$$
by our choice of $\lambda$ as desired. The second condition
$(\epsilon \otimes 1) \circ (1 \otimes \eta) = \text{id}$
of Categories, Definition \ref{categories-definition-dual}
is proved in exactly the
same manner.

\medskip\noindent
Note that $p_1^*a$ and $\text{pr}_2^*a$ restrict to the same
cohomology class on $\Delta \subset X \times X$. Moreover we
have $\mathcal{C}_{\Delta/X \times X} = \Omega^1_\Delta$ which
is the restriction of $p_1^*\Omega^1_X$. Hence
Lemma \ref{lemma-A5-A6-imply} implies
$\gamma([\Delta]) \cup p_1^*a = \gamma([\Delta]) \cup p_2^*a$
and the proof is complete.
\end{proof}

\begin{remark}[Uniqueness of trace maps]
\label{remark-trace}
Assume given data (D0), (D1), and (D2') satisfying axioms (A1) -- (A7).
Let $X$ be a smooth projective scheme over $k$ which is nonempty
and equidimensional of dimension $d$. Combining what was said in
the proofs of Lemma \ref{lemma-poincare-duality} and
Homology, Lemma \ref{homology-lemma-left-dual-graded-vector-spaces}
we see that
$$
\gamma([\Delta]) \in \bigoplus\nolimits_i H^i(X) \otimes H^{2d - i}(X)(d)
$$
defines a perfect duality between $H^i(X)$ and $H^{2d - i}(X)(d)$
for all $i$.
In particular, the linear map $\int_X = \lambda : H^{2d}(X)(d) \to F$ of
axiom (A6) is unique! We will call the linear map $\int_X$ the trace map
of $X$ from now on.
\end{remark}

\begin{lemma}
\label{lemma-trace-product}
Assume given data (D0), (D1), and (D2') satisfying axioms (A1) -- (A7).
Then axiom (B) of Section \ref{section-axioms} holds.
\end{lemma}

\begin{proof}
Axiom (B)(a) is immediate from axiom (A5).
Let $X$ and $Y$ be nonempty smooth projective schemes over $k$
equidimensional of dimensions $d$ and $e$. To see that axiom (B)(b)
holds, observe that the diagonal $\Delta_{X \times Y}$ of $X \times Y$
is the intersection product of the pullbacks of the diagonals
$\Delta_X$ of $X$ and $\Delta_Y$ of $Y$ by the projections
$p : X \times Y \times X \times Y \to X \times X$ and
$q : X \times Y \times X \times Y \to Y \times Y$.
Compatibility of $\gamma$ with intersection products then gives
that
$$
\gamma([\Delta_{X \times Y}]) \in
H^{2d + 2e}(X \times Y \times X \times Y)(d + e)
$$
is the cup product of the pullbacks of $\gamma([\Delta_X])$
and $\gamma([\Delta_Y])$ by $p$ and $q$. Write
$$
\gamma([\Delta_{X \times Y}]) = \sum \eta_{X \times Y, i}
\text{ with }
\eta_{X \times Y, i} \in
H^i(X \times Y) \otimes H^{2d + 2e - i}(X \times Y)(d + e)
$$
and simiarly $\gamma([\Delta_X]) = \sum \eta_{X, i}$ and
$\gamma([\Delta_Y]) = \sum \eta_{Y, i}$. The observation above
implies we have
$$
\eta_{X \times Y, 0} =
\sum\nolimits_{i \in \mathbf{Z}} p^*\eta_{X, i} \cup q^*\eta_{Y, -i}
$$
(If our cohomology theory vanishes in negative degrees, which will
be true in almost all cases, then only the term for $i = 0$ contributes
and $\eta_{X \times Y, 0}$ lies in
$H^0(X) \otimes H^0(Y) \otimes H^{2d}(X)(d) \otimes H^{2e}(Y)(e)$ as expected,
but we don't need this.) Since $\lambda_X : H^{2d}(X)(d) \to F$ and 
$\lambda_Y : H^{2e}(Y)(e) \to F$ send $\eta_{X, 0}$ and $\eta_{Y, 0}$
to $1$ in $H^*(X)$ and $H^*(Y)$, we see that $\lambda_X \otimes \lambda_Y$
sends $\eta_{X \times Y, 0}$ to $1$ in
$H^*(X) \otimes H^*(Y) = H^*(X \times Y)$ and the proof is complete.
\end{proof}

\begin{lemma}
\label{lemma-trace-base}
Assume given data (D0), (D1), and (D2') satisfying axioms (A1) -- (A7).
Then axiom (C)(d) of Section \ref{section-axioms} holds.
\end{lemma}

\begin{proof}
We have $\gamma([\Spec(k)]) = 1 \in H^*(\Spec(k))$ by construction.
Since
$$
H^0(\Spec(k)) = F,\quad
H^0(\Spec(k) \times \Spec(k)) = H^0(\Spec(k)) \otimes_F H^0(\Spec(k))
$$
the map $\int_{\Spec(k)} = \lambda$ of axiom (A6) must send $1$ to $1$
because we have seen that
$\int_{\Spec(k) \times \Spec(k)} = \int_{\Spec(k)} \int_{\Spec(k)}$
in Lemma \ref{lemma-trace-product}.
\end{proof}

\noindent
Assume given data (D0), (D1), and (D2') satisfying axioms (A1) -- (A7).
Then we obtain data (D0), (D1), (D2), and (D3) of
Section \ref{section-axioms}
satisfying axioms (A), (B) and (C)(a), (C)(c), and (C)(d)
of Section \ref{section-axioms}, see
Lemmas \ref{lemma-poincare-duality}, \ref{lemma-trace-product}, and
\ref{lemma-trace-base}.
Moreover, we have the pushforwards $f_* : H^*(X) \to H^*(Y)$
as constructed in Section \ref{section-axioms}. The only axiom of
Section \ref{section-axioms}
which isn't clear yet is axiom (C)(b).

\begin{lemma}
\label{lemma-ok-for-projective-bundle}
Assume given data (D0), (D1), and (D2') satisfying axioms (A1) -- (A7).
Let $p : P \to X$ be as in axiom (A3) with $X$ nonempty equidimensional.
Then $\gamma$ commutes with pushforward along $p$.
\end{lemma}

\begin{proof}
It suffices to prove this on generators for $\CH_*(P)$.
Thus it suffices to prove this for a cycle class of the
form $\xi^i \cdot p^*\alpha$ where $0 \leq i \leq r - 1$
and $\alpha \in \CH_*(X)$. Note that $p_*(\xi^i \cdot p^*\alpha) = 0$
if $i < r - 1$ and $p_*(\xi^{r - 1} \cdot p^*\alpha) = \alpha$.
On the other hand, we have
$\gamma(\xi^i \cdot p^*\alpha) = c^i \cup p^*\gamma(\alpha)$
and by the projection formula (Lemma \ref{lemma-pushforward})
we have
$$
p_*\gamma(\xi^i \cdot p^*\alpha) = p_*(c^i) \cup \gamma(\alpha)
$$
Thus it suffices to show that $p_*c^i = 0$ for $i < r - 1$ and
$p_*c^{r - 1} = 1$. Equivalently, it suffices to prove that
$\lambda_P : H^{2d + 2r - 2}(P)(d + r - 1) \to F$ defined by
the rules
$$
\lambda_P(c^i \cup p^*(a)) =
\left\{
\begin{matrix}
0 & \text{if} & i < r - 1 \\
\int_X(a) & \text{if} & i = r - 1
\end{matrix}
\right.
$$
satisfies the condition of axiom (A5). This follows from the
computation of the class of the diagonal of $P$ in
Lemma \ref{lemma-diagonal-projective-bundle}.
\end{proof}

\begin{lemma}
\label{lemma-integrate-1}
Assume given data (D0), (D1), and (D2') satisfying axioms (A1) -- (A7).
If $k'/k$ is a Galois extension, then we have
$\int_{\Spec(k')} 1 = [k' : k]$.
\end{lemma}

\begin{proof}
We observe that
$$
\Spec(k') \times \Spec(k') =
\coprod\nolimits_{\sigma \in \text{Gal}(k'/k)}
(\Spec(\sigma) \times \text{id})^{-1} \Delta
$$
as cycles on $\Spec(k') \times \Spec(k')$.
Our construction of $\gamma$ always sends $[X]$ to $1$ in $H^0(X)$. Thus
$1 \otimes 1 = 1 = \sum (\Spec(\sigma) \times \text{id})^*\gamma([\Delta])$.
Denote $\lambda : H^0(\Spec(k')) \to F$ the map from
axiom (A6), in other words $(\text{id} \otimes \lambda)(\gamma(\Delta)) = 1$
in $H^0(\Spec(k'))$. We obtain
\begin{align*}
\lambda(1) 1
& =
(\text{id} \otimes \lambda)(1 \otimes 1) \\
& =
(\text{id} \otimes \lambda)(
\sum (\Spec(\sigma) \times \text{id})^*\gamma([\Delta])) \\
& =
\sum (\Spec(\sigma) \times \text{id})^*(
(\text{id} \otimes \lambda)(\gamma([\Delta])) \\
& =
\sum (\Spec(\sigma) \times \text{id})^*(1) \\
& =
[k' : k]
\end{align*}
Since $\lambda$ is another name for $\int_{\Spec(k')}$
(Remark \ref{remark-trace}) the proof is complete.
\end{proof}

\begin{lemma}
\label{lemma-enough}
Assume given data (D0), (D1), and (D2') satisfying axioms (A1) -- (A7).
In order to show that $\gamma$ commutes with pushforward it suffices
to show that $i_*(1) = \gamma([Z])$ if $i : Z \to X$ is a closed
immersion of nonempty smooth projective equidimensional schemes over $k$.
\end{lemma}

\begin{proof}
We will use without further mention that we've constructed our
cycle class map $\gamma$ in Lemma \ref{lemma-cycle-classes}
compatible with intersection products and pullbacks and that
we've already shown axioms
(A), (B), (C)(a), (C)(c), and (C)(d) of Section \ref{section-axioms}, see
Lemma \ref{lemma-poincare-duality},
Remark \ref{remark-trace}, and
Lemmas \ref{lemma-trace-product} and \ref{lemma-trace-base}.
In particular, we may use (for example) Lemma \ref{lemma-pushforward}
to see that pushforward on $H^*$ is compatible with composition
and satisfies the projection formula.

\medskip\noindent
Let $f : X \to Y$ be a morphism of nonempty
equidimensional smooth projective schemes over $k$.
We are trying to show $f_*\gamma(\alpha) = \gamma(f_*\alpha)$
for any cycle class $\alpha$ on $X$.
We can write $\alpha$ as a $\mathbf{Q}$-linear combination of products of
Chern classes of locally free $\mathcal{O}_X$-modules
(Chow Homology, Lemma \ref{chow-lemma-K-tensor-Q}).
Thus we may assume $\alpha$ is a product of Chern classes of
finite locally free $\mathcal{O}_X$-modules
$\mathcal{E}_1, \ldots, \mathcal{E}_r$.
Pick $p : P \to X$ as in the splitting principle
(Chow Homology, Lemma \ref{chow-lemma-splitting-principle}).
By Chow Homology, Remark \ref{chow-remark-the-proof-shows-more}
we see that $p$ is a composition of projective space bundles and
that $\alpha = p_*(\xi_1 \cap \ldots \cap \xi_d \cap \cdot p^*\alpha)$
where $\xi_i$ are first Chern classes of invertible modules.
By Lemma \ref{lemma-ok-for-projective-bundle}
we know that $p_*$ commutes with cycle classes.
Thus it suffices to prove the property for the composition
$f \circ p$. Since $p^*\mathcal{E}_1, \ldots, p^*\mathcal{E}_r$
have filtrations whose successive quotients are invertible
modules, this reduces us to the case where $\alpha$ is
of the form $\xi_1 \cap \ldots \cap \xi_t \cap [X]$
for some first Chern classes $\xi_i$ of invertible modules $\mathcal{L}_i$.

\medskip\noindent
Assume
$\alpha = c_1(\mathcal{L}_1) \cap \ldots \cap c_1(\mathcal{L}_t) \cap [X]$
for some invertible modules $\mathcal{L}_i$ on $X$.
Let $\mathcal{L}$ be an ample invertible $\mathcal{O}_X$-module.
For $n \gg 0$ the invertible $\mathcal{O}_X$-modules
$\mathcal{L}^{\otimes n}$ and
$\mathcal{L}_1 \otimes \mathcal{L}^{\otimes n}$ are both
very ample on $X$ over $k$, see
Morphisms, Lemma \ref{morphisms-lemma-invertible-add-enough-ample-very-ample}.
Since $c_1(\mathcal{L}_1) = c_1(\mathcal{L}_1 \otimes \mathcal{L}^{\otimes n})
- c_1(\mathcal{L}^{\otimes n})$ this reduces us to the case where
$\mathcal{L}_1$ is very ample. Repeating this with $\mathcal{L}_i$
for $i = 2, \ldots, t$ we reduce to the case where $\mathcal{L}_i$
is very ample on $X$ over $k$ for all $i = 1, \ldots, t$.

\medskip\noindent
Assume $k$ is infinite and $\alpha = 
c_1(\mathcal{L}_1) \cap \ldots \cap c_1(\mathcal{L}_t) \cap [X]$
for some very ample invertible modules $\mathcal{L}_i$ on $X$ over $k$.
By Bertini in the form of Varieties, Lemma \ref{varieties-lemma-bertini}
we can successively find regular sections $s_i$ of $\mathcal{L}_i$
such that the schemes $Z(s_1) \cap \ldots \cap Z(s_i)$
are smooth over $k$ and of codimension $i$ in $X$.
By the construction of capping with the first class of
an invertible module (going back to
Chow Homology, Definition \ref{chow-definition-divisor-invertible-sheaf}),
this reduces us to the case where $\alpha = [Z]$
for some nonempty smooth closed subscheme $Z \subset X$ which
is equidimensional.

\medskip\noindent
Assume $\alpha = [Z]$ where $Z \subset X$ is a smooth closed subscheme.
Choose a closed embedding $X \to \mathbf{P}^n$. We can factor $f$ as
$$
X \to Y \times \mathbf{P}^n \to Y
$$
Since we know the result for the second morphism by
Lemma \ref{lemma-ok-for-projective-bundle}
it suffices to prove the result when
$\alpha = [Z]$ where $i : Z \to X$ is a closed immersion 
and $f$ is a closed immersion.
Then $j = f \circ i$ is a closed embedding too.
Using the hypothesis for $i$ and $j$ we win.

\medskip\noindent
We still have to prove the lemma in case $k$ is finite. We urge the
reader to skip the rest of the proof. Everything we said above continues
to work, except that we do not know we can choose the sections
$s_i$ cutting out our $Z$ over $k$ as $k$ is finite. However, we do
know that we can find $s_i$ over the algebraic closure $\overline{k}$
of $k$ (by the same lemma). This means that we can
find a finite extension $k'/k$ such that our sections $s_i$
are defined over $k'$. Denote $\pi : X_{k'} \to X$ the projection.
The arguments above shows that we get the desired conclusion
(from the assumption in the lemma)
for the cycle $\pi^*\alpha$ and the morphism
$f \circ \pi : X_{k'} \to Y$.
We have $\pi_*\pi^*\alpha = [k' : k] \alpha$, see
Chow Homology, Lemma \ref{chow-lemma-finite-flat}.
On the other hand, we have
$$
\pi_*\gamma(\pi^*\alpha) = \pi_*\pi^*\gamma(\alpha) =
\gamma(\alpha) \pi_*1
$$
by the projection formula for our cohomology theory. Observe
that $\pi$ is a projection (!) and hence we have
$\pi_*(1) = \int_{\Spec(k')}(1) 1$ by
Lemma \ref{lemma-pr2star}. Thus to finish the proof in the
finite field case, it suffices to prove that
$\int_{\Spec(k')}(1) = [k' : k]$ which we do in
Lemma \ref{lemma-integrate-1}.
\end{proof}

\noindent
In the lemmas below we use the Grassmannians defined and constructed
in Constructions, Section \ref{constructions-section-grassmannian}.

\begin{lemma}
\label{lemma-grassmanian}
Assume given data (D0), (D1), and (D2') satisfying axioms (A1) -- (A7).
Given integers $0 < l < n$ and a nonempty equidimensional
smooth projective scheme $X$ over $k$ consider the projection morphism
$p : X \times \mathbf{G}(l, n) \to X$.
Then $\gamma$ commutes with pushforward along $p$.
\end{lemma}

\begin{proof}
If $l = 1$ or $l = n - 1$ then $p$ is a projective bundle and
the result follows from Lemma \ref{lemma-ok-for-projective-bundle}.
In general there exists a morphism
$$
h : Y \to X \times \mathbf{G}(l, n)
$$
such that both $h$ and $p \circ h$ are compositions of projective
space bundles. Namely, denote $\mathbf{G}(1, 2, \ldots, l; n)$
the partial flag variety. Then the morphism
$$
\mathbf{G}(1, 2, \ldots, l; n) \to \mathbf{G}(l, n)
$$
is a compostion of projective space bundles and similarly the
structure morphism $\mathbf{G}(1, 2, \ldots, l; n) \to \Spec(k)$
is of this form. Thus we may set $Y = X \times \mathbf{G}(1, 2, \ldots, l; n)$.
Since every cycle on $X \times \mathbf{G}(l, n)$ is the pushforward of
a cycle on $Y$, the result for $Y \to X$ and the result for
$Y \to X \times \mathbf{G}(l, n)$ imply the result for $p$.
\end{proof}

\begin{lemma}
\label{lemma-enough-better}
Assume given data (D0), (D1), and (D2') satisfying axioms (A1) -- (A7).
In order to show that $\gamma$ commutes with pushforward it suffices
to show that $i_*(1) = \gamma([Z])$ if $i : Z \to X$ is a closed
immersion of nonempty smooth projective equidimensional schemes over $k$
such that the class of $\mathcal{C}_{Z/X}$ in $K_0(Z)$ is the
pullback of a class in $K_0(X)$.
\end{lemma}

\begin{proof}
By Lemma \ref{lemma-enough} it suffices to show that $i_*(1) = \gamma([Z])$
if $i : Z \to X$ is a closed immersion of nonempty
smooth projective equidimensional
schemes over $k$. Say $Z$ has codimension $r$ in $X$.
Let $\mathcal{L}$ be a sufficiently ample invertible module on $X$.
Choose $n > 0$ and a surjection
$$
\mathcal{O}_Z^{\oplus n} \to \mathcal{C}_{Z/X} \otimes \mathcal{L}|_Z
$$
This gives a morphism $g : Z \to \mathbf{G}(n - r, n)$
to the Grassmannian over $k$, see
Constructions, Section \ref{constructions-section-grassmannian}.
Consider the composition
$$
Z \to X \times \mathbf{G}(n - r, n) \to X
$$
Pushforward along the second morphism is compatible with classes
of cycles by Lemma \ref{lemma-grassmanian}. The conormal sheaf $\mathcal{C}$
of the closed immersion $Z \to X \times \mathbf{G}(n - r, n)$ sits in
a short exact sequence
$$
0 \to \mathcal{C}_{Z/X} \to \mathcal{C} \to
g^*\Omega_{\mathbf{G}(n - r, n)} \to 0
$$
See More on Morphisms, Lemma
\ref{more-morphisms-lemma-two-unramified-morphisms-formally-smooth}.
Since $\mathcal{C}_{Z/X} \otimes \mathcal{L}|_Z$ is the pull
back of a finite locally free sheaf on $\mathbf{G}(n - r, n)$
we conclude that the class of $\mathcal{C}$ in $K_0(Z)$
is the pullback of a class in $K_0(X \times \mathbf{G}(n - r, n))$.
Hence we have the property for $Z \to X \times \mathbf{G}(n - r, n)$
and we conclude.
\end{proof}

\begin{lemma}
\label{lemma-injective-H0}
Assume given data (D0), (D1), and (D2') satisfying axioms (A1) -- (A7).
If $k''/k'/k$ are finite separable field extensions, then
$H^0(\Spec(k')) \to H^0(\Spec(k''))$ is injective.
\end{lemma}

\begin{proof}
We may replace $k''$ by its normal closure over $k$
which is Galois over $k$, see
Fields, Lemma \ref{fields-lemma-normal-closure-galois}.
Then $k''$ is Galois over $k'$ as well, see
Fields, Lemma \ref{fields-lemma-galois-goes-up}.
We deduce we have an isomorphism
$$
k' \otimes_k k'' \longrightarrow
\prod\nolimits_{\sigma \in \text{Gal}(k''/k')} k'',\quad
\eta \otimes \zeta \longmapsto (\eta \sigma(\zeta))_\sigma
$$
This produces an isomorphism
$\coprod_\sigma \Spec(k'') \to \Spec(k') \times \Spec(k'')$
which on cohomology produces the isomorphism
$$
H^*(\Spec(k')) \otimes_F H^*(\Spec(k''))
\to
\prod\nolimits_\sigma H^*(\Spec(k'')),\quad
a' \otimes a'' \longmapsto (\pi^*a' \cup \Spec(\sigma)^*a'')_\sigma
$$
where $\pi : \Spec(k'') \to \Spec(k')$ is the morphism
corresponding to the inclusion of $k'$ in $k''$.
We conclude the lemma is true by taking $a'' = 1$.
\end{proof}

\begin{lemma}
\label{lemma-pushforward-blowup}
Assume given data (D0), (D1), and (D2') satisfying axioms (A1) -- (A8).
Let $b : X' \to X$ be a blowing up of a smooth projective scheme $X$
over $k$ which is nonempty equidimensional of dimension $d$
in a nonwhere dense smooth center $Z$. Then $b_*(1) = 1$.
\end{lemma}

\begin{proof}
We may replace $X$ by a connected component of $X$ (some details
omitted). Thus we may assume $X$ is connected and hence irreducible.
Set $k' = \Gamma(X, \mathcal{O}_X) = \Gamma(X', \mathcal{O}_{X'})$;
we omit the proof of the equality. Choose a closed point $x' \in X'$
which isn't contained in the exceptional divisor and whose residue field
$k''$ is separable over $k$; this is possible by
Varieties, Lemma \ref{varieties-lemma-smooth-separable-closed-points-dense}.
Denote $x \in X$ the image (whose residue field is equal to $k''$
as well of course). Consider the diagram
$$
\xymatrix{
x' \times X' \ar[r] \ar[d] & X' \times X' \ar[d] \\
x \times X \ar[r] & X \times X
}
$$
The class of the diagonal $\Delta = \Delta_X$ pulls back to the class of the
``diagonal point'' $\delta_x : x \to x \times X$ and similarly for the class of
the diagonal $\Delta'$. On the other hand, the diagonal point $\delta_x$
pulls back to the diagonal point $\delta_{x'}$ by the left vertical arrow.
Write $\gamma([\Delta]) = \sum \eta_i$ with
$\eta_i \in H^i(X) \otimes H^{2d - i}(X)(d)$ and
$\gamma([\Delta']) = \sum \eta'_i$ with
$\eta'_i \in H^i(X') \otimes H^{2d - i}(X')(d)$.
The arguments above show that $\eta_0$ and $\eta'_0$ map to the same
class in
$$
H^0(x') \otimes_F H^{2d}(X')(d)
$$
We have $H^0(\Spec(k')) = H^0(X) = H^0(X')$ by axiom (A8).
By Lemma \ref{lemma-injective-H0} this common value maps injectively
into $H^0(x')$. We conclude that $\eta_0$ maps to $\eta'_0$ by the map
$$
H^0(X) \otimes_F H^{2d}(X)(d)
\longrightarrow
H^0(X') \otimes_F H^{2d}(X')(d)
$$
This means that $\int_X$ is equal to $\int_{X'}$ composed with
the pullback map. This proves the lemma.
\end{proof}

\begin{lemma}
\label{lemma-done}
Assume given data (D0), (D1), and (D2') satisfying axioms (A1) -- (A8).
Then the cycle class map $\gamma$ commutes with pushforward.
\end{lemma}

\begin{proof}
Let $i : Z \to X$ be as in Lemma \ref{lemma-enough-better}. Consider
the diagram
$$
\xymatrix{
E \ar[r]_j \ar[d]_\pi & X' \ar[d]^b \\
Z \ar[r]^i & X
}
$$
Let $\theta \in \CH^{r - 1}(X')$ be as in
Lemma \ref{lemma-divide-pullback-good-blowing-up}.
Then $\pi_*j^!\theta = [Z]$ in $\CH_*(Z)$ implies that
$\pi_*\gamma(j^!\theta) = 1$ by Lemma \ref{lemma-ok-for-projective-bundle}
because $\pi$ is a projective space bundle.
Hence we see that
$$
i_*(1) = i_*(\pi_*(\gamma(j^!\theta))) =
b_*j_*(j^*\gamma(\theta)) =
b_*(j_*(1) \cup \gamma(\theta))
$$
We have $j_*(1) = \gamma([E])$ by (A9). Thus this is equal to
$$
b_*(\gamma([E]) \cup \gamma(\theta)) =
b_*(\gamma([E] \cdot \theta)) =
b_*(\gamma(b^*[Z])) =
b_*b^*\gamma([Z]) = b_*(1) \cup \gamma([Z])
$$
Since $b_*(1) = 1$ by Lemma \ref{lemma-pushforward-blowup} the
proof is complete.
\end{proof}

\begin{proposition}
\label{proposition-get-weil}
Assume given data (D0), (D1), and (D2') satisfying axioms (A1) -- (A8).
Then we have a Weil cohomology theory.
\end{proposition}

\begin{proof}
We have axioms (A), (B) and (C)(a), (C)(c), and (C)(d) of
Section \ref{section-axioms} by
Lemmas \ref{lemma-poincare-duality}, \ref{lemma-trace-product}, and
\ref{lemma-trace-base}.
We have axiom (C)(b) by
Lemma \ref{lemma-done}.
Finally, the additional condition of
Definition \ref{definition-weil-cohomology-theory}
holds because it is the same as our axiom (A8).
\end{proof}

\noindent
The following lemma is sometimes useful to show that we get a
Weil cohomology theory over a nonclosed field by reducing to a
closed one.

\begin{lemma}
\label{lemma-check-over-extension}
Let $k'/k$ be an extension of fields. Let $F'/F$ be an extension
of fields of characteristic $0$. Assume given
\begin{enumerate}
\item data (D0), (D1), (D2') for $k$ and $F$ denoted
$F(1), H^*, c_1^H$,
\item data (D0), (D1), (D2') for $k'$ and $F'$ denoted
$F'(1), (H')^*, c_1^{H'}$, and
\item an isomorphism $F(1) \otimes_F F' \to F'(1)$, functorial isomorphisms
$H^*(X) \otimes_F F' \to (H')^*(X_{k'})$ on the category of smooth projective
schemes $X$ over $k$ such that the diagrams
$$
\xymatrix{
\Pic(X) \ar[r]_{c_1^H} \ar[d] & H^2(X)(1) \ar[d] \\
\Pic(X_{k'}) \ar[r]^{c_1^{H'}} & (H')^2(X_{k'})(1)
}
$$
commute.
\end{enumerate}
In this case, if $F'(1), (H')^*, c_1^{H'}$ satisfy axioms (A1) -- (A9),
then the same is true for $F(1), H^*, c_1^H$.
\end{lemma}

\begin{proof}
We go by the axioms one by one.

\medskip\noindent
Axiom (A1). We have to show $H^*(\emptyset) = 0$ and that
$(i^*, j^*) : H^*(X \amalg Y) \to H^*(X) \times H^*(Y)$
is an isomorphism where $i$ and $j$ are the coprojections.
By the functorial nature of the isomorphisms
$H^*(X) \otimes_F F' \to (H')^*(X_{k'})$ this
follows from linear algebra: if $\varphi : V \to W$ is an $F$-linear map
of $F$-vector spaces, then $\varphi$ is an isomorphism if and only if
$\varphi_{F'} : V \otimes_F F' \to W \otimes_F F'$ is an isomorphism.

\medskip\noindent
Axiom (A2). This means that given a morphism $f : X \to Y$ of smooth projective
schemes over $k$ and an invertible $\mathcal{O}_Y$-module $\mathcal{N}$
we have $f^*c_1^H(\mathcal{L}) = c_1^H(f^*\mathcal{L})$. This is immediately
clear from the corresponding property for $c_1^{H'}$, the commutative
diagrams in the lemma, and the fact that the canonical map
$V \to V \otimes_F F'$ is injective for any $F$-vector space $V$.

\medskip\noindent
Axiom (A3). This follows from the principle stated in the proof of
axiom (A1) and compatibility of $c_1^H$ and $c_1^{H'}$.

\medskip\noindent
Axiom (A4). Let $i : Y \to X$ be the inclusion of an effective
Cartier divisor over $k$ with both $X$ and $Y$ smooth and projective
over $k$. For $a \in H^*(X)$ with
$i^*a = 0$ we have to show $a \cup c_1^H(\mathcal{O}_X(Y)) = 0$.
Denote $a' \in (H')^*(X_{k'})$ the image of $a$.
The assumption implies that $(i')^*a' = 0$ where $i' : Y_{k'} \to X_{k'}$
is the base change of $i$. Hence we get
$a' \cup c_1^{H'}(\mathcal{O}_{X_{k'}}(Y_{k'})) = 0$ by the axiom
for $(H')^*$. Since $a' \cup c_1^{H'}(\mathcal{O}_{X_{k'}}(Y_{k'}))$
is the image of $a \cup c_1^H(\mathcal{O}_X(Y))$ we conclude by
the princple stated in the proof of axiom (A2).

\medskip\noindent
Axiom (A5). This means that $H^*(\Spec(k)) = F$ and that for $X$ and $Y$ smooth
projective over $k$ the map $H^*(X) \otimes_F H^*(Y) \to H^*(X \times Y)$,
$a \otimes b \mapsto p^*(a) \cup q^*(b)$ is an isomorphism
where $p$ and $q$ are the projections. This follows from the principle
stated in the proof of axiom (A1).

\medskip\noindent
We interrupt the flow of the arguments to show that for every
smooth projective scheme $X$ over $k$ the diagram
$$
\xymatrix{
\CH^*(X) \ar[r]_-\gamma \ar[d]_{g^*} & \bigoplus H^{2i}(X)(i) \ar[d] \\
\CH^*(X_{k'}) \ar[r]^-{\gamma'} & \bigoplus (H')^{2i}(X_{k'})(i)
}
$$
commutes. Observe that we have $\gamma$ as we know axioms
(A1) -- (A4) already; see Lemma \ref{lemma-cycle-classes}.
Also, the left vertical arrow is the one discussed in
Chow Homology, Section \ref{chow-section-change-base}
for the morphism of base schemes $g : \Spec(k') \to \Spec(k)$.
More precisely, it is the map given in
Chow Homology, Lemma \ref{chow-lemma-pullback-base-change}.
Pick $\alpha \in \CH^*(X)$. Write $\alpha = ch(\beta) \cap [X]$
in $\CH^*(X) \otimes \mathbf{Q}$
for some $\beta \in K_0(\textit{Vect}(X)) \otimes \mathbf{Q}$
so that $\gamma(\alpha) = ch^{H}(\beta)$; this is our construction of $\gamma$.
Since the map of Chow Homology, Lemma \ref{chow-lemma-pullback-base-change}
is compatible with capping with Chern classes by
Chow Homology, Lemma \ref{chow-lemma-pullback-base-change-chern-classes}
we see that $g^*\alpha = ch((X_{k'} \to X)^*\beta) \cap [X_{k'}]$.
Hence $\gamma'(g^*\alpha) = ch^{H'}((X_{k'} \to X)^*\beta)$.
Thus commutativity of the diagram will hold if for any locally
free $\mathcal{O}_X$-module $\mathcal{E}$ of rank $r$ and $0 \leq i \leq r$
the element $c_i^H(\mathcal{E})$ of $H^{2i}(X)(i)$
maps to the element $c_i^{H'}(\mathcal{E}_{k'})$ in $(H')^{2i}(X_{k'})(i)$.
Because we have the projective space bundle formula for both
$X$ and $X'$ we may replace $X$ by a projective space bundle
over $X$ finitely many times to show this. Thus we may assume
$\mathcal{E}$ has a filtration whose graded pieces are
invertible $\mathcal{O}_X$-modules
$\mathcal{L}_1, \ldots, \mathcal{L}_r$.
See Chow Homology, Lemma \ref{chow-lemma-splitting-principle} and
Remark \ref{chow-remark-the-proof-shows-more}.
Then $c^H_i(\mathcal{E}$ is the $i$th elementary symmetric polynomial
in $c^H_1(\mathcal{L}_1), \ldots, c^H_1(\mathcal{L}_r)$
and we conclude by our assumption that we have agreement for
first Chern classes.

\medskip\noindent
Axiom (A6). Suppose given $F$-vector spaces
$V$, $W$, an element $v \in V$, and a tensor $\xi \in V \otimes_F W$.
Denote $V' = V \otimes_F F'$, $W' = W \otimes_F F'$ and $v'$, $\xi'$
the images of $v$, $\xi$ in $V'$, $V' \otimes_{F'} W'$. The linear algebra
principle we will use in the proof of axiom (A6) is the following:
there exists an $F$-linear map $\lambda : W \to F$ such that
$(1 \otimes \lambda)\xi = v$ if and only if there exists an $F'$-linear
map $\lambda' : W \otimes_F F' \to F'$ such that
$(1 \otimes \lambda')\xi' = v'$.

\medskip\noindent
Let $X$ be a nonempty equidimensional smooth projective scheme
over $k$ of dimension $d$. Denote $\gamma = \gamma([\Delta])$
in $H^{2d}(X \times X)(d)$ (unadorned fibre products will be over $k$).
Observe/recall that this makes sense as we know axioms (A1) -- (A4) already;
see Lemma \ref{lemma-cycle-classes}. We may decompose
$$
\gamma = \sum \gamma_i, \quad
\gamma_i \in H^i(X) \otimes_F H^{2d - i}(X)(d)
$$
in the K\"unneth decomposition. Similarly, denote
$\gamma' = \gamma([\Delta']) = \sum \gamma'_i$
in $(H')^{2d}(X_{k'} \times_{k'} X_{k'})(d)$.
By the linear algebra princple mentioned above, it suffices
to show that $\gamma_0$ maps to $\gamma'_0$ in
$(H')^0(X) \otimes_{F'} (H')^{2d}(X')(d)$.
By the compatibility of K\"unneth decompositions
we see that it suffice to show that $\gamma$ maps to
$\gamma'$ in
$$
(H')^{2d}(X_{k'} \times_{k'} X_{k'})(d) = (H')^{2d}((X \times X)_{k'})(d)
$$
Since $\Delta_{k'} = \Delta'$ this follows from the discussion above.

\medskip\noindent
Axiom (A7). This follows from the linear algebra fact: a
linear map $V \to W$ of $F$-vector spaces is injective
if and only if $V \otimes_F F' \to W \otimes_F F'$ is injective.

\medskip\noindent
Axiom (A8). Follows from the linear algebra fact used in
the proof of axiom (A1).

\medskip\noindent
Axiom (A9). Let $X$ be a nonempty smooth projective scheme over $k$
equidimensional of dimension $d$. Let $i : Y \to X$ be a nonempty
effective Cartier divisor smooth over $k$.
Let $\lambda_Y$ and $\lambda_X$ be as in axiom (A6) for $X$ and $Y$.
We have to show: for $a \in H^{2d - 2}(X)(d - 1)$
we have $\lambda_Y(i^*(a)) = \lambda_X(a \cup c_1^H(\mathcal{O}_X(Y))$.
By Remark \ref{remark-trace}
we know that $\lambda_X : H^{2d}(X)(d) \to F$ and
$\lambda_Y : H^{2d - 2}(Y)(d - 1)$ are uniquely
determined by the requirement in axiom (A6).
Having said this, it follows from our proof of axiom (A6) for $H^*$ above
that $\lambda_X \otimes \text{id}_{F'}$ corresponds to $\lambda_{X_{k'}}$
via the given identification $H^{2d}(X)(d) \otimes_F F' = H^{2d}(X_{k'})(d)$.
Thus the fact that we know axiom (A9) for $F'(1), (H')^*, c_1^{H'}$
implies the axiom for $F(1), H^*, c_1^H$ by a diagram chase.
This completes the proof of the lemma.
\end{proof}















\begin{multicols}{2}[\section{Other chapters}]
\noindent
Preliminaries
\begin{enumerate}
\item \hyperref[introduction-section-phantom]{Introduction}
\item \hyperref[conventions-section-phantom]{Conventions}
\item \hyperref[sets-section-phantom]{Set Theory}
\item \hyperref[categories-section-phantom]{Categories}
\item \hyperref[topology-section-phantom]{Topology}
\item \hyperref[sheaves-section-phantom]{Sheaves on Spaces}
\item \hyperref[sites-section-phantom]{Sites and Sheaves}
\item \hyperref[stacks-section-phantom]{Stacks}
\item \hyperref[fields-section-phantom]{Fields}
\item \hyperref[algebra-section-phantom]{Commutative Algebra}
\item \hyperref[brauer-section-phantom]{Brauer Groups}
\item \hyperref[homology-section-phantom]{Homological Algebra}
\item \hyperref[derived-section-phantom]{Derived Categories}
\item \hyperref[simplicial-section-phantom]{Simplicial Methods}
\item \hyperref[more-algebra-section-phantom]{More on Algebra}
\item \hyperref[smoothing-section-phantom]{Smoothing Ring Maps}
\item \hyperref[modules-section-phantom]{Sheaves of Modules}
\item \hyperref[sites-modules-section-phantom]{Modules on Sites}
\item \hyperref[injectives-section-phantom]{Injectives}
\item \hyperref[cohomology-section-phantom]{Cohomology of Sheaves}
\item \hyperref[sites-cohomology-section-phantom]{Cohomology on Sites}
\item \hyperref[dga-section-phantom]{Differential Graded Algebra}
\item \hyperref[dpa-section-phantom]{Divided Power Algebra}
\item \hyperref[sdga-section-phantom]{Differential Graded Sheaves}
\item \hyperref[hypercovering-section-phantom]{Hypercoverings}
\end{enumerate}
Schemes
\begin{enumerate}
\setcounter{enumi}{25}
\item \hyperref[schemes-section-phantom]{Schemes}
\item \hyperref[constructions-section-phantom]{Constructions of Schemes}
\item \hyperref[properties-section-phantom]{Properties of Schemes}
\item \hyperref[morphisms-section-phantom]{Morphisms of Schemes}
\item \hyperref[coherent-section-phantom]{Cohomology of Schemes}
\item \hyperref[divisors-section-phantom]{Divisors}
\item \hyperref[limits-section-phantom]{Limits of Schemes}
\item \hyperref[varieties-section-phantom]{Varieties}
\item \hyperref[topologies-section-phantom]{Topologies on Schemes}
\item \hyperref[descent-section-phantom]{Descent}
\item \hyperref[perfect-section-phantom]{Derived Categories of Schemes}
\item \hyperref[more-morphisms-section-phantom]{More on Morphisms}
\item \hyperref[flat-section-phantom]{More on Flatness}
\item \hyperref[groupoids-section-phantom]{Groupoid Schemes}
\item \hyperref[more-groupoids-section-phantom]{More on Groupoid Schemes}
\item \hyperref[etale-section-phantom]{\'Etale Morphisms of Schemes}
\end{enumerate}
Topics in Scheme Theory
\begin{enumerate}
\setcounter{enumi}{41}
\item \hyperref[chow-section-phantom]{Chow Homology}
\item \hyperref[intersection-section-phantom]{Intersection Theory}
\item \hyperref[pic-section-phantom]{Picard Schemes of Curves}
\item \hyperref[weil-section-phantom]{Weil Cohomology Theories}
\item \hyperref[adequate-section-phantom]{Adequate Modules}
\item \hyperref[dualizing-section-phantom]{Dualizing Complexes}
\item \hyperref[duality-section-phantom]{Duality for Schemes}
\item \hyperref[discriminant-section-phantom]{Discriminants and Differents}
\item \hyperref[derham-section-phantom]{de Rham Cohomology}
\item \hyperref[local-cohomology-section-phantom]{Local Cohomology}
\item \hyperref[algebraization-section-phantom]{Algebraic and Formal Geometry}
\item \hyperref[curves-section-phantom]{Algebraic Curves}
\item \hyperref[resolve-section-phantom]{Resolution of Surfaces}
\item \hyperref[models-section-phantom]{Semistable Reduction}
\item \hyperref[functors-section-phantom]{Functors and Morphisms}
\item \hyperref[equiv-section-phantom]{Derived Categories of Varieties}
\item \hyperref[pione-section-phantom]{Fundamental Groups of Schemes}
\item \hyperref[etale-cohomology-section-phantom]{\'Etale Cohomology}
\item \hyperref[crystalline-section-phantom]{Crystalline Cohomology}
\item \hyperref[proetale-section-phantom]{Pro-\'etale Cohomology}
\item \hyperref[relative-cycles-section-phantom]{Relative Cycles}
\item \hyperref[more-etale-section-phantom]{More \'Etale Cohomology}
\item \hyperref[trace-section-phantom]{The Trace Formula}
\end{enumerate}
Algebraic Spaces
\begin{enumerate}
\setcounter{enumi}{64}
\item \hyperref[spaces-section-phantom]{Algebraic Spaces}
\item \hyperref[spaces-properties-section-phantom]{Properties of Algebraic Spaces}
\item \hyperref[spaces-morphisms-section-phantom]{Morphisms of Algebraic Spaces}
\item \hyperref[decent-spaces-section-phantom]{Decent Algebraic Spaces}
\item \hyperref[spaces-cohomology-section-phantom]{Cohomology of Algebraic Spaces}
\item \hyperref[spaces-limits-section-phantom]{Limits of Algebraic Spaces}
\item \hyperref[spaces-divisors-section-phantom]{Divisors on Algebraic Spaces}
\item \hyperref[spaces-over-fields-section-phantom]{Algebraic Spaces over Fields}
\item \hyperref[spaces-topologies-section-phantom]{Topologies on Algebraic Spaces}
\item \hyperref[spaces-descent-section-phantom]{Descent and Algebraic Spaces}
\item \hyperref[spaces-perfect-section-phantom]{Derived Categories of Spaces}
\item \hyperref[spaces-more-morphisms-section-phantom]{More on Morphisms of Spaces}
\item \hyperref[spaces-flat-section-phantom]{Flatness on Algebraic Spaces}
\item \hyperref[spaces-groupoids-section-phantom]{Groupoids in Algebraic Spaces}
\item \hyperref[spaces-more-groupoids-section-phantom]{More on Groupoids in Spaces}
\item \hyperref[bootstrap-section-phantom]{Bootstrap}
\item \hyperref[spaces-pushouts-section-phantom]{Pushouts of Algebraic Spaces}
\end{enumerate}
Topics in Geometry
\begin{enumerate}
\setcounter{enumi}{81}
\item \hyperref[spaces-chow-section-phantom]{Chow Groups of Spaces}
\item \hyperref[groupoids-quotients-section-phantom]{Quotients of Groupoids}
\item \hyperref[spaces-more-cohomology-section-phantom]{More on Cohomology of Spaces}
\item \hyperref[spaces-simplicial-section-phantom]{Simplicial Spaces}
\item \hyperref[spaces-duality-section-phantom]{Duality for Spaces}
\item \hyperref[formal-spaces-section-phantom]{Formal Algebraic Spaces}
\item \hyperref[restricted-section-phantom]{Algebraization of Formal Spaces}
\item \hyperref[spaces-resolve-section-phantom]{Resolution of Surfaces Revisited}
\end{enumerate}
Deformation Theory
\begin{enumerate}
\setcounter{enumi}{89}
\item \hyperref[formal-defos-section-phantom]{Formal Deformation Theory}
\item \hyperref[defos-section-phantom]{Deformation Theory}
\item \hyperref[cotangent-section-phantom]{The Cotangent Complex}
\item \hyperref[examples-defos-section-phantom]{Deformation Problems}
\end{enumerate}
Algebraic Stacks
\begin{enumerate}
\setcounter{enumi}{93}
\item \hyperref[algebraic-section-phantom]{Algebraic Stacks}
\item \hyperref[examples-stacks-section-phantom]{Examples of Stacks}
\item \hyperref[stacks-sheaves-section-phantom]{Sheaves on Algebraic Stacks}
\item \hyperref[criteria-section-phantom]{Criteria for Representability}
\item \hyperref[artin-section-phantom]{Artin's Axioms}
\item \hyperref[quot-section-phantom]{Quot and Hilbert Spaces}
\item \hyperref[stacks-properties-section-phantom]{Properties of Algebraic Stacks}
\item \hyperref[stacks-morphisms-section-phantom]{Morphisms of Algebraic Stacks}
\item \hyperref[stacks-limits-section-phantom]{Limits of Algebraic Stacks}
\item \hyperref[stacks-cohomology-section-phantom]{Cohomology of Algebraic Stacks}
\item \hyperref[stacks-perfect-section-phantom]{Derived Categories of Stacks}
\item \hyperref[stacks-introduction-section-phantom]{Introducing Algebraic Stacks}
\item \hyperref[stacks-more-morphisms-section-phantom]{More on Morphisms of Stacks}
\item \hyperref[stacks-geometry-section-phantom]{The Geometry of Stacks}
\end{enumerate}
Topics in Moduli Theory
\begin{enumerate}
\setcounter{enumi}{107}
\item \hyperref[moduli-section-phantom]{Moduli Stacks}
\item \hyperref[moduli-curves-section-phantom]{Moduli of Curves}
\end{enumerate}
Miscellany
\begin{enumerate}
\setcounter{enumi}{109}
\item \hyperref[examples-section-phantom]{Examples}
\item \hyperref[exercises-section-phantom]{Exercises}
\item \hyperref[guide-section-phantom]{Guide to Literature}
\item \hyperref[desirables-section-phantom]{Desirables}
\item \hyperref[coding-section-phantom]{Coding Style}
\item \hyperref[obsolete-section-phantom]{Obsolete}
\item \hyperref[fdl-section-phantom]{GNU Free Documentation License}
\item \hyperref[index-section-phantom]{Auto Generated Index}
\end{enumerate}
\end{multicols}


\bibliography{my}
\bibliographystyle{amsalpha}

\end{document}
